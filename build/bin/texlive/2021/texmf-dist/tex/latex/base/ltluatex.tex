%%
%% This is file `ltluatex.tex',
%% generated with the docstrip utility.
%%
%% The original source files were:
%%
%% ltluatex.dtx  (with options: `tex,plain')
%% 
%% This is a generated file.
%% 
%% The source is maintained by the LaTeX Project team and bug
%% reports for it can be opened at https://latex-project.org/bugs.html
%% (but please observe conditions on bug reports sent to that address!)
%% 
%% 
%% Copyright (C) 1993-2021
%% The LaTeX Project and any individual authors listed elsewhere
%% in this file.
%% 
%% This file was generated from file(s) of the LaTeX base system.
%% --------------------------------------------------------------
%% 
%% It may be distributed and/or modified under the
%% conditions of the LaTeX Project Public License, either version 1.3c
%% of this license or (at your option) any later version.
%% The latest version of this license is in
%%    https://www.latex-project.org/lppl.txt
%% and version 1.3c or later is part of all distributions of LaTeX
%% version 2008 or later.
%% 
%% This file has the LPPL maintenance status "maintained".
%% 
%% This file may only be distributed together with a copy of the LaTeX
%% base system. You may however distribute the LaTeX base system without
%% such generated files.
%% 
%% The list of all files belonging to the LaTeX base distribution is
%% given in the file `manifest.txt'. See also `legal.txt' for additional
%% information.
%% 
%% The list of derived (unpacked) files belonging to the distribution
%% and covered by LPPL is defined by the unpacking scripts (with
%% extension .ins) which are part of the distribution.
\ifx\newluafunction\undefined\else\expandafter\endinput\fi
\ifx
  \ProvidesFile\undefined\begingroup\def\ProvidesFile
  #1#2[#3]{\endgroup\immediate\write-1{File: #1 #3}}
\fi
\ProvidesFile{ltluatex.tex}%
[2021/10/15 v1.1v
  LuaTeX support for plain TeX (core)
]
\edef\etatcatcode{\the\catcode`\@}
\catcode`\@=11
\ifnum\luatexversion<60 %
  \wlog{***************************************************}
  \wlog{* LuaTeX version too old for ltluatex support *}
  \wlog{***************************************************}
  \expandafter\endinput
\fi
\long\def\@gobble#1{}
\long\def\@firstofone#1{#1}
\directlua{tex.enableprimitives("",tex.extraprimitives("luatex"))}
\ifx\e@alloc\@undefined
  \ifx\documentclass\@undefined
    \ifx\loccount\@undefined
      % \iffalse meta-comment
%
% Copyright 1997, 1998, 2008 2015 2016 LaTeX Project and Peter Breitenlohner.
% 
% This file (etex.sty) may be distributed and/or modified under the
% conditions of the LaTeX Project Public License, either version 1.3 of
% this license or (at your option) any later version.  The latest
% version of this license is in
%   http://www.latex-project.org/lppl.txt
% and version 1.3 or later is part of all distributions of LaTeX
% version 2003/12/01 or later.
% 
% This work has the LPPL maintenance status "maintained".
% 
% The Current Maintainer of this work is David Carlisle.
% https://github.com/davidcarlisle/dpctex/issues
% \fi

\NeedsTeXFormat{LaTeX2e}
\ProvidesPackage{etex}
%        [1997/08/12 v0.1 eTeX basic definition package (DPC)]
%        [1998/03/26 v2.0 eTeX basic definition package (PEB)]
%        [2015/03/02 v2.1 eTeX basic definition package (PEB,DPC)]
%        [2015/07/06 v2.2 eTeX basic definition package (PEB,DPC)]
%        [2015/07/08 v2.3 eTeX basic definition package (PEB,DPC)]
%        [2015/09/02 v2.4 eTeX basic definition package (PEB,DPC)]
%        [2016/01/07 v2.5 eTeX basic definition package (PEB,DPC)]
%        [2016/01/11 v2.6 eTeX basic definition package (PEB,DPC)]
         [2016/08/01 v2.7 eTeX basic definition package (PEB,DPC)]

%%%%%%%%%%%%%%%%%%%%%%%%%%%%%%%%%%%%%%%%%%%%%%%%%%%%%%%%%%%%%%%%%%%%%%%%

%% A basic interface to some etex primitives, closely modeled on
%% etex.src and etexdefs.lib provided by the core etex team.

%% The etex.src `module' system is not copied here, the standard
%% LaTeX package option mechanism is used instead,
%% however the package options match the module names.
%% (Currently grouptypes, interactionmodes, nodetypes, iftypes.)
%% The individual type names are different too: We use, e.g.,
%%
%% `\bottomleveltype' and `\simplegrouptype' instead of
%% `\grouptypes{bottomlevel}' and `\grouptypes{simple}'.

%%%%%%%%%%%%%%%%%%%%%%%%%%%%%%%%%%%%%%%%%%%%%%%%%%%%%%%%%%%%%%%%%%%%%%%%

%% Other Comments...

%% The names of the `interactionmodes' are not too good.
%% In particular \scroll and \batch are likely to clash with existing
%% uses. These names have been changed into \batchinteractionmode,
%% \scrollinteractionmode etc.
%% Similarly, the names of the `groupetypes' have been changed, in
%% particular \mathgroup would conflict with the LaTeX kernel.

%% \etex logo could have the same trick as \LaTeXe to pick up a bold
%% epsilon when needed. (Not done here, I hate wasting tokens on logos.)
%% This version does have a \m@th not in the original.

%% The \globcountvector, \loccountvector, etc. allocation macros are
%% not (yet) implemented.

%% Currently if run on a standard TeX, the package generates an error.
%% Perhaps it should instead load some code to try to fake
%% the new etex primitives in that case???
%% Likewise, the package generates an error when used with e-TeX V 1

%% The etex.src language mechanism is not copied here. That facility
%% does not use any of the etex features. LaTeX should be customised
%% using the same hyphen.cfg mechanism as for a format built with a
%% standard TeX.

%% David Carlisle

%% Upgraded for e-TeX V 2.0
%% Peter Breitenlohner

%%%%%%%%%%%%%%%%%%%%%%%%%%%%%%%%%%%%%%%%%%%%%%%%%%%%%%%%%%%%%%%%%%%%%%%%



\ifx\eTeXversion\@undefined
  \PackageError{etex}
    {This package may only be run using an\MessageBreak
     etex in extended mode}
    {Perhaps you forgot the `*' when making the format with (e)initex.%
    }
\fi

\ifnum\eTeXversion<2
  \PackageError{etex}
    {This package requires e-TeX V 2}
    {You are probably using the obsolete e-TeX V 1.%
    }
\fi

% 2.2
% Check if the new latex 2015/01/01 allocation is already using
% extended reisters. If so it is too late to change allocation scheme.
% Older versions of LaTeX would have given an error when the classic
% TeX registers were all allocated, but newer formats allocate from
% the extended range, so usually this package is not needed.
\@tempswafalse
\ifnum\count10>\@cclv\@tempswatrue\else
\ifnum\count11>\@cclv\@tempswatrue\else
\ifnum\count12>\@cclv\@tempswatrue\else
\ifnum\count13>\@cclv\@tempswatrue\else
\ifnum\count14>\@cclv\@tempswatrue\else
\ifnum\count15>\@cclv\@tempswatrue
\fi\fi\fi\fi\fi\fi

\if@tempswa
\PackageWarningNoLine{etex}{%
Extended allocation already in use.\MessageBreak
etex.sty code will not be used.\MessageBreak
To force etex package to load, add\MessageBreak
\string\RequirePackage{etex}\MessageBreak
at the start of the document}

% 2.5 define the global allocation to be the standard ones
% as extended allocation is already in use. Helps with
% compatibility with some packages that use these commands 
% after loading etex.
% 2.6 avoid error from outer if used with (e)plain
\expandafter\let\csname globcount\expandafter\endcsname
                \csname newcount\endcsname
\expandafter\let\csname globdimen\expandafter\endcsname
                \csname newdimen\endcsname
\expandafter\let\csname globskip\expandafter\endcsname
                \csname newskip\endcsname
\expandafter\let\csname globmuskip\expandafter\endcsname
                \csname newmuskip\endcsname
\expandafter\let\csname globtoks\expandafter\endcsname
                \csname newtoks\endcsname
\expandafter\let\csname globmarks\expandafter\endcsname
                \csname newmarks\endcsname
% end of 2.5/2.6 change

\expandafter\endinput\fi

% End of 2.2 addition.

% 2.3 move option handling after the above error checks.
\DeclareOption{grouptypes}{\catcode`\G=9}
\DeclareOption{interactionmodes}{\catcode`\I=9}
\DeclareOption{nodetypes}{\catcode`\N=9}
\DeclareOption{iftypes}{\catcode`\C=9}
\DeclareOption{localalloclog}{\let\et@xwlog\wlog} % the default
\DeclareOption{localallocnolog}{\let\et@xwlog\@gobble} % be quiet
\DeclareOption{localallocshow}{\let\et@xwlog\typeout} % debugging
% End of 2.3 addition.

% v2.7
% \extrafloats does not work with this package
% but make it give a sensible error, not mis-parse \ifnum.
%
% Note that using \extrafloats earlier might not be safe as
% it could in principle clash with registers used for local allocation.
% However it probably works (as local allocation is used locally...).
% A better fix would be not to load this package with current LaTeX.
% This current etex package is just provided to force the old behaviour
% and such documents should not be using new features such as \extrafloats).
\ifdefined\extrafloats
\def\extrafloats#1{%
  \PackageError{etex}{%
    \noexpand\extrafloats is incompatible with etex.sty allocation.\MessageBreak
    Try using \noexpand\extrafloats before loading etex}%
    \@ehc}
\fi
% end of v2.7 change

\def\eTeX{%
  $\m@th\varepsilon$-\TeX}

\def\tracingall{%
  \tracingcommands\thr@@        % etex
  \tracingstats\tw@
  \tracingpages\@ne
  \tracinglostchars\tw@         % etex
  \tracingmacros\tw@
  \tracingparagraphs\@ne
  \tracingrestores\@ne
  \tracinggroups\@ne            % etex
  \tracingifs\@ne               % etex
  \tracingscantokens\@ne        % etex
  \tracingnesting\@ne           % etex
  \tracingassigns\@ne           % etex
  \errorcontextlines\maxdimen
  \showoutput}

\def\loggingall{%
  \tracingall
  \tracingonline\z@}

\def\tracingnone{%
  \tracingonline\z@
  \showboxdepth\m@ne
  \showboxbreadth\m@ne
  \tracingoutput\z@
  \errorcontextlines\m@ne
  \tracingassigns\z@
  \tracingnesting\z@
  \tracingscantokens\z@
  \tracingifs\z@
  \tracinggroups\z@
  \tracingrestores\z@
  \tracingparagraphs\z@
  \tracingmacros\z@
  \tracinglostchars\@ne
  \tracingpages\z@
  \tracingstats\z@
  \tracingcommands\z@}

%% Register allocation
%% We have to adjust the Plain TeX / LaTeX register allocation counts
%% for our slightly modified book-keeping, but first we allocate our
%% insertion counter \et@xins, because \insc@ount of Plain TeX / LaTeX
%% will be used differently.

\newcount\et@xins

\advance\count10 by 1 % \count10=23 % allocates \count registers 23, 24, ...
\advance\count11 by 1 % \count11=10 % allocates \dimen registers 10, 11, ...
\advance\count12 by 1 % \count12=10 % allocates \skip registers 10, 11, ...
\advance\count13 by 1 % \count13=10 % allocates \muskip registers 10, 11, ...
\advance\count14 by 1 % \count14=10 % allocates \box registers 10, 11, ...
\advance\count15 by 1 % \count15=10 % allocates \toks registers 10, 11, ...
\advance\count16 by 1 % \count16=0 % allocates input streams 0, 1, ...
\advance\count17 by 1 % \count17=0 % allocates output streams 0, 1, ...
\advance\count18 by 1 % \count18=4 % allocates math families 4, 5, ...
\advance\count19 by 1 % \count19=0 % allocates \language codes 0, 1, ...

\et@xins=\insc@unt % \et@xins=255 % allocates insertions 254, 253, ...


%% To ensure working in LaTeX 2015 release do define \newcount etc
%% with their pre 2015 LaTeX definitions
\def\newcount{\alloc@0\count\countdef\insc@unt}
\def\newdimen{\alloc@1\dimen\dimendef\insc@unt}
\def\newskip{\alloc@2\skip\skipdef\insc@unt}
\def\newmuskip{\alloc@3\muskip\muskipdef\@cclvi}
\def\newbox{\alloc@4\box\chardef\insc@unt}
\def\newtoks{\alloc@5\toks\toksdef\@cclvi}
\def\newread{\alloc@6\read\chardef\sixt@@n}
\def\newwrite{\alloc@7\write\chardef\sixt@@n}
\def\new@mathgroup{\alloc@8\mathgroup\chardef\sixt@@n}
\let\newfam\new@mathgroup
\def\newlanguage{\alloc@9\language\chardef\@cclvi}

%% When the normal register pool for \count, \dimen, \skip, \muskip,
%% \box, or \toks registers is exhausted, we switch to the extended pool.

\def\alloc@#1#2#3#4#5%
 {\ifnum\count1#1<#4% make sure there's still room
    \allocationnumber\count1#1
    \global\advance\count1#1\@ne
    \global#3#5\allocationnumber
    \wlog{\string#5=\string#2\the\allocationnumber}%
  \else\ifnum#1<6
    \begingroup \escapechar\m@ne
    \expandafter\alloc@@\expandafter{\string#2}#5%
  \else\errmessage{No room for a new #2}\fi\fi
 }

%% The \expandafter construction used here allows the generation of
%% \newcount and \globcount from #1=count.

\def\alloc@@#1#2%
 {\endgroup % restore \escapechar
  \wlog{Normal \csname#1\endcsname register pool exhausted,
    switching to extended pool.}%
  \global\expandafter\let
    \csname new#1\expandafter\endcsname
    \csname glob#1\endcsname
  \csname new#1\endcsname#2%
 }

%% We do change the LaTeX definition of \newinsert

\def\newinsert#1{% make sure there's still room for ...
  \ch@ck0\et@xins\count{% ... a \count, ...
    \ch@ck1\et@xins\dimen{% ... \dimen, ...
      \ch@ck2\et@xins\skip{% ... \skip, ...
        \ch@ck4\et@xins\box{% ... and \box register
  \global\advance\et@xins\m@ne
  \unless\ifnum\insc@unt<\et@xins \global\insc@unt\et@xins \fi
  \allocationnumber\et@xins
  \global\chardef#1\allocationnumber
  \wlog{\string#1=\string\insert\the\allocationnumber}}}}}}

\def\ch@ck#1#2#3#4%
 {\ifnum\count1#1<#2#4\else\errmessage{No room for a new #3}\fi}

%% And we define \reserveinserts, so that you can say \reserveinserts{17}
%% in order to reserve room for up to 17 additional insertion classes, that
%% will not be taken away by \newcount, \newdimen, \newskip, or \newbox.

% 2.4 Remove \outer to match LaTeX allocations
% which are never \outer unlike plain TeX.

%\outer
\def\reserveinserts#1%
 {\global\insc@unt\numexpr \et@xins \ifnum#1>\z@ -#1\fi \relax}

% Now, we define \globcount, \globbox, etc., so that you can say
% \globcount\foo and \foo will be defined (with \countdef) to be the
% next count register from the vastly larger but somewhat less efficient
% extended register pool. We also define \loccount, etc., but these
% register definitions are local to the current group.

\count260=277 % globally allocates \count registers 277, 278, ...
\count261=256 % globally allocates \dimen registers 256, 257, ...
\count262=256 % globally allocates \skip registers 256, 257, ...
\count263=256 % globally allocates \muskip registers 256, 257, ...
\count264=256 % globally allocates \box registers 256, 257, ...
\count265=256 % globally allocates \toks registers 256, 257, ...
\count266=1 % globally allocates \marks classes 1, 2, ...

\count270=32768 % locally allocates \count registers 32767, 32766, ...
\count271=32768 % ditto for \dimen registers
\count272=32768 % ditto for \skip registers
\count273=32768 % ditto for \muskip registers
\count274=32768 % ditto for \box registers
\count275=32768 % ditto for \toks registers
\count276=32768 % ditto for \marks classes

% \count registers 256-259 and 267-269 are not (yet) used

% \def \et@xglob #1#2#3#4% <offset>, <type>, <method>, <register>
% \def \et@xloc #1#2#3#4% <offset>, <type>, <method>, <register>

\def \globcount  {\et@xglob 0\count  \countdef}
\def \loccount   {\et@xloc  0\count  \countdef}
\def \globdimen  {\et@xglob 1\dimen  \dimendef}
\def \locdimen   {\et@xloc  1\dimen  \dimendef}
\def \globskip   {\et@xglob 2\skip   \skipdef}
\def \locskip    {\et@xloc  2\skip   \skipdef}
\def \globmuskip {\et@xglob 3\muskip \muskipdef}
\def \locmuskip  {\et@xloc  3\muskip \muskipdef}
\def \globbox    {\et@xglob 4\box    \mathchardef}
\def \locbox     {\et@xloc  4\box    \mathchardef}
\def \globtoks   {\et@xglob 5\toks   \toksdef}
\def \loctoks    {\et@xloc  5\toks   \toksdef}
\def \globmarks  {\et@xglob 6\marks  \mathchardef}
\def \locmarks   {\et@xloc  6\marks  \mathchardef}

\let\newmarks=\globmarks %% this used to be \newmark for e-TeX V 1.1

\def\et@xglob#1#2#3#4%
 {\et@xchk#1#2{% make sure there's still room
  \allocationnumber=\count26#1%
  \global\advance\count26#1\@ne
  \global#3#4\allocationnumber
  \wlog{\string#4=\string#2\the\allocationnumber}}%
 }

\def\et@xloc#1#2#3#4%
 {\et@xchk#1#2{% make sure there's still room
  \advance\count27#1by\m@ne
  \allocationnumber=\count27#1%
  #3#4=\allocationnumber
  \et@xwlog{\string#4=\string#2\the\allocationnumber\space(local)}}%
 }

%% The allocation messages for local allocations use \et@xwlog, such that
%% these messages can easily be switched on/off

\let\et@xwlog=\wlog

\def\et@xchk#1#2#3%
 {\ifnum\count26#1<\count27#1 #3\else\errmessage{No room for a new #2}\fi}

% Next we define \globcountblk, \loccountblk, etc., so that you can
% say \globcountblk\foo{17} and \foo will be defined (with \mathchardef)
% as the first (the zeroth?) of a block of 17 consecutive registers.
% Thus the user is intended to reference elements <\foo+0> to <\foo+n-1>,
% where n is the length of the block allocated.

% \def \et@xgblk #1#2#3#4% <offset>, <type>, <register>, <size>
% \def \et@xlblk #1#2#3#4% <offset>, <type>, <register>, <size>

\def\globcountblk  {\et@xgblk 0\count  }
\def\loccountblk   {\et@xlblk 0\count  }
\def\globdimenblk  {\et@xgblk 1\dimen  }
\def\locdimenblk   {\et@xlblk 1\dimen  }
\def\globskipblk   {\et@xgblk 2\skip   }
\def\locskipblk    {\et@xlblk 2\skip   }
\def\globmuskipblk {\et@xgblk 3\muskip }
\def\locmuskipblk  {\et@xlblk 3\muskip }
\def\globboxblk    {\et@xgblk 4\box    }
\def\locboxblk     {\et@xlblk 4\box    }
\def\globtoksblk   {\et@xgblk 5\toks   }
\def\loctoksblk    {\et@xlblk 5\toks   }
\def\globmarksblk  {\et@xgblk 6\marks  }
\def\locmarksblk   {\et@xlblk 6\marks  }

% \def\et@xchkblk#1#1#3#4% <offset>, <type>, <size>, <action>

\def\et@xgblk#1#2#3#4%
 {\et@xchkblk#1#2{#4}% make sure there's still room
   {\allocationnumber\count26#1%
    \global\advance\count26#1by#4%
    \global\mathchardef#3\allocationnumber
    \wlog{\string#3=\string#2blk{\number#4} at
      \the\allocationnumber}%
   }%
 }

\def\et@xlblk#1#2#3#4%
 {\et@xchkblk#1#2{#4}% make sure there's still room
   {\advance\count27#1-#4%
    \allocationnumber\count27#1%
    \mathchardef#3\allocationnumber
    \et@xwlog{\string#3=\string#2blk{\number#4} at
      \the\allocationnumber\space(local)}%
   }%
 }

\def\et@xchkblk#1#2#3#4%
 {\ifnum#3<\z@
    \errmessage{Negative register block size \number#3}%
  \else\ifnum\numexpr\count26#1+#3>\count27#1%
    \errmessage{No room for new #2block of size \number#3}%
  \else #4\fi \fi
 }

\catcode`\G=14
\catcode`\I=14
\catcode`\N=14
\catcode`\C=14

\ProcessOptions

%% Declare names for `grouptypes'

G \chardef \bottomleveltype       =  0 % for the outside world
G \chardef \simplegrouptype       =  1 % for local structure only
G \chardef \hboxgrouptype         =  2 % for `\hbox{}'
G \chardef \adjustedhboxgrouptype =  3 % for `\hbox{}' in vertical mode
G \chardef \vboxgrouptype         =  4 % for `\vbox{}'
G \chardef \vtopgrouptype         =  5 % for `\vtop{}'
G \chardef \aligngrouptype        =  6 % for `\halign{}', `\valign{}'
G \chardef \noaligngrouptype      =  7 % for `\noalign{}'
G \chardef \outputgrouptype       =  8 % for output routine
G \chardef \mathgrouptype         =  9 % for, e.g, `^{}'
G \chardef \discgrouptype         = 10 % for `\discretionary{}{}{}'
G \chardef \insertgrouptype       = 11 % for `\insert{}', `\vadjust{}'
G \chardef \vcentergrouptype      = 12 % for `\vcenter{}'
G \chardef \mathchoicegrouptype   = 13 % for `\mathchoice{}{}{}{}'
G \chardef \semisimplegrouptype   = 14 % for `\begingroup...\endgroup'
G \chardef \mathshiftgrouptype    = 15 % for `$...$'
G \chardef \mathleftgrouptype     = 16 % for `\left...\right'

%% Declare names for `interactionmodes'

I \chardef \batchinteractionmode     = 0 % omits all stops and omits terminal output
I \chardef \nonstopinteractionmode   = 1 % omits all stops
I \chardef \scrollinteractionmode    = 2 % omits error stops
I \chardef \errorstopinteractionmode = 3 % stops at every opportunity to interact

%% Declare names for `nodetypes'

N \chardef \charnode     =  0 % character nodes
N \chardef \hlistnode    =  1 % hlist nodes
N \chardef \vlistnode    =  2 % vlist nodes
N \chardef \rulenode     =  3 % rule nodes
N \chardef \insnode      =  4 % insertion nodes
N \chardef \marknode     =  5 % a mark node
N \chardef \adjustnode   =  6 % an adjust node
N \chardef \ligaturenode =  7 % a ligature node
N \chardef \discnode     =  8 % a discretionary node
N \chardef \whatsitnode  =  9 % special extension nodes
N \chardef \mathnode     = 10 % a math node
N \chardef \gluenode     = 11 % node that points to a glue specification
N \chardef \kernnode     = 12 % a kern node
N \chardef \penaltynode  = 13 % a penalty node
N \chardef \unsetnode    = 14 % an unset node
N \chardef \mathsnodes   = 15 % nodes that occur only in maths mode

%% Declare names for `iftypes'

C \chardef \charif     =  1 % \if
C \chardef \catif      =  2 % \ifcat
C \chardef \numif      =  3 % \ifnum
C \chardef \dimif      =  4 % \ifdim
C \chardef \oddif      =  5 % \ifodd
C \chardef \vmodeif    =  6 % \ifvmode
C \chardef \hmodeif    =  7 % \ifhmode
C \chardef \mmodeif    =  8 % \ifmmode
C \chardef \innerif    =  9 % \ifinner
C \chardef \voidif     = 10 % \ifvoid
C \chardef \hboxif     = 11 % \ifhbox
C \chardef \vboxif     = 12 % \ifvbox
C \chardef \xif        = 13 % \ifx
C \chardef \eofif      = 14 % \ifeof
C \chardef \trueif     = 15 % \iftrue
C \chardef \falseif    = 16 % \iffalse
C \chardef \caseif     = 17 % \ifcase
C \chardef \definedif  = 18 % \ifdefined
C \chardef \csnameif   = 19 % \ifcsname
C \chardef \fontcharif = 20 % \iffontchar

\catcode`\G=11
\catcode`\I=11
\catcode`\N=11
\catcode`\C=11

%
    \fi
    \catcode`\@=11 %
    \outer\expandafter\def\csname newfam\endcsname
                          {\alloc@8\fam\chardef\et@xmaxfam}
  \else
    \RequirePackage{etex}
    \expandafter\def\csname newfam\endcsname
                    {\alloc@8\fam\chardef\et@xmaxfam}
    \expandafter\let\expandafter\new@mathgroup\csname newfam\endcsname
  \fi
\edef \et@xmaxregs {\ifx\directlua\@undefined 32768\else 65536\fi}
\edef \et@xmaxfam {\ifx\Umathcode\@undefined\sixt@@n\else\@cclvi\fi}
\count 270=\et@xmaxregs % locally allocates \count registers
\count 271=\et@xmaxregs % ditto for \dimen registers
\count 272=\et@xmaxregs % ditto for \skip registers
\count 273=\et@xmaxregs % ditto for \muskip registers
\count 274=\et@xmaxregs % ditto for \box registers
\count 275=\et@xmaxregs % ditto for \toks registers
\count 276=\et@xmaxregs % ditto for \marks classes
\expandafter\let\csname newcount\expandafter\expandafter\endcsname
                \csname globcount\endcsname
\expandafter\let\csname newdimen\expandafter\expandafter\endcsname
                \csname globdimen\endcsname
\expandafter\let\csname newskip\expandafter\expandafter\endcsname
                \csname globskip\endcsname
\expandafter\let\csname newbox\expandafter\expandafter\endcsname
                \csname globbox\endcsname
\chardef\e@alloc@top=65535
\let\e@alloc@chardef\chardef
\def\e@alloc#1#2#3#4#5#6{%
  \global\advance#3\@ne
  \e@ch@ck{#3}{#4}{#5}#1%
  \allocationnumber#3\relax
  \global#2#6\allocationnumber
  \wlog{\string#6=\string#1\the\allocationnumber}}%
\gdef\e@ch@ck#1#2#3#4{%
  \ifnum#1<#2\else
    \ifnum#1=#2\relax
      #1\@cclvi
      \ifx\count#4\advance#1 10 \fi
    \fi
    \ifnum#1<#3\relax
    \else
      \errmessage{No room for a new \string#4}%
    \fi
  \fi}%
\expandafter\csname newcount\endcsname\e@alloc@attribute@count
\expandafter\csname newcount\endcsname\e@alloc@ccodetable@count
\expandafter\csname newcount\endcsname\e@alloc@luafunction@count
\expandafter\csname newcount\endcsname\e@alloc@whatsit@count
\expandafter\csname newcount\endcsname\e@alloc@bytecode@count
\expandafter\csname newcount\endcsname\e@alloc@luachunk@count
\fi
\ifx\e@alloc@attribute@count\@undefined
  \countdef\e@alloc@attribute@count=258
  \e@alloc@attribute@count=\z@
\fi
\def\newattribute#1{%
  \e@alloc\attribute\attributedef
    \e@alloc@attribute@count\m@ne\e@alloc@top#1%
}
\def\setattribute#1#2{#1=\numexpr#2\relax}
\def\unsetattribute#1{#1=-"7FFFFFFF\relax}
\ifx\e@alloc@ccodetable@count\@undefined
  \countdef\e@alloc@ccodetable@count=259
  \e@alloc@ccodetable@count=\z@
\fi
\def\newcatcodetable#1{%
  \e@alloc\catcodetable\chardef
    \e@alloc@ccodetable@count\m@ne{"8000}#1%
  \initcatcodetable\allocationnumber
}
\newcatcodetable\catcodetable@initex
\newcatcodetable\catcodetable@string
\begingroup
  \def\setrangecatcode#1#2#3{%
    \ifnum#1>#2 %
      \expandafter\@gobble
    \else
      \expandafter\@firstofone
    \fi
      {%
        \catcode#1=#3 %
        \expandafter\setrangecatcode\expandafter
          {\number\numexpr#1 + 1\relax}{#2}{#3}
      }%
  }
  \@firstofone{%
    \catcodetable\catcodetable@initex
      \catcode0=12 %
      \catcode13=12 %
      \catcode37=12 %
      \setrangecatcode{65}{90}{12}%
      \setrangecatcode{97}{122}{12}%
      \catcode92=12 %
      \catcode127=12 %
      \savecatcodetable\catcodetable@string
    \endgroup
  }%
\newcatcodetable\catcodetable@latex
\newcatcodetable\catcodetable@atletter
\begingroup
  \def\parseunicodedataI#1;#2;#3;#4\relax{%
    \parseunicodedataII#1;#3;#2 First>\relax
  }%
  \def\parseunicodedataII#1;#2;#3 First>#4\relax{%
    \ifx\relax#4\relax
      \expandafter\parseunicodedataIII
    \else
      \expandafter\parseunicodedataIV
    \fi
      {#1}#2\relax%
  }%
  \def\parseunicodedataIII#1#2#3\relax{%
    \ifnum 0%
      \if L#21\fi
      \if M#21\fi
      >0 %
      \catcode"#1=11 %
    \fi
  }%
  \def\parseunicodedataIV#1#2#3\relax{%
    \read\unicoderead to \unicodedataline
    \if L#2%
      \count0="#1 %
      \expandafter\parseunicodedataV\unicodedataline\relax
    \fi
  }%
  \def\parseunicodedataV#1;#2\relax{%
    \loop
      \unless\ifnum\count0>"#1 %
        \catcode\count0=11 %
        \advance\count0 by 1 %
    \repeat
  }%
  \def\storedpar{\par}%
  \chardef\unicoderead=\numexpr\count16 + 1\relax
  \openin\unicoderead=UnicodeData.txt %
  \loop\unless\ifeof\unicoderead %
    \read\unicoderead to \unicodedataline
    \unless\ifx\unicodedataline\storedpar
      \expandafter\parseunicodedataI\unicodedataline\relax
    \fi
  \repeat
  \closein\unicoderead
  \@firstofone{%
    \catcode64=12 %
    \savecatcodetable\catcodetable@latex
    \catcode64=11 %
    \savecatcodetable\catcodetable@atletter
   }
\endgroup
\ifx\e@alloc@luafunction@count\@undefined
  \countdef\e@alloc@luafunction@count=260
  \e@alloc@luafunction@count=\z@
\fi
\def\newluafunction{%
  \e@alloc\luafunction\e@alloc@chardef
    \e@alloc@luafunction@count\m@ne\e@alloc@top
}
\ifx\e@alloc@whatsit@count\@undefined
  \countdef\e@alloc@whatsit@count=261
  \e@alloc@whatsit@count=\z@
\fi
\def\newwhatsit#1{%
  \e@alloc\whatsit\e@alloc@chardef
    \e@alloc@whatsit@count\m@ne\e@alloc@top#1%
}
\ifx\e@alloc@bytecode@count\@undefined
  \countdef\e@alloc@bytecode@count=262
  \e@alloc@bytecode@count=\z@
\fi
\def\newluabytecode#1{%
  \e@alloc\luabytecode\e@alloc@chardef
    \e@alloc@bytecode@count\m@ne\e@alloc@top#1%
}

\ifx\e@alloc@luachunk@count\@undefined
  \countdef\e@alloc@luachunk@count=263
  \e@alloc@luachunk@count=\z@
\fi
\def\newluachunkname#1{%
  \e@alloc\luachunk\e@alloc@chardef
    \e@alloc@luachunk@count\m@ne\e@alloc@top#1%
    {\escapechar\m@ne
    \directlua{lua.name[\the\allocationnumber]="\string#1"}}%
}
\def\now@and@everyjob#1{%
  \everyjob\expandafter{\the\everyjob
    #1%
  }%
  #1%
}
  \begingroup
    \attributedef\attributezero=0 %
    \chardef     \charzero     =0 %
    \countdef    \CountZero    =0 %
    \dimendef    \dimenzero    =0 %
    \mathchardef \mathcharzero =0 %
    \muskipdef   \muskipzero   =0 %
    \skipdef     \skipzero     =0 %
    \toksdef     \tokszero     =0 %
    \directlua{require("ltluatex")}
  \endgroup
\catcode`\@=\etatcatcode\relax
\endinput
%%
%% End of file `ltluatex.tex'.
