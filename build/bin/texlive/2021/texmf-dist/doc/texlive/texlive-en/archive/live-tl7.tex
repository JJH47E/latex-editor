%
% change history (started May 13th 2002)
% 2002/05/13: added tex-langafrican to list of collections
% 2002/05/14: corrected example files, other small errors noted by Volker
% 2002/05/18: win32 updates, by Fabrice
% 2002/05/25: remove mention of sizes, and bsr-interpolated; add Gutenberg
\documentclass{article}
\advance\textwidth by 1in
\advance\oddsidemargin by -1in
\advance\evensidemargin by -1in
\newsavebox{\warnbox}
\def\Q#1{\par\vskip6pt\leftline{#1}\par}
\setcounter{tocdepth}{2}
\pretolerance=1000
\tolerance=1500
\hbadness=3000
\vbadness=3000
\hyphenpenalty=400
\renewcommand{\topfraction}{0.85}
\renewcommand{\floatpagefraction}{0.86}
\renewcommand{\textfraction}{0.1}
\setcounter{topnumber}{5}
\setcounter{totalnumber}{5}
\def\eTeX{$\varepsilon$-\TeX}
\def\Dash{---}
\def\hyph{-}
\def\OMEGA{$\Omega$}
\def\bs{{\protect\normalfont\ttfamily\char'134}}
\DeclareRobustCommand{\cs}[1]{{\normalfont\ttfamily\char`\\#1}}
\def\La{La}
\let\textsl\textit
\let\sl\it
\let\slfamily\itfamily
\def\ttdefault{cmtt}
\usepackage{tex-live}
\begin{document}
%\makeatletter\@ifundefined{HRestore}{}{\HRestore\noalign}\makeatother
\author{Sebastian Rahtz\\
\texttt{sebastian.rahtz@oucs.ox.ac.uk}
%Michel Goossens\\
%\texttt{m.goossens@cern.ch}
}
\title{The \protect\TeX\ Live Guide, 7th edition}
\date{May 2002}
\maketitle
\begin{multicols}{2}
\tableofcontents
\listoftables
\end{multicols}
\section{Introduction}\label{sec:intro}
This documentation describes the
  main features of the \TeXLive{} 7 \CD{}\Dash a \TeX{}/\LaTeX{}
  distribution for Unix, Linux, MacOSX, and Windows32 systems that includes
  \TeX{}, \LaTeXe{}, \MF, \MP, \cmdname{Makeindex}, and \BibTeX{}; and
  a wide-ranging set of macros, fonts and documentation conforming to
  the \emph{\TeX{} Directory Standard} (\TDS{})\Dash which can be used
  with nearly every \TeX{} setup.
  
  This \TeX{} package uses the \Webc{} version 7.3.7
  implementation of the programs, which tries to make \TeX{}ing as
  easy as possible, and takes full advantage of the efficient and
  highly customizable \KPS{} library from Karl Berry and Olaf Weber.
  It can be run either directly from the \CD{} or installed on a hard
  disk.

Most of the runnable systems on the \CD{} include a large set of
drivers and support programs for \TeX, including \cmdname{dvips}
(PostScript driver), \cmdname{dvipdfm} (dvi to PDF),
\cmdname{xdvi} (X Windows previewer),
\cmdname{dvilj} (HP LaserJet driver), \cmdname{lacheck} (\LaTeX{}
syntax checker), \cmdname{tex4ht} (\TeX{} to HTML converter),
\cmdname{dviconcat} and \cmdname{dviselect}, \cmdname{dv2dt} and
  \cmdname{dt2dv} (\texttt{dvi} to ASCII and vice versa), and Angus
  Duggan's PostScript utilities.

\subsection{Extensions to \TeX}

The \TeXLive{} runnable systems contain three extended versions of 
standard \TeX:
\begin{enumerate}
\item \eTeX, which adds a small but powerful set of new primitives,
  and the \TeXXeT{} extensions for left to right typesetting; in
  default mode, \eTeX{} is 100\% compatible with ordinary \TeX. See
 \OnCD{texmf/doc/etex/base/etex_man.pdf} on the \CD{} for details.
\item pdf\TeX, which can optionally write Acrobat PDF format instead
  of \dvi{}. You will find the user manual in
  \OnCD{texmf/doc/pdftex/pdftex-l.pdf}. The
  file \OnCD{texmf/doc/pdftex/samplepdf/samplepdf.tex} shows how it is
  used. The \LaTeX{} \pkgname{hyperref} package has an option
  `\optname{pdftex}', which turns on all the program features.
\item \OMEGA{} (Omega), which works internally with 16-bit characters,
  using Unicode; this allows it to directly work with almost all the
  world's scripts simultaneously. It also supports dynamically loaded
  `\OMEGA{} Translation Processes' (\acro{OTP}s), which allow the user
  to define complex transformations to be performed on arbitrary
  streams of input. See \OnCD{texmf/doc/omega/base/doc-1.8.tex} for some
  (not necessarily up to date) documentation.
\end{enumerate} 
\eTeX{} (version 2.1) is stable, although subsequent releases will add
new functionality.  pdf\TeX{} (version 1.00b) is also stable,
but is still being improved. 
\OMEGA{} (version 1.23) is under development; the version
on this \CD{} is that current as of May 2002. 


\section[Structure and contents of the CD-ROM]{Structure 
  and contents of the \protect\CD{}}
The important \CD{} top-level directories are listed below.
\begin{description}
\item[bin] The \TeX{} family programs, arranged in separate platform
  directories.
\item[Books] Examples related to some books about \TeX.
\item[FAQ] Frequently Asked Questions, in English, French, and German.
\item[MacOSX] Support programs for MacOSX users
\item[info] Documentation in \acro{GNU} `info' format for the \TeX{}
  system.
\item[man] Documentation in the form of Unix man pages for the \TeX{}
  system.
\item[setupw32] directory contains material for installation and use
  under Windows (see section \ref{sec:wintex} on p.~\pageref{sec:wintex}).
\item[source] The source of all programs, including the main \Webc{}
  \TeX{} and \MF{} distributions. These are stored in a bzip2-compressed
  tar archive.
\item[support] Various bits of \TeX-related software which are
  \emph{not} installed by default, support programs, and a complete 
  distribution of \cmdname{Ghostscript} version 7.05.
  You can find here some other programs (editors, \TeX\ shells), which 
  are usually absent from Windows installations, dedicated for beginners.
  They can be installed with the \texttt{TeXSetup.exe} Windows installation
  program.
\item[texmf] The main support tree of macros, fonts and documentation;
\item[usergrps] Material about \TeX\ User Groups.
\end{description}

There are also two installation scripts for Unix systems, 
\texttt{install-cd.sh} and \texttt{install-pkg.sh}; we discuss
them on in section \ref{sec:install} on p.~\pageref{sec:install}.

\subsection[Packages and collections]{Packages and collections}
\label{packages}
The \TeXLive{} \texttt{texmf} tree consists of various `collections',
each of which refers to a set of `packages', of which there are over 700 
on the \CD{}. Normal installation allows the
user to copy all of a collection to a local hard disk from the \CD{}, but
it is also possible to install just one package of a collection.

The collections add functionality to a \TeX\ system. One of them, named
`tex-basic', is necessary for almost all \TeX\ tasks,
and two others, called `tex-latex' and `tex-pdftex' are highly
recommended for most users. All others are optional. 
The collections (defined by XML files in \texttt{texmf/tpm/collections})
and their short descriptions are listed below.

\begin{description}
     \item[tex-basic]
   These files are regarded as basic for any \TeX\ system, covering
plain \TeX\ macros, Computer Modern fonts, and configuration for common
drivers.
    \item[tex-bibtexextra]
      Additional, extensive libraries of BibTeX styles and bibliographies.
    \item[tex-chemistry]
      Essential chemistry
    \item[tex-context]
      Hans Hagen's powerful macro package, ConText
    \item[tex-documentation]
      Assorted useful documentation and guides
    \item[tex-etex]
      Support files for an extended \TeX\
    \item[tex-extrabin]
      Various useful, but non-essential, support programs. Includes
            programs and macros for texinfo system; programs for dvi file 
                  manipulation, etc.
    \item[tex-fontbin]
     Programs for conversion between font formats, testing fonts 
           (virtual fonts stuff, .gf and .pk manipulation, 
                 mft, fontinst, etc.)
    \item[tex-fontsextra]
      All sorts of extra fonts
    \item[tex-formatsextra]
      A collection of \TeX\ `formats', ie large-scale
macro packages designed to be dumped into .fmt file
    \item[tex-games]
      Setups for typesetting various board games, including chess
    \item[tex-genericextra]
      This is a mixed bag of macro packages and fonts which do not seem
to belong elsewhere
    \item[tex-htmlxml]
      Packages to convert \LaTeX\ to XML/HTML, and typeset XML/SGML
    \item[tex-langafrican]
      Support for some African scripts  
    \item[tex-langarmenian]
      Essential armenian
    \item[tex-langcjk]
Essential CJK (Chinese, Japanese, Korean) macros and fonts
    \item[tex-langcroatian]
      Essential croatian
    \item[tex-langcyrillic]
       Fonts and macro packages to typeset Cyrillic texts.
    \item[tex-langczechslovak]
      Pick this if you want Czech/Slovak fonts and other packages
    \item[tex-langdanish]
      Essential Danish
    \item[tex-langdutch]
      Essential Dutch
    \item[tex-langfinnish]
      Essential Finnish
    \item[tex-langfrench]
      Essential French
    \item[tex-langgerman]
      Essential German
    \item[tex-langgreek]
      Essential Greek
    \item[tex-langhungarian]
      Essential Hungarian
    \item[tex-langindic]
      Essential Indic
    \item[tex-langitalian]
      Essential Italian
    \item[tex-langlatin]
      Essential Latin
    \item[tex-langmanju]
      Essential Manju
    \item[tex-langmongolian]
      Essential mongolian
    \item[tex-langnorwegian]
      Essential Norwegian
    \item[tex-langother]
        Other languages
    \item[tex-langpolish]
      Pick this if you want Polish fonts and other packages
    \item[tex-langportuguese]
      Essential Portuguese
    \item[tex-langspanish]
      Essential Spanish
    \item[tex-langswedish]
      Essential Swedish
    \item[tex-langtibetan]
    Fonts and support for typesetting Tibetan
    \item[tex-langukenglish]
      Essential UK English
    \item[tex-langvietnamese]
      Essential Vietnamese
    \item[tex-latex]
      These packages are either mandated by the core \LaTeX\ team,
or very commonly recommended
    \item[tex-latexextra]
      A large collection of add-on packages for \LaTeX\
    \item[tex-mathextra]
      Extra math
    \item[tex-metapost]
      MetaPost (and MetaFont) drawing packages
    \item[tex-music]
      Music typesetting packages
    \item[tex-omega]
      Omega, a 16-bit extended \TeX\ by John Plaice and Yannis Haralambous
    \item[tex-pdftex]
      Support files for Han The Thanh's variant of \TeX\ which can
generate PDF output
    \item[tex-pictures]
      Essential graphics
    \item[tex-plainextra]
      Plain \TeX\ extra macros
    \item[tex-psfonts]
      Essential psfonts
    \item[tex-psutils]
      Utilities to manipulate PostScript files
    \item[tex-publishers]
      Essential publishers
    \item[tex-t1utils]
      Utilities to manipulate Type1 fonts
    \item[tex-texbooks]
      Examples and other material from various books about TeX/LaTeX.
    \item[tex-theses]
      Macro packages from various Universities for their thesis styles
    \item[tex-ttfutils]
      Utilities to manipulate TrueType fonts
    \item[win32-support]
         You can choose individual tools from this collection.
         There are many \TeX\ oriented editors, graphics files
         toolsets, etc.
\end{description}

The directory \texttt{texmf/tpm/packages} contains lists 
of all files in each package (used by the installation programs).

\section{Installation and use under Unix}
\label{sec:install}
You can use the \TeXLive{} \CD{} in three ways:
\begin{enumerate}
\item You can mount the \CD{} on your file system, run
the  \texttt{install-cd.sh} script, and select the option \verb|<R>| 
  (`do not install files, set up to run off CD-ROM'), 
  and run everything off the \CD; this takes very
  little disk space, and gives you immediate access to everything on
  the \CD; although the performance will not be optimal, it is
  perfectly acceptable on, for instance, PCs running Linux. You could
  also copy the entire CD contents to your hard disk and work in this way.
\item You can install all or part of the system to your local hard
  disk; this is the best method for many people, if they have enough
  disk space to spare (a minimum of about 100 megabytes, or 300
  megabytes for a recommended good-sized system).
\item You can install selected packages to work either with your
  existing \TeX{} system or a \TeXLive{} system you installed earlier.
\end{enumerate}
Each of these methods is described in more detail in the following
sections.

\ifSingleColumn
\begin{figure*}
  \begin{center}
    \leavevmode
\fi
\begin{warningbox}
\textbf{Warning: } This \CD{} is in ISO 9660 (High Sierra) format,
with Rock Ridge and Joliet extensions. In order to take full advantage
of the \CD{} on a Unix system, your system needs to be able to use
the Rock Ridge extensions.  Please consult the documentation for your
\cmdname{mount} command to see if it is possible.  If you have several
different machines on a local network, see if you can mount the
\CD{} on one which \emph{does} support Rock Ridge, and use this with
the others.

\leavevmode\quad
Linux, FreeBSD, Sun, SGI and DEC Alpha systems should be able to use
the \CD{} with no problems. We would appreciate receiving detailed advice
from other system users who also succeed, for future versions of this
documentation.

\leavevmode\quad
The discussion below about installation assumes you have been able to mount
the \CD{} with full Rock Ridge compatibility.
\end{warningbox}
\ifSingleColumn
  \end{center}
\end{figure*}
\fi

%\begin{addbymb}
\subsection[Pre-installation procedure for MacOSX users]
{Pre-installation procedure for MacOSX users}
If you do not run MacOSX, you should skip this section.

The \texttt{install-cd.sh} script is a \emph{sh} script (begins with 
``\verb=#!/bin/sh=''), but
on MacOSX \cmdname{sh} is unable to run it because 
\cmdname{sh} is 
emulated. However, \cmdname{bash} will run it. Unfortunately (again) 
\cmdname{bash} is not 
installed by default on MacOSX\footnote{there are some suggestions this will 
change in the future and even 
that \cmdname{bash} will be used to emulate \cmdname{sh},
in which case it might be true 
that in future versions of MacOSX, the script will just work.}.
\begin{enumerate}
\item\label{test-bash} (optional) See if \path|bash| is already installed. 
Launch \path|Terminal|
(\path|/Applications/utilities/Terminal|) and type in a window
\begin{alltt}
>> \Ucom{rehash; which bash}
\end{alltt}
the answer will be:
\begin{itemize}
\item the bash location (e.g. \path|/bin/bash| or  \path|/usr/local/bin/bash|) if installed;
\item \texttt{bash: command not found} if not installed.
\end{itemize}

If bash is already installed, skip to \ref{std-install}

\item bash installation:
\begin{description}
\item[Mac friendly procedure] 
Look in the \path|MacOSX| directory of the \CD{}
for an image disk named \path|bash.dmg|. mount that 
file by double-clicking it. The disk image (volume) will be mounted. 
Then start the \path|i-Installer| application on that volume. You will be asked 
to authenticate, if you have never seen that before, you might not have 
enough privileges to install. Just enter your own user name and 
password. Hit install. \path|bash| will be installed on your system.
\item[Terminal procedure]
\begin{enumerate}
\item Log in as an admin user, at least a user with Admin privileges or 
sudo user or as the System Administrator.
\item Open the \path|MacOSX| directory on the \CD{} and copy 
\path|bash.tar.gz| 
to your home directory (\texttt{\~{}/})
\item Then launch \path|Terminal| and type or copy/paste the 
line below  in a Terminal window):
\begin{alltt}
>> \Ucom{(cd /usr/local/; sudo tar xvzf ~/bash.tar.gz)}
\end{alltt}
type a carriage-return: you will be asked for your password, and 
\path|bash| will be 
installed.
\item Quit \path|Terminal|.
\end{enumerate}
\end{description}
\item Now after using either install method, goto \ref{test-bash}:
you must obtain \path|/usr/local/bin/bash|\ldots{} (if not, try to log out and in).
\item\label{std-install} The installation procedure is the 
same on MacOSX than on other UNIX platforms (as MacOSX is UNIX, 
that's quite normal!). Nevertheless, you should read what follows:
\begin{itemize}
    \item Note that in all commands of the following sections, 
\cmdname{sh} must be replaced by \texttt{sudo bash}:
\begin{alltt}
>> \Ucom{sh install-cd.sh}
\end{alltt}
becomes
\begin{alltt}
>> \Ucom{sudo bash install-cd.sh}
\end{alltt}
etc.
\item On MacOSX, cd are auto-mounted: you \emph{don't} need to use 
\texttt{mount}. The \CD{} will be mounted in the \path|/Volumes/| 
directory: in order to make it the current directory, you just have to 
type in \path|Terminal|:
\begin{alltt}
>> \Ucom{cd /Volumes/TeXLive-7...}
\end{alltt}
(complete this line with the real name of the \CD{}: using 
``auto completion'' by pressing the \texttt{<tab>} 
key will do it).
\end{itemize}
\end{enumerate}
%\end{addbymb}

\subsection[Running \protect\TeXLive{} from the CD-ROM]{Running \protect\TeXLive{} from the \protect\CD}
The organisation of \Webc{} means that you can run programs simply by
adding the appropriate directory under \path|bin| on the \CD{} to
your \texttt{PATH}, and the support files will all be found with no further
ado.  The following shows the list of available systems and the
corresponding directories. \textbf{Only x86 Linux, Mac OSX, and
  Windows are available on the default CD. You need to ask
for the Unix CD if you need the other systems.}

\begin{flushleft}
\begin{tabular}{@{}l>{\ttfamily}ll}
Compaq Alpha Linux & alpha-linux & CD2\\
Compaq Alphaev5 OSF 4.0d & alphaev5-osf4.0d  & CD2\\
HP9000 HPUX 10.20 & hppa2.0-hpux10.20  & CD2\\
IBM RS 6000 AIX 4.2.* & rs6000-aix4.2.1.0  & CD2\\
Intel x86 Solaris 2.8 & i386-solaris2.8 & CD2\\
Intel x86 with GNU/Linux & i386-linux  & CD1\\
Mac OSX & powerpc-darwin5.3 & CD1\\
Sun Sparc Solaris 2.7 & sparc-solaris2.7  & CD2\\
Windows 9X/ME/NT/2K/XP & win32  & CD1\\
\end{tabular}
\end{flushleft}

You may worry that when you subsequently make fonts or change
configuration, things will go wrong because you cannot change files on
the \CD{}. However, you can maintain a parallel, writeable, \TeX{}
tree on your hard disk; this is searched before the main tree on the
\CD{}. The default location is \path|texmf-var| on the CD
(which does not exist!), so you \emph{must} override this by setting
the \texttt{VARTEXMF} environment variable.


Thus \emph{sh} or \emph{bash} users on an Intel PC
running Linux can mount the \TeXLive{} \CD{} on
\file{/mnt/cdrom} by issuing the command:

\begin{alltt}
>> \Ucom{mount -t iso9660 /dev/cdrom /mnt/cdrom}
\end{alltt}
Then they should change the current directory to \path|/mnt/cdrom|, run
\begin{alltt}
>> \Ucom{sh install-cd.sh}
\end{alltt}
and select the option \verb|<R>| (\emph{do not install files, set up to run
off CD-ROM}). After that, they should include the directory containing
the binaries for the given architecture into the search path
by updating the \texttt{PATH} variable.
\begin{verbatim}
PATH=/mnt/cdrom/bin/i386-linux:$PATH
export PATH
VARTEXMF=/usr/TeX/texmf-var
export VARTEXMF
\end{verbatim}
%\begin{addbymb}
\begin{description}
    \item[MacOSX users] On MacOSX, the default shell is \texttt{tcsh}:
\begin{verbatim}
setenv PATH /Volumes/<cd-name>/bin/powerpc-darwin5.3:${PATH}
setenv VARTEXMF /usr/TeX/texmf-var
\end{verbatim}
\end{description}
%\end{addbymb}    
For convenience, these statements can also be entered into the
\texttt{.profile} script.
%\begin{addbymb}
(for \texttt{tcsh} on MacOSX, 
\verb=~/Library/init/tcsh/rc.mine=).
%\end{addbymb} 

If in doubt, ask your local system support guru to help you work out
how to mount your \CD{} or which directory to use for your system.

Appropriate support files will be installed on your hard disk the
first time you need them. You can edit and change local configuration
files which are stored to the directory designated
by \texttt{\$VARTEXMF}. Any format file that is needed
will be generated and stored here.

%------------------------------------------------
\subsection{Installing \protect\TeXLive{} to a hard disk}
All of the necessary steps to install all or part of the distribution
on your hard disk are achieved by mounting the \CD{}, changing to the 
top-level directory, and typing:
\begin{alltt}
>> \Ucom{sh install-cd.sh}
\end{alltt}
(On some Unix systems, you may need to use \cmdname{sh5} or
\cmdname{bash}.)  This script works by accessing lists of collections
and packages from the \CD{}, and trying to guess what sort of computer
system you are on. It should start by displaying the following:
\verbatiminput{../examples/ex1.tex}
It will then show the main control screen
(Figure~\ref{ins1}), which lets you change five
things:
\begin{enumerate}
\item the type of system you are on, or want to install for;
\item the \emph{installation scheme} you want to use (eg full,
  recommended, basic etc)
\item the collections you want to change from
the installation scheme (they are organised into two sets:
\emph{standard collections} and \emph{language collections});
\item the location on your hard disk to put the files;
\item some runtime behaviour features.
\end{enumerate}

\begin{figure*}
\begin{center}
\ifnum\Status=2
\boxedverbatiminput{../examples/ex2.tex}
\else
\begin{minipage}{0.8\textwidth}
\boxedverbatiminput{../examples/ex2.tex}
\end{minipage}
\fi
\caption{Main control screen}\label{ins1}
\end{center}
\end{figure*}

You choose options by typing a letter or number and pressing `return'.
In the example, a Linux GNU/Linux system has been detected, the
default set of collections will be installed, and
the default installation directory is \path|/usr/TeX|; note
that the disk space required for the current installation
configuration is also displayed. If you make a suggested setup, you
need about 60 megabytes of disk free; however, the basic setup will
only take about 30 megabytes, and you can enhance it with selected
packages as you need them. 

%\begin{addbymb}
\begin{description}
    \item[MacOSX users] Most frontends (\emph{TeXShop}, 
    \emph{ITeXMac}\ldots) use the \teTeX{} default location
    which is \path|/usr/local/teTeX|, so, Mac users could find interest
    in installing \TeXLive{} in \path|/usr/local/teTeX| 
rather than in \path|/usr/TeX|.
\end{description}
%\end{addbymb}

Under the directory you choose for installation, the installation
script will put the binaries in a subdirectory of \path|bin|, and the
support tree in \path|texmf|. An additional tree \path|texmf-var|
will contain copies of configuration files (except the main 
\path|texmf.cnf|), which are to be modified by \path|texconfig| program.
This tree will also store generated format files for \TeX, \MF, etc.

\begin{figure*}
\begin{center}
\ifnum\Status=2
\boxedverbatiminput{../examples/ex3.tex}
\else
\begin{minipage}{.80\linewidth}
\footnotesize
\boxedverbatiminput{../examples/ex3.tex}
\end{minipage}
\fi
\caption{Selecting standard collections}\label{ins2}
\end{center}
\end{figure*}

\begin{figure*}
\begin{center}
\ifnum\Status=2
\boxedverbatiminput{../examples/ex4.tex}
\else
\begin{minipage}{.80\linewidth}
\footnotesize
\boxedverbatiminput{../examples/ex4.tex}
\end{minipage}
\fi
\caption{Selecting language collections}\label{ins3}
\end{center}
\end{figure*}

When you choose |<C>| for \emph{standard collections}, you will see
the display of available collections (Figure \ref{ins2}).  Each
collection --- TeX macro files, Metafont font families, and so on ---
consists of several packages.  You can toggle their inclusion on or
off by pressing the key.  Note that the selection letter keys are case
sensitive.

When you choose |<L>| for \emph{language collections}, you will see
the display of available language support collections (Figure
\ref{ins3}). Each collection consists of several packages, which
provide features like hyphenation files and fonts.

The |<O>| for \emph{options} item lets you decide whether to make new fonts
be created in another location (if you want the main package mounted
read-only for most users), and whether to make symbolic links for the
binaries, \cmdname{man} and \acro{GNU} \cmdname{info} pages in the `standard'
locations; you'll need `root' permissions for tasks to do this, of course.

When you are finished, return to the main screen, and ask the
installation to start. It will take each of the collections and
systems that you requested, consult the list of files on the \CD{},
and build a master list of files to transfer. These will then be
copied to your hard disk. If you installed a system, 
an initialization sequence is now run
(creating format files, etc.). When this has finished, all you need do
is add the correct subdirectory of \texttt{bin} in the \TeX{}
installation to your path, and start using \TeX. If you want, you can
move the binaries up one level, e.g.\ from
\path|/usr/local/bin/alpha-osf3.2|
to \path|/usr/local/bin|; if you do this, however, you must 
edit \path|texmf/web2c/texmf.cnf| (see Appendix~\ref{app:texmf.cnf})
and change the line near the start which reads
\begin{verbatim}
  TEXMFMAIN = $SELFAUTOPARENT
\end{verbatim}
to
\begin{verbatim}
  TEXMFMAIN = $SELFAUTODIR
\end{verbatim}

If you move the whole installation to another directory tree entirely, you
need to edit \envname{TEXMFMAIN} to specify the support tree explicitly, and
set \envname{TEXMFCNF} in your environment to \path|$TEXMFMAIN/texmf/web2c|.

%------------------------------------------------

\subsection{Installing individual packages from \protect\TeXLive{} to a hard
  disk} You may want to use the \TeXLive{} \CD{} to either update an
existing setup, or add features to an earlier installation from the
\CD{}. The main installation program is intended for the first time
only, and subsequently you should use the \texttt{install-pkg.sh}
script on the \CD{}. Run this by mounting the \CD{}, changing to
the mounted directory, and typing

\begin{alltt}
>> \Ucom{sh install-pkg.sh \emph{options}}
\end{alltt}


The script supports nine options; the first four let you set the
individual package you want to install, the whole collection (i.e.,
\texttt{tex-mathextra}), the name of the mounted \CD{} directory, and
the name of the directory containing the list files (normally these
latter two will be set automatically):

\begin{tabular}{ll}
\texttt{-{}-package=}\emph{name} & \\
\texttt{-{}-collection=}\emph{name} & \\
\texttt{-{}-cddir=}\emph{name} & \\
\texttt{-{}-listdir=}\emph{name} & \\
\end{tabular}

What actually happens is controlled by four more switches; the first
two allow you to exclude documentation or source files from the
installation, the third stops the default action of running
\texttt{mktexlsr} on completion to rebuild the file database, and
the last does nothing but list the files that would be installed:

\begin{tabular}{ll}
\texttt{-{}-nodoc} & \\
\texttt{-{}-nosrc} & \\
\texttt{-{}-nohash} & \\
\texttt{-{}-listonly} & \\
\end{tabular}

Finally, you can specify that, instead of installing the files, the script
should make a \cmdname{tar} archive in a specified location:

\begin{tabular}{ll}
\texttt{-{}-archive=}\emph{name} & \\
\end{tabular}

Thus, if we simply wanted to see the files that make up the package
\pkgname{fancyhdr} before we installed it, our command and output would be
as follows:

\begin{alltt}
\ifSingleColumn>> \Ucom{sh install-pkg.sh --package=fancyhdr --listonly}
\else>> \Ucom{sh install-pkg.sh -{}-package=fancyhdr \bs}
>>                           \Ucom{-{}-listonly}
\fi{}
texmf/doc/latex/fancyhdr/fancyhdr.dvi
texmf/doc/latex/fancyhdr/fancyhdr.tex
texmf/lists/fancyhdr
texmf/source/latex/fancyhdr/README
texmf/source/latex/fancyhdr/fancyheadings.new
texmf/tex/latex/fancyhdr/extramarks.sty
texmf/tex/latex/fancyhdr/fancyhdr.sty
texmf/tex/latex/fancyhdr/fixmarks.sty
\end{alltt}
Other examples of usage are:
\begin{itemize}
\item Install the \LaTeX{} package \texttt{natbib}:
\begin{alltt}
>> \Ucom{sh install-pkg.sh --package=natbib}
\end{alltt}
\item Install the \LaTeX{} package \texttt{alg} with no source files and no
documentation:
\begin{alltt}
\ifSingleColumn>> \Ucom{sh install-pkg.sh  --package=alg --nosrc --nodoc}
\else>> \Ucom{sh install-pkg.sh  -{}-package=alg \bs}
>>                  \Ucom{-{}-nosrc -{}-nodoc}
\fi\end{alltt}
\item Install all the packages available in the collection containing additional
Plain \TeX\ macros:
\begin{alltt}
>> \Ucom{sh install-pkg.sh --collection=tex-plainextra}
\end{alltt}
\item Place all files which are needed for PSTricks 
in a \cmdname{tar} file in \path|/tmp|:
\begin{alltt}
\ifSingleColumn>> \Ucom{sh install-pkg.sh --package=pstricks --archive=/tmp/pstricks.tar}
\else
>> \Ucom{sh install-pkg.sh -{}-package=pstricks \bs}
>>          \Ucom{-{}-archive=/tmp/pstricks.tar}
\fi\end{alltt}
\end{itemize}

\subsection{The texconfig program}
\label{ssec:tex}

After the installation program has copied all files to their final
locations, you can use a program called \cmdname{texconfig} that
allows you to configure the system to fit your local needs. This can
be called at any other time to change your setup, with a full-screen
(which requires the \cmdname{dialog} program, included as part of the
binary packages) or command-line interface.
It should be used for all maintenance, such as changes of installed
printers, or rebuilding the file database. Both modes have help text
to guide you through the facilities.

\section{Installation and use under Windows}
\label{sec:wintex}
This section only applies to systems running Windows 9x, \acro{ME},
\acro{NT}, \acro{2K} or \acro{XP}.

It is also necessary to have your Windows set up so that it uses the
Microsoft Joliet extensions for reading \CD s; simply look at the
\CD{} in Explorer and see whether it shows long, mixed-case, file
names. If it does not, you cannot use the ready-to-run system on the
\CD.

This Win32 \TeX{} systems includes a \texttt{dvi} previewer,
\textsf{Windvi}, which is similar in usage to the established Unix
\textsf{xdvi}. The documentation can be found in
\OnCD{texmf/doc/html/windvi/windvi.html}.

\subsection[The TeXLive.exe program]{The \texttt{TeXLive.exe} program}

\begin{figure*}
 \begin{center}
  \ifnum \Status=1
    \includegraphics[width=.7\textwidth]{../pictures/Welcome-to-TeXLive}
  \else
    \ifnum \Status=2
\includegraphics[bb=20 20 575 432]{../pictures/Welcome-to-TeXLive.gif}
    \else
        \includegraphics[width=.7\textwidth]{../pictures/Welcome-to-TeXLive}
    \fi
  \fi
 \end{center}
 \caption{``Welcome to \TeXLive'' window}\label{graph:welcome}
\end{figure*}

If your computer is configured to let the \CD{} autostart, then a
dialog box with a menu bar will popup on the screen, and you will have several choices
from there:
\begin{itemize}
\item Install \TeX{} on your hard disk,
\item Install TeX-oriented editors on your hard disk,
\item Install support packages on your hard disk (Ghostscript, NetPBM,
  \ldots)
\item Do some maintenance on you \TeX{} system,
\item Remove the \TeX{} system,
\item Use \TeX{} off the \CD{},
\item Cleanup the temporary files created on your hard disk when using
  \TeX{} off the \CD{},
\item Update some of the DLLs on your system,
\item Browse some documentation: \TeXLive{} documentation, TUG web
  pages, \fpTeX web pages,
\item Run the \cmdname{TeXdocTK} application to find specific documentation.
\end{itemize}

If your \CD{} does not autostart, you can explicitly run the program
by double clicking on \path|bin/win32/TeXLive.exe| on the \CD{} from
the explorer window.

\subsection[Running \protect\TeXLive{} from the CD-ROM]{Running from 
the \protect\CD{}}
You can run all the \TeX{} programs directly off the \CD, and have
access to all the macros and fonts immediately, at the price of a
slower performance than if you install on the hard disk. 
To work effectively, one needs to modify
environment variables and to create some small auxiliary directories on a
hard disk.  These directories will contain necessary configuration files
allowing the user to modify programs settings and to generate a necessary
format file. Moreover, automatically generated font files will be stored
there too.

Should you want to run \TeX{} this way, you will have to follow these
steps:
\begin{enumerate}
\item from the menu, chose \verb|Explore CD-Rom|, then  
  \verb|Select a text editor|, a dialog box will open to select some \path|.exe|
  program. 
  
  This program needs to be a \TeX{} oriented editor. It must be able
  to run the \TeX{} compiler, previewer and any other needed tool. If
  you don't have one already installed on your system, you can install
  one from the \CD{}, details section~{\ref{sec:texlive-install}}.

  \emph{There is no way  we can guess if the program you will
    select is actually a text editor, so be careful}.
  Here is a list of frequently used \TeX{} editors:
\begin{center}
  \begin{tabular}[ht]{ll}
    GNU Emacs & \path|c:\Program Files\NTEmacs\bin\runemacs.exe| \\
    XEmacs & \path|c:\Program Files\XEmacs\XEmacs-21.2\i586-pc-win32\xemacs.exe| \\
    WinShell & \path|c:\Program Files\WinShell\WinShell.exe| \\
    WinEdt & \path|c:\Program Files\WinEdt Team\WinEdt\WinEdt.exe|\\
    TeXnicCenter & \path|c:\Program Files\TeXnicCenter\TEXCNTR.exe|
  \end{tabular}
\end{center}
 The program
  selected will be memorized as the editor to use for future runs.

\item from the menu, chose \verb|Explore CD-Rom|, then  
  \verb|Run TeX off CD-Rom|. The environment will be modified, a small
  temporary directory created and some configuration files copied
  there. Then, the selected editor selected will be launched, and you
  will be able to type in some text, let \TeX{} typeset it and the
  view it or print it.

  If Ghostscript is not detected on your machine, you will be warned
  that rendering your DVI files might fail. You can install it from
  the \path|Install|, \path|Support| menu item. See
  section~\ref{sec:texlive-install} for details.

\item you can select a different text editor any time you want.

\item if you chose \verb|Cleanup CD-Rom setup|, everything \TeX{}
  needed will be  removed, comprised the selection of your text
  editor, but not the extra packages you may have downloaded and
  installed. If you installed \cmdname{WinShell} or \cmdname{NTEmacs}, 
they won't be removed.
\end{enumerate}

The editor is run inside a modified environment. A temporary TDS
compliant texmf tree is build in the temporary area of your computer.
It is needed to store files that may be build on the fly like pk font
files or format files. Configuration files are copied from the \CD{}
to this texmf tree, so that you can edit them if needed.  The \path|ls-R|
database is computed for this texmf tree.
Then  the \texttt{PATH} and \texttt{TEXMFCNF} environment
variables are set locally, and the  editor is run in
this local environment. From within your editor\footnote{Actually, you can
  state any other program than a text editor, and select your command
  processor for example. You will then get a console with the right
  settings to use \TeX{} from the \CD{}.}, you have access to a full
\TeXLive{} environment, all files referenced on the \CD{}.


\smallskip {\small\noindent \textbf{[For advanced users:]} You can
  also use the small batch file \path|mkloctex.bat| to be called in a
  directory \path|setupw32| of the \CD.  From the Start menu select
  `Run', then browse CD drive and select \path|mkloctex.bat|. Before
  starting it, you should add two parameters separated by a space: the
  letter of your CD drive and the letter of the hard disk where you
  want to install the \TeX\ directory.  The whole line should read,
  e.g., \verb+d:\setupw32\mkloctex.bat d c+.  When installation is
  complete, please read carefully the information on screen. If you
  are running Windows 9x/ME, then you will have to restart Windows.  }

\subsection{Installing editors or support packages}
\label{sec:texlive-install}

You can already use the \path|TeXSetup.exe| program to install a
single, not \TeX{} dependent package. This might be either an editor
like \cmdname{WinShell} or \cmdname{NTEmacs},
or also a support package like \cmdname{NetPBM}
(graphics formats conversion) or \cmdname{Ghostscript}.

Some of the packages are not free, or not with the same meaning as for
the rest of the \CD{}. These packages have been made available through
the Internet. You need to enable an Internet connection in order to
install them. Chosing the \path|Enable Internet access| subitem will
search your system for an active Internet connection, or start one if
possible. \emph{If your computer is not connected to the Internet,
  then the timeout might be long, 30s or more}. So try to enable it
only if you know you have a connection.

Only a few packages are available from the \CD{}, but the most
important of them: \cmdname{NTEmacs} and \cmdname{WinShell} for the
editors, \cmdname{Ghostscript} and \cmdname{NetPBM} for the other
support packages. \cmdname{NetPBM} is needed for running \TeX{}4ht.

The downloadable packages are sometimes huge: \cmdname{Perl} is 10Mb,
\cmdname{XEmacs} is 50Mb, so be warned that it can take a lot of time
to install such things. \path|TeXSetup.exe| does not yet provide an
estimation of the time needed to complete the download.

When installing these packages, \path|TeXSetup| is working in
unattended mode. However, the programs that have their own installer
like WinEdt or Ghostscript for example will require human interaction.

Those packages who have no specific installer will be unpacked and configured
for your system. You will be required to select some directory where
they will be installed. The directory to select should be the root
directory of the whole installation. Assuming you want to install
NTEmacs and NetPBM, the archive files already contain the
\path|NTEmacs\| and \path|NetPBM\| part, so the directory you have to
select is something like \path|c:\Local| or \path|c:\Program Files|.

\subsection{Installing to your hard disk}

Installation is started  by letting the CD autostart, and
selecting the item \verb|Install| from the menu, then the subitem
\verb|TeXLive|. This will invoke the \path|TeXSetup.exe|. You can also find
it in the \path|bin/win32| directory and run it. \path|TeXSetup.exe|
is a Windows wizard and it will display several pages while running.

\begin{description}
\item[Welcome Page] 
  You can choose a \emph{quick} installation from
  there, in this case, the installation will run without any human
  assistance from beginning to end, with all the default settings
  (Figure~\ref{graph:setup-src},
  \ifnum \Status = 1
  on the left%
  \else\ifnum \Status=2
  at the top%
  \else
  on the left%
  \fi\fi
  ). However, if you chose to install any support program that has its
  own installer like WinEdt or Ghostscript, your intervention might be required.
  If you have enough privileges (administrator or power user rights)
  under a Windows version where it is applicable, then you can decide
  to install the \TeXLive{} for all users or for yourself only by
  checking the appropriate box.
  
  \begin{figure*}[!htb]
  The \cmdname{TeXSetup} Wizard\hfill
  Source directories for the \TeXLive{} files

 \begin{center}
  \ifnum \Status=1
    \includegraphics[width=.48\textwidth]{../pictures/setup-wizard.jpg}\hfill%
    \includegraphics[width=.48\textwidth]{../pictures/source-for-texlive.jpg}
  \else
    \ifnum \Status=2
\includegraphics[bb=0 0 506 403]{../pictures/setup-wizard.gif}
\includegraphics[bb=0 0 506 403]{../pictures/source-for-texlive.gif}
    \else
        \includegraphics[width=.48\textwidth]{../pictures/setup-wizard}%
                        \hfill%
        \includegraphics[width=.48\textwidth]{../pictures/source-for-texlive}
    \fi
  \fi
 \caption{The \TeXLive{} setup wizard}\label{graph:setup-src}
 \end{center}
\end{figure*}
\item[Source Page] 
  This page is a bit complex. It will allow you to select two source
  directories for your \TeXLive{} system
  (Figure~\ref{graph:setup-src}, 
  \ifnum \Status = 1
  on the right%
  \else\ifnum \Status=2
  at the bottom%
  \else
  on the right%
  \fi\fi
  ). You will need a \emph{local
  source directory} and possibly a \emph{remote source directory}.

  Why do we need both these directories? The very files of the
  \TeXLive{} system are on the \CD{}, but some other packages useful
  under a Win32 system are not, either because of space lacking or
  because their license was not compatible with the \TeXLive{}'s one.
  You need to enable Internet downloading if you want to install these
  support packages.

  However, don't panic: the default parameters of the setup will allow
  you to install a full system using the \CD{} only. Simply, you won't
  have \file{WinEdt} for example, but you will be able to install it
  later.

  So you can take your files from:
  \begin{itemize}
  \item the \CD{} or any similar flat tree of files, available through
    some standard directory (this means the \CD{} can be mounted on
    some remote machine and be made available through network
    sharing),
  \item a set of  \file{.zip} files (this is the case when you are
    installing the
    \fpTeX{} distribution),
  \item the Internet, in this case, the program takes care of
    downloading the \file{.zip} files it needs for you.
  \end{itemize}
  This option is available only if you enable Internet file
  downloading in the right part of the page. You also need to
  configure this Internet access by selecting to connect either using
  Internet Explorer~5 \file{wininet.dll}, or using a direct connection (\texttt{ftp},
   \texttt{http}).
% or using a proxy server.
%   Last, you can be assisted in defining the \emph{local source
%     directory} and \emph{remote source directory} which will be used
%   to copy the files onto your hard disk. The \texttt{browse} buttons
%   allow to select a directory for the former, and an url among a list
%   of mirrors for the latter. 
\item[Root Page] 
  On this page, you will tell where you want the files to be
  installed (Figure~\ref{graph:root-schm}, 
  \ifnum \Status = 1
  on the left%
  \else\ifnum \Status=2
  at the top%
  \else
  on the left%
  \fi\fi
  ). Only the root directory really matters, the other ones
  are set according to the root one. You may want to make
  \path|$TEXMFEXTRA| point to some TDS compliant directory with other
  \TeX{} files or assign a different value to 
  \path|$HOMETEXMF|, which is set by default to whatever Windows think
  is your `HOME' location.
\begin{figure*}[!htb]
  Root and directories\hfill%
  Scheme selection
 \begin{center}
  \ifnum \Status=1
    \includegraphics[width=.46\textwidth]{../pictures/root-of-installation.jpg}\hfill%
    \includegraphics[width=.46\textwidth]{../pictures/scheme-selection.jpg}
  \else
    \ifnum \Status=2
\includegraphics[bb=0 0 506 403]{../pictures/root-of-installation.gif}
\includegraphics[bb=0 0 506 403]{../pictures/scheme-selection.gif}
    \else
        \includegraphics[width=.46\textwidth]{../pictures/root-of-installation}%
                        \hfill%
        \includegraphics[width=.46\textwidth]{../pictures/scheme-selection}
    \fi
  \fi
 \caption{\TeXLive-Setup: Root and directories / Schemes}\label{graph:root-schm}
 \end{center}
\end{figure*}
\item[Get TPM Page] 
  This page does not require any manual intervention. The \file{.tpm}
  files which describe collections and packages are retrieved
  (possibly from the Internet), unzipped if needed and parsed.
\item[Schemes Page]
  On this page, you will select the global scheme of your
  installation (Figure~\ref{graph:root-schm}, 
  \ifnum \Status = 1
  on the right%
  \else\ifnum \Status=2
  at the bottom%
  \else
  on the right%
  \fi\fi
  ). A scheme is a large set of files targeted at some kind
  of usage. There are 3 generic schemes for basic, recommended and
  full installation. The other ones are devoted to LUGs (what GUST or
  GUTenberg propose for their members) or applications (XML and
  \TeX{}).
  When a scheme is selected, it is still possible to refine the
  selection by checking the appropriate box. If doing so, you will be
  presented the packages page to change your selection, else you will
  jump to the review page.
\item[Packages Page] 
  Collections and packages are presented in a tree form
  (Figure~\ref{graph:win32-support}, 
  \ifnum \Status = 1
  on the left%
  \else\ifnum \Status=2
  at the top%
  \else
  on the left%
  \fi\fi
  ). The links
  in the tree are dependency links. Collections \emph{depend on}
  packages and maybe other collections, and it is the same for each
  package. You can select any package or collection individually, but
  your request will be granted only if the object is not requested by
  another one which is selected. For example, you can't deselect
  \texttt{tex-basic} without deselecting all the collections that
  request it.

  The \texttt{win32-support} collection displayed on the picture is Win32 specific. It
  holds a number of bonus packages (Figure~\ref{graph:win32-support}, 
  \ifnum \Status = 1
  on the right%
  \else\ifnum \Status=2
  at the bottom%
  \else
  on the right%
  \fi\fi
  ) which can be installed
  automatically and individually: Ghostscript, the PostScript
  interpreter, \TeX{} oriented editors, tools like Perl,
  \LaTeX{}2HTML, etc. \emph{None of these packages are selected by
    default}. Some of them have an Internet Explorer icon on their
  right, this means that they are not on the \CD{} and they will be
  available only if you previously enabled  Internet
  downloading. \emph{This collection cannot be selected entirely at
    once: you need to select the packages individually}. This is to
  avoid unwanted downloads of huge files.

  On this page, you also have the information about disk space
  needed, for each object, and for the set of those who are selected,
  and also the disk space available on the partition selected for the
  installation. Last, you can choose to install or not the
  documentation files and source files associated with each package.
\item[Review Page] 
  You will find there a summary of your choices
  (Figure~\ref{graph:review}, 
  \ifnum \Status = 1
  on the left%
  \else\ifnum \Status=2
  at the top%
  \else
  on the left%
  \fi\fi
  ). It is still time to
  go back to change them.
\begin{figure*}[!htb]
  Packages Page\hfill%
  Win32 Support
  \begin{center}
  \ifnum \Status=1
    \includegraphics[width=.48\textwidth]{../pictures/package-selection}\hfill%
    \includegraphics[width=.48\textwidth]{../pictures/win32-support}
  \else
    \ifnum \Status=2
\includegraphics[bb=0 0 506 403]{../pictures/package-selection.gif}\hfill%
\includegraphics[bb=0 0 506 403]{../pictures/win32-support.gif}
    \else
        \includegraphics[width=.48\textwidth]{../pictures/package-selection}%
                          \hfill%
        \includegraphics[width=.48\textwidth]{../pictures/win32-support}
    \fi
  \fi
 \end{center}
 \caption{Packages Page / Win32 goodies}
 \label{graph:win32-support}
\end{figure*}
\item[Files Copy Page] 
  The selected files are copied on your hard disk   (Figure~\ref{graph:file-copy}, 
  \ifnum \Status = 1
  on the right%
  \else\ifnum \Status=2
  at the bottom%
  \else
  on the right%
  \fi\fi
  ). All the files not
  yet available on your local disk are first downloaded from the
  remote source directory on the Internet. Then every package is
  unpacked (if \path|.zip| files), or copied from the \CD{}.
\item[Configuration Page]  Several  packages  need some  configuration
  step to  make them usable (Figure~\ref{graph:configuration},  
  \ifnum \Status = 1
  on the left%
  \else\ifnum \Status=2
  at the top%
  \else
  on the left%
  \fi\fi
  ). Also the  \TeXLive{} system  needs some post-processing  step
  (format  files generation,  ls-R   databases generation, environment
  variables, etc.). All these operations are  done there, some of them
  can be lengthy.
\begin{figure*}[!htb]
  \textbf{Review Page}\hfill%
  \textbf{File Copy Page}
  \begin{center}
  \ifnum \Status=1
      \includegraphics[width=.48\textwidth]{../pictures/review-settings}\hfill%
      \includegraphics[width=.48\textwidth]{../pictures/file-copy}
  \else
    \ifnum \Status=2
          \includegraphics[bb=0 0 506 403]{../pictures/review-settings.gif}\hfill%
          \includegraphics[bb=0 0 506 403]{../pictures/file-copy.gif}
    \else
          \includegraphics[width=.48\textwidth]{../pictures/review-settings}\hfill%
          \includegraphics[width=.48\textwidth]{../pictures/file-copy}%
    \fi
  \fi
 \end{center}
 \caption{Review Page / File Copy Page}
 \label{graph:review}\label{graph:file-copy}
\end{figure*}
\item[Final Page] The installation being over, you may want to display
  the Windows specific documentation (HTML format) and / or the log
  file of the setup process  (Figure~\ref{graph:final}
  \ifnum \Status = 1
  on the right%
  \else\ifnum \Status=2
  at the bottom%
  \else
  on the right%
  \fi\fi
  ). If it is needed (Win9x/WinME), you will
  be asked to reboot your computer.
\end{description}

\begin{figure*}[!htb]
  \textbf{Configuration Page}\hfill%
  \textbf{Final Page}
  \begin{center}
  \ifnum \Status=1
      \includegraphics[width=.48\textwidth]{../pictures/configuration}\hfill%
      \includegraphics[width=.48\textwidth]{../pictures/congratulations}
  \else
    \ifnum \Status=2
          \includegraphics[bb=0 0 506 403]{../pictures/configuration.gif}\hfill%
          \includegraphics[bb=0 0 506 403]{../pictures/congratulations.gif}
    \else
          \includegraphics[width=.48\textwidth]{../pictures/configuration}\hfill%
          \includegraphics[width=.48\textwidth]{../pictures/congratulations}%
    \fi
  \fi
 \end{center}
 \caption{Configuration Page / Final page}
\label{graph:configuration}\label{graph:final}
\end{figure*}

Please be aware that the choice of cluster size on DOS disk
partitions can radically affect the size of your \TeX\
installation. The support tree has hundreds of small files, and it is
not unusual for a complete installation to take up to 4 times the
amount of space used on the \CD.


\section[Maintenance and other aspects of the \protect\TeXLive{}
installation under Windows]{Maintenance and other aspects of the \protect\TeXLive{}
installation under Windows}

\subsection{What's different under Win32 from the standard \Webc ?}
The Win32 version of \Webc{} has some specific features that need to
be noticed.

\begin{description}
\item[\KPS{}] the hash-tables that \KPS{} builds are quite large for
  the \TeXLive{}. In order to cut down the starting time of any
  \KPS{}-enabled program, these hash-tables have been put in shared
  memory. This way, when you chain the execution of several such
  programs, like \path|tex| calling \path|mpost| calling \path|tex|,
  the overhead when starting each of the programs but the first will
  be reduced. This change is hidden to the user, except if you set the
  debug flag of kpathsea to the \path|-1| value: you will then trace
  access to the shared memory, which is not what you want (it is
  accessed very often!). What is useful in a log trace of the shared
  memory access is still to be defined, so the situation might evolve
  in the future.
\item[\cmdname{kpsecheck}] this command provides some option that did
  not fit well into \cmdname{kpsewhich}. It will allow you to list all
  the files that occur multiple times across your texmf trees. This
  could be handy, but most of the time you will also get unwanted
  output (like dozens of \path|README| files)\footnote{It is noticeable
    that all these files result in clashes inside the \KPS{}-hashing
    mechanism, fortunately, \KPS{} never look for these files.}. For
  this reason, you can combine the \path|-multiple-occurences| with 2
  other options for including or excluding any filename that match
  some pattern (you can request for several patterns).

  The \cmdname{kpsecheck} command will also report the status of
  shared memory: in use or not used. That might be useful to know 
  because if the status reported is ``in use'', that means one or
  several processes are working, and the effect of any
  \cmdname{mktexlsr} command will be delayed until the next time where
  no \KPS{} linked process will be running.

  Last, this same command will report about the location it thinks
  Ghostscript can be found. Under Win32, for many programs, it is
  easier to use the Ghostscript dll, and find it by using the
  Ghostscript registry key than to change the \path|PATH|, which has a
  limited length anyway.
\item[\Webc{}] the engines have a few more options than the ones from regular 
  \Webc{}, and one option with a different behaviour:
  \begin{itemize}
  \item the \path|-fmt| option behaves differently. Previously and
    with the regular \Webc{} distribution, this option has 2 different
    meanings when you are in ``ini'' or ``vir'' mode. Under Win32, it
    has the same meaning: preload the format file specified with the
    argument. The meaning of building a format of such name in ``ini''
    mode is obtained by the new \path|-job-name| option. 
  \item \path|-job-name| option: allows to set the name of the file
    resulting from the compilation process. In normal mode, it will
    change the base name of all files produced (\path|.dvi|,
    \path|.aux|, etc.), while in ``ini'' mode, it will set the name of
    the format file written.
  \item \path|-halt-on-error| stop the compilation at the first error.
  \item \path|-job-time| set the job time to the same timestamp as the
    file given in argument.
  \item \path|-output-directory| allow to write all the output files in the
    specified directory.
  \item \path|-time-statistics| print statistics about the job run
    time. It is to be noted that Win9x not being a true multitasking
    operating system, it has no reliable timer for short periods, so
    the value printed is an approximation. Under NT/2K/XP, the result
    is quite accurate with user time and system time values allocated
    for this run. For Unix
    users: the \path|time| command is not usually available to Windows
    users.
  \end{itemize}
\end{description}

\subsection{Adding packages to your installation}

You will find an option in the \texttt{TeXLive} menu (or go to
\texttt{Start -> Programs -> TeXLive -> Add TeX package} menu) to run again
\file{TeXSetup.exe}, but in maintenance mode this time. The steps you
will go through are almost identical to the ones the first time you
run it.

The only different step is about the packages selection page. In
maintenance mode, the list of installed packages is compared to the
list of packages available from your source directories. Packages that
are not installed will be displayed in green, out of date packages
will be displayed in red and up to date, installed packages are
displayed in black.

This way, you can choose to add or upgrade components, either from
your \CD{} or from the
Internet, where you are likely to find some more recent version of
your packages.

It is up to you to select which ones of the packages you want to
install. The rest of the process is similar to the first installation.

If you want to add files that are not provided by the \TeXLive{} (or
\fpTeX{}) distribution, it is recommended to put them in the
\path|$TEXMFLOCAL| directory. This way, you will be safe against
upgrades of the \TeXLive{} software.

The directory pointed to by \path|$TEXMFLOCAL| is initially empty. If
you want to add there the support file for Maple (symbolic computation
program) for example, you will have to put the style files in:\\
\path|c:\Program Files\TeXLive\texmf-local\tex\latex\maple\|\\
and the documentation files in:\\
\path|c:\Program Files\TeXLive\texmf-local\doc\latex\maple\|

Next, \textbf{do not forget to rebuild the ls-R databases files},
either by using the right menu (Start -> Programs -> TeXLive
  -> Maintenance), either by manually running the \file{mktexlsr}
  command.

\subsection[Removing \protect\TeXLive{} from your hard disk]{Removing
    \protect\TeXLive{} from your hard disk}

The uninstall procedure is available either from the
\file{TeXLive.exe} program, from the \path|TeXLive| menu or from the
control panel (Start menu -> Control Panel, Add/Remove Programs
option). This procedure will cleanup your hard disk of most of the
\TeXLive{} files. However, \TeX{} is a system that is creating files
and there is no mechanism to keep track of all of them. Moreover,
Win32 support packages have their own uninstall procedure, which you
will have to run separately (provided you want to get rid of them). Last,
the files you may have stored in \path|$TEXMFLOCAL| won't be
removed. So, even if the vast majority of files are automatically
removed by the uninstall procedure, you will have to do some manual
cleanup to actually remove all of them.

\subsection{Running \texttt{TeXSetup.exe} from the command line}

The \path|TeXSetup.exe| program has a number of other interesting
options. You can get the list by running:
\begin{verbatim}
c:\>TeXSetup --help
\end{verbatim}

Here is the description:
\begin{description}
\item[\path|--automatic-reboot|] reboot without waiting user
  confirmation once installation is over;
\item[\path|--dry-run|] do nothing, just log everything that will be
  done without this option;
\item[\path|--quick|] use the recommended installation and default
  directories, ask nothing up to rebooting;
\item[\path|--net-method (=ie5/direct)|] enable to download components with
  restricted licenses from the Internet (either using direct
  connection of Internet Explorer 5 DLLs): you need to have an available
  network connection and some of the packages are huge;
\item[\path|--remote-source-directory <url>|] this is the base url for the remote packages;
\item[\path|--local-source-directory <dir>|] by default, \path|TeXSetup.exe|
  will guess  the  root directory of the set of files you want it to act on, if you ever
  upgrade \path|TeXSetup.exe|, you won't be able to copy the new version
  onto your \CD{}, so you will need to use this option to specify the
  root of the \CD{};
\item[\path|--installation-directory <dir>|] this is the root of your
  installation, all files will be copied under this location. The
  default value is \verb+c:\Program Files\TeXLive+;
\item[\path|--texmfmain-directory <dir>|]
\item[\path|--texmflocal-directory <dir>|]
\item[\path|--texmfextra-directory <dir>|]
\item[\path|--texmfhome-directory <dir>|]
\item[\path|--vartexmf-directory <dir>|]
\item[\path|--vartexfonts-directory <dir>|] these are the directories
  used to configure the location of your files. They map directly to
  the \path|texmf.cnf| variables.
\item[\path|--with-source-files(=yes/no)|] copy the source files
  for \TeX{} packages, default value is no;
\item[\path|--with-documentation-files(=yes/no)|] copy documentation files for \TeX{}
  packages. Default value is yes. Beware: this is only documentation about specific
  packages, general documentation will be installed anyway;
\item[\path|--program-folder <folder>|] the name of the folder under
  which you will find the menus;
\item[\path|--add-package <pkg>|] this is used to add or update a specific
  package after a first regular installation;
\item[\path|--scheme <pkg>|] install the named scheme instead of the
  default \path|texlive-recommended| scheme;
\item[\path|--maintenance|] mostly the same as \path|--add-package|
  without specifying a package to add;
\item[\path|--uninstall|] this option will remove anything \TeX{} related coming from
  the \CD{}, which means there can be files left if you added style
  files or format files, and also that supplementary tools will not be
  removed\footnote{This option is still a bit crude as of \today};
\item[\path|--help|] this option opens up a box with the list of options.
\end{description}
  
\subsection{Network installation}

\KPS{} knows about UNC names, so you can use them to get your TEXMF
tree from the network. But there is better than this.
All the support files and configuration files, everything except the files in the
\path|bin/win32| are shareable with a \teTeX{} or Unix \TeXLive{} installation. That means
you can use Samba either to mount from an NT server to a Unix
workstation or the converse. Several strategies are possible:
\begin{itemize}
\item Put everything on the server. Just add each set of files for the os and 
architecture you want to use in the \path|bin| directory. That means
for example \path|bin/win32| and \path|bin/i386-linux|. Next configure
your main variables. You can use UNC names to
point to the right directories under Win32.
\item Install a local copy for the binaries and format files. In this
  case, assign \path|$TEXMFMAIN| to the main \path|texmf| tree that
  will be accessed remotely. Set \path|$VARTEXMF| to be a local
  directory which will hold local configuration files and on-the-fly
  generated files.
\end{itemize}

\subsection{Personal Configurations}

\subsubsection{\cmdname{WinShell}}
\label{sec:winshell}

Start this program from the Start menu or from the Desktop shortcut.
Go to \textbf{Options} \verb+->+ \textbf{Program Calls}.

\begin{itemize}\itemsep 0pt  
\item In the DVIWin tab, if the filename is \textbf{yap}, replace it
with \textbf{windvi.exe}.
\item In Ghostview tab, make sure it gives the correct path for
gsview32.exe:\\
\hspace*{1em} \verb+C:\ghostgum\gsview\gsview32.exe+ (for 3.6, the
current version)\\ 
or \\
\hspace*{1em} \verb+C:\gstools\gsview\gsview32.exe+ (for the older
version)

\end{itemize}

Click \textbf{OK}.

Note that the install process sets all files with the \texttt{.tex}
extension to open with \cmdname{WinShell}. Unless you plan to use another editor
(such as WinEdt or Emacs), this is appropriate.

Unfortunately,  \cmdname{WinShell} does   not  have a  spell-checking feature.   
However, if  you  have  installed the \path|tex-extrabin|  collection,
your installation includes \textbf{Ispell}  (a spell  checking program
found  on  most  Unix systems).  The   executable is in  your  PATH so
\verb+ispell.exe+  will be found if you  invoke it from  a DOS window. 
If you installed documentation, look
at\\
\verb+C:\Program Files\TeXLive\texmf\doc\html\manpages\ispell.html+\\
for   information   on     Ispell.     (Otherwise,   you  can     find
\verb+ispell.html+ on the CD.)  If you rely on spell checking, you may
want      to   add     an    Ispell   icon      to    \cmdname{WinShell}.    See
subsection~\ref{winshell:ispell} of this document for how to do that.

For an excellent commercial (but inexpensive) spelling checker,
see \url{http://www.microspell.com}.

%Appendix B also contains other information about using \cmdname{WinShell}, such as:
%
%\begin{itemize}
%\item How to apply a bug fix you probably need 
%\item How to print directly from \cmdname{WinShell} 
%\item How to use the Project feature for multiple-file documents  
%\end{itemize}

\cmdname{WinShell} also has on-line help, which you can access via the
$\textbf{\underbar{?}}$ on the menu bar.

Other information about using \cmdname{WinShell} can be found in
section \ref{winshell:more} on p.~\pageref{winshell:more}.

\subsubsection{Dvips}

The configuration file for dvips can be found in\\
\verb+C:\Program Files\TeXLive\texmf-var\dvips\config\config.ps+\\
You may open  it with any editor  (\cmdname{WinShell}  will do fine) and  change
some parameters:

\begin{description}
\item [fonts] you can change the default printer \MF{} mode or printer
  resolution in case dvips needs to generate PK fonts. By default it
  is configured to use type1 versions of the CM fonts, so it should
  not call \cmdname{mktexpk} too often;
\item[printer] you can tell dvips where you want to print by
  default. If the `o' option is not followed by a printer name, then a
  \file{.ps} PostScript file is written. You can give dvips a printer
  name such as:
\begin{verbatim}
o lpt1:
% o | lpr -S server -P myprinter
% o \\server\myprinter
\end{verbatim}
\item[paper] Next, you might want 
  to change the paper size from European (A4) to US letter
  by making the US letter the first paper size mentioned in the file.
  Scroll to the group of lines beginning with ``@''. Move 
  the appropriate lines so that this section begins with the lines:\\
\hspace*{1em}    @ letterSize 8.5in 11in\\
\hspace*{1em}    @ letter 8.5in 11in\\
\hspace*{1em}    @+ \%\%BeginPaperSize: Letter\\
\hspace*{1em}    @+ letter\\
\hspace*{1em}    @+ \%\%EndPaperSize
\end{description}

\textbf{Note:} The current \TeXLive{} distribution has implemented (for
the first time!) the procedure of making always up-to-date fontmaps files
for Dvips and Pdftex. This is done by \cmdname{updmap} program
during installation, as well as during any font package addition.
If you add new packages by hand, edit the file \verb+updmap.cfg+
in \path|$VARTEXMF/web2c|.

\subsubsection{Pdftex}
If you want to use the program pdf{}latex to convert directly to
pdf format, and you are using US letter file, edit the file\\
\verb+C:\Program Files\TeXLive\texmf-var\pdftex\config\pdftex.cfg+\\
and change ``page\_width'' and ``page\_height'' to specify letter-size paper.
These entries should read:\\
\hspace*{1em}   page\_width 8.5 true in\\
\hspace*{1em}   page\_height 11 true in\\[6pt]
Save the file and exit the editor.   

\subsubsection{GSView}
Starting from versions compatible with Ghostscript 6.50, GSView is not
free anymore, but shareware. So it is not on the \CD{} anymore.

You may want to change the papersize to US letter size. If so,
open GSView from the Start menu.  \\
From the \textbf{Media} menu, select \textbf{Letter}.

Also, there are  menu settings that are supposed to give you the
most readable screen image:\\[4pt]
 From \textbf{Media} \verb+->+ \textbf{Display Settings}, 
set \textbf{Text Alpha} 
and \textbf{Graphics Alpha} both to 4 bits.

Note that the installation process has set all .ps and .eps files 
to automatically open with GSView.

For printing instructions, see section~\ref{printing} below.

\subsubsection{WinDvi}
\label{sub:windvi}

The \file{TeXSetup.exe} program takes care of associating the files
with the \file{.dvi} extension with \file{Windvi}.

Open it from the Start menu (Programs \verb+->+ TeXLive \verb+->+ DVI Viewer).  
You can set it for US letter paper by going to \textbf{View} \verb+->+ \textbf{Options}
and next to \textbf{Papertype},
selecting US (8.5\verb+"+ x 11\verb+"+). 
Click \textbf{OK}. 
Exit Windvi.
You can change other parameters from there, like the ability to
execute commands included in path|\special{}|.
Also, the first time you 
view any .dvi file, you may find the magnification too large. Zoom out
until you get an appropriate size.

%There are two important parameters related to printing that can't yet
%be set from the dialog boxes, namely the resolution and the \MF{} mode
%for the printer. You can set them once for all by specifying them on
%the command line:
%\begin{verbatim}
%c:\>windvi -p 360 -mfmode canonbjc foo.dvi
%\end{verbatim}
%When you will exit Windvi, these parameters will be stored in the
%configuration file. The available modes are found in this file:\\
%\path|c:\Program Files\TeXLive\texmf\metafont\misc\modes.mf|

All the configuration for Windvi is stored in the
\path|$HOME/windvi.cnf| file. You can find it by running this command
at the prompt:
\begin{verbatim}
c:\>kpsewhich --expand-var $HOME/windvi.cnf
\end{verbatim}

Should you have problems with Windvi, please remove the configuration
file and test your problem against a vanilla configuration.

\subsection{Testing}

You can test \cmdname{WinShell} by opening the file \verb+sample2e.tex+, found in
\path|C:\Local\TeX\texmf\tex\latex\base|. The \LaTeX\ source should appear
on the screen. Process it by clicking on the \LaTeX\ icon on the toolbar,
then view it by clicking on the Preview (Windvi) icon. 

At first, when you preview files with Windvi, it will create fonts because
screen fonts were not installed. After a while, you will have created most
of the fonts you use, and you will rarely see the font-creation window.
Return to \cmdname{WinShell} and try dvips, then GSView.

\textbf{Hint for the future:} If a \LaTeX\ run stops because \LaTeX\ 
cannot find a file, you can press \textbf{Ctrl-z} to quit. 

\subsection{Printing}\label{printing}

It is possible to print from Windvi. In this case, printing will
be done using the Windows unified printer driver. By definition, it
is compatible with all printers. However, there is some drawback: it
can generate some huge spool files, and some (older) versions of Windows just
don't like them. The advantage is that you can use features like
embedding BMP or WMF images. You also need to make sure that the
printer parameters are correctly set (subsection~\ref{sub:windvi}),
else you will get scaled printing (printing at 600dpi on a 300dpi
printer will give you only one quadrant of your page).


Printing is faster and more reliable if you run dvips to make a
.ps file and then print from GSView. To print from GSView, first select
\textbf{Print...} from the \textbf{File} menu. A Print window will appear.

If you will be using a PostScript printer, \textit{be sure to select
\textbf{PostScript Printer}}.  In the newer version this is done in
the ``Print Method'' box at the bottom left of the Print window. You
can then select any of the printers that you have previously installed
on your PC.  If you fail to check the box for PostScript Printer,
printing will not work.
  
If you will be using your own non-PostScript printer, select
\textbf{Ghostscript device} in the ``Print Method'' box, then click on
the button to the right labelled \textbf{djet500} and select your
printer type from the list that pops up. (In the older version of
GSView, make sure PostScript Printer is \textit{not} selected, then
select your printer type from the ``Device'' list.)
  
If you use \cmdname{WinShell} and a PostScript printer, probably the most
convenient way to print is to add an icon to the \cmdname{WinShell} toolbar that
invokes dvips in a way that sends the output directly to a default
printer.  For detailed instructions on how to do this, see 
\ref{winshell:print} on p.~\pageref{winshell:print}
(\textit{More About \cmdname{WinShell}}).

\subsection{More About \cmdname{WinShell}}\label{winshell:more}
\subsubsection{Installing Bug Fixes}

\cmdname{WinShell}'s author (\href{mailto:idb@winshell.de}{Ingo de Boer},
thanks to him) sometimes releases beta versions of the next \cmdname{WinShell}
version which are also bug fixes. You can grab them from
\url{http://www.winshell.de}. Usually they are \file{.zip} files that
only require to be unpacked in \cmdname{WinShell} directory (\path|c:\Program
Files\WinShell| by default), either using WinZip or a similar tool, or
by using \file{unzip} on the command line. If you got some
\file{winshellbugfix.zip} file and that you saved it in the \cmdname{WinShell}
directory, then you need to run:
\begin{verbatim}
c:\>cd c:\"Program Files"\WinShell
c:\>c:\local\bin\unzip winshellbugfix.zip
\end{verbatim}
Say `yes' if you are asked if some files should be overwritten. The
\path|unzip.exe| programme can be found in the
\path|support/gnu-utils| package. If you
do not have it on your machine, you can use any archiver tool like
WinZip to achieve the same effect.

\subsubsection{Using the Project Feature}
If your document is split into several files
(for example a thesis), look into \cmdname{WinShell}'s ``Project'' feature. From the
\textbf{Project} menu, you give the project a name (e.g., Thesis), 
supply the name of the main (or root) file, and then ``add'' other
files. These filenames display on the left of the screen where you can
double click the names to view and switch between them. Clicking
the \textbf{\LaTeX} icon always processes the main file.

Note the icons on the toolbar for toggling the project space (on the left)
and the log space (at the bottom). If you are not using the Project
feature, you may want to toggle off the space on the left, using the full
screen width to display your file.

\subsubsection{Printing from \cmdname{WinShell} to a PostScript Printer}
\label{winshell:print}

The Dvips icon on the \cmdname{WinShell} toolbar puts the PostScript output in a file,
which you can then view with GSView and print from there if you choose.
However, it's convenient to add a \cmdname{WinShell} ``program call'' to dvips which
sends the output directly to a designated PostScript printer. The steps
below show how to do this for the printer \textbf{vclw}; you should
substitute the name of your most frequently-used printer for \textbf{vclw}.
\begin{enumerate}
\item Make the program aware of the printer:
\begin{itemize}

\item Open \cmdname{WinShell}, go to \textbf{Options} \verb+->+ 
\textbf{Program Calls} \verb+->+ \textbf{User defined}.

\item Click on \textbf{Tool 1} in the list on the right and fill in the 
fields to the left as follows:\vspace{-6pt}
\begin{quote}
Name: \verb+Print+\\
exe file: \verb+dvips+\\
cmd-line: \verb+ -D600 %m -o vclw+\\
Uncheck the box for ``DVIPS first''
\end{quote}%\vspace{-12pt}
\item Click \textbf{OK}
\end{itemize}
\item Add Print to the toolbar:
\begin{itemize}
\item Go to  \textbf{Options} \verb+->+ \textbf{View} \verb+->+ \textbf{Customize}.
\item In the Category box, select \textbf{User-Programs}.
\item Select \textbf{Print} and drag it to the toolbar, placing it just to
  the right of the GSView icon.
\item You then have a choice of ``Image only'', ``Text only'', or ``Image and
Text''. The easiest is to select ``Text only'' and click \textbf{OK}.
You should then see \textbf{Print} on the toolbar. (If you prefer, you can
select ``Image only'', then ``Edit'', and edit the displayed picture to your
satisfaction.)
\end{itemize}
\end{enumerate}
Now, to print a \LaTeX\ document, just click on the \textbf{Print} icon to
send to your selected printer. To use a different printer, you will need to
click on the \textbf{Dvips} icon to print to a file. Then click on the
GSView icon and use GSView to send to any printer you have installed on
your PC.

\subsubsection{Adding Ispell to \cmdname{WinShell}}
\label{winshell:ispell}
\begin{enumerate}
\item Add Ispell to User Tools:
\begin{itemize}
\item Open \cmdname{WinShell}, go to \textbf{Options} \verb+->+ 
\textbf{Program Calls} \verb+->+ \textbf{User defined}.
\item In the list on the right, click on \textbf{Tool 1} (or \textbf{Tool
    2} if you have already used \textbf{Tool 1}) and fill in the 
fields to the left as follows:\vspace{-6pt}
\begin{quote}
Name: \verb+Ispell+\\
exe file: \verb+ispell+\\
cmd-line: \verb+ -t -d american %c.tex+\\
Uncheck the boxes for ``LaTeX first'' and ``DVIPS first''
\end{quote}%\vspace{-12pt}
\item Click \textbf{OK}
\end{itemize}

\item Add Ispell to the toolbar: %\vspace{-14pt}
\begin{itemize}\itemsep 0pt
\item Go to  \textbf{Options} \verb+->+ \textbf{View} \verb+->+ \textbf{Customize}.
\item In the Category box, select \textbf{User-Programs}.
\item Select \textbf{Ispell} and drag it to the toolbar, placing it just to
  the right of the GSView icon (or the last icon you added).
\item You then have a choice of ``Image only'', ``Text only'', or ``Image and
Text''. The easiest is to select ``Text only'' and click \textbf{OK}.
You should then see \textbf{Ispell} on the toolbar. (If you prefer, you can
select ``Image only'', then ``Edit'', and edit the displayed picture to your
satisfaction.)
\end{itemize}
\end{enumerate}
Now, when you have a \LaTeX\ document open, you can click on
\textbf{Ispell} to perform spell checking. Ispell will open another window
and display the first misspelled word on the left with the filename on the
right. Below that you will see the context in which the misspelling appears;
often several suggestions for replacements are also displayed. To replace
the word, enter the number corresponding to the desired replacement. Other
possible responses are listed below; for example, you can press the space
bar to ignore the misspelled word. For more information on Ispell, read the
manual page: \verb+C:\Program Files\TeXLive\texmf\doc\html\manpages\ispell.html+.

Note that when you replace a word, you will not see the correction in your
\cmdname{WinShell} window until you close the file (click the X in the upper right
corner) and then open it again (use the File menu).

\subsection{Tips and tricks about the Win32 platform}

\subsubsection{Different flavors of Win32}

What we call Win32 is not an operating system by itself. It is a set
of functions -- and a large one\footnote{Around 12000 functions in the
  header files of the Microsoft SDK} -- that you can use to write
programs for different operating systems of the Windows family.

Windows comes in different flavors:
\begin{itemize}
\item Win95, Win98 and WinME, which \emph{are not true multitasking,
    multithreading} environments. They are the latest -- and hopefully
  last -- metamorphosis of DOS. This can be more or less proven by the
  fact that when booting, the PC will load the \path|command.com|
  interpreter, and if you stop the boot process at this point, you can
  ask for the current (DOS) version and it will answer something like
  'MS-DOS 7.0' (at least for the old versions of Win9x);
\item Windows NT, which is a new operating system  written from scratch,
  capable of
  true multitasking behaviour, and loaded with high level features;
\item Windows 2K, written on an NT basis, with all the bells and
  whistles of Win98.
\item Windows XP, which comes with Personal and Pro flavors. This is
  the last step in merging both lines of products (Win9x based and NT
  based). XP is written on an NT basis.
\end{itemize}

Win9x  are able    to run  32~bits programs    and 16~bits   programs
concurrently.  But the  operating  system by   itself is  not entirely
written in 32~bits mode, and does not support memory protection: 16bits
applications can overwrite parts of the operating system memory!  Some
parts of  the system like the GDI  (Graphical Device Interface) manage
limited resources like  bitmaps, fonts, pens and so  on for the set of
all programs that run  concurrently. All the bitmaps headers available
at the same time  can't amount for more  than 64kb. This  explains the
performance tool   and the fact that you   can put your  system on his
knees by making intensive use of graphic objects for example.

NT, 2K and XP do not suffer from these limitations, and neither from
other Win9x limitations. They are true multitasking environments, with
protected memory. They are much more responsive than Win9x because
of better memory management, better file system and so on.

\subsubsection{Command line prompt}

You will wonder: ``why would I need to use a command line prompt when
I have Windows?''.

Good question. The problem is of very general nature. Not all
operations can be done easily using only a GUI. Command line gives you
programming power -- assuming a clever command interpreter.

But the problem here is more fundamental: \TeX{} is \emph{a batch}
tool. Not an interactive one. \TeX{} needs to compute the best
layout for each page, resolve cross-references and so on. This can be
done only by a global processing of the document. It is not (yet) a
task that can be done interactively.

This means that you should use \TeX{} from a command line. In fact the
situation is not so bad. There is an advantage to write command line
tools for complex processing: they are better debugged, because not
tied to GUI problems, and GUI tools can be designed to interface the
command line tools. This is the case for \TeX{} where you will
interact with it most of the time through a GUI text editor -- see
section~\ref{sec:winshell} for example.

However, you may need to use the command line prompt in a number of
situations, by example in case of problems and you want to debug your
setup -- see
  section~\ref{sec:troubleshooting}.

  \begin{description}
  \item[Win9x] You will open a command line prompt by looking either
    for the MS-DOS icon in the ``Start->Programs'' menu, either by
    choosing ``Start->Run'' menu and typing in \path|command.com|
  \item[NT, 2K, XP] You will open a command line prompt by looking
    for the  ``Command Prompt'' in the ``Start->Accessories''
    menu\footnote{These locations may change across different OS
      versions.}. You can also choose the ``Start->Run'' menu and type
    in \path|cmd.exe|, which is the name of the brand new command
    interpreter for NT\footnote{Which explains why it is untrue to
      call this a \emph{DOS} box under NT!}.
\end{description}

\subsubsection{Path separators}

The Win32 API understands both \path|/| and \path|\| characters as
PATH separators. But the command interpreters do not! So whenever a
path name is used programmatically, you can use both separators, and
even mix them up in the same path name. But on the command line, you
must type \path|\| as path separator. The reason is compatibility:
the command processor used the \path|/| to introduce arguments to
commands.

All this to say: do not be surprised to read path names written using
the Unix convention; \fpTeX{} is a port of \Webc, and aims to be compatible
across platforms. For this reason, all the configuration files that
need to specify path names use the Unix convention.

\subsubsection{File systems}
\label{sec:clusters}

The worse feature of Win9x with regard to \TeX{} is probably the
so-called FAT file system. \TeX{} uses many many small files, with
size around 1kb -- 3kb. The FAT file system is old, and predates by
far the multi-gigabytes hard disks we have today. It means it can't
manage efficiently the 30000 \TeX{} files found on the \CD{}. The FAT
file system will allocate a minimum of 32kb for \emph{any} file on a
huge partition. It means that \TeX{} will use much more disk space
than it actually needs.

The other, more modern, file systems available -- namely FAT32 and
NTFS -- do not have this drawback. They manage clusters of 4kb
only\footnote{You can lower the limit to 512 bytes on NTFS}.

\subsubsection{How to add some directory to your PATH}

There are pairs of variables and values which behave much like global
variables inside your programs. The set of those variables is called the
environment. Each program is initialized with a copy of the
environment when it is run. It can request and change the
value of any variable. The changes happen in the copy of the
environment, and is not at all propagated to the other running
programs.

Your PATH is a special environment variable used to search for
programs you want to run.
There is a different procedure to change it for Win9x, WinME and NT/2K/XP:

\begin{description}
\item[Windows 95/98] Edit your \path|autoexec.bat|. In this file should be a line
  starting with \texttt{PATH=} and followed by a list of directories separated
  by \path|;|. Please add the directory with the executables in this line.
  After this, this line could look as follows:
\begin{verbatim}
PATH=c:\windows;c:\windows\system;c:\"Program Files"\TeXLive\bin\win32
\end{verbatim}
\item[Windows ME] You need to run the special program
  \path|c:\windows\system\msconfig.exe| to be able to change any
  environment variable. From this program, select the `Environment'
  tab, and then add or modify the  variable you want. You will be asked to reboot the
  machine upon any change.
\item[Windows NT/2K/XP]
  Click left on \texttt{Start --> Settings --> Control Panel}. Now the window
  with the control panel icons opens. Double click on System. The
  System Properties window opens. Click on the tab
  \textsf{Environment} or look for a button named `Environment
  Variables' among the dialog boxes. Now
  you can change the environment variables for your user account. Note:
  There are also displayed the environment settings for the system.
  Normally, you can't change the system variables unless you have
  administrator rights on your machine. If you want to change the \texttt{PATH}
  for all users, you will have to contact your system administrator or
  be the system administrator yourself--in the later case you should
  know what you are doing.
  
  If there  is already a \texttt{PATH}  setting for your user account,
  left click on \texttt{PATH}.  In the field \textsf{Variable} appears
  \texttt{PATH}  while  the field  Value  shows the current setting of
  \texttt{PATH} as  a list of directories  separated  by \path|;|. Add
  the   directory   where   the   executables   are    located   (e.g. 
  \path|c:\Program Files\TeXLive\bin\win32|).  If  there isn't  a \texttt{PATH} variable
  for your user  account, simply click  in the field Variable and type
  in  \texttt{PATH}, click in the field \textsf{Value}
  and type in the directory with the executables.  Important: Click on
  the \textsf{Apply} button before clicking \textsf{Ok}, otherwise the
  changes to \texttt{PATH} won't apply to your system. Be careful when
  changing the environment settings.
\end{description}

The best way to be sure that a variable has been properly set is to
open a console and type:
\begin{verbatim}
set VARIABLE
\end{verbatim}
which should return the corresponding value.

\subsubsection{\TeX{} engines}

If you have a look at the \Webc{} documentation,  you will read that all
the various \TeX{}  derived programs  use the  same base  engine.  For
example, \path|tex.exe| and \path|latex.exe|  are exact copies  of the
same program, but each one will use a  different format file, based on
its calling name.

Under Unix, this feature is implemented through \emph{symbolic
  links}. It saves up a bit of disk space, because some engines are
used with many different format files.

The Win32 API does not know about file links. So to save up almost
the same amount of memory, all the \TeX{} base engines have been put
in DLLs (\emph{Dynamic Linked Library}). This means that you will have
the following layout:
\begin{alltt}
13/05/2002  17:06                3 584 latex.exe
13/05/2002  17:06              266 240 tex.dll
13/05/2002  17:06                3 584 tex.exe
\end{alltt}
and the \path|latex.exe| file is nothing but a rough copy of
\path|tex.exe| using the same core \path|tex.dll|.
The same trick has been used for the \path|mktex*.exe| family of programs which are
linked to the \path|mktex.dll| library.

In fact, a generic tool called \path|lnexe.exe| is provided to build the
equivalent of Unix hard links for executable files only under Win32.

\subsection{In case of problems}
\label{sec:troubleshooting}

\subsubsection{What to do if  \texttt{latex} does not
  find your files?}

\begin{itemize}
\item \cmdname{kpsewhich} is the tool of choice to debug any
problem.  Unfortunately, kpsewhich outputs debug information to
stderr, and the Windows console does not know how to redirect stderr
to a file\footnote{Well, NT and Win2k consoles know how to do
  that. But the trick will work for any console.}.  
For diagnostic purposes you can temporarily set an
environment variable (in DOS box):
\begin{verbatim}
SET KPATHSEA_DEBUG_OUTPUT=err.log
\end{verbatim}
You can also set the debug level: 
\begin{verbatim}
SET KPATHSEA_DEBUG=-1
\end{verbatim}

If you want to redirect stderr to stdout, which is not possible under
either W9x or NT/2K/XP, then just do:
\begin{verbatim}
SET KPATHSEA_DEBUG_OUTPUT=con:
\end{verbatim}
This way you can capture both stdout and stderr in the same file.

\item Assuming the installation has been done in \path|c:/Program Files/TeXLive|, check
  the following values: \\
  {\small
  \begin{tabular}{ll}
    \path|kpsewhich -expand-path $SELFAUTOPARENT| &  \path|c:/Program Files/TeXLive| \\
    \path|kpsewhich -expand-path $TEXMF| & \path|c:/Program Files/TeXLive/texmf| \\
    \path|kpsewhich -expand-path $TEXMFCNF| &
    \path|.;c:/Program Files/TeXLive/texmf/web2c;| \\
                                        & \path|c:/Program Files/TeXLive/bin/win32;| \\
                                        & \path|c:/Program Files/TeXLive/bin;| \\
                                        & \path|c:/Program Files/TeXLive| \\
    \path|kpsewhich -expand-var $TEXINPUTS| & \path|.;c:/Program Files/TeXLive/texmf/tex//|
  \end{tabular}
}
\item If you have other \TeX{}-related values already set in your
  environment, please, remove them. They are overriding the ones in
  texmf.cnf.
\item Check the values from:\\
{\small
  \begin{tabular}{ll}
    \texttt{kpsewhich cmr10.tfm} & \path|c:/Program Files/TeXLive/texmf/fonts/tfm/public/cm/cmr10.tfm|\\
    \texttt{kpsewhich latex.fmt}& \path|c:/Program Files/TeXLive/texmf/web2c/latex.fmt|
  \end{tabular}
}
\item At this point, if everything is correct, \path|tex.exe| and
  co. should work. If it is not the case, you will need to play with
  the \path|-debug=n| option from \path|kpsewhich|, and check back all
  the values. Try to identify and report the problem.
\end{itemize}

\subsubsection{What to do if your setup still does not work as expected?}

There are several questions to ask about:
\begin{enumerate}
\item Is \file{tex.exe} on my \path|PATH|? 
\item Is the \path|TEXMFCNF| variable correctly set to
  \path|c:/Program Files/TeXLive/texmf-var/web2c| (default value)?
\item Are there any errors in the log file generated by the
  \file{TeXSetup.exe} program? Errors are flagged with the sequence
  \path|Error|.
% \item If everything seems correct up there, then you need to read the
%   documentation in  \path|texmf/doc/fptex/fptex.pdf| or
%   \path|texmf/doc/html/fptex/fptex.html| for a more detailed procedure
%   (available on your hard disk or on the \CD{}).
\item One can also go to \url{http://www.tug.org/tex-live.html} and
  check for any bug fix.
\item The Windows distribution on the \CD{}  is no more no less than
  the  \fpTeX{} distribution, so you can also go to the Web pages at
  \url{http://www.fptex.org}, or consider subscribing to the \fpTeX{}
  mailing-list by consulting
  \url{http://www.tug.org/mailman/listinfo/fptex} .
\end{enumerate}


The \TeXLive{} software is complex and made of more than 250 programs
and around 40000 files from various sources. It is quite difficult to
predict all possible causes for problems. Nevertheless, we will do our
best to help you in every case.


\subsection{Compiling the source files}

You have the whole set of source files,  comprised for
Windows in the \path|source/source.tar.bz2| archive available on the
\CD{}. To be able to compile the whole distribution for Windows, you
will need:
\begin{itemize}
\item Windows 2K/XP
\item Microsoft Visual Studio .Net
\item a set of Unix tools (\texttt{sed},
  \texttt{grep}, \texttt{gawk}, etc.) and also Perl, Flex and Bison,
\item to adjust the paths in the
  \path|win32/make/common.mak| file according to your installation
\item adjust the paths in the  Perl script file
  \path|win32/perl/build.pl|,
\item run the compilation from the  \path|win32/| directory using this
  command:
\begin{verbatim}
c:\texlive\source\win32>perl ./perl/build.pl --install --log=install.log
\end{verbatim}
\end{itemize}

There is a lot of work to do to make this process easier and cleaner.

\subsection{Where to get more information?}

The Win32 \TeX{} distribution on the \CD{} is also known as
\fpTeX. Only the packaging differs, but \fpTeX{} is no more no less
than the current \TeXLive\ release for Windows.

The \fpTeX home on the Web is at:\\
\url{http://www.fptex.org/}

The current \fpTeX release is available from any
\href{http://www.ctan.org}{CTAN} %(see also the section
%\ref{sec:ctan}) 
site in the directory~:\\
\noindent\url{ftp://ctan.tug.org/tex-archive/systems/win32/fptex/}.

The main ftp site for \fpTeX\ is \url{ftp://ftp.dante.de/pub/fptex/} from
where beta versions of \fpTeX\ and additionnal tools are available.
This main site  is (partially) mirrored  daily by the CTAN backbones in their
\path|systems/win32/fptex| directory.

% You can reach me at my email address:
% \url{mailto:Fabrice.Popineau@supelec.fr}.

The \TeX{} Users Group  is kindly hosting  a mailing-list dedicated to
\fpTeX. This is a  very low volume one.  It is used for announcements,
bugs reports or as well to discuss about improvements or various users
problems.     To      subscribe,       read     the      page       at
\url{http://www.tug.org/mailman/listinfo/fptex}.  The   mailing   list
address is \EmailURL{fptex@tug.org}.

\section{Building on a new Unix platform}
If you have a platform for which we have not provided binary sources,
you will need to compile \TeX{} and friends from scratch. This is not
as hard as it sounds. What you need is all in the directory
\texttt{source} on the \CD{}.

You should first install the support tree  from the \TeXLive{} \CD{} (do
a basic install, with no system binaries chosen).

\subsection{Prerequisites} 
You will need about 100 megabytes of disk space to compile all of
\TeX{} and its support programs. You'll also need an \acro{ANSI} C
compiler, a \cmdname{make} utility, a lexical scanner, and a parser
generator. The \acro{GNU} utilities (\cmdname{gcc}, \acro{GNU}
\cmdname{make}, \cmdname{m4}, \cmdname{flex}, \cmdname{bison}) are the
most widely tested on different platforms.  \cmdname{gcc}-2.7.*
\cmdname{flex}-2.4.7 and \acro{GNU}\,\cmdname{make}-3.72.1 or newer
should work well. You may be able to work with other C compilers and
\cmdname{make} programs, but you will need a good understanding of
building Unix programs to sort out problems. The command
\texttt{uname} must return a sensible value.



\subsection{Configuration}
First, unpack the source from the compressed
\texttt{tar} file in the directory \dirname{source} to your disk and change
directory to where you placed it.  Decide where the `root' of the
installation will be, e.g.\ \path|/usr/local| or
\path|/usr/local/TeX|.  Obviously you should use the same location
that you specified when you installed the support tree.

Now, start the build process by running \textsf{configure} with 
a command-line like
\begin{alltt}
>> \Ucom{./configure -prefix=/usr/local/TeX}
\end{alltt}

The `prefix' directory is the one where you installed the support
tree; the directory layout that will be used is as follows (where
\$TEXDIR stands for the directory you chose):

\noindent
\begin{tabular}{>{\ttfamily}ll@{}}
  \dirname{$TEXDIR/man}                  & Unix manual pages\\
  \dirname{$TEXDIR/share/texmf}          & main tree with fonts,\\
                                         & \qquad macros, etc\\
  \dirname{$TEXDIR/info}                 & \acro{GNU} style info manuals\\
  \dirname{$TEXDIR/bin/$PLATFORM}        & binaries\\
\end{tabular}
%$

You can omit the use of `\dirname{share/}' part for the
\dirname{texmf} directory if you want, as
\dirname{$TEXDIR/share/texmf} and \dirname{$TEXDIR/texmf} are
auto-detected by configure.  If you choose something different, you
have to specify that directory with the \verb|--datadir| option of
\textsf{configure}.


If you want to leave out the \dirname{$PLATFORM} directory level
(i.e.\ put the binaries directly into \dirname{$TEXDIR/bin}), specify
the \verb|--disable-multiplatform| option for \textsf{configure}.


Have a look at the output of \texttt{./configure --help} for more
options you can use (such as omitting optional packages such as \OMEGA\
or \eTeX).

\subsection{Running \textsf{make}}
Make sure the shell variable \texttt{noclobber} is not set, and 
then type 
\begin{alltt}
>> \Ucom{make world}
\end{alltt}
and relax\ldots. 

It could also be useful to log all the output, e.g.\ by 
typing
\begin{alltt}
>> \Ucom{sh -c "make world >world.log 2>\&1" \&}
\end{alltt}
Before you think that everything is ok, please check the log file for
errors (\acro{GNU} \cmdname{make} always uses the string
``\texttt{Error:}'' whenever a command returns an error code) and
check if all binaries are built:
\begin{alltt}
>> \Ucom{cd /usr/local/TeX/bin/i686-pc-linux-gnu}
>> \Ucom{ls | wc}
\end{alltt}
The result should be 209. 

If you need special privileges for \texttt{make install}, you can
run two \cmdname{make} jobs in separate runs:
\begin{alltt}
>> \Ucom{make all}
>> \Ucom{su}
>> \Ucom{make install strip}
\end{alltt}

\subsection{Final configuration steps}
Set up your \envname{PATH} to include the directory containing the
just-installed binaries (e.g.\ 
\dirname{/usr/local/TeX/bin/mips-sgi-irix6.5}); similarly,
\envname{MANPATH} and \envname{INFOPATH} to include the relevant newly
installed subdirectories, i.e.\ \dirname{$TEXDIR/man} and
\dirname{$TEXDIR/info}.

The program \textsf{texconfig} allows you to set the defaults for
hyphenation, paper size, print command, \MF{} mode, etc. You can
run this command interactively and see what options it offers, or type
\begin{alltt}
>> \Ucom{texconfig help}
\end{alltt}

For example, if you are not using A4 format paper, you can make
`lettersize' the default using:
\begin{alltt}
>> \Ucom{texconfig dvips paper letter}
>> \Ucom{texconfig xdvi paper us}
\end{alltt}

\section{A user's guide to the \protect\Webc{} system}

%--------------------------------------------------------

\Webc{} contains a set of \TeX-related programs, i.e., \TeX{} itself,
\MF{}, \MP, \BibTeX{}, etc.  The original implementation was by Tomas
Rokicki who, in 1987, developed a first \TeX{}-to-C system adapting
change files under Unix, which were primarily the work of Howard
Trickey and Pavel Curtis.  Tim Morgan became the maintainer of the
system, and during this period the name changed to Web-to-C\@.  In
1990, Karl Berry took over the work, assisted by dozens of additional
contributors, and in 1997 he handed the baton to Olaf Weber.  The
latest result is \Webc{} Version 7.3, which was released in March
1999. Our version has some updates for the forthcoming new release,
and identifies itself as 7.3.7

The \Webc{} 7.3 system runs on Unix, Windows 3.1, 9x/ME/\acro{NT}/\acro{2K}/\acro{XP},
\acro{DOS}, and other operating systems. It uses Knuth's
original sources for \TeX{} and other basic programs written in \web{}
and translates them into C source code. Moreover, the system offers a
large set of macros and functions developed to augment the original
\TeX{} software. The core \TeX{} family components are:

\begin{description}
\renewcommand{\makelabel}[1]{\descriptionlabel{\mdseries\cmdname{#1}}}
\item[bibtex]    Maintaining bibliographies.
\item[dmp]       \cmdname{troff} to MPX (\MP{} pictures).
\item[dvicopy]   Produces modified copy of \dvi{} file.
\item[dvitomp]   \dvi{} to MPX (MetaPost pictures).
\item[dvitype]   \dvi{} to human-readable text.
\item[gftodvi]   Generic font proofsheets.
\item[gftopk]    Generic to packed fonts.
\item[gftype]    GF to human-readable text.
\item[makempx]   \MP{} label typesetting.
\item[mf]        Creating typeface families.
\item[mft]       Prettyprinting \MF{} source.
\item[mpost]     Creating technical diagrams.
\item[mpto]      MetaPost label extraction.
\item[newer]     Compare modification times.
\item[patgen]    Creating hyphenation patterns.
\item[pktogf]    Packed to generic fonts.
\item[pktype]    PK to human-readable text.
\item[pltotf]    Property list to TFM.
\item[pooltype]  Display \web{} pool files.
\item[tangle]    \web{} to Pascal.
\item[tex]       Typesetting.
\item[tftopl]    TFM to property list.
\item[vftovp]    Virtual font to virtual property list
\item[vptovf]    Virtual property list to virtual font.
\item[weave]     \web{} to \TeX.
\end{description}
\noindent The precise functions and syntax of these programs are described in
the documentation of the individual packages or of \Webc{} itself.
However, knowing a few principles governing the whole family of
programs will help you to benefit optimally from your \Webc{}
installation.

All programs honor the standard \acro{GNU} options:
\begin{description}
\item[\texttt{-{}-help\ \ \ }] print basic usage summary.
\item[\texttt{-{}-verbose}] print detailed progress report.
\item[\texttt{-{}-version}] print version information, then exit.
\end{description}

For locating files the \Webc{} programs use the path searching library
\KPS{}. This library uses a combination of environment variables and a
few configuration files to optimize searching the \TeX{} directory
tree. \Webc{} can handle more than one directory tree
simultaneously, which is useful if one wants to maintain \TeX's
standard distribution and local extensions in two distinct trees. To
speed up file searches the root of each tree has a file \file{ls-R},
containing an entry showing the name and relative pathname for all
files ``hanging'' under that root.

\subsection{\protect\KPS\ path searching}

Let us first describe the generic path searching mechanism of the \KPS{}
library. 

We call a \emph{search path} a colon- or semicolon\hyph sepa\-rated list
of \emph{path elements}, which are basically directory names.  A
search path can come from (a combination of) many sources.  To look up
a file \samp{my-file} along a path \samp{.:/dir}, \KPS{} checks each
element of the path in turn: first \file{./my-file}, then
\file{/dir/my-file}, returning the first match (or possibly all
matches).

In order to adapt optimally to all operating systems' conventions, on
non-Unix systems \KPS{} can use filename separators different from
``colon'' (\samp{:}) and ``slash'' (\samp{/}).

To check a particular path element \var{p}, \KPS{} first checks if a
prebuilt database (see ``Filename data\-base'' on 
page~\pageref{Filename-database}) applies to \var{p}, i.e., if the database
is in a directory that is a prefix of \var{p}.  If so, the path
specification is matched against the contents of the database.

If the database does not exist, or does not apply to this path
element, or contains no matches, the filesystem is searched (if this
was not forbidden by a specification starting with \samp{!!}\ and if
the file being searched for must exist).  \KPS{} constructs the list
of directories that correspond to this path element, and then checks
in each for the file being sought.

The ``file must exist'' condition comes into play with \samp{.vf} files and
input files read by \TeX's \cs{openin} command.  Such files may not
exist (e.g., \file{cmr10.vf}), and so it would be wrong to search the
disk for them.  Therefore, if you fail to update \file{ls-R} when you
install a new \samp{.vf} file, it will never be found.
Each path element is checked in turn: first the database, then the
disk.  If a match is found, the search stops and the result is
returned.

Although the simplest and most common path element is a directory
name, \KPS{} supports additional features in search paths: layered
default values, environment variable names, config file values, users'
home directories, and recursive subdirectory searching.  Thus, we say
that \KPS{} \emph{expands} a path element, meaning it transforms all
the specifications into basic directory name or names.  This is
described in the following sections in the same order as it takes
place.

Note that if the filename being searched for is absolute or explicitly
relative, i.e., starts with \samp{/} or \samp{./} or \samp{../},
\KPS{} simply checks if that file exists.

\ifSingleColumn
\else
\begin{figure*}
\verbatiminput{../examples/ex5.tex}
\setlength{\abovecaptionskip}{0pt}
  \caption{An illustrative configuration file sample}
  \label{fig:config-sample}
\end{figure*}
\fi

\subsubsection{Path sources}
\label{Path-sources}

A search path can come from many sources.  In the order in which
\KPS{} uses them:

\begin{enumerate}
\item 
  A user-set environment variable, for instance, \envname{TEXINPUTS}\@.
  Environment variables with a period and a program name appended
  override; e.g., if \samp{latex} is the name of the program being run,
  then \envname{TEXINPUTS.latex} will override \envname{TEXINPUTS}.
\item 
  A program-specific configuration file, for exam\-ple, a line
  ``\texttt{S /a:/b}'' in \cmdname{dvips}'s \file{config.ps}.
\item   A \KPS{} configuration file \file{texmf.cnf}, containing a line like
  \samp{TEXINPUTS=/c:/d} (see below).  
\item The compile-time default.
\end{enumerate} 
\noindent You can see each of these values for a given
search path by using the debugging options (see ``Debugging actions''
on page~\pageref{Debugging}).

\subsubsection{Config files}

\tolerance=3500
\KPS{} reads \emph{runtime configuration files} named \file{texmf.cnf}
for search path and other definitions.  The search path used to look
for these files is named \envname{TEXMFCNF} (by default such a file lives
in the \file{texmf/web2c} subdirectory).  \emph{All}
\file{texmf.cnf} files in the search path will be read and definitions
in earlier files override those in later files.  Thus, with a search
path of \verb|.:$TEXMF|, values from \file{./texmf.cnf} override those
from \verb|$TEXMF/texmf.cnf|.

\tolerance=1500

While reading the description of the format of the file
\file{texmf.cnf} below, please also refer to
appendix~\ref{app:texmf.cnf}, starting on
page~\pageref{app:texmf.cnf}, which lists the \file{texmf.cnf} file on
the \CD.

\begin{itemize}
\item 
  Comments start with ``\texttt{\%}'' and continue to the end of the line.
\item 
  Blank lines are ignored.
\item
  A \bs{} at the end of a line acts as a continuation character,
  i.e., the next line is appended.  Whitespace at the beginning of
  continuation lines is not ignored.
\item 
  Each remaining line has the form:
\begin{alltt}
  \emph{variable}[.\emph{progname}] [=] \emph{value}
\end{alltt}%
  where the \samp{=} and surrounding whitespace are optional.
\item 
  The ``\texttt{\var{variable}}'' name may contain any character other
  than whitespace, \samp{=}, or \samp{.}, but sticking to
  \samp{A-Za-z\_} is safest.
\item 
  If ``\texttt{.\var{progname}}'' is present, the definition only
  applies if the program that is running is named
  \texttt{\var{progname}} or \texttt{\var{progname}.exe}.  This allows
  different flavors of \TeX{} to have different search paths, for
  example.
\item ``\texttt{\var{value}}'' may contain any characters except
  ``\texttt{\%}'' and \samp{@}.  The
  ``\texttt{\$\var{var}.\var{prog}}'' feature is not available on the
  right-hand side; instead, you must use an additional variable.  A
  \samp{;}\ in ``\texttt{\var{value}}'' is translated to \samp{:}\ if
  running under Unix; this is useful to be able to have a single
  \file{texmf.cnf} for Unix, MSDOS and Windows systems.
\item 
  All definitions are read before anything is expanded, so
  variables can be referenced before they are defined.
\end{itemize}
A configuration file fragment illustrating most of these points is
\ifSingleColumn
shown below:

\verbatiminput{../examples/ex5.tex}
\else
shown in Figure~\ref{fig:config-sample}.
\fi

\subsubsection{Path expansion}
\label{Path-expansion}


\KPS{} recognizes certain special characters and constructions in
search paths, similar to those available in Unix shells.  As a
general example, the  complex path,
\verb+~$USER/{foo,bar}//baz+, expands to all subdirectories under
directories \file{foo} and \file{bar} in \texttt{\$USER}'s home
directory that contain a directory or file \file{baz}.  These
expansions are explained in the sections below.
%$
\subsubsection{Default expansion}
\label{Default-expansion}

\tolerance=2500
If the highest-priority search path (see ``Path sources'' on
page~\pageref{Path-sources}) contains an \emph{extra colon} (i.e.,
leading, trailing, or doubled), \KPS{} inserts at that point the
next-highest-prio\-rity search path that is defined.  If that inserted
path has an extra colon, the same happens with the next highest.  For
example, given an environment variable setting

\tolerance=1500

\begin{alltt}
>> \Ucom{setenv TEXINPUTS /home/karl:}
\end{alltt}
and a \code{TEXINPUTS} value from \file{texmf.cnf} of

\begin{alltt}
  .:\$TEXMF//tex
\end{alltt}
then the final value used for searching will be:

\begin{alltt}
  /home/karl:.:\$TEXMF//tex
\end{alltt}

Since it would be useless to insert the default value in more than one
place, \KPS{} changes only one extra \samp{:}\ and leaves any others in
place: it checks first for a leading \samp{:}, then a trailing
\samp{:}, then a doubled \samp{:}.

\subsubsection{Brace expansion}

A useful feature is brace expansion, which means that, for instance,
\verb+v{a,b}w+ expands to \verb+vaw:vbw+. Nesting is allowed.
This can be used to implement multiple \TeX{} hierarchies, by
assigning a brace list to \code{\$TEXMF}.
For example, in \file{texmf.cnf}, you find 
the following definition:
\begin{verbatim}
  TEXMF = {$HOMETEXMF,$TEXMFLOCAL,!!$VARTEXMF,!!$TEXMFMAIN}
\end{verbatim}
Using this you can then write something like
\begin{verbatim}
  TEXINPUTS = .;$TEXMF/tex//
\end{verbatim}
%$
which means that, after looking in the current directory, the
\code{\$HOMETEXMF/tex}, \code{\$TEXMFLOCAL/tex}, \code{\$VARTEXMF/tex}
and \code{\$TEXMFMAIN/tex} trees \emph{only}) will be searched (the
last two use using \file{ls-R} data base files). It is a convenient
way for running two parallel \TeX{} structures, one ``frozen'' (on a
\CD, for instance) and the other being continuously updated with new
versions as they become available.  By using the \code{\$TEXMF}
variable in all definitions, one is sure to always search the
up-to-date tree first.

\subsubsection{Subdirectory expansion}
\label{Subdirectory-expansion}

Two or more consecutive slashes in a path element following a directory
\var{d} is replaced by all subdirectories of \var{d}: first those
subdirectories directly under \var{d}, then the subsubdirectories under
those, and so on.  At each level, the order in which the directories are
searched is \emph{unspecified}.

If you specify any filename components after the \samp{//}, only
subdirectories with matching components are included.  For example,
\samp{/a//b} expands into directories \file{/a/1/b}, \file{/a/2/b},
\file{/a/1/1/b}, and so on, but not \file{/a/b/c} or \file{/a/1}.

Multiple \samp{//} constructs in a path are possible, but
\samp{//} at the beginning of a path is ignored.

\subsubsection{List of special characters and their meaning: a summary}

The following list summarises the meaning of special characters
in \KPS{} configuration files.

\newcommand{\CODE}[1]{\makebox[2.2em][l]{\code{#1}}}
\begin{description}
\item[\CODE{:}] Separator in path specification; at the beginning or
  the end of a path it substitutes the default path expansion.
\item[\CODE{;}] Separator on non-Unix systems (acts like \code{:}).
\item[\CODE{\$}] Variable expansion.
\item[\CODE{\string~}] Represents the user's home directory. 
\item[\CODE{\char`\{\dots\char`\}}] Brace expansion, e.g.,
    \verb+a{1,2}b+ will become \verb+a1b:a2b+.
\item[\CODE{//}] Subdirectory expansion (can occur anywhere in
    a path, except at its beginning).
\item[\CODE{\%}] Start of comment.
\item[\CODE{\bs}] Continuation character (allows multi-line entries).
\item[\CODE{!!}] Search \emph{only} database to locate file, \emph{do
    not} search the disk.
\end{description}


\subsection{Filename databases}
\label{Filename-database}


\KPS{} goes to some lengths to minimize disk accesses for searches.
Nevertheless, at installations with enough directories, searching each
possible directory for a given file can take an excessively long time
(this is especially true if many hundreds of font directories have to
be traversed.)  Therefore, \KPS{} can use an externally-built
``database'' file named \file{ls-R} that maps files to directories,
thus avoiding the need to exhaustively search the disk.

A second database file \file{aliases} allows you to give additional
names to the files listed in \file{ls-R}.  This can be helpful to
adapt to DOS-like ``8.3'' filename conventions in source files.

\subsubsection{The filename database}
\label{ls-R}

As explained above, the name of the main filename database must be
\file{ls-R}.  You can put one at the root of each \TeX{} hierarchy in
your installation that you wish to be searched (\code{\$TEXMF} by
default); most sites have only one hierarchy.  \KPS{} looks for
\file{ls-R} files along the \code{TEXMFDBS} path.

The recommended way to create and maintain \samp{ls-R} is to run the
\code{mktexlsr} script included with the distribution. It is invoked
by the various \samp{mktex}\dots\ scripts.  In principle, this script
just runs the command
\begin{alltt}
cd \var{/your/texmf/root} && ls -LAR ./ >ls-R
\end{alltt}
presuming your system's \code{ls} produces the right output format
(\acro{GNU}'s \code{ls} is all right).  To ensure that the database is always
up to date, it is easiest to rebuild it regularly via \code{cron}, so
that for changes in the installed files\Dash perhaps after installing or
updating a \LaTeX{} package\Dash the file \file{ls-R} is automatically
updated.

If a file is not found in the database, by default \KPS{} goes ahead
and searches the disk. If a particular path element begins with
\samp{!!}, however, \emph{only} the database will be searched for that
element, never the disk.


\subsubsection{kpsewhich: Standalone path searching}
\label{Invoking-kpsewhich}

The \texttt{kpsewhich} program exercises path searching independent of any
particular application.  This can be useful as a sort of \code{find}
program to locate files in \TeX{} hierarchies (this is used heavily in
the distributed \samp{mktex}\dots\ scripts).

\begin{alltt}
>> \Ucom{kpsewhich \var{option}\dots{} \var{filename}\dots{}}
\end{alltt}
The options specified in ``\texttt{\var{option}}'' can start with either \samp{-}
or \samp{-{}-}, and any unambiguous abbreviation is accepted.

\KPS{} looks up each non-option argument on the command line as a
filename, and returns the first file found. There is no option to
return all the files with a particular name (you can run the Unix
\samp{find} utility for that).

The more important options are described next.

\begin{description}
\item[\texttt{-{}-dpi=\var{num}}]\mbox{} 
  Set the resolution to ``\texttt{\var{num}}''; this only affects \samp{gf}
  and \samp{pk} lookups.  \samp{-D} is a synonym, for compatibility
  with \cmdname{dvips}.  Default is 600.
\item[\texttt{-{}-format=\var{name}}]\mbox{}\\
  Set the format for lookup to ``\texttt{\var{name}}''.  By default, the
  format is guessed from the filename. For formats which do not have
  an associated unambiguous suffix, such as \MP{} support files and
  \cmdname{dvips} configuration files, you have to specify the name as
  found in the first column of Table~\ref{TABKpathsea}, which lists
  currently recognized names, a description, 
  associated environment variables%,
  \footnote{You can find definitions for these environment variables
    in the file \file{texmf.cnf} (page~\pageref{app:texmf.cnf})}, and
  possible file extensions.
\end{description}


\def\KpathKey#1#2#3#4#5{% name, number, description, variables,
                        % suffixes
\SetRowColor#1 & #3 & #4 & #5\\}
\def\HEAD#1{\multicolumn{1}{l}{\emph{#1}}}
%
\begin{small}
% a footnoterule immediately under a bottom-of-page rule looks dead
% silly, so we suppress it
\renewcommand\footnoterule{\relax}
%
\begin{longtable}{@{}%
     >{\ttfamily}P{.16\textwidth}%               Col.1
     P{.3\textwidth}%                            Col 2
     >{\ttfamily\footnotesize}P{.29\textwidth}%  Col 3
     >{\ttfamily}P{.14\textwidth}%               Col 4
@{}}
\caption{Kpathsea file types}\label{TABKpathsea}\\
\emph{Name} & \HEAD{Description} & \HEAD{Variables}
& \HEAD{Suffixes}\\
\hline
\endfirsthead
\multicolumn{3}{l}{Kpathsea file types \emph{continued}}\\
\emph{Name} & \HEAD{Description} & \HEAD{Variables}
& \HEAD{Suffixes}\\
\hline
\noalign{\vspace{2pt}}
\endhead
\mbox{}\\
\hline
\endfoot
\KpathKey
 {afm}
 {4}
 {Adobe font metrics}
 {AFMFONTS}
 {.afm}
\KpathKey
 {base}
 {5}
 {Metafont memory dump}
 {MFBASES, TEXMFINI}
 {.base}
\KpathKey
 {bib}
 {6}
 {\BibTeX{} bibliography source}
 {BIBINPUTS, TEXBIB}
 {.bib}
\KpathKey
 {}
 {2}
 {bitmap fonts}
 {GLYPHFONTS, TEXFONTS}
 {}
\KpathKey
 {bst}
 {7}
 {\BibTeX{} style files}
 {BSTINPUTS}
 {.bst}
\KpathKey
 {cnf}
 {8}
 {Runtime configuration files}
 {TEXMFCNF}
 {.cnf}
\KpathKey
 {dvips config}
 {34}
 {\textsf{dvips} configuration files, e.g., \file{config.ps}
  and \file{psfonts.map}}
 {TEXCONFIG}
 {.map}
\KpathKey
 {fmt}
 {10}
 {\TeX{} memory dump}
 {TEXFORMATS, TEXMFINI}
 {.fmt, .efmt, .efm}
\KpathKey
 {gf}
 {0}
 {generic font bitmap}
 {GFFONTS}
 {.gf}
\KpathKey
 {graphic/figure}
 {25}
 {Encapsulated PostScript figures}
 {TEXPICTS, TEXINPUTS}
 {.eps, .epsi}
\KpathKey
 {ist}
 {35}
 {\textsf{makeindex} style files}
 {TEXINDEXSTYLE, INDEXSTYLE}
 {.ist}
\KpathKey
 {ls-R}
 {9}
 {Filename databases}
 {TEXMFDBS}
 {}
\KpathKey
 {map}
 {11}
 {Fontmaps}
 {TEXFONTMAPS}
 {.map}
\KpathKey
 {mem}
 {12}
 {MetaPost memory dump}
 {MPMEMS, TEXMFINI}
 {.mem}
\KpathKey
 {mf}
 {13}
 {Metafont source}
 {MFINPUTS}
 {.mf}
\KpathKey
 {mfpool}
 {14}
 {Metafont program strings}
 {MFPOOL, TEXMFINI}
 {.pool}
\KpathKey
 {mft}
 {15}
 {MFT style file}
 {MFTINPUTS}
 {.mft}
\KpathKey
 {}
 {41}
 {miscellaneous fonts}
 {MISCFONTS}
 {}
\KpathKey
 {mp}
 {16}
 {MetaPost source}
 {MPINPUTS}
 {.mp}
\KpathKey
 {mppool}
 {17}
 {MetaPost program strings}
 {MPPOOL, TEXMFINI}
 {.pool}
\KpathKey
 {MetaPost support}
 {18}
 {MetaPost support files, used by DMP}
 {MPSUPPORT}
 {}
\KpathKey
 {ocp}
 {19}
 {\OMEGA{} compiled process files}
 {OCPINPUTS}
 {.ocp}
\KpathKey
 {ofm}
 {20}
 {\OMEGA{} font metrics}
 {OFMFONTS, TEXFONTS}
 {.ofm, .tfm}
\KpathKey
 {opl}
 {21}
 {\OMEGA{} property lists}
 {OPLFONTS, TEXFONTS}
 {.opl}
\KpathKey
 {otp}
 {22}
 {\OMEGA{} translation process files}
 {OTPINPUTS}
 {.otp}
\KpathKey
 {ovf}
 {23}
 {\OMEGA{} virtual fonts}
 {OVFFONTS, TEXFONTS}
 {.ovf}
\KpathKey
 {ovp}
 {24}
 {\OMEGA{} virtual property lists}
 {OVPFONTS, TEXFONTS}
 {.ovp}
\KpathKey
 {pk}
 {1}
 {packed bitmap fonts}
 {\emph{program}FONTS \textrm{(\texttt{\emph{program}} being
     \textsmaller{\cmdname{XDVI}},
     etc.)}, PKFONTS, TEXPKS, GLYPHFONTS, TEXFONTS}
 {.pk}
\KpathKey
 {PostScript header}
 {30}
 {downloadable PostScript}
 {TEXPSHEADERS, PSHEADERS}
 {.pro, .enc}
\KpathKey
 {tex}
 {26}
 {\TeX{} source}
 {TEXINPUTS}
 {.tex, .cls, .sty, .clo, .def}
\KpathKey
 {TeX system documentation}
 {27}
 {Documentation files for the \TeX{} system}
 {TEXDOCS}
 {}
\KpathKey
 {TeX system sources}
 {29}
 {Source files for the \TeX{} system}
 {TEXSOURCES}
 {}
\KpathKey
 {texpool}
 {28}
 {\TeX{} program strings}
 {TEXPOOL, TEXMFINI}
 {.pool}
\KpathKey
 {tfm}
 {3}
 {\TeX{} font metrics}
 {TFMFONTS, TEXFONTS}
 {.tfm}
\KpathKey
 {Troff fonts}
 {31}
 {Troff fonts, used by DMP}
 {TRFONTS}
 {}
\KpathKey
 {truetype fonts}
 {36}
 {TrueType outline fonts}
 {TTFONTS}
 {.ttf, .ttc}
\KpathKey
 {type1 fonts}
 {32}
 {Type 1 PostScript outline fonts}
 {T1FONTS, T1INPUTS, TEXPSHEADERS, DVIPSHEADERS}
 {.pfa, .pfb}
\KpathKey
 {type42 fonts}
 {37}
 {Type 42 PostScript outline fonts}
 {T42FONTS}
 {}
\KpathKey
 {vf}
 {33}
 {virtual fonts}
 {VFFONTS, TEXFONTS}
 {.vf}
\KpathKey
 {web2c files}
 {38}
 {\Webc{} support files}
 {WEB2C}
 {}
\KpathKey
 {other text files}
 {39}
 {text files used by `\textsf{foo}'}
 {FOOINPUTS}
 {}
\KpathKey
 {other binary files}
 {40}
 {binary files used by `\textsf{foo}'}
 {FOOINPUTS}
 {}
\end{longtable}
\end{small}


\begin{multicols}{2}

\begin{description}
\item[] The last two entries in Table~\ref{TABKpathsea} are special
  cases, where the paths and environment variables depend on the name
  of the program: the variable name is constructed by converting the
  program name to upper case, and then appending \texttt{INPUTS}.
  
  The environment variables are set by default in the configuration
  file \file{texmf.cnf}. It is only when you want to override one or
  more of the values specified in that file that you might want to set
  them explicitly in your execution environment.

  Note that the \samp{-{}-format} and \samp{-{}-path} options are mutually
  exclusive.
\item[\texttt{-{}-mode=\var{string}}]\mbox{}\\
  Set the mode name to ``\texttt{\var{string}}''; this only affects \samp{gf}
  and \samp{pk} lookups.  No default: any mode will be found.
\item[\texttt{-{}-must-exist}]\mbox{}\\
  Do everything possible to find the files, notably including
  searching the disk.  By default, only the \file{ls-R} database is
  checked, in the interest of efficiency.
\item[\texttt{-{}-path=\var{string}}]\mbox{}\\
  Search along the path ``\texttt{\var{string}}'' (colon-%
  separated as usual),
  instead of guessing the search path from the filename.  \samp{//}
  and all the usual expansions are supported.
  The options \samp{-{}-path} and \samp{-{}-format} are mutually
  exclusive.
\item[\texttt{-{}-progname=\var{name}}]\mbox{}\\
  Set the program name to ``\texttt{\var{name}}''.
  This can affect the search paths via the ``\texttt{.\var{prognam}}''
  feature in configuration files.
  The default is \samp{kpsewhich}.
\item[\texttt{-{}-show-path=\var{name}}]\mbox{}\\
  shows the path used for file lookups of file type ``\texttt{\var{name}}''.
  Either a filename extension (\samp{.pk}, \samp{.vf}, etc.) or a
  name can be used, just as with \samp{-{}-format} option.
\item[\texttt{-{}-debug=\var{num}}]\mbox{}\\
  sets the debugging options to ``\texttt{\var{num}}''.
\end{description}


\subsubsection{Examples of use}
\label{SExamplesofuse}

Let us now have a look at \KPS{} in action.

\begin{alltt}
>> \Ucom{kpsewhich  article.cls}
/usr/local/texmf/tex/latex/base/article.cls
\end{alltt}
We are looking for the file \file{article.cls}. Since the \samp{.cls}
suffix is unambiguous we do not need to specify that we want to look for a
file of type ``tex'' (\TeX{} source file directories). We find it in
the subdirectory \file{tex/latex/base} below the \samp{TEXMF} root
directory.  Similarly, all of the following are found without problems
thanks to their unambiguous suffix.
\begin{alltt}
>> \Ucom{kpsewhich array.sty}
   /usr/local/texmf/tex/latex/tools/array.sty
>> \Ucom{kpsewhich latin1.def}
   /usr/local/texmf/tex/latex/base/latin1.def
>> \Ucom{kpsewhich size10.clo}
   /usr/local/texmf/tex/latex/base/size10.clo
>> \Ucom{kpsewhich small2e.tex}
   /usr/local/texmf/tex/latex/base/small2e.tex
>> \Ucom{kpsewhich tugboat.bib}
   /usr/local/texmf/bibtex/bib/beebe/tugboat.bib
\end{alltt}

The latter is a \BibTeX{} bibliography database for \textsl{TUGBoat} articles.

\begin{alltt}
>> \Ucom{kpsewhich cmr10.pk}
\end{alltt}
Font bitmap glyph files of type \file{.pk} are used by display
programs like \cmdname{dvips} and \cmdname{xdvi}.  Nothing is returned in
this case since there are no pre-generated Computer Modern \samp{.pk}
files on our system (since we use the Type1 versions on the \CD{}).
\begin{alltt}
>> \Ucom{kpsewhich ecrm1000.pk}
\ifSingleColumn   /usr/local/texmf/fonts/pk/ljfour/jknappen/ec/ecrm1000.600pk
\else
/usr/local/texmf/fonts/pk/ljfour/jknappen/
...                         ec/ecrm1000.600pk
\fi\end{alltt}
For the extended Computer Modern files we had to generate \samp{.pk}
files, and since the default \MF{} mode on our installation is
\texttt{ljfour} with a base resolution of 600 dpi (dots per inch),
this instantiation is returned.
\begin{alltt}
>> \Ucom{kpsewhich -dpi=300 ecrm1000.pk}
\end{alltt}
In this case, when specifying that we are interested in a resolution
of 300dpi (\texttt{-dpi=300}) we see that no such font is available on
the system. In fact, a program like \cmdname{dvips} or \cmdname{xdvi} would
go off and actually build the \texttt{.pk} files at the required
resolution using the script \cmdname{mktexpk}.

Next we turn our attention to \cmdname{dvips}'s header and configuration
files.  We first look at one of the commonly used files, the general
prolog \file{tex.pro} for \TeX{} support, before turning our attention
to the generic configuration file (\file{config.ps}) and the
PostScript font map \file{psfonts.map}.  As the \samp{.ps} suffix is
ambiguous we have to specify explicitly which type we are considering
(``\texttt{dvips config}'') for the file \texttt{config.ps}.
\begin{alltt}
>> \Ucom{kpsewhich tex.pro}
   /usr/local/texmf/dvips/base/tex.pro
>> \Ucom{kpsewhich --format="dvips config" config.ps}
   /usr/local/texmf/config/config.ps
>> \Ucom{kpsewhich psfonts.map}
   /usr/local/texmf/dvips/base/psfonts.map
\end{alltt}

We now take a closer look at the URW Times PostScript support files.
The name for these in Berry's font naming scheme is ``\texttt{utm}''.
The first file we look at is the configuration file, which contains
the name of the map file:
\begin{alltt}
>> \Ucom{kpsewhich --format="dvips config" config.utm}
/usr/local/texmf/dvips/psnfss/config.utm
\end{alltt}
The contents of that file is
\begin{alltt}
  p +utm.map
\end{alltt}
which points to the file \file{utm.map}, which we want to
locate next.
\begin{alltt}
>> \Ucom{kpsewhich --format="dvips config" utm.map}
   /usr/local/texmf/dvips/psnfss/utm.map
\end{alltt}
This map file defines the file names of the Type1 PostScript fonts in
the URW collection.  Its contents look like (we only show part of the
lines):
\begin{alltt}
  utmb8r  NimbusRomNo9L-Medi    ... <utmb8a.pfb
  utmbi8r NimbusRomNo9L-MediItal... <utmbi8a.pfb
  utmr8r  NimbusRomNo9L-Regu    ... <utmr8a.pfb
  utmri8r NimbusRomNo9L-ReguItal... <utmri8a.pfb
  utmbo8r NimbusRomNo9L-Medi    ... <utmb8a.pfb
  utmro8r NimbusRomNo9L-Regu    ... <utmr8a.pfb
\end{alltt}
Let us, for instance, take the Times Regular instance
\file{utmr8a.pfb} and find its position in the \file{texmf} directory
tree by using a search for Type1 font files:
\begin{alltt}
>> \Ucom{kpsewhich utmr8a.pfb}
\ifSingleColumn   /usr/local/texmf/fonts/type1/urw/utm/utmr8a.pfb
\else   /usr/local/texmf/fonts/type1/
...                       urw/utm/utmr8a.pfb
\fi\end{alltt}

It should be evident from these few examples how you can easily locate
the whereabouts of a given file. This is especially important if you
suspect that the wrong version of a file is picked up somehow, since
\cmdname{kpsewhich} will show you the first file encountered.

\subsubsection{Debugging actions}
\label{Debugging}

Sometimes it is necessary to investigate how a program resolves
file references. To make this feasible in a convenient way \KPS{} 
offers various debug levels:
\begin{itemize}
\item[\texttt{\ 1}] \texttt{stat} calls (file tests). When running
  with an up-to-date \file{ls-R} database this should almost give no
  output.
\item[\texttt{\ 2}] References to hash tables (like \file{ls-R}
  database, map files, configuration files).
\item[\texttt{\ 4}] File open and close operations.
\item[\texttt{\ 8}] General path information for file types
  searched by \KPS. This is useful to find out where a particular
  path for the file was defined.
\item[\texttt{16}] Directory list for each path element (only relevant
  for searches on disk).
\item[\texttt{32}] File searches.
\end{itemize}
A value of \texttt{-1} will set all the above options; in practice you
will probably always use these levels if you need any debugging.

Similarly, with the \cmdname{dvips} program, by setting a
combination of debug switches, one can follow in detail where files are being picked
up from.  Alternatively, when a file is not found, the debug trace
shows in which directories the program looks for the given file, so
that one can get an indication what the problem~is.

Generally speaking, as most programs call the \KPS{} library
internally, one can select a debug option by using the
\envname{KPATHSEA\_DEBUG} environment variable, and setting it to (a
combination of) values as described in the above list.

(Note for Windows users: it is not easy to redirect
all messages to a file in this system. For diagnostic purposes
you can temporarily \texttt{SET KPATHSEA\_DEBUG\_OUTPUT=err.log}).

Let us consider, as an example, a small \LaTeX{} source file,
\file{hello-world.tex}, which contains the following input.
\begin{verbatim}
  \documentclass{article}
  \begin{document}
  Hello World!
  \end{document}
\end{verbatim}
This little file only uses the font \file{cmr10}, so let us look
how \cmdname{dvips} prepares the PostScript file (we want to use the Type1
version of the Computer Modern fonts, hence the option \texttt{-Pcms}).
\begin{alltt}
>> \Ucom{dvips -d4100 hello-world -Pcms -o}
\end{alltt}
In this case we have combined \cmdname{dvips}'s debug class 4 (font
paths) with \KPS's path element expansion (see \cmdname{dvips} Reference
Manual, \OnCD{texmf/doc/html/dvips/dvips_toc.html}).
%\ifSingleColumn
%We get something
%like the following (we have rearranged the output for easier display):
%\begin{alltt}\small
%\begin{figure*}[tp]
\centering
\begin{BVerbatim}[fontsize=\small]
debug:start search(file=texmf.cnf, must_exist=1, find_all=1,
  path=.:/usr/local/bin/texlive:/usr/local/bin:
       /usr/local/bin/texmf/web2c:/usr/local:
       /usr/local/texmf/web2c:/.:/./teTeX/TeX/texmf/web2c:).
kdebug:start search(file=ls-R, must_exist=1, find_all=1,
  path=~/tex:/usr/local/texmf).
kdebug:search(ls-R) =>/usr/local/texmf/ls-R
kdebug:start search(file=aliases, must_exist=1, find_all=1,
  path=~/tex:/usr/local/texmf).
kdebug:search(aliases) => /usr/local/texmf/aliases
kdebug:start search(file=config.ps, must_exist=0, find_all=0,
  path=.:~/tex:!!/usr/local/texmf/dvips//).
kdebug:search(config.ps) => /usr/local/texmf/dvips/config/config.ps
kdebug:start search(file=/root/.dvipsrc, must_exist=0, find_all=0,
  path=.:~/tex:!!/usr/local/texmf/dvips//).
search(file=/home/goossens/.dvipsrc, must_exist=1, find_all=0,
  path=.:~/tex/dvips//:!!/usr/local/texmf/dvips//).
kdebug:search($HOME/.dvipsrc) =>
kdebug:start search(file=config.cms, must_exist=0, find_all=0,
  path=.:~/tex/dvips//:!!/usr/local/texmf/dvips//).
kdebug:search(config.cms)
=>/usr/local/texmf/dvips/cms/config.cms
\end{BVerbatim}
\caption{Finding configuration files}\label{fig:dvipsdbga}

\bigskip

\begin{BVerbatim}[fontsize=\small]
kdebug:start search(file=texc.pro, must\_exist=0, find\_all=0,
  path=.:~/tex/dvips//:!!/usr/local/texmf/dvips//:
       ~/tex/fonts/type1//:!!/usr/local/texmf/fonts/type1//).
kdebug:search(texc.pro) => /usr/local/texmf/dvips/base/texc.pro
\end{BVerbatim}
\caption{Finding the prolog file}\label{fig:dvipsdbgb}

\bigskip

\begin{BVerbatim}[fontsize=\small]
kdebug:start search(file=cmr10.tfm, must\_exist=1, find\_all=0,
  path=.:~/tex/fonts/tfm//:!!/usr/local/texmf/fonts/tfm//:
       /var/tex/fonts/tfm//).
kdebug:search(cmr10.tfm) => /usr/local/texmf/fonts/tfm/public/cm/cmr10.tfm
kdebug:start search(file=texps.pro, must\_exist=0, find\_all=0,
   ...
<texps.pro>
kdebug:start search(file=cmr10.pfb, must\_exist=0, find\_all=0,
  path=.:~/tex/dvips//:!!/usr/local/texmf/dvips//:
       ~/tex/fonts/type1//:!!/usr/local/texmf/fonts/type1//).
kdebug:search(cmr10.pfb) => /usr/local/texmf/fonts/type1/public/cm/cmr10.pfb
<cmr10.pfb>[1]
\end{BVerbatim}
\caption{Finding the font file}\label{fig:dvipsdbgc}
\end{figure*}

%\end{alltt}
%\else
The output (slightly rearranged) appears in
Figure~\ref{fig:dvipsdbga}.
\begin{figure*}[tp]
\centering
\begin{BVerbatim}[fontsize=\small]
debug:start search(file=texmf.cnf, must_exist=1, find_all=1,
  path=.:/usr/local/bin/texlive:/usr/local/bin:
       /usr/local/bin/texmf/web2c:/usr/local:
       /usr/local/texmf/web2c:/.:/./teTeX/TeX/texmf/web2c:).
kdebug:start search(file=ls-R, must_exist=1, find_all=1,
  path=~/tex:/usr/local/texmf).
kdebug:search(ls-R) =>/usr/local/texmf/ls-R
kdebug:start search(file=aliases, must_exist=1, find_all=1,
  path=~/tex:/usr/local/texmf).
kdebug:search(aliases) => /usr/local/texmf/aliases
kdebug:start search(file=config.ps, must_exist=0, find_all=0,
  path=.:~/tex:!!/usr/local/texmf/dvips//).
kdebug:search(config.ps) => /usr/local/texmf/dvips/config/config.ps
kdebug:start search(file=/root/.dvipsrc, must_exist=0, find_all=0,
  path=.:~/tex:!!/usr/local/texmf/dvips//).
search(file=/home/goossens/.dvipsrc, must_exist=1, find_all=0,
  path=.:~/tex/dvips//:!!/usr/local/texmf/dvips//).
kdebug:search($HOME/.dvipsrc) =>
kdebug:start search(file=config.cms, must_exist=0, find_all=0,
  path=.:~/tex/dvips//:!!/usr/local/texmf/dvips//).
kdebug:search(config.cms)
=>/usr/local/texmf/dvips/cms/config.cms
\end{BVerbatim}
\caption{Finding configuration files}\label{fig:dvipsdbga}

\bigskip

\begin{BVerbatim}[fontsize=\small]
kdebug:start search(file=texc.pro, must\_exist=0, find\_all=0,
  path=.:~/tex/dvips//:!!/usr/local/texmf/dvips//:
       ~/tex/fonts/type1//:!!/usr/local/texmf/fonts/type1//).
kdebug:search(texc.pro) => /usr/local/texmf/dvips/base/texc.pro
\end{BVerbatim}
\caption{Finding the prolog file}\label{fig:dvipsdbgb}

\bigskip

\begin{BVerbatim}[fontsize=\small]
kdebug:start search(file=cmr10.tfm, must\_exist=1, find\_all=0,
  path=.:~/tex/fonts/tfm//:!!/usr/local/texmf/fonts/tfm//:
       /var/tex/fonts/tfm//).
kdebug:search(cmr10.tfm) => /usr/local/texmf/fonts/tfm/public/cm/cmr10.tfm
kdebug:start search(file=texps.pro, must\_exist=0, find\_all=0,
   ...
<texps.pro>
kdebug:start search(file=cmr10.pfb, must\_exist=0, find\_all=0,
  path=.:~/tex/dvips//:!!/usr/local/texmf/dvips//:
       ~/tex/fonts/type1//:!!/usr/local/texmf/fonts/type1//).
kdebug:search(cmr10.pfb) => /usr/local/texmf/fonts/type1/public/cm/cmr10.pfb
<cmr10.pfb>[1]
\end{BVerbatim}
\caption{Finding the font file}\label{fig:dvipsdbgc}
\end{figure*}
\cmdname{dvips} starts by locating its working files. First,
\file{texmf.cnf} is found, which gives the definitions of the search
paths for the other files, then the file database \file{ls-R} (to
optimize file searching) and the file \file{aliases}, which makes it
possible to declare several names (e.g., a short \acro{DOS}-like ``8.3'' and
a more natural longer version) for the same file.  Then \cmdname{dvips}
goes on to find the generic configuration file \file{config.ps}
before looking for the customization file \file{.dvipsrc} (which, in
this case is \emph{not found}).  Finally, \cmdname{dvips} locates the
config file for the Computer Modern PostScript fonts \file{config.cms}
(this was initiated with the \texttt{-Pcms} option on the \cmdname{dvips}
command).  This file contains the list of the ``map'' files which
define the relation between the \TeX{}, PostScript and file system
names of the fonts.
\begin{alltt}
>> \Ucom{more /usr/local/texmf/dvips/cms/config.cms}
   p +ams.map
   p +cms.map
   p +cmbkm.map
   p +amsbkm.map
\end{alltt}
\cmdname{dvips} thus goes on to find all these files, plus the generic
map file \file{psfonts.map}, which is always loaded (it contains
declarations for commonly used PostScript fonts; see the last part of
Section \ref{SExamplesofuse} for more details about PostScript map
file handling).

At this point \cmdname{dvips} identifies itself to the user\ldots
\begin{alltt}\ifSingleColumn
This is dvips 5.78 Copyright 1998 Radical Eye Software (www.radicaleye.com)
\else\small{}This is dvips 5.78 Copyright 1998 Radical Eye...
\fi\end{alltt}% decided to accept slight overrun in tub case
\ifSingleColumn
\ldots and then goes on to look for the prolog file \file{texc.pro}, 
\begin{alltt}\small
kdebug:start search(file=texc.pro, must\_exist=0, find\_all=0, 
  path=.:~/tex/dvips//:!!/usr/local/texmf/dvips//:
       ~/tex/fonts/type1//:!!/usr/local/texmf/fonts/type1//).
kdebug:search(texc.pro) => /usr/local/texmf/dvips/base/texc.pro
\end{alltt}
\else
then goes on to look for the prolog file \file{texc.pro} (see
Figure~\ref{fig:dvipsdbgb}).

\fi
After having found the file in question, \cmdname{dvips} outputs
date and time, and informs us that it will generate the
file \file{hello-world.ps}, then that it needs the font file
\file{cmr10}, and that the latter is declared as ``resident'':
\begin{alltt}\small
TeX output 1998.02.26:1204' -> hello-world.ps 
Defining font () cmr10 at 10.0pt 
Font cmr10 <CMR10> is resident.
\end{alltt}
Now the search is on for the file \file{cmr10.tfm}, which is found,
then a few more prolog files (not shown) are referenced, and finally
the Type1 instance \file{cmr10.pfb} of the font is located and
included in the output file (see last line).
\begin{alltt}\small
kdebug:start search(file=cmr10.tfm, must\_exist=1, find\_all=0, 
  path=.:~/tex/fonts/tfm//:!!/usr/local/texmf/fonts/tfm//:
       /var/tex/fonts/tfm//).
kdebug:search(cmr10.tfm) => /usr/local/texmf/fonts/tfm/public/cm/cmr10.tfm
kdebug:start search(file=texps.pro, must\_exist=0, find\_all=0, 
   ...
<texps.pro>
kdebug:start search(file=cmr10.pfb, must\_exist=0, find\_all=0, 
  path=.:~/tex/dvips//:!!/usr/local/texmf/dvips//:
       ~/tex/fonts/type1//:!!/usr/local/texmf/fonts/type1//).
kdebug:search(cmr10.pfb) => /usr/local/texmf/fonts/type1/public/cm/cmr10.pfb
<cmr10.pfb>[1] 
\end{alltt}

\subsection{Runtime options}

Another of the nice features of \Webc{} is its possibility to
control a number of memory parameters (in particular, array sizes) via
the runtime file \file{texmf.cnf} read by \KPS{}. The listing of
\file{texmf.cnf} is shown in Appendix~\ref{app:texmf.cnf}, starting on
page~\pageref{app:texmf.cnf}; the settings of all parameters can be
found in Part 3 of that file. The more important control variables are:
\begin{description}
\item[\texttt{main\_memory}] 
  Total words of memory available, for
  \TeX{}, \MF{} and \MP.  You must make a new format file for each
  different setting. For instance, you could generate a ``huge''
  version of \TeX{}, and call the format file \texttt{hugetex.fmt}.
  Using the standard way of specifying the program name used by \KPS{},
  the particular value of the \texttt{main\_memory} variable will then
  be read from \file{texmf.cnf} (compare the generic value
  and the ``huge'' one instantiated by \texttt{hugetex},
  etc.).
\item[\texttt{extra\_mem\_bot}]
  Extra space for ``large'' \TeX{} data structures:
  boxes, glue, breakpoints, etc.  Especially useful if you use
  \PiCTeX{}.
\item[\texttt{font\_mem\_size}]
  Number of words for font information available for \TeX. This
  is more or less the total size of all TFM files read.
\item[\texttt{hash\_extra}]
  Additional space for the hash table of control sequence names.
  Approximately 10,000 control sequences can be stored in the main
  hash table; if you have a large book with numerous cross-references,
  this might not be enough. You can see that both the 
  \cmdname{hugetex} and \cmdname{pdflatex} program invocations ask for
  an extra 15,000 control sequences (the default value of
  \texttt{hash\_extra} is zero).
\end{description}
Of course, this facility is no substitute for truly dynamic arrays and
memory allocation, but since this is extremely difficult to implement
in present \TeX, these runtime parameters provide a practical compromise
allowing some flexibility.

%--------------------------------------------------------


\section{Acknowledgements}\label{sec:acknowl}
This edition of TeXLive is edited by Sebastian Rahtz, with the
major contributors being Fabrice Popineau, who has worked away
unceasingly at the Win32 part of the package (especially the setup!)
and contributed in many different ways with ideas, advice and code;
and Staszek Wawrykiewicz, who provided great checking feedback, and
co-ordinated the Polish contributions. Kaja Christiansen performed a
vital role in endless recompilations on assorted Unix platforms, and
Robin Laakso coordinated production for TUG. Vladimir Volovich 
did great work on cleaning the source and making other improvements,
while Gerben Wierda did all the work for MacOSX.

We are very grateful for past and present help from
\begin{itemize}
\item The German-speaking \TeX{} Users (DANTE e.V.), who provided a machine
 on which the master of the CD-ROM is developed and maintained; and Rainer
Sch\"opf and Reinhard Zierke who look after it;
\item The Perforce company, for providing a free copy of their
excellent  change management system, which we have used to
 manage the CD-ROM
contents;
\item Karl Berry, who provided the original \Webc{} distribution, and
  has continued to give invaluable advice, encouragement, and help;
\item Mimi Burbank, who arranged access 
 at the Florida State University School of Computational Science
 and Information Technology
 to a slew of different
  computers to compile \TeX{} on, and acted as an essential guinea-pig
  whenever asked;
\item Kaja Christiansen, who provided essential feedback, compilation, and
  documentation preparation;
\item Thomas Esser, without whose marvellous \teTeX{} package this
  \CD{} would certainly not exist, and whose continual help makes it a
  better product;
\item Michel Goossens, who co-authored the documentation;
\item Eitan Gurari, whose \TeX4ht was used to create the \HTML{}
  version of this documentation, and who worked tirelessly to improve
  it at short notice;
\item Art Ogawa and Pat Monohon, who coordinated releases for TUG;
\item Petr Olsak, who coordinated and checked all the Czech/Slovak
  material very carefully;
\item Olaf Weber, for his patient assembly and maintenance of Web2c;
\item Graham Williams, on whose work the catalogue of packages depends.
\end{itemize}

Gerhard Wilhelms, Volker Schaa, Fabrice Popineau, Janka
Chleb\'\i{}kov\'a, Staszek Wawrykiewicz, Erik Frambach, and Ulrik
Vieth kindly translated documentation at various times into their
respective languages, checked other documentation, and provided very
welcome feedback.


\section{History}

This \CD{} distribution is a joint effort by many \TeX{} Users
Groups, including those from Germany, the Netherlands, the UK, France, 
the Czech Republic/Slovakia, India, Poland and Russia, as well as
the international TUG.
Discussion began in late 1993 when the Dutch \TeX{} Users Group was
starting work on its 4All\TeX{} \CD{} for \acro{MS-DOS} users, and it was
hoped at that time to issue a single, rational, \CD{} for all
systems. This was far too ambitious a target, but it did spawn not
only the very successful 4All\TeX{} \CD{}, but also the \acro{TUG}
Technical Council working group on a \emph{\TeX{} Directory
  Structure}, which specified how to create consistent and manageable
collections of \TeX{} support files. The final draft of the \TDS{} was
published in the December 1995 issue of \textsl{TUGboat}, and it was clear from
an early stage that one desirable product would be a model structure
on \CD{}. The \CD{} you now have is a very direct result of the
working group's deliberations. It was also clear that the success of
the 4All\TeX{} \CD{} showed that Unix users would benefit from a
similarly easy system, and this is the other main strand of \TeXLive.

We undertook to make a new Unix-based \TDS{} \CD{} in the autumn of
1995, and quickly identified Thomas Esser's \teTeX{} as the ideal
setup, as it already had multi-platform support and was built with
portability across file systems in mind. Thomas agreed to help, and
work began seriously at the start of 1996. The first edition was
released in May 1996. At the start of 1997, Karl Berry completed a
major new release of his \Webc{} package, which included nearly all
the features which Thomas Esser had added in \teTeX, and we decided to
base the 2nd edition of the \CD{} on the standard \Webc, with the
addition of \teTeX's \texttt{texconfig} script. The 3rd edition of the
\CD{} was based on a major revision of \Webc, 7.2, by Olaf Weber; at the
same time, a new revision of \teTeX was being made, 
and \TeXLive{} shares almost all of its features. The 4th edition
followed the same pattern, using a new version of te\TeX, and a new
release of \Webc{} (7.3).  The system now included a complete Windows setup.

For the 5th edition (March 2000) many parts of the \CD{} were revised
and checked, updating hundreds of packages. Package details were
stored in XML files. But the major change for \TeX\ Live 5 was that
all non-free software was removed. Everything on this \CD{} should be
compatible with the Debian Free Software Guidelines
(\url{http://www.debian.org/intro/free}); we have done our best to
check the license conditions of all packages, but we would very much
appreciate hearing of any mistakes.

The 6th edition (July 2001) had a lot material updated. The major
change was a new install concept: the user could select a more exact set
of needed collections. Language-related collections were completely
reorganized, so selecting any of them installs not only macros, fonts,
etc., but also prepares an appropriate \texttt{language.dat}.

The 7th edition of 2002 had as a major addition a setup
for MacOSX, and the usual myriad of updates to all sorts
of packages and programs. An important goal was integration
of the source back with teTeX, to correct the drift apart
in versions 5 and 6.

\section{Future versions}

\emph{This \CD{} is not a perfect product!}  We plan to re-issue it
once a year, and would like to provide more help material, more
utilities, more installation programs, and (of course) an
ever-improved and checked tree of macros and fonts. This work is all
done by hard-pressed volunteers in their limited spare time, and a
great deal remains to be done. If you can help, don't hesitate to put
your name forward!

Corrections, suggestions and additions for future revisions
should be sent to:\hfill\null
\begin{quote}
Sebastian Rahtz\\
{}7 Stratfield Road\\
Oxford OX2 7BG\\
United Kingdom\\
\EmailURL{rahtz@tug.org}
\end{quote}


Updates, notes, and suggestions will be made available on \CTAN{} 
in \path|info/texlive|. A \acro{WWW} page for
information and ordering details is at
\url{http://www.tug.org/tex-live.html}.

\end{multicols}

\Section{The texmf.cnf file}
\label{app:texmf.cnf}
\begin{small}
\listinginput{1}{../examples/texmf.cnf}
\end{small}

\end{document}



Notes on Nicolle Zellner's problems:
------------------------------------
- autoexec.bat gave error, ``too many parameters''.
fix: in setvars.bat file, put "" around %PATH% to take care of spaces.

- ``out of environment space'' when rebooting meant that autoexec.bat was
not executing properly-- path and texmfhome variable were not being set.
fix: put in config.sys file the line: shell=c:\command.com /e:1024 /p
(and reboot probably)

[Note: alternative advice written by Pat: 
  The environment space is an area of memory in which the OS stores
  variables such as path. To increase the memory size do the following:

  Open a MS-DOS window. Click on the Properties icon (looks like a hand
  holding a sheet of paper). Click on the Memory tab in the Properties
  window. Change the value under Initial Environment: to 4096. Apply the
  properties. Click OK on the warning about needing to restart your computer.
  The new environment memory will be in effect after the reboot.]

- windvi did not retain settings.
fix: in setvars.bat file, add command: set Home=c:\Home and also create 
the directory c:\Home.
copy windvi.cnf to c:\Home\
reboot

-ispell doesn't work (can't write to c:\tmp, can't create dictionary)
In addition to the the c:\Home addition above, create the folder:
d:\tmp\  (d: is where the files are that she is using.)
Now ispell should work from the command line.
Still can't get it to work from within \cmdname{WinShell}. (change cmd line to %c.tex,
it was reading the project main file)



Notes on installing WinEdt:
---------------------------
If you choose to install WinEdt, you will be asked for a folder name: any
folder name will do, as this is for temporary storage. An acceptable
choice is \verb+C:\Winedt+. Click \textbf{unzip}, then \textbf{OK}, 
and finally \textbf{Close}.
---

later, to complete WinEdt installation:

Run "winedt5.exe" and complete the setup procedure. 

Get from  ftp://ftp.dante.de/pub/fptex/supplementary/ 
30/08/2000  20:17               70 844 winedt5-cnf.zip
10/09/2000  10:43              839 086 winedt5-patch4.zip

then unpack (use unzip or winzip with subdir creation enabled, NOT pkunzkip) 
 "winedt5-patch4.zip" inside the "c:\program files\winedt" directory 
then at last unpack "winedt5-cnf.zip".

When this is done, go inside the 'Options->Configurations' menu,
choose the "fpTeX Direct" option. 

If windvi previewer doesn't work, go to options->menu setup-> accessories
and change the Menu Setup so the call for DVI Preview is analogous to
the call for Dvips-- i.e., don't use "DVIVIEW" in the command line, 
but rather fptex-bin and dviwin.exe. 


