%% BEGIN fancybox-doc
%%
%% COPYRIGHT 1992, by Timothy Van Zandt, Timothy.VAN-ZANDT@insead.edu.
%% COPYRIGHT 2010, by Herbert Voss hvoss@tug.org.
%%
%% Documentation for fancybox.sty.
%%
%% This file may be distributed and/or modified under the conditions of
%% the LaTeX Project Public License, either version 1.2 of this license
%% or (at your option) any later version.  The latest version of this
%% license is in:
%% 
%%    http://www.latex-project.org/lppl.txt
%% 
%% and version 1.2 or later is part of all distributions of LaTeX version
%% 1999/12/01 or later.
%%
%%
%%
%% This creates two temporary files: \jobname.ex1 and \jobname.tmp

\documentclass[12pt]{article}
\usepackage [T1]{fontenc}
\usepackage{mathpazo,url}

\usepackage{fancybox}
\let\FBfv\fileversion
\usepackage{url}

\VerbatimFootnotes
\makeatletter
% Short meta (works in verbatim. Can't use < for other purposes.
\catcode`\<=13 \def<#1>{{\rm\it #1\/}}    % <meta> (works in verbatim)

% Short verbatim. " can appear in verbatim environments.
\def\temp{\Verb"}
\expandafter\def\expandafter\dospecials\expandafter{\dospecials\do\"}
\catcode`\"=13
\let"\temp

% Verbatim item:
\newcommand{\vitem}{\SaveVerb[{\def\bf{}\item[\UseVerb{\MyTemp}]}]{\MyTemp}}

%% EXAMPLES:

\setlength{\fboxsep}{6pt}

\begin{VerbatimOut}{\jobname.ex1}
% 1. Save example verbatim to \jobname.tmp,
% 2. Input verbatim with \catcode`\"=14 (" is a comment).
% 3. Input again with \catcode\`"=9 (" is ignored).

\renewcommand{\EveryVerbatimLine}[2]{}
\renewcommand{\EveryVerbOutLine}[2]{}
\newcommand\BeginExample{%
  \VerbatimEnvironment\begin{VerbatimOut}{\jobname.tmp}}
\newcommand\EndExample{%
  \end{VerbatimOut}%.
  \renewcommand{\EveryVerbatimLine}{}%
  \renewcommand{\EveryVerbatimCodes}{\catcode`\"=14}%
  \LVerbatimInput{\jobname.tmp}%
  \catcode`\"=9}
\newenvironment{example}{\BeginExample}{\EndExample
  \begin{center}\input{\jobname.tmp}\end{center}}
\newenvironment{example*}{\BeginExample}%
  {\EndExample \input{\jobname.tmp}}
\newenvironment{example**}{\BeginExample}%
  {\EndExample \globaldefs=1 \input{\jobname.tmp}}
\end{VerbatimOut}

\input{\jobname.ex1}
\makeatother

\title{Documentation for fancybox.sty:\\
  Box tips and tricks for \LaTeX\thanks{Documentation revised by Herbert Vo\ss}}
\author{Timothy Van Zandt\\ \url{Timothy.VAN-ZANDT@insead.edu}}
\date{Version \FBfv\\ \today}


\begin{document}
\maketitle
\begin{abstract}
"fancybox.sty", together with its documentation, gives extensive answers to
and solutions for many questions about how to frame or rotate this or that in
\LaTeX. It also contains commands for shadow, double and oval frames.
\end{abstract}
\vfill
Thanks to Heiko Oberdiek, Marcin Wolinski.
\clearpage
\tableofcontents
\clearpage
\section{Fancy frames}

"fancybox.sty" has five variants of \LaTeX's "\fbox" command:
\begin{quote}\raggedright
"\shadowbox", "\doublebox", "\ovalbox" (with "\thinlines") and "\Ovalbox"
(with "\thicklines").
\end{quote}
Here are examples:\footnote{In this documentation, the default value of
"\fboxsep" has been changed from 3pt to 6pt.}
\begin{example}
  \shadowbox{\large\bf New Glarus Birdwatch}
\end{example}
\begin{example}
  \doublebox{\large\bf New Glarus Birdwatch}
\end{example}
\begin{example}
  \ovalbox{\large\bf New Glarus Birdwatch}
\end{example}
\begin{example}
  \Ovalbox{\large\bf New Glarus Birdwatch}
\end{example}

The distance between the box and the frame is "\fboxsep", as with \LaTeX's
"\fbox" command. The commands use other parameters as well:
\begin{description}
\vitem"\shadowbox" The width of the frame is "\fboxrule" (the same as with
"\fbox"). The width of the shadow is "\shadowsize" (default: "4pt").

\vitem"\doublebox" The width of the inner frame is .75"\fboxrule", and the
width of the outer frame is 1.5"\fboxrule". The distance between the two
frames is 1.5"\fboxrule" plus .5pt.

\vitem"\ovalbox" The width of the frame is set by the "\thinlines"
declaration. The diameter of the corner arcs is set with the "\cornersize"
command.
\begin{LVerbatim}
  \cornersize{<num>}
\end{LVerbatim}
sets the diameter of the corners arcs to <num> times the lessor of the width
and height of the box.
\begin{LVerbatim}
  \cornersize*{<dim>}
\end{LVerbatim}
sets the diameter of the corner arcs to <dim>. This is all approximate,
because \LaTeX\ has a limited range of arc sizes to choose from. The default
is
\begin{LVerbatim}
  \cornersize{.5}
\end{LVerbatim}

\vitem"\Ovalbox" This is like "\ovalbox", except that the width of the lines
is set by the "\thicklines" declaration.
\end{description}

There are no analogs to \LaTeX's "\framebox" command, which has various
optional arguments not supported by "\fbox". You can get the exact same
functionality by putting the argument of the above framing commands in a
"\makebox".

There is also a variant "\fancyoval" of \LaTeX's "\oval" picture object. The
difference is that "\oval" always makes the diameter of the corner arcs as
large as possible, and "\fancyoval" uses the "\cornersize" command to set the
diameter.

\section{A short course on boxes}

\begingroup\sloppy
The "\shadowbox", "\doublebox", "\ovalbox" and "\Ovalbox" commands described
in the previous section are examples of LR-box commands, meaning that their
argument is processed in LR mode. \LaTeX\ LR-box commands include "\mbox",
"\makebox", "\fbox", "\framebox", "\sbox" and "\savebox". All the PSTricks
commands whose argument is text are LR-box commands, including, e.g, the
framing, rotating, scaling and positioning commands, and some of the node
commands. Any rotation command is an LR-box command.

\endgroup

The purpose of the rest of this documentation is to provide answers to, and
solutions for, frequently asked questions about using LR-box commands with
\LaTeX. I will use "\fbox" for the leading example of a box framing
command,\footnote{In the examples using "\fbox", be aware that the default
value of "\fboxsep" has been changed in this documentation from 3pt to 6pt.}
and "\rotateleft" for the leading example of a box rotation command.
("fancybox.sty" does not contain a "\rotateleft" command, as this must be
implemented via "\special"'s, but there are numerous box-rotation style files
around.) However, most of what is said here applies to any LR-box command.

In each LR-box command, the text is processed in restricted horizontal mode,
which is referred to as ``LR-mode'' in Lamport's {\em \LaTeX: User's Guide and
Reference Manual}. In restricted horizontal mode, the input, consisting of
regular characters and boxes, is made into one (long or short) line. There is
no line-breaking, nor can there be vertical mode material such as an entire
displayed equation. However, the fact that you can include another box means
that this isn't really a restriction.

For one thing, alignment environments such as \LaTeX's "tabular" are just
boxes, and thus present no problem. Picture environments and the LR-box
commands themselves are also just boxes. Entire paragraphs or other vertical
mode material such as displayed equations can be nested in a "\parbox" or
"minipage".

\section{Defining LR-box environments}

To frame a "minipage", you have to write
\begin{LVerbatim}
  \fbox{%
    \begin{minipage}{3in}
      blah
    \end{minipage}}
\end{LVerbatim}
You might want to define an environment "fminipage" that frames its contents,
but you can't use
\begin{LVerbatim}
  \newenvironment{fminipage}%
    {\fbox{\begin{minipage}}%
    {\end{minipage}}}
\end{LVerbatim}
because the braces are not balanced in the definition.

"fancybox.sty" contains an "Sbox" environment that makes it easy to define
your own LR-box environments. It is a variant of \LaTeX's "\sbox" command that
saves the contents of the environment in a storage bin that can be retrieved
with the command "\TheSbox".\footnote{The difference between
\begin{LVerbatim}
  \begin{Sbox}
    blah
  \end{Sbox}
  \TheSbox
\end{LVerbatim}
and
\begin{LVerbatim}
  \newsavebox{\mybox}
  \sbox{\mybox}{blah}
  \usebox{\mybox}
\end{LVerbatim}
is that "Sbox" saves the contents globally, and "\TheSbox" erases the contents
globally.}
For example, here is a framed minipage:
\begin{LVerbatim}
  \begin{Sbox}
    \begin{minipage}{3in}
      blah
    \end{minipage}
  \end{Sbox}
  \fbox{\TheSbox}
\end{LVerbatim}
and here is an "fminipage" environment that works:
\begin{example**}
  \newenvironment{fminipage}%
    {\begin{Sbox}\begin{minipage}}%
    {\end{minipage}\end{Sbox}\fbox{\TheSbox}}
\end{example**}
Let's see that it really works:
\begin{example}
  \begin{fminipage}{2in}
    Since the former doesn't use braces to delimit
    the contents of the box, $\ldots$
  \end{fminipage}
\end{example}


\section{Math}

In-line math, or pieces of a displayed equation (as opposed to a whole
equation), are horizontal mode material, but most LR-box commands switch out
of math mode when they occur in math mode. Thus, you have to explicitly switch
back in to math mode when desired.\footnote{This is {\em not} true for the
PSTricks LR-box commands.}
For example:
\begin{example}
  $x + y = \fbox{$\Omega$}$
\end{example}

You also have to explicitly write
\begin{quote}
  "\scriptstyle", "\scriptscriptstyle" or "\displaystyle"
\end{quote}
if you want one of these special math styles. For example, here I will frame
an equation, but not the equation number:
\begin{example*}
  \begin{equation}
    \fbox{$\displaystyle
      \int_{\Omega_0} \zeta(\omega) d\omega
      \geq \bar{r}$}
  \end{equation}
\end{example*}

Entire displayed equations or "eqnarray" environments work differently because
they are vertical mode material. Thus, they have to go inside a "\parbox" or
"minipage". E.g.,
\begin{example*}
  \newlength{\mylength}
  \[
    \setlength{\fboxsep}{15pt}
    \setlength{\mylength}{\linewidth}
    \addtolength{\mylength}{-2\fboxsep}
    \addtolength{\mylength}{-2\fboxrule}
    \fbox{%
      \parbox{\mylength}{
        \setlength{\abovedisplayskip}{0pt}
        \setlength{\belowdisplayskip}{0pt}
        \begin{equation}
          x + y = z
        \end{equation}}}
  \]
\end{example*}
The outer "\[" "\]" are just used to display the boxed equation, rather than
actually switch into math mode. Note how I set the width of the "\parbox" so
that the displayed box would exactly have width "\linewidth".\footnote{That is
what "\mylength" is for. It is better to define a single scratch length that
you reuse rather than creating a new one each time.}

 I also set the display skips to "0pt" and increased the size of "\fboxsep" so
that I would have the same distance all around between the equation and the
frame.

This is again a mouthful, and so I might instead define:\footnote{The reason
for using "\minipage" instead of "\begin{minipage}", and so on, is that with
AmS-\LaTeX, "\begin" and "\end" cannot appear in the definition of a new
equation environment.}
\begin{example**}
  \newenvironment{FramedEqn}%
    {\setlength{\fboxsep}{15pt}
      \setlength{\mylength}{\linewidth}%
      \addtolength{\mylength}{-2\fboxsep}%
      \addtolength{\mylength}{-2\fboxrule}%
      \Sbox
      \minipage{\mylength}%
         \setlength{\abovedisplayskip}{0pt}%
         \setlength{\belowdisplayskip}{0pt}%
         \equation}%
    {\endequation\endminipage\endSbox
      \[\fbox{\TheSbox}\]}
\end{example**}
"fancybox.sty" doesn't bother defining any such environments, because there
are too many possible designs. But let's see if the one above works:
\begin{example*}
  \begin{FramedEqn}
    \Rightarrow P\sim\xi(P_\gamma)- \frac{1}{3}
  \end{FramedEqn}
\end{example*}

"fancybox.sty" contains a "Beqnarray" environment, which is like the
"eqnarray" environment but it is not vertical mode material. Instead, it
produces a box just large enough to hold all the equations. For example:
\begin{example}
  \fbox{%
    \begin{Beqnarray*}
      x & = & y\\
      y & > & x \\
      \int_4^5 f(x)dx & = & \sum_{i\in F} x_i
    \end{Beqnarray*}}
\end{example}
The unstarred version produces standard equation numbers on the right (even
with the "leqno" style option and AmS-\LaTeX). It might not work with special
equation numbering macros.


\section{Floats}\label{floats}

A common mistake is to put a whole "table", "figure" or other float
environment inside an LR-box command. Instead, you should put everything that
is {\em inside} the environment (including the "\caption", if you want that
boxed too) inside a "minipage" of the desired width, and then put the
"minipage" inside the LR-box command.

For example:
\begin{example*}
  \begin{table}[h]
    \begin{center}
      \fbox{%
        \begin{minipage}{.8\textwidth}
          \begin{center}
            \begin{tabular}{rl}
              foo & bar
            \end{tabular}
          \end{center}
          \caption{A table of foo and bar.}
        \end{minipage}}
    \end{center}
  \end{table}
\end{example*}
Note how I had to use "center" twice: once to center the framed box, and again
to center the stuff inside the box.

That is a mouthful, and so I might define a "FramedTable" environment like the
following, which sets the size of the "minipage" so that the framed box is
exactly the width of the page (no need for the first "center" environment this
time):
\begin{example**}
  \newenvironment{FramedTable}%
    {\begin{table}[h]
      \begin{Sbox}%
      \setlength{\mylength}{\textwidth}%
      \addtolength{\mylength}{-2\fboxsep}%
      \addtolength{\mylength}{-2\fboxrule}%
      \begin{minipage}{\mylength}}%
    {\end{minipage}\end{Sbox}\fbox{\TheSbox}\end{table}}%
\end{example**}
Now let's see if it works:
\begin{example*}
  \begin{FramedTable}
    \begin{center}
      \begin{tabular}{rl}
        foo & bar
      \end{tabular}
    \end{center}
    \caption{A table of foo and bar.}
  \end{FramedTable}
\end{example*}

The most common reason to want to rotate an entire float, caption and all, is
to put it on a page by itself in landscape mode, centered both horizontally
and vertically. Compared to the table framing we did above, we just have to
replace "\fbox" by our box rotation command (e.g., "\rotateleft" or whatever),
set the width of the minipage to "\textheight" (if you want to use the full
size of the page), and use the float position specifier "[p]". "fancybox.sty"
contains an environment,
\begin{LVerbatim}
  \begin{landfloat}{<float>}{<rotation command>}
    ...
  \end{landfloat}
\end{LVerbatim}
that automates this. It has two arguments: the name of the floating
environment, and your rotation command. For example, if "\rotateleft{foo}"
rotates "foo" by 90 degrees, and you want a landscape mode "table", then try
\begin{LVerbatim}
  \begin{landfloat}{table}{\rotateleft}
    ...
  \end{landfloat}
\end{LVerbatim}
If the whole document is in landscape mode, then "landfloat" gives you a
portrait-mode float---good for a table that is too tall to fit in landscape
mode.

If you don't add a caption to a float, it doesn't matter much what floating
environment you use (e.g., "table", "figure" or whatever). Thus, you can put
anything in a landscape float. For example, suppose I have a very wide
equation. Then I can write:
\begin{LVerbatim}
  \begin{landfloat}{table}{\rotateleft}
    \begin{equation}
      ...
    \end{equation}
  \end{landfloat}
\end{LVerbatim}
"fancybox.sty" defines a generic caption, "\GenericCaption", that doesn't
affect the numbering of floats, doesn't make an entry in a list of floats, and
doesn't add anything to the argument you give. I could have used this if I
wanted to add a caption to the previous example.

\section{Center, flushleft and flushright}

There are two ways to box a "center", "flushleft" or "flushright" environment.
If you have long lines and you want \TeX\ to do the line breaking, then put
the environment inside a "minipage".\footnote{List and other such environments
work best inside a "minipage" environment rather than a "\parbox".} If you
have short lines and you want the frame to adjust itself to the size of the
longest line, then use "fancybox.sty"'s
\begin{quote}
  "Bcenter", "Bflushleft" or "Bflushright"
\end{quote}
environment instead.

These are basically just "tabular" environments with a single column, and so
each line should end with "\\", or "\\["<dim>"]" to insert extra space, and
the text on each line must be balanced. Also, each line is sitting in its own
group, and so, e.g., if you want to change the font for the whole environment,
you should do this before the environment rather than after.
Like the "tabular" environment, the box that is produced has the baseline at
the center, unless you include an optional argument "[t]" to align the box
with the top line or "[b]" to align the box with the bottom line.

For example:
\begin{example}
  \setlength{\fboxsep}{10pt}%
  \fbox{%
    \begin{Bcenter}
      Love of life\\
      and other short stories\\
      by Policarpa Salabarrieta
  \end{Bcenter}}
\end{example}
Compare this with:
\begin{example}
  \setlength{\fboxsep}{10pt}%
  \fbox{%
    \begin{minipage}{6cm}
    \begin{center}
      Love of life and other short stories
      by Policarpa Salabarrieta
    \end{center}
    \end{minipage}}
\end{example}

In either case, if we want the resulting framed box centered, we have to
include it inside another "center" environment.

Here is another example:
\begin{example}
  My list: \fbox{%
    \begin{Bflushleft}[b]
      Galanga root\\
      Coconut\\
      Tempeh
    \end{Bflushleft}}
\end{example}


\section{Lists}

There are again two ways to box a list environment such as
\begin{quote}
  "itemize", "enumerate" or "description".
\end{quote}
You can put it in a "minipage" of pre-determined size and let \TeX\ do the
line breaking, or you can use
\begin{quote}
  "Bitemize", "Benumerate" or "Bdescription"
\end{quote}
instead, and let the box adjust to the size of the longest line.

For example.
\begin{example}
  \fbox{%
    \begin{Bitemize}
      \item Groceries
      \item Clean hamster cages
      \item Pick up Peter
    \end{Bitemize}}
\end{example}

 Most of the usual list parameters are irrelevant except for "\itemsep" and
"\labelsep". These environments are also based on the "tabular" environment,
and so each item should be balanced text. You can't write "\vspace{4pt}"
either, but you can insert an extra amount of space before an item by writing
"\item(4pt)" (or "\item(4pt)[label]" if you have a label). Also, you can break
lines within an item using "\\" or "\\[<dim>]". For example:
\begin{example}
  \fbox{%
    \begin{Bdescription}
      \item[David] Groceries
      \item[Eli]  Hamster cages\\
                  Surreal numbers
      \item(3pt)[Doris] Pick up Peter
    \end{Bdescription}}
\end{example}

These environments also have an optional argument, "[t]" to align the box with
the top line, and "[b]" to align the box with the bottom line.
\begin{example}
  To do:
  \fbox{\setlength{\itemsep}{0pt}%
    \begin{Benumerate}[t]
      \item Groceries
      \item Hamster cages
      \item Pick up Peter
    \end{Benumerate}}
\end{example}

There is also a generic "\Blist" command that is analogous to \LaTeX's
"\list".  It has the same two obligatory arguments, plus a third optional
"[t]" or "[b]" argument for changing the alignment.

\section{Superimposing boxes}

The command
\begin{LVerbatim}
  \boxput*(<x>,<y>){<LR stuff1>}{<LR stuff2>}
\end{LVerbatim}
puts <LR stuff1> either behind (the default) or in front of (with the "*") <LR
stuff2>.\footnote{You will only notice the difference between "\boxput" and
"\boxput*" if you are using color implemented by "\special"'s} The resulting
box has the dimensions of <LR stuff2>.

The coordinates "(<x>,<y>)" determine where the center of <LR stuff1> is
positioned. For example, "(0,0)" puts it at the center of <LR stuff2>, "(0,1)"
puts it at the center-top, and "(-1,-1)" puts it in the bottom-left corner.

More generally, the origin of the coordinate system is at the center of <LR
stuff2>, one unit in the vertical direction is half the vertical size of <LR
stuff2>, and one unit in the horizontal direction is half the width of <LR
stuff2>. Thus, <x> and <y> should always be numbers (without units such as
"pt" or "cm"), with one exception: If <y> is "b" or "B", <LR stuff1> is
positioned vertically at the baseline of <LR stuff2>. "(<x>,<y>)" is
optional---the default is "(0,0)".

Except for the funny coordinate system, "\boxput" is like the "\put" command
in a "picture". In particular, you can use "\makebox" in <LR stuff1> to
fine-tune the positioning, and <LR-stuff1> can contain a "picture"
environment.

You might use "\boxput" to put a ``water mark'' in the background of a box, or
to put a label next to a box, when you don't want this label to take up any
space. Here is a lazy example:
\begin{example}
  \boxput{\makebox(0,0){\Huge Censored!}}{\parbox{3in}{%
    The origin of the coordinate system is at the center of
    {\em LR stuff2}, and one unit in the x-direction
    is half the width of {\em LR stuff2}.}}
\end{example}

This would be a lot more interesting using PSTricks, with ``Censored!'' in the
foreground, rotated 30 degrees, and red:
\begin{LVerbatim}
  \boxput*{\rput{30}{\Huge\red Censored}}{\parbox{3in}{%
    blah blah}}
\end{LVerbatim}

Here is another example using PSTricks:
\begin{LVerbatim}
  \newcommand{\titledframe}[2]{%
    \boxput*(0,1){\psframebox*{#1}}%
      {\psframebox[framesep=12pt]{#2}}}
\end{LVerbatim}
\newcommand{\titledframe}[2]{%
  \boxput*(0,1){#1}%
    {\fboxsep=12pt \fbox{#2}}}
The following example illustrated roughly how it works, but ``My title'' does
not blot out the frame behind it because this documentation does not use
PSTricks. \begin{example}
  \titledframe{My title}{%
    \parbox{2in}{The title is superimposed
      on the top part of the frame.}}
\end{example}

\section{Framing a whole page}

The commands
\begin{LVerbatim}
  \fancyput*(<x>,<y>){<LR stuff>}
  \thisfancyput*(<x>,<y>){<LR stuff>}
\end{LVerbatim}
are pretty much like "\put" commands in a \LaTeX\ picture environment whose
origin is 1 inch down and to the right from the top left corner of the
page.\footnote{Don't blame me for \TeX's peculiar 1 inch margins.} The only
differences are that (i) that any LR-mode material is permitted (including
\LaTeX\ "picture" environment, of course), (ii) the coordinate is optional
"(0pt,0pt)" is substituted by default), and (iii) if the coordinate is
included, you {\em must} specify the units.

"\thisfancyput" affects only the current page, and is a global declaration
(analogous to "\thispagestyle").

If you include the optional "*", then the command adds to, rather than
replaces, other things that have been inserted with "\fancyput" or
"\thisfancyput".

These commands are particularly useful for framing a page, because you can get
a frame that is, e.g., 1 inch from each side of the physical page without
having to worry about what margins you are using for the document. Here is an
example:
\begin{example*}
  \thisfancyput(3.25in,-4.5in){%
    \setlength{\unitlength}{1in}\fancyoval(7,9.5)}
\end{example*}
You could also use "\fancyput" to add some kind of ``watermark'' or background
image (e.g., a light gray ``DRAFT'').

There are other commands that directly frame or in some other way box the page
of text:
\begin{LVerbatim}
  \fancypage{<cmds1>}{<cmds2>}
  \thisfancypage{<cmds1>}{<cmds2>}
\end{LVerbatim}

Each finished page, before adding the headers and footers, (and thus having
width and height "\textwidth" and "\textheight", is boxed with
\begin{LVerbatim}
  <cmds1>{<pagebox>}
\end{LVerbatim}
Thus, <cmds1> should be, or should end with, a command whose argument can be a
box, such as "\fbox" or "\rotateleft".

Then the headers and footers are added, using the new width of the page, and
this is boxed with
\begin{LVerbatim}
  <cmds2>{<pagebox>}
\end{LVerbatim}
The same rules apply to <cmds2> as to <cmds1>.

Here is an example:
\begin{example*}
  \thisfancypage{%
    \setlength{\fboxsep}{8pt}%
    \setlength{\shadowsize}{8pt}%
    \shadowbox}{}
\end{example*}

{\bf Warning:} The commands described in this section change \LaTeX's output
routine, and may not work with document styles that do the same. Also, bad
arguments can cause serious errors with uninformative error messages.

\section{Switching to landscape mode midstream}

The most common reason to switch to landscape mode midstream is to rotate a
float, such as a table or figure. This was discussed in Section \ref{floats}.

If you want to rotate one or more pages, without rotating the headers and
footers, use the
\begin{LVerbatim}
  \begin{LandScape}{<cmd>}
    ...
  \end{LandScape}
\end{LVerbatim}
environment. <cmd> should be the command for rotating the page 90 degrees to
the left or right. (E.g., "\rotateleft", or "\rotate[l]".)

If you want to rotate the headers, footers and margins as well, use the
\begin{LVerbatim}
  \begin{Landscape}{<paperwidth>}{<paperheight>}{<cmd>}
    ...
  \end{Landscape}
\end{LVerbatim}
environment (the small "s" makes the difference) to rotate the pages left
(counterclockwise), and use the "Landscape*" environment (same arguments) to
rotate the pages right (clockwise). The three arguments are the width of the
paper, the height of the paper, and the rotation command you are using. For
example, if I have a portrait mode document using the US 8.5in by 11in paper,
and if "\rotateleft{foo}" rotates "foo" 90 degrees counterclockwise, then I
can write
\begin{LVerbatim}
  \begin{Landscape}{8.5in}{11in}{\rotateleft}
\end{LVerbatim}

You can use "\LandScape", "\Landscape" and "\Landscape*", rather then the
"LandScape", "Landscape" and "Landscape*" environments, if you want the rest
of the document to be in landscape mode.

If your document is being printed in landscape mode, then these environments 
switch to portrait mode.

For example, suppose I have a landscape mode document, and I want to switch to
portrait mode for the rest of the document, rotating the pages to the
``right'' with "\rotateright". Then I would write
\begin{LVerbatim}
  \Landscape*{11in}{8.5in}{\rotateright}
\end{LVerbatim}

These environments switch the text width and height, leaving the margins
exactly as they were before. It is quite possible that you want to make other
changes to the page parameters after switching to landscape mode, but as
Lamport points out the \LaTeX\ {\em User's Guide and Reference Manual}, this
generally doesn't work right in the middle of the document. "fancybox.sty" has
a command "\UsePageParameters" which gets around this. It should be used right
after you change the page parameters (and the page parameter changes should
come right after the beginning of the landscape environment, or "\clearpage").

{\bf Warning:} The commands and environments described in this section change
\LaTeX's output routine, and may not work with document styles that do the
same. Also, bad arguments can cause serious errors with uninformative error
messages.

\section{Verbatim}

If you try to frame some verbatim text by typing
\begin{LVerbatim}
  \fbox{%
    \begin{minipage}{5cm}
      \begin{verbatim}
        \My \Program \Listing
        if { foo } { bar } fi
      \end{verbatim}
    \end{minipage}}
\end{LVerbatim}
you will get nonsense at best. This is because the argument to "\fbox" is read
before the "\begin{verbatim}" is processed. But then it is too late for \TeX\
to go back and interpret the contents of the verbatim environment literally
rather than as special \TeX\ commands and characters.

One solution is to use the "Sbox" environment:
\begin{example}
  \begin{Sbox}
  \begin{minipage}{5cm}
  \begin{verbatim}
  \My \Program \Listing
  if { foo } { bar } fi
  \end{verbatim}
  \end{minipage}
  \end{Sbox}
  \setlength{\fboxsep}{8pt}
  \fbox{\TheSbox}
\end{example}

"fancybox.sty" also contains a command that ``fixes'' \LaTeX's LR-box commands
for use with verbatim text:
\begin{LVerbatim}
  \VerbBox{<cmd>}{<LR stuff>}
\end{LVerbatim}
This is like
\begin{LVerbatim}
  <cmd>{<LR stuff>}
\end{LVerbatim}
but <LR stuff> can contain verbatim text.\footnote{Or other tricks that
involve "\catcode" changes, as occurs with some foreign language macros.} For
example:
\begin{example}
  \VerbBox{\fbox}{\verb+\foo{bar}+}
\end{example}

For footnotes, put the command "\VerbatimFootnotes" in the preamble, and then
you can use verbatim commands or environments in the argument of "\footnote".
This is an optional feature because it might conflict with somebody else's
modification of the footnote system.

If you try to define your own framed verbatim environment with
\begin{LVerbatim}
  \newenvironment{FramedVerb}%
    {\begin{Sbox}\begin{minipage}{5in}\begin{verbatim}}
    {\end{verbatim}\end{minipage}\end{Sbox}
      \setlength{\fboxsep}{8pt}\fbox{\TheSbox}}
\end{LVerbatim}
and then type
\begin{LVerbatim}
  \begin{FramedVerb}
    if { foo } { bar } fi
  \end{FramedVerb}
\end{LVerbatim}
you will again run into trouble because after the "\begin{verbatim}", \LaTeX\
is searching for the literal string "\end{verbatim}" as the end of the
verbatim text. It just skips right over the "\end{FramedVerb}" and may well
continue to the end of the file or until it throws up.

"fancybox.sty" contains some verbatim environments that get around this
problem and that have other advantages for LR-boxing verbatim listings, when
compared to the standard \LaTeX\ "verbatim" environment. Admittedly, many of
their special features have nothing to do with boxes.

Here are the basic verbatim environments:
\begin{description}
  \vitem"Verbatim" Works pretty much like \LaTeX's "verbatim".
  \vitem"LVerbatim" Like "Verbatim", but "list" rather than "trivlist" is used
to display the listing, and so it is indented from the left margin. (This is
what I am using for verbatim listings in this document.)
  \vitem"BVerbatim[<pos>]" Produces a box with the same width as the longest
verbatim line. The baseline is in the center, unless you include the optional
argument "[t]" for alignment with the top line or "[b]" for alignment with the
bottom line.
  \vitem"VerbatimOut{<file>}" Writes the verbatim text to <file>.
  \vitem"SaveVerbatim{<cmd>}" Saves the verbatim text as <cmd>. <cmd> is
defined globally, without checking whether <cmd> is already defined. Use
obviously innocuous names like "\MyTemp".
\end{description}

{\bf Important:} For any of these verbatim environments, or new verbatim
environments you define yourself (see below), nothing should come after
"\begin{Verbatim}" or before "\end{Verbatim}" on the same line --- not even
spaces!%
  \footnote{If you need to allow something to come before "\end{Verbatim}",
then you have two options:
  \begin{itemize}
  \item Put the command "\AltGetVerbatim" in the preamble. This switches to a
scheme where anything preceding "\end{Verbatim}" is simply ignored. This can
cause problems if you do really weird things with active characters or other
commands within the verbatim environment (e.g., active conditionals that are
not balanced within a line of verbatim text), but in this case you are
probably a good enough hacker to use the next option.
  \item "\EndVerbatimTokens" is a token register that you can set to the
tokens that should precede "\end{Verbatim}" on the same line, with their
verbatim "\catcode"'s.
  \end{itemize}}
If you put something after "\begin{Verbatim}" on the same line, it is simply
ignored. However, if you put something after "\end{Verbatim}" on the same
line, or if you misspell "\end{Verbatim}", you will get an error such as
\begin{LVerbatim}
  ! File ended while scanning use of \Verbatim.
\end{LVerbatim}
and the document will end at that point.

You can define new verbatim environments using "\newenvironment". You just
have to start the definition with
\begin{LVerbatim}
  \VerbatimEnvironment
\end{LVerbatim}
 For example, here is the framed verbatim environment we tried earlier:
\begin{example**}
  \newenvironment{FramedVerb}%
    {\VerbatimEnvironment
      \begin{Sbox}\begin{minipage}{5cm}\begin{Verbatim}}%
    {\end{Verbatim}\end{minipage}\end{Sbox}
      \setlength{\fboxsep}{8pt}\fbox{\TheSbox}}
\end{example**}
Let's give it a try:
\begin{example}
  \begin{FramedVerb}
  if { foo } { bar } fi
  \end{FramedVerb}
\end{example}

Here are three commands for inputting a whole file verbatim. The file must end
with a new line.
\begin{description}
  \vitem"\VerbatimInput{<file>}" Like "\Verbatim".
  \vitem"\LVerbatimInput{<file>}" Like "\LVerbatim".
  \vitem"\BVerbatimInput[<pos>]{<file>}" Like "\BVerbatim".
\end{description}

Here are three commands for making use of verbatim text that has been saved to
a command:
\begin{description}
  \vitem"\UseVerbatim{<cmd>}" Like "\Verbatim".
  \vitem"\LUseVerbatim{<cmd>}" Like "\LVerbatim".
  \vitem"\BUseVerbatim[<pos>]{<cmd>}" Like "\BVerbatim".
\end{description}
The "SaveVerbatim" environment and the "\UseVerbatim" commands are useful for 
including verbatim text in the argument of "\marginpar", "\fbox" and other
commands. For example, here is another way to define the "FramedVerb"
environment:
\begin{LVerbatim}
  \newenvironment{FramedVerb}%
    {\VerbatimEnvironment
      \begin{SaveVerbatim}{\MyTemp}}%
    {\end{SaveVerbatim}%
      \setlength{\fboxsep}{8pt}%
      \fbox{\begin{minipage}{5cm}\UseVerbatim{\MyTemp}
        \end{minipage}}}
\end{LVerbatim}

Here are some verbatim commands for short-pieces of (in-line) verbatim text:
\begin{description}
  \vitem"\Verb<char> <literal> <char>" \ \\
  Like \LaTeX's "\verb" command, but it will complain if it encounters a new
line in <literal>.\footnote{Be careful that your word processing does not
insert one for you.} For example:
\begin{example*}
  "\begin{quote}
  The main use for the \Verb+SaveVerbatim+
  environment and the \Verb+\UseVerbatim+
  commands is to include $\ldots$
  "\end{quote}
\end{example*}
  \vitem"\UseVerb{<cmd>}" Like "\UseVerbatim", but without any particular
formatting. It is intended for including short pieces of literal text saved
with "\SaveVerb" (below).\footnote{But it can also be used for multiple lines
saved with the "SaveVerbatim" environment if you want to do the formatting
yourself. E.g., try this in a tabbing environment with "\VerbatimTab"
appropriately defined.}
  \vitem"\SaveVerb[<whatever>]{<cmd>}<char> <literal> <char>"\ \\
  This is like "\Verb", but it saves <literal> as <cmd>, and then returns to
the optional argument <whatever>. Like the "SaveVerbatim" environment, it
defines <cmd> globally without checking whether <cmd> is already defined.
Without the optional argument, the most common use is for including verbatim
text in a "\marginpar", "\section" or other command argument.

  The optional argument can be used for special tricks. For example, all the
listings of commands in this documentation use "\vitem" in a "description"
environment, where "\vitem" is defined by:\footnote{The braces enclosing the
optional argument of "\SaveVerb" prevent the "]" inside the argument from
being mistaken for the end of the argument.}
\begin{LVerbatim}
  \newcommand{\vitem}%
    {\SaveVerb[{\item[\UseVerb{\MyTemp}]}]{\MyTemp}}
\end{LVerbatim}
Whereas
\begin{LVerbatim}
  \item[\Verb"\foo"]
\end{LVerbatim}
would not work because after "\item" reads its argument it is too late to
interpret "\foo" literally,
\begin{LVerbatim}
  \vitem"foo"
\end{LVerbatim}
does work because it is equivalent to
\begin{LVerbatim}
  \SaveVerb{\MyTemp}"foo"\item[\UseVerb{\MyTemp}]
\end{LVerbatim}
\end{description}


These environments and commands use various parameters that make it easy to
customize their behavior. However, until you want to find the need for such
customization, you might as well ignore the rest of this section.

Internally, "fancybox.sty" separates the reading and formatting of verbatim
text. Most of the environments and commands perform both functions, but
"SaveVerbatim" and "\SaveVerb" only read the text, while "UseVerbatim" (and
company) and "\UseVerb" only format the text. "VerbatimOut" gets special
treatment. The parameters that apply to each class of verbatim environment or
command is listed in Table \ref{verbtable}.

\begin{table}
\begin{center}
\renewcommand{\arraystretch}{0}
\setlength{\tabcolsep}{7pt}
\newcommand{\vs}{& \\[\tabcolsep]}
\newcommand{\VS}{\\ \vs \hline \vs}
\begin{tabular}{|l|l|}
\hline \vs
{\em Where} & {\em What}
\\ \vs \hline
\hline \vs
  \begin{Bflushleft}
    Environments\\
    that format
  \end{Bflushleft}
&
  \begin{Bflushleft}
    "\VerbatimSpace"\\
    "\VerbatimTab"\\
    "\VerbatimFont"\\
    "\VerbatimFuzz"\\
    "\EveryVerbatimLine"\\
    "\EveryVerbatim"\\
    "\ThisVerb"
  \end{Bflushleft}
\VS
  \begin{Bflushleft}
    Environments\\
    that read
  \end{Bflushleft}
&
  \begin{Bflushleft}
    "\EveryVerbatimCodes"\\
    "\ThisVerbCodes"
  \end{Bflushleft}
\VS
  \begin{Bflushleft}
    "\Verb" and\\
    "\UseVerb"
  \end{Bflushleft}
&
  \begin{Bflushleft}
    "\VerbSpace"\\
    "\VerbTab"\\
    "\VerbFont"\\
    "\EveryVerb"\\
    "\ThisVerb"
  \end{Bflushleft}
\VS
  \begin{Bflushleft}
    "\Verb" and\\
    "\SaveVerb"
  \end{Bflushleft}
&
  \begin{Bflushleft}
    "\EveryVerbCodes"\\
    "\ThisVerbCodes"
  \end{Bflushleft}
\VS
  \begin{Bflushleft}
    "VerbatimOut"
  \end{Bflushleft}
&
  \begin{Bflushleft}
    "\EveryVerbOutCodes"\\
    "\ThisVerbCodes"\\
    "\EveryVerbOutLine"\\
    "\ThisVerb"
  \end{Bflushleft}
\\ \vs \hline
\end{tabular}
\caption{Parameters for the verbatim environments and
commands.\label{verbtable}}
\end{center}
\end{table}

All the parameters, including "\VerbatimFuzz", are ordinary commands, and
should be changed with "\renewcommand".

Here is a description of each of the parameters for environments that format
the verbatim text:
\begin{description}
  \vitem"\VerbatimSpace" The insertion text for spaces. The default is "\ ",
which produces a blank space. Change it to "\ttspace" to get \ttspace.
  \vitem"\VerbatimTab" The insertion text for tabs. The default is
\begin{LVerbatim}
  \ \ \ \ \ \ \ \
\end{LVerbatim}
  \vitem"\VerbatimFont" The font to use for verbatim text. The default is
"\tt"
  \vitem"\VerbatimFuzz" This is the amount by which lines can be too long in a
"Verbatim" or "LVerbatim" environment before you get overfull "\hbox"
warnings. This threshold is usually .1pt, but the default definition of
"\VerbatimFuzz" is "2pt" because verbatim lines won't break and are therefore
often too long.
  \vitem"\EveryVerbatimLine" This is inserted at the beginning of each line of
verbatim environments or verbatim files. By default it does nothing. I like to
indent each line in the verbatim environment in the input file by 2 spaces, so
I define
\begin{LVerbatim}
  \renewcommand{\EveryVerbatimLine}[2]{}
\end{LVerbatim}
to eat those spaces. (But I have to remember to put in two spaces or space
markers for blank lines too.) You might also use it to number the lines. For
example:
\begin{example}
  \newcounter{VerbLineNo}
  \renewcommand{\EveryVerbatimLine}%
    {\makebox[10pt][r]{%
       \stepcounter{VerbLineNo}%
       \tiny\rm\arabic{VerbLineNo}}%
    \hspace{10pt}}
  \renewcommand{\EveryVerbatim}%
    {\setcounter{VerbLineNo}{0}}
  \begin{SaveVerbatim}{\MyTemp}
  \setlength{\fboxsep}{15pt}
  \setlength{\mylength}{\linewidth}
  \end{SaveVerbatim}
  \fbox{\BUseVerbatim{\MyTemp}}
\end{example}
  \vitem"\EveryVerbatim" Whatever else you want to say before formatting the
verbatim text. By default, it does nothing.
  \vitem"\ThisVerb" This is executed before any of the commands above, and
then its value is cleared. Use this to customize a single verbatim formatting
environment.
\end{description}

Here is a description of the parameters for environments that read the
verbatim text:
\begin{description}
  \vitem"\EveryVerbatimCodes" This command is inserted just before reading the
verbatim text. Use it to play with "\catcode"'s (see the {\em \TeX book}). For
example, I might type
\begin{LVerbatim}
  \renewcommand{\EveryVerbatimCodes}{\catcode`\"=14}
\end{LVerbatim}
if I want to use \Verb+"+ as a comment character in verbatim text.%
%% The use of SaveVerbatim and \SaveVerb is not necessary here because
%% \VerbatimFootnotes is in effect. This is just a test.
\begin{SaveVerbatim}{\MyTemp}
  \def\MyQuote{"}    % \MyQuote is now the character ",
  \def\temp{\Verb"}  %    in case I need it.
  \catcode`\"=13     % Now " is like a command.
  \let"\temp         % Now "foo" is like \Verb"foo"
  \def\do{\noexpand\do\noexpand}      % Now " can be used in verbatim
  \edef\dospecials{\dospecials\do\"}  % environments anyway.
\end{SaveVerbatim}\SaveVerb{\Foo}+"+\SaveVerb{\FFoo}+\Verb"foo"+%
\footnote{Here is another "\catcode" trick. We make \UseVerb{\Foo} a short
verbatim command, so that we can say \UseVerb{\Foo foo\Foo} instead of
\UseVerb{\FFoo}:
\LUseVerbatim{\MyTemp}}
  \vitem"\ThisVerbCodes" This command is executed before "\EveryVerbCodes",
and then it is cleared. Use this to fool with the "\catcode"`s of a single
verbatim environment.
\end{description}

The parameters for "\Verb", "\UseVerb" and "\SaveVerb" and the "VerbatimOut"
environment are analogous to the similar commands for other environments.

Here is an example of the use of "\ThisVerb" to define a variant of "\Verb"
that uses \ttspace\ to mark spaces:
\begin{LVerbatim}
  \newcommand{\SVerb}{%
    \renewcommand{\ThisVerb}%
      {\renewcommand{\VerbatimSpace}{\ttspace}}%
    \Verb}
\end{LVerbatim}

Finally, without further comment, here are the definitions of the "example"
and "example*" environments that were used for the examples in this document:
{\renewcommand{\EveryVerbatimLine}{}\LVerbatimInput{\jobname.ex1}}


\end{document}
%% END fancybox.doc
