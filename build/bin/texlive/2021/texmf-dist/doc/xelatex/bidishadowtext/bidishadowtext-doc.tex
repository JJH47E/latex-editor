\documentclass{ltxdoc}
\usepackage{holtxdoc}
\begin{document}
\title{The \xpackage{bidishadowtext} Package}
\author{Vafa Khalighi\\\xemail{persian-tex@tug.org}}
\maketitle
\vskip 0pt plus 3fill
\fbox{%
\begin{minipage}{\dimexpr(\textwidth-2\fboxsep-2\fboxrule)}
If you want to report any bugs or typos and corrections in the documentation,
or ask for any new features, or suggest any improvements, or ask any questions
about the package, then please do not send any direct email to me; I will not 
answer any direct email. Instead please use the issue tracker:

\medskip
  \centerline{\url{https://github.com/vafa/bidishadowtext/issues}}

\medskip
In doing so, please always explain your issue well enough, always include
a minimal working example showing the issue, and always choose the appropriate
label for your query (i.e. if you are reporting any bugs, choose `Bug' label). 
\end{minipage}
}
\tableofcontents
 \section{Introduction}
The \xpackage{shadowtext} package allows you to have colored shadow text; unfortunately this package does not work well together with \xpackage{bidi} package.


The \xpackage{bidishadowtext} package is a re-implementation of \xpackage{shadowtext} package adding bidi support.

\section{Documentation}
All the commands of \xpackage{shadowtext} package are prefixed with \texttt{bidi} in \xpackage{bidishadowtext} package; for instance instead using \cs{shadowtext} command; you will need to use \cs{bidishadowtext} command. You must always load \xpackage{bidishadowtext} package before \xpackage{bidi} package (or any other packages that uses \xpackage{bidi} package internally like \xpackage{xepersian} package).

\end{document}