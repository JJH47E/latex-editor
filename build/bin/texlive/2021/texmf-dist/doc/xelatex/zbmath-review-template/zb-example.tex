% Copyright 2021 by FIZ-Karlsruhe
%
% This file is part of the ctan package zbmath-review-template
% and may be distributed and/or modified under the
% conditions of the CC-BY-SA 4.0 license:
%
% https://creativecommons.org/licenses/by-sa/4.0/
%
% This file contains an example for the usage of the class zbMATH.
% The text was first published at https://zbmath.org/1203.15025.
%


\documentclass{zbMATH}

\title{Random matrices: universality of ESDs and the circular law} % title of the reviewed article
\author{Tao, Terence; Vu, Van; Krishnapur, Manjunath} % authors of the reviewed article

\begin{document}
% generate the title
\maketitle

% display copyright in footer
\makefooter

% text of the review
This paper is concerned with the convergence of empirical spectral distributions (ESD) of random matrices, both in the sense of convergence in probability and in the almost sure sense.


Recall that for each \(n\in {\mathbb N}\), if \(F_n\) is a random variable taking values in some Hausdorff topological space \(X\) and let \(F\in X\), then we say that
{\parindent5mm
\begin{itemize}
    \item[1)] \(F_n\) converges in probability to \(F\) if \(\lim_{n\to \infty} P(F_n\in V) = 1\) for every neighborhood \(V\) of \(F\).
    \item[2)] \(F_n\) converges almost surely to \(F\) if \(P(\lim _{n\to \infty} F_n=F) = 1\).
\end{itemize}}
A fundamental problem in the theory of random matrices is to determine the limiting distribution of the ESD of a random matrix ensemble (either in probability or in the almost sure sense) as the size of the random matrix tends to infinity. The universality phenomenon (a well-known intuition) asserts that the limiting distribution should not depend on the particular distribution of the entries. In the 1950s Wigner proved that the ESD of an \(n\times n\) Hermitian matrix with (upper diagonal) entries being i.i.d. Gaussian random variables converge to the semicircle law \(F\) with density 
\[
\rho(x)=\begin{cases} \frac 1{2\pi} \sqrt {4-x^2}, &|x|\leq 2\\ 0 &|x|>0 \end{cases}
\]
Wigner's result was then extended in different forms and one of them is:

Theorem 1.3. Let \(A_n\) be the \(n \times n\) Hermitian random matrix whose upper diagonal entries are i.i.d. complex random variables with mean \(0\) and variance \(1\). The ESD of \(\frac 1{\sqrt n}A_n\) then converges (in both convergence modes) to the semicircle distribution.

 The well-known circular law conjecture asserts that Theorem 1.3 deals with the non-Hermitian matrices: 

Conjecture 1.4. (Circular law conjecture) Let \(A_n\) be the \(n \times n\) random matrix whose entries are i.i.d. complex random variables with mean \(0\) and variance \(1\). The ESD of \(\frac 1{\sqrt n}A_n\) then converges (in both convergence modes) to the uniform distribution on the unit disk.

Works towards the circular law conjecture have been done by many authors including Mehta, Edelman, Girko, Bai, Götze and Tikhomirov, Pan and Zhou and the present authors (see the references) under some assumptions on the distribution. For example, Mehta (complex Gaussian distribution), Edelman (real Gaussian distribution).

The authors prove the above conjecture (Theorem 1.10). They first establish some universality principles (Theorem 1.5 and Corollary 1.8). It turns out the limit distribution of the ESD of a random matrix ensemble \(A_n\) depends only on the mean and variance of its entries, under a mild size condition on the mean \({\mathbf E}A_n\) and under the assumption that the matrix \(A_n-{\mathbf E}A_n\) has i.i.d. entries. Because of Corollary 1.8, the proof is reduced to the case where the entries are Gaussian (Lindeberg trick). Thus from Mehta's result the circular law conjecture follows.

As a related result, they show how laws for this ESD follow from laws for the singular value distribution of \(\frac 1{\sqrt n}A_n-zI\) for complex \(z\) (see Theorem 1.15). Other interesting results are obtained.

Appendix A reviews some matrix inequalities and Appendix B is authored by M. Krishnapur.

Some typos and comments on Appendix A: Lemma A.2 ``generalized eigenvalues'' should be ``eigenvalues''. In the proof, ``orthogonal \(Q\)'' should be ``unitary \(Q\)'' (indeed the proof follows directly from Schur's triangularization theorem). In the proof of Lemma A.4, \(X_j\), \(W_j\) should be \(X_j^*\), \(W_j^*\) respectively. Indeed the majorization between the vectors \((\sigma_1(A)^{-1},\dots , \sigma_{n'}(A)^{-2})\) and \((\text{dist} (X_1,W_1)^{-2}, \dots, \text{dist} (X_{n'},W_{n'})^{-2})\) holds by applying Schur's result on the positive definite matrix \((AA^*)^{-1}\).

% name of the reviewer
\reviewer{Tin Yau Tam (Auburn)}

%Optional
%If you want you can add MSC Classes and keywords.
\msc{15B52 60F15 60B20 15A18} % msc classes separated by space

\keywords{circular law; eigenvalues; random matrices; universality; empirical spectral distributions; convergence in probability; limiting distribution; semicircle law; Hermitian random matrix} % keywords separated by semicolon ';'


\end{document}