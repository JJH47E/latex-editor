\documentclass{article}
\usepackage[xe]{shipunov2}
\usepackage{multicol}

\setmainfont[Mapping=tex-text]{CMU Serif}

\renewcommand*{\indexname}{Указатель русских названий родов}

\makeindex

\begin{document}

\subsection{Семейство Лютиковые --- Ranunculaceae\label{Ran}}

Травы с очередными, часто расчлененными листьями без
прилистников; цветки обычно обоеполые, околоцветник простой,
лепестковидный или расчленен на чашечку и венчик, тычинки и
пестики в неопределенном числе, плод --- многолистовка или
многоорешек.

\Z1. Листочки околоцветника белые или бледно-зеленые, не
опадающие. Листья пальчаторассеченные, кожистые, зимующие. Плод
--- многолистовка \TT Морозник\index{Морозник } кавказский (\NN Helleborus
caucasicus A. Br.)

\AN Признаки иные \T 2.

\Z2. Околоцветник простой, венчиковидный \T 3.

\AN Околоцветник двойной \T 4.

\newpage

\Z3. Листочков околоцветника 5, желтых \TT
Ветреница\index{Ветреница} лютичная (\NN Anemone ranunculoides
L.)

\AN Листочков околоцветника больше 5, голубых или синих \TT
Вет\-ре\-ни\-ца нежная (\NN Anemone blanda Schott \& Kotschy)

\ZZ4(2). Листья цельные, чашелистиков 3. Растение голое \TT
Чистяк\index{Чистяк} калужницелистный (\NN Ficaria calthifolia
Reichenb.)

\AN Листья рассеченные. Чашелистиков 5 (иногда опадают после
того, как цветок раскрылся) \T 5.

\Z5. Растение густо опушено мягкими, шелковистыми волосками.
Цветоложе голое \TT Лютик\index{Лютик} крупноцветковый (\NN
Ranunculus grandiflorus C.A.~Mey.)

\AN Растение слабоопушенное. Цветоложе покрыто волосками \TT
Лютик каппадокийский (\NN Ranunculus cappadocicus Willd.)

\printindex

\end{document}
