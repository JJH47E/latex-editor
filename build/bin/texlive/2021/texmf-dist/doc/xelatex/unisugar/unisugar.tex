\documentclass{ltxdoc} % Process with xelatex -shell-esc
\usepackage{unisugar}
\usepackage[colorlinks=true]{hyperref}
\usepackage{bashful}
\usepackage{varioref}

⌘let⌘use␣package=⌘usepackage
⌘use␣package{xspace}
⌘let⌘new␣command=⌘newcommand
⌘new␣command⌘unisugar{⌘texttt{unisugar}⌘xspace}

⌘use␣package{graphicx}
⌘use␣package{metalogo}
⌘use␣package{hyperref}
⌘use␣package{polyglossia}
  ⌘setmainlanguage{english}
  ⌘setotherlanguage{hebrew}
  ⌘newfontfamily⌘hebrewfont[Script=Hebrew]{David CLM}
  ⌘newfontfamily⌘codefont{DejaVu Sans Mono}

⌘let⌘new␣environment=⌘newenvironment

⌘new␣command⌘command␣key{{⌘codefont⌘⌘}}
⌘new␣command⌘return␣key{{⌘codefont⌘⏎}⌘xspace}

⌘new␣command⌘Unicode{⌘textsc{Unicode}⌘xspace}
⌘new␣environment{code}{%
  ⌘begingroup
  ⌘quote
  ⌘small
  ⌘codefont
  ⌘tt
}{%
  ⌘endquote
  ⌘endgroup
}
⌘new␣command⌘me{unisugar}
⌘title{The ⌘textsf{⌘me} Package\thanks{
  Copyright ⌘copyright{} 2011 by Yossi Gil  
    ⌘url{mailto:yogi@cs.technion.ac.il}.
  This work may be distributed and/or modified under the conditions of the
    ⌘emph{⌘LaTeX{} Project Public License} (LPPL), either version 1.3 of this 
    license or (at your option) any later version.  
The latest version of this license is in 
  ⌘url{http://www.latex-project.org/lppl.txt} and version 1.3 or later 
  is part of all distributions of ⌘LaTeX{} version 2005/12/01 or later.
This work has the LPPL maintenance status `maintained'.
The Current Maintainer of this work is Yossi Gil. 
This work consists of the files ⌘texttt{⌘me.tex} and ⌘texttt{⌘me.sty} 
  and the derived file
  ⌘texttt{⌘me.pdf}
}}

⌘author{Yossi Gil⌘thanks{⌘url{mailto:yogi@cs.Technion.ac.IL}}⏎
   ⌘normalsize Department of Computer Science⏎
   ⌘normalsize The Technion---Israel Institute of Technology⏎
   ⌘normalsize Technion City, Haifa 32000, Israel
}

⌘date{{⌘makeatletter
     ⌘date@unisugar\thanks{
        This document describes ⌘unisugar ⌘version@unisugar.}}}
        
        

⌘begin{document}
⌘bash[stdoutFile=README]
cat << EOF
The unisugar package 0.92
=========================

This package requires a TeX-alike system that uses native Unicode input system;
current examples are XeTeX and LuaTeX. The package provides syntactic sugar for 
selected LaTeX commands, allowing these to be replaced with their Unicode- 
character counterpart, e.g., a Unicode bullet can be used instead of a \item,
and a pilcrow can be used instead of \paragraph.

The intent is to minimize the use of English left-to-right characters in 
documents whose main language is written right-to-left, since mixing characters
of different directionality confuses both text editors and human beings.

Using this package, you may find yourself typing a bit less, provided you can 
configure your text editor or keyboard driver to generate the handful of Unicode
characters defined by this package. More importantly, the package is useful in 
defining macros whose name is composed of right-to-left characters and in 
minimizing mixed directionality text in right-to-left documents.

This package may be distributed and/or modified under the LaTeX Project Public 
License (LPPL), version 1.3 or higher (your choice). The latest version of this 
license can be found at: http://www.latex-project.org/lppl.txt

This work is author-maintained(as per LPPL maintenance status) by Yossi Gil
<yogi@cs.Technion.ac.IL>
EOF
\END

⌘maketitle
⌘begin{abstract}
This package provides syntactic sugar
  for ⌘LaTeX{} commands, using selected 
    ⌘href{http://www.unicode.org/standard/standard.html}⌘Unicode 
    characters:
Selected ⌘Unicode characters can be used as shorthand for certain 
  ⌘LaTeX{} commands.
The package also makes it possible to use the familiar 
  command key symbol,~⌘command␣key{} as a prefix of ⌘TeX{}'s 
  macros (the backlash character,~⌘textbackslash, can
  still be used).
And it allows the use of visual space,~⌘␣, 
  within the names of macros, thus making it easier to 
  write macros whose names are composed of more than one word.
  
In this document I describe these syntactical extensions,
  and explain why they make it easier to compose
  right-to-left documents.
⌘end{abstract}

{⌘small⌘tableofcontents}

⌘parindent 1.5ex
⌘parskip 0.5em

§ Introduction
You do not really ⌘emph{need} this package. 
As its name implies, all it offers is what people call 
  ``syntactic sugar''---the ability to use a selected
    number of ⌘textsc{⌘Unicode} characters instead of and within 
    the most common ⌘LaTeX{} commands.
    
Some may appreciate the fact that using this package, 
  the text you type---the input to ⌘LaTeX{}---will look 
  a bit more neat
  and infinitesimally closer to the output.
Others will argue against it, mentioning that the input document 
  will be less portable,
  and less ``⌘LaTeX-like'' (whatever this term means).
 
Still, using this package, 
  you will find yourself typing a bit less, 
  provided you can configure your text editor or 
  keyboard driver to generate
  the handful of ⌘Unicode characters defined by this package.
  
More importantly, if your input files include right-to-left text,
  you are likely to find this package indispensable.
If your document is indeed going to include right-to-left text, please
  pay special attention to Section~☝{Section:rtl:divisions},
    which describes how ⌘unisugar should help you with
    your document divisioning directives, and 
    to Section~☝{Section:rtl:commands},
    which explains how ⌘unisugar makes it easier
    to intermix ⌘LaTeX{} commands with your text.
If however you are not likely to include right-to-left text in 
  your documents,
  you do not need to read these sections.

    
§ User Guide
Simply apply a ⌘verb+\usepackage+ directive to use
  this package.
⌘begin{code}
⌘verb+\usepackage{unisugar}+
⌘end{code}

There are no package options at this time.

You would then have to use a ⌘Unicode based version of ⌘LaTeX{}, 
  that is ⌘XeLaTeX{} or ⌘LuaLaTeX{}
  to process your input file, e.g., for processing the document 
    you are now reading,   
  I typed at my shell command prompt
⌘begin{code}
xelatex unisugar.tex
⌘end{code}

§§ Document Divisioning Commands

Two of ⌘Unicode's typographical characters,~⌘texttt{⌘¶}
   (code point B6, the pilcrow sign) and~⌘texttt{⌘§}
   (code point A7, the section sign) are employed
   in ⌘unisugar to make shorthands 
  for ⌘LaTeX{} traditional document divisionining commands:
  ⌘verb+\section+, 
  ⌘verb+\subsection+, ⌘verb+\subsubsection+, and the lesser units,
  ⌘verb+\paragraph+, 
  and ⌘verb+\subparagraph+ commands.
  
Thus, instead of writing 
⌘begin{code}
⌘textbackslash section⌘{User Guide⌘}
⌘end{code}
at the beginning of this section, I typed just 
⌘begin{code}
⌘§ User Guide
⌘end{code}
Similarly, instead of
⌘begin{code}
⌘textbackslash subsection⌘{Document Divisioning Commands⌘}
⌘end{code}
I typed 
⌘begin{code}
⌘§⌘§ Document Divisioning Commands
⌘end{code} 
to generate the header of this subsection.

Observe that I did not need to type nor an opening curly
  bracket,~⌘verb+{+, neither a closing curly bracket,~⌘verb+}+.
The division's title extends from the~⌘texttt{⌘§}
  character (in case of a section), or the~⌘texttt{⌘§⌘§} 
  characters pair (in case of a subsection)
    until the end of the line.
  
  
Table ☝{Table:divisions} summarizes the shorthands for the document 
  divisioning commands.
The original versions are always there, in case you need
   to use the starred version of the divisioning command, 
   or pass an optional argument to it.
    
⌘begin{table}[!Hht]
⌘small
⌘begin{center}
⌘begin{tabular}{lll}
Division            & ⌘LaTeX{}              & ⌘unisugar ⏎
\hline
\hline
part                & ⌘verb+\part+          &  --- ⏎
chapter             & ⌘verb+\chapter+       &  --- ⏎
section             & ⌘verb+\section+       & ⌘texttt{\§} ⏎
sub-section         & ⌘verb+\section+       & ⌘texttt{\§\§} ⏎
sub-sub-section     & ⌘verb+\subsection+    & ⌘texttt{\§\§\§} ⏎
paragraph           & ⌘verb+\paragraph+     & ⌘texttt{\¶} ⏎
sub-paragraph       & ⌘verb+\subparagraph+  & ⌘texttt{\¶\¶} ⏎
\hline
⌘end{tabular}
\caption{Shorthands for divisioning commands}
⌖{Table:divisions}
⌘end{center}
⌘end{table}

§§ Right-to-left Text Editing
⌖{Section:rtl:divisions}
There is a great advantage of using these sugared replacements 
  in composing left-to-right documents.
Think of a a Hebrew document including a section 
  named ⌘texthebrew{מבוא} (Which means ``Introduction'' in Hebrew).
Then, with plain ⌘XeLaTeX{}, the document would start with 
  a directive
⌘begin{code}
⌘textbackslash{}section⌘{⌘fontspec{Courier New}
מבוא%
⌘rm⌘tt⌘}
⌘end{code}
Even if your text editor can manage well mixed directionality text,
  you will find editing the above line a bit  confusing.
The reason is that the character following the opening 
  curly brackets is 
  not~⌘texthebrew{א} but rather ⌘texthebrew{מ}.
  
As the cursor moves forward beginning at the first character in
  the line, it hits the opening curly brackets, and then you may
  expect it to proceed to the adjacent 
  letter ⌘texthebrew{א}.
This so called ⌘emph{visual} flow is 
  in incorrect. 
The more correct ⌘emph{logical} flow prescribes 
  that the cursor should instead ``jump'' to the 
  to the letter ⌘texthebrew{מ}, which is the first letter of 
  the word ⌘texthebrew{מבוא}.

Some editors adhere to the visual flow, others, more modern
  editors, to the logical flow.
But experience shows that both are quite confusing.
  
Our sugared version of the divisioning directives 
  offer a more sane alternative.
You can write instead
⌘begin{hebrew}
⌘begin{code}
\§ ⌘fontspec{Courier New} מבוא
⌘end{code} 
⌘end{hebrew}

To understand why the latter is so much better than the former,
  you need to know a bit about the manner in which ⌘Unicode deals with 
  text directionality.
Broadly speaking, characters come
  in three major varieties:
⌘begin{enumerate}
• Left-to-right directed characters, including e.g., Latin characters.
• Right-to-left directed characters, including e.g., 
      characters of the Hebrew
  alphabet.
• Undirected characters, including the digits 0-9, punctuation characters, 
    and characters such as ⌘§, ⌘¶, ⌘␣, and ⌘command␣key{}
    which are not part of specific writing script.
⌘end{enumerate}
⌘Unicode assigns a direction to each line
  according to the first strongly directed character of that line, and then
  proceeds to assigning directionality to subsequences of characters 
    occurring in that line.⌘footnote{%
The full algorithm is fairly complicated, taking into account
  ``weak'' directionality of some of the undirected characters,
    explicit directionality markers and nested directionality levels;
    details can be found here ⌘url{http://unicode.org/reports/tr9}.}
    
Most text editors follow ⌘Unicode's algorithm.
The line 
⌘begin{code}
⌘textbackslash{}section\{\fontspec{Courier New} מבוא%
⌘rm⌘tt⌘}
⌘end{code}
will thus be classified as left-to-right, with
  the ⌘texthebrew{מבוא} portion classified right-to-left, leading
  to the visual vs.\ logical confusion and to irritating 
  cursor jumps.
  
Further, the fact that the entire line is left-to-right will
   lead text editors such as ⌘texttt{gedit} to place
   the first character of that line at the left most position
   in the window.
This might be confusing even further since the section body will, 
  most likely,
  be classified as right-to-left, and hence will be laid out
  on the screen with the first character at the right-most 
  position.
  
  
⌘begin{figure}[!htbp]
⌘begin{center}
	⌘includegraphics[scale=0.3]{traditional.png}
⌘end{center}
⌘caption{Using traditional sectioning directive with right-to-left text}
⌖{Figure:traditional}
⌘end{figure}
  
Figure~☝{Figure:traditional} depicting the use of ⌘texttt{gedit}
  to compose a Hebrew ⌘LaTeX{} document
  may help in visualizing the difficulty.
Line~18 containing the section title is laid out left-to-right,
  despite the fact that the title is written Hebrew, in
  visual discrepancy with lines~19 and on, containing the section body,
  which are laid out right-to-left.
  
In contrast, the sugared version of our document divisioning command
⌘begin{hebrew}
⌘begin{code}
\fontspec{Courier New}
\§ ⌘fontspec{Courier New} מבוא
⌘end{code} 
⌘end{hebrew}
has no left-to-right characters, and hence will be classified 
  as being entirely right-to-left:
The cursor will not jump as it moves 
  across that line.
  
  
Compare Figure~☝{Figure:traditional} with Figure~☝{Figure:sugar}
  in which the sugared version of the ⌘verb+\section+ directive is used.
We see that in  Figure~☝{Figure:sugar} ⌘emph{all} lines
  are laid out right-to-left.
  
  
⌘begin{figure}[!hbtp]
⌘begin{center}
	⌘includegraphics[scale=0.3]{sugar.png}
⌘end{center}
⌘caption{Using sugared sectioning directive with right-to-left text}
⌖{Figure:sugar}
⌘end{figure}

§§ Other Useful ⌘Unicode Characters
The above sugar replacement for divisioning commands
  does not scale well.
With the exception of mathematical symbols, it is difficult 
  to find reasonable substitutes for ⌘Unicode replacements 
  for the majority of ⌘LaTeX{} commands.
  
Still, four more ⌘Unicode characters are used by ⌘unisugar
  as aliases for common ⌘LaTeX{} commands:
⌘begin{enumerate}
• Code point 2022, the bullet, rendered as ⌘•, is
  yet another name for the ⌘verb+\item+ command.
To obtain this item, I typed
⌘begin{code}
\• Code point 2022, the bullet, rendered as ⌘textbackslash\•,⏎ 
is yet another name for the ⌘verb-\verb+\item+- command.⏎
  To obtain this item, I typed:⏎
⌘end{code}
• Code point 23CE, the return symbol, rendered 
  as~⌘return␣key is an alias for ⌘verb+\\+.
  The ⌘verb+\author+ directive of this manuscript was
  typed out as 
⌘begin{code}
⌘verb+\author{+\\
⌘verb+  Yossi Gil+\\
⌘verb+    ⌘thanks{\url{mailto:yogi@cs.technion.ac.il}}+⌘return␣key⏎
⌘verb+  ⌘normalsize Department of Computer Science+⌘return␣key⏎
⌘verb+  ⌘normalsize The Technion---Israel Institute+\\
⌘verb+                   of Technology+⌘return␣key⏎
⌘verb+  ⌘normalsize Technion City, Haifa 32000, Israel+\\
⌘verb+}+
⌘end{code}  
• Code point 2316 (position indicator), rendered as 
      {\fontspec{DejaVu Sans Mono}\⌖}
      is an alias for ⌘LaTeX's ⌘verb+\label+ command.
  To generate a label for Table ☝{Table:divisions},
  I wrote:
⌘begin{code}
{\fontspec{Courier New}\⌖}⌘verb+{Table:divisions}+
⌘end{code} 
• Code point 261D  (white up pointing index) rendered as 
    {\fontspec{DejaVu Sans Mono}\☝}
    is an alias ⌘LaTeX{}'s ⌘verb+\ref+ command.
    To reference Table ☝{Table:divisions}, I wrote
⌘begin{code}
   To reference Table {\fontspec{DejaVu Sans Mono}\☝}\{Table:divisions\}, 
   I wrote
⌘end{code} 

• Code point 2026 (horizontal ellipsis), rendered as \verb+…+, serves as a sugar nickname for 
  \verb+\ldots+, the lower ellipsis command.
  
⌘end{enumerate}
 

§§ Extended Syntax for Commands
It is futile to try to introduce an pictorial, easy to remember, 
  symbol for each of the
  ⌘LaTeX{} commands in ordinary use, or even for a substantial
  portion of these.
As large as it is, the ⌘Unicode character repertoire
  simply does not include icons that associated visually with notions
  such as ⌘verb-\verb+-, ⌘verb+\begin{description}+, etc.
And even if it was, such a large set would be difficult to 
  memorize.
Worse, methods for producing so many characters would be cumbersome.

Thus, you would have to type ⌘LaTeX{} commands every so often.
This package offers a slightly better syntax for writing these.
  

First, ⌘Unicode's code point 2318, rendered as ⌘command␣key, 
  is used in many computing systems
   to denote the command key. 
With ⌘unisugar,the~⌘command␣key{}
  character can be used as a control sequence
  prefix,
So, instead of writing at the beginning of this document
⌘begin{code}
⌘verb+\begin{document}+⏎
⌘verb+\maketitle+⏎
⌘verb+\begin{abstract}+⏎
⌘verb+This package provides syntactic sugar+⌘ldots
⌘end{code}
I wrote
⌘begin{code}
⌘command␣key⌘verb+begin{document}+⏎
⌘command␣key⌘verb+maketitle+⏎
⌘command␣key⌘verb+begin{abstract}+⏎
⌘verb+This package provides syntactic sugar+⌘ldots
⌘end{code}

Second, ⌘unisugar, extends to the usual set of~52 letters (a--z and A--Z)
  ⌘Unicode character 2423,
  the open box, which looks like like visible space in its 
  rendering~⌘␣.
This character can thus participate in control sequences. 
The intention is that it will serve for separating words
  in the case that your control sequence is composed of 
  several words.
  
 
The names of a large number of  ⌘LaTeX{} commands are made 
  from two words.
There are even a dozen or so control sequences whose
  name consists of three words, e.g., ⌘verb+\enlargethispage+
  and ⌘verb+\addcontentsline+.
Although ⌘unisugar does not provide aliases for any of
  these multi-word commands, you can do so yourself.
For example, at the preamble of this document, right after
  ⌘verb+\usepackage{unisugar}+, 
I wrote
⌘begin{code}
⌘command␣key{}let⌘command␣key{}use␣package=⌘command␣key{}usepackage⏎
⌘command␣key{}use␣package⌘{xspace⌘}⏎
⌘command␣key{}let⌘command␣key{}new␣command=⌘command␣key{}newcommand⏎
⌘command␣key{}new␣command⌘command␣key{}%
  unisugar⌘{⌘command␣key{}texttt⌘{unisugar⌘}⌘command␣key{}xspace⌘}⏎
⌘end{code}

§§ Intermixing Commands with Right-to-Left Text
⌖{Section:rtl:commands}
You may not appreciate so much the advantage of 
  typing ⌘Unicode's~⌘command␣key{}
  instead of plain ASCII's~⌘texttt{⌘textbackslash}.
Granted, on most keyboards, typing~⌘texttt{⌘textbackslash}
  would be easier.
  
However, the nice property of ⌘command␣key{} is that it directionally 
  neutral.
You would have to think about a sentence 
  involving at least one ⌘LaTeX⌘ control sequence
  and/or a slash character to understand what I mean.
You would not have to think so hard,
  since, this last sentence, namely,
⌘begin{quote}
``⌘emph{You would have to think about a sentence 
  involving at least one ⌘LaTeX{} control sequence
  and/or a slash character to understand what I mean.}''
⌘end{quote} 
is a perfect example, since
  it is a sentence which involves a ⌘LaTeX{} control sequence,
  since the ⌘LaTeX⌘ logo is printed out using
  the ``⌘verb+\LaTeX\+" control sequence.
Further, this sentence includes the slash character.

Typing this sentence in English is fairly straightforward. 
⌘begin{code}
You would have to think about a sentence 
  involving at least one ⌘verb+\LaTeX\ +control sequence
  and/or a slash character to understand what I mean.
⌘end{code}

Let me translate this sentence into Hebrew for you. 

⌘begin{hebrew}
יהיה עליך לחשוב על משפט המכיל לפחות פקודת בקרה אחת 
של ⌘LaTeX⌘ ו/או לוכסן בכדי להבין למה אני מתכוון.
⌘end{hebrew}


What input does ⌘LaTeX{} require in order to produce the above? 
Using backslashes and traditional ⌘LaTeX{} notation I would have written
⌘begin{hebrew}
⌘begin{code}⌘fontspec{Courier New}
יהיה עליך לחשוב על משפט המכיל לפחות פקודת בקרה 
אחת ⏎ 
של 
⌘verb+\LaTeX\+ 
ו/או לוכסן בכדי להבין למה אני מתכוון.
⌘end{code}
⌘end{hebrew}

Figure~☝{Figure:gedit-mixed-traditional} shows what this might look on
  an actual text editor.
This may not seem too complicated, but a closer look 
  would reveal that the production and inspection of this ⌘LaTeX{} 
  is hindered by two or three annoying hidden issues.


⌘begin{figure}[!htbp]
⌘begin{center}
%Produced from:
%
%
%Let me translate this sentence into Hebrew for you.
%
%⌘begin{hebrew}
%יהיה עליך לחשוב על משפט המכיל לפחות פקודת בקרה אחת 
%של ⌘LaTeX\ ו/או לוכסן בכדי להבין למה אני מתכוון.
%⌘end{hebrew}
%
%
%
%
⌘includegraphics[scale=0.3]{gedit-mixed-traditional.png}
⌘end{center}
⌘caption{A Hebrew sentence containing a control sequence (without ⌘unisugar).}
⌖{Figure:gedit-mixed-traditional}
⌘end{figure}


First, the two backslash characters in the figure are not really 
  ⌘emph{back}slashes. 
The reason is that Hebrew is written right-to-left, and the ⌘textbackslash{} 
  character leans in the text direction, and should therefore be considered 
  a ⌘emph{forward} slash in this context.
  
The second annoyance is the distinction between the 
  two occurrences of this character:
  the first is at the beginning of the control sequence ⌘verb+\LaTeX+,
  denoting that the control sequence starts at that point.
The second occurrence is at the end of the same control sequence,
  with the purpose of
  escaping the space that follows.
This escape is required to prevent the control sequence from consuming
  the spaces that follow.
  
The distinction between the first and which is the last ``backslash''
  is clear in the case that the enclosing sentence is 
  written left-to-right.
 But, both humans and text processing devices may be confused
  when the enclosing sentence is right-to-left.

{⌘small⌘textsl
(There is yet a third difficulty in the above sentence which I will 
  not address here.
  The English forward slash character is used in Hebrew for separating the day, 
  month and 
  year components of a date. 
  This is natural, since dates involve digits, and these are written,
    even in Hebrew, left to right.
  Most Hebrew authors extrapolate this convention to the separation of
  Hebrew words by a ``forward'' (left-leaning) slash as in the phrase
  ⌘texthebrew{⌘fontspec{Courier New}%
  ו/או
  }%
  in the above. 
  Other authors would right this phrase with the slash leaning in
    the text direction, i.e., 
     ⌘texthebrew{⌘fontspec{Courier New}%
  ו⌘textbackslash{}או
  }%
  )}
  
The fact that the ⌘command␣key{} character does not 
  lean neither left nor right,
  takes care of the first annoyance.
  
The remedy for the second is simpler---use a pair of curly brackets
  to mark the end of the control sequence.
  
Thus, with ⌘unisugar, I would write:
⌘begin{hebrew}
⌘begin{code}
⌘fontspec{Courier New}
יהיה עליך לחשוב על משפט המכיל לפחות פקודת בקרה 
אחת ⏎ של 
⌘setLTR⌘verb+{}+LaTeX⌘command␣key{}⌘setRTL
ו/או לוכסן בכדי להבין למה אני מתכוון.
⌘end{code}
⌘end{hebrew}

Figure~☝{Figure:gedit-mixed-traditional} shows this sugared input as it
  appears in the gedit editor.
We can see that the entire control sequence phrase 
  is written left-to-right, with the escape character
  first, and the pair of curly brackets last.
  

⌘begin{figure}[!htbp]
⌘begin{center}
%
%Produced from:
%
%Let me translate this sentence into Hebrew for you.
%
%⌘begin{hebrew}
%יהיה עליך לחשוב על משפט המכיל לפחות פקודת בקרה אחת 
%של {}LaTeX⌘ ו/או לוכסן בכדי להבין למה אני מתכוון.
%⌘end{hebrew}
%
%
%
%
%
%
⌘includegraphics[scale=0.3]{gedit-mixed-sugar.png}
⌘end{center}
⌘caption{A Hebrew sentence containing a control sequence (with ⌘unisugar).}
⌖{Figure:gedit-mixed-sugar}
⌘end{figure}

Evidently, mixed directionality text is slightly easier and clearer.  
But we can do even better, if we allow Hebrew characters
in control sequences. 
This is carried out by (a yet to be published) another package,
  named ⌘texttt{sukkar}, which, relying on ⌘unisugar does precisely this
  and more.
Package ⌘texttt{sukkar} also translates many common ⌘LaTeX{}
  commands to Hebrew, and since juxtaposition of words looks
  weird in Hebrew, it uses the~⌘␣{} Unicode character 
  to separate words.
With ⌘texttt{sukkar} 
  one can write, e.g., 
⌘begin{hebrew}⌘codefont⌘command␣key⌘fontspec{Courier New}
עשה⌘␣{}כותרת
⌘end{hebrew}
  instead of ⌘verb+\make_title+.
  
  
§ History
⌘begin{description}
•[Version 0.9] Initial release.
•[Version 0.91] Placed under LPPL.
•[Version 0.92] Added U+2026 HORIZONTAL ELLIPSIS, as nickname for ⌘verb+\ldots+.
⌘end{description}
    

§ Acknowledgements
Will Robertson advised gave the advise of 
	using ⌘XeLaTeX{} and ⌘texttt{polyglossia}
  to circumvent a bug of ⌘texttt{utf8x}.
I pay tribute to Bruno Le Floch and Martin Scharrer who together 
  devised the mechanism that made it possible to 
  define a command which takes the rest of the line as argument.
Martin Scharrer and Will Robertson encouraged me to work on this package.
Vafa Khalighi  
	devotion to bidirectional text processing with ⌘LaTeX{}
  was truly inspirational.
  




⌘end{document}
