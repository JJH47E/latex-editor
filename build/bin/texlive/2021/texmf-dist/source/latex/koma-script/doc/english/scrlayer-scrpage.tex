% ======================================================================
% scrlayer-scrpage.tex
% Copyright (c) Markus Kohm, 2013-2019
%
% This file is part of the LaTeX2e KOMA-Script bundle.
%
% This work may be distributed and/or modified under the conditions of
% the LaTeX Project Public License, version 1.3c of the license.
% The latest version of this license is in
%   http://www.latex-project.org/lppl.txt
% and version 1.3c or later is part of all distributions of LaTeX 
% version 2005/12/01 or later and of this work.
%
% This work has the LPPL maintenance status "author-maintained".
%
% The Current Maintainer and author of this work is Markus Kohm.
%
% This work consists of all files listed in manifest.txt.
% ----------------------------------------------------------------------
% scrlayer-scrpage.tex
% Copyright (c) Markus Kohm, 2013-2019
%
% Dieses Werk darf nach den Bedingungen der LaTeX Project Public Lizenz,
% Version 1.3c, verteilt und/oder veraendert werden.
% Die neuste Version dieser Lizenz ist
%   http://www.latex-project.org/lppl.txt
% und Version 1.3c ist Teil aller Verteilungen von LaTeX
% Version 2005/12/01 oder spaeter und dieses Werks.
%
% Dieses Werk hat den LPPL-Verwaltungs-Status "author-maintained"
% (allein durch den Autor verwaltet).
%
% Der Aktuelle Verwalter und Autor dieses Werkes ist Markus Kohm.
% 
% Dieses Werk besteht aus den in manifest.txt aufgefuehrten Dateien.
% ======================================================================
%
% Chapter about scrlayer-scrpage of the KOMA-Script guide
%
% ----------------------------------------------------------------------
%
% Kapitel ueber scrlayer-scrpage in der KOMA-Script-Anleitung
%
% ============================================================================

\KOMAProvidesFile{scrlayer-scrpage.tex}%
                 [$Date: 2019-12-06 11:44:37 +0100 (Fri, 06 Dec 2019) $
                  KOMA-Script guide (chapter: scrlayer-scrpage)]
\translator{Markus Kohm\and Jana Schubert\and Jens H\"uhne\and Karl Hagen}

% Date version of the translated file: 2019-11-29

\chapter[{Headers and Footers with \Package{scrlayer-scrpage}}]
  {Headers\ChangedAt{v3.12}{\Package{scrlayer-scrpage}} and
  Footers with \Package{scrlayer-scrpage}}
\labelbase{scrlayer-scrpage}
%
\BeginIndexGroup
\BeginIndex{Package}{scrlayer-scrpage}%
\begin{Explain}
  Until version 3.11b of \KOMAScript, the \Package{scrpage2}%
  \IndexPackage[indexmain]{scrpage2}\important{\Package{scrpage2}} package was
  the recommended way to customise headers and footers beyond the options
  provided by the \PageStyle{headings}, \PageStyle{myheadings},
  \PageStyle{plain}, and \PageStyle{empty} page styles of the \KOMAScript{}
  classes.
  \iffalse%
  The still older \Package{scrpage}\IndexPackage{scrpage} package was marked
  obsolete in 2001 and removed from the regular \KOMAScript{} distribution in
  2013.\par
  \fi%
  Since 2013, the \hyperref[cha:scrlayer]{\Package{scrlayer}}%
  \important{\hyperref[cha:scrlayer]{\Package{scrlayer}}}%
  \IndexPackage{scrlayer} package has been included as a basic module of
  \KOMAScript. This package provides a layer model and a new page-style model
  based upon it. However, the package's interface is almost too flexible and
  consequently not easy for the average user to comprehend. For more
  information about this interface, see \autoref{cha:scrlayer} in
  \autoref{part:forExperts}. However, a few of the options that are actually
  part of \Package{scrlayer}, and which are therefore taken up again in that
  chapter, are also documented here because they are required to use
  \Package{scrlayer-scrpage}.

  Many users are already familiar with the commands from \Package{scrpage2}.
  For this reason, \Package{scrlayer-scrpage} provides a method for
  manipulating headers and footers which is based on \Package{scrlayer}, is
  largely compatible with \Package{scrpage2}, and at the same time greatly
  expands the user interface. If you are already familiar with
  \Package{scrpage2} and refrain from direct calls to its internal commands,
  you can usually use \Package{scrlayer-scrpage} as a drop-in replacement.
  This also applies to most examples using \Package{scrpage2} found in
  \LaTeX{} books or on the Internet.%
  \iffalse%
    \iffree{}{\par With the release of this book, \Package{scrlayer-scrpage}
      for \KOMAScript{} is recommended as the tool of choice when it comes to
      changing the predefined page styles. Using the obsolete package
      \Package{scrpage2}\IndexPackage[indexmain]{scrpage2}%
      \important{\Package{scrpage2}} is now deprecated. Therefore, the
      commands for this outdated package are no longer part of this book. If
      you encounter older documents that still use \Package{scrpage2},
      consider switching to \Package{scrlayer-scrpage}. Notwithstanding, this
      chapter does contain some hints for using \Package{scrpage2}.}%
  \fi
\end{Explain}

In addition to \Package{scrlayer-scrpage}\iffree{ or \Package{scrpage2}}{},
you could also use \Package{fancyhdr}\IndexPackage{fancyhdr} (see
\cite{package:fancyhdr}) to configure the headers and footers of pages.
However, this package has no support for several \KOMAScript{} features,
for example the element scheme (see \DescRef{\LabelBase.cmd.setkomafont},
\DescRef{\LabelBase.cmd.addtokomafont}, and
\DescRef{\LabelBase.cmd.usekomafont} in \autoref{sec:maincls.textmarkup},
\DescPageRef{maincls.cmd.setkomafont}) or the configurable numbering format
for dynamic headers (see the \DescRef{maincls.option.numbers} option and,
for example, \DescRef{\LabelBase.cmd.chaptermarkformat} in
\autoref{sec:maincls.structure}, \DescPageRef{maincls.option.numbers} and
\DescPageRef{maincls.cmd.chaptermarkformat}). Hence, if you are using a
\KOMAScript{} class, you should use the new \Package{scrlayer-scrpage}
package. \iffree{If you have problems, you can still use
\Package{scrpage2}.}{\ignorespaces} Of course, you can also use
\Package{scrlayer-scrpage} with other classes, such as the standard \LaTeX{}
ones.

Apart from the features described in this chapter, \Package{scrlayer-scrpage}
provides several more functions that are likely only of interest to a very
small number of users and therefore are described in
\autoref{cha:scrlayer-scrpage-experts} of \autoref{part:forExperts}, starting
at \autopageref{cha:scrlayer-scrpage-experts}. Nevertheless, if the options
described in \autoref{part:forAuthors} are insufficient for your purposes, you
should examine \autoref{cha:scrlayer-scrpage-experts}.

\LoadCommonFile{options} % \section{Early or late Selection of Options}

\LoadCommonFile{headfootheight} % \section{Header and Footer Height}

\LoadCommonFile{textmarkup} % \section{Text Markup}

\section{Using Predefined Page Styles}
\seclabel{predefined.pagestyles}

The easiest way to create custom headers and footers with
\Package{scrlayer-scrpage} is to use one of the predefined page styles.
%
\iffalse % Umbruchoptimierung
  This section introduces the page styles defined by the
  \Package{scrlayer-scrpage} package as it loads. It also explains the
  commands that you can use to create basic settings for these page
  styles. You can find further options, commands, and background information
  in the following sections and in
  \autoref{sec:scrlayer-scrpage-experts.pagestyle.pairs} in
  \autoref{part:forExperts}.%
\fi

\begin{Declaration}
  \PageStyle{scrheadings}%
  \PageStyle{plain.scrheadings}
\end{Declaration}
The \Package{scrlayer-scrpage} package provides two page styles that you can
reconfigure to your liking. The first page style is
\PageStyle{scrheadings}\important{\PageStyle{scrheadings}}, which is intended
as a page style with running heads. Its defaults are similar to the page style
\PageStyle{headings}\IndexPagestyle{headings} of the standard \LaTeX{} or
\KOMAScript{} classes. You can configure exactly what appears in the header or
footer with the commands and options described below.

The second page style is \PageStyle{plain.scrheadings}%
\important{\PageStyle{plain.scrheadings}}, which is intended to be a style
with no running head. Its defaults resemble those of the
\PageStyle{plain}\IndexPagestyle{plain} page style of the standard or
\KOMAScript{} classes. You can configure exactly what appears in the header or
footer with the commands and options described below.

You could, of course, configure \PageStyle{scrheadings} to be a page style
without a running head and \PageStyle{plain.scrheadings} to be a page style
with a running head. It is, however, advisable to adhere to the conventions
mentioned above. The two page styles mutually influence one another. Once you
apply one of these page styles, \PageStyle{scrheadings} will become accessible
as \PageStyle{headings}\important{\PageStyle{headings}}%
\IndexPagestyle{headings} and the page style \PageStyle{plain.scrheadings}
will become accessible as \PageStyle{plain}\important{\PageStyle{plain}}%
\IndexPagestyle{plain}. Thus, if you use a class or package that automatically
switches between \PageStyle{headings} and \PageStyle{plain}, you only need to
select \PageStyle{scrheadings} or \PageStyle{plain.scrheadings} once. Direct
patches to the corresponding classes or packages are not necessary. This pair
of page styles can thus serve as a drop-in replacement for
\PageStyle{headings} and \PageStyle{plain}. If you need more such pairs,
please refer to \autoref{sec:scrlayer-scrpage-experts.pagestyle.pairs} in
\autoref{part:forExperts}.%
\EndIndexGroup


\begin{Declaration}
  \Macro{lehead}\OParameter{plain.scrheadings content}%
                \Parameter{scrheadings content}%
  \Macro{cehead}\OParameter{plain.scrheadings content}%
                \Parameter{scrheadings content}%
  \Macro{rehead}\OParameter{plain.scrheadings content}%
                \Parameter{scrheadings content}%
  \Macro{lohead}\OParameter{plain.scrheadings content}%
                \Parameter{scrheadings content}%
  \Macro{cohead}\OParameter{plain.scrheadings content}%
                \Parameter{scrheadings content}%
  \Macro{rohead}\OParameter{plain.scrheadings content}%
                \Parameter{scrheadings content}
\end{Declaration}
You can set the contents of the header for the
\DescRef{\LabelBase.pagestyle.plain.scrheadings} and
\DescRef{\LabelBase.pagestyle.scrheadings} page styles with these commands.
The optional argument sets the content of an element of the
\DescRef{\LabelBase.pagestyle.plain.scrheadings} page style, while the
mandatory argument sets the content of the corresponding element of the
\DescRef{\LabelBase.pagestyle.scrheadings} page style.

The contents of even\,---\,or left-hand\,---\,pages\textnote{left-hand pages}
can be set with \Macro{lehead}, \Macro{cehead}, and \Macro{rehead}. The
``\texttt{e}'' appearing as the second letter of the commands' names stands
for ``\emph{even}''.

The contents of odd\,---\,or right-hand\,---\,pages\textnote{right-hand pages}
can be set with \Macro{lohead}, \Macro{cohead}, and \Macro{rohead}. The
``\texttt{o}'' appearing as the second letter of the commands' names stands
for ``\emph{odd}''.

Note\textnote{Attention!} that in one-sided printing, only right-hand pages
exist, and \LaTeX{} designates these as odd pages regardless of their page
number.

Each header consists of a left-aligned\textnote{left aligned} element that can
be set with \Macro{lehead} or \Macro{lohead}. The ``\texttt{l}'' appearing as
the first letter of the commands' names stands for ``\emph{left aligned}''.

Similarly, each header has a centred\textnote{centred} element that can be set
with \Macro{cehead} or \Macro{cohead}. The ``\texttt{c}'' appearing as the
first letter of the command' names stands for ``\emph{centred}''.

Likewise, each header has a right-aligned\textnote{right aligned} element that
can be set with \Macro{rehead} or \Macro{rohead}. The ``\texttt{r}'' appearing
as the first letter of the commands' names stands for ``\emph{right
aligned}''.

\BeginIndexGroup
\BeginIndex{FontElement}{pagehead}\LabelFontElement{pagehead}%
\BeginIndex{FontElement}{pageheadfoot}\LabelFontElement{pageheadfoot}%
These elements do not have individual font attributes that you can
change using the commands \DescRef{\LabelBase.cmd.setkomafont} and
\DescRef{\LabelBase.cmd.addtokomafont} (see \autoref{sec:maincls.textmarkup},
\DescPageRef{maincls.cmd.setkomafont}). Instead, they use an element named
\FontElement{pagehead}. Before this element is applied, the
\FontElement{pageheadfoot} element will also be applied. See
\autoref{tab:scrlayer-scrpage.fontelements} for the defaults of these
elements.%
\EndIndexGroup

The meaning of each command for headers in two-sided printing is illustrated
in \autoref{fig:scrlayer-scrpage.head}.%
%
\begin{figure}[tp]
  \centering
  \begin{picture}(\textwidth,30mm)(0,-10mm)
    \thinlines
    \small\ttfamily
    % left/even page
    \put(0,20mm){\line(1,0){.49\textwidth}}%
    \put(0,0){\line(0,1){20mm}}%
    \multiput(0,0)(0,-1mm){10}{\line(0,-1){.5mm}}%
    \put(.49\textwidth,5mm){\line(0,1){15mm}}%
    \put(.05\textwidth,10mm){%
      \color{ImageRed}%
      \put(-.5em,0){\line(1,0){4em}}%
      \multiput(3.5em,0)(.25em,0){5}{\line(1,0){.125em}}%
      \put(-.5em,0){\line(0,1){\baselineskip}}%
      \put(-.5em,\baselineskip){\line(1,0){4em}}%
      \multiput(3.5em,\baselineskip)(.25em,0){5}{\line(1,0){.125em}}%
      \makebox(4em,5mm)[l]{\Macro{lehead}}%
    }%
    \put(.465\textwidth,10mm){%
      \color{ImageBlue}%
      \put(-4em,0){\line(1,0){4em}}%
      \multiput(-4em,0)(-.25em,0){5}{\line(1,0){.125em}}%
      \put(0,0){\line(0,1){\baselineskip}}%
      \put(-4em,\baselineskip){\line(1,0){4em}}%
      \multiput(-4em,\baselineskip)(-.25em,0){5}{\line(1,0){.125em}}%
      \put(-4.5em,0){\makebox(4em,5mm)[r]{\Macro{rehead}}}%
    }%
    \put(.2525\textwidth,10mm){%
      \color{ImageGreen}%
      \put(-2em,0){\line(1,0){4em}}%
      \multiput(2em,0)(.25em,0){5}{\line(1,0){.125em}}%
      \multiput(-2em,0)(-.25em,0){5}{\line(1,0){.125em}}%
      \put(-2em,\baselineskip){\line(1,0){4em}}%
      \multiput(2em,\baselineskip)(.25em,0){5}{\line(1,0){.125em}}%
      \multiput(-2em,\baselineskip)(-.25em,0){5}{\line(1,0){.125em}}%
      \put(-2em,0){\makebox(4em,5mm)[c]{\Macro{cehead}}}%
    }%
    % right/odd page
    \put(.51\textwidth,20mm){\line(1,0){.49\textwidth}}%
    \put(.51\textwidth,5mm){\line(0,1){15mm}}%
    \put(\textwidth,0){\line(0,1){20mm}}%
    \multiput(\textwidth,0)(0,-1mm){10}{\line(0,-1){.5mm}}%
    \put(.5325\textwidth,10mm){%
      \color{ImageBlue}%
      \put(0,0){\line(1,0){4em}}%
      \multiput(4em,0)(.25em,0){5}{\line(1,0){.125em}}%
      \put(0,0){\line(0,1){\baselineskip}}%
      \put(0em,\baselineskip){\line(1,0){4em}}%
      \multiput(4em,\baselineskip)(.25em,0){5}{\line(1,0){.125em}}%
      \put(.5em,0){\makebox(4em,5mm)[l]{\Macro{lohead}}}%
    }%
    \put(.965\textwidth,10mm){%
      \color{ImageRed}%
      \put(-4em,0){\line(1,0){4em}}%
      \multiput(-4em,0)(-.25em,0){5}{\line(1,0){.125em}}%
      \put(0,0){\line(0,1){\baselineskip}}%
      \put(-4em,\baselineskip){\line(1,0){4em}}%
      \multiput(-4em,\baselineskip)(-.25em,0){5}{\line(1,0){.125em}}%
      \put(-4.5em,0){\makebox(4em,5mm)[r]{\Macro{rohead}}}%
    }%
    \put(.75\textwidth,10mm){%
      \color{ImageGreen}%
      \put(-2em,0){\line(1,0){4em}}%
      \multiput(2em,0)(.25em,0){5}{\line(1,0){.125em}}%
      \multiput(-2em,0)(-.25em,0){5}{\line(1,0){.125em}}%
      \put(-2em,\baselineskip){\line(1,0){4em}}%
      \multiput(2em,\baselineskip)(.25em,0){5}{\line(1,0){.125em}}%
      \multiput(-2em,\baselineskip)(-.25em,0){5}{\line(1,0){.125em}}%
      \put(-2em,0){\makebox(4em,5mm)[c]{\Macro{cohead}}}%
    }%
    % commands for both pages
    \color{ImageBlue}%
    \put(.5\textwidth,0){\makebox(0,\baselineskip)[c]{\Macro{ihead}}}%
    \color{ImageGreen}%
    \put(.5\textwidth,-5mm){\makebox(0,\baselineskip)[c]{\Macro{chead}}}
    \color{ImageRed}%
    \put(.5\textwidth,-10mm){\makebox(0,\baselineskip)[c]{\Macro{ohead}}}
    \put(\dimexpr.5\textwidth-2em,.5\baselineskip){%
      \color{ImageBlue}%
      \put(0,0){\line(-1,0){1.5em}}%
      \put(-1.5em,0){\vector(0,1){5mm}}%
      \color{ImageGreen}%
      \put(0,-1.25\baselineskip){\line(-1,0){\dimexpr .25\textwidth-2em\relax}}%
      \put(-\dimexpr
      .25\textwidth-2em\relax,-1.25\baselineskip){\vector(0,1){\dimexpr
          5mm+1.25\baselineskip\relax}}
      \color{ImageRed}%
      \put(0,-2.5\baselineskip){\line(-1,0){\dimexpr .45\textwidth-4em\relax}}%
      \put(-\dimexpr
      .45\textwidth-4em\relax,-2.5\baselineskip){\vector(0,1){\dimexpr
          5mm+2.5\baselineskip\relax}}
    }%
    \put(\dimexpr.5\textwidth+2em,.5\baselineskip){%
      \color{ImageBlue}%
      \put(0,0){\line(1,0){1.5em}}%
      \put(1.5em,0){\vector(0,1){5mm}}%
      \color{ImageGreen}%
      \put(0,-1.25\baselineskip){\line(1,0){\dimexpr .25\textwidth-2em\relax}}
      \put(\dimexpr
      .25\textwidth-2em\relax,-1.25\baselineskip){\vector(0,1){\dimexpr
          5mm+1.25\baselineskip\relax}}
      \color{ImageRed}%
      \put(0,-2.5\baselineskip){\line(1,0){\dimexpr .45\textwidth-4em\relax}}
      \put(\dimexpr
      .45\textwidth-4em\relax,-2.5\baselineskip){\vector(0,1){\dimexpr
          5mm+2.5\baselineskip\relax}}
   }%
  \end{picture}
  \caption[Commands for setting the page header]%
          {The meaning of the commands for setting the contents of page headers
          shown on a two-page schematic}
  \label{fig:scrlayer-scrpage.head}
\end{figure}
%
\begin{Example}
  Suppose you're writing a short article and you want the author's name to
  appear on the left side of the page and the article's title to appear
  right. You can write, for example:
\begin{lstcode}
  \documentclass{scrartcl}
  \usepackage{scrlayer-scrpage}
  \lohead{John Doe}
  \rohead{Page style with \KOMAScript}
  \begin{document}
  \title{Page styles with \KOMAScript}
  \author{John Doe}
  \maketitle
  \end{document}
\end{lstcode}
  But what happens? On the first page there's only a page number in the
  footer, while the header remains empty!

  The explanation is simple: The \Class{scrartcl} class, like the default
  \Class{article} class, switches to the \PageStyle{plain} page style for the
  page which contains the title. After the command
  \DescRef{maincls.cmd.pagestyle}\PParameter{scrheadings} in the preamble of
  our example, this actually refers to the
  \DescRef{\LabelBase.pagestyle.plain.scrheadings} page style. The default for
  this page style when using a \KOMAScript{} class is an empty page header and
  a page number in the footer. In the example, the optional arguments of
  \Macro{lohead} and \Macro{rohead} are omitted, so the
  \DescRef{\LabelBase.pagestyle.plain.scrheadings} page style remains
  unchanged and the result for the first page is actually correct.

  Now add enough text to the example after \DescRef{maincls.cmd.maketitle} 
  so that a second page is printed. You can simply add
  \Macro{usepackage}\PParameter{lipsum}\IndexPackage{lipsum} to the document
  preamble and \Macro{lipsum}\IndexCmd{lipsum} below
  \DescRef{maincls.cmd.maketitle}. You will see that the header of the second
  page now contains the author and the document title as we wanted.

  For comparison, you should also add the optional argument to
  \Macro{lohead} and \Macro{rohead}. Change the example as follows:
\begin{lstcode}
  \documentclass{scrartcl}
  \usepackage{scrlayer-scrpage}
  \lohead[John Doe]
         {John Doe}
  \rohead[Page style with \KOMAScript]
         {Page style with \KOMAScript}
  \begin{document}
  \title{Page styles with \KOMAScript}
  \author{John Doe}
  \maketitle
  \end{document}
\end{lstcode}
  Now you have a header on the first page just above the title itself.
  That is because you have reconfigured page style
  \DescRef{\LabelBase.pagestyle.plain.scrheadings} with the two optional
  arguments. As you probably appreciate, it would be better to leave this page
  style unchanged, as a running head above the document title is rather
  annoying.
  
  By the way, as an alternative to configuring
  \DescRef{\LabelBase.pagestyle.plain.scrheadings} you could, if you were
  using a \KOMAScript{} class, have changed the page style for pages that
  contain title headers. See \DescRef{maincls.cmd.titlepagestyle}%
  \important{\DescRef{maincls.cmd.titlepagestyle}}%
  \IndexCmd{titlepagestyle} in \autoref{sec:maincls.pagestyle},
  \DescPageRef{maincls.cmd.titlepagestyle}.
\end{Example}

Note\textnote{Attention!} that you should never put a section
heading or section number directly into the header using one of these
commands. Because of the asynchronous way that \TeX{} lays out and outputs
pages, doing so can easily result in the wrong number or heading text in the
running head. Instead you should use the mark mechanism, ideally in
conjunction with the procedures explained in the next section.%
\EndIndexGroup

\begin{Declaration}
  \Macro{lehead*}\OParameter{plain.scrheadings content}%
                \Parameter{scrheadings content}%
  \Macro{cehead*}\OParameter{plain.scrheadings content}%
                \Parameter{scrheadings content}%
  \Macro{rehead*}\OParameter{plain.scrheadings content}%
                \Parameter{scrheadings content}%
  \Macro{lohead*}\OParameter{plain.scrheadings content}%
                \Parameter{scrheadings content}%
  \Macro{cohead*}\OParameter{plain.scrheadings content}%
                \Parameter{scrheadings content}%
  \Macro{rohead*}\OParameter{plain.scrheadings content}%
                \Parameter{scrheadings content}
\end{Declaration}
The starred versions\ChangedAt{v3.14}{\Package{scrlayer-scrpage}} of the
previously described commands differ from the ordinary versions only if you
omit the optional argument \PName{plain.scrheadings content}. In this case,
the version without the star does not change the contents of
\DescRef{\LabelBase.pagestyle.plain.scrheadings}. The starred version, on the
other hand, uses the mandatory argument \PName{scrheading content}
for \DescRef{\LabelBase.pagestyle.plain.scrheadings} as well. So if both
arguments should be the same, you can simply use the starred version with only
the mandatory argument.%

\begin{Example}
  You can shorten the previous example using the starred versions of
  \DescRef{\LabelBase.cmd.lohead} and \DescRef{\LabelBase.cmd.rohead}:
\begin{lstcode}
  \documentclass{scrartcl}
  \usepackage{scrlayer-scrpage}
  \lohead*{John Doe}
  \rohead*{Page style with \KOMAScript}
  \begin{document}
  \title{Page styles with \KOMAScript}
  \author{John Doe}
  \maketitle
  \end{document}
\end{lstcode}%
\end{Example}%
\EndIndexGroup
\ExampleEndFix


\begin{Declaration}
  \Macro{lefoot}\OParameter{plain.scrheadings content}%
                \Parameter{scrheadings content}%
  \Macro{cefoot}\OParameter{plain.scrheadings content}%
                \Parameter{scrheadings content}%
  \Macro{refoot}\OParameter{plain.scrheadings content}%
                \Parameter{scrheadings content}%
  \Macro{lofoot}\OParameter{plain.scrheadings content}%
                \Parameter{scrheadings content}%
  \Macro{cofoot}\OParameter{plain.scrheadings content}%
                \Parameter{scrheadings content}%
  \Macro{rofoot}\OParameter{plain.scrheadings content}%
                \Parameter{scrheadings content}
\end{Declaration}
You can define the contents of the footer for
\DescRef{\LabelBase.pagestyle.scrheadings} and
\DescRef{\LabelBase.pagestyle.plain.scrheadings} with these commands. The
optional argument defines the content of an element of
\DescRef{\LabelBase.pagestyle.plain.scrheadings}, while the mandatory argument
sets the content of the corresponding element of
\DescRef{\LabelBase.pagestyle.scrheadings}.

The contents of even\,---\,or left-hand\,---\,pages\textnote{left-hand page}
are set with \Macro{lefoot}, \Macro{cefoot}, and \Macro{refoot}. The
``\texttt{e}'' appearing as the second letter of the commands' names stands
for ``\emph{even}''.

The contents of odd\,---\,or right-hand\,---\,pages\textnote{right-hand page}
are set with \Macro{lofoot}, \Macro{cofoot}, and \Macro{rofoot}. The
``\texttt{o}'' appearing as the second letter of the commands' names stands
for ``\emph{odd}''.

Note\textnote{Attention!} that in one-sided printing, only right-hand pages
exist, and \LaTeX{} designates these as odd pages regardless of their page
number.

Each footer consists of a left-aligned\textnote{left aligned} element that can
be set with \Macro{lefoot} or \Macro{lofoot}. The ``\texttt{l}'' appearing as
the first letter of the commands' names stands for ``\emph{left aligned}''.

Similarly, each footer has a centred\textnote{centred} element that can be set
with \Macro{cefoot} or \Macro{cofoot}. The ``\texttt{c}'' in the first letter
of the command' names stands for ``\emph{centred}''.

Likewise, each footer has a right-aligned\textnote{right aligned} element that
can be set with \Macro{refoot} or \Macro{rofoot}. The ``\texttt{r}'' in the
first letter of the commands' names stands for ``\emph{right aligned}''.

\BeginIndexGroup
\BeginIndex{FontElement}{pagefoot}\LabelFontElement{pagefoot}%
\BeginIndex{FontElement}{pageheadfoot}\LabelFontElement[foot]{pageheadfoot}%
However, these elements do not have individual font attributes that can be
changed with the \DescRef{\LabelBase.cmd.setkomafont} and
\DescRef{\LabelBase.cmd.addtokomafont} commands (see
\autoref{sec:maincls.textmarkup}, \DescPageRef{maincls.cmd.setkomafont}).
Instead, they use an element named
\FontElement{pagefoot}\important{\FontElement{pagefoot}}. Before this element
is applied, the font element
\FontElement{pageheadfoot}\important{\FontElement{pageheadfoot}} is also
applied. See \autoref{tab:scrlayer-scrpage.fontelements} for the defaults of
the fonts of these elements.%
\EndIndexGroup

The meaning of each command for footers in two-sided printing is illustrated
in \autoref{fig:scrlayer-scrpage.foot}.%
%
\begin{figure}[bp]
  \centering
  \begin{picture}(\textwidth,30mm)
    \thinlines
    \small\ttfamily
    % left page
    \put(0,0){\line(1,0){.49\textwidth}}%
    \put(0,0){\line(0,1){20mm}}%
    \multiput(0,20mm)(0,1mm){10}{\line(0,1){.5mm}}%
    \put(.49\textwidth,0){\line(0,1){15mm}}%
    \put(.05\textwidth,5mm){%
      \color{ImageRed}%
      \put(-.5em,0){\line(1,0){4em}}%
      \multiput(3.5em,0)(.25em,0){5}{\line(1,0){.125em}}%
      \put(-.5em,0){\line(0,1){\baselineskip}}%
      \put(-.5em,\baselineskip){\line(1,0){4em}}%
      \multiput(3.5em,\baselineskip)(.25em,0){5}{\line(1,0){.125em}}%
      \makebox(4em,5mm)[l]{\Macro{lefoot}}%
    }%
    \put(.465\textwidth,5mm){%
      \color{ImageBlue}%
      \put(-4em,0){\line(1,0){4em}}%
      \multiput(-4em,0)(-.25em,0){5}{\line(1,0){.125em}}%
      \put(0,0){\line(0,1){\baselineskip}}%
      \put(-4em,\baselineskip){\line(1,0){4em}}%
      \multiput(-4em,\baselineskip)(-.25em,0){5}{\line(1,0){.125em}}%
      \put(-4.5em,0){\makebox(4em,5mm)[r]{\Macro{refoot}}}%
    }%
    \put(.2525\textwidth,5mm){%
      \color{ImageGreen}%
      \put(-2em,0){\line(1,0){4em}}%
      \multiput(2em,0)(.25em,0){5}{\line(1,0){.125em}}%
      \multiput(-2em,0)(-.25em,0){5}{\line(1,0){.125em}}%
      \put(-2em,\baselineskip){\line(1,0){4em}}%
      \multiput(2em,\baselineskip)(.25em,0){5}{\line(1,0){.125em}}%
      \multiput(-2em,\baselineskip)(-.25em,0){5}{\line(1,0){.125em}}%
      \put(-2em,0){\makebox(4em,5mm)[c]{\Macro{cefoot}}}%
    }%
    % right page
    \put(.51\textwidth,0){\line(1,0){.49\textwidth}}%
    \put(.51\textwidth,0){\line(0,1){15mm}}%
    \put(\textwidth,0){\line(0,1){20mm}}%
    \multiput(\textwidth,20mm)(0,1mm){10}{\line(0,1){.5mm}}%
    \put(.5325\textwidth,5mm){%
      \color{ImageBlue}%
      \put(0,0){\line(1,0){4em}}%
      \multiput(4em,0)(.25em,0){5}{\line(1,0){.125em}}%
      \put(0,0){\line(0,1){\baselineskip}}%
      \put(0em,\baselineskip){\line(1,0){4em}}%
      \multiput(4em,\baselineskip)(.25em,0){5}{\line(1,0){.125em}}%
      \put(.5em,0){\makebox(4em,5mm)[l]{\Macro{lofoot}}}%
    }%
    \put(.965\textwidth,5mm){%
      \color{ImageRed}%
      \put(-4em,0){\line(1,0){4em}}%
      \multiput(-4em,0)(-.25em,0){5}{\line(1,0){.125em}}%
      \put(0,0){\line(0,1){\baselineskip}}%
      \put(-4em,\baselineskip){\line(1,0){4em}}%
      \multiput(-4em,\baselineskip)(-.25em,0){5}{\line(1,0){.125em}}%
      \put(-4.5em,0){\makebox(4em,5mm)[r]{\Macro{rofoot}}}%
    }%
    \put(.75\textwidth,5mm){%
      \color{ImageGreen}%
      \put(-2em,0){\line(1,0){4em}}%
      \multiput(2em,0)(.25em,0){5}{\line(1,0){.125em}}%
      \multiput(-2em,0)(-.25em,0){5}{\line(1,0){.125em}}%
      \put(-2em,\baselineskip){\line(1,0){4em}}%
      \multiput(2em,\baselineskip)(.25em,0){5}{\line(1,0){.125em}}%
      \multiput(-2em,\baselineskip)(-.25em,0){5}{\line(1,0){.125em}}%
      \put(-2em,0){\makebox(4em,5mm)[c]{\Macro{cofoot}}}%
    }%
    % both pages
    \color{ImageBlue}%
    \put(.5\textwidth,15mm){\makebox(0,\baselineskip)[c]{\Macro{ifoot}}}%
    \color{ImageGreen}%
    \put(.5\textwidth,20mm){\makebox(0,\baselineskip)[c]{\Macro{cfoot}}}
    \color{ImageRed}%
    \put(.5\textwidth,25mm){\makebox(0,\baselineskip)[c]{\Macro{ofoot}}}
    \put(\dimexpr.5\textwidth-2em,.5\baselineskip){%
      \color{ImageBlue}%
      \put(0,15mm){\line(-1,0){1.5em}}%
      \put(-1.5em,15mm){\vector(0,-1){5mm}}%
      \color{ImageGreen}%
      \put(0,20mm){\line(-1,0){\dimexpr .25\textwidth-2em\relax}}%
      \put(-\dimexpr .25\textwidth-2em\relax,20mm){\vector(0,-1){10mm}}%
      \color{ImageRed}%
      \put(0,25mm){\line(-1,0){\dimexpr .45\textwidth-4em\relax}}%
      \put(-\dimexpr .45\textwidth-4em\relax,25mm){\vector(0,-1){15mm}}%
    }%
    \put(\dimexpr.5\textwidth+2em,.5\baselineskip){%
      \color{ImageBlue}%
      \put(0,15mm){\line(1,0){1.5em}}%
      \put(1.5em,15mm){\vector(0,-1){5mm}}%
      \color{ImageGreen}%
      \put(0,20mm){\line(1,0){\dimexpr .25\textwidth-2em\relax}}%
      \put(\dimexpr .25\textwidth-2em\relax,20mm){\vector(0,-1){10mm}}%
      \color{ImageRed}%
      \put(0,25mm){\line(1,0){\dimexpr .45\textwidth-4em\relax}}%
      \put(\dimexpr .45\textwidth-4em\relax,25mm){\vector(0,-1){15mm}}%
    }%
  \end{picture}
  \caption[Commands for setting the page footer]%
          {The meaning of the commands for setting the contents of page
            footers shown on a two-page schematic}%
  \label{fig:scrlayer-scrpage.foot}
\end{figure}
%
\begin{Example}
  Let's return to the example of the short article. Let's say you want to
  specify the publisher in the left side of the footer. You would change the
  example above to:
\begin{lstcode}
  \documentclass{scrartcl}
  \usepackage{scrlayer-scrpage}
  \lohead{John Doe}
  \rohead{Page style with \KOMAScript}
  \lofoot{Smart Alec Publishing}
  \usepackage{lipsum}
  \begin{document}
  \title{Page styles with \KOMAScript}
  \author{John Doe}
  \maketitle
  \lipsum
  \end{document}
\end{lstcode}
  Once again the publisher is not printed on the first page with the title.
  The reason is the same as in the example with
  \DescRef{\LabelBase.cmd.lohead} above. And the solution for getting the
  publisher on the first page is similar:
\begin{lstcode}
  \lofoot[Smart Alec Publishing]
         {Smart Alec Publishing}
\end{lstcode}
  Now you decide\textnote{font change}\important{\FontElement{pageheadfoot}}%
  \IndexFontElement{pageheadfoot} that the header and footer should use an
  upright but smaller font in place of the default slanted font:
\begin{lstcode}
  \setkomafont{pageheadfoot}{\small}
\end{lstcode}
  In addition, the header, but not the footer, should be bold:
\begin{lstcode}
  \setkomafont{pagehead}{\bfseries}
\end{lstcode}
  It is important\textnote{Attention!} that this command does not occur until
  after \Package{scrpage-scrlayer} has been loaded because the \KOMAScript{}
  class defines \DescRef{\LabelBase.fontelement.pagehead} as an alias for 
  \DescRef{\LabelBase.fontelement.pageheadfoot}. Only by loading
  \Package{scrpage-scrlayer} will \DescRef{\LabelBase.fontelement.pagehead}
  become an element independent of
  \DescRef{\LabelBase.fontelement.pageheadfoot}.

  Now add one more \Macro{lipsum} and the
  \Option{twoside}\IndexOption{twoside}\important{\Option{twoside}} option
  when   loading \Class{scrartcl}. First of all, you will see the page number
  moves from the centre to the outer margin of the page footer, due to the
  changed defaults of \DescRef{\LabelBase.pagestyle.scrheadings} and
  \DescRef{\LabelBase.pagestyle.plain.scrheadings} for two-sided printing with
  a \KOMAScript{} class.

  Simultaneously, the author, document title, and publisher will vanish from
  page~2. They only appear on page~3. That's because we've only used
  commands for odd pages. You can recognise this by the ``\texttt{o}'' in the
  second position of the command names.

  Now, we could simply copy those commands and replace the ``\texttt{o}'' with
  an ``\texttt{e}'' to define the contents of \emph{even} pages. But with
  two-sided printing, it makes more sense to use mirror-inverted elements,
  i.\,e. the left element of an even page should become the right element of
  the odd page and visa versa. To achieve this, we also replace the first
  letter ``\texttt{l}'' with ``\texttt{r}'':
\begin{lstcode}
  \documentclass[twoside]{scrartcl}
  \usepackage{scrlayer-scrpage}
  \lohead{John Doe}
  \rohead{Page style with \KOMAScript}
  \lofoot[Smart Alec Publishing]
         {Smart Alec Publishing}
  \rehead{John Doe}
  \lohead{Page style with \KOMAScript}
  \refoot[Smart Alec Publishing]
         {Smart Alec Publishing}
  \usepackage{lipsum}
  \begin{document}
  \title{Page styles with \KOMAScript}
  \author{John Doe}
  \maketitle
  \lipsum\lipsum
  \end{document}
\end{lstcode}
\end{Example}
%
Since it is a bit cumbersome to define left and right pages separately in
cases such as the previous example, a simpler solution for this common case is
introduced below.

Allow me once again an important note:\textnote{Attention!} you should
never put a section heading or section number directly into the footer using
one of these commands. Because of the asynchronous way that \TeX{} lays out and
outputs pages, doing so can easily result in the wrong number or heading text
in the running head. Instead you should use the mark mechanism, ideally in
conjunction with the procedures explained in the next section.%
\EndIndexGroup


\begin{Declaration}
  \Macro{lefoot*}\OParameter{plain.scrheadings content}%
                \Parameter{scrheadings content}%
  \Macro{cefoot*}\OParameter{plain.scrheadings content}%
                \Parameter{scrheadings content}%
  \Macro{refoot*}\OParameter{plain.scrheadings content}%
                \Parameter{scrheadings content}%
  \Macro{lofoot*}\OParameter{plain.scrheadings content}%
                \Parameter{scrheadings content}%
  \Macro{cofoot*}\OParameter{plain.scrheadings content}%
                \Parameter{scrheadings content}%
  \Macro{rofoot*}\OParameter{plain.scrheadings content}%
                \Parameter{scrheadings content}
\end{Declaration}
The starred versions\ChangedAt{v3.14}{\Package{scrlayer-scrpage}} of the
previously described commands differ only if you omit the optional argument
\OParameter{plain.scrheadings content}. In this case, the version without the
star does not change the contents of
\DescRef{\LabelBase.pagestyle.plain.scrheadings}. The starred version, on the
other hand, uses the mandatory argument \PName{scrheading content} for
\DescRef{\LabelBase.pagestyle.plain.scrheadings} as well. So if both arguments
should be the same, you can simply use the starred version with just the
mandatory argument.%

\begin{Example}
  You can shorten the previous example using the star versions of
  \DescRef{\LabelBase.cmd.lofoot} and \DescRef{\LabelBase.cmd.refoot}:
\begin{lstcode}
  \documentclass[twoside]{scrartcl}
  \usepackage{scrlayer-scrpage}
  \lohead{John Doe}
  \rohead{Page style with \KOMAScript}
  \lofoot*{Smart Alec Publishing}
  \rehead{John Doe}
  \lohead{Page style with \KOMAScript}
  \refoot*{Smart Alec Publishing}
  \usepackage{lipsum}
  \begin{document}
  \title{Page styles with \KOMAScript}
  \author{John Doe}
  \maketitle
  \lipsum\lipsum
  \end{document}
\end{lstcode}
\end{Example}
%
\EndIndexGroup
\ExampleEndFix


\begin{Declaration}
  \Macro{ohead}\OParameter{plain.scrheadings content}%
                \Parameter{scrheadings content}%
  \Macro{chead}\OParameter{plain.scrheadings content}%
                \Parameter{scrheadings content}%
  \Macro{ihead}\OParameter{plain.scrheadings content}%
                \Parameter{scrheadings content}%
  \Macro{ofoot}\OParameter{plain.scrheadings content}%
                \Parameter{scrheadings content}%
  \Macro{cfoot}\OParameter{plain.scrheadings content}%
                \Parameter{scrheadings content}%
  \Macro{ifoot}\OParameter{plain.scrheadings content}%
                \Parameter{scrheadings content}
\end{Declaration}
To configure the headers and footers for two-sided printing with the
previously described commands, you would have to configure the left and right
sides separately from one another. In most cases, however, the left and right
sides are more or less symmetrical. An item that appears on the left of an
even page should appear on the right of an odd page and vice versa. Centred
elements are usually centred on both sides.

To simplify the definition of such symmetric page styles,
\Package{scrlayer-scrpage} has shortcuts. The \Macro{ohead} command
corresponds to a call to both \DescRef{\LabelBase.cmd.lehead} and
\DescRef{\LabelBase.cmd.rohead}. The \Macro{chead} command corresponds to a
call to both \DescRef{\LabelBase.cmd.cehead} and
\DescRef{\LabelBase.cmd.cohead}. And the \Macro{ihead} command corresponds to
a call to both \DescRef{\LabelBase.cmd.rehead} and
\DescRef{\LabelBase.cmd.lohead}. The same applies to the equivalent commands
for the page footer. An outline of these relationships can also be found in
\autoref{fig:scrlayer-scrpage.head} on \autopageref{fig:scrlayer-scrpage.head}
and \autoref{fig:scrlayer-scrpage.foot} on
\autopageref{fig:scrlayer-scrpage.foot}.
%
\begin{Example}
  You can simplify the previous example using the new commands:
\begin{lstcode}
  \documentclass[twoside]{scrartcl}
  \usepackage{scrlayer-scrpage}
  \ihead{John Doe}
  \ohead{Page style with \KOMAScript}
  \ifoot[Smart Alec Publishing]
        {Smart Alec Publishing}
  \usepackage{lipsum}
  \begin{document}
  \title{Page styles with \KOMAScript}
  \author{John Doe}
  \maketitle
  \lipsum\lipsum
  \end{document}
\end{lstcode}
\iffalse%
  As you can see, you can use half the number of commands but get the same
  result. %
\fi%
\end{Example}%
Because one-sided printing treats all pages as odd pages, these commands are
synonymous with the corresponding right-side commands when in one-sided mode.
Therefore in most cases you will only need these six commands instead of the
twelve described before.

Allow me once again an important note:\textnote{Attention!} you should never
put a section heading or section number directly into the footer using one of
these commands. Because of the asynchronous way that \TeX{} lays out and
outputs pages, doing so can easily result in the wrong number or heading text
in the running head. Instead you should use the mark mechanism, ideally in
conjunction with the procedures explained in the next section.%
\EndIndexGroup


\begin{Declaration}
  \Macro{ohead*}\OParameter{plain.scrheadings content}%
                \Parameter{scrheadings content}%
  \Macro{chead*}\OParameter{plain.scrheadings content}%
                \Parameter{scrheadings content}%
  \Macro{ihead*}\OParameter{plain.scrheadings content}%
                \Parameter{scrheadings content}%
  \Macro{ofoot*}\OParameter{plain.scrheadings content}%
                \Parameter{scrheadings content}%
  \Macro{cfoot*}\OParameter{plain.scrheadings content}%
                \Parameter{scrheadings content}%
  \Macro{ifoot*}\OParameter{plain.scrheadings content}%
                \Parameter{scrheadings content}
\end{Declaration}
The previously described commands also have starred
versions\ChangedAt{v3.14}{\Package{scrlayer-scrpage}} that differ only if you
omit the optional argument \OParameter{plain.scrheadings content}. In this
case, the version without a star does not change the content of
\DescRef{\LabelBase.pagestyle.plain.scrheadings}. The version with the star,
on the other hand, also uses the mandatory argument \PName{scrheadings
content} for \DescRef{\LabelBase.pagestyle.plain.scrheadings}. So if both
arguments should be the same, you can simply use the starred version with only
the mandatory argument.%

\begin{Example}
  You can shorten the previous example using the star version of
  \DescRef{\LabelBase.cmd.ifoot}:
\begin{lstcode}
  \documentclass[twoside]{scrartcl}
  \usepackage{scrlayer-scrpage}
  \ihead{John Doe}
  \ohead{Page style with \KOMAScript}
  \ifoot*{Smart Alec Publishing}
  \usepackage{lipsum}
  \begin{document}
  \title{Page styles with \KOMAScript}
  \author{John Doe}
  \maketitle
  \lipsum\lipsum
  \end{document}
\end{lstcode}%
\end{Example}%
\EndIndexGroup


\begin{Declaration}
  \OptionVName{pagestyleset}{setting}
\end{Declaration}
\BeginIndex{Option}{pagestyleset~=KOMA-Script}%
\BeginIndex{Option}{pagestyleset~=standard}%
The examples above refer several times to the default settings of the page
styles \DescRef{\LabelBase.pagestyle.scrheadings}\IndexPagestyle{scrheadings}
and \DescRef{\LabelBase.pagestyle.plain.scrheadings}%
\IndexPagestyle{plain.scrheadings}. In fact, \Package{scrlayer-scrpage}
currently provides two different defaults for these page styles. You can
select them manually with the \Option{pagestyleset} option.

The 
\PValue{KOMA-Script}\important{\OptionValue{pagestyleset}{KOMA-Script}}
\PName{setting} selects the defaults, which are also set automatically if the
option is not specified and a \KOMAScript{} class is detected. In two-sided
printing, \DescRef{\LabelBase.pagestyle.scrheadings} uses outer-aligned
running heads in the header and outer-aligned page numbers in the footer.
In one-sided printing, the running head will be printed in the
middle of the header and the page number in the middle of the footer. Upper-
and lower-case letters are used in the automatic running heads as they
actually appear in the sectioning headings. This corresponds to the
\OptionValueRef{\LabelBase}{markcase}{used}\IndexOption{markcase~=used}%
\important{\OptionValueRef{\LabelBase}{markcase}{used}} option. The
\DescRef{\LabelBase.pagestyle.plain.scrheadings} page style has no running
heads, but the page numbers are printed in the same manner.

However, if the \hyperref[cha:scrlttr2]{\Class{scrlttr2}}%
\important{\hyperref[cha:scrlttr2]{\Class{scrlttr2}}}%
\IndexClass{scrlttr2} class is detected, the default settings are based on the
page styles of that class. See \autoref{sec:scrlttr2.pagestyle}, 
\autopageref{sec:scrlttr2.pagestyle}.

The 
\PValue{standard}\important{\OptionValue{pagestyleset}{standard}}
\PName{setting} selects defaults that match the page styles of the standard
classes. This is also activated automatically if the option has not been
specified and no \KOMAScript{} class is detected. In this case, for two-sided
printing \DescRef{\LabelBase.pagestyle.scrheadings} uses running heads
inner-aligned in the header, and the page numbers will be printed\,---\,also
in the header\,---\,outer-aligned. One-sided printing uses the same settings,
but since only right-hand pages exist in this mode, the running head will
always be left-aligned and the page number right-aligned. The automatic
running heads\,---\,despite considerable typographic objections\,---\,are
converted to capital letters, as they would be with
\OptionValueRef{\LabelBase}{markcase}{upper}\IndexOption{markcase~=upper}%
\important{\OptionValueRef{\LabelBase}{markcase}{upper}}. In one-sided
printing, the \DescRef{\LabelBase.pagestyle.plain.scrheadings} page style
differs considerably from \DescRef{\LabelBase.pagestyle.scrheadings} because
the page number is printed in the middle of the footer.
Unlike\textnote{\KOMAScript{} vs. standard classes} the \PageStyle{plain} page
style in the standard classes,
\DescRef{\LabelBase.pagestyle.plain.scrheadings} omits the page number in
two-sided printing. The standard classes print the page number in the middle
of the footer, which does not match the rest of the page styles in two-sided
printing.
\iffalse % Umbruchkorrekturtext
  If you want the page number back
\begin{lstcode}
  \cfoot[\pagemark]{}
\end{lstcode}
  will restore it. %
\fi%
The running head is omitted in \DescRef{\LabelBase.pagestyle.plain.scrheadings}.

Note\textnote{Attention!} that using this option activates the
\DescRef{\LabelBase.pagestyle.scrheadings}\IndexPagestyle{scrheadings}%
\important{\DescRef{\LabelBase.pagestyle.scrheadings}} page style.
\iffalse% Umbruchkorrektur
  This also applies if you use the option within the document.%
\fi
%
\EndIndexGroup


\LoadCommonFile{pagestylemanipulation} % \section{Manipulating Defined Page Styles}

\begin{Declaration}
  \OptionVName{headwidth}{width\textup{:}offset\textup{:}offset}%
  \OptionVName{footwidth}{width\textup{:}offset\textup{:}offset}
\end{Declaration}
By default the header\Index{header>width} and footer\Index{footer>width} are
as wide as the type area. However, you can change this using these
\KOMAScript{} options. The value \PName{width} is the desired width of the
header or footer. The \PName{offset} defines how far the header or footer
should be moved towards the outer\,---\,in one-sided printing to the
right\,---\,margin. All three\ChangedAt{v3.14}{\Package{scrlayer-scrpage}}
values are optional and can be omitted. If you omit a value, you can also omit
the associated colon to the left of it. If only one \PName{offset} is
specified, it is used for both odd and even pages. Otherwise, the first
\PName{offset} is used for odd and the second \PName{offset} for even pages in
two-sided mode. If you only use one value without a colon, this will be the
\PName{width}.

For both the \PName{width} and the \PName{offset} you can use any valid length
value, \LaTeX{} length, \TeX{} dimension, or \TeX{} skip. In addition, you can
use an \eTeX{} dimension expression with the basic arithmetic operations
\texttt{+}, \texttt{-}, \texttt{*}, \texttt{/}, and parentheses. See
\cite[section~3.5]{manual:eTeX} for more information on such expressions. See
\autoref{sec:scrlayer-scrpage.options} for more information on using a
\LaTeX{} length as an option value. The \PName{width} can also be one of the
symbolic values shown in \autoref{tab:scrlayer-scrpage.symbolic.values}.

By default the header and the footer are the width of the text area. The
default \PName{offset} depends on the selected \PName{width}. One-sided
printing typically uses half the difference between \PName{width} and the
width of the text area. This centres the header horizontally above the text
area. Two-sided printing, on the other hand, uses only a third of the
difference between \PName{width} and the width of the text area. However, if
\PName{width} is the width of the whole text area and the marginal note
column, the default \PName{offset} will be zero. If this is too complicated
for you, you should simply specify the desired \PName{offset} yourself.
%
\begin{table}
  \centering
  \caption[Symbolic values for the \Option{headwidth} and \Option{footwidth}
  options]{Available symbolic values for the \PName{width} value of options
    \Option{headwidth} and \Option{footwidth}}
  \label{tab:scrlayer-scrpage.symbolic.values}
  \begin{desctabular}
    \entry{\PValue{foot}}{%
      the current width of the footer%
    }%
    \entry{\PValue{footbotline}}{%
      the current length of the horizontal line below the footer%
    }%
    \entry{\PValue{footsepline}}{%
      the current length of the horizontal line above the footer%
    } \entry{\PValue{head}}{%
      the current width of the header%
    }%
    \entry{\PValue{headsepline}}{%
      the current length of the horizontal line below the header%
    }%
    \entry{\PValue{headtopline}}{%
      the current length of the horizontal line above the header%
    }%
    \entry{\PValue{marginpar}}{%
      the width of the marginal note column including the distance
      between the text area and the marginal note column%
    }%
    \entry{\PValue{page}}{%
      the width of the page taking into account any binding correction defined
      with the help of the \Package{typearea} package (see the
      \DescRef{typearea.option.BCOR} option in
      \autoref{sec:typearea.typearea}, \DescPageRef{typearea.option.BCOR})%
    }%
    \entry{\PValue{paper}}{%
      the width of the paper without considering any binding correction%
    }%
    \entry{\PValue{text}}{%
      the width of the text area%
    }%
    \entry{\PValue{textwithmarginpar}}{%
      the width of the text area including the marginal note column and
      the distance between the two (Note: only in this case is
      the default for \PName{offset} zero)%
    }%
  \end{desctabular}
\end{table}
%
\EndIndexGroup


\begin{Declaration}
  \OptionVName{headtopline}{thickness\textup{:}length}%
  \OptionVName{headsepline}{thickness\textup{:}length}%
  \OptionVName{footsepline}{thickness\textup{:}length}%
  \OptionVName{footbotline}{thickness\textup{:}length}
\end{Declaration}
\BeginIndex{Option}{headtopline~=\PName{thickness\textup{:}length}}%
\BeginIndex{Option}{headsepline~=\PName{thickness\textup{:}length}}%
\BeginIndex{Option}{footsepline~=\PName{thickness\textup{:}length}}%
\BeginIndex{Option}{footbotline~=\PName{thickness\textup{:}length}}%
The \KOMAScript{} classes provide only one separation line below the header
and another above the footer, and you can only switch these lines on or off.
But the \Package{scrlayer-scrpage} package also lets you place lines above the
header and below the footer. And for all four lines, you can not only switch
them on an off but also configure their \PName{length} and \PName{thickness}.

Both values are optional. If you omit the \PName{thickness}, a default value
of 0.4\Unit{pt} is used, producing a so-called \emph{hairline}. If you omit
the \PName{length}, the width of the header or footer will be used. If you
omit both, you can also omit the colon. If you use only one value without
colon, this is the \PName{thickness}.

Of course, the \PName{length} can be not just shorter than the current width
of the header or footer but also longer. See also the options 
\DescRef{\LabelBase.option.ilines}\IndexOption{ilines}%
\important{\DescRef{\LabelBase.option.ilines},
\DescRef{\LabelBase.option.clines}, \DescRef{\LabelBase.option.olines}},
\DescRef{\LabelBase.option.clines}\IndexOption{clines} and
\DescRef{\LabelBase.option.olines}\IndexOption{olines} later in this section.

\BeginIndexGroup
\BeginIndex{FontElement}{headtopline}\LabelFontElement{headtopline}%
\BeginIndex{FontElement}{headsepline}\LabelFontElement{headsepline}%
\BeginIndex{FontElement}{footsepline}\LabelFontElement{footsepline}%
\BeginIndex{FontElement}{footbotline}\LabelFontElement{footbotline}%
In addition to the length and thickness, you can also change the colour of the
lines. Initially the colour depends on the colour of the header or footer. In
addition to this, however, the settings of the corresponding elements
\important[i]{\FontElement{headtopline}\\
  \FontElement{headsepline}\\
  \FontElement{footsepline}\\
  \FontElement{footbotline}} \FontElement{headtopline},
\FontElement{headsepline}, \FontElement{footsepline} and
\FontElement{footbotline} are applied. You can
change these using the \DescRef{\LabelBase.cmd.setkomafont} or
\DescRef{\LabelBase.cmd.addtokomafont} commands (see
\autoref{sec:maincls.textmarkup}, \DescPageRef{maincls.cmd.setkomafont}).
By default these elements are empty, so they do not change the current font or
colour. Font changes at this point, unlike colour changes, make little sense
anyway and are therefore not recommended for these elements.%
\EndIndexGroup
%
\EndIndexGroup


\begin{Declaration}
  \OptionVName{plainheadtopline}{simple switch}%
  \OptionVName{plainheadsepline}{simple switch}%
  \OptionVName{plainfootsepline}{simple switch}%
  \OptionVName{plainfootbotline}{simple switch}
\end{Declaration}
You can use these options to apply the settings for the lines to the
\PageStyle{plain} page style. You can find the available values for
\PName{simple switch} in \autoref{tab:truefalseswitch} on
\autopageref{tab:truefalseswitch}. If one of these options is activated, the
\PageStyle{plain} page style will use the line settings given by the options
and commands described above. If the option is deactivated, the
\PageStyle{plain} will not show the corresponding line.%
\EndIndexGroup


\begin{Declaration}
  \Option{ilines}%
  \Option{clines}%
  \Option{olines}
\end{Declaration}
As previously explained, dividing lines for the header or footer can be longer
or shorter than the width of the header or footer respectively. But the
question remains how these lines are aligned. By default, all lines are
aligned to the left margin in one-sided printing and to the inner margin in
two-sided printing. This corresponds to using the \Option{ilines} option.
Alternatively, you can use the \Option{clines} option to centre the lines with
respect to the width of the header or footer, or the \Option{olines} option to
align them to the outer (or right) margin.%
\EndIndexGroup
%
\EndIndexGroup

%%% Local Variables:
%%% mode: latex
%%% mode: flyspell
%%% coding: us-ascii
%%% ispell-local-dictionary: "en_GB"
%%% TeX-master: "../guide"
%%% End: 
