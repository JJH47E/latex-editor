% \iffalse meta-comment
%
%% File: l3token.dtx
%
% Copyright (C) 2005-2022 The LaTeX Project
%
% It may be distributed and/or modified under the conditions of the
% LaTeX Project Public License (LPPL), either version 1.3c of this
% license or (at your option) any later version.  The latest version
% of this license is in the file
%
%    https://www.latex-project.org/lppl.txt
%
% This file is part of the "l3kernel bundle" (The Work in LPPL)
% and all files in that bundle must be distributed together.
%
% -----------------------------------------------------------------------
%
% The development version of the bundle can be found at
%
%    https://github.com/latex3/latex3
%
% for those people who are interested.
%
%<*driver>
\documentclass[full,kernel]{l3doc}
\begin{document}
  \DocInput{\jobname.dtx}
\end{document}
%</driver>
% \fi
%
% \title{^^A
%   The \pkg{l3token} package\\ Token manipulation^^A
% }
%
% \author{^^A
%  The \LaTeX{} Project\thanks
%    {^^A
%      E-mail:
%        \href{mailto:latex-team@latex-project.org}
%          {latex-team@latex-project.org}^^A
%    }^^A
% }
%
% \date{Released 2022-02-24}
%
% \maketitle
%
% \begin{documentation}
%
% This module deals with tokens. Now this is perhaps not the most
% precise description so let's try with a better description: When
% programming in \TeX{}, it is often desirable to know just what a
% certain token is: is it a control sequence or something
% else. Similarly one often needs to know if a control sequence is
% expandable or not, a macro or a primitive, how many arguments it
% takes etc. Another thing of great importance (especially when it
% comes to document commands) is looking ahead in the token stream to
% see if a certain character is present and maybe even remove it or
% disregard other tokens while scanning. This module provides
% functions for both and as such has two primary function
% categories: |\token_| for anything that deals with tokens and
% |\peek_| for looking ahead in the token stream.
%
% Most functions we describe here can be used on control sequences,
% as those are tokens as well.
%
% It is important to distinguish two aspects of a token: its
% \enquote{shape} (for lack of a better word), which affects the
% matching of delimited arguments and the comparison of token lists
% containing this token, and its \enquote{meaning}, which affects
% whether the token expands or what operation it performs.  One can have
% tokens of different shapes with the same meaning, but not the
% converse.
%
% For instance, \cs{if:w}, \cs{if_charcode:w}, and \cs[no-index]{tex_if:D} are
% three names for the same internal operation of \TeX{}, namely the
% primitive testing the next two characters for equality of their
% character code.  They have the same meaning hence behave identically
% in many situations.  However, \TeX{} distinguishes them when searching
% for a delimited argument.  Namely, the example function
% |\show_until_if:w| defined below takes everything until \cs{if:w}
% as an argument, despite the presence of other copies of \cs{if:w}
% under different names.
% \begin{verbatim}
% \cs_new:Npn \show_until_if:w #1 \if:w { \tl_show:n {#1} }
% \show_until_if:w \tex_if:D \if_charcode:w \if:w
% \end{verbatim}
% A list of all possible shapes and a list of all possible meanings are
% given in section~\ref{sec:l3token:all-tokens}.
%
% \section{Creating character tokens}
%
% \begin{function}[updated = 2015-11-12]
%   {
%     \char_set_active_eq:NN,  \char_set_active_eq:Nc,
%     \char_gset_active_eq:NN, \char_gset_active_eq:Nc
%   }
%   \begin{syntax}
%      \cs{char_set_active_eq:NN} \meta{char} \meta{function}
%   \end{syntax}
%   Sets the behaviour of the \meta{char} in situations where it is
%   active (category code $13$) to be equivalent to that of the
%   \meta{function}. The category code of the \meta{char} is
%   \emph{unchanged} by this process. The \meta{function} may itself
%   be an active character.
% \end{function}
%
% \begin{function}[added = 2015-11-12]
%   {
%     \char_set_active_eq:nN,  \char_set_active_eq:nc,
%     \char_gset_active_eq:nN, \char_gset_active_eq:nc
%   }
%   \begin{syntax}
%      \cs{char_set_active_eq:nN} \Arg{integer expression} \meta{function}
%   \end{syntax}
%   Sets the behaviour of the \meta{char} which has character
%   code as given by the \meta{integer expression} in situations
%   where it is active (category code $13$) to be equivalent to that of the
%   \meta{function}. The category code of the \meta{char} is
%   \emph{unchanged} by this process. The \meta{function} may itself
%   be an active character.
% \end{function}
%
% \begin{function}[EXP, added = 2015-09-09, updated = 2019-01-16]
%   {\char_generate:nn}
%   \begin{syntax}
%      \cs{char_generate:nn} \Arg{charcode} \Arg{catcode}
%   \end{syntax}
%   Generates a character token of the given \meta{charcode} and \meta{catcode}
%   (both of which may be integer expressions). The \meta{catcode} may be
%   one of
%   \begin{itemize}
%     \item $1$ (begin group)
%     \item $2$ (end group)
%     \item $3$ (math toggle)
%     \item $4$ (alignment)
%     \item $6$ (parameter)
%     \item $7$ (math superscript)
%     \item $8$ (math subscript)
%     \item $11$ (letter)
%     \item $12$ (other)
%     \item $13$ (active)
%   \end{itemize}
%   and other values raise an error. The \meta{charcode} may be any one valid
%   for the engine in use.
%   Active characters cannot be generated in older versions of \XeTeX{}.
%   Another way to build token lists with unusual category codes is
%   \cs{regex_replace:nnN} |{.*}| \Arg{replacement} \meta{tl~var}.
%   \begin{texnote}
%     Exactly two expansions are needed to produce the character.
%   \end{texnote}
% \end{function}
%
% \begin{function}[added = 2020-01-09, EXP]
%   {
%     \char_lowercase:N, \char_uppercase:N,
%     \char_titlecase:N, \char_foldcase:N,
%     \char_str_lowercase:N, \char_str_uppercase:N,
%     \char_str_titlecase:N, \char_str_foldcase:N
%   }
%   \begin{syntax}
%     \cs{char_lowercase:N} \meta{char}
%   \end{syntax}
%   Converts the \meta{char} to the equivalent case-changed character
%   as detailed by the function name (see \cs{str_foldcase:n}
%   and \cs{text_titlecase:n} for details of these terms). The case mapping
%   is carried out with no context-dependence (\emph{cf.}~\cs{text_uppercase:n},
%   \emph{etc.}) The \texttt{str} versions always generate \enquote{other}
%   (category code $12$) characters, whilst the standard versions generate
%   characters with the category code of the \meta{char} (i.e.~only
%   the character code changes).
% \end{function}
%
% \begin{variable}[added = 2011-09-05]{\c_catcode_other_space_tl}
%   Token list containing one character with category code $12$,
%   (\enquote{other}), and character code $32$ (space).
% \end{variable}
%
% \section{Manipulating and interrogating character tokens}
%
% \begin{function}[updated = 2015-11-11]
%   {
%     \char_set_catcode_escape:N           ,
%     \char_set_catcode_group_begin:N      ,
%     \char_set_catcode_group_end:N        ,
%     \char_set_catcode_math_toggle:N      ,
%     \char_set_catcode_alignment:N        ,
%     \char_set_catcode_end_line:N         ,
%     \char_set_catcode_parameter:N        ,
%     \char_set_catcode_math_superscript:N ,
%     \char_set_catcode_math_subscript:N   ,
%     \char_set_catcode_ignore:N           ,
%     \char_set_catcode_space:N            ,
%     \char_set_catcode_letter:N           ,
%     \char_set_catcode_other:N            ,
%     \char_set_catcode_active:N           ,
%     \char_set_catcode_comment:N          ,
%     \char_set_catcode_invalid:N
%   }
%   \begin{syntax}
%     \cs{char_set_catcode_letter:N} \meta{character}
%   \end{syntax}
%   Sets the category code of the \meta{character} to that indicated in
%   the function name. Depending on the current category code of the
%   \meta{token} the escape token may also be needed:
%   \begin{verbatim}
%     \char_set_catcode_other:N \%
%   \end{verbatim}
%   The assignment is local.
% \end{function}
%
% \begin{function}[updated = 2015-11-11]
%   {
%     \char_set_catcode_escape:n           ,
%     \char_set_catcode_group_begin:n      ,
%     \char_set_catcode_group_end:n        ,
%     \char_set_catcode_math_toggle:n       ,
%     \char_set_catcode_alignment:n        ,
%     \char_set_catcode_end_line:n         ,
%     \char_set_catcode_parameter:n        ,
%     \char_set_catcode_math_superscript:n ,
%     \char_set_catcode_math_subscript:n   ,
%     \char_set_catcode_ignore:n           ,
%     \char_set_catcode_space:n            ,
%     \char_set_catcode_letter:n           ,
%     \char_set_catcode_other:n            ,
%     \char_set_catcode_active:n           ,
%     \char_set_catcode_comment:n          ,
%     \char_set_catcode_invalid:n
%   }
%   \begin{syntax}
%     \cs{char_set_catcode_letter:n} \Arg{integer expression}
%   \end{syntax}
%   Sets the category code of the \meta{character} which has character
%   code as given by the \meta{integer expression}. This version can be
%   used to set up characters which cannot otherwise be given
%   (\emph{cf.}~the \texttt{N}-type variants). The assignment is local.
% \end{function}
%
% \begin{function}[updated = 2015-11-11]{\char_set_catcode:nn}
%   \begin{syntax}
%     \cs{char_set_catcode:nn} \Arg{intexpr_1} \Arg{intexpr_2}
%   \end{syntax}
%   These functions set the category code of the \meta{character} which
%   has character code as given by the \meta{integer expression}.
%   The first \meta{integer expression}
%   is the character code and the second is the category code to apply.
%   The setting applies within the current \TeX{} group. In general, the
%   symbolic functions \cs[no-index]{char_set_catcode_\meta{type}} should be preferred,
%   but there are cases where these lower-level functions may be useful.
% \end{function}
%
% \begin{function}[EXP]{\char_value_catcode:n}
%   \begin{syntax}
%     \cs{char_value_catcode:n} \Arg{integer expression}
%   \end{syntax}
%   Expands to the current category code of the \meta{character} with
%   character code given by the
%   \meta{integer expression}.
% \end{function}
%
% \begin{function}{\char_show_value_catcode:n}
%   \begin{syntax}
%     \cs{char_show_value_catcode:n} \Arg{integer expression}
%   \end{syntax}
%   Displays the current category code of the \meta{character} with
%   character code given by the  \meta{integer expression} on the
%   terminal.
% \end{function}
%
% \begin{function}[updated = 2015-08-06]{\char_set_lccode:nn}
%   \begin{syntax}
%     \cs{char_set_lccode:nn} \Arg{intexpr_1} \Arg{intexpr_2}
%   \end{syntax}
%   Sets up the behaviour of the \meta{character} when
%   found inside \cs{text_lowercase:n}, such that \meta{character_1}
%   will be converted into \meta{character_2}. The two \meta{characters}
%   may be specified using an \meta{integer expression} for the character code
%   concerned. This may include the \TeX{} |`|\meta{character}
%   method for converting a single character into its character
%   code:
%   \begin{verbatim}
%     \char_set_lccode:nn { `\A } { `\a } % Standard behaviour
%     \char_set_lccode:nn { `\A } { `\A + 32 }
%     \char_set_lccode:nn { 50 } { 60 }
%   \end{verbatim}
%   The setting applies within the current \TeX{} group.
% \end{function}
%
% \begin{function}[EXP]{\char_value_lccode:n}
%   \begin{syntax}
%     \cs{char_value_lccode:n} \Arg{integer expression}
%   \end{syntax}
%   Expands to the current lower case code of the \meta{character} with
%   character code given by the
%   \meta{integer expression}.
% \end{function}
%
% \begin{function}{\char_show_value_lccode:n}
%   \begin{syntax}
%     \cs{char_show_value_lccode:n} \Arg{integer expression}
%   \end{syntax}
%   Displays the current lower case code of the \meta{character} with
%   character code given by the  \meta{integer expression} on the
%   terminal.
% \end{function}
%
% \begin{function}[updated = 2015-08-06]{\char_set_uccode:nn}
%   \begin{syntax}
%     \cs{char_set_uccode:nn} \Arg{intexpr_1} \Arg{intexpr_2}
%   \end{syntax}
%   Sets up the behaviour of the \meta{character} when
%   found inside \cs{text_uppercase:n}, such that \meta{character_1}
%   will be converted into \meta{character_2}. The two \meta{characters}
%   may be specified using an \meta{integer expression} for the character code
%   concerned. This may include the \TeX{} |`|\meta{character}
%   method for converting a single character into its character
%   code:
%   \begin{verbatim}
%     \char_set_uccode:nn { `\a } { `\A } % Standard behaviour
%     \char_set_uccode:nn { `\A } { `\A - 32 }
%     \char_set_uccode:nn { 60 } { 50 }
%   \end{verbatim}
%   The setting applies within the current \TeX{} group.
% \end{function}
%
% \begin{function}[EXP]{\char_value_uccode:n}
%   \begin{syntax}
%     \cs{char_value_uccode:n} \Arg{integer expression}
%   \end{syntax}
%   Expands to the current upper case code of the \meta{character} with
%   character code given by the
%   \meta{integer expression}.
% \end{function}
%
% \begin{function}{\char_show_value_uccode:n}
%   \begin{syntax}
%     \cs{char_show_value_uccode:n} \Arg{integer expression}
%   \end{syntax}
%   Displays the current upper case code of the \meta{character} with
%   character code given by the  \meta{integer expression} on the
%   terminal.
% \end{function}
%
% \begin{function}[updated = 2015-08-06]{\char_set_mathcode:nn}
%   \begin{syntax}
%     \cs{char_set_mathcode:nn} \Arg{intexpr_1} \Arg{intexpr_2}
%   \end{syntax}
%   This function sets up the math code of \meta{character}.
%   The \meta{character} is specified as
%   an \meta{integer expression} which will be used as the character
%   code of the relevant character. The setting applies within the
%   current \TeX{} group.
% \end{function}
%
% \begin{function}[EXP]{\char_value_mathcode:n}
%   \begin{syntax}
%     \cs{char_value_mathcode:n} \Arg{integer expression}
%   \end{syntax}
%   Expands to the current math code of the \meta{character} with
%   character code given by the
%   \meta{integer expression}.
% \end{function}
%
% \begin{function}{\char_show_value_mathcode:n}
%   \begin{syntax}
%     \cs{char_show_value_mathcode:n} \Arg{integer expression}
%   \end{syntax}
%   Displays the current math code of the \meta{character} with
%   character code given by the  \meta{integer expression} on the
%   terminal.
% \end{function}
%
% \begin{function}[updated = 2015-08-06]{\char_set_sfcode:nn}
%   \begin{syntax}
%     \cs{char_set_sfcode:nn} \Arg{intexpr_1} \Arg{intexpr_2}
%   \end{syntax}
%   This function sets up the space factor for the \meta{character}.
%   The \meta{character} is specified as
%   an \meta{integer expression} which will be used as the character
%   code of the relevant character. The setting applies within the
%   current \TeX{} group.
% \end{function}
%
% \begin{function}[EXP]{\char_value_sfcode:n}
%   \begin{syntax}
%     \cs{char_value_sfcode:n} \Arg{integer expression}
%   \end{syntax}
%   Expands to the current space factor for the \meta{character} with
%   character code given by the
%   \meta{integer expression}.
% \end{function}
%
% \begin{function}{\char_show_value_sfcode:n}
%   \begin{syntax}
%     \cs{char_show_value_sfcode:n} \Arg{integer expression}
%   \end{syntax}
%   Displays the current space factor for the \meta{character} with
%   character code given by the  \meta{integer expression} on the
%   terminal.
% \end{function}
%
% \begin{variable}[added = 2012-01-23, updated = 2015-11-11]{\l_char_active_seq}
%   Used to track which tokens may require special handling at the document
%   level as they are (or have been at some point)
%   of category \meta{active} (catcode~$13$). Each entry in
%   the sequence consists of a single escaped token, for example |\~|.
%   Active tokens should be added to the sequence when they are defined for
%   general document use.
% \end{variable}
%
% \begin{variable}[added = 2012-01-23, updated = 2015-11-11]{\l_char_special_seq}
%   Used to track which tokens will require special handling when working with
%   verbatim-like material at the document level as they are not of categories
%   \meta{letter} (catcode~$11$) or \meta{other} (catcode~$12$). Each entry in
%   the sequence consists of a single escaped token, for example |\\| for the
%   backslash or |\{| for an opening brace. ^^A \}
%   Escaped tokens should be added to the sequence when they are defined for
%   general document use.
% \end{variable}
%
% \section{Generic tokens}
%
% \begin{variable}[module = token]
%   {
%     \c_group_begin_token,
%     \c_group_end_token,
%     \c_math_toggle_token,
%     \c_alignment_token,
%     \c_parameter_token,
%     \c_math_superscript_token,
%     \c_math_subscript_token,
%     \c_space_token
%   }
%   These are implicit tokens which have the category code described
%   by their name. They are used internally for test purposes but
%   are also available to the programmer for other uses.
% \end{variable}
%
% \begin{variable}
%   {
%     \c_catcode_letter_token,
%     \c_catcode_other_token
%   }
%   These are implicit tokens which have the category code described
%   by their name. They are used internally for test purposes and should
%   not be used other than for category code tests.
% \end{variable}
%
% \begin{variable}{\c_catcode_active_tl}
%   A token list containing an active token. This is used internally
%   for test purposes and should not be used other than in
%   appropriately-constructed category code tests.
% \end{variable}
%
% \section{Converting tokens}
%
% \begin{function}[EXP]{\token_to_meaning:N, \token_to_meaning:c}
%   \begin{syntax}
%     \cs{token_to_meaning:N} \meta{token}
%   \end{syntax}
%   Inserts the current meaning of the \meta{token} into the input
%   stream as a series of characters of category code $12$ (other).
%   This is the primitive \TeX{} description of the \meta{token},
%   thus for example both functions defined by \cs{cs_set_nopar:Npn}
%   and token list variables defined using \cs{tl_new:N} are described
%   as |macro|s.
%   \begin{texnote}
%     This is the \TeX{} primitive \tn{meaning}.
%     The \meta{token} can thus be an explicit space tokens or an
%     explicit begin-group or end-group character token (|{|~or~|}| when
%     normal \TeX{} category codes apply) even though these are not
%     valid \texttt{N}-type arguments.
%   \end{texnote}
% \end{function}
%
% \begin{function}[EXP]{\token_to_str:N, \token_to_str:c}
%   \begin{syntax}
%     \cs{token_to_str:N} \meta{token}
%   \end{syntax}
%   Converts the given \meta{token} into a series of characters with
%   category code $12$ (other). If the \meta{token} is a control
%   sequence, this will start with the current escape character with
%   category code $12$ (the escape character is part of the
%   \meta{token}). This function requires only a single expansion.
%   \begin{texnote}
%     \cs{token_to_str:N} is the \TeX{} primitive \tn{string} renamed.
%     The \meta{token} can thus be an explicit space tokens or an
%     explicit begin-group or end-group character token (|{|~or~|}| when
%     normal \TeX{} category codes apply) even though these are not
%     valid \texttt{N}-type arguments.
%   \end{texnote}
% \end{function}
%
% \section{Token conditionals}
%
% \begin{function}[EXP,pTF]{\token_if_group_begin:N}
%   \begin{syntax}
%     \cs{token_if_group_begin_p:N} \meta{token} \\
%     \cs{token_if_group_begin:NTF} \meta{token} \Arg{true code} \Arg{false code}
%   \end{syntax}
%   Tests if \meta{token} has the category code of a begin group token
%   (|{| when normal \TeX{} category codes are in ^^A }
%   force).
%   Note that an explicit begin group token cannot be tested in this way,
%   as it is not a valid \texttt{N}-type argument.
% \end{function}
%
% \begin{function}[EXP,pTF]{\token_if_group_end:N}
%   \begin{syntax}
%     \cs{token_if_group_end_p:N} \meta{token} \\
%     \cs{token_if_group_end:NTF} \meta{token} \Arg{true code} \Arg{false code}
%    \end{syntax}
%   Tests if \meta{token} has the category code of an end group token
%   (^^A {
%   |}| when normal \TeX{} category codes are in force).
%   Note that an explicit end group token cannot be tested in this way,
%   as it is not a valid \texttt{N}-type argument.
% \end{function}
%
% \begin{function}[EXP,pTF]{\token_if_math_toggle:N}
%   \begin{syntax}
%     \cs{token_if_math_toggle_p:N} \meta{token} \\
%     \cs{token_if_math_toggle:NTF} \meta{token} \Arg{true code} \Arg{false code}
%   \end{syntax}
%   Tests if \meta{token} has the category code of a math shift token
%   (|$| when normal \TeX{} category codes are in force).
% \end{function}
%
% \begin{function}[EXP,pTF]{\token_if_alignment:N}
%   \begin{syntax}
%     \cs{token_if_alignment_p:N} \meta{token} \\
%     \cs{token_if_alignment:NTF} \meta{token} \Arg{true code} \Arg{false code}
%   \end{syntax}
%   Tests if \meta{token} has the category code of an alignment token
%   (|&| when normal \TeX{} category codes are in force).
% \end{function}
%
% \begin{function}[EXP,pTF]{\token_if_parameter:N}
%   \begin{syntax}
%     \cs{token_if_parameter_p:N} \meta{token} \\
%     \cs{token_if_parameter:NTF} \meta{token} \Arg{true code} \Arg{false code}
%   \end{syntax}
%   Tests if \meta{token} has the category code of a macro parameter token
%   (|#| when normal \TeX{} category codes are in force).
% \end{function}
%
% \begin{function}[EXP,pTF]{\token_if_math_superscript:N}
%   \begin{syntax}
%     \cs{token_if_math_superscript_p:N} \meta{token} \\
%     \cs{token_if_math_superscript:NTF} \meta{token} \Arg{true code} \Arg{false code}
%   \end{syntax}
%   Tests if \meta{token} has the category code of a superscript token
%   (|^| when normal \TeX{} category codes are in force).
% \end{function}
%
% \begin{function}[EXP,pTF]{\token_if_math_subscript:N}
%   \begin{syntax}
%     \cs{token_if_math_subscript_p:N} \meta{token} \\
%     \cs{token_if_math_subscript:NTF} \meta{token} \Arg{true code} \Arg{false code}
%   \end{syntax}
%   Tests if \meta{token} has the category code of a subscript token
%   (|_| when normal \TeX{} category codes are in force).
% \end{function}
%
% \begin{function}[EXP,pTF]{\token_if_space:N}
%   \begin{syntax}
%     \cs{token_if_space_p:N} \meta{token} \\
%     \cs{token_if_space:NTF} \meta{token} \Arg{true code} \Arg{false code}
%   \end{syntax}
%   Tests if \meta{token} has the category code of a space token.
%   Note that an explicit space token with character code $32$ cannot
%   be tested in this way, as it is not a valid \texttt{N}-type argument.
% \end{function}
%
% \begin{function}[EXP,pTF]{\token_if_letter:N}
%   \begin{syntax}
%     \cs{token_if_letter_p:N} \meta{token} \\
%     \cs{token_if_letter:NTF} \meta{token} \Arg{true code} \Arg{false code}
%   \end{syntax}
%   Tests if \meta{token} has the category code of a letter token.
% \end{function}
%
% \begin{function}[EXP,pTF]{\token_if_other:N}
%   \begin{syntax}
%     \cs{token_if_other_p:N} \meta{token} \\
%     \cs{token_if_other:NTF} \meta{token} \Arg{true code} \Arg{false code}
%   \end{syntax}
%   Tests if \meta{token} has the category code of an \enquote{other}
%   token.
% \end{function}
%
% \begin{function}[EXP,pTF]{\token_if_active:N}
%   \begin{syntax}
%     \cs{token_if_active_p:N} \meta{token} \\
%     \cs{token_if_active:NTF} \meta{token} \Arg{true code} \Arg{false code}
%   \end{syntax}
%   Tests if \meta{token} has the category code of an active character.
% \end{function}
%
% \begin{function}[EXP,pTF]{\token_if_eq_catcode:NN}
%   \begin{syntax}
%     \cs{token_if_eq_catcode_p:NN} \meta{token_1} \meta{token_2} \\
%     \cs{token_if_eq_catcode:NNTF} \meta{token_1} \meta{token_2} \Arg{true code} \Arg{false code}
%   \end{syntax}
%   Tests if the two \meta{tokens} have the same category code.
% \end{function}
%
% \begin{function}[EXP,pTF]{\token_if_eq_charcode:NN}
%   \begin{syntax}
%     \cs{token_if_eq_charcode_p:NN} \meta{token_1} \meta{token_2} \\
%     \cs{token_if_eq_charcode:NNTF} \meta{token_1} \meta{token_2} \Arg{true code} \Arg{false code}
%   \end{syntax}
%   Tests if the two \meta{tokens} have the same character code.
% \end{function}
%
% \begin{function}[EXP,pTF]{\token_if_eq_meaning:NN}
%   \begin{syntax}
%     \cs{token_if_eq_meaning_p:NN} \meta{token_1} \meta{token_2} \\
%     \cs{token_if_eq_meaning:NNTF} \meta{token_1} \meta{token_2} \Arg{true code} \Arg{false code}
%   \end{syntax}
%   Tests if the two \meta{tokens} have the same meaning when expanded.
% \end{function}
%
% \begin{function}[updated = 2011-05-23, EXP,pTF]{\token_if_macro:N}
%   \begin{syntax}
%     \cs{token_if_macro_p:N} \meta{token} \\
%     \cs{token_if_macro:NTF} \meta{token} \Arg{true code} \Arg{false code}
%   \end{syntax}
%   Tests if the \meta{token} is a \TeX{} macro.
% \end{function}
%
% \begin{function}[EXP,pTF]{\token_if_cs:N}
%   \begin{syntax}
%     \cs{token_if_cs_p:N} \meta{token} \\
%     \cs{token_if_cs:NTF} \meta{token} \Arg{true code} \Arg{false code}
%   \end{syntax}
%   Tests if the \meta{token} is a control sequence.
% \end{function}
%
% \begin{function}[EXP,pTF]{\token_if_expandable:N}
%   \begin{syntax}
%     \cs{token_if_expandable_p:N} \meta{token} \\
%     \cs{token_if_expandable:NTF} \meta{token} \Arg{true code} \Arg{false code}
%   \end{syntax}
%   Tests if the \meta{token} is expandable. This test returns \meta{false}
%   for an undefined token.
% \end{function}
%
% \begin{function}[EXP,pTF, updated=2012-01-20]{\token_if_long_macro:N}
%   \begin{syntax}
%     \cs{token_if_long_macro_p:N} \meta{token} \\
%     \cs{token_if_long_macro:NTF} \meta{token} \Arg{true code} \Arg{false code}
%   \end{syntax}
%   Tests if the \meta{token} is a long macro.
% \end{function}
%
% \begin{function}[EXP,pTF, updated=2012-01-20]{\token_if_protected_macro:N}
%   \begin{syntax}
%     \cs{token_if_protected_macro_p:N} \meta{token} \\
%     \cs{token_if_protected_macro:NTF} \meta{token} \Arg{true code} \Arg{false code}
%   \end{syntax}
%   Tests if the \meta{token} is a protected macro: for a macro which
%   is both protected and long this returns \texttt{false}.
% \end{function}
%
% \begin{function}[EXP,pTF, updated=2012-01-20]{\token_if_protected_long_macro:N}
%   \begin{syntax}
%     \cs{token_if_protected_long_macro_p:N} \meta{token} \\
%     \cs{token_if_protected_long_macro:NTF} \meta{token} \Arg{true code} \Arg{false code}
%   \end{syntax}
%   Tests if the \meta{token} is a protected long macro.
% \end{function}
%
% \begin{function}[EXP,pTF, updated=2012-01-20]{\token_if_chardef:N}
%   \begin{syntax}
%     \cs{token_if_chardef_p:N} \meta{token} \\
%     \cs{token_if_chardef:NTF} \meta{token} \Arg{true code} \Arg{false code}
%   \end{syntax}
%   Tests if the \meta{token} is defined to be a chardef.
%   \begin{texnote}
%     Booleans, boxes and small integer constants are implemented as
%     \tn{chardef}s.
%   \end{texnote}
% \end{function}
%
% \begin{function}[EXP,pTF, updated=2012-01-20]{\token_if_mathchardef:N}
%   \begin{syntax}
%     \cs{token_if_mathchardef_p:N} \meta{token} \\
%     \cs{token_if_mathchardef:NTF} \meta{token} \Arg{true code} \Arg{false code}
%   \end{syntax}
%   Tests if the \meta{token} is defined to be a mathchardef.
% \end{function}
%
% \begin{function}[EXP,pTF, added=2020-10-27]{\token_if_font_selection:N}
%   \begin{syntax}
%     \cs{token_if_font_selection_p:N} \meta{token} \\
%     \cs{token_if_font_selection:NTF} \meta{token} \Arg{true code} \Arg{false code}
%   \end{syntax}
%   Tests if the \meta{token} is defined to be a font selection command.
% \end{function}
%
% \begin{function}[EXP,pTF, updated=2012-01-20]{\token_if_dim_register:N}
%   \begin{syntax}
%     \cs{token_if_dim_register_p:N} \meta{token} \\
%     \cs{token_if_dim_register:NTF} \meta{token} \Arg{true code} \Arg{false code}
%   \end{syntax}
%   Tests if the \meta{token} is defined to be a dimension register.
% \end{function}
%
% \begin{function}[EXP,pTF, updated=2012-01-20]{\token_if_int_register:N}
%   \begin{syntax}
%     \cs{token_if_int_register_p:N} \meta{token} \\
%     \cs{token_if_int_register:NTF} \meta{token} \Arg{true code} \Arg{false code}
%   \end{syntax}
%   Tests if the \meta{token} is defined to be a integer register.
%   \begin{texnote}
%     Constant integers may be implemented as integer registers,
%     \tn{chardef}s, or \tn{mathchardef}s depending on their value.
%   \end{texnote}
% \end{function}
%
% \begin{function}[EXP,pTF, added=2012-02-15]{\token_if_muskip_register:N}
%   \begin{syntax}
%     \cs{token_if_muskip_register_p:N} \meta{token} \\
%     \cs{token_if_muskip_register:NTF} \meta{token} \Arg{true code} \Arg{false code}
%   \end{syntax}
%   Tests if the \meta{token} is defined to be a muskip register.
% \end{function}
%
% \begin{function}[EXP,pTF, updated=2012-01-20]{\token_if_skip_register:N}
%   \begin{syntax}
%     \cs{token_if_skip_register_p:N} \meta{token} \\
%     \cs{token_if_skip_register:NTF} \meta{token} \Arg{true code} \Arg{false code}
%   \end{syntax}
%   Tests if the \meta{token} is defined to be a skip register.
% \end{function}
%
% \begin{function}[EXP,pTF, updated=2012-01-20]{\token_if_toks_register:N}
%   \begin{syntax}
%     \cs{token_if_toks_register_p:N} \meta{token} \\
%     \cs{token_if_toks_register:NTF} \meta{token} \Arg{true code} \Arg{false code}
%   \end{syntax}
%   Tests if the \meta{token} is defined to be a toks register
%   (not used by \LaTeX3).
% \end{function}
%
% \begin{function}[updated = 2020-09-11, EXP,pTF]{\token_if_primitive:N}
%   \begin{syntax}
%     \cs{token_if_primitive_p:N} \meta{token} \\
%     \cs{token_if_primitive:NTF} \meta{token} \Arg{true code} \Arg{false code}
%   \end{syntax}
%   Tests if the \meta{token} is an engine primitive. In \LuaTeX{}
%   this includes primitive-like commands defined using |{token.set_lua}|.
% \end{function}
%
% \begin{function}[added = 2020-12-03, EXP, noTF]
%   {\token_case_catcode:Nn, \token_case_charcode:Nn, \token_case_meaning:Nn}
%   \begin{syntax}
%     \cs{token_case_meaning:NnTF} \meta{test token} \\
%     ~~"{" \\
%     ~~~~\meta{token case_1} \Arg{code case_1} \\
%     ~~~~\meta{token case_2} \Arg{code case_2} \\
%     ~~~~\ldots \\
%     ~~~~\meta{token case_n} \Arg{code case_n} \\
%     ~~"}" \\
%     ~~\Arg{true code}
%     ~~\Arg{false code}
%   \end{syntax}
%   This function compares the \meta{test token} in turn with each of
%   the \meta{token cases}. If the two are equal (as described for
%   \cs{token_if_eq_catcode:NNTF}, \cs{token_if_eq_charcode:NNTF} and
%   \cs{token_if_eq_meaning:NNTF}, respectively) then the associated
%   \meta{code} is left in the input stream and other cases are
%   discarded. If any of the cases are matched, the \meta{true code} is
%   also inserted into the input stream (after the code for the
%   appropriate case), while if none match then the \meta{false code} is
%   inserted. The functions \cs{token_case_catcode:Nn},
%   \cs{token_case_charcode:Nn}, and \cs{token_case_meaning:Nn}, which
%   do nothing if there is no match, are also available.
% \end{function}
%
% \section{Peeking ahead at the next token}
%
% There is often a need to look ahead at the next token in the input
% stream while leaving it in place. This is handled using the
% \enquote{peek} functions. The generic \cs{peek_after:Nw} is provided
% along with a family of predefined tests for common cases.  As peeking
% ahead does \emph{not} skip spaces the predefined tests include both a
% space-respecting and space-skipping version.  In addition, using
% \cs{peek_analysis_map_inline:n}, one can map through the following
% tokens in the input stream and repeatedly perform some tests.
%
% \begin{function}{\peek_after:Nw}
%   \begin{syntax}
%     \cs{peek_after:Nw} \meta{function} \meta{token}
%   \end{syntax}
%   Locally sets the test variable \cs{l_peek_token} equal to \meta{token}
%   (as an implicit token, \emph{not} as a token list), and then
%   expands the \meta{function}. The \meta{token} remains in
%   the input stream as the next item after the \meta{function}.
%   The \meta{token} here may be \verb*| |, |{| or |}| (assuming
%   normal \TeX{} category codes), \emph{i.e.}~it is not necessarily the
%   next argument which would be grabbed by a normal function.
% \end{function}
%
% \begin{function}{\peek_gafter:Nw}
%   \begin{syntax}
%     \cs{peek_gafter:Nw} \meta{function} \meta{token}
%   \end{syntax}
%   Globally sets the test variable \cs{g_peek_token} equal to \meta{token}
%   (as an implicit token, \emph{not} as a token list), and then
%   expands the \meta{function}. The \meta{token} remains in
%   the input stream as the next item after the \meta{function}.
%   The \meta{token} here may be \verb*| |, |{| or |}| (assuming
%   normal \TeX{} category codes), \emph{i.e.}~it is not necessarily the
%   next argument which would be grabbed by a normal function.
% \end{function}
%
% \begin{variable}{\l_peek_token}
%  Token set by \cs{peek_after:Nw} and available for testing
%  as described above.
% \end{variable}
%
% \begin{variable}{\g_peek_token}
%  Token set by \cs{peek_gafter:Nw} and available for testing
%  as described above.
% \end{variable}
%
% \begin{function}[updated = 2012-12-20, TF]{\peek_catcode:N}
%   \begin{syntax}
%     \cs{peek_catcode:NTF} \meta{test token} \Arg{true code} \Arg{false code}
%   \end{syntax}
%   Tests if the next \meta{token} in the input stream has the same
%   category code as the \meta{test token} (as defined by the test
%   \cs{token_if_eq_catcode:NNTF}). Spaces are respected by the test
%   and the \meta{token} is left in the input stream after
%   the \meta{true code} or \meta{false code} (as appropriate to the
%   result of the test).
% \end{function}
%
% \begin{function}[updated = 2012-12-20, TF]{\peek_catcode_remove:N}
%   \begin{syntax}
%     \cs{peek_catcode_remove:NTF} \meta{test token} \Arg{true code} \Arg{false code}
%   \end{syntax}
%   Tests if the next \meta{token} in the input stream has the same
%   category code as the \meta{test token} (as defined by the test
%   \cs{token_if_eq_catcode:NNTF}). Spaces are respected by the test
%   and the \meta{token} is removed from the input stream if the
%   test is true. The function then places either the
%   \meta{true code} or \meta{false code} in the input stream (as
%   appropriate to the result of the test).
% \end{function}
%
% \begin{function}[updated = 2012-12-20, TF]{\peek_charcode:N}
%   \begin{syntax}
%     \cs{peek_charcode:NTF} \meta{test token} \Arg{true code} \Arg{false code}
%   \end{syntax}
%   Tests if the next \meta{token} in the input stream has the same
%   character code as the \meta{test token} (as defined by the test
%   \cs{token_if_eq_charcode:NNTF}). Spaces are respected by the test
%   and the \meta{token} is left in the input stream after
%   the \meta{true code} or \meta{false code} (as appropriate to the
%   result of the test).
% \end{function}
%
% \begin{function}[updated = 2012-12-20, TF]{\peek_charcode_remove:N}
%   \begin{syntax}
%     \cs{peek_charcode_remove:NTF} \meta{test token} \Arg{true code} \Arg{false code}
%   \end{syntax}
%   Tests if the next \meta{token} in the input stream has the same
%   character code as the \meta{test token} (as defined by the test
%   \cs{token_if_eq_charcode:NNTF}). Spaces are respected by the test
%   and the \meta{token} is removed from the input stream if the
%   test is true. The function then places either the
%   \meta{true code} or \meta{false code} in the input stream (as
%   appropriate to the result of the test).
% \end{function}
%
% \begin{function}[updated = 2011-07-02, TF]{\peek_meaning:N}
%   \begin{syntax}
%     \cs{peek_meaning:NTF} \meta{test token} \Arg{true code} \Arg{false code}
%   \end{syntax}
%   Tests if the next \meta{token} in the input stream has the same
%   meaning as the \meta{test token} (as defined by the test
%   \cs{token_if_eq_meaning:NNTF}). Spaces are respected by the test
%   and the \meta{token} is left in the input stream after
%   the \meta{true code} or \meta{false code} (as appropriate to the
%   result of the test).
% \end{function}
%
% \begin{function}[updated = 2011-07-02, TF]{\peek_meaning_remove:N}
%   \begin{syntax}
%     \cs{peek_meaning_remove:NTF} \meta{test token} \Arg{true code} \Arg{false code}
%   \end{syntax}
%   Tests if the next \meta{token} in the input stream has the same
%   meaning as the \meta{test token} (as defined by the test
%   \cs{token_if_eq_meaning:NNTF}). Spaces are respected by the test
%   and the \meta{token} is removed from the input stream if the
%   test is true. The function then places either the
%   \meta{true code} or \meta{false code} in the input stream (as
%   appropriate to the result of the test).
% \end{function}
%
% \begin{function}[added = 2018-10-01]{\peek_remove_spaces:n}
%   \begin{syntax}
%     \cs{peek_remove_spaces:n} \Arg{code}
%   \end{syntax}
%   Peeks ahead and detect if the following token is a space (category code
%   $10$ and character code $32$). If so, removes the token and checks the
%   next token. Once a non-space token is found, the \meta{code} will be
%   inserted into the input stream. Typically this will contain a \texttt{peek}
%   operation, but this is not required.
% \end{function}
%
% \begin{function}[added = 2022-01-10]{\peek_remove_filler:n}
%   \begin{syntax}
%     \cs{peek_remove_filler:n} \Arg{code}
%   \end{syntax}
%   Peeks ahead and detect if the following token is a space (category code
%   $10$) or has meaning equal to \cs{scan_stop:}. If so,
%   removes the token and checks the next token. If neither of these cases
%   apply, expands the next token using \texttt{f}-type expansion, then checks
%   the resulting leading token in the same way. If after expansion the next
%   token is neither of the two test cases, the \meta{code} will be inserted
%   into the input stream. Typically this will contain a \texttt{peek}
%   operation, but this is not required.
%   \begin{texnote}
%     This is essentially a macro-based implementation of how \TeX{} handles
%     the search for a left brace after for example \tn{everypar}, except that
%     any non-expandable token cleanly ends the \meta{filler} (i.e.~it does not
%     lead to a \TeX{} error).
%
%     In contrast to \TeX{}'s filler removal, a construct
%     \verb|\exp_not:N \foo| will be treated in the same way as \verb|\foo|.
%   \end{texnote}
% \end{function}
%
% \begin{function}[TF, updated = 2012-12-20]{\peek_N_type:}
%   \begin{syntax}
%     \cs{peek_N_type:TF} \Arg{true code} \Arg{false code}
%   \end{syntax}
%   Tests if the next \meta{token} in the input stream can be safely
%   grabbed as an \texttt{N}-type argument. The test is \meta{false}
%   if the next \meta{token} is either an explicit or implicit
%   begin-group or end-group token (with any character code), or
%   an explicit or implicit space character (with character code $32$
%   and category code $10$), or an outer token (never used in \LaTeX3)
%   and \meta{true} in all other cases.
%   Note that a \meta{true} result ensures that the next \meta{token} is
%   a valid \texttt{N}-type argument. However, if the next \meta{token}
%   is for instance \cs{c_space_token}, the test takes the
%   \meta{false} branch, even though the next \meta{token} is in fact
%   a valid \texttt{N}-type argument. The \meta{token} is left
%   in the input stream after the \meta{true code} or \meta{false code}
%   (as appropriate to the result of the test).
% \end{function}
%
% \begin{function}[added = 2020-12-03]{\peek_analysis_map_inline:n}
%   \begin{syntax}
%     \cs{peek_analysis_map_inline:n} \Arg{inline function}
%   \end{syntax}
%   Repeatedly removes one \meta{token} from the input stream and
%   applies the \meta{inline function} to it, until
%   \cs{peek_analysis_map_break:} is called.  The \meta{inline function}
%   receives three arguments for each \meta{token} in the input stream:
%   \begin{itemize}
%   \item \meta{tokens}, which both \texttt{o}-expand and
%     \texttt{x}-expand to the \meta{token}. The detailed form of
%     \meta{tokens} may change in later releases.
%   \item \meta{char code}, a decimal representation of the character
%     code of the \meta{token}, $-1$ if it is a control sequence.
%   \item \meta{catcode}, a capital hexadecimal digit which denotes the
%     category code of the \meta{token} (0:~control sequence,
%     1:~begin-group, 2:~end-group, 3:~math shift, 4:~alignment tab,
%     6:~parameter, 7:~superscript, 8:~subscript, A:~space, B:~letter,
%     C:~other, D:~active).  This can be converted to an integer by
%     writing |"|\meta{catcode}.
%   \end{itemize}
%   These arguments are the same as for \cs{tl_analysis_map_inline:nn}
%   defined in \pkg{l3tl-analysis}.  The \meta{char code} and
%   \meta{catcode} do not take the meaning of a control sequence or
%   active character into account: for instance, upon encountering the
%   token \cs{c_group_begin_token} in the input stream,
%   \cs{peek_analysis_map_inline:n} calls the \meta{inline function}
%   with |#1| being \cs{exp_not:n} |{| \cs{c_group_begin_token} |}|
%   (with the current implementation),
%   |#2|~being~$-1$, and
%   |#3|~being~$0$, as for any other control sequence.  In contrast,
%   upon encountering an explicit begin-group token~|{|, % ^^A |}|
%   the \meta{inline function} is called with arguments
%   \cs{exp_after:wN} |{| \cs{if_false:} |}| \cs{fi:}, $123$ and~$1$.
%
%   The mapping is done at the current group level, \emph{i.e.}~any
%   local assignments made by the \meta{inline function} remain in
%   effect after the loop.  Within the code, \cs{l_peek_token} is set
%   equal (as a token, not a token list) to the token under
%   consideration.
% \end{function}
%
% \begin{function}[added = 2020-12-03]
%   {\peek_analysis_map_break:, \peek_analysis_map_break:n}
%   \begin{syntax}
%     \cs{peek_analysis_map_inline:n}
%       |{| \dots{} \cs{peek_analysis_map_break:n} \Arg{code} |}|
%   \end{syntax}
%   Stops the \cs{peek_analysis_map_inline:n} loop from seeking more
%   tokens, and inserts \meta{code} in the input stream (empty for
%   \cs{peek_analysis_map_break:}).
% \end{function}
%
% \begin{function}[added = 2020-12-03, TF]{\peek_regex:n, \peek_regex:N}
%   \begin{syntax}
%     \cs{peek_regex:nTF} \Arg{regex} \Arg{true code} \Arg{false code}
%   \end{syntax}
%   Tests if the \meta{tokens} that follow in the input stream match the
%   \meta{regular expression}.  Any \meta{tokens} that have been read
%   are left in the input stream after the \meta{true code} or
%   \meta{false code} (as appropriate to the result of the test).  See
%   \pkg{l3regex} for documentation of the syntax of regular
%   expressions.  The \meta{regular expression} is implicitly anchored
%   at the start, so for instance \cs{peek_regex:nTF}~|{|~|a|~|}| is
%   essentially equivalent to \cs{peek_charcode:NTF}~|a|.
%   \begin{texnote}
%     Implicit character tokens are correctly considered by
%     \cs{peek_regex:nTF} as control sequences, while functions that
%     inspect individual tokens (for instance \cs{peek_charcode:NTF})
%     only take into account their meaning.
%
%     The \cs{peek_regex:nTF} function only inspects as few tokens as
%     necessary to determine whether the regular expression matches.
%     For instance \cs{peek_regex:nTF} \verb"{ abc | [a-z] }" |{ } { }|
%     |abc| will only inspect the first token~|a| even though the first
%     branch |abc| of the alternative is preferred in functions such as
%     \cs{peek_regex_remove_once:n}.  This may have an effect on
%     tokenization if the input stream has not yet been tokenized and
%     category codes are changed.
%   \end{texnote}
% \end{function}
%
% \begin{function}[added = 2020-12-03, TF]
%   {\peek_regex_remove_once:n, \peek_regex_remove_once:N}
%   \begin{syntax}
%     \cs{peek_regex_remove_once:nTF} \Arg{regex} \Arg{true code} \Arg{false code}
%   \end{syntax}
%   Tests if the \meta{tokens} that follow in the input stream match the
%   \meta{regex}.  If the test is true, the \meta{tokens} are removed
%   from the input stream and the \meta{true code} is inserted, while if
%   the test is false, the \meta{false code} is inserted followed by the
%   \meta{tokens} that were originally in the input stream.
%   See \pkg{l3regex} for documentation of the syntax of
%   regular expressions.  The \meta{regular expression} is implicitly
%   anchored at the start, so for instance
%   \cs{peek_regex_remove_once:nTF}~|{|~|a|~|}| is essentially equivalent to
%   \cs{peek_charcode_remove:NTF}~|a|.
%   \begin{texnote}
%     Implicit character tokens are correctly considered by
%     \cs{peek_regex_remove_once:nTF} as control sequences, while functions
%     that inspect individual tokens (for instance
%     \cs{peek_charcode:NTF}) only take into account their meaning.
%   \end{texnote}
% \end{function}
%
% \begin{function}[added = 2020-12-03, noTF]
%   {\peek_regex_replace_once:nn, \peek_regex_replace_once:Nn}
%   \begin{syntax}
%     \cs{peek_regex_replace_once:nnTF} \Arg{regex} \Arg{replacement} \Arg{true code} \Arg{false code}
%   \end{syntax}
%   If the \meta{tokens} that follow in the input stream match the
%   \meta{regex}, replaces them according to the \meta{replacement} as
%   for \cs{regex_replace_once:nnN}, and leaves the result in the input
%   stream, after the \meta{true code}.  Otherwise, leaves \meta{false
%   code} followed by the \meta{tokens} that were originally in the
%   input stream, with no modifications.  See \pkg{l3regex} for
%   documentation of the syntax of regular expressions and of the
%   \meta{replacement}: for instance |\0| in the \meta{replacement} is
%   replaced by the tokens that were matched in the input stream.  The
%   \meta{regular expression} is implicitly anchored at the start.  In
%   contrast to \cs{regex_replace_once:nnN}, no error arises if the
%   \meta{replacement} leads to an unbalanced token list: the tokens are
%   inserted into the input stream without issue.
%   \begin{texnote}
%     Implicit character tokens are correctly considered by
%     \cs{peek_regex_replace_once:nnTF} as control sequences, while
%     functions that inspect individual tokens (for instance
%     \cs{peek_charcode:NTF}) only take into account their meaning.
%   \end{texnote}
% \end{function}
%
% \section{Description of all possible tokens}
% \label{sec:l3token:all-tokens}
%
% Let us end by reviewing every case that a given token can fall into.
% This section is quite technical and some details are only meant for
% completeness.  We distinguish the meaning of the token, which controls
% the expansion of the token and its effect on \TeX{}'s state, and its
% shape, which is used when comparing token lists such as for delimited
% arguments.  Two tokens of the same shape must have the same meaning,
% but the converse does not hold.
%
% A token has one of the following shapes.
% \begin{itemize}
%   \item A control sequence, characterized by the sequence of
%     characters that constitute its name: for instance, \cs{use:n} is a
%     five-letter control sequence.
%   \item An active character token, characterized by its character code
%     (between $0$ and $1114111$ for \LuaTeX{} and \XeTeX{} and less for
%     other engines) and category code~$13$.
%   \item A character token, characterized by its character code and
%     category code (one of $1$, $2$, $3$, $4$, $6$, $7$, $8$, $10$,
%     $11$ or~$12$ whose meaning is described below).
% \end{itemize}
% There are also a few internal tokens.  The following list may be
% incomplete in some engines.
% \begin{itemize}
%   \item Expanding \tn{the}\tn{font} results in a token that looks
%     identical to the command that was used to select the current font
%     (such as \tn{tenrm}) but it differs from it in shape.
%   \item A \enquote{frozen} |\relax|, which differs from the primitive in
%     shape (but has the same meaning), is inserted when the closing \tn{fi} of a
%     conditional is encountered before the conditional is evaluated.
%   \item Expanding \tn{noexpand} \meta{token} (when the \meta{token} is
%     expandable) results in an internal token, displayed (temporarily)
%     as \cs[module = {}]{notexpanded: \meta{token}}, whose shape coincides with the
%     \meta{token} and whose meaning differs from \tn{relax}.
%   \item An |\outer endtemplate:| can be encountered when peeking ahead
%     at the next token; this expands to another internal token,
%     |end of alignment template|.
%   \item Tricky programming might access a frozen |\endwrite|.
%   \item Some frozen tokens can only be accessed in interactive
%     sessions: |\cr|, |\right|, |\endgroup|, |\fi|, |\inaccessible|.
%   \item In \LuaTeX{}, there is also the strange case of \enquote{bytes}
%     |^^^^^^1100|$x$$y$ where $x,y$ are any two lowercase hexadecimal
%     digits, so that the hexadecimal number ranges from
%     $"\text{110000}=1114112$ to~$"\text{1100ff}=1114367$.  These are
%     used to output individual bytes to files, rather than UTF-8.  For
%     the purposes of token comparisons they behave like non-expandable
%     primitive control sequences (\emph{not characters}) whose
%     \tn{meaning} is \verb*|the character | followed by the given byte.
%     If this byte is in the range |80|--|ff| this gives an ``invalid
%     utf-8 sequence'' error: applying \cs{token_to_str:N} or
%     \cs{token_to_meaning:N} to these tokens is unsafe.  Unfortunately,
%     they don't seem to be detectable safely by any means except perhaps
%     Lua code.
% \end{itemize}
%
% The meaning of a (non-active) character token is fixed by its category
% code (and character code) and cannot be changed.  We call these
% tokens \emph{explicit} character tokens.  Category codes that a
% character token can have are listed below by giving a sample output of
% the \TeX{} primitive \tn{meaning}, together with their \LaTeX3 names
% and most common example:
% \begin{itemize}
%   \item[1] begin-group character (|group_begin|, often |{|),
%   \item[2] end-group character (|group_end|, often |}|),
%   \item[3] math shift character (|math_toggle|, often |$|), %^^A $
%   \item[4] alignment tab character (|alignment|, often |&|),
%   \item[6] macro parameter character (|parameter|, often |#|),
%   \item[7] superscript character (|math_superscript|, often |^|),
%   \item[8] subscript character (|math_subscript|, often |_|),
%   \item[10] blank space (|space|, often character code~$32$),
%   \item[11] the letter (|letter|, such as |A|),
%   \item[12] the character (|other|, such as |0|).
% \end{itemize}
% Category code~$13$ (|active|) is discussed below.  Input characters
% can also have several other category codes which do not lead to
% character tokens for later processing: $0$~(|escape|),
% $5$~(|end_line|), $9$~(|ignore|), $14$~(|comment|), and
% $15$~(|invalid|).
%
% The meaning of a control sequence or active character can be identical
% to that of any character token listed above (with any character code),
% and we call such tokens \emph{implicit} character tokens.  The
% meaning is otherwise in the following list:
% \begin{itemize}
%   \item a macro, used in \LaTeX3 for most functions and some variables
%     (|tl|, |fp|, |seq|, \ldots{}),
%   \item a primitive such as \tn{def} or \tn{topmark}, used in \LaTeX3
%     for some functions,
%   \item a register such as \tn{count}|123|, used in \LaTeX3{} for the
%     implementation of some variables (|int|, |dim|, \ldots{}),
%   \item a constant integer such as \tn{char}|"56| or \tn{mathchar}|"121|,
%   \item a font selection command,
%   \item undefined.
% \end{itemize}
% Macros can be \tn{protected} or not, \tn{long} or not (the opposite of
% what \LaTeX3 calls |nopar|), and \tn{outer} or not (unused in
% \LaTeX3).  Their \tn{meaning} takes the form
% \begin{quote}
%   \meta{prefix} |macro:|\meta{argument}|->|\meta{replacement}
% \end{quote}
% where \meta{prefix} is among \tn{protected}\tn{long}\tn{outer},
% \meta{argument} describes parameters that the macro expects, such as
% |#1#2#3|, and \meta{replacement} describes how the parameters are
% manipulated, such as~|\int_eval:n{#2+#1*#3}|.
%
% Now is perhaps a good time to mention some subtleties relating to
% tokens with category code $10$ (space).  Any input character with this
% category code (normally, space and tab characters) becomes a normal
% space, with character code~$32$ and category code~$10$.
%
% When a macro takes an undelimited argument, explicit space characters
% (with character code $32$ and category code $10$) are ignored.  If the
% following token is an explicit character token with category code $1$
% (begin-group) and an arbitrary character code, then \TeX{} scans ahead
% to obtain an equal number of explicit character tokens with category
% code $1$ (begin-group) and $2$ (end-group), and the resulting list of
% tokens (with outer braces removed) becomes the argument.  Otherwise, a
% single token is taken as the argument for the macro: we call such
% single tokens \enquote{N-type}, as they are suitable to be used as an
% argument for a function with the signature~\texttt{:N}.
%
% When a macro takes a delimited argument \TeX{} scans ahead until
% finding the delimiter (outside any pairs of begin-group/end-group
% explicit characters), and the resulting list of tokens (with outer
% braces removed) becomes the argument.  Note that explicit space
% characters at the start of the argument are \emph{not} ignored in this
% case (and they prevent brace-stripping).
%
% \end{documentation}
%
% \begin{implementation}
%
% \section{\pkg{l3token} implementation}
%
%    \begin{macrocode}
%<*package>
%    \end{macrocode}
%
%    \begin{macrocode}
%<*tex>
%    \end{macrocode}
%
%    \begin{macrocode}
%<@@=char>
%    \end{macrocode}
%
% \subsection{Internal auxiliaries}
%
% \begin{variable}{\s_@@_stop}
%   Internal scan mark.
%    \begin{macrocode}
\scan_new:N \s_@@_stop
%    \end{macrocode}
% \end{variable}
%
% \begin{variable}{\q_@@_no_value}
%   Internal recursion quarks.
%    \begin{macrocode}
\quark_new:N \q_@@_no_value
%    \end{macrocode}
% \end{variable}
%
% \begin{macro}[pTF]{\@@_quark_if_no_value:N}
%   Functions to query recursion quarks.
%    \begin{macrocode}
\__kernel_quark_new_conditional:Nn \@@_quark_if_no_value:N { TF }
%    \end{macrocode}
% \end{macro}
%
% \subsection{Manipulating and interrogating character tokens}
%
% \begin{macro}{\char_set_catcode:nn}
% \begin{macro}{\char_value_catcode:n}
% \begin{macro}{\char_show_value_catcode:n}
%   Simple wrappers around the primitives.
%    \begin{macrocode}
\cs_new_protected:Npn \char_set_catcode:nn #1#2
  { \tex_catcode:D \int_eval:n {#1} = \int_eval:n {#2} \exp_stop_f: }
\cs_new:Npn \char_value_catcode:n #1
  { \tex_the:D \tex_catcode:D \int_eval:n {#1} \exp_stop_f: }
\cs_new_protected:Npn \char_show_value_catcode:n #1
  { \exp_args:Nf \tl_show:n { \char_value_catcode:n {#1} } }
%    \end{macrocode}
% \end{macro}
% \end{macro}
% \end{macro}
%
% \begin{macro}
%   {
%     \char_set_catcode_escape:N           ,
%     \char_set_catcode_group_begin:N      ,
%     \char_set_catcode_group_end:N        ,
%     \char_set_catcode_math_toggle:N      ,
%     \char_set_catcode_alignment:N        ,
%     \char_set_catcode_end_line:N         ,
%     \char_set_catcode_parameter:N        ,
%     \char_set_catcode_math_superscript:N ,
%     \char_set_catcode_math_subscript:N   ,
%     \char_set_catcode_ignore:N           ,
%     \char_set_catcode_space:N            ,
%     \char_set_catcode_letter:N           ,
%     \char_set_catcode_other:N            ,
%     \char_set_catcode_active:N           ,
%     \char_set_catcode_comment:N          ,
%     \char_set_catcode_invalid:N
%   }
%    \begin{macrocode}
\cs_new_protected:Npn \char_set_catcode_escape:N #1
  { \char_set_catcode:nn { `#1 } { 0 } }
\cs_new_protected:Npn \char_set_catcode_group_begin:N #1
  { \char_set_catcode:nn { `#1 } { 1 } }
\cs_new_protected:Npn \char_set_catcode_group_end:N #1
  { \char_set_catcode:nn { `#1 } { 2 } }
\cs_new_protected:Npn \char_set_catcode_math_toggle:N #1
  { \char_set_catcode:nn { `#1 } { 3 } }
\cs_new_protected:Npn \char_set_catcode_alignment:N #1
  { \char_set_catcode:nn { `#1 } { 4 } }
\cs_new_protected:Npn \char_set_catcode_end_line:N #1
  { \char_set_catcode:nn { `#1 } { 5 } }
\cs_new_protected:Npn \char_set_catcode_parameter:N #1
  { \char_set_catcode:nn { `#1 } { 6 } }
\cs_new_protected:Npn \char_set_catcode_math_superscript:N #1
  { \char_set_catcode:nn { `#1 } { 7 } }
\cs_new_protected:Npn \char_set_catcode_math_subscript:N #1
  { \char_set_catcode:nn { `#1 } { 8 } }
\cs_new_protected:Npn \char_set_catcode_ignore:N #1
  { \char_set_catcode:nn { `#1 } { 9 } }
\cs_new_protected:Npn \char_set_catcode_space:N #1
  { \char_set_catcode:nn { `#1 } { 10 } }
\cs_new_protected:Npn \char_set_catcode_letter:N #1
  { \char_set_catcode:nn { `#1 } { 11 } }
\cs_new_protected:Npn \char_set_catcode_other:N #1
  { \char_set_catcode:nn { `#1 } { 12 } }
\cs_new_protected:Npn \char_set_catcode_active:N #1
  { \char_set_catcode:nn { `#1 } { 13 } }
\cs_new_protected:Npn \char_set_catcode_comment:N #1
  { \char_set_catcode:nn { `#1 } { 14 } }
\cs_new_protected:Npn \char_set_catcode_invalid:N #1
  { \char_set_catcode:nn { `#1 } { 15 } }
%    \end{macrocode}
% \end{macro}
%
% \begin{macro}
%   {
%     \char_set_catcode_escape:n           ,
%     \char_set_catcode_group_begin:n      ,
%     \char_set_catcode_group_end:n        ,
%     \char_set_catcode_math_toggle:n      ,
%     \char_set_catcode_alignment:n        ,
%     \char_set_catcode_end_line:n         ,
%     \char_set_catcode_parameter:n        ,
%     \char_set_catcode_math_superscript:n ,
%     \char_set_catcode_math_subscript:n   ,
%     \char_set_catcode_ignore:n           ,
%     \char_set_catcode_space:n            ,
%     \char_set_catcode_letter:n           ,
%     \char_set_catcode_other:n            ,
%     \char_set_catcode_active:n           ,
%     \char_set_catcode_comment:n          ,
%     \char_set_catcode_invalid:n
%   }
%    \begin{macrocode}
\cs_new_protected:Npn \char_set_catcode_escape:n #1
  { \char_set_catcode:nn {#1} { 0 } }
\cs_new_protected:Npn \char_set_catcode_group_begin:n #1
  { \char_set_catcode:nn {#1} { 1 } }
\cs_new_protected:Npn \char_set_catcode_group_end:n #1
  { \char_set_catcode:nn {#1} { 2 } }
\cs_new_protected:Npn \char_set_catcode_math_toggle:n #1
  { \char_set_catcode:nn {#1} { 3 } }
\cs_new_protected:Npn \char_set_catcode_alignment:n #1
  { \char_set_catcode:nn {#1} { 4 } }
\cs_new_protected:Npn \char_set_catcode_end_line:n #1
  { \char_set_catcode:nn {#1} { 5 } }
\cs_new_protected:Npn \char_set_catcode_parameter:n #1
  { \char_set_catcode:nn {#1} { 6 } }
\cs_new_protected:Npn \char_set_catcode_math_superscript:n #1
  { \char_set_catcode:nn {#1} { 7 } }
\cs_new_protected:Npn \char_set_catcode_math_subscript:n #1
  { \char_set_catcode:nn {#1} { 8 } }
\cs_new_protected:Npn \char_set_catcode_ignore:n #1
  { \char_set_catcode:nn {#1} { 9 } }
\cs_new_protected:Npn \char_set_catcode_space:n #1
  { \char_set_catcode:nn {#1} { 10 } }
\cs_new_protected:Npn \char_set_catcode_letter:n #1
  { \char_set_catcode:nn {#1} { 11 } }
\cs_new_protected:Npn \char_set_catcode_other:n #1
  { \char_set_catcode:nn {#1} { 12 } }
\cs_new_protected:Npn \char_set_catcode_active:n #1
  { \char_set_catcode:nn {#1} { 13 } }
\cs_new_protected:Npn \char_set_catcode_comment:n #1
  { \char_set_catcode:nn {#1} { 14 } }
\cs_new_protected:Npn \char_set_catcode_invalid:n #1
  { \char_set_catcode:nn {#1} { 15 } }
%    \end{macrocode}
% \end{macro}
%
% \begin{macro}{\char_set_mathcode:nn}
% \begin{macro}{\char_value_mathcode:n}
% \begin{macro}{\char_show_value_mathcode:n}
% \begin{macro}{\char_set_lccode:nn}
% \begin{macro}{\char_value_lccode:n}
% \begin{macro}{\char_show_value_lccode:n}
% \begin{macro}{\char_set_uccode:nn}
% \begin{macro}{\char_value_uccode:n}
% \begin{macro}{\char_show_value_uccode:n}
% \begin{macro}{\char_set_sfcode:nn}
% \begin{macro}{\char_value_sfcode:n}
% \begin{macro}{\char_show_value_sfcode:n}
%   Pretty repetitive, but necessary!
%    \begin{macrocode}
\cs_new_protected:Npn \char_set_mathcode:nn #1#2
  { \tex_mathcode:D \int_eval:n {#1} = \int_eval:n {#2} \exp_stop_f: }
\cs_new:Npn \char_value_mathcode:n #1
  { \tex_the:D \tex_mathcode:D \int_eval:n {#1} \exp_stop_f: }
\cs_new_protected:Npn \char_show_value_mathcode:n #1
  { \exp_args:Nf \tl_show:n { \char_value_mathcode:n {#1} } }
\cs_new_protected:Npn \char_set_lccode:nn #1#2
  { \tex_lccode:D \int_eval:n {#1} = \int_eval:n {#2} \exp_stop_f: }
\cs_new:Npn \char_value_lccode:n #1
  { \tex_the:D \tex_lccode:D \int_eval:n {#1} \exp_stop_f: }
\cs_new_protected:Npn \char_show_value_lccode:n #1
  { \exp_args:Nf \tl_show:n { \char_value_lccode:n {#1} } }
\cs_new_protected:Npn \char_set_uccode:nn #1#2
  { \tex_uccode:D \int_eval:n {#1} = \int_eval:n {#2} \exp_stop_f: }
\cs_new:Npn \char_value_uccode:n #1
  { \tex_the:D \tex_uccode:D \int_eval:n {#1} \exp_stop_f: }
\cs_new_protected:Npn \char_show_value_uccode:n #1
  { \exp_args:Nf \tl_show:n { \char_value_uccode:n {#1} } }
\cs_new_protected:Npn \char_set_sfcode:nn #1#2
  { \tex_sfcode:D \int_eval:n {#1} = \int_eval:n {#2} \exp_stop_f: }
\cs_new:Npn \char_value_sfcode:n #1
  { \tex_the:D \tex_sfcode:D \int_eval:n {#1} \exp_stop_f: }
\cs_new_protected:Npn \char_show_value_sfcode:n #1
  { \exp_args:Nf \tl_show:n { \char_value_sfcode:n {#1} } }
%    \end{macrocode}
% \end{macro}
% \end{macro}
% \end{macro}
% \end{macro}
% \end{macro}
% \end{macro}
% \end{macro}
% \end{macro}
% \end{macro}
% \end{macro}
% \end{macro}
% \end{macro}
%
% \begin{variable}{\l_char_active_seq, \l_char_special_seq}
%   Two sequences for dealing with special characters. The first is characters
%   which may be active, the second longer list is for \enquote{special}
%   characters more generally. Both lists are escaped so that for example
%   bulk code assignments can be carried out. In both cases, the order is
%   by \textsc{ascii} character code (as is done in for example
%   \cs{ExplSyntaxOn}).
%    \begin{macrocode}
\seq_new:N \l_char_special_seq
\seq_set_split:Nnn \l_char_special_seq { }
  { \  \" \# \$ \% \& \\ \^ \_ \{ \} \~ }
\seq_new:N \l_char_active_seq
\seq_set_split:Nnn \l_char_active_seq { }
  { \" \$ \& \^ \_ \~ }
%    \end{macrocode}
% \end{variable}
%
% \subsection{Creating character tokens}
%
% \begin{macro}
%   {
%     \char_set_active_eq:NN, \char_gset_active_eq:NN,
%     \char_set_active_eq:Nc, \char_gset_active_eq:Nc,
%     \char_set_active_eq:nN, \char_gset_active_eq:nN,
%     \char_set_active_eq:nc, \char_gset_active_eq:nc
%   }
%   Four simple functions with very similar definitions, so set up using
%   an auxiliary.
%   These are similar to \LuaTeX{}'s \tn{letcharcode} primitive.
%    \begin{macrocode}
\group_begin:
  \char_set_catcode_active:N \^^@
  \cs_set_protected:Npn \@@_tmp:nN #1#2
    {
      \cs_new_protected:cpn { #1 :nN } ##1
        {
          \group_begin:
            \char_set_lccode:nn { `\^^@ } { ##1 }
          \tex_lowercase:D { \group_end: #2 ^^@ }
        }
      \cs_new_protected:cpx { #1 :NN } ##1
        { \exp_not:c { #1 : nN } { `##1 } }
    }
  \@@_tmp:nN { char_set_active_eq }  \cs_set_eq:NN
  \@@_tmp:nN { char_gset_active_eq } \cs_gset_eq:NN
\group_end:
\cs_generate_variant:Nn \char_set_active_eq:NN  { Nc }
\cs_generate_variant:Nn \char_gset_active_eq:NN { Nc }
\cs_generate_variant:Nn \char_set_active_eq:nN  { nc }
\cs_generate_variant:Nn \char_gset_active_eq:nN { nc }
%    \end{macrocode}
% \end{macro}
%
% \begin{macro}{\@@_int_to_roman:w}
%   For efficiency in 8-bit engines, we use the faster primitive approach
%   to making roman numerals.
%    \begin{macrocode}
\cs_new_eq:NN \@@_int_to_roman:w \tex_romannumeral:D
%    \end{macrocode}
% \end{macro}
%
% \begin{macro}[EXP]{\char_generate:nn}
% \begin{macro}[EXP]{\@@_generate_aux:nn}
% \begin{macro}[EXP]{\@@_generate_aux:nnw, \@@_generate_auxii:nnw}
% \begin{variable}{\l_@@_tmp_tl}
% \begin{macro}[EXP]{\@@_generate_invalid_catcode:}
%   The aim here is to generate characters of (broadly) arbitrary category
%   code. Where possible, that is done using engine support (\XeTeX{},
%   \LuaTeX{}). There are though various issues which are covered below. At
%   the interface layer, turn the two arguments into integers up-front so
%   this is only done once.
%    \begin{macrocode}
\cs_new:Npn \char_generate:nn #1#2
  {
    \exp:w \exp_after:wN \@@_generate_aux:w
      \int_value:w \int_eval:n {#1} \exp_after:wN ;
      \int_value:w \int_eval:n {#2} ;
  }
%    \end{macrocode}
%   Before doing any actual conversion, first some special case filtering.
%   Spaces are out here as \LuaTeX{} emulation only makes normal (charcode
%   $32$ spaces). However, |^^@| is filtered out separately as that can't be
%   done with macro emulation either, so is flagged up separately. That
%   done, hand off to the engine-dependent part.
%    \begin{macrocode}
\cs_new:Npn \@@_generate_aux:w #1 ; #2 ;
  {
    \if_int_compare:w #2 = 10 \exp_stop_f:
      \if_int_compare:w #1 =  \c_zero_int
        \msg_expandable_error:nn { char } { null-space }
      \else:
        \msg_expandable_error:nn { char } { space }
      \fi:
    \else:
      \if_int_odd:w 0
          \if_int_compare:w #2 < 1  \exp_stop_f: 1 \fi:
          \if_int_compare:w #2 = 5  \exp_stop_f: 1 \fi:
          \if_int_compare:w #2 = 9  \exp_stop_f: 1 \fi:
          \if_int_compare:w #2 > 13 \exp_stop_f: 1 \fi: \exp_stop_f:
        \msg_expandable_error:nn { char }
          { invalid-catcode }
      \else:
        \if_int_odd:w 0
          \if_int_compare:w #1 < \c_zero_int 1 \fi:
          \if_int_compare:w #1 > \c_max_char_int 1 \fi: \exp_stop_f:
          \msg_expandable_error:nn { char }
            { out-of-range }
        \else:
          \@@_generate_aux:nnw {#1} {#2}
        \fi:
      \fi:
    \fi:
    \exp_end:
  }
\tl_new:N \l_@@_tmp_tl
%    \end{macrocode}
%   Engine-dependent definitions are now needed for the implementation. For
%   \LuaTeX{} and \XeTeX{} there is engine-level support.
%   They can do cases that macro emulation can't. All of those are filtered
%   out here using a primitive-based boolean expression to avoid fixing
%   the category code of the null character used in the false branch
%   (for 8-bit engines).
%   The final level is the basic definition at the engine level: the arguments
%   here are integers so there is no need to worry about them too much.
%   Older versions of  \XeTeX{} cannot generate active characters so we filter
%   that:
%   at some future stage that may change: the slightly odd ordering of
%   auxiliaries reflects that.
%    \begin{macrocode}
\group_begin:
  \char_set_catcode_active:N \^^L
  \cs_set:Npn ^^L { }
  \char_set_catcode_other:n { 0 }
  \if_int_odd:w 0
      \sys_if_engine_luatex:T { 1 }
      \sys_if_engine_xetex:T { 1 } \exp_stop_f:
      \cs_new:Npn \@@_generate_aux:nnw #1#2#3 \exp_end:
        {
          #3
          \exp_after:wN \exp_end:
          \tex_Ucharcat:D #1 \exp_stop_f: #2 \exp_stop_f:
        }
      \cs_if_exist:NF \tex_expanded:D
        {
          \cs_new_eq:NN \@@_generate_auxii:nnw \@@_generate_aux:nnw
          \cs_gset:Npn \@@_generate_aux:nnw #1#2#3 \exp_end:
            {
              #3
              \if_int_compare:w #2 = 13 \exp_stop_f:
                \msg_expandable_error:nn { char } { active }
              \else:
                \@@_generate_auxii:nnw {#1} {#2}
              \fi:
              \exp_end:
            }
        }
  \else:
%    \end{macrocode}
%   For engines where \tn{Ucharcat} isn't available or emulated, we have
%   to work in macros, and cover only the $8$-bit range. The first stage is
%   to build up a |tl| containing |^^@| with each category code that can
%   be accessed in this way, with an error set up for the other cases. This
%   is all done such that it can be quickly accessed using a |\if_case:w|
%   low-level conditional. There are a few things to notice here.
%   As |^^L| is |\outer| we need to locally set it to avoid a problem.
%   To get open/close braces into the list, they are set up using |\if_false:|
%   pairing and are then |x|-type expanded together into the desired form.
%    \begin{macrocode}
      \tl_set:Nn \l_@@_tmp_tl { \exp_not:N \or: }
      \char_set_catcode_group_begin:n { 0 } % {
      \tl_put_right:Nn \l_@@_tmp_tl { ^^@ \if_false: } }
      \char_set_catcode_group_end:n { 0 }
      \tl_put_right:Nn \l_@@_tmp_tl { { \fi: \exp_not:N \or: ^^@ } % }
      \__kernel_tl_set:Nx \l_@@_tmp_tl { \l_@@_tmp_tl }
      \char_set_catcode_math_toggle:n { 0 }
      \tl_put_right:Nn \l_@@_tmp_tl { \or: ^^@ }
      \char_set_catcode_alignment:n { 0 }
      \tl_put_right:Nn \l_@@_tmp_tl { \or: ^^@ }
      \tl_put_right:Nn \l_@@_tmp_tl { \or: }
      \char_set_catcode_parameter:n { 0 }
      \tl_put_right:Nn \l_@@_tmp_tl { \or: ^^@ }
      \char_set_catcode_math_superscript:n { 0 }
      \tl_put_right:Nn \l_@@_tmp_tl { \or: ^^@ }
      \char_set_catcode_math_subscript:n { 0 }
      \tl_put_right:Nn \l_@@_tmp_tl { \or: ^^@ }
      \tl_put_right:Nn \l_@@_tmp_tl { \or: }
%    \end{macrocode}
%   For making spaces, there needs to be an |o|-type expansion of a |\use:n|
%   (or some other tokenization) to avoid dropping the space.
%    \begin{macrocode}
      \char_set_catcode_space:n { 0 }
      \tl_put_right:No \l_@@_tmp_tl { \use:n { \or: } ^^@ }
      \char_set_catcode_letter:n { 0 }
      \tl_put_right:Nn \l_@@_tmp_tl { \or: ^^@ }
      \char_set_catcode_other:n { 0 }
      \tl_put_right:Nn \l_@@_tmp_tl { \or: ^^@ }
      \char_set_catcode_active:n { 0 }
      \tl_put_right:Nn \l_@@_tmp_tl { \or: ^^@ }
%    \end{macrocode}
%   Convert the above temporary list into a series of constant token
%   lists, one for each character code, using \cs{tex_lowercase:D} to
%   convert |^^@| in each case. The \texttt{x}-type expansion ensures
%   that \cs{tex_lowercase:D} receives the contents of the token list.
%   |^^L| is awkward hence this is done in three parts: up to |^^L|,
%   |^^L| itslef and above |^L|. Notice that at this stage |^^@| is active.
%    \begin{macrocode}
      \cs_set_protected:Npn \@@_tmp:n #1
        {
          \char_set_lccode:nn { 0 } {#1}
          \char_set_lccode:nn { 32 } {#1}
          \exp_args:Nx \tex_lowercase:D
            {
              \tl_const:Nn
                \exp_not:c { c_@@_ \@@_int_to_roman:w #1 _tl }
                { \exp_not:o \l_@@_tmp_tl }
            }
        }
      \int_step_function:nnN { 0 } { 11 }  \@@_tmp:n
      \group_begin:
        \tl_replace_once:Nnn \l_@@_tmp_tl { ^^@ } { \ERROR }
        \@@_tmp:n { 12 }
      \group_end:
      \int_step_function:nnN { 13 } { 255 } \@@_tmp:n
%    \end{macrocode}
%   As \TeX{} is very unhappy if it finds an alignment character inside
%   a primitive \tn{halign} even when skipping false branches, some
%   precautions are required. \TeX{} is happy if the token is hidden
%   between braces within \cs{if_false:} \dots{} \cs{fi:}.
%    \begin{macrocode}
      \cs_new:Npn \@@_generate_aux:nnw #1#2#3 \exp_end:
        {
          #3
          \if_false: { \fi:
          \exp_after:wN \exp_after:wN
          \exp_after:wN \exp_end:
          \exp_after:wN \exp_after:wN
          \if_case:w #2
            \exp_last_unbraced:Nv \exp_stop_f:
              { c_@@_ \@@_int_to_roman:w #1 _tl }
          \or: }
          \fi:
        }
  \fi:
\group_end:
%    \end{macrocode}
% \end{macro}
% \end{variable}
% \end{macro}
% \end{macro}
% \end{macro}
%
% \begin{macro}[EXP]{\char_to_utfviii_bytes:n}
% \begin{macro}[EXP]{\@@_to_utfviii_bytes_auxi:n}
% \begin{macro}[EXP]{\@@_to_utfviii_bytes_auxii:Nnn}
% \begin{macro}[EXP]{\@@_to_utfviii_bytes_auxiii:n}
% \begin{macro}[EXP]
%   {
%     \@@_to_utfviii_bytes_outputi:nw   ,
%     \@@_to_utfviii_bytes_outputii:nw  ,
%     \@@_to_utfviii_bytes_outputiii:nw ,
%     \@@_to_utfviii_bytes_outputiv:nw
%   }
% \begin{macro}[EXP]
%   {\@@_to_utfviii_bytes_output:nnn, \@@_to_utfviii_bytes_output:fnn}
% \begin{macro}[EXP]{\@@_to_utfviii_bytes_end:}
%   This code converts a codepoint into the correct UTF-8 representation.
%   In terms of the algorithm itself, see
%   \url{https://en.wikipedia.org/wiki/UTF-8} for the octet pattern.
%    \begin{macrocode}
\cs_new:Npn \char_to_utfviii_bytes:n #1
  {
    \exp_args:Nf \@@_to_utfviii_bytes_auxi:n
      { \int_eval:n {#1} }
  }
\cs_new:Npn \@@_to_utfviii_bytes_auxi:n #1
  {
    \if_int_compare:w #1 > "80 \exp_stop_f:
      \if_int_compare:w #1 < "800 \exp_stop_f:
        \@@_to_utfviii_bytes_outputi:nw
          { \@@_to_utfviii_bytes_auxii:Nnn C {#1} { 64 } }
        \@@_to_utfviii_bytes_outputii:nw
          { \@@_to_utfviii_bytes_auxiii:n {#1} }
      \else:
        \if_int_compare:w #1 < "10000 \exp_stop_f:
          \@@_to_utfviii_bytes_outputi:nw
            { \@@_to_utfviii_bytes_auxii:Nnn E {#1} { 64 * 64 } }
          \@@_to_utfviii_bytes_outputii:nw
            {
              \@@_to_utfviii_bytes_auxiii:n
                { \int_div_truncate:nn {#1} { 64 } }
            }
          \@@_to_utfviii_bytes_outputiii:nw
            { \@@_to_utfviii_bytes_auxiii:n {#1} }
        \else:
          \@@_to_utfviii_bytes_outputi:nw
            {
              \@@_to_utfviii_bytes_auxii:Nnn F
                 {#1} { 64 * 64 * 64 }
            }
          \@@_to_utfviii_bytes_outputii:nw
            {
              \@@_to_utfviii_bytes_auxiii:n
                { \int_div_truncate:nn {#1} { 64 * 64 } }
            }
          \@@_to_utfviii_bytes_outputiii:nw
            {
              \@@_to_utfviii_bytes_auxiii:n
                { \int_div_truncate:nn {#1} { 64 } }
            }
          \@@_to_utfviii_bytes_outputiv:nw
            { \@@_to_utfviii_bytes_auxiii:n {#1} }
        \fi:
      \fi:
    \else:
      \@@_to_utfviii_bytes_outputi:nw {#1}
    \fi:
    \@@_to_utfviii_bytes_end: { } { } { } { }
  }
\cs_new:Npn \@@_to_utfviii_bytes_auxii:Nnn #1#2#3
  {  "#10 + \int_div_truncate:nn {#2} {#3} }
\cs_new:Npn \@@_to_utfviii_bytes_auxiii:n #1
  { \int_mod:nn {#1} { 64 } + 128 }
\cs_new:Npn \@@_to_utfviii_bytes_outputi:nw
  #1 #2 \@@_to_utfviii_bytes_end: #3
  { \@@_to_utfviii_bytes_output:fnn { \int_eval:n {#1} } { } {#2} }
\cs_new:Npn \@@_to_utfviii_bytes_outputii:nw
  #1 #2 \@@_to_utfviii_bytes_end: #3#4
  { \@@_to_utfviii_bytes_output:fnn { \int_eval:n {#1} } { {#3} } {#2} }
\cs_new:Npn \@@_to_utfviii_bytes_outputiii:nw
  #1 #2 \@@_to_utfviii_bytes_end: #3#4#5
  {
    \@@_to_utfviii_bytes_output:fnn
      { \int_eval:n {#1} } { {#3} {#4} } {#2}
  }
\cs_new:Npn \@@_to_utfviii_bytes_outputiv:nw
  #1 #2 \@@_to_utfviii_bytes_end: #3#4#5#6
  {
    \@@_to_utfviii_bytes_output:fnn
      { \int_eval:n {#1} } { {#3} {#4} {#5} } {#2}
  }
\cs_new:Npn \@@_to_utfviii_bytes_output:nnn #1#2#3
  {
    #3
    \@@_to_utfviii_bytes_end: #2 {#1}
  }
\cs_generate_variant:Nn \@@_to_utfviii_bytes_output:nnn { f }
\cs_new:Npn \@@_to_utfviii_bytes_end: { }
%    \end{macrocode}
% \end{macro}
% \end{macro}
% \end{macro}
% \end{macro}
% \end{macro}
% \end{macro}
% \end{macro}
%
% \begin{macro}[rEXP]{\char_to_nfd:N}
% \begin{macro}[rEXP]{\@@_to_nfd:n}
% \begin{macro}[rEXP]{\@@_to_nfd:Nw}
%   Look up any \textsc{nfd} and recursively produce the result.
%    \begin{macrocode}
\cs_new:Npn \char_to_nfd:N #1
  {
    \cs_if_exist:cTF { c_@@_nfd_ \token_to_str:N #1 _ tl }
      {
        \exp_after:wN \exp_after:wN \exp_after:wN \@@_to_nfd:Nw
          \exp_after:wN \exp_after:wN \exp_after:wN #1
            \cs:w c_@@_nfd_ \token_to_str:N #1 _ tl \cs_end:
              \s_@@_stop
      }
      { \exp_not:n {#1} }
  }
\cs_set_eq:NN \@@_to_nfd:n \char_to_nfd:N
\cs_new:Npn \@@_to_nfd:Nw #1#2#3 \s_@@_stop
  {
    \exp_args:Ne \@@_to_nfd:n
      { \char_generate:nn { `#2 } { \@@_change_case_catcode:N #1 } }
    \tl_if_blank:nF {#3}
      {
        \exp_args:Ne \@@_to_nfd:n
          { \char_generate:nn { `#3 } { \char_value_catcode:n { `#3 } } }
      }
  }
%    \end{macrocode}
% \end{macro}
% \end{macro}
% \end{macro}
%
% \begin{macro}[EXP]
%   {
%     \char_lowercase:N, \char_uppercase:N,
%     \char_titlecase:N, \char_foldcase:N
%   }
% \begin{macro}[EXP]{\@@_change_case:nNN}
% \begin{macro}[EXP]{\@@_change_case:nN}
% \begin{macro}[EXP]{\@@_change_case_multi:nN, \@@_change_case_multi:vN}
% \begin{macro}[EXP]{\@@_change_case_multi:NNNNw}
% \begin{macro}[EXP]{\@@_change_case:NNN}
% \begin{macro}[EXP]{\@@_change_case:NNNN}
% \begin{macro}[EXP]{\@@_change_case:NN}
% \begin{macro}[EXP]{\@@_change_case_catcode:N}
% \begin{macro}[EXP]
%   {
%     \char_str_lowercase:N, \char_str_uppercase:N,
%     \char_str_titlecase:N, \char_str_foldcase:N
%   }
% \begin{macro}[EXP]{\@@_str_change_case:nNN}
% \begin{macro}[EXP]{\@@_str_change_case:nN}
%   To ensure that the category codes produced are predictable, every character
%   is re-generated even if it is otherwise unchanged. This makes life a little
%   interesting when we might have multiple output characters: we have to
%   grab each of them and case change them in reverse order to maintain
%   \texttt{f}-type expandability.
%    \begin{macrocode}
\cs_new:Npn \char_lowercase:N #1
  { \@@_change_case:nNN { lower } \char_value_lccode:n #1 }
\cs_new:Npn \char_uppercase:N #1
  { \@@_change_case:nNN { upper } \char_value_uccode:n #1 }
\cs_new:Npn \char_titlecase:N #1
  {
    \tl_if_exist:cTF { c_@@_titlecase_ \token_to_str:N #1 _tl }
      {
        \@@_change_case_multi:vN
          { c_@@_titlecase_ \token_to_str:N #1 _tl } #1
      }
      { \char_uppercase:N #1 }
  }
\cs_new:Npn \char_foldcase:N #1
  { \@@_change_case:nNN { fold } \char_value_lccode:n #1 }
\cs_new:Npn \@@_change_case:nNN #1#2#3
  {
    \tl_if_exist:cTF { c_@@_ #1 case_ \token_to_str:N #3 _tl }
      {
        \@@_change_case_multi:vN
          { c_@@_ #1 case_ \token_to_str:N #3 _tl } #3
      }
      { \exp_args:Nf \@@_change_case:nN { #2 { `#3 } } #3 }
  }
\cs_new:Npn \@@_change_case:nN #1#2
  {
    \int_compare:nNnTF {#1} = 0
      { #2 }
      { \char_generate:nn {#1} { \@@_change_case_catcode:N #2 } }
  }
\cs_new:Npn \@@_change_case_multi:nN #1#2
  { \@@_change_case_multi:NNNNw #2 #1 \q_@@_no_value \q_@@_no_value \s_@@_stop }
\cs_generate_variant:Nn \@@_change_case_multi:nN { v }
\cs_new:Npn \@@_change_case_multi:NNNNw #1#2#3#4#5 \s_@@_stop
  {
    \@@_quark_if_no_value:NTF #4
      {
        \@@_quark_if_no_value:NTF #3
          { \@@_change_case:NN #1 #2 }
          { \@@_change_case:NNN #1 #2#3 }
      }
      { \@@_change_case:NNNN #1 #2#3#4 }
  }
\cs_new:Npn \@@_change_case:NNN #1#2#3
  {
    \exp_args:Nnf \use:nn
      { \@@_change_case:NN #1 #2 }
      { \@@_change_case:NN #1 #3 }
  }
\cs_new:Npn \@@_change_case:NNNN #1#2#3#4
  {
    \exp_args:Nnff \use:nnn
      { \@@_change_case:NN #1 #2 }
      { \@@_change_case:NN #1 #3 }
      { \@@_change_case:NN #1 #4 }
  }
\cs_new:Npn \@@_change_case:NN #1#2
  { \char_generate:nn { `#2 } { \@@_change_case_catcode:N #1 } }
\cs_new:Npn \@@_change_case_catcode:N #1
  {
    \if_catcode:w \exp_not:N #1 \c_math_toggle_token
      3
    \else:
      \if_catcode:w \exp_not:N #1 \c_alignment_token
        4
      \else:
        \if_catcode:w \exp_not:N #1 \c_math_superscript_token
          7
        \else:
          \if_catcode:w \exp_not:N #1 \c_math_subscript_token
            8
          \else:
            \if_catcode:w \exp_not:N #1 \c_space_token
              10
            \else:
             \if_catcode:w \exp_not:N #1 \c_catcode_letter_token
               11
             \else:
               \if_catcode:w \exp_not:N #1 \c_catcode_other_token
                 12
               \else:
                 13
               \fi:
             \fi:
            \fi:
          \fi:
        \fi:
      \fi:
    \fi:
  }
%    \end{macrocode}
%   Same story for the string version, except category code is easier
%   to follow. This of course makes this version significantly faster.
%    \begin{macrocode}
\cs_new:Npn \char_str_lowercase:N #1
  { \@@_str_change_case:nNN { lower } \char_value_lccode:n #1 }
\cs_new:Npn \char_str_uppercase:N #1
  { \@@_str_change_case:nNN { upper } \char_value_uccode:n #1 }
\cs_new:Npn \char_str_titlecase:N #1
  {
    \tl_if_exist:cTF { c_@@_titlecase_ \token_to_str:N #1 _tl }
      { \tl_to_str:c { c_@@_titlecase_ \token_to_str:N #1 _tl } }
      { \char_str_uppercase:N #1 }
  }
\cs_new:Npn \char_str_foldcase:N #1
  { \@@_str_change_case:nNN { fold } \char_value_lccode:n #1 }
\cs_new:Npn \@@_str_change_case:nNN #1#2#3
  {
    \tl_if_exist:cTF { c_@@_ #1 case_ \token_to_str:N #3 _tl }
      { \tl_to_str:c { c_@@_ #1 case_ \token_to_str:N #3 _tl } }
      { \exp_args:Nf \@@_str_change_case:nN { #2 { `#3 } } #3 }
  }
\cs_new:Npn \@@_str_change_case:nN #1#2
  {
    \int_compare:nNnTF {#1} = 0
      { \tl_to_str:n {#2} }
      { \char_generate:nn {#1} { 12 } }
  }
\bool_lazy_or:nnF
  { \cs_if_exist_p:N \tex_luatexversion:D }
  { \cs_if_exist_p:N \tex_XeTeXversion:D }
  {
    \cs_set:Npn \@@_str_change_case:nN #1#2
      { \tl_to_str:n {#2} }
  }
%    \end{macrocode}
% \end{macro}
% \end{macro}
% \end{macro}
% \end{macro}
% \end{macro}
% \end{macro}
% \end{macro}
% \end{macro}
% \end{macro}
% \end{macro}
% \end{macro}
% \end{macro}
%
% \begin{macro}{\c_catcode_other_space_tl}
%   Create a space with category code $12$: an \enquote{other} space.
%    \begin{macrocode}
\tl_const:Nx \c_catcode_other_space_tl { \char_generate:nn { `\  } { 12 } }
%    \end{macrocode}
% \end{macro}
%
% \subsection{Generic tokens}
%
%    \begin{macrocode}
%<@@=token>
%    \end{macrocode}
%
% \begin{variable}{\s_@@_mark, \s_@@_stop}
%   Internal scan marks.
%    \begin{macrocode}
\scan_new:N \s_@@_mark
\scan_new:N \s_@@_stop
%    \end{macrocode}
% \end{variable}
%
% \begin{macro}{\token_to_meaning:N, \token_to_meaning:c}
% \begin{macro}{\token_to_str:N, \token_to_str:c}
%   These are all defined in \pkg{l3basics}, as they are needed
%   \enquote{early}. This is just a reminder!
% \end{macro}
% \end{macro}
%
% \begin{macro}
%   {
%     \c_group_begin_token,
%     \c_group_end_token,
%     \c_math_toggle_token,
%     \c_alignment_token,
%     \c_parameter_token,
%     \c_math_superscript_token,
%     \c_math_subscript_token,
%     \c_space_token,
%     \c_catcode_letter_token,
%     \c_catcode_other_token
%   }
%   We define these useful tokens. For the brace and space tokens things have
%   to be done by hand: the formal argument spec.~for \cs{cs_new_eq:NN} does
%   not cover them so we do things by hand. (As currently coded it would
%   \emph{work} with \cs{cs_new_eq:NN} but that's not really a great idea to
%   show off: we want people to stick to the defined interfaces and that
%   includes us.) So that these few odd names go into the log when appropriate
%   there is a need to hand-apply the \cs{__kernel_chk_if_free_cs:N} check.
%    \begin{macrocode}
\group_begin:
  \__kernel_chk_if_free_cs:N \c_group_begin_token
  \tex_global:D \tex_let:D \c_group_begin_token {
  \__kernel_chk_if_free_cs:N \c_group_end_token
  \tex_global:D \tex_let:D \c_group_end_token }
  \char_set_catcode_math_toggle:N \*
  \cs_new_eq:NN \c_math_toggle_token *
  \char_set_catcode_alignment:N \*
  \cs_new_eq:NN \c_alignment_token *
  \cs_new_eq:NN \c_parameter_token #
  \cs_new_eq:NN \c_math_superscript_token ^
  \char_set_catcode_math_subscript:N \*
  \cs_new_eq:NN \c_math_subscript_token *
  \__kernel_chk_if_free_cs:N \c_space_token
  \use:n { \tex_global:D \tex_let:D \c_space_token = ~ } ~
  \cs_new_eq:NN \c_catcode_letter_token a
  \cs_new_eq:NN \c_catcode_other_token 1
\group_end:
%    \end{macrocode}
% \end{macro}
%
% \begin{variable}{\c_catcode_active_tl}
%   Not an implicit token!
%    \begin{macrocode}
\group_begin:
  \char_set_catcode_active:N \*
  \tl_const:Nn \c_catcode_active_tl { \exp_not:N * }
\group_end:
%    \end{macrocode}
% \end{variable}
%
% \subsection{Token conditionals}
%
% \begin{macro}[pTF]{\token_if_group_begin:N}
%   Check if token is a begin group token. We use the constant
%   \cs{c_group_begin_token} for this.
%    \begin{macrocode}
\prg_new_conditional:Npnn \token_if_group_begin:N #1 { p , T ,  F , TF }
  {
    \if_catcode:w \exp_not:N #1 \c_group_begin_token
      \prg_return_true: \else: \prg_return_false: \fi:
  }
%    \end{macrocode}
% \end{macro}
%
% \begin{macro}[pTF]{\token_if_group_end:N}
%   Check if token is a end group token. We use the constant
%   \cs{c_group_end_token} for this.
%    \begin{macrocode}
\prg_new_conditional:Npnn \token_if_group_end:N #1 { p , T ,  F , TF }
  {
    \if_catcode:w \exp_not:N #1 \c_group_end_token
      \prg_return_true: \else: \prg_return_false: \fi:
  }
%    \end{macrocode}
% \end{macro}
%
% \begin{macro}[pTF]{\token_if_math_toggle:N}
%   Check if token is a math shift token. We use the constant
%   \cs{c_math_toggle_token} for this.
%    \begin{macrocode}
\prg_new_conditional:Npnn \token_if_math_toggle:N #1 { p , T ,  F , TF }
  {
    \if_catcode:w \exp_not:N #1 \c_math_toggle_token
      \prg_return_true: \else: \prg_return_false: \fi:
  }
%    \end{macrocode}
% \end{macro}
%
% \begin{macro}[pTF]{\token_if_alignment:N}
%   Check if token is an alignment tab token. We use the constant
%   \cs{c_alignment_token} for this.
%    \begin{macrocode}
\prg_new_conditional:Npnn \token_if_alignment:N #1 { p , T ,  F , TF }
  {
    \if_catcode:w \exp_not:N #1 \c_alignment_token
      \prg_return_true: \else: \prg_return_false: \fi:
  }
%    \end{macrocode}
% \end{macro}
%
% \begin{macro}[pTF]{\token_if_parameter:N}
%   Check if token is a parameter token. We use the constant
%   \cs{c_parameter_token} for this. We have to trick \TeX{} a bit to
%   avoid an error message: within a group we prevent
%   \cs{c_parameter_token} from behaving like a macro parameter character.
%   The definitions of \cs{prg_new_conditional:Npnn} are global, so they
%   remain after the group.
%    \begin{macrocode}
\group_begin:
\cs_set_eq:NN \c_parameter_token \scan_stop:
\prg_new_conditional:Npnn \token_if_parameter:N #1 { p , T ,  F , TF }
  {
    \if_catcode:w \exp_not:N #1 \c_parameter_token
      \prg_return_true: \else: \prg_return_false: \fi:
  }
\group_end:
%    \end{macrocode}
% \end{macro}
%
% \begin{macro}[pTF]{\token_if_math_superscript:N}
%   Check if token is a math superscript token. We use the constant
%   \cs{c_math_superscript_token} for this.
%    \begin{macrocode}
\prg_new_conditional:Npnn \token_if_math_superscript:N #1
  { p , T ,  F , TF }
  {
    \if_catcode:w \exp_not:N #1 \c_math_superscript_token
      \prg_return_true: \else: \prg_return_false: \fi:
  }
%    \end{macrocode}
% \end{macro}
%
% \begin{macro}[pTF]{\token_if_math_subscript:N}
%   Check if token is a math subscript token. We use the constant
%   \cs{c_math_subscript_token} for this.
%    \begin{macrocode}
\prg_new_conditional:Npnn \token_if_math_subscript:N #1 { p , T ,  F , TF }
  {
    \if_catcode:w \exp_not:N #1 \c_math_subscript_token
      \prg_return_true: \else: \prg_return_false: \fi:
  }
%    \end{macrocode}
% \end{macro}
%
% \begin{macro}[pTF]{\token_if_space:N}
%   Check if token is a space token. We use the constant
%   \cs{c_space_token} for this.
%    \begin{macrocode}
\prg_new_conditional:Npnn \token_if_space:N #1 { p , T ,  F , TF }
  {
    \if_catcode:w \exp_not:N #1 \c_space_token
      \prg_return_true: \else: \prg_return_false: \fi:
  }
%    \end{macrocode}
% \end{macro}
%
% \begin{macro}[pTF]{\token_if_letter:N}
%   Check if token is a letter token. We use the constant
%   \cs{c_catcode_letter_token} for this.
%    \begin{macrocode}
\prg_new_conditional:Npnn \token_if_letter:N #1 { p , T ,  F , TF }
  {
    \if_catcode:w \exp_not:N #1 \c_catcode_letter_token
      \prg_return_true: \else: \prg_return_false: \fi:
  }
%    \end{macrocode}
% \end{macro}
%
% \begin{macro}[pTF]{\token_if_other:N}
%   Check if token is an other char token. We use the constant
%   \cs{c_catcode_other_token} for this.
%    \begin{macrocode}
\prg_new_conditional:Npnn \token_if_other:N #1 { p , T ,  F , TF }
  {
    \if_catcode:w \exp_not:N #1 \c_catcode_other_token
      \prg_return_true: \else: \prg_return_false: \fi:
  }
%    \end{macrocode}
% \end{macro}
%
% \begin{macro}[pTF]{\token_if_active:N}
%   Check if token is an active char token. We use the constant
%   \cs{c_catcode_active_tl} for this. A technical point is that
%   \cs{c_catcode_active_tl} is in fact a macro expanding to
%   |\exp_not:N *|, where |*| is active.
%    \begin{macrocode}
\prg_new_conditional:Npnn \token_if_active:N #1 { p , T ,  F , TF }
  {
    \if_catcode:w \exp_not:N #1 \c_catcode_active_tl
      \prg_return_true: \else: \prg_return_false: \fi:
  }
%    \end{macrocode}
% \end{macro}
%
% \begin{macro}[pTF]{\token_if_eq_meaning:NN}
%   Check if the tokens |#1| and |#2| have same meaning.
%    \begin{macrocode}
\prg_new_conditional:Npnn \token_if_eq_meaning:NN #1#2 { p , T ,  F , TF }
  {
    \if_meaning:w  #1  #2
      \prg_return_true: \else: \prg_return_false: \fi:
  }
%    \end{macrocode}
% \end{macro}
%
% \begin{macro}[pTF]{\token_if_eq_catcode:NN}
%  Check if the tokens |#1| and |#2| have same category code.
%    \begin{macrocode}
\prg_new_conditional:Npnn \token_if_eq_catcode:NN #1#2 { p , T ,  F , TF }
  {
    \if_catcode:w \exp_not:N #1 \exp_not:N #2
      \prg_return_true: \else: \prg_return_false: \fi:
  }
%    \end{macrocode}
% \end{macro}
%
% \begin{macro}[pTF]{\token_if_eq_charcode:NN}
%  Check if the tokens |#1| and |#2| have same character code.
%    \begin{macrocode}
\prg_new_conditional:Npnn \token_if_eq_charcode:NN #1#2 { p , T ,  F , TF }
  {
    \if_charcode:w \exp_not:N #1 \exp_not:N #2
      \prg_return_true: \else: \prg_return_false: \fi:
  }
%    \end{macrocode}
% \end{macro}
%
% \begin{macro}[pTF]{\token_if_macro:N}
% \begin{macro}{\@@_if_macro_p:w}
%   When a token is a macro, \cs{token_to_meaning:N} always outputs
%   something like |\long macro:#1->#1| so we could naively check to
%   see if the meaning contains |->|. However, this can fail the five
%   \tn[no-index]{...mark} primitives, whose meaning has the form
%   |...mark:|\meta{user material}. The problem is that the
%   \meta{user material} can contain |->|.
%
%   However, only characters, macros, and marks can contain the colon
%   character. The idea is thus to grab until the first |:|, and analyse
%   what is left. However, macros can have any combination of |\long|,
%   |\protected| or |\outer| (not used in \LaTeX3) before the string
%   |macro:|. We thus only select the part of the meaning between
%   the first |ma| and the first following |:|. If this string is
%   |cro|, then we have a macro. If the string is |rk|, then we have
%   a mark. The string can also be |cro parameter character | for a
%   colon with a weird category code (namely the usual category code
%   of |#|). Otherwise, it is empty.
%
%   This relies on the fact that |\long|, |\protected|, |\outer|
%   cannot contain |ma|, regardless of the escape character, even if
%   the escape character is |m|\ldots{}
%
%   Both |ma| and |:| must be of category code $12$ (other), so are
%   detokenized.
%
%    \begin{macrocode}
\use:x
  {
    \prg_new_conditional:Npnn \exp_not:N \token_if_macro:N ##1
      { p , T ,  F , TF }
      {
        \exp_not:N \exp_after:wN \exp_not:N \@@_if_macro_p:w
        \exp_not:N \token_to_meaning:N ##1 \tl_to_str:n { ma : }
          \s_@@_stop
      }
    \cs_new:Npn \exp_not:N  \@@_if_macro_p:w
      ##1 \tl_to_str:n { ma } ##2 \c_colon_str ##3 \s_@@_stop
  }
      {
        \str_if_eq:nnTF { #2 } { cro }
          { \prg_return_true: }
          { \prg_return_false: }
      }
%    \end{macrocode}
%  \end{macro}
%  \end{macro}
%
% \begin{macro}[pTF]{\token_if_cs:N}
%   Check if token has same catcode as a control sequence. This
%   follows the same pattern as for \cs{token_if_letter:N} \emph{etc.}
%   We use \cs{scan_stop:} for this.
%    \begin{macrocode}
\prg_new_conditional:Npnn \token_if_cs:N #1 { p , T ,  F , TF }
  {
    \if_catcode:w \exp_not:N #1 \scan_stop:
      \prg_return_true: \else: \prg_return_false: \fi:
  }
%    \end{macrocode}
% \end{macro}
%
% \begin{macro}[pTF]{\token_if_expandable:N}
%   Check if token is expandable. We use the fact that \TeX{}
%   temporarily converts \cs{exp_not:N} \meta{token} into \cs{scan_stop:}
%   if \meta{token} is expandable.  An \texttt{undefined} token is not
%   considered as expandable.  No problem nesting the conditionals,
%   since the third |#1| is only skipped if it is non-expandable (hence
%   not part of \TeX{}'s conditional apparatus).
%    \begin{macrocode}
\prg_new_conditional:Npnn \token_if_expandable:N #1 { p , T ,  F , TF }
  {
    \exp_after:wN \if_meaning:w \exp_not:N #1 #1
      \prg_return_false:
    \else:
      \if_cs_exist:N #1
        \prg_return_true:
      \else:
        \prg_return_false:
      \fi:
    \fi:
  }
%    \end{macrocode}
% \end{macro}
%
% \begin{macro}
%   {
%     \@@_delimit_by_char":w,
%     \@@_delimit_by_count:w,
%     \@@_delimit_by_dimen:w,
%     \@@_delimit_by_ font:w,
%     \@@_delimit_by_macro:w,
%     \@@_delimit_by_muskip:w,
%     \@@_delimit_by_skip:w,
%     \@@_delimit_by_toks:w,
%   }
%   These auxiliary functions are used below to define some
%   conditionals which detect whether the \tn{meaning} of their
%   argument begins with a particular string.  Each auxiliary takes an
%   argument delimited by a string, a second one delimited by
%   \cs{s_@@_stop}, and returns the first one and its delimiter.
%   This result is eventually compared to another string.
%   Note that the ``font'' auxiliary is delimited by a space followed by
%   ``\texttt{font}''.  This avoids an unnecessary check for the
%   \tn{font} primitive below.
%    \begin{macrocode}
\group_begin:
\cs_set_protected:Npn \@@_tmp:w #1
  {
    \use:x
      {
        \cs_new:Npn \exp_not:c { @@_delimit_by_ #1 :w }
            ####1 \tl_to_str:n {#1} ####2 \s_@@_stop
          { ####1 \tl_to_str:n {#1} }
      }
  }
\@@_tmp:w { char" }
\@@_tmp:w { count }
\@@_tmp:w { dimen }
\@@_tmp:w { ~ font }
\@@_tmp:w { macro }
\@@_tmp:w { muskip }
\@@_tmp:w { skip }
\@@_tmp:w { toks }
\group_end:
%    \end{macrocode}
% \end{macro}
%
% \begin{macro}[pTF]
%   {
%     \token_if_chardef:N,               \token_if_mathchardef:N,
%                                        \token_if_long_macro:N,
%     \token_if_protected_macro:N,       \token_if_protected_long_macro:N,
%     \token_if_font_selection:N,
%     \token_if_dim_register:N,          \token_if_int_register:N,
%     \token_if_muskip_register:N,
%     \token_if_skip_register:N,         \token_if_toks_register:N,
%   }
%   Each of these conditionals tests whether its argument's
%   \tn{meaning} starts with a given string.  This is essentially done
%   by having an auxiliary grab an argument delimited by the string and
%   testing whether the argument was empty.  Of course, a copy of this
%   string must first be added to the end of the \tn{meaning} to avoid
%   a runaway argument in case it does not contain the string.  Two
%   complications arise.  First, the escape character is not fixed, and
%   cannot be included in the delimiter of the auxiliary function (this
%   function cannot be defined on the fly because tests must remain
%   expandable): instead the first argument of the auxiliary (plus the
%   delimiter to avoid complications with trailing spaces) is compared
%   using \cs{str_if_eq:eeTF} to the result of applying
%   \cs{token_to_str:N} to a control sequence.  Second, the
%   \tn{meaning} of primitives such as \tn{dimen} or \tn{dimendef}
%   starts in the same way as registers such as
%   \tn{dimen}\texttt{123}, so they must be tested for.
%
%   Characters used as delimiters must have catcode~$12$
%   and are obtained through \cs{tl_to_str:n}.  This requires doing all
%   definitions within \texttt{x}-expansion.  The temporary function
%   \cs{@@_tmp:w} used to define each conditional receives three
%   arguments: the name of the conditional, the auxiliary's delimiter
%   (also used to name the auxiliary), and the string to which one
%   compares the auxiliary's result.  Note that the \tn{meaning} of a
%   protected long macro starts with |\protected\long macro|, with no
%   space after |\protected| but a space after |\long|, hence the
%   mixture of \cs{token_to_str:N} and \cs{tl_to_str:n}.
%
%   For the first six conditionals, \cs{cs_if_exist:cT} turns out to
%   be \texttt{false} (thanks to the leading space for \texttt{font}),
%   and the code boils down to a string comparison
%   between the result of the auxiliary on the \tn{meaning} of the
%   conditional's argument~|####1|, and~|#3|.  Both are evaluated at
%   run-time, as this is important to get the correct escape character.
%
%   The other five conditionals have additional code that compares the
%   argument~|####1| to two \TeX{} primitives which would wrongly be
%   recognized as registers otherwise.  Despite using \TeX{}'s
%   primitive conditional construction, this does not break
%   when~|####1| is itself a conditional, because branches of the
%   conditionals are only skipped if |####1|~is one of the two
%   primitives that are tested for (which are not \TeX{} conditionals).
%    \begin{macrocode}
\group_begin:
\cs_set_protected:Npn \@@_tmp:w #1#2#3
  {
    \use:x
      {
        \prg_new_conditional:Npnn \exp_not:c { token_if_ #1 :N } ####1
          { p , T ,  F , TF }
          {
            \cs_if_exist:cT { tex_ #2 :D }
              {
                \exp_not:N \if_meaning:w ####1 \exp_not:c { tex_ #2 :D }
                \exp_not:N \prg_return_false:
                \exp_not:N \else:
                \exp_not:N \if_meaning:w ####1 \exp_not:c { tex_ #2 def:D }
                \exp_not:N \prg_return_false:
                \exp_not:N \else:
              }
            \exp_not:N \str_if_eq:eeTF
              {
                \exp_not:N \exp_after:wN
                \exp_not:c { @@_delimit_by_ #2 :w }
                \exp_not:N \token_to_meaning:N ####1
                ? \tl_to_str:n {#2} \s_@@_stop
              }
              { \exp_not:n {#3} }
              { \exp_not:N \prg_return_true: }
              { \exp_not:N \prg_return_false: }
            \cs_if_exist:cT { tex_ #2 :D }
              {
                \exp_not:N \fi:
                \exp_not:N \fi:
              }
          }
      }
  }
\@@_tmp:w { chardef } { char" } { \token_to_str:N \char" }
\@@_tmp:w { mathchardef } { char" } { \token_to_str:N \mathchar" }
\@@_tmp:w { long_macro } { macro } { \tl_to_str:n { \long } macro }
\@@_tmp:w { protected_macro } { macro }
  { \tl_to_str:n { \protected } macro }
\@@_tmp:w { protected_long_macro } { macro }
  { \token_to_str:N \protected \tl_to_str:n { \long } macro }
\@@_tmp:w { font_selection } { ~ font } { select ~ font }
\@@_tmp:w { dim_register } { dimen } { \token_to_str:N \dimen }
\@@_tmp:w { int_register } { count } { \token_to_str:N \count }
\@@_tmp:w { muskip_register } { muskip } { \token_to_str:N \muskip }
\@@_tmp:w { skip_register } { skip } { \token_to_str:N \skip }
\@@_tmp:w { toks_register } { toks } { \token_to_str:N \toks }
\group_end:
%    \end{macrocode}
% \end{macro}
%
% \begin{macro}[pTF]{\token_if_primitive:N}
% \begin{macro}{\@@_if_primitive:NNw,
%     \@@_if_primitive_space:w,
%     \@@_if_primitive_nullfont:N,
%     \@@_if_primitive_loop:N,
%     \@@_if_primitive:Nw,
%     \@@_if_primitive_undefined:N,
%     \@@_if_primitive_lua:N}
%^^A See http://groups.google.com/group/comp.text.tex/browse_thread/thread/0a72666873f8753d#
%
%   We filter out macros first, because they cause endless trouble later
%   otherwise.
%
%   Primitives are almost distinguished by the fact that the result
%   of \cs{token_to_meaning:N} is formed from letters only. Every other
%   token has either a space (e.g., |the letter A|), a digit
%   (e.g., |\count123|) or a double quote (e.g., |\char"A|).
%
%   Ten exceptions: on the one hand, \cs{tex_undefined:D} is not a
%   primitive, but its meaning is |undefined|, only letters;
%   on the other hand, \tn{space}, \tn{italiccorr},
%   \tn{hyphen}, \tn{firstmark}, \tn{topmark},
%   \tn{botmark}, \tn{splitfirstmark}, \tn{splitbotmark},
%   and \tn{nullfont} are primitives, but have non-letters
%   in their meaning.
%
%   We start by removing the two first (non-space) characters from
%   the meaning. This removes the escape character (which may be
%   nonexistent depending on \tn{endlinechar}), and takes care
%   of three of the exceptions: \tn{space}, \tn{italiccorr}
%   and \tn{hyphen}, whose meaning is at most two characters.
%   This leaves a string terminated by some |:|, and \cs{s_@@_stop}.
%
%   The meaning of each one of the five \tn[no-index]{...mark} primitives
%   has the form \meta{letters}|:|\meta{user material}. In other words,
%   the first non-letter is a colon. We remove everything after the first
%   colon.
%
%   We are now left with a string, which we must analyze. For primitives,
%   it contains only letters. For non-primitives, it contains either
%   |"|, or a space, or a digit. Two exceptions remain: \cs{tex_undefined:D},
%   which is not a primitive, and \tn{nullfont}, which is a primitive.
%
%   Spaces cannot be grabbed in an undelimited way, so we check them
%   separately. If there is a space, we test for \tn{nullfont}.
%   Otherwise, we go through characters one by one, and stop at the
%   first character less than |`A| (this is not quite a test for
%   \enquote{only letters}, but is close enough to work in this context).
%   If this first character is |:| then we have a primitive, or
%   \cs{tex_undefined:D}, and if it is |"| or a digit, then the token
%   is not a primitive.
%
%   For \LuaTeX{} we use a different implementation which just looks at the
%   command code for the token and compaes it to a list of non-primitives.
%   Again, \tn{nullfont} is a special case because it is the only primitive
%   with the normally non-primitive |set_font| command code.
%
%   In LuaMeta\TeX{} some of the command names are different, so we check for
%   both versions. The first one is always the \LuaTeX{} version.
%
%    \begin{macrocode}
\sys_if_engine_luatex:TF
  {
%</tex>
%<*lua>
do
  local get_next = token.get_next
  local get_command = token.get_command
  local get_index = token.get_index
  local get_mode = token.get_mode or token.get_index
  local cmd = command_id
  local set_font = cmd'get_font'
  local biggest_char = token.biggest_char and token.biggest_char()
                    or status.getconstants().max_character_code

  local mode_below_biggest_char = {}
  local index_not_nil = {}
  local mode_not_null = {}
  local non_primitive = {
    [cmd'left_brace'] = true,
    [cmd'right_brace'] = true,
    [cmd'math_shift'] = true,
    [cmd'mac_param' or cmd'parameter'] = mode_below_biggest_char,
    [cmd'sup_mark' or cmd'superscript'] = true,
    [cmd'sub_mark' or cmd'subscript'] = true,
    [cmd'endv' or cmd'ignore'] = true,
    [cmd'spacer'] = true,
    [cmd'letter'] = true,
    [cmd'other_char'] = true,
    [cmd'tab_mark' or cmd'alignment_tab'] = mode_below_biggest_char,
    [cmd'char_given'] = true,
    [cmd'math_given' or 'math_char_given'] = true,
    [cmd'xmath_given' or 'math_char_xgiven'] = true,
    [cmd'set_font'] = mode_not_null,
    [cmd'undefined_cs'] = true,
    [cmd'call'] = true,
    [cmd'long_call' or cmd'protected_call'] = true,
    [cmd'outer_call' or cmd'tolerant_call'] = true,
    [cmd'long_outer_call' or cmd'tolerant_protected_call'] = true,
    [cmd'assign_glue' or cmd'register_glue'] = index_not_nil,
    [cmd'assign_mu_glue' or cmd'register_mu_glue'] = index_not_nil,
    [cmd'assign_toks' or cmd'register_toks'] = index_not_nil,
    [cmd'assign_int' or cmd'register_int'] = index_not_nil,
    [cmd'assign_attr' or cmd'register_attribute'] = true,
    [cmd'assign_dimen' or cmd'register_dimen'] = index_not_nil,
  }

  luacmd("@@_if_primitive_lua:N", function()
    local tok = get_next()
    local is_non_primitive = non_primitive[get_command(tok)]
    return put_next(
           is_non_primitive == true
             and false_tok
        or is_non_primitive == nil
             and true_tok
        or is_non_primitive == mode_not_null
             and (get_mode(tok) == 0 and true_tok or false_tok)
        or is_non_primitive == index_not_nil
             and (get_index(tok) and false_tok or true_tok)
        or is_non_primitive == mode_below_biggest_char
             and (get_mode(tok) > biggest_char and true_tok or false_tok))
  end, "global")
end
%</lua>
%<*tex>
    \prg_new_conditional:Npnn \token_if_primitive:N #1 { p , T , F , TF }
      {
        \@@_if_primitive_lua:N #1
      }
  }
  {
    \tex_chardef:D \c_@@_A_int = `A ~ %
    \use:x
      {
        \prg_new_conditional:Npnn \exp_not:N \token_if_primitive:N ##1
          { p , T , F , TF }
          {
            \exp_not:N \token_if_macro:NTF ##1
              \exp_not:N \prg_return_false:
              {
                \exp_not:N \exp_after:wN \exp_not:N \@@_if_primitive:NNw
                \exp_not:N \token_to_meaning:N ##1
                  \tl_to_str:n { : : : } \s_@@_stop ##1
              }
          }
        \cs_new:Npn \exp_not:N \@@_if_primitive:NNw
          ##1##2 ##3 \c_colon_str ##4 \s_@@_stop
          {
            \exp_not:N \tl_if_empty:oTF
              { \exp_not:N \@@_if_primitive_space:w ##3 ~ }
              {
                \exp_not:N \@@_if_primitive_loop:N ##3
                  \c_colon_str \s_@@_stop
              }
              { \exp_not:N \@@_if_primitive_nullfont:N }
          }
      }
    \cs_new:Npn \@@_if_primitive_space:w #1 ~ { }
    \cs_new:Npn \@@_if_primitive_nullfont:N #1
      {
        \if_meaning:w \tex_nullfont:D #1
          \prg_return_true:
        \else:
          \prg_return_false:
        \fi:
      }
    \cs_new:Npn \@@_if_primitive_loop:N #1
      {
        \if_int_compare:w `#1 < \c_@@_A_int %
          \exp_after:wN \@@_if_primitive:Nw
          \exp_after:wN #1
        \else:
          \exp_after:wN \@@_if_primitive_loop:N
        \fi:
      }
    \cs_new:Npn \@@_if_primitive:Nw #1 #2 \s_@@_stop
      {
        \if:w : #1
          \exp_after:wN \@@_if_primitive_undefined:N
        \else:
          \prg_return_false:
          \exp_after:wN \use_none:n
        \fi:
      }
    \cs_new:Npn \@@_if_primitive_undefined:N #1
      {
        \if_cs_exist:N #1
          \prg_return_true:
        \else:
          \prg_return_false:
        \fi:
      }
  }
%    \end{macrocode}
% \end{macro}
% \end{macro}
%
% \begin{macro}[EXP, noTF]
%   {\token_case_catcode:Nn, \token_case_charcode:Nn, \token_case_meaning:Nn}
% \begin{macro}[EXP]{\@@_case:NNnTF, \@@_case:NNw, \@@_case_end:nw}
%   The aim here is to allow the case statement to be evaluated
%   using a known number of expansion steps (two), and without
%   needing to use an explicit \enquote{end of recursion} marker.
%   That is achieved by using the test input as the final case,
%   as this is always true. The trick is then to tidy up
%   the output such that the appropriate case code plus either
%   the \texttt{true} or \texttt{false} branch code is inserted.
%    \begin{macrocode}
\cs_new:Npn \token_case_catcode:Nn #1#2
  { \exp:w \@@_case:NNnTF \token_if_eq_catcode:NNTF #1 {#2} { } { } }
\cs_new:Npn \token_case_catcode:NnT #1#2#3
  { \exp:w \@@_case:NNnTF \token_if_eq_catcode:NNTF #1 {#2} {#3} { } }
\cs_new:Npn \token_case_catcode:NnF #1#2
  { \exp:w \@@_case:NNnTF \token_if_eq_catcode:NNTF #1 {#2} { } }
\cs_new:Npn \token_case_catcode:NnTF
  { \exp:w \@@_case:NNnTF \token_if_eq_catcode:NNTF }
\cs_new:Npn \token_case_charcode:Nn #1#2
  { \exp:w \@@_case:NNnTF \token_if_eq_charcode:NNTF #1 {#2} { } { } }
\cs_new:Npn \token_case_charcode:NnT #1#2#3
  { \exp:w \@@_case:NNnTF \token_if_eq_charcode:NNTF #1 {#2} {#3} { } }
\cs_new:Npn \token_case_charcode:NnF #1#2
  { \exp:w \@@_case:NNnTF \token_if_eq_charcode:NNTF #1 {#2} { } }
\cs_new:Npn \token_case_charcode:NnTF
  { \exp:w \@@_case:NNnTF \token_if_eq_charcode:NNTF }
\cs_new:Npn \token_case_meaning:Nn #1#2
  { \exp:w \@@_case:NNnTF \token_if_eq_meaning:NNTF #1 {#2} { } { } }
\cs_new:Npn \token_case_meaning:NnT #1#2#3
  { \exp:w \@@_case:NNnTF \token_if_eq_meaning:NNTF #1 {#2} {#3} { } }
\cs_new:Npn \token_case_meaning:NnF #1#2
  { \exp:w \@@_case:NNnTF \token_if_eq_meaning:NNTF #1 {#2} { } }
\cs_new:Npn \token_case_meaning:NnTF
  { \exp:w \@@_case:NNnTF \token_if_eq_meaning:NNTF }
\cs_new:Npn \@@_case:NNnTF #1#2#3#4#5
  {
    \@@_case:NNw #1 #2 #3 #2 { }
    \s_@@_mark {#4}
    \s_@@_mark {#5}
    \s_@@_stop
  }
\cs_new:Npn \@@_case:NNw #1#2#3#4
  {
    #1 #2 #3
      { \@@_case_end:nw {#4} }
      { \@@_case:NNw #1 #2 }
  }
%    \end{macrocode}
%   To tidy up the recursion, there are two outcomes. If there was a hit to
%   one of the cases searched for, then |#1| is the code to insert,
%   |#2| is the \emph{next} case to check on and |#3| is all of
%   the rest of the cases code. That means that |#4| is the \texttt{true}
%   branch code, and |#5| tidies up the spare \cs{s_@@_mark} and the
%   \texttt{false} branch. On the other hand, if none of the cases matched
%   then we arrive here using the \enquote{termination} case of comparing
%   the search with itself. That means that |#1| is empty, |#2| is
%   the first \cs{s_@@_mark} and so |#4| is the \texttt{false} code (the
%   \texttt{true} code is mopped up by |#3|).
%    \begin{macrocode}
\cs_new:Npn \@@_case_end:nw #1#2#3 \s_@@_mark #4#5 \s_@@_stop
  { \exp_end: #1 #4 }
%    \end{macrocode}
% \end{macro}
% \end{macro}
%
% \subsection{Peeking ahead at the next token}
%
%    \begin{macrocode}
%<@@=peek>
%    \end{macrocode}
%
% Peeking ahead is implemented using a two part mechanism. The
% outer level provides a defined interface to the lower level material.
% This allows a large amount of code to be shared. There are four
% cases:
% \begin{enumerate}
%   \item peek at the next token;
%   \item peek at the next non-space token;
%   \item peek at the next token and remove it;
%   \item peek at the next non-space token and remove it.
% \end{enumerate}
%
% \begin{variable}{\l_peek_token}
% \begin{variable}{\g_peek_token}
%   Storage tokens which are publicly documented: the token peeked.
%    \begin{macrocode}
\cs_new_eq:NN \l_peek_token ?
\cs_new_eq:NN \g_peek_token ?
%    \end{macrocode}
% \end{variable}
% \end{variable}
%
% \begin{variable}{\l_@@_search_token}
%   The token to search for as an implicit token:
%   \emph{cf.}~\cs{l_@@_search_tl}.
%    \begin{macrocode}
\cs_new_eq:NN \l_@@_search_token ?
%    \end{macrocode}
% \end{variable}
%
% \begin{variable}{\l_@@_search_tl}
%   The token to search for as an explicit token:
%   \emph{cf.}~\cs{l_@@_search_token}.
%    \begin{macrocode}
\tl_new:N \l_@@_search_tl
%    \end{macrocode}
% \end{variable}
%
% \begin{macro}
%   {\@@_true:w, \@@_true_aux:w, \@@_false:w, \@@_tmp:w}
%   Functions used by the branching and space-stripping code.
%    \begin{macrocode}
\cs_new:Npn \@@_true:w  { }
\cs_new:Npn \@@_true_aux:w  { }
\cs_new:Npn \@@_false:w { }
\cs_new:Npn \@@_tmp:w { }
%    \end{macrocode}
% \end{macro}
%
% \begin{variable}{\s_@@_mark,\s_@@_stop}
%   Internal scan marks.
%    \begin{macrocode}
\scan_new:N \s_@@_mark
\scan_new:N \s_@@_stop
%    \end{macrocode}
% \end{variable}
%
% \begin{macro}[EXP]{\@@_use_none_delimit_by_s_stop:w}
%   Functions to gobble up to a scan mark.
%    \begin{macrocode}
\cs_new:Npn \@@_use_none_delimit_by_s_stop:w #1 \s_@@_stop { }
%    \end{macrocode}
% \end{macro}
%
% \begin{macro}{\peek_after:Nw}
% \begin{macro}{\peek_gafter:Nw}
%   Simple wrappers for \tn{futurelet}: no arguments absorbed
%   here.
%    \begin{macrocode}
\cs_new_protected:Npn \peek_after:Nw
  { \tex_futurelet:D \l_peek_token }
\cs_new_protected:Npn \peek_gafter:Nw
  { \tex_global:D \tex_futurelet:D \g_peek_token }
%    \end{macrocode}
% \end{macro}
% \end{macro}
%
% \begin{macro}{\@@_true_remove:w}
%   A function to remove the next token and then regain control.
%    \begin{macrocode}
\cs_new_protected:Npn \@@_true_remove:w
  {
    \tex_afterassignment:D \@@_true_aux:w
    \cs_set_eq:NN \@@_tmp:w
  }
%    \end{macrocode}
% \end{macro}
%
% \begin{macro}{\peek_remove_spaces:n, \@@_remove_spaces:}
%   Repeatedly use \cs{@@_true_remove:w} to remove a space and call
%   \cs{@@_true_aux:w}.
%    \begin{macrocode}
\cs_new_protected:Npn \peek_remove_spaces:n #1
  {
    \cs_set:Npx \@@_false:w { \exp_not:n {#1} }
    \group_align_safe_begin:
    \cs_set:Npn \@@_true_aux:w { \peek_after:Nw \@@_remove_spaces: }
    \@@_true_aux:w
  }
\cs_new_protected:Npn \@@_remove_spaces:
  {
    \if_meaning:w \l_peek_token \c_space_token
      \exp_after:wN \@@_true_remove:w
    \else:
      \group_align_safe_end:
      \exp_after:wN \@@_false:w
    \fi:
  }
%    \end{macrocode}
% \end{macro}
%
% \begin{macro}{\peek_remove_filler:n}
% \begin{macro}{\@@_remove_filler:w}
% \begin{macro}{\@@_remove_filler:}
% \begin{macro}{\@@_remove_filler_expand:w}
%   Here we expand the input, removing spaces and \cs{scan_stop:} tokens until
%   we reach a non-expandable token. At that stage we re-insert the payload.
%   To deal with the problem of |&| tokens, we have to put the align-safe
%   group in the correct place.
%    \begin{macrocode}
\cs_new_protected:Npn \peek_remove_filler:n #1
  {
    \cs_set:Npn \@@_true_aux:w { \@@_remove_filler:w }
    \cs_set:Npx \@@_false:w
      {
        \exp_not:N \group_align_safe_end:
        \exp_not:n {#1}
      }
    \group_align_safe_begin:
    \@@_remove_filler:w
  }
\cs_new_protected:Npn \@@_remove_filler:w
  {
    \exp_after:wN \peek_after:Nw \exp_after:wN \@@_remove_filler:
    \exp:w \exp_end_continue_f:w
  }
%    \end{macrocode}
%   Here we can nest conditionals as \cs{l_peek_token} is only skipped over in
%   the nested one if it's a space: no problems with conditionals or outer
%   tokens.
%    \begin{macrocode}
\cs_new_protected:Npn \@@_remove_filler:
  {
    \if_catcode:w \exp_not:N \l_peek_token \c_space_token
      \exp_after:wN \@@_true_remove:w
    \else:
      \if_meaning:w \l_peek_token \scan_stop:
        \exp_after:wN \exp_after:wN \exp_after:wN
          \@@_true_remove:w
      \else:
        \exp_after:wN \exp_after:wN \exp_after:wN
          \@@_remove_filler_expand:w
      \fi:
    \fi:
  }
%    \end{macrocode}
%   To deal with undefined control sequences in the same way \TeX{} does,
%   we need to check for expansion manually.
%    \begin{macrocode}
\cs_new_protected:Npn \@@_remove_filler_expand:w
  {
    \exp_after:wN \if_meaning:w \exp_not:N \l_peek_token \l_peek_token
      \exp_after:wN \@@_false:w
    \else:
      \exp_after:wN \@@_remove_filler:w
    \fi:
  }
%    \end{macrocode}
% \end{macro}
% \end{macro}
% \end{macro}
% \end{macro}
%
% \begin{macro}{\@@_token_generic_aux:NNNTF}
%   The generic functions store the test token in both implicit and
%   explicit modes, and the \texttt{true} and \texttt{false} code as
%   token lists, more or less. The two branches have to be absorbed here
%   as the input stream needs to be cleared for the peek function itself.
%   Here, |#1| is \cs{@@_true_remove:w} when removing the token and
%   \cs{@@_true_aux:w} otherwise.
%    \begin{macrocode}
\cs_new_protected:Npn \@@_token_generic_aux:NNNTF #1#2#3#4#5
  {
    \group_align_safe_begin:
    \cs_set_eq:NN \l_@@_search_token #3
    \tl_set:Nn \l_@@_search_tl {#3}
    \cs_set:Npx \@@_true_aux:w
      {
        \exp_not:N \group_align_safe_end:
        \exp_not:n {#4}
      }
    \cs_set_eq:NN \@@_true:w #1
    \cs_set:Npx \@@_false:w
      {
        \exp_not:N \group_align_safe_end:
        \exp_not:n {#5}
      }
    \peek_after:Nw #2
  }
%    \end{macrocode}
% \end{macro}
%
% \begin{macro}[TF]{\@@_token_generic:NN, \@@_token_remove_generic:NN}
%   For token removal there needs to be a call to the auxiliary
%   function which does the work.
%    \begin{macrocode}
\cs_new_protected:Npn \@@_token_generic:NNTF
  { \@@_token_generic_aux:NNNTF \@@_true_aux:w }
\cs_new_protected:Npn \@@_token_generic:NNT #1#2#3
  { \@@_token_generic:NNTF #1 #2 {#3} { } }
\cs_new_protected:Npn \@@_token_generic:NNF #1#2#3
  { \@@_token_generic:NNTF #1 #2 { } {#3} }
\cs_new_protected:Npn \@@_token_remove_generic:NNTF
  { \@@_token_generic_aux:NNNTF \@@_true_remove:w }
\cs_new_protected:Npn \@@_token_remove_generic:NNT #1#2#3
  { \@@_token_remove_generic:NNTF #1 #2 {#3} { } }
\cs_new_protected:Npn \@@_token_remove_generic:NNF #1#2#3
  { \@@_token_remove_generic:NNTF #1 #2 { } {#3} }
%    \end{macrocode}
% \end{macro}
%
% \begin{macro}{\@@_execute_branches_meaning:}
%   The meaning test is straight forward.
%    \begin{macrocode}
\cs_new:Npn \@@_execute_branches_meaning:
  {
    \if_meaning:w \l_peek_token \l_@@_search_token
      \exp_after:wN \@@_true:w
    \else:
      \exp_after:wN \@@_false:w
    \fi:
  }
%    \end{macrocode}
% \end{macro}
%
% \begin{macro}{\@@_execute_branches_catcode:, \@@_execute_branches_charcode:}
% \begin{macro}
%   {
%     \@@_execute_branches_catcode_aux:    ,
%     \@@_execute_branches_catcode_auxii:N ,
%     \@@_execute_branches_catcode_auxiii:
%   }
%   The catcode and charcode tests are very similar, and in order to use
%   the same auxiliaries we do something a little bit odd, firing
%   \cs{if_catcode:w} and \cs{if_charcode:w} before finding the operands
%   for those tests, which are only given in the |auxii:N| and
%   |auxiii:| auxiliaries.  For our purposes, three kinds of tokens may
%   follow the peeking function:
%   \begin{itemize}
%     \item control sequences which are not equal to a non-active
%       character token (\emph{e.g.}, macro, primitive);
%     \item active characters which are not equal to a non-active
%       character token (\emph{e.g.}, macro, primitive);
%     \item explicit non-active character tokens, or control sequences
%       or active characters set equal to a non-active character token.
%   \end{itemize}
%   The first two cases are not distinguishable simply using \TeX{}'s
%   \tn{futurelet}, because we can only access the \tn{meaning} of
%   tokens in that way.  In those cases, detected thanks to a
%   comparison with \cs{scan_stop:}, we grab the following token, and
%   compare it explicitly with the explicit search token stored in
%   \cs{l_@@_search_tl}.  The \cs{exp_not:N} prevents outer macros
%   (coming from non-\LaTeX3 code) from blowing up.  In the third case,
%   \cs{l_peek_token} is good enough for the test, and we compare it
%   again with the explicit search token.  Just like the peek token, the
%   search token may be of any of the three types above, hence the need
%   to use the explicit token that was given to the peek function.
%    \begin{macrocode}
\cs_new:Npn \@@_execute_branches_catcode:
  { \if_catcode:w \@@_execute_branches_catcode_aux: }
\cs_new:Npn \@@_execute_branches_charcode:
  { \if_charcode:w \@@_execute_branches_catcode_aux: }
\cs_new:Npn \@@_execute_branches_catcode_aux:
  {
        \if_catcode:w \exp_not:N \l_peek_token \scan_stop:
          \exp_after:wN \exp_after:wN
          \exp_after:wN \@@_execute_branches_catcode_auxii:N
          \exp_after:wN \exp_not:N
        \else:
          \exp_after:wN \@@_execute_branches_catcode_auxiii:
        \fi:
  }
\cs_new:Npn \@@_execute_branches_catcode_auxii:N #1
  {
        \exp_not:N #1
        \exp_after:wN \exp_not:N \l_@@_search_tl
      \exp_after:wN \@@_true:w
    \else:
      \exp_after:wN \@@_false:w
    \fi:
    #1
  }
\cs_new:Npn \@@_execute_branches_catcode_auxiii:
  {
        \exp_not:N \l_peek_token
        \exp_after:wN \exp_not:N \l_@@_search_tl
      \exp_after:wN \@@_true:w
    \else:
      \exp_after:wN \@@_false:w
    \fi:
  }
%    \end{macrocode}
% \end{macro}
% \end{macro}
%
% \begin{macro}[TF]
%   {
%     \peek_catcode:N,
%     \peek_catcode_remove:N,
%     \peek_charcode:N,
%     \peek_charcode_remove:N,
%     \peek_meaning:N,
%     \peek_meaning_remove:N,
%   }
%   The public functions themselves cannot be defined using
%   \cs{prg_new_conditional:Npnn}.  Instead, the |TF|, |T|, |F| variants
%   are defined in terms of corresponding variants of
%   \cs{@@_token_generic:NNTF} or \cs{@@_token_remove_generic:NNTF},
%   with first argument one of \cs{@@_execute_branches_catcode:},
%   \cs{@@_execute_branches_charcode:}, or
%   \cs{@@_execute_branches_meaning:}.
%    \begin{macrocode}
\tl_map_inline:nn { { catcode } { charcode } { meaning } }
  {
    \tl_map_inline:nn { { } { _remove } }
      {
        \tl_map_inline:nn { { TF } { T } { F } }
          {
            \cs_new_protected:cpx { peek_ #1 ##1 :N ####1 }
              {
                \exp_not:c { @@_token ##1 _generic:NN ####1 }
                \exp_not:c { @@_execute_branches_ #1 : }
              }
          }
      }
  }
%    \end{macrocode}
% \end{macro}
%
% \begin{macro}[TF]{\peek_N_type:}
% \begin{macro}
%   {\@@_execute_branches_N_type:, \@@_N_type:w, \@@_N_type_aux:nnw}
%   All tokens are \texttt{N}-type tokens, except in four cases:
%   begin-group tokens, end-group tokens, space tokens with character
%   code~$32$, and outer tokens.  Since \cs{l_peek_token} might be
%   outer, we cannot use the convenient \cs{bool_if:nTF} function, and
%   must resort to the old trick of using \tn{ifodd} to expand a set of
%   tests.  The \texttt{false} branch of this test is taken if the token
%   is one of the first three kinds of non-\texttt{N}-type tokens
%   (explicit or implicit), thus we call \cs{@@_false:w}.  In the
%   \texttt{true} branch, we must detect outer tokens, without impacting
%   performance too much for non-outer tokens.  The first filter is to
%   search for \texttt{outer} in the \tn{meaning} of \cs{l_peek_token}.
%   If that is absent, \cs{@@_use_none_delimit_by_s_stop:w} cleans up, and
%   we call \cs{@@_true:w}.  Otherwise, the token can be a non-outer
%   macro or a primitive mark whose parameter or replacement text
%   contains \texttt{outer}, it can be the primitive \tn{outer}, or it
%   can be an outer token.  Macros and marks would have \texttt{ma} in
%   the part before the first occurrence of \texttt{outer}; the meaning
%   of \tn{outer} has nothing after \texttt{outer}, contrarily to outer
%   macros; and that covers all cases, calling \cs{@@_true:w} or
%   \cs{@@_false:w} as appropriate.  Here, there is no \meta{search
%     token}, so we feed a dummy \cs{scan_stop:} to the
%   \cs{@@_token_generic:NNTF} function.
%    \begin{macrocode}
\group_begin:
  \cs_set_protected:Npn \@@_tmp:w #1 \s_@@_stop
    {
      \cs_new_protected:Npn \@@_execute_branches_N_type:
        {
          \if_int_odd:w
              \if_catcode:w \exp_not:N \l_peek_token {   \c_zero_int \fi:
              \if_catcode:w \exp_not:N \l_peek_token }   \c_zero_int \fi:
              \if_meaning:w \l_peek_token \c_space_token \c_zero_int \fi:
              \c_one_int
            \exp_after:wN \@@_N_type:w
              \token_to_meaning:N \l_peek_token
              \s_@@_mark \@@_N_type_aux:nnw
              #1 \s_@@_mark \@@_use_none_delimit_by_s_stop:w
              \s_@@_stop
            \exp_after:wN \@@_true:w
          \else:
            \exp_after:wN \@@_false:w
          \fi:
        }
      \cs_new_protected:Npn \@@_N_type:w ##1 #1 ##2 \s_@@_mark ##3
        { ##3 {##1} {##2} }
    }
  \exp_after:wN \@@_tmp:w \tl_to_str:n { outer } \s_@@_stop
\group_end:
\cs_new_protected:Npn \@@_N_type_aux:nnw #1 #2 #3 \fi:
  {
    \fi:
    \tl_if_in:noTF {#1} { \tl_to_str:n {ma} }
      { \@@_true:w }
      { \tl_if_empty:nTF {#2} { \@@_true:w } { \@@_false:w } }
  }
\cs_new_protected:Npn \peek_N_type:TF
  {
    \@@_token_generic:NNTF
      \@@_execute_branches_N_type: \scan_stop:
  }
\cs_new_protected:Npn \peek_N_type:T
  { \@@_token_generic:NNT \@@_execute_branches_N_type: \scan_stop: }
\cs_new_protected:Npn \peek_N_type:F
  { \@@_token_generic:NNF \@@_execute_branches_N_type: \scan_stop: }
%    \end{macrocode}
% \end{macro}
% \end{macro}
%
%    \begin{macrocode}
%</tex>
%    \end{macrocode}
%
%    \begin{macrocode}
%</package>
%    \end{macrocode}
%
% \end{implementation}
%
% \PrintIndex
