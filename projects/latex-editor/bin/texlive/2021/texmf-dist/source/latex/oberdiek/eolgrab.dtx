% \iffalse meta-comment
%
% File: eolgrab.dtx
% Version: 2016/05/16 v1.1
% Info: Catch arguments delimited by end of line
%
% Copyright (C)
%    2011 Heiko Oberdiek
%    2016-2019 Oberdiek Package Support Group
%    https://github.com/ho-tex/oberdiek/issues
%
% This work may be distributed and/or modified under the
% conditions of the LaTeX Project Public License, either
% version 1.3c of this license or (at your option) any later
% version. This version of this license is in
%    https://www.latex-project.org/lppl/lppl-1-3c.txt
% and the latest version of this license is in
%    https://www.latex-project.org/lppl.txt
% and version 1.3 or later is part of all distributions of
% LaTeX version 2005/12/01 or later.
%
% This work has the LPPL maintenance status "maintained".
%
% The Current Maintainers of this work are
% Heiko Oberdiek and the Oberdiek Package Support Group
% https://github.com/ho-tex/oberdiek/issues
%
% The Base Interpreter refers to any `TeX-Format',
% because some files are installed in TDS:tex/generic//.
%
% This work consists of the main source file eolgrab.dtx
% and the derived files
%    eolgrab.sty, eolgrab.pdf, eolgrab.ins, eolgrab.drv,
%    eolgrab-example-ltx.tex, eolgrab-example-env.tex,
%    eolgrab-example-sec.tex, eolgrab-test1.tex,
%    eolgrab-test2.tex.
%
% Distribution:
%    CTAN:macros/latex/contrib/oberdiek/eolgrab.dtx
%    CTAN:macros/latex/contrib/oberdiek/eolgrab.pdf
%
% Unpacking:
%    (a) If eolgrab.ins is present:
%           tex eolgrab.ins
%    (b) Without eolgrab.ins:
%           tex eolgrab.dtx
%    (c) If you insist on using LaTeX
%           latex \let\install=y% \iffalse meta-comment
%
% File: eolgrab.dtx
% Version: 2016/05/16 v1.1
% Info: Catch arguments delimited by end of line
%
% Copyright (C)
%    2011 Heiko Oberdiek
%    2016-2019 Oberdiek Package Support Group
%    https://github.com/ho-tex/oberdiek/issues
%
% This work may be distributed and/or modified under the
% conditions of the LaTeX Project Public License, either
% version 1.3c of this license or (at your option) any later
% version. This version of this license is in
%    https://www.latex-project.org/lppl/lppl-1-3c.txt
% and the latest version of this license is in
%    https://www.latex-project.org/lppl.txt
% and version 1.3 or later is part of all distributions of
% LaTeX version 2005/12/01 or later.
%
% This work has the LPPL maintenance status "maintained".
%
% The Current Maintainers of this work are
% Heiko Oberdiek and the Oberdiek Package Support Group
% https://github.com/ho-tex/oberdiek/issues
%
% The Base Interpreter refers to any `TeX-Format',
% because some files are installed in TDS:tex/generic//.
%
% This work consists of the main source file eolgrab.dtx
% and the derived files
%    eolgrab.sty, eolgrab.pdf, eolgrab.ins, eolgrab.drv,
%    eolgrab-example-ltx.tex, eolgrab-example-env.tex,
%    eolgrab-example-sec.tex, eolgrab-test1.tex,
%    eolgrab-test2.tex.
%
% Distribution:
%    CTAN:macros/latex/contrib/oberdiek/eolgrab.dtx
%    CTAN:macros/latex/contrib/oberdiek/eolgrab.pdf
%
% Unpacking:
%    (a) If eolgrab.ins is present:
%           tex eolgrab.ins
%    (b) Without eolgrab.ins:
%           tex eolgrab.dtx
%    (c) If you insist on using LaTeX
%           latex \let\install=y% \iffalse meta-comment
%
% File: eolgrab.dtx
% Version: 2016/05/16 v1.1
% Info: Catch arguments delimited by end of line
%
% Copyright (C)
%    2011 Heiko Oberdiek
%    2016-2019 Oberdiek Package Support Group
%    https://github.com/ho-tex/oberdiek/issues
%
% This work may be distributed and/or modified under the
% conditions of the LaTeX Project Public License, either
% version 1.3c of this license or (at your option) any later
% version. This version of this license is in
%    https://www.latex-project.org/lppl/lppl-1-3c.txt
% and the latest version of this license is in
%    https://www.latex-project.org/lppl.txt
% and version 1.3 or later is part of all distributions of
% LaTeX version 2005/12/01 or later.
%
% This work has the LPPL maintenance status "maintained".
%
% The Current Maintainers of this work are
% Heiko Oberdiek and the Oberdiek Package Support Group
% https://github.com/ho-tex/oberdiek/issues
%
% The Base Interpreter refers to any `TeX-Format',
% because some files are installed in TDS:tex/generic//.
%
% This work consists of the main source file eolgrab.dtx
% and the derived files
%    eolgrab.sty, eolgrab.pdf, eolgrab.ins, eolgrab.drv,
%    eolgrab-example-ltx.tex, eolgrab-example-env.tex,
%    eolgrab-example-sec.tex, eolgrab-test1.tex,
%    eolgrab-test2.tex.
%
% Distribution:
%    CTAN:macros/latex/contrib/oberdiek/eolgrab.dtx
%    CTAN:macros/latex/contrib/oberdiek/eolgrab.pdf
%
% Unpacking:
%    (a) If eolgrab.ins is present:
%           tex eolgrab.ins
%    (b) Without eolgrab.ins:
%           tex eolgrab.dtx
%    (c) If you insist on using LaTeX
%           latex \let\install=y% \iffalse meta-comment
%
% File: eolgrab.dtx
% Version: 2016/05/16 v1.1
% Info: Catch arguments delimited by end of line
%
% Copyright (C)
%    2011 Heiko Oberdiek
%    2016-2019 Oberdiek Package Support Group
%    https://github.com/ho-tex/oberdiek/issues
%
% This work may be distributed and/or modified under the
% conditions of the LaTeX Project Public License, either
% version 1.3c of this license or (at your option) any later
% version. This version of this license is in
%    https://www.latex-project.org/lppl/lppl-1-3c.txt
% and the latest version of this license is in
%    https://www.latex-project.org/lppl.txt
% and version 1.3 or later is part of all distributions of
% LaTeX version 2005/12/01 or later.
%
% This work has the LPPL maintenance status "maintained".
%
% The Current Maintainers of this work are
% Heiko Oberdiek and the Oberdiek Package Support Group
% https://github.com/ho-tex/oberdiek/issues
%
% The Base Interpreter refers to any `TeX-Format',
% because some files are installed in TDS:tex/generic//.
%
% This work consists of the main source file eolgrab.dtx
% and the derived files
%    eolgrab.sty, eolgrab.pdf, eolgrab.ins, eolgrab.drv,
%    eolgrab-example-ltx.tex, eolgrab-example-env.tex,
%    eolgrab-example-sec.tex, eolgrab-test1.tex,
%    eolgrab-test2.tex.
%
% Distribution:
%    CTAN:macros/latex/contrib/oberdiek/eolgrab.dtx
%    CTAN:macros/latex/contrib/oberdiek/eolgrab.pdf
%
% Unpacking:
%    (a) If eolgrab.ins is present:
%           tex eolgrab.ins
%    (b) Without eolgrab.ins:
%           tex eolgrab.dtx
%    (c) If you insist on using LaTeX
%           latex \let\install=y\input{eolgrab.dtx}
%        (quote the arguments according to the demands of your shell)
%
% Documentation:
%    (a) If eolgrab.drv is present:
%           latex eolgrab.drv
%    (b) Without eolgrab.drv:
%           latex eolgrab.dtx; ...
%    The class ltxdoc loads the configuration file ltxdoc.cfg
%    if available. Here you can specify further options, e.g.
%    use A4 as paper format:
%       \PassOptionsToClass{a4paper}{article}
%
%    Programm calls to get the documentation (example):
%       pdflatex eolgrab.dtx
%       makeindex -s gind.ist eolgrab.idx
%       pdflatex eolgrab.dtx
%       makeindex -s gind.ist eolgrab.idx
%       pdflatex eolgrab.dtx
%
% Installation:
%    TDS:tex/generic/oberdiek/eolgrab.sty
%    TDS:doc/latex/oberdiek/eolgrab.pdf
%    TDS:doc/latex/oberdiek/example/eolgrab-example-ltx.tex
%    TDS:doc/latex/oberdiek/example/eolgrab-example-env.tex
%    TDS:doc/latex/oberdiek/example/eolgrab-example-sec.tex
%    TDS:source/latex/oberdiek/eolgrab.dtx
%
%<*ignore>
\begingroup
  \catcode123=1 %
  \catcode125=2 %
  \def\x{LaTeX2e}%
\expandafter\endgroup
\ifcase 0\ifx\install y1\fi\expandafter
         \ifx\csname processbatchFile\endcsname\relax\else1\fi
         \ifx\fmtname\x\else 1\fi\relax
\else\csname fi\endcsname
%</ignore>
%<*install>
\input docstrip.tex
\Msg{************************************************************************}
\Msg{* Installation}
\Msg{* Package: eolgrab 2016/05/16 v1.1 Catch arguments delimited by end of line (HO)}
\Msg{************************************************************************}

\keepsilent
\askforoverwritefalse

\let\MetaPrefix\relax
\preamble

This is a generated file.

Project: eolgrab
Version: 2016/05/16 v1.1

Copyright (C)
   2011 Heiko Oberdiek
   2016-2019 Oberdiek Package Support Group

This work may be distributed and/or modified under the
conditions of the LaTeX Project Public License, either
version 1.3c of this license or (at your option) any later
version. This version of this license is in
   https://www.latex-project.org/lppl/lppl-1-3c.txt
and the latest version of this license is in
   https://www.latex-project.org/lppl.txt
and version 1.3 or later is part of all distributions of
LaTeX version 2005/12/01 or later.

This work has the LPPL maintenance status "maintained".

The Current Maintainers of this work are
Heiko Oberdiek and the Oberdiek Package Support Group
https://github.com/ho-tex/oberdiek/issues


The Base Interpreter refers to any `TeX-Format',
because some files are installed in TDS:tex/generic//.

This work consists of the main source file eolgrab.dtx
and the derived files
   eolgrab.sty, eolgrab.pdf, eolgrab.ins, eolgrab.drv,
   eolgrab-example-ltx.tex, eolgrab-example-env.tex,
   eolgrab-example-sec.tex, eolgrab-test1.tex,
   eolgrab-test2.tex.

\endpreamble
\let\MetaPrefix\DoubleperCent

\generate{%
  \file{eolgrab.ins}{\from{eolgrab.dtx}{install}}%
  \file{eolgrab.drv}{\from{eolgrab.dtx}{driver}}%
  \usedir{tex/generic/oberdiek}%
  \file{eolgrab.sty}{\from{eolgrab.dtx}{package}}%
  \usedir{doc/latex/oberdiek/example}%
  \file{eolgrab-example-ltx.tex}{\from{eolgrab.dtx}{example-ltx}}%
  \file{eolgrab-example-env.tex}{\from{eolgrab.dtx}{example-env}}%
  \file{eolgrab-example-sec.tex}{\from{eolgrab.dtx}{example-sec}}%
%  \usedir{doc/latex/oberdiek/test}%
%  \file{eolgrab-test1.tex}{\from{eolgrab.dtx}{test1}}%
%  \file{eolgrab-test2.tex}{\from{eolgrab.dtx}{test2}}%
}

\catcode32=13\relax% active space
\let =\space%
\Msg{************************************************************************}
\Msg{*}
\Msg{* To finish the installation you have to move the following}
\Msg{* file into a directory searched by TeX:}
\Msg{*}
\Msg{*     eolgrab.sty}
\Msg{*}
\Msg{* To produce the documentation run the file `eolgrab.drv'}
\Msg{* through LaTeX.}
\Msg{*}
\Msg{* Happy TeXing!}
\Msg{*}
\Msg{************************************************************************}

\endbatchfile
%</install>
%<*ignore>
\fi
%</ignore>
%<*driver>
\NeedsTeXFormat{LaTeX2e}
\ProvidesFile{eolgrab.drv}%
  [2016/05/16 v1.1 Catch arguments delimited by end of line (HO)]%
\documentclass{ltxdoc}
\usepackage{holtxdoc}[2011/11/22]
\begin{document}
  \DocInput{eolgrab.dtx}%
\end{document}
%</driver>
% \fi
%
%
%
% \GetFileInfo{eolgrab.drv}
%
% \title{The \xpackage{eolgrab} package}
% \date{2016/05/16 v1.1}
% \author{Heiko Oberdiek\thanks
% {Please report any issues at \url{https://github.com/ho-tex/oberdiek/issues}}}
%
% \maketitle
%
% \begin{abstract}
% This package implements a generic argument grabber
% to catch an argument that is delimited by the line end.
% \end{abstract}
%
% \tableofcontents
%
% \urldef\urltexhax\url{https://tug.org/mailman/listinfo/texhax}
% \def\eol/{^^A
%   \meta{\texttt{EOL}}^^A
% }
%
% \section{Documentation}
%
% The starting point for this package was a feature
% request of Arno Trautmann in the mailing list
% \texttt{texhax}\footnote{Info page for mailing list
% \texttt{texhax}: \urltexhax} \cite{trautmann-request}.
% A macro \cs{eolsection} should behave like \cs{section},
% but the argument should be delimited by the line end
% instead of given in curly braces:
% \begin{quote}
%   |\eolsection My Title|
% \end{quote}
% Phil Taylor answered this with an implementation for \cs{eolsection}.
% Because this feature could be useful for other macros as well,
% I answered with an implementation of \cs{eolgrab} as general solution
% \cite{oberdiek-eolgrab}.
%
% Both formats \hologo{plainTeX} and \hologo{LaTeX} are supported
% by the package, see the example for \cs{eolsection} below.
%
% \begin{declcs}{eolgrab} \M{code} \meta{argument} \eol/
% \end{declcs}
% Macro \cs{eolgrab} takes two arguments. The first argument
% is \meta{code}, a classical undelimited \hologo{TeX} macro argument.
% The second argument is delimited by the line end \eol/.
% The macro calls \meta{code} with \meta{argument} as argument
% in curly braces. Because the catcode of the line end is changed,
% \cs{eolgrab} will not work in the argument of other macros.
% Macro \cs{eolgrab} is made robust if either \hologo{eTeX}'s
% \cs{protected} or \hologo{LaTeX}'s \cs{DeclareRobustCommand}
% is available.
%
% \begin{declcs}{eolgrabopt} \M{code} \meta{argument} \eol/
% \end{declcs}
% Macro \cs{eolgrabopt} passes \meta{argument} as optional
% argument to \meta{code} if \meta{argument} is not empty.
% \begin{quote}
%   |\eolgrabopt\item foo|
% \end{quote}
% becomes to
% \begin{quote}
%   |\item[{foo}]|
% \end{quote}
% The curly argument braces are added to support square brackets
% inside \meta{argument}.
% If the \meta{argument} is empty:
% \begin{quote}
%   |\eolgrabopt\item|
% \end{quote}
% then
% \begin{quote}
%   |\item|
% \end{quote}
% is called without optional argument.
%
% \subsection{Examples}
%
% \begin{itemize}
% \item
%   The line
%   \begin{quote}
%     |\eolgrab\section My Title|
%   \end{quote}
%   is equivalent to
%   \begin{quote}
%     |\section{My Title}|
%   \end{quote}
% \item
%   The next example uses the star form of \cs{section}. Then the
%   command to be called consists of two tokens. Therefore the
%   first argument of \cs{eolgrab} needs curly braces:
%   \begin{quote}
%     |\eolgrab{\section*}My Title|
%   \end{quote}
%   becomes
%   \begin{quote}
%     |\section*{My Title}|
%   \end{quote}
% \item
%   Now \hologo{LaTeX}'s \cs{PackageError} is used. This macro
%   has three arguments, the package or class name, the message
%   text and the help text. A standard help text of \hologo{LaTeX}
%   is used as given in macro \cs{@ehc}. The second argument,
%   the message text is used as argument, delimited by line end:
%   \begin{quote}
%     |\eolgrab{\PackageError{foobar}}%|\\
%     |Some error message text\MessageBreak%|\\
%     |with several lines|\\
%     |\@ehc|
%   \end{quote}
%   In the first two lines of the example, the line end is suppressed
%   by the comment character (percent), thus the argument is delimited
%   by the line end of the third line. The result is:
%   \begin{quote}
%     |\PackageError{foobar}{Some error message text\MessageBreak|\\
%     |with several lines}\@ehc|
%   \end{quote}
% \item
%   The original request for macro \cs{eolsection}, see above,
%   can be implemented easily with the help of \cs{eolgrab}.
%   Example for \hologo{LaTeX}:
%   \begin{quote}
%     |\usepackage{eolgrab}|\\
%     |\newcommand*{\eolsection}{\eolgrab\section}|
%   \end{quote}
%   Example for \hologo{plainTeX}:
%   \begin{quote}
%     |\input eolgrab.sty\relax|\\
%     |\def\eolsection{\eolgrab\section}|
%   \end{quote}
%   And a sophisticated variant for \hologo{LaTeX} that also supports
%   the star syntax and the optional argument:
%    \begin{macrocode}
%<*example-sec>
\documentclass{article}
\usepackage{eolgrab}
\makeatletter
\newcommand*{\eolsection}{%
  \@ifstar{%
    \eolgrab{\section*}%
  }{%
    \@ifnextchar[{%
      \eoloptsection
    }{%
      \eolgrab\section
    }%
  }%
}
\newcommand*{\eoloptsection}[1][]{%
  \eolgrab{\section[{#1}]}%
}
\makeatother
\begin{document}
\tableofcontents
\eolsection Section without star and optional argument
\eolsection*Section with star
\eolsection[Short section title]Long section title
\end{document}
%</example-sec>
%    \end{macrocode}
% \end{itemize}
%
% \subsubsection{Small \hologo{LaTeX} document as example}
%
%    \begin{macrocode}
%<*example-ltx>
\RequirePackage{eolgrab}
\eolgrab\documentclass article
\eolgrab\begin document
\eolgrab\section Hello World
\eolgrab\emph Some text
\eolgrab\end document
%</example-ltx>
%    \end{macrocode}
%
% \subsubsection{\hologo{LaTeX} document with environments}
%
%    \begin{macrocode}
%<*example-env>
\documentclass{article}
\usepackage{eolgrab}
\newcommand*{\Begin}{\eolgrab   \begin}
\newcommand*{\End  }{\eolgrab   \end  }
\newcommand*{\Item }{\eolgrabopt\item }
\Begin document
  \Begin itemize
    \Item
      first item
    \Item
      second item
  \End itemize
  \Begin description
    \Item foo
      is the first syllable of foobar.
    \Item bar
      is the second syllable of foobar.
  \End description
\End document
%</example-env>
%    \end{macrocode}
%
% \subsection{Limitations}
%
% Macro \cs{eolgrab} needs to catch the line end.
% If \hologo{TeX} reads a line, then it throws away the line end
% characters (carriage return, line feed) and removes spaces at the
% end of the line. Then it adds the character with the character code
% that is given by \cs{endlinechar} at the end of the line. The category
% code of the inserted character is given by the current value of
% its \cs{catcode}. If \cs{endlinechar} is not a valid character code
% (especially if it is negative), then no character is added.
%
% In \hologo{plainTeX} and \hologo{LaTeX} the standard settings
% of the inserted endline character is the character with code 13
% (or |^||^||M| in \hologo{TeX} notation) with catcode 5 (end of line).
% That means the inserted end of line character behaves like a space
% token. For example, it is removed after macro names. Therefore
% \cs{eolgrab} changes the catcode.
%
% Therefore \cs{eolgrab} has some limitations:
% \begin{itemize}
% \item
%   Like other verbatim stuff, the macro \cs{eolgrab} cannot be
%   used in the argument of other macros. \cs{eolgrab} want to
%   change the catcode of the end of line character. If this
%   character is read before, because it is processed as argument
%   of another macro, the catcode is already set and is not reassigned
%   later if \cs{eolgrab} changes the category code for this character
%   code.
% \item
%   The argument must not contain the end of line character.
%   Otherwise the first end of line character is already taken
%   as delimiter, leaving the rest of the line outside the argument.
% \item
%   Because \cs{eolgrab} is probably mostly used in the line with
%   the delimited argument. Therefore changes of \cs{endlinechar}
%   will not affect the current line.
% \end{itemize}
%
% \StopEventually{
% }
%
% \section{Implementation}
%
%    \begin{macrocode}
%<*package>
%    \end{macrocode}
%
% \subsection{Reload check and package identification}
%    Reload check, especially if the package is not used with \LaTeX.
%    \begin{macrocode}
\begingroup\catcode61\catcode48\catcode32=10\relax%
  \catcode13=5 % ^^M
  \endlinechar=13 %
  \catcode35=6 % #
  \catcode39=12 % '
  \catcode44=12 % ,
  \catcode45=12 % -
  \catcode46=12 % .
  \catcode58=12 % :
  \catcode64=11 % @
  \catcode123=1 % {
  \catcode125=2 % }
  \expandafter\let\expandafter\x\csname ver@eolgrab.sty\endcsname
  \ifx\x\relax % plain-TeX, first loading
  \else
    \def\empty{}%
    \ifx\x\empty % LaTeX, first loading,
      % variable is initialized, but \ProvidesPackage not yet seen
    \else
      \expandafter\ifx\csname PackageInfo\endcsname\relax
        \def\x#1#2{%
          \immediate\write-1{Package #1 Info: #2.}%
        }%
      \else
        \def\x#1#2{\PackageInfo{#1}{#2, stopped}}%
      \fi
      \x{eolgrab}{The package is already loaded}%
      \aftergroup\endinput
    \fi
  \fi
\endgroup%
%    \end{macrocode}
%    Package identification:
%    \begin{macrocode}
\begingroup\catcode61\catcode48\catcode32=10\relax%
  \catcode13=5 % ^^M
  \endlinechar=13 %
  \catcode35=6 % #
  \catcode39=12 % '
  \catcode40=12 % (
  \catcode41=12 % )
  \catcode44=12 % ,
  \catcode45=12 % -
  \catcode46=12 % .
  \catcode47=12 % /
  \catcode58=12 % :
  \catcode64=11 % @
  \catcode91=12 % [
  \catcode93=12 % ]
  \catcode123=1 % {
  \catcode125=2 % }
  \expandafter\ifx\csname ProvidesPackage\endcsname\relax
    \def\x#1#2#3[#4]{\endgroup
      \immediate\write-1{Package: #3 #4}%
      \xdef#1{#4}%
    }%
  \else
    \def\x#1#2[#3]{\endgroup
      #2[{#3}]%
      \ifx#1\@undefined
        \xdef#1{#3}%
      \fi
      \ifx#1\relax
        \xdef#1{#3}%
      \fi
    }%
  \fi
\expandafter\x\csname ver@eolgrab.sty\endcsname
\ProvidesPackage{eolgrab}%
  [2016/05/16 v1.1 Catch arguments delimited by end of line (HO)]%
%    \end{macrocode}
%
% \subsection{Catcodes}
%
%    \begin{macrocode}
\begingroup\catcode61\catcode48\catcode32=10\relax%
  \catcode13=5 % ^^M
  \endlinechar=13 %
  \catcode123=1 % {
  \catcode125=2 % }
  \catcode64=11 % @
  \def\x{\endgroup
    \expandafter\edef\csname eolgrab@AtEnd\endcsname{%
      \endlinechar=\the\endlinechar\relax
      \catcode13=\the\catcode13\relax
      \catcode32=\the\catcode32\relax
      \catcode35=\the\catcode35\relax
      \catcode61=\the\catcode61\relax
      \catcode64=\the\catcode64\relax
      \catcode123=\the\catcode123\relax
      \catcode125=\the\catcode125\relax
    }%
  }%
\x\catcode61\catcode48\catcode32=10\relax%
\catcode13=5 % ^^M
\endlinechar=13 %
\catcode35=6 % #
\catcode64=11 % @
\catcode123=1 % {
\catcode125=2 % }
\def\TMP@EnsureCode#1#2{%
  \edef\eolgrab@AtEnd{%
    \eolgrab@AtEnd
    \catcode#1=\the\catcode#1\relax
  }%
  \catcode#1=#2\relax
}
\TMP@EnsureCode{40}{12}% (
\TMP@EnsureCode{41}{12}% )
\TMP@EnsureCode{42}{12}% *
\TMP@EnsureCode{46}{12}% .
\TMP@EnsureCode{47}{12}% /
\TMP@EnsureCode{91}{12}% [
\TMP@EnsureCode{93}{12}% ]
\TMP@EnsureCode{94}{7}% ^
\edef\eolgrab@AtEnd{\eolgrab@AtEnd\noexpand\endinput}
%    \end{macrocode}
%
% \subsection{Resources}
%
%    \begin{macrocode}
\begingroup\expandafter\expandafter\expandafter\endgroup
\expandafter\ifx\csname RequirePackage\endcsname\relax
  \input ltxcmds.sty\relax
  \input infwarerr.sty\relax
\else
  \RequirePackage{ltxcmds}[2010/12/04]%
  \RequirePackage{infwarerr}[2010/04/08]%
\fi
%    \end{macrocode}
%
%    \begin{macro}{\eolgrab@ifdefinable}
%    \begin{macrocode}
\ltx@IfUndefined{@ifdefinable}{%
  \def\eolgrab@ifdefinable#1#2{%
    \ltx@ifundefined{#1}{#2}{%
      \@PackageError{eolgrab}{%
        Command \ltx@backslashchar#1 already defined%
      }\@ehc
    }%
  }%
}{%
  \def\eolgrab@ifdefinable#1{%
    \expandafter\@ifdefinable\csname#1\endcsname
  }%
}
%    \end{macrocode}
%    \end{macro}
%
% \subsection{Macro \cs{eolgrab}}
%
%    \begin{macro}{\eolgrab}
%    \begin{macrocode}
\eolgrab@ifdefinable{eolgrab}{%
  \ltx@IfUndefined{protected}{%
    \ltx@IfUndefined{DeclareRobustCommand}{%
      \def\eolgrab#1%
    }{%
      \newcommand\eolgrab{}%
      \DeclareRobustCommand*\eolgrab
    }%
  }{%
    \protected\def\eolgrab#1%
  }{%
    \begingroup
    \endlinechar=13 %
    \catcode13=\ltx@active
    \eolgrab@{#1}%
  }%
}
%    \end{macrocode}
%    \end{macro}
%
%    \begin{macro}{\eolgrabopt}
%    \begin{macrocode}
\eolgrab@ifdefinable{eolgrabopt}{%
  \ltx@IfUndefined{protected}{%
    \ltx@IfUndefined{DeclareRobustCommand}{%
      \def\eolgrabopt#1%
    }{%
      \newcommand\eolgrabopt{}%
      \DeclareRobustCommand*\eolgrabopt
    }%
  }{%
    \protected\def\eolgrabopt#1%
  }{%
    \begingroup
    \endlinechar=13 %
    \catcode13=\ltx@active
    \eolgrab@opt{#1}%
  }%
}
%    \end{macrocode}
%    \end{macro}
%
%    \begin{macrocode}
\begingroup
  \catcode13=\ltx@active %
\ltx@firstofone{\endgroup %
%    \end{macrocode}
%    \begin{macro}{\eolgrab@}
%    \begin{macrocode}
  \def\eolgrab@#1#2^^M{%
    \endgroup %
    #1{#2}%
  }%
%    \end{macrocode}
%    \end{macro}
%    \begin{macro}{\eolgrab@opt}
%    \begin{macrocode}
  \def\eolgrab@opt#1#2^^M{%
    \endgroup %
    \ltx@ifempty{#2}{%
      #1%
    }{%
      #1[{#2}]%
    }%
  }%
%    \end{macrocode}
%    \end{macro}
%    \begin{macrocode}
}%
%    \end{macrocode}
%
%    \begin{macrocode}
\eolgrab@AtEnd%
%    \end{macrocode}
%    \begin{macrocode}
%</package>
%    \end{macrocode}
%% \section{Installation}
%
% \subsection{Download}
%
% \paragraph{Package.} This package is available on
% CTAN\footnote{\CTANpkg{eolgrab}}:
% \begin{description}
% \item[\CTAN{macros/latex/contrib/oberdiek/eolgrab.dtx}] The source file.
% \item[\CTAN{macros/latex/contrib/oberdiek/eolgrab.pdf}] Documentation.
% \end{description}
%
%
% \paragraph{Bundle.} All the packages of the bundle `oberdiek'
% are also available in a TDS compliant ZIP archive. There
% the packages are already unpacked and the documentation files
% are generated. The files and directories obey the TDS standard.
% \begin{description}
% \item[\CTANinstall{install/macros/latex/contrib/oberdiek.tds.zip}]
% \end{description}
% \emph{TDS} refers to the standard ``A Directory Structure
% for \TeX\ Files'' (\CTANpkg{tds}). Directories
% with \xfile{texmf} in their name are usually organized this way.
%
% \subsection{Bundle installation}
%
% \paragraph{Unpacking.} Unpack the \xfile{oberdiek.tds.zip} in the
% TDS tree (also known as \xfile{texmf} tree) of your choice.
% Example (linux):
% \begin{quote}
%   |unzip oberdiek.tds.zip -d ~/texmf|
% \end{quote}
%
% \subsection{Package installation}
%
% \paragraph{Unpacking.} The \xfile{.dtx} file is a self-extracting
% \docstrip\ archive. The files are extracted by running the
% \xfile{.dtx} through \plainTeX:
% \begin{quote}
%   \verb|tex eolgrab.dtx|
% \end{quote}
%
% \paragraph{TDS.} Now the different files must be moved into
% the different directories in your installation TDS tree
% (also known as \xfile{texmf} tree):
% \begin{quote}
% \def\t{^^A
% \begin{tabular}{@{}>{\ttfamily}l@{ $\rightarrow$ }>{\ttfamily}l@{}}
%   eolgrab.sty & tex/generic/oberdiek/eolgrab.sty\\
%   eolgrab.pdf & doc/latex/oberdiek/eolgrab.pdf\\
%   example/eolgrab-example-ltx.tex & doc/latex/oberdiek/example/eolgrab-example-ltx.tex\\
%   example/eolgrab-example-env.tex & doc/latex/oberdiek/example/eolgrab-example-env.tex\\
%   example/eolgrab-example-sec.tex & doc/latex/oberdiek/example/eolgrab-example-sec.tex\\
%   eolgrab.dtx & source/latex/oberdiek/eolgrab.dtx\\
% \end{tabular}^^A
% }^^A
% \sbox0{\t}^^A
% \ifdim\wd0>\linewidth
%   \begingroup
%     \advance\linewidth by\leftmargin
%     \advance\linewidth by\rightmargin
%   \edef\x{\endgroup
%     \def\noexpand\lw{\the\linewidth}^^A
%   }\x
%   \def\lwbox{^^A
%     \leavevmode
%     \hbox to \linewidth{^^A
%       \kern-\leftmargin\relax
%       \hss
%       \usebox0
%       \hss
%       \kern-\rightmargin\relax
%     }^^A
%   }^^A
%   \ifdim\wd0>\lw
%     \sbox0{\small\t}^^A
%     \ifdim\wd0>\linewidth
%       \ifdim\wd0>\lw
%         \sbox0{\footnotesize\t}^^A
%         \ifdim\wd0>\linewidth
%           \ifdim\wd0>\lw
%             \sbox0{\scriptsize\t}^^A
%             \ifdim\wd0>\linewidth
%               \ifdim\wd0>\lw
%                 \sbox0{\tiny\t}^^A
%                 \ifdim\wd0>\linewidth
%                   \lwbox
%                 \else
%                   \usebox0
%                 \fi
%               \else
%                 \lwbox
%               \fi
%             \else
%               \usebox0
%             \fi
%           \else
%             \lwbox
%           \fi
%         \else
%           \usebox0
%         \fi
%       \else
%         \lwbox
%       \fi
%     \else
%       \usebox0
%     \fi
%   \else
%     \lwbox
%   \fi
% \else
%   \usebox0
% \fi
% \end{quote}
% If you have a \xfile{docstrip.cfg} that configures and enables \docstrip's
% TDS installing feature, then some files can already be in the right
% place, see the documentation of \docstrip.
%
% \subsection{Refresh file name databases}
%
% If your \TeX~distribution
% (\TeX\,Live, \mikTeX, \dots) relies on file name databases, you must refresh
% these. For example, \TeX\,Live\ users run \verb|texhash| or
% \verb|mktexlsr|.
%
% \subsection{Some details for the interested}
%
% \paragraph{Unpacking with \LaTeX.}
% The \xfile{.dtx} chooses its action depending on the format:
% \begin{description}
% \item[\plainTeX:] Run \docstrip\ and extract the files.
% \item[\LaTeX:] Generate the documentation.
% \end{description}
% If you insist on using \LaTeX\ for \docstrip\ (really,
% \docstrip\ does not need \LaTeX), then inform the autodetect routine
% about your intention:
% \begin{quote}
%   \verb|latex \let\install=y\input{eolgrab.dtx}|
% \end{quote}
% Do not forget to quote the argument according to the demands
% of your shell.
%
% \paragraph{Generating the documentation.}
% You can use both the \xfile{.dtx} or the \xfile{.drv} to generate
% the documentation. The process can be configured by the
% configuration file \xfile{ltxdoc.cfg}. For instance, put this
% line into this file, if you want to have A4 as paper format:
% \begin{quote}
%   \verb|\PassOptionsToClass{a4paper}{article}|
% \end{quote}
% An example follows how to generate the
% documentation with pdf\LaTeX:
% \begin{quote}
%\begin{verbatim}
%pdflatex eolgrab.dtx
%makeindex -s gind.ist eolgrab.idx
%pdflatex eolgrab.dtx
%makeindex -s gind.ist eolgrab.idx
%pdflatex eolgrab.dtx
%\end{verbatim}
% \end{quote}
%
% \begin{thebibliography}{9}
% \raggedright
% \bibitem{trautmann-request}
% Arno Trautmann,
% \textit{[texhax] read argument until EOL};
% mailing list \texttt{texthax@tug.org},
% 2011-01-06;
% \url{https://tug.org/pipermail/texhax/2011-January/016517.html}.
%
% \bibitem{taylor}
% Philip Taylor,
% \textit{Re: [texhax] read argument until EOL};
% mailing list \texttt{texhax@tug.org},
% 2011-01-06;
% \url{https://tug.org/pipermail/texhax/2011-January/016519.html}.
%
% \bibitem{oberdiek-eolgrab}
% Heiko Oberdiek,
% \textit{Re: [texhax] read argument until EOL};
% mailing list \texttt{texhax@tug.org},
% 2011-01-06;
% \url{https://tug.org/pipermail/texhax/2011-January/016526.html}.
%
% \end{thebibliography}
%
% \begin{History}
%   \begin{Version}{2011/01/12 v1.0}
%   \item
%     First public version.
%   \end{Version}
%   \begin{Version}{2016/05/16 v1.1}
%   \item
%     Documentation updates.
%   \end{Version}
% \end{History}
%
% \PrintIndex
%
% \Finale
\endinput

%        (quote the arguments according to the demands of your shell)
%
% Documentation:
%    (a) If eolgrab.drv is present:
%           latex eolgrab.drv
%    (b) Without eolgrab.drv:
%           latex eolgrab.dtx; ...
%    The class ltxdoc loads the configuration file ltxdoc.cfg
%    if available. Here you can specify further options, e.g.
%    use A4 as paper format:
%       \PassOptionsToClass{a4paper}{article}
%
%    Programm calls to get the documentation (example):
%       pdflatex eolgrab.dtx
%       makeindex -s gind.ist eolgrab.idx
%       pdflatex eolgrab.dtx
%       makeindex -s gind.ist eolgrab.idx
%       pdflatex eolgrab.dtx
%
% Installation:
%    TDS:tex/generic/oberdiek/eolgrab.sty
%    TDS:doc/latex/oberdiek/eolgrab.pdf
%    TDS:doc/latex/oberdiek/example/eolgrab-example-ltx.tex
%    TDS:doc/latex/oberdiek/example/eolgrab-example-env.tex
%    TDS:doc/latex/oberdiek/example/eolgrab-example-sec.tex
%    TDS:source/latex/oberdiek/eolgrab.dtx
%
%<*ignore>
\begingroup
  \catcode123=1 %
  \catcode125=2 %
  \def\x{LaTeX2e}%
\expandafter\endgroup
\ifcase 0\ifx\install y1\fi\expandafter
         \ifx\csname processbatchFile\endcsname\relax\else1\fi
         \ifx\fmtname\x\else 1\fi\relax
\else\csname fi\endcsname
%</ignore>
%<*install>
\input docstrip.tex
\Msg{************************************************************************}
\Msg{* Installation}
\Msg{* Package: eolgrab 2016/05/16 v1.1 Catch arguments delimited by end of line (HO)}
\Msg{************************************************************************}

\keepsilent
\askforoverwritefalse

\let\MetaPrefix\relax
\preamble

This is a generated file.

Project: eolgrab
Version: 2016/05/16 v1.1

Copyright (C)
   2011 Heiko Oberdiek
   2016-2019 Oberdiek Package Support Group

This work may be distributed and/or modified under the
conditions of the LaTeX Project Public License, either
version 1.3c of this license or (at your option) any later
version. This version of this license is in
   https://www.latex-project.org/lppl/lppl-1-3c.txt
and the latest version of this license is in
   https://www.latex-project.org/lppl.txt
and version 1.3 or later is part of all distributions of
LaTeX version 2005/12/01 or later.

This work has the LPPL maintenance status "maintained".

The Current Maintainers of this work are
Heiko Oberdiek and the Oberdiek Package Support Group
https://github.com/ho-tex/oberdiek/issues


The Base Interpreter refers to any `TeX-Format',
because some files are installed in TDS:tex/generic//.

This work consists of the main source file eolgrab.dtx
and the derived files
   eolgrab.sty, eolgrab.pdf, eolgrab.ins, eolgrab.drv,
   eolgrab-example-ltx.tex, eolgrab-example-env.tex,
   eolgrab-example-sec.tex, eolgrab-test1.tex,
   eolgrab-test2.tex.

\endpreamble
\let\MetaPrefix\DoubleperCent

\generate{%
  \file{eolgrab.ins}{\from{eolgrab.dtx}{install}}%
  \file{eolgrab.drv}{\from{eolgrab.dtx}{driver}}%
  \usedir{tex/generic/oberdiek}%
  \file{eolgrab.sty}{\from{eolgrab.dtx}{package}}%
  \usedir{doc/latex/oberdiek/example}%
  \file{eolgrab-example-ltx.tex}{\from{eolgrab.dtx}{example-ltx}}%
  \file{eolgrab-example-env.tex}{\from{eolgrab.dtx}{example-env}}%
  \file{eolgrab-example-sec.tex}{\from{eolgrab.dtx}{example-sec}}%
%  \usedir{doc/latex/oberdiek/test}%
%  \file{eolgrab-test1.tex}{\from{eolgrab.dtx}{test1}}%
%  \file{eolgrab-test2.tex}{\from{eolgrab.dtx}{test2}}%
}

\catcode32=13\relax% active space
\let =\space%
\Msg{************************************************************************}
\Msg{*}
\Msg{* To finish the installation you have to move the following}
\Msg{* file into a directory searched by TeX:}
\Msg{*}
\Msg{*     eolgrab.sty}
\Msg{*}
\Msg{* To produce the documentation run the file `eolgrab.drv'}
\Msg{* through LaTeX.}
\Msg{*}
\Msg{* Happy TeXing!}
\Msg{*}
\Msg{************************************************************************}

\endbatchfile
%</install>
%<*ignore>
\fi
%</ignore>
%<*driver>
\NeedsTeXFormat{LaTeX2e}
\ProvidesFile{eolgrab.drv}%
  [2016/05/16 v1.1 Catch arguments delimited by end of line (HO)]%
\documentclass{ltxdoc}
\usepackage{holtxdoc}[2011/11/22]
\begin{document}
  \DocInput{eolgrab.dtx}%
\end{document}
%</driver>
% \fi
%
%
%
% \GetFileInfo{eolgrab.drv}
%
% \title{The \xpackage{eolgrab} package}
% \date{2016/05/16 v1.1}
% \author{Heiko Oberdiek\thanks
% {Please report any issues at \url{https://github.com/ho-tex/oberdiek/issues}}}
%
% \maketitle
%
% \begin{abstract}
% This package implements a generic argument grabber
% to catch an argument that is delimited by the line end.
% \end{abstract}
%
% \tableofcontents
%
% \urldef\urltexhax\url{https://tug.org/mailman/listinfo/texhax}
% \def\eol/{^^A
%   \meta{\texttt{EOL}}^^A
% }
%
% \section{Documentation}
%
% The starting point for this package was a feature
% request of Arno Trautmann in the mailing list
% \texttt{texhax}\footnote{Info page for mailing list
% \texttt{texhax}: \urltexhax} \cite{trautmann-request}.
% A macro \cs{eolsection} should behave like \cs{section},
% but the argument should be delimited by the line end
% instead of given in curly braces:
% \begin{quote}
%   |\eolsection My Title|
% \end{quote}
% Phil Taylor answered this with an implementation for \cs{eolsection}.
% Because this feature could be useful for other macros as well,
% I answered with an implementation of \cs{eolgrab} as general solution
% \cite{oberdiek-eolgrab}.
%
% Both formats \hologo{plainTeX} and \hologo{LaTeX} are supported
% by the package, see the example for \cs{eolsection} below.
%
% \begin{declcs}{eolgrab} \M{code} \meta{argument} \eol/
% \end{declcs}
% Macro \cs{eolgrab} takes two arguments. The first argument
% is \meta{code}, a classical undelimited \hologo{TeX} macro argument.
% The second argument is delimited by the line end \eol/.
% The macro calls \meta{code} with \meta{argument} as argument
% in curly braces. Because the catcode of the line end is changed,
% \cs{eolgrab} will not work in the argument of other macros.
% Macro \cs{eolgrab} is made robust if either \hologo{eTeX}'s
% \cs{protected} or \hologo{LaTeX}'s \cs{DeclareRobustCommand}
% is available.
%
% \begin{declcs}{eolgrabopt} \M{code} \meta{argument} \eol/
% \end{declcs}
% Macro \cs{eolgrabopt} passes \meta{argument} as optional
% argument to \meta{code} if \meta{argument} is not empty.
% \begin{quote}
%   |\eolgrabopt\item foo|
% \end{quote}
% becomes to
% \begin{quote}
%   |\item[{foo}]|
% \end{quote}
% The curly argument braces are added to support square brackets
% inside \meta{argument}.
% If the \meta{argument} is empty:
% \begin{quote}
%   |\eolgrabopt\item|
% \end{quote}
% then
% \begin{quote}
%   |\item|
% \end{quote}
% is called without optional argument.
%
% \subsection{Examples}
%
% \begin{itemize}
% \item
%   The line
%   \begin{quote}
%     |\eolgrab\section My Title|
%   \end{quote}
%   is equivalent to
%   \begin{quote}
%     |\section{My Title}|
%   \end{quote}
% \item
%   The next example uses the star form of \cs{section}. Then the
%   command to be called consists of two tokens. Therefore the
%   first argument of \cs{eolgrab} needs curly braces:
%   \begin{quote}
%     |\eolgrab{\section*}My Title|
%   \end{quote}
%   becomes
%   \begin{quote}
%     |\section*{My Title}|
%   \end{quote}
% \item
%   Now \hologo{LaTeX}'s \cs{PackageError} is used. This macro
%   has three arguments, the package or class name, the message
%   text and the help text. A standard help text of \hologo{LaTeX}
%   is used as given in macro \cs{@ehc}. The second argument,
%   the message text is used as argument, delimited by line end:
%   \begin{quote}
%     |\eolgrab{\PackageError{foobar}}%|\\
%     |Some error message text\MessageBreak%|\\
%     |with several lines|\\
%     |\@ehc|
%   \end{quote}
%   In the first two lines of the example, the line end is suppressed
%   by the comment character (percent), thus the argument is delimited
%   by the line end of the third line. The result is:
%   \begin{quote}
%     |\PackageError{foobar}{Some error message text\MessageBreak|\\
%     |with several lines}\@ehc|
%   \end{quote}
% \item
%   The original request for macro \cs{eolsection}, see above,
%   can be implemented easily with the help of \cs{eolgrab}.
%   Example for \hologo{LaTeX}:
%   \begin{quote}
%     |\usepackage{eolgrab}|\\
%     |\newcommand*{\eolsection}{\eolgrab\section}|
%   \end{quote}
%   Example for \hologo{plainTeX}:
%   \begin{quote}
%     |\input eolgrab.sty\relax|\\
%     |\def\eolsection{\eolgrab\section}|
%   \end{quote}
%   And a sophisticated variant for \hologo{LaTeX} that also supports
%   the star syntax and the optional argument:
%    \begin{macrocode}
%<*example-sec>
\documentclass{article}
\usepackage{eolgrab}
\makeatletter
\newcommand*{\eolsection}{%
  \@ifstar{%
    \eolgrab{\section*}%
  }{%
    \@ifnextchar[{%
      \eoloptsection
    }{%
      \eolgrab\section
    }%
  }%
}
\newcommand*{\eoloptsection}[1][]{%
  \eolgrab{\section[{#1}]}%
}
\makeatother
\begin{document}
\tableofcontents
\eolsection Section without star and optional argument
\eolsection*Section with star
\eolsection[Short section title]Long section title
\end{document}
%</example-sec>
%    \end{macrocode}
% \end{itemize}
%
% \subsubsection{Small \hologo{LaTeX} document as example}
%
%    \begin{macrocode}
%<*example-ltx>
\RequirePackage{eolgrab}
\eolgrab\documentclass article
\eolgrab\begin document
\eolgrab\section Hello World
\eolgrab\emph Some text
\eolgrab\end document
%</example-ltx>
%    \end{macrocode}
%
% \subsubsection{\hologo{LaTeX} document with environments}
%
%    \begin{macrocode}
%<*example-env>
\documentclass{article}
\usepackage{eolgrab}
\newcommand*{\Begin}{\eolgrab   \begin}
\newcommand*{\End  }{\eolgrab   \end  }
\newcommand*{\Item }{\eolgrabopt\item }
\Begin document
  \Begin itemize
    \Item
      first item
    \Item
      second item
  \End itemize
  \Begin description
    \Item foo
      is the first syllable of foobar.
    \Item bar
      is the second syllable of foobar.
  \End description
\End document
%</example-env>
%    \end{macrocode}
%
% \subsection{Limitations}
%
% Macro \cs{eolgrab} needs to catch the line end.
% If \hologo{TeX} reads a line, then it throws away the line end
% characters (carriage return, line feed) and removes spaces at the
% end of the line. Then it adds the character with the character code
% that is given by \cs{endlinechar} at the end of the line. The category
% code of the inserted character is given by the current value of
% its \cs{catcode}. If \cs{endlinechar} is not a valid character code
% (especially if it is negative), then no character is added.
%
% In \hologo{plainTeX} and \hologo{LaTeX} the standard settings
% of the inserted endline character is the character with code 13
% (or |^||^||M| in \hologo{TeX} notation) with catcode 5 (end of line).
% That means the inserted end of line character behaves like a space
% token. For example, it is removed after macro names. Therefore
% \cs{eolgrab} changes the catcode.
%
% Therefore \cs{eolgrab} has some limitations:
% \begin{itemize}
% \item
%   Like other verbatim stuff, the macro \cs{eolgrab} cannot be
%   used in the argument of other macros. \cs{eolgrab} want to
%   change the catcode of the end of line character. If this
%   character is read before, because it is processed as argument
%   of another macro, the catcode is already set and is not reassigned
%   later if \cs{eolgrab} changes the category code for this character
%   code.
% \item
%   The argument must not contain the end of line character.
%   Otherwise the first end of line character is already taken
%   as delimiter, leaving the rest of the line outside the argument.
% \item
%   Because \cs{eolgrab} is probably mostly used in the line with
%   the delimited argument. Therefore changes of \cs{endlinechar}
%   will not affect the current line.
% \end{itemize}
%
% \StopEventually{
% }
%
% \section{Implementation}
%
%    \begin{macrocode}
%<*package>
%    \end{macrocode}
%
% \subsection{Reload check and package identification}
%    Reload check, especially if the package is not used with \LaTeX.
%    \begin{macrocode}
\begingroup\catcode61\catcode48\catcode32=10\relax%
  \catcode13=5 % ^^M
  \endlinechar=13 %
  \catcode35=6 % #
  \catcode39=12 % '
  \catcode44=12 % ,
  \catcode45=12 % -
  \catcode46=12 % .
  \catcode58=12 % :
  \catcode64=11 % @
  \catcode123=1 % {
  \catcode125=2 % }
  \expandafter\let\expandafter\x\csname ver@eolgrab.sty\endcsname
  \ifx\x\relax % plain-TeX, first loading
  \else
    \def\empty{}%
    \ifx\x\empty % LaTeX, first loading,
      % variable is initialized, but \ProvidesPackage not yet seen
    \else
      \expandafter\ifx\csname PackageInfo\endcsname\relax
        \def\x#1#2{%
          \immediate\write-1{Package #1 Info: #2.}%
        }%
      \else
        \def\x#1#2{\PackageInfo{#1}{#2, stopped}}%
      \fi
      \x{eolgrab}{The package is already loaded}%
      \aftergroup\endinput
    \fi
  \fi
\endgroup%
%    \end{macrocode}
%    Package identification:
%    \begin{macrocode}
\begingroup\catcode61\catcode48\catcode32=10\relax%
  \catcode13=5 % ^^M
  \endlinechar=13 %
  \catcode35=6 % #
  \catcode39=12 % '
  \catcode40=12 % (
  \catcode41=12 % )
  \catcode44=12 % ,
  \catcode45=12 % -
  \catcode46=12 % .
  \catcode47=12 % /
  \catcode58=12 % :
  \catcode64=11 % @
  \catcode91=12 % [
  \catcode93=12 % ]
  \catcode123=1 % {
  \catcode125=2 % }
  \expandafter\ifx\csname ProvidesPackage\endcsname\relax
    \def\x#1#2#3[#4]{\endgroup
      \immediate\write-1{Package: #3 #4}%
      \xdef#1{#4}%
    }%
  \else
    \def\x#1#2[#3]{\endgroup
      #2[{#3}]%
      \ifx#1\@undefined
        \xdef#1{#3}%
      \fi
      \ifx#1\relax
        \xdef#1{#3}%
      \fi
    }%
  \fi
\expandafter\x\csname ver@eolgrab.sty\endcsname
\ProvidesPackage{eolgrab}%
  [2016/05/16 v1.1 Catch arguments delimited by end of line (HO)]%
%    \end{macrocode}
%
% \subsection{Catcodes}
%
%    \begin{macrocode}
\begingroup\catcode61\catcode48\catcode32=10\relax%
  \catcode13=5 % ^^M
  \endlinechar=13 %
  \catcode123=1 % {
  \catcode125=2 % }
  \catcode64=11 % @
  \def\x{\endgroup
    \expandafter\edef\csname eolgrab@AtEnd\endcsname{%
      \endlinechar=\the\endlinechar\relax
      \catcode13=\the\catcode13\relax
      \catcode32=\the\catcode32\relax
      \catcode35=\the\catcode35\relax
      \catcode61=\the\catcode61\relax
      \catcode64=\the\catcode64\relax
      \catcode123=\the\catcode123\relax
      \catcode125=\the\catcode125\relax
    }%
  }%
\x\catcode61\catcode48\catcode32=10\relax%
\catcode13=5 % ^^M
\endlinechar=13 %
\catcode35=6 % #
\catcode64=11 % @
\catcode123=1 % {
\catcode125=2 % }
\def\TMP@EnsureCode#1#2{%
  \edef\eolgrab@AtEnd{%
    \eolgrab@AtEnd
    \catcode#1=\the\catcode#1\relax
  }%
  \catcode#1=#2\relax
}
\TMP@EnsureCode{40}{12}% (
\TMP@EnsureCode{41}{12}% )
\TMP@EnsureCode{42}{12}% *
\TMP@EnsureCode{46}{12}% .
\TMP@EnsureCode{47}{12}% /
\TMP@EnsureCode{91}{12}% [
\TMP@EnsureCode{93}{12}% ]
\TMP@EnsureCode{94}{7}% ^
\edef\eolgrab@AtEnd{\eolgrab@AtEnd\noexpand\endinput}
%    \end{macrocode}
%
% \subsection{Resources}
%
%    \begin{macrocode}
\begingroup\expandafter\expandafter\expandafter\endgroup
\expandafter\ifx\csname RequirePackage\endcsname\relax
  \input ltxcmds.sty\relax
  \input infwarerr.sty\relax
\else
  \RequirePackage{ltxcmds}[2010/12/04]%
  \RequirePackage{infwarerr}[2010/04/08]%
\fi
%    \end{macrocode}
%
%    \begin{macro}{\eolgrab@ifdefinable}
%    \begin{macrocode}
\ltx@IfUndefined{@ifdefinable}{%
  \def\eolgrab@ifdefinable#1#2{%
    \ltx@ifundefined{#1}{#2}{%
      \@PackageError{eolgrab}{%
        Command \ltx@backslashchar#1 already defined%
      }\@ehc
    }%
  }%
}{%
  \def\eolgrab@ifdefinable#1{%
    \expandafter\@ifdefinable\csname#1\endcsname
  }%
}
%    \end{macrocode}
%    \end{macro}
%
% \subsection{Macro \cs{eolgrab}}
%
%    \begin{macro}{\eolgrab}
%    \begin{macrocode}
\eolgrab@ifdefinable{eolgrab}{%
  \ltx@IfUndefined{protected}{%
    \ltx@IfUndefined{DeclareRobustCommand}{%
      \def\eolgrab#1%
    }{%
      \newcommand\eolgrab{}%
      \DeclareRobustCommand*\eolgrab
    }%
  }{%
    \protected\def\eolgrab#1%
  }{%
    \begingroup
    \endlinechar=13 %
    \catcode13=\ltx@active
    \eolgrab@{#1}%
  }%
}
%    \end{macrocode}
%    \end{macro}
%
%    \begin{macro}{\eolgrabopt}
%    \begin{macrocode}
\eolgrab@ifdefinable{eolgrabopt}{%
  \ltx@IfUndefined{protected}{%
    \ltx@IfUndefined{DeclareRobustCommand}{%
      \def\eolgrabopt#1%
    }{%
      \newcommand\eolgrabopt{}%
      \DeclareRobustCommand*\eolgrabopt
    }%
  }{%
    \protected\def\eolgrabopt#1%
  }{%
    \begingroup
    \endlinechar=13 %
    \catcode13=\ltx@active
    \eolgrab@opt{#1}%
  }%
}
%    \end{macrocode}
%    \end{macro}
%
%    \begin{macrocode}
\begingroup
  \catcode13=\ltx@active %
\ltx@firstofone{\endgroup %
%    \end{macrocode}
%    \begin{macro}{\eolgrab@}
%    \begin{macrocode}
  \def\eolgrab@#1#2^^M{%
    \endgroup %
    #1{#2}%
  }%
%    \end{macrocode}
%    \end{macro}
%    \begin{macro}{\eolgrab@opt}
%    \begin{macrocode}
  \def\eolgrab@opt#1#2^^M{%
    \endgroup %
    \ltx@ifempty{#2}{%
      #1%
    }{%
      #1[{#2}]%
    }%
  }%
%    \end{macrocode}
%    \end{macro}
%    \begin{macrocode}
}%
%    \end{macrocode}
%
%    \begin{macrocode}
\eolgrab@AtEnd%
%    \end{macrocode}
%    \begin{macrocode}
%</package>
%    \end{macrocode}
%% \section{Installation}
%
% \subsection{Download}
%
% \paragraph{Package.} This package is available on
% CTAN\footnote{\CTANpkg{eolgrab}}:
% \begin{description}
% \item[\CTAN{macros/latex/contrib/oberdiek/eolgrab.dtx}] The source file.
% \item[\CTAN{macros/latex/contrib/oberdiek/eolgrab.pdf}] Documentation.
% \end{description}
%
%
% \paragraph{Bundle.} All the packages of the bundle `oberdiek'
% are also available in a TDS compliant ZIP archive. There
% the packages are already unpacked and the documentation files
% are generated. The files and directories obey the TDS standard.
% \begin{description}
% \item[\CTANinstall{install/macros/latex/contrib/oberdiek.tds.zip}]
% \end{description}
% \emph{TDS} refers to the standard ``A Directory Structure
% for \TeX\ Files'' (\CTANpkg{tds}). Directories
% with \xfile{texmf} in their name are usually organized this way.
%
% \subsection{Bundle installation}
%
% \paragraph{Unpacking.} Unpack the \xfile{oberdiek.tds.zip} in the
% TDS tree (also known as \xfile{texmf} tree) of your choice.
% Example (linux):
% \begin{quote}
%   |unzip oberdiek.tds.zip -d ~/texmf|
% \end{quote}
%
% \subsection{Package installation}
%
% \paragraph{Unpacking.} The \xfile{.dtx} file is a self-extracting
% \docstrip\ archive. The files are extracted by running the
% \xfile{.dtx} through \plainTeX:
% \begin{quote}
%   \verb|tex eolgrab.dtx|
% \end{quote}
%
% \paragraph{TDS.} Now the different files must be moved into
% the different directories in your installation TDS tree
% (also known as \xfile{texmf} tree):
% \begin{quote}
% \def\t{^^A
% \begin{tabular}{@{}>{\ttfamily}l@{ $\rightarrow$ }>{\ttfamily}l@{}}
%   eolgrab.sty & tex/generic/oberdiek/eolgrab.sty\\
%   eolgrab.pdf & doc/latex/oberdiek/eolgrab.pdf\\
%   example/eolgrab-example-ltx.tex & doc/latex/oberdiek/example/eolgrab-example-ltx.tex\\
%   example/eolgrab-example-env.tex & doc/latex/oberdiek/example/eolgrab-example-env.tex\\
%   example/eolgrab-example-sec.tex & doc/latex/oberdiek/example/eolgrab-example-sec.tex\\
%   eolgrab.dtx & source/latex/oberdiek/eolgrab.dtx\\
% \end{tabular}^^A
% }^^A
% \sbox0{\t}^^A
% \ifdim\wd0>\linewidth
%   \begingroup
%     \advance\linewidth by\leftmargin
%     \advance\linewidth by\rightmargin
%   \edef\x{\endgroup
%     \def\noexpand\lw{\the\linewidth}^^A
%   }\x
%   \def\lwbox{^^A
%     \leavevmode
%     \hbox to \linewidth{^^A
%       \kern-\leftmargin\relax
%       \hss
%       \usebox0
%       \hss
%       \kern-\rightmargin\relax
%     }^^A
%   }^^A
%   \ifdim\wd0>\lw
%     \sbox0{\small\t}^^A
%     \ifdim\wd0>\linewidth
%       \ifdim\wd0>\lw
%         \sbox0{\footnotesize\t}^^A
%         \ifdim\wd0>\linewidth
%           \ifdim\wd0>\lw
%             \sbox0{\scriptsize\t}^^A
%             \ifdim\wd0>\linewidth
%               \ifdim\wd0>\lw
%                 \sbox0{\tiny\t}^^A
%                 \ifdim\wd0>\linewidth
%                   \lwbox
%                 \else
%                   \usebox0
%                 \fi
%               \else
%                 \lwbox
%               \fi
%             \else
%               \usebox0
%             \fi
%           \else
%             \lwbox
%           \fi
%         \else
%           \usebox0
%         \fi
%       \else
%         \lwbox
%       \fi
%     \else
%       \usebox0
%     \fi
%   \else
%     \lwbox
%   \fi
% \else
%   \usebox0
% \fi
% \end{quote}
% If you have a \xfile{docstrip.cfg} that configures and enables \docstrip's
% TDS installing feature, then some files can already be in the right
% place, see the documentation of \docstrip.
%
% \subsection{Refresh file name databases}
%
% If your \TeX~distribution
% (\TeX\,Live, \mikTeX, \dots) relies on file name databases, you must refresh
% these. For example, \TeX\,Live\ users run \verb|texhash| or
% \verb|mktexlsr|.
%
% \subsection{Some details for the interested}
%
% \paragraph{Unpacking with \LaTeX.}
% The \xfile{.dtx} chooses its action depending on the format:
% \begin{description}
% \item[\plainTeX:] Run \docstrip\ and extract the files.
% \item[\LaTeX:] Generate the documentation.
% \end{description}
% If you insist on using \LaTeX\ for \docstrip\ (really,
% \docstrip\ does not need \LaTeX), then inform the autodetect routine
% about your intention:
% \begin{quote}
%   \verb|latex \let\install=y% \iffalse meta-comment
%
% File: eolgrab.dtx
% Version: 2016/05/16 v1.1
% Info: Catch arguments delimited by end of line
%
% Copyright (C)
%    2011 Heiko Oberdiek
%    2016-2019 Oberdiek Package Support Group
%    https://github.com/ho-tex/oberdiek/issues
%
% This work may be distributed and/or modified under the
% conditions of the LaTeX Project Public License, either
% version 1.3c of this license or (at your option) any later
% version. This version of this license is in
%    https://www.latex-project.org/lppl/lppl-1-3c.txt
% and the latest version of this license is in
%    https://www.latex-project.org/lppl.txt
% and version 1.3 or later is part of all distributions of
% LaTeX version 2005/12/01 or later.
%
% This work has the LPPL maintenance status "maintained".
%
% The Current Maintainers of this work are
% Heiko Oberdiek and the Oberdiek Package Support Group
% https://github.com/ho-tex/oberdiek/issues
%
% The Base Interpreter refers to any `TeX-Format',
% because some files are installed in TDS:tex/generic//.
%
% This work consists of the main source file eolgrab.dtx
% and the derived files
%    eolgrab.sty, eolgrab.pdf, eolgrab.ins, eolgrab.drv,
%    eolgrab-example-ltx.tex, eolgrab-example-env.tex,
%    eolgrab-example-sec.tex, eolgrab-test1.tex,
%    eolgrab-test2.tex.
%
% Distribution:
%    CTAN:macros/latex/contrib/oberdiek/eolgrab.dtx
%    CTAN:macros/latex/contrib/oberdiek/eolgrab.pdf
%
% Unpacking:
%    (a) If eolgrab.ins is present:
%           tex eolgrab.ins
%    (b) Without eolgrab.ins:
%           tex eolgrab.dtx
%    (c) If you insist on using LaTeX
%           latex \let\install=y\input{eolgrab.dtx}
%        (quote the arguments according to the demands of your shell)
%
% Documentation:
%    (a) If eolgrab.drv is present:
%           latex eolgrab.drv
%    (b) Without eolgrab.drv:
%           latex eolgrab.dtx; ...
%    The class ltxdoc loads the configuration file ltxdoc.cfg
%    if available. Here you can specify further options, e.g.
%    use A4 as paper format:
%       \PassOptionsToClass{a4paper}{article}
%
%    Programm calls to get the documentation (example):
%       pdflatex eolgrab.dtx
%       makeindex -s gind.ist eolgrab.idx
%       pdflatex eolgrab.dtx
%       makeindex -s gind.ist eolgrab.idx
%       pdflatex eolgrab.dtx
%
% Installation:
%    TDS:tex/generic/oberdiek/eolgrab.sty
%    TDS:doc/latex/oberdiek/eolgrab.pdf
%    TDS:doc/latex/oberdiek/example/eolgrab-example-ltx.tex
%    TDS:doc/latex/oberdiek/example/eolgrab-example-env.tex
%    TDS:doc/latex/oberdiek/example/eolgrab-example-sec.tex
%    TDS:source/latex/oberdiek/eolgrab.dtx
%
%<*ignore>
\begingroup
  \catcode123=1 %
  \catcode125=2 %
  \def\x{LaTeX2e}%
\expandafter\endgroup
\ifcase 0\ifx\install y1\fi\expandafter
         \ifx\csname processbatchFile\endcsname\relax\else1\fi
         \ifx\fmtname\x\else 1\fi\relax
\else\csname fi\endcsname
%</ignore>
%<*install>
\input docstrip.tex
\Msg{************************************************************************}
\Msg{* Installation}
\Msg{* Package: eolgrab 2016/05/16 v1.1 Catch arguments delimited by end of line (HO)}
\Msg{************************************************************************}

\keepsilent
\askforoverwritefalse

\let\MetaPrefix\relax
\preamble

This is a generated file.

Project: eolgrab
Version: 2016/05/16 v1.1

Copyright (C)
   2011 Heiko Oberdiek
   2016-2019 Oberdiek Package Support Group

This work may be distributed and/or modified under the
conditions of the LaTeX Project Public License, either
version 1.3c of this license or (at your option) any later
version. This version of this license is in
   https://www.latex-project.org/lppl/lppl-1-3c.txt
and the latest version of this license is in
   https://www.latex-project.org/lppl.txt
and version 1.3 or later is part of all distributions of
LaTeX version 2005/12/01 or later.

This work has the LPPL maintenance status "maintained".

The Current Maintainers of this work are
Heiko Oberdiek and the Oberdiek Package Support Group
https://github.com/ho-tex/oberdiek/issues


The Base Interpreter refers to any `TeX-Format',
because some files are installed in TDS:tex/generic//.

This work consists of the main source file eolgrab.dtx
and the derived files
   eolgrab.sty, eolgrab.pdf, eolgrab.ins, eolgrab.drv,
   eolgrab-example-ltx.tex, eolgrab-example-env.tex,
   eolgrab-example-sec.tex, eolgrab-test1.tex,
   eolgrab-test2.tex.

\endpreamble
\let\MetaPrefix\DoubleperCent

\generate{%
  \file{eolgrab.ins}{\from{eolgrab.dtx}{install}}%
  \file{eolgrab.drv}{\from{eolgrab.dtx}{driver}}%
  \usedir{tex/generic/oberdiek}%
  \file{eolgrab.sty}{\from{eolgrab.dtx}{package}}%
  \usedir{doc/latex/oberdiek/example}%
  \file{eolgrab-example-ltx.tex}{\from{eolgrab.dtx}{example-ltx}}%
  \file{eolgrab-example-env.tex}{\from{eolgrab.dtx}{example-env}}%
  \file{eolgrab-example-sec.tex}{\from{eolgrab.dtx}{example-sec}}%
%  \usedir{doc/latex/oberdiek/test}%
%  \file{eolgrab-test1.tex}{\from{eolgrab.dtx}{test1}}%
%  \file{eolgrab-test2.tex}{\from{eolgrab.dtx}{test2}}%
}

\catcode32=13\relax% active space
\let =\space%
\Msg{************************************************************************}
\Msg{*}
\Msg{* To finish the installation you have to move the following}
\Msg{* file into a directory searched by TeX:}
\Msg{*}
\Msg{*     eolgrab.sty}
\Msg{*}
\Msg{* To produce the documentation run the file `eolgrab.drv'}
\Msg{* through LaTeX.}
\Msg{*}
\Msg{* Happy TeXing!}
\Msg{*}
\Msg{************************************************************************}

\endbatchfile
%</install>
%<*ignore>
\fi
%</ignore>
%<*driver>
\NeedsTeXFormat{LaTeX2e}
\ProvidesFile{eolgrab.drv}%
  [2016/05/16 v1.1 Catch arguments delimited by end of line (HO)]%
\documentclass{ltxdoc}
\usepackage{holtxdoc}[2011/11/22]
\begin{document}
  \DocInput{eolgrab.dtx}%
\end{document}
%</driver>
% \fi
%
%
%
% \GetFileInfo{eolgrab.drv}
%
% \title{The \xpackage{eolgrab} package}
% \date{2016/05/16 v1.1}
% \author{Heiko Oberdiek\thanks
% {Please report any issues at \url{https://github.com/ho-tex/oberdiek/issues}}}
%
% \maketitle
%
% \begin{abstract}
% This package implements a generic argument grabber
% to catch an argument that is delimited by the line end.
% \end{abstract}
%
% \tableofcontents
%
% \urldef\urltexhax\url{https://tug.org/mailman/listinfo/texhax}
% \def\eol/{^^A
%   \meta{\texttt{EOL}}^^A
% }
%
% \section{Documentation}
%
% The starting point for this package was a feature
% request of Arno Trautmann in the mailing list
% \texttt{texhax}\footnote{Info page for mailing list
% \texttt{texhax}: \urltexhax} \cite{trautmann-request}.
% A macro \cs{eolsection} should behave like \cs{section},
% but the argument should be delimited by the line end
% instead of given in curly braces:
% \begin{quote}
%   |\eolsection My Title|
% \end{quote}
% Phil Taylor answered this with an implementation for \cs{eolsection}.
% Because this feature could be useful for other macros as well,
% I answered with an implementation of \cs{eolgrab} as general solution
% \cite{oberdiek-eolgrab}.
%
% Both formats \hologo{plainTeX} and \hologo{LaTeX} are supported
% by the package, see the example for \cs{eolsection} below.
%
% \begin{declcs}{eolgrab} \M{code} \meta{argument} \eol/
% \end{declcs}
% Macro \cs{eolgrab} takes two arguments. The first argument
% is \meta{code}, a classical undelimited \hologo{TeX} macro argument.
% The second argument is delimited by the line end \eol/.
% The macro calls \meta{code} with \meta{argument} as argument
% in curly braces. Because the catcode of the line end is changed,
% \cs{eolgrab} will not work in the argument of other macros.
% Macro \cs{eolgrab} is made robust if either \hologo{eTeX}'s
% \cs{protected} or \hologo{LaTeX}'s \cs{DeclareRobustCommand}
% is available.
%
% \begin{declcs}{eolgrabopt} \M{code} \meta{argument} \eol/
% \end{declcs}
% Macro \cs{eolgrabopt} passes \meta{argument} as optional
% argument to \meta{code} if \meta{argument} is not empty.
% \begin{quote}
%   |\eolgrabopt\item foo|
% \end{quote}
% becomes to
% \begin{quote}
%   |\item[{foo}]|
% \end{quote}
% The curly argument braces are added to support square brackets
% inside \meta{argument}.
% If the \meta{argument} is empty:
% \begin{quote}
%   |\eolgrabopt\item|
% \end{quote}
% then
% \begin{quote}
%   |\item|
% \end{quote}
% is called without optional argument.
%
% \subsection{Examples}
%
% \begin{itemize}
% \item
%   The line
%   \begin{quote}
%     |\eolgrab\section My Title|
%   \end{quote}
%   is equivalent to
%   \begin{quote}
%     |\section{My Title}|
%   \end{quote}
% \item
%   The next example uses the star form of \cs{section}. Then the
%   command to be called consists of two tokens. Therefore the
%   first argument of \cs{eolgrab} needs curly braces:
%   \begin{quote}
%     |\eolgrab{\section*}My Title|
%   \end{quote}
%   becomes
%   \begin{quote}
%     |\section*{My Title}|
%   \end{quote}
% \item
%   Now \hologo{LaTeX}'s \cs{PackageError} is used. This macro
%   has three arguments, the package or class name, the message
%   text and the help text. A standard help text of \hologo{LaTeX}
%   is used as given in macro \cs{@ehc}. The second argument,
%   the message text is used as argument, delimited by line end:
%   \begin{quote}
%     |\eolgrab{\PackageError{foobar}}%|\\
%     |Some error message text\MessageBreak%|\\
%     |with several lines|\\
%     |\@ehc|
%   \end{quote}
%   In the first two lines of the example, the line end is suppressed
%   by the comment character (percent), thus the argument is delimited
%   by the line end of the third line. The result is:
%   \begin{quote}
%     |\PackageError{foobar}{Some error message text\MessageBreak|\\
%     |with several lines}\@ehc|
%   \end{quote}
% \item
%   The original request for macro \cs{eolsection}, see above,
%   can be implemented easily with the help of \cs{eolgrab}.
%   Example for \hologo{LaTeX}:
%   \begin{quote}
%     |\usepackage{eolgrab}|\\
%     |\newcommand*{\eolsection}{\eolgrab\section}|
%   \end{quote}
%   Example for \hologo{plainTeX}:
%   \begin{quote}
%     |\input eolgrab.sty\relax|\\
%     |\def\eolsection{\eolgrab\section}|
%   \end{quote}
%   And a sophisticated variant for \hologo{LaTeX} that also supports
%   the star syntax and the optional argument:
%    \begin{macrocode}
%<*example-sec>
\documentclass{article}
\usepackage{eolgrab}
\makeatletter
\newcommand*{\eolsection}{%
  \@ifstar{%
    \eolgrab{\section*}%
  }{%
    \@ifnextchar[{%
      \eoloptsection
    }{%
      \eolgrab\section
    }%
  }%
}
\newcommand*{\eoloptsection}[1][]{%
  \eolgrab{\section[{#1}]}%
}
\makeatother
\begin{document}
\tableofcontents
\eolsection Section without star and optional argument
\eolsection*Section with star
\eolsection[Short section title]Long section title
\end{document}
%</example-sec>
%    \end{macrocode}
% \end{itemize}
%
% \subsubsection{Small \hologo{LaTeX} document as example}
%
%    \begin{macrocode}
%<*example-ltx>
\RequirePackage{eolgrab}
\eolgrab\documentclass article
\eolgrab\begin document
\eolgrab\section Hello World
\eolgrab\emph Some text
\eolgrab\end document
%</example-ltx>
%    \end{macrocode}
%
% \subsubsection{\hologo{LaTeX} document with environments}
%
%    \begin{macrocode}
%<*example-env>
\documentclass{article}
\usepackage{eolgrab}
\newcommand*{\Begin}{\eolgrab   \begin}
\newcommand*{\End  }{\eolgrab   \end  }
\newcommand*{\Item }{\eolgrabopt\item }
\Begin document
  \Begin itemize
    \Item
      first item
    \Item
      second item
  \End itemize
  \Begin description
    \Item foo
      is the first syllable of foobar.
    \Item bar
      is the second syllable of foobar.
  \End description
\End document
%</example-env>
%    \end{macrocode}
%
% \subsection{Limitations}
%
% Macro \cs{eolgrab} needs to catch the line end.
% If \hologo{TeX} reads a line, then it throws away the line end
% characters (carriage return, line feed) and removes spaces at the
% end of the line. Then it adds the character with the character code
% that is given by \cs{endlinechar} at the end of the line. The category
% code of the inserted character is given by the current value of
% its \cs{catcode}. If \cs{endlinechar} is not a valid character code
% (especially if it is negative), then no character is added.
%
% In \hologo{plainTeX} and \hologo{LaTeX} the standard settings
% of the inserted endline character is the character with code 13
% (or |^||^||M| in \hologo{TeX} notation) with catcode 5 (end of line).
% That means the inserted end of line character behaves like a space
% token. For example, it is removed after macro names. Therefore
% \cs{eolgrab} changes the catcode.
%
% Therefore \cs{eolgrab} has some limitations:
% \begin{itemize}
% \item
%   Like other verbatim stuff, the macro \cs{eolgrab} cannot be
%   used in the argument of other macros. \cs{eolgrab} want to
%   change the catcode of the end of line character. If this
%   character is read before, because it is processed as argument
%   of another macro, the catcode is already set and is not reassigned
%   later if \cs{eolgrab} changes the category code for this character
%   code.
% \item
%   The argument must not contain the end of line character.
%   Otherwise the first end of line character is already taken
%   as delimiter, leaving the rest of the line outside the argument.
% \item
%   Because \cs{eolgrab} is probably mostly used in the line with
%   the delimited argument. Therefore changes of \cs{endlinechar}
%   will not affect the current line.
% \end{itemize}
%
% \StopEventually{
% }
%
% \section{Implementation}
%
%    \begin{macrocode}
%<*package>
%    \end{macrocode}
%
% \subsection{Reload check and package identification}
%    Reload check, especially if the package is not used with \LaTeX.
%    \begin{macrocode}
\begingroup\catcode61\catcode48\catcode32=10\relax%
  \catcode13=5 % ^^M
  \endlinechar=13 %
  \catcode35=6 % #
  \catcode39=12 % '
  \catcode44=12 % ,
  \catcode45=12 % -
  \catcode46=12 % .
  \catcode58=12 % :
  \catcode64=11 % @
  \catcode123=1 % {
  \catcode125=2 % }
  \expandafter\let\expandafter\x\csname ver@eolgrab.sty\endcsname
  \ifx\x\relax % plain-TeX, first loading
  \else
    \def\empty{}%
    \ifx\x\empty % LaTeX, first loading,
      % variable is initialized, but \ProvidesPackage not yet seen
    \else
      \expandafter\ifx\csname PackageInfo\endcsname\relax
        \def\x#1#2{%
          \immediate\write-1{Package #1 Info: #2.}%
        }%
      \else
        \def\x#1#2{\PackageInfo{#1}{#2, stopped}}%
      \fi
      \x{eolgrab}{The package is already loaded}%
      \aftergroup\endinput
    \fi
  \fi
\endgroup%
%    \end{macrocode}
%    Package identification:
%    \begin{macrocode}
\begingroup\catcode61\catcode48\catcode32=10\relax%
  \catcode13=5 % ^^M
  \endlinechar=13 %
  \catcode35=6 % #
  \catcode39=12 % '
  \catcode40=12 % (
  \catcode41=12 % )
  \catcode44=12 % ,
  \catcode45=12 % -
  \catcode46=12 % .
  \catcode47=12 % /
  \catcode58=12 % :
  \catcode64=11 % @
  \catcode91=12 % [
  \catcode93=12 % ]
  \catcode123=1 % {
  \catcode125=2 % }
  \expandafter\ifx\csname ProvidesPackage\endcsname\relax
    \def\x#1#2#3[#4]{\endgroup
      \immediate\write-1{Package: #3 #4}%
      \xdef#1{#4}%
    }%
  \else
    \def\x#1#2[#3]{\endgroup
      #2[{#3}]%
      \ifx#1\@undefined
        \xdef#1{#3}%
      \fi
      \ifx#1\relax
        \xdef#1{#3}%
      \fi
    }%
  \fi
\expandafter\x\csname ver@eolgrab.sty\endcsname
\ProvidesPackage{eolgrab}%
  [2016/05/16 v1.1 Catch arguments delimited by end of line (HO)]%
%    \end{macrocode}
%
% \subsection{Catcodes}
%
%    \begin{macrocode}
\begingroup\catcode61\catcode48\catcode32=10\relax%
  \catcode13=5 % ^^M
  \endlinechar=13 %
  \catcode123=1 % {
  \catcode125=2 % }
  \catcode64=11 % @
  \def\x{\endgroup
    \expandafter\edef\csname eolgrab@AtEnd\endcsname{%
      \endlinechar=\the\endlinechar\relax
      \catcode13=\the\catcode13\relax
      \catcode32=\the\catcode32\relax
      \catcode35=\the\catcode35\relax
      \catcode61=\the\catcode61\relax
      \catcode64=\the\catcode64\relax
      \catcode123=\the\catcode123\relax
      \catcode125=\the\catcode125\relax
    }%
  }%
\x\catcode61\catcode48\catcode32=10\relax%
\catcode13=5 % ^^M
\endlinechar=13 %
\catcode35=6 % #
\catcode64=11 % @
\catcode123=1 % {
\catcode125=2 % }
\def\TMP@EnsureCode#1#2{%
  \edef\eolgrab@AtEnd{%
    \eolgrab@AtEnd
    \catcode#1=\the\catcode#1\relax
  }%
  \catcode#1=#2\relax
}
\TMP@EnsureCode{40}{12}% (
\TMP@EnsureCode{41}{12}% )
\TMP@EnsureCode{42}{12}% *
\TMP@EnsureCode{46}{12}% .
\TMP@EnsureCode{47}{12}% /
\TMP@EnsureCode{91}{12}% [
\TMP@EnsureCode{93}{12}% ]
\TMP@EnsureCode{94}{7}% ^
\edef\eolgrab@AtEnd{\eolgrab@AtEnd\noexpand\endinput}
%    \end{macrocode}
%
% \subsection{Resources}
%
%    \begin{macrocode}
\begingroup\expandafter\expandafter\expandafter\endgroup
\expandafter\ifx\csname RequirePackage\endcsname\relax
  \input ltxcmds.sty\relax
  \input infwarerr.sty\relax
\else
  \RequirePackage{ltxcmds}[2010/12/04]%
  \RequirePackage{infwarerr}[2010/04/08]%
\fi
%    \end{macrocode}
%
%    \begin{macro}{\eolgrab@ifdefinable}
%    \begin{macrocode}
\ltx@IfUndefined{@ifdefinable}{%
  \def\eolgrab@ifdefinable#1#2{%
    \ltx@ifundefined{#1}{#2}{%
      \@PackageError{eolgrab}{%
        Command \ltx@backslashchar#1 already defined%
      }\@ehc
    }%
  }%
}{%
  \def\eolgrab@ifdefinable#1{%
    \expandafter\@ifdefinable\csname#1\endcsname
  }%
}
%    \end{macrocode}
%    \end{macro}
%
% \subsection{Macro \cs{eolgrab}}
%
%    \begin{macro}{\eolgrab}
%    \begin{macrocode}
\eolgrab@ifdefinable{eolgrab}{%
  \ltx@IfUndefined{protected}{%
    \ltx@IfUndefined{DeclareRobustCommand}{%
      \def\eolgrab#1%
    }{%
      \newcommand\eolgrab{}%
      \DeclareRobustCommand*\eolgrab
    }%
  }{%
    \protected\def\eolgrab#1%
  }{%
    \begingroup
    \endlinechar=13 %
    \catcode13=\ltx@active
    \eolgrab@{#1}%
  }%
}
%    \end{macrocode}
%    \end{macro}
%
%    \begin{macro}{\eolgrabopt}
%    \begin{macrocode}
\eolgrab@ifdefinable{eolgrabopt}{%
  \ltx@IfUndefined{protected}{%
    \ltx@IfUndefined{DeclareRobustCommand}{%
      \def\eolgrabopt#1%
    }{%
      \newcommand\eolgrabopt{}%
      \DeclareRobustCommand*\eolgrabopt
    }%
  }{%
    \protected\def\eolgrabopt#1%
  }{%
    \begingroup
    \endlinechar=13 %
    \catcode13=\ltx@active
    \eolgrab@opt{#1}%
  }%
}
%    \end{macrocode}
%    \end{macro}
%
%    \begin{macrocode}
\begingroup
  \catcode13=\ltx@active %
\ltx@firstofone{\endgroup %
%    \end{macrocode}
%    \begin{macro}{\eolgrab@}
%    \begin{macrocode}
  \def\eolgrab@#1#2^^M{%
    \endgroup %
    #1{#2}%
  }%
%    \end{macrocode}
%    \end{macro}
%    \begin{macro}{\eolgrab@opt}
%    \begin{macrocode}
  \def\eolgrab@opt#1#2^^M{%
    \endgroup %
    \ltx@ifempty{#2}{%
      #1%
    }{%
      #1[{#2}]%
    }%
  }%
%    \end{macrocode}
%    \end{macro}
%    \begin{macrocode}
}%
%    \end{macrocode}
%
%    \begin{macrocode}
\eolgrab@AtEnd%
%    \end{macrocode}
%    \begin{macrocode}
%</package>
%    \end{macrocode}
%% \section{Installation}
%
% \subsection{Download}
%
% \paragraph{Package.} This package is available on
% CTAN\footnote{\CTANpkg{eolgrab}}:
% \begin{description}
% \item[\CTAN{macros/latex/contrib/oberdiek/eolgrab.dtx}] The source file.
% \item[\CTAN{macros/latex/contrib/oberdiek/eolgrab.pdf}] Documentation.
% \end{description}
%
%
% \paragraph{Bundle.} All the packages of the bundle `oberdiek'
% are also available in a TDS compliant ZIP archive. There
% the packages are already unpacked and the documentation files
% are generated. The files and directories obey the TDS standard.
% \begin{description}
% \item[\CTANinstall{install/macros/latex/contrib/oberdiek.tds.zip}]
% \end{description}
% \emph{TDS} refers to the standard ``A Directory Structure
% for \TeX\ Files'' (\CTANpkg{tds}). Directories
% with \xfile{texmf} in their name are usually organized this way.
%
% \subsection{Bundle installation}
%
% \paragraph{Unpacking.} Unpack the \xfile{oberdiek.tds.zip} in the
% TDS tree (also known as \xfile{texmf} tree) of your choice.
% Example (linux):
% \begin{quote}
%   |unzip oberdiek.tds.zip -d ~/texmf|
% \end{quote}
%
% \subsection{Package installation}
%
% \paragraph{Unpacking.} The \xfile{.dtx} file is a self-extracting
% \docstrip\ archive. The files are extracted by running the
% \xfile{.dtx} through \plainTeX:
% \begin{quote}
%   \verb|tex eolgrab.dtx|
% \end{quote}
%
% \paragraph{TDS.} Now the different files must be moved into
% the different directories in your installation TDS tree
% (also known as \xfile{texmf} tree):
% \begin{quote}
% \def\t{^^A
% \begin{tabular}{@{}>{\ttfamily}l@{ $\rightarrow$ }>{\ttfamily}l@{}}
%   eolgrab.sty & tex/generic/oberdiek/eolgrab.sty\\
%   eolgrab.pdf & doc/latex/oberdiek/eolgrab.pdf\\
%   example/eolgrab-example-ltx.tex & doc/latex/oberdiek/example/eolgrab-example-ltx.tex\\
%   example/eolgrab-example-env.tex & doc/latex/oberdiek/example/eolgrab-example-env.tex\\
%   example/eolgrab-example-sec.tex & doc/latex/oberdiek/example/eolgrab-example-sec.tex\\
%   eolgrab.dtx & source/latex/oberdiek/eolgrab.dtx\\
% \end{tabular}^^A
% }^^A
% \sbox0{\t}^^A
% \ifdim\wd0>\linewidth
%   \begingroup
%     \advance\linewidth by\leftmargin
%     \advance\linewidth by\rightmargin
%   \edef\x{\endgroup
%     \def\noexpand\lw{\the\linewidth}^^A
%   }\x
%   \def\lwbox{^^A
%     \leavevmode
%     \hbox to \linewidth{^^A
%       \kern-\leftmargin\relax
%       \hss
%       \usebox0
%       \hss
%       \kern-\rightmargin\relax
%     }^^A
%   }^^A
%   \ifdim\wd0>\lw
%     \sbox0{\small\t}^^A
%     \ifdim\wd0>\linewidth
%       \ifdim\wd0>\lw
%         \sbox0{\footnotesize\t}^^A
%         \ifdim\wd0>\linewidth
%           \ifdim\wd0>\lw
%             \sbox0{\scriptsize\t}^^A
%             \ifdim\wd0>\linewidth
%               \ifdim\wd0>\lw
%                 \sbox0{\tiny\t}^^A
%                 \ifdim\wd0>\linewidth
%                   \lwbox
%                 \else
%                   \usebox0
%                 \fi
%               \else
%                 \lwbox
%               \fi
%             \else
%               \usebox0
%             \fi
%           \else
%             \lwbox
%           \fi
%         \else
%           \usebox0
%         \fi
%       \else
%         \lwbox
%       \fi
%     \else
%       \usebox0
%     \fi
%   \else
%     \lwbox
%   \fi
% \else
%   \usebox0
% \fi
% \end{quote}
% If you have a \xfile{docstrip.cfg} that configures and enables \docstrip's
% TDS installing feature, then some files can already be in the right
% place, see the documentation of \docstrip.
%
% \subsection{Refresh file name databases}
%
% If your \TeX~distribution
% (\TeX\,Live, \mikTeX, \dots) relies on file name databases, you must refresh
% these. For example, \TeX\,Live\ users run \verb|texhash| or
% \verb|mktexlsr|.
%
% \subsection{Some details for the interested}
%
% \paragraph{Unpacking with \LaTeX.}
% The \xfile{.dtx} chooses its action depending on the format:
% \begin{description}
% \item[\plainTeX:] Run \docstrip\ and extract the files.
% \item[\LaTeX:] Generate the documentation.
% \end{description}
% If you insist on using \LaTeX\ for \docstrip\ (really,
% \docstrip\ does not need \LaTeX), then inform the autodetect routine
% about your intention:
% \begin{quote}
%   \verb|latex \let\install=y\input{eolgrab.dtx}|
% \end{quote}
% Do not forget to quote the argument according to the demands
% of your shell.
%
% \paragraph{Generating the documentation.}
% You can use both the \xfile{.dtx} or the \xfile{.drv} to generate
% the documentation. The process can be configured by the
% configuration file \xfile{ltxdoc.cfg}. For instance, put this
% line into this file, if you want to have A4 as paper format:
% \begin{quote}
%   \verb|\PassOptionsToClass{a4paper}{article}|
% \end{quote}
% An example follows how to generate the
% documentation with pdf\LaTeX:
% \begin{quote}
%\begin{verbatim}
%pdflatex eolgrab.dtx
%makeindex -s gind.ist eolgrab.idx
%pdflatex eolgrab.dtx
%makeindex -s gind.ist eolgrab.idx
%pdflatex eolgrab.dtx
%\end{verbatim}
% \end{quote}
%
% \begin{thebibliography}{9}
% \raggedright
% \bibitem{trautmann-request}
% Arno Trautmann,
% \textit{[texhax] read argument until EOL};
% mailing list \texttt{texthax@tug.org},
% 2011-01-06;
% \url{https://tug.org/pipermail/texhax/2011-January/016517.html}.
%
% \bibitem{taylor}
% Philip Taylor,
% \textit{Re: [texhax] read argument until EOL};
% mailing list \texttt{texhax@tug.org},
% 2011-01-06;
% \url{https://tug.org/pipermail/texhax/2011-January/016519.html}.
%
% \bibitem{oberdiek-eolgrab}
% Heiko Oberdiek,
% \textit{Re: [texhax] read argument until EOL};
% mailing list \texttt{texhax@tug.org},
% 2011-01-06;
% \url{https://tug.org/pipermail/texhax/2011-January/016526.html}.
%
% \end{thebibliography}
%
% \begin{History}
%   \begin{Version}{2011/01/12 v1.0}
%   \item
%     First public version.
%   \end{Version}
%   \begin{Version}{2016/05/16 v1.1}
%   \item
%     Documentation updates.
%   \end{Version}
% \end{History}
%
% \PrintIndex
%
% \Finale
\endinput
|
% \end{quote}
% Do not forget to quote the argument according to the demands
% of your shell.
%
% \paragraph{Generating the documentation.}
% You can use both the \xfile{.dtx} or the \xfile{.drv} to generate
% the documentation. The process can be configured by the
% configuration file \xfile{ltxdoc.cfg}. For instance, put this
% line into this file, if you want to have A4 as paper format:
% \begin{quote}
%   \verb|\PassOptionsToClass{a4paper}{article}|
% \end{quote}
% An example follows how to generate the
% documentation with pdf\LaTeX:
% \begin{quote}
%\begin{verbatim}
%pdflatex eolgrab.dtx
%makeindex -s gind.ist eolgrab.idx
%pdflatex eolgrab.dtx
%makeindex -s gind.ist eolgrab.idx
%pdflatex eolgrab.dtx
%\end{verbatim}
% \end{quote}
%
% \begin{thebibliography}{9}
% \raggedright
% \bibitem{trautmann-request}
% Arno Trautmann,
% \textit{[texhax] read argument until EOL};
% mailing list \texttt{texthax@tug.org},
% 2011-01-06;
% \url{https://tug.org/pipermail/texhax/2011-January/016517.html}.
%
% \bibitem{taylor}
% Philip Taylor,
% \textit{Re: [texhax] read argument until EOL};
% mailing list \texttt{texhax@tug.org},
% 2011-01-06;
% \url{https://tug.org/pipermail/texhax/2011-January/016519.html}.
%
% \bibitem{oberdiek-eolgrab}
% Heiko Oberdiek,
% \textit{Re: [texhax] read argument until EOL};
% mailing list \texttt{texhax@tug.org},
% 2011-01-06;
% \url{https://tug.org/pipermail/texhax/2011-January/016526.html}.
%
% \end{thebibliography}
%
% \begin{History}
%   \begin{Version}{2011/01/12 v1.0}
%   \item
%     First public version.
%   \end{Version}
%   \begin{Version}{2016/05/16 v1.1}
%   \item
%     Documentation updates.
%   \end{Version}
% \end{History}
%
% \PrintIndex
%
% \Finale
\endinput

%        (quote the arguments according to the demands of your shell)
%
% Documentation:
%    (a) If eolgrab.drv is present:
%           latex eolgrab.drv
%    (b) Without eolgrab.drv:
%           latex eolgrab.dtx; ...
%    The class ltxdoc loads the configuration file ltxdoc.cfg
%    if available. Here you can specify further options, e.g.
%    use A4 as paper format:
%       \PassOptionsToClass{a4paper}{article}
%
%    Programm calls to get the documentation (example):
%       pdflatex eolgrab.dtx
%       makeindex -s gind.ist eolgrab.idx
%       pdflatex eolgrab.dtx
%       makeindex -s gind.ist eolgrab.idx
%       pdflatex eolgrab.dtx
%
% Installation:
%    TDS:tex/generic/oberdiek/eolgrab.sty
%    TDS:doc/latex/oberdiek/eolgrab.pdf
%    TDS:doc/latex/oberdiek/example/eolgrab-example-ltx.tex
%    TDS:doc/latex/oberdiek/example/eolgrab-example-env.tex
%    TDS:doc/latex/oberdiek/example/eolgrab-example-sec.tex
%    TDS:source/latex/oberdiek/eolgrab.dtx
%
%<*ignore>
\begingroup
  \catcode123=1 %
  \catcode125=2 %
  \def\x{LaTeX2e}%
\expandafter\endgroup
\ifcase 0\ifx\install y1\fi\expandafter
         \ifx\csname processbatchFile\endcsname\relax\else1\fi
         \ifx\fmtname\x\else 1\fi\relax
\else\csname fi\endcsname
%</ignore>
%<*install>
\input docstrip.tex
\Msg{************************************************************************}
\Msg{* Installation}
\Msg{* Package: eolgrab 2016/05/16 v1.1 Catch arguments delimited by end of line (HO)}
\Msg{************************************************************************}

\keepsilent
\askforoverwritefalse

\let\MetaPrefix\relax
\preamble

This is a generated file.

Project: eolgrab
Version: 2016/05/16 v1.1

Copyright (C)
   2011 Heiko Oberdiek
   2016-2019 Oberdiek Package Support Group

This work may be distributed and/or modified under the
conditions of the LaTeX Project Public License, either
version 1.3c of this license or (at your option) any later
version. This version of this license is in
   https://www.latex-project.org/lppl/lppl-1-3c.txt
and the latest version of this license is in
   https://www.latex-project.org/lppl.txt
and version 1.3 or later is part of all distributions of
LaTeX version 2005/12/01 or later.

This work has the LPPL maintenance status "maintained".

The Current Maintainers of this work are
Heiko Oberdiek and the Oberdiek Package Support Group
https://github.com/ho-tex/oberdiek/issues


The Base Interpreter refers to any `TeX-Format',
because some files are installed in TDS:tex/generic//.

This work consists of the main source file eolgrab.dtx
and the derived files
   eolgrab.sty, eolgrab.pdf, eolgrab.ins, eolgrab.drv,
   eolgrab-example-ltx.tex, eolgrab-example-env.tex,
   eolgrab-example-sec.tex, eolgrab-test1.tex,
   eolgrab-test2.tex.

\endpreamble
\let\MetaPrefix\DoubleperCent

\generate{%
  \file{eolgrab.ins}{\from{eolgrab.dtx}{install}}%
  \file{eolgrab.drv}{\from{eolgrab.dtx}{driver}}%
  \usedir{tex/generic/oberdiek}%
  \file{eolgrab.sty}{\from{eolgrab.dtx}{package}}%
  \usedir{doc/latex/oberdiek/example}%
  \file{eolgrab-example-ltx.tex}{\from{eolgrab.dtx}{example-ltx}}%
  \file{eolgrab-example-env.tex}{\from{eolgrab.dtx}{example-env}}%
  \file{eolgrab-example-sec.tex}{\from{eolgrab.dtx}{example-sec}}%
%  \usedir{doc/latex/oberdiek/test}%
%  \file{eolgrab-test1.tex}{\from{eolgrab.dtx}{test1}}%
%  \file{eolgrab-test2.tex}{\from{eolgrab.dtx}{test2}}%
}

\catcode32=13\relax% active space
\let =\space%
\Msg{************************************************************************}
\Msg{*}
\Msg{* To finish the installation you have to move the following}
\Msg{* file into a directory searched by TeX:}
\Msg{*}
\Msg{*     eolgrab.sty}
\Msg{*}
\Msg{* To produce the documentation run the file `eolgrab.drv'}
\Msg{* through LaTeX.}
\Msg{*}
\Msg{* Happy TeXing!}
\Msg{*}
\Msg{************************************************************************}

\endbatchfile
%</install>
%<*ignore>
\fi
%</ignore>
%<*driver>
\NeedsTeXFormat{LaTeX2e}
\ProvidesFile{eolgrab.drv}%
  [2016/05/16 v1.1 Catch arguments delimited by end of line (HO)]%
\documentclass{ltxdoc}
\usepackage{holtxdoc}[2011/11/22]
\begin{document}
  \DocInput{eolgrab.dtx}%
\end{document}
%</driver>
% \fi
%
%
%
% \GetFileInfo{eolgrab.drv}
%
% \title{The \xpackage{eolgrab} package}
% \date{2016/05/16 v1.1}
% \author{Heiko Oberdiek\thanks
% {Please report any issues at \url{https://github.com/ho-tex/oberdiek/issues}}}
%
% \maketitle
%
% \begin{abstract}
% This package implements a generic argument grabber
% to catch an argument that is delimited by the line end.
% \end{abstract}
%
% \tableofcontents
%
% \urldef\urltexhax\url{https://tug.org/mailman/listinfo/texhax}
% \def\eol/{^^A
%   \meta{\texttt{EOL}}^^A
% }
%
% \section{Documentation}
%
% The starting point for this package was a feature
% request of Arno Trautmann in the mailing list
% \texttt{texhax}\footnote{Info page for mailing list
% \texttt{texhax}: \urltexhax} \cite{trautmann-request}.
% A macro \cs{eolsection} should behave like \cs{section},
% but the argument should be delimited by the line end
% instead of given in curly braces:
% \begin{quote}
%   |\eolsection My Title|
% \end{quote}
% Phil Taylor answered this with an implementation for \cs{eolsection}.
% Because this feature could be useful for other macros as well,
% I answered with an implementation of \cs{eolgrab} as general solution
% \cite{oberdiek-eolgrab}.
%
% Both formats \hologo{plainTeX} and \hologo{LaTeX} are supported
% by the package, see the example for \cs{eolsection} below.
%
% \begin{declcs}{eolgrab} \M{code} \meta{argument} \eol/
% \end{declcs}
% Macro \cs{eolgrab} takes two arguments. The first argument
% is \meta{code}, a classical undelimited \hologo{TeX} macro argument.
% The second argument is delimited by the line end \eol/.
% The macro calls \meta{code} with \meta{argument} as argument
% in curly braces. Because the catcode of the line end is changed,
% \cs{eolgrab} will not work in the argument of other macros.
% Macro \cs{eolgrab} is made robust if either \hologo{eTeX}'s
% \cs{protected} or \hologo{LaTeX}'s \cs{DeclareRobustCommand}
% is available.
%
% \begin{declcs}{eolgrabopt} \M{code} \meta{argument} \eol/
% \end{declcs}
% Macro \cs{eolgrabopt} passes \meta{argument} as optional
% argument to \meta{code} if \meta{argument} is not empty.
% \begin{quote}
%   |\eolgrabopt\item foo|
% \end{quote}
% becomes to
% \begin{quote}
%   |\item[{foo}]|
% \end{quote}
% The curly argument braces are added to support square brackets
% inside \meta{argument}.
% If the \meta{argument} is empty:
% \begin{quote}
%   |\eolgrabopt\item|
% \end{quote}
% then
% \begin{quote}
%   |\item|
% \end{quote}
% is called without optional argument.
%
% \subsection{Examples}
%
% \begin{itemize}
% \item
%   The line
%   \begin{quote}
%     |\eolgrab\section My Title|
%   \end{quote}
%   is equivalent to
%   \begin{quote}
%     |\section{My Title}|
%   \end{quote}
% \item
%   The next example uses the star form of \cs{section}. Then the
%   command to be called consists of two tokens. Therefore the
%   first argument of \cs{eolgrab} needs curly braces:
%   \begin{quote}
%     |\eolgrab{\section*}My Title|
%   \end{quote}
%   becomes
%   \begin{quote}
%     |\section*{My Title}|
%   \end{quote}
% \item
%   Now \hologo{LaTeX}'s \cs{PackageError} is used. This macro
%   has three arguments, the package or class name, the message
%   text and the help text. A standard help text of \hologo{LaTeX}
%   is used as given in macro \cs{@ehc}. The second argument,
%   the message text is used as argument, delimited by line end:
%   \begin{quote}
%     |\eolgrab{\PackageError{foobar}}%|\\
%     |Some error message text\MessageBreak%|\\
%     |with several lines|\\
%     |\@ehc|
%   \end{quote}
%   In the first two lines of the example, the line end is suppressed
%   by the comment character (percent), thus the argument is delimited
%   by the line end of the third line. The result is:
%   \begin{quote}
%     |\PackageError{foobar}{Some error message text\MessageBreak|\\
%     |with several lines}\@ehc|
%   \end{quote}
% \item
%   The original request for macro \cs{eolsection}, see above,
%   can be implemented easily with the help of \cs{eolgrab}.
%   Example for \hologo{LaTeX}:
%   \begin{quote}
%     |\usepackage{eolgrab}|\\
%     |\newcommand*{\eolsection}{\eolgrab\section}|
%   \end{quote}
%   Example for \hologo{plainTeX}:
%   \begin{quote}
%     |\input eolgrab.sty\relax|\\
%     |\def\eolsection{\eolgrab\section}|
%   \end{quote}
%   And a sophisticated variant for \hologo{LaTeX} that also supports
%   the star syntax and the optional argument:
%    \begin{macrocode}
%<*example-sec>
\documentclass{article}
\usepackage{eolgrab}
\makeatletter
\newcommand*{\eolsection}{%
  \@ifstar{%
    \eolgrab{\section*}%
  }{%
    \@ifnextchar[{%
      \eoloptsection
    }{%
      \eolgrab\section
    }%
  }%
}
\newcommand*{\eoloptsection}[1][]{%
  \eolgrab{\section[{#1}]}%
}
\makeatother
\begin{document}
\tableofcontents
\eolsection Section without star and optional argument
\eolsection*Section with star
\eolsection[Short section title]Long section title
\end{document}
%</example-sec>
%    \end{macrocode}
% \end{itemize}
%
% \subsubsection{Small \hologo{LaTeX} document as example}
%
%    \begin{macrocode}
%<*example-ltx>
\RequirePackage{eolgrab}
\eolgrab\documentclass article
\eolgrab\begin document
\eolgrab\section Hello World
\eolgrab\emph Some text
\eolgrab\end document
%</example-ltx>
%    \end{macrocode}
%
% \subsubsection{\hologo{LaTeX} document with environments}
%
%    \begin{macrocode}
%<*example-env>
\documentclass{article}
\usepackage{eolgrab}
\newcommand*{\Begin}{\eolgrab   \begin}
\newcommand*{\End  }{\eolgrab   \end  }
\newcommand*{\Item }{\eolgrabopt\item }
\Begin document
  \Begin itemize
    \Item
      first item
    \Item
      second item
  \End itemize
  \Begin description
    \Item foo
      is the first syllable of foobar.
    \Item bar
      is the second syllable of foobar.
  \End description
\End document
%</example-env>
%    \end{macrocode}
%
% \subsection{Limitations}
%
% Macro \cs{eolgrab} needs to catch the line end.
% If \hologo{TeX} reads a line, then it throws away the line end
% characters (carriage return, line feed) and removes spaces at the
% end of the line. Then it adds the character with the character code
% that is given by \cs{endlinechar} at the end of the line. The category
% code of the inserted character is given by the current value of
% its \cs{catcode}. If \cs{endlinechar} is not a valid character code
% (especially if it is negative), then no character is added.
%
% In \hologo{plainTeX} and \hologo{LaTeX} the standard settings
% of the inserted endline character is the character with code 13
% (or |^||^||M| in \hologo{TeX} notation) with catcode 5 (end of line).
% That means the inserted end of line character behaves like a space
% token. For example, it is removed after macro names. Therefore
% \cs{eolgrab} changes the catcode.
%
% Therefore \cs{eolgrab} has some limitations:
% \begin{itemize}
% \item
%   Like other verbatim stuff, the macro \cs{eolgrab} cannot be
%   used in the argument of other macros. \cs{eolgrab} want to
%   change the catcode of the end of line character. If this
%   character is read before, because it is processed as argument
%   of another macro, the catcode is already set and is not reassigned
%   later if \cs{eolgrab} changes the category code for this character
%   code.
% \item
%   The argument must not contain the end of line character.
%   Otherwise the first end of line character is already taken
%   as delimiter, leaving the rest of the line outside the argument.
% \item
%   Because \cs{eolgrab} is probably mostly used in the line with
%   the delimited argument. Therefore changes of \cs{endlinechar}
%   will not affect the current line.
% \end{itemize}
%
% \StopEventually{
% }
%
% \section{Implementation}
%
%    \begin{macrocode}
%<*package>
%    \end{macrocode}
%
% \subsection{Reload check and package identification}
%    Reload check, especially if the package is not used with \LaTeX.
%    \begin{macrocode}
\begingroup\catcode61\catcode48\catcode32=10\relax%
  \catcode13=5 % ^^M
  \endlinechar=13 %
  \catcode35=6 % #
  \catcode39=12 % '
  \catcode44=12 % ,
  \catcode45=12 % -
  \catcode46=12 % .
  \catcode58=12 % :
  \catcode64=11 % @
  \catcode123=1 % {
  \catcode125=2 % }
  \expandafter\let\expandafter\x\csname ver@eolgrab.sty\endcsname
  \ifx\x\relax % plain-TeX, first loading
  \else
    \def\empty{}%
    \ifx\x\empty % LaTeX, first loading,
      % variable is initialized, but \ProvidesPackage not yet seen
    \else
      \expandafter\ifx\csname PackageInfo\endcsname\relax
        \def\x#1#2{%
          \immediate\write-1{Package #1 Info: #2.}%
        }%
      \else
        \def\x#1#2{\PackageInfo{#1}{#2, stopped}}%
      \fi
      \x{eolgrab}{The package is already loaded}%
      \aftergroup\endinput
    \fi
  \fi
\endgroup%
%    \end{macrocode}
%    Package identification:
%    \begin{macrocode}
\begingroup\catcode61\catcode48\catcode32=10\relax%
  \catcode13=5 % ^^M
  \endlinechar=13 %
  \catcode35=6 % #
  \catcode39=12 % '
  \catcode40=12 % (
  \catcode41=12 % )
  \catcode44=12 % ,
  \catcode45=12 % -
  \catcode46=12 % .
  \catcode47=12 % /
  \catcode58=12 % :
  \catcode64=11 % @
  \catcode91=12 % [
  \catcode93=12 % ]
  \catcode123=1 % {
  \catcode125=2 % }
  \expandafter\ifx\csname ProvidesPackage\endcsname\relax
    \def\x#1#2#3[#4]{\endgroup
      \immediate\write-1{Package: #3 #4}%
      \xdef#1{#4}%
    }%
  \else
    \def\x#1#2[#3]{\endgroup
      #2[{#3}]%
      \ifx#1\@undefined
        \xdef#1{#3}%
      \fi
      \ifx#1\relax
        \xdef#1{#3}%
      \fi
    }%
  \fi
\expandafter\x\csname ver@eolgrab.sty\endcsname
\ProvidesPackage{eolgrab}%
  [2016/05/16 v1.1 Catch arguments delimited by end of line (HO)]%
%    \end{macrocode}
%
% \subsection{Catcodes}
%
%    \begin{macrocode}
\begingroup\catcode61\catcode48\catcode32=10\relax%
  \catcode13=5 % ^^M
  \endlinechar=13 %
  \catcode123=1 % {
  \catcode125=2 % }
  \catcode64=11 % @
  \def\x{\endgroup
    \expandafter\edef\csname eolgrab@AtEnd\endcsname{%
      \endlinechar=\the\endlinechar\relax
      \catcode13=\the\catcode13\relax
      \catcode32=\the\catcode32\relax
      \catcode35=\the\catcode35\relax
      \catcode61=\the\catcode61\relax
      \catcode64=\the\catcode64\relax
      \catcode123=\the\catcode123\relax
      \catcode125=\the\catcode125\relax
    }%
  }%
\x\catcode61\catcode48\catcode32=10\relax%
\catcode13=5 % ^^M
\endlinechar=13 %
\catcode35=6 % #
\catcode64=11 % @
\catcode123=1 % {
\catcode125=2 % }
\def\TMP@EnsureCode#1#2{%
  \edef\eolgrab@AtEnd{%
    \eolgrab@AtEnd
    \catcode#1=\the\catcode#1\relax
  }%
  \catcode#1=#2\relax
}
\TMP@EnsureCode{40}{12}% (
\TMP@EnsureCode{41}{12}% )
\TMP@EnsureCode{42}{12}% *
\TMP@EnsureCode{46}{12}% .
\TMP@EnsureCode{47}{12}% /
\TMP@EnsureCode{91}{12}% [
\TMP@EnsureCode{93}{12}% ]
\TMP@EnsureCode{94}{7}% ^
\edef\eolgrab@AtEnd{\eolgrab@AtEnd\noexpand\endinput}
%    \end{macrocode}
%
% \subsection{Resources}
%
%    \begin{macrocode}
\begingroup\expandafter\expandafter\expandafter\endgroup
\expandafter\ifx\csname RequirePackage\endcsname\relax
  \input ltxcmds.sty\relax
  \input infwarerr.sty\relax
\else
  \RequirePackage{ltxcmds}[2010/12/04]%
  \RequirePackage{infwarerr}[2010/04/08]%
\fi
%    \end{macrocode}
%
%    \begin{macro}{\eolgrab@ifdefinable}
%    \begin{macrocode}
\ltx@IfUndefined{@ifdefinable}{%
  \def\eolgrab@ifdefinable#1#2{%
    \ltx@ifundefined{#1}{#2}{%
      \@PackageError{eolgrab}{%
        Command \ltx@backslashchar#1 already defined%
      }\@ehc
    }%
  }%
}{%
  \def\eolgrab@ifdefinable#1{%
    \expandafter\@ifdefinable\csname#1\endcsname
  }%
}
%    \end{macrocode}
%    \end{macro}
%
% \subsection{Macro \cs{eolgrab}}
%
%    \begin{macro}{\eolgrab}
%    \begin{macrocode}
\eolgrab@ifdefinable{eolgrab}{%
  \ltx@IfUndefined{protected}{%
    \ltx@IfUndefined{DeclareRobustCommand}{%
      \def\eolgrab#1%
    }{%
      \newcommand\eolgrab{}%
      \DeclareRobustCommand*\eolgrab
    }%
  }{%
    \protected\def\eolgrab#1%
  }{%
    \begingroup
    \endlinechar=13 %
    \catcode13=\ltx@active
    \eolgrab@{#1}%
  }%
}
%    \end{macrocode}
%    \end{macro}
%
%    \begin{macro}{\eolgrabopt}
%    \begin{macrocode}
\eolgrab@ifdefinable{eolgrabopt}{%
  \ltx@IfUndefined{protected}{%
    \ltx@IfUndefined{DeclareRobustCommand}{%
      \def\eolgrabopt#1%
    }{%
      \newcommand\eolgrabopt{}%
      \DeclareRobustCommand*\eolgrabopt
    }%
  }{%
    \protected\def\eolgrabopt#1%
  }{%
    \begingroup
    \endlinechar=13 %
    \catcode13=\ltx@active
    \eolgrab@opt{#1}%
  }%
}
%    \end{macrocode}
%    \end{macro}
%
%    \begin{macrocode}
\begingroup
  \catcode13=\ltx@active %
\ltx@firstofone{\endgroup %
%    \end{macrocode}
%    \begin{macro}{\eolgrab@}
%    \begin{macrocode}
  \def\eolgrab@#1#2^^M{%
    \endgroup %
    #1{#2}%
  }%
%    \end{macrocode}
%    \end{macro}
%    \begin{macro}{\eolgrab@opt}
%    \begin{macrocode}
  \def\eolgrab@opt#1#2^^M{%
    \endgroup %
    \ltx@ifempty{#2}{%
      #1%
    }{%
      #1[{#2}]%
    }%
  }%
%    \end{macrocode}
%    \end{macro}
%    \begin{macrocode}
}%
%    \end{macrocode}
%
%    \begin{macrocode}
\eolgrab@AtEnd%
%    \end{macrocode}
%    \begin{macrocode}
%</package>
%    \end{macrocode}
%% \section{Installation}
%
% \subsection{Download}
%
% \paragraph{Package.} This package is available on
% CTAN\footnote{\CTANpkg{eolgrab}}:
% \begin{description}
% \item[\CTAN{macros/latex/contrib/oberdiek/eolgrab.dtx}] The source file.
% \item[\CTAN{macros/latex/contrib/oberdiek/eolgrab.pdf}] Documentation.
% \end{description}
%
%
% \paragraph{Bundle.} All the packages of the bundle `oberdiek'
% are also available in a TDS compliant ZIP archive. There
% the packages are already unpacked and the documentation files
% are generated. The files and directories obey the TDS standard.
% \begin{description}
% \item[\CTANinstall{install/macros/latex/contrib/oberdiek.tds.zip}]
% \end{description}
% \emph{TDS} refers to the standard ``A Directory Structure
% for \TeX\ Files'' (\CTANpkg{tds}). Directories
% with \xfile{texmf} in their name are usually organized this way.
%
% \subsection{Bundle installation}
%
% \paragraph{Unpacking.} Unpack the \xfile{oberdiek.tds.zip} in the
% TDS tree (also known as \xfile{texmf} tree) of your choice.
% Example (linux):
% \begin{quote}
%   |unzip oberdiek.tds.zip -d ~/texmf|
% \end{quote}
%
% \subsection{Package installation}
%
% \paragraph{Unpacking.} The \xfile{.dtx} file is a self-extracting
% \docstrip\ archive. The files are extracted by running the
% \xfile{.dtx} through \plainTeX:
% \begin{quote}
%   \verb|tex eolgrab.dtx|
% \end{quote}
%
% \paragraph{TDS.} Now the different files must be moved into
% the different directories in your installation TDS tree
% (also known as \xfile{texmf} tree):
% \begin{quote}
% \def\t{^^A
% \begin{tabular}{@{}>{\ttfamily}l@{ $\rightarrow$ }>{\ttfamily}l@{}}
%   eolgrab.sty & tex/generic/oberdiek/eolgrab.sty\\
%   eolgrab.pdf & doc/latex/oberdiek/eolgrab.pdf\\
%   example/eolgrab-example-ltx.tex & doc/latex/oberdiek/example/eolgrab-example-ltx.tex\\
%   example/eolgrab-example-env.tex & doc/latex/oberdiek/example/eolgrab-example-env.tex\\
%   example/eolgrab-example-sec.tex & doc/latex/oberdiek/example/eolgrab-example-sec.tex\\
%   eolgrab.dtx & source/latex/oberdiek/eolgrab.dtx\\
% \end{tabular}^^A
% }^^A
% \sbox0{\t}^^A
% \ifdim\wd0>\linewidth
%   \begingroup
%     \advance\linewidth by\leftmargin
%     \advance\linewidth by\rightmargin
%   \edef\x{\endgroup
%     \def\noexpand\lw{\the\linewidth}^^A
%   }\x
%   \def\lwbox{^^A
%     \leavevmode
%     \hbox to \linewidth{^^A
%       \kern-\leftmargin\relax
%       \hss
%       \usebox0
%       \hss
%       \kern-\rightmargin\relax
%     }^^A
%   }^^A
%   \ifdim\wd0>\lw
%     \sbox0{\small\t}^^A
%     \ifdim\wd0>\linewidth
%       \ifdim\wd0>\lw
%         \sbox0{\footnotesize\t}^^A
%         \ifdim\wd0>\linewidth
%           \ifdim\wd0>\lw
%             \sbox0{\scriptsize\t}^^A
%             \ifdim\wd0>\linewidth
%               \ifdim\wd0>\lw
%                 \sbox0{\tiny\t}^^A
%                 \ifdim\wd0>\linewidth
%                   \lwbox
%                 \else
%                   \usebox0
%                 \fi
%               \else
%                 \lwbox
%               \fi
%             \else
%               \usebox0
%             \fi
%           \else
%             \lwbox
%           \fi
%         \else
%           \usebox0
%         \fi
%       \else
%         \lwbox
%       \fi
%     \else
%       \usebox0
%     \fi
%   \else
%     \lwbox
%   \fi
% \else
%   \usebox0
% \fi
% \end{quote}
% If you have a \xfile{docstrip.cfg} that configures and enables \docstrip's
% TDS installing feature, then some files can already be in the right
% place, see the documentation of \docstrip.
%
% \subsection{Refresh file name databases}
%
% If your \TeX~distribution
% (\TeX\,Live, \mikTeX, \dots) relies on file name databases, you must refresh
% these. For example, \TeX\,Live\ users run \verb|texhash| or
% \verb|mktexlsr|.
%
% \subsection{Some details for the interested}
%
% \paragraph{Unpacking with \LaTeX.}
% The \xfile{.dtx} chooses its action depending on the format:
% \begin{description}
% \item[\plainTeX:] Run \docstrip\ and extract the files.
% \item[\LaTeX:] Generate the documentation.
% \end{description}
% If you insist on using \LaTeX\ for \docstrip\ (really,
% \docstrip\ does not need \LaTeX), then inform the autodetect routine
% about your intention:
% \begin{quote}
%   \verb|latex \let\install=y% \iffalse meta-comment
%
% File: eolgrab.dtx
% Version: 2016/05/16 v1.1
% Info: Catch arguments delimited by end of line
%
% Copyright (C)
%    2011 Heiko Oberdiek
%    2016-2019 Oberdiek Package Support Group
%    https://github.com/ho-tex/oberdiek/issues
%
% This work may be distributed and/or modified under the
% conditions of the LaTeX Project Public License, either
% version 1.3c of this license or (at your option) any later
% version. This version of this license is in
%    https://www.latex-project.org/lppl/lppl-1-3c.txt
% and the latest version of this license is in
%    https://www.latex-project.org/lppl.txt
% and version 1.3 or later is part of all distributions of
% LaTeX version 2005/12/01 or later.
%
% This work has the LPPL maintenance status "maintained".
%
% The Current Maintainers of this work are
% Heiko Oberdiek and the Oberdiek Package Support Group
% https://github.com/ho-tex/oberdiek/issues
%
% The Base Interpreter refers to any `TeX-Format',
% because some files are installed in TDS:tex/generic//.
%
% This work consists of the main source file eolgrab.dtx
% and the derived files
%    eolgrab.sty, eolgrab.pdf, eolgrab.ins, eolgrab.drv,
%    eolgrab-example-ltx.tex, eolgrab-example-env.tex,
%    eolgrab-example-sec.tex, eolgrab-test1.tex,
%    eolgrab-test2.tex.
%
% Distribution:
%    CTAN:macros/latex/contrib/oberdiek/eolgrab.dtx
%    CTAN:macros/latex/contrib/oberdiek/eolgrab.pdf
%
% Unpacking:
%    (a) If eolgrab.ins is present:
%           tex eolgrab.ins
%    (b) Without eolgrab.ins:
%           tex eolgrab.dtx
%    (c) If you insist on using LaTeX
%           latex \let\install=y% \iffalse meta-comment
%
% File: eolgrab.dtx
% Version: 2016/05/16 v1.1
% Info: Catch arguments delimited by end of line
%
% Copyright (C)
%    2011 Heiko Oberdiek
%    2016-2019 Oberdiek Package Support Group
%    https://github.com/ho-tex/oberdiek/issues
%
% This work may be distributed and/or modified under the
% conditions of the LaTeX Project Public License, either
% version 1.3c of this license or (at your option) any later
% version. This version of this license is in
%    https://www.latex-project.org/lppl/lppl-1-3c.txt
% and the latest version of this license is in
%    https://www.latex-project.org/lppl.txt
% and version 1.3 or later is part of all distributions of
% LaTeX version 2005/12/01 or later.
%
% This work has the LPPL maintenance status "maintained".
%
% The Current Maintainers of this work are
% Heiko Oberdiek and the Oberdiek Package Support Group
% https://github.com/ho-tex/oberdiek/issues
%
% The Base Interpreter refers to any `TeX-Format',
% because some files are installed in TDS:tex/generic//.
%
% This work consists of the main source file eolgrab.dtx
% and the derived files
%    eolgrab.sty, eolgrab.pdf, eolgrab.ins, eolgrab.drv,
%    eolgrab-example-ltx.tex, eolgrab-example-env.tex,
%    eolgrab-example-sec.tex, eolgrab-test1.tex,
%    eolgrab-test2.tex.
%
% Distribution:
%    CTAN:macros/latex/contrib/oberdiek/eolgrab.dtx
%    CTAN:macros/latex/contrib/oberdiek/eolgrab.pdf
%
% Unpacking:
%    (a) If eolgrab.ins is present:
%           tex eolgrab.ins
%    (b) Without eolgrab.ins:
%           tex eolgrab.dtx
%    (c) If you insist on using LaTeX
%           latex \let\install=y\input{eolgrab.dtx}
%        (quote the arguments according to the demands of your shell)
%
% Documentation:
%    (a) If eolgrab.drv is present:
%           latex eolgrab.drv
%    (b) Without eolgrab.drv:
%           latex eolgrab.dtx; ...
%    The class ltxdoc loads the configuration file ltxdoc.cfg
%    if available. Here you can specify further options, e.g.
%    use A4 as paper format:
%       \PassOptionsToClass{a4paper}{article}
%
%    Programm calls to get the documentation (example):
%       pdflatex eolgrab.dtx
%       makeindex -s gind.ist eolgrab.idx
%       pdflatex eolgrab.dtx
%       makeindex -s gind.ist eolgrab.idx
%       pdflatex eolgrab.dtx
%
% Installation:
%    TDS:tex/generic/oberdiek/eolgrab.sty
%    TDS:doc/latex/oberdiek/eolgrab.pdf
%    TDS:doc/latex/oberdiek/example/eolgrab-example-ltx.tex
%    TDS:doc/latex/oberdiek/example/eolgrab-example-env.tex
%    TDS:doc/latex/oberdiek/example/eolgrab-example-sec.tex
%    TDS:source/latex/oberdiek/eolgrab.dtx
%
%<*ignore>
\begingroup
  \catcode123=1 %
  \catcode125=2 %
  \def\x{LaTeX2e}%
\expandafter\endgroup
\ifcase 0\ifx\install y1\fi\expandafter
         \ifx\csname processbatchFile\endcsname\relax\else1\fi
         \ifx\fmtname\x\else 1\fi\relax
\else\csname fi\endcsname
%</ignore>
%<*install>
\input docstrip.tex
\Msg{************************************************************************}
\Msg{* Installation}
\Msg{* Package: eolgrab 2016/05/16 v1.1 Catch arguments delimited by end of line (HO)}
\Msg{************************************************************************}

\keepsilent
\askforoverwritefalse

\let\MetaPrefix\relax
\preamble

This is a generated file.

Project: eolgrab
Version: 2016/05/16 v1.1

Copyright (C)
   2011 Heiko Oberdiek
   2016-2019 Oberdiek Package Support Group

This work may be distributed and/or modified under the
conditions of the LaTeX Project Public License, either
version 1.3c of this license or (at your option) any later
version. This version of this license is in
   https://www.latex-project.org/lppl/lppl-1-3c.txt
and the latest version of this license is in
   https://www.latex-project.org/lppl.txt
and version 1.3 or later is part of all distributions of
LaTeX version 2005/12/01 or later.

This work has the LPPL maintenance status "maintained".

The Current Maintainers of this work are
Heiko Oberdiek and the Oberdiek Package Support Group
https://github.com/ho-tex/oberdiek/issues


The Base Interpreter refers to any `TeX-Format',
because some files are installed in TDS:tex/generic//.

This work consists of the main source file eolgrab.dtx
and the derived files
   eolgrab.sty, eolgrab.pdf, eolgrab.ins, eolgrab.drv,
   eolgrab-example-ltx.tex, eolgrab-example-env.tex,
   eolgrab-example-sec.tex, eolgrab-test1.tex,
   eolgrab-test2.tex.

\endpreamble
\let\MetaPrefix\DoubleperCent

\generate{%
  \file{eolgrab.ins}{\from{eolgrab.dtx}{install}}%
  \file{eolgrab.drv}{\from{eolgrab.dtx}{driver}}%
  \usedir{tex/generic/oberdiek}%
  \file{eolgrab.sty}{\from{eolgrab.dtx}{package}}%
  \usedir{doc/latex/oberdiek/example}%
  \file{eolgrab-example-ltx.tex}{\from{eolgrab.dtx}{example-ltx}}%
  \file{eolgrab-example-env.tex}{\from{eolgrab.dtx}{example-env}}%
  \file{eolgrab-example-sec.tex}{\from{eolgrab.dtx}{example-sec}}%
%  \usedir{doc/latex/oberdiek/test}%
%  \file{eolgrab-test1.tex}{\from{eolgrab.dtx}{test1}}%
%  \file{eolgrab-test2.tex}{\from{eolgrab.dtx}{test2}}%
}

\catcode32=13\relax% active space
\let =\space%
\Msg{************************************************************************}
\Msg{*}
\Msg{* To finish the installation you have to move the following}
\Msg{* file into a directory searched by TeX:}
\Msg{*}
\Msg{*     eolgrab.sty}
\Msg{*}
\Msg{* To produce the documentation run the file `eolgrab.drv'}
\Msg{* through LaTeX.}
\Msg{*}
\Msg{* Happy TeXing!}
\Msg{*}
\Msg{************************************************************************}

\endbatchfile
%</install>
%<*ignore>
\fi
%</ignore>
%<*driver>
\NeedsTeXFormat{LaTeX2e}
\ProvidesFile{eolgrab.drv}%
  [2016/05/16 v1.1 Catch arguments delimited by end of line (HO)]%
\documentclass{ltxdoc}
\usepackage{holtxdoc}[2011/11/22]
\begin{document}
  \DocInput{eolgrab.dtx}%
\end{document}
%</driver>
% \fi
%
%
%
% \GetFileInfo{eolgrab.drv}
%
% \title{The \xpackage{eolgrab} package}
% \date{2016/05/16 v1.1}
% \author{Heiko Oberdiek\thanks
% {Please report any issues at \url{https://github.com/ho-tex/oberdiek/issues}}}
%
% \maketitle
%
% \begin{abstract}
% This package implements a generic argument grabber
% to catch an argument that is delimited by the line end.
% \end{abstract}
%
% \tableofcontents
%
% \urldef\urltexhax\url{https://tug.org/mailman/listinfo/texhax}
% \def\eol/{^^A
%   \meta{\texttt{EOL}}^^A
% }
%
% \section{Documentation}
%
% The starting point for this package was a feature
% request of Arno Trautmann in the mailing list
% \texttt{texhax}\footnote{Info page for mailing list
% \texttt{texhax}: \urltexhax} \cite{trautmann-request}.
% A macro \cs{eolsection} should behave like \cs{section},
% but the argument should be delimited by the line end
% instead of given in curly braces:
% \begin{quote}
%   |\eolsection My Title|
% \end{quote}
% Phil Taylor answered this with an implementation for \cs{eolsection}.
% Because this feature could be useful for other macros as well,
% I answered with an implementation of \cs{eolgrab} as general solution
% \cite{oberdiek-eolgrab}.
%
% Both formats \hologo{plainTeX} and \hologo{LaTeX} are supported
% by the package, see the example for \cs{eolsection} below.
%
% \begin{declcs}{eolgrab} \M{code} \meta{argument} \eol/
% \end{declcs}
% Macro \cs{eolgrab} takes two arguments. The first argument
% is \meta{code}, a classical undelimited \hologo{TeX} macro argument.
% The second argument is delimited by the line end \eol/.
% The macro calls \meta{code} with \meta{argument} as argument
% in curly braces. Because the catcode of the line end is changed,
% \cs{eolgrab} will not work in the argument of other macros.
% Macro \cs{eolgrab} is made robust if either \hologo{eTeX}'s
% \cs{protected} or \hologo{LaTeX}'s \cs{DeclareRobustCommand}
% is available.
%
% \begin{declcs}{eolgrabopt} \M{code} \meta{argument} \eol/
% \end{declcs}
% Macro \cs{eolgrabopt} passes \meta{argument} as optional
% argument to \meta{code} if \meta{argument} is not empty.
% \begin{quote}
%   |\eolgrabopt\item foo|
% \end{quote}
% becomes to
% \begin{quote}
%   |\item[{foo}]|
% \end{quote}
% The curly argument braces are added to support square brackets
% inside \meta{argument}.
% If the \meta{argument} is empty:
% \begin{quote}
%   |\eolgrabopt\item|
% \end{quote}
% then
% \begin{quote}
%   |\item|
% \end{quote}
% is called without optional argument.
%
% \subsection{Examples}
%
% \begin{itemize}
% \item
%   The line
%   \begin{quote}
%     |\eolgrab\section My Title|
%   \end{quote}
%   is equivalent to
%   \begin{quote}
%     |\section{My Title}|
%   \end{quote}
% \item
%   The next example uses the star form of \cs{section}. Then the
%   command to be called consists of two tokens. Therefore the
%   first argument of \cs{eolgrab} needs curly braces:
%   \begin{quote}
%     |\eolgrab{\section*}My Title|
%   \end{quote}
%   becomes
%   \begin{quote}
%     |\section*{My Title}|
%   \end{quote}
% \item
%   Now \hologo{LaTeX}'s \cs{PackageError} is used. This macro
%   has three arguments, the package or class name, the message
%   text and the help text. A standard help text of \hologo{LaTeX}
%   is used as given in macro \cs{@ehc}. The second argument,
%   the message text is used as argument, delimited by line end:
%   \begin{quote}
%     |\eolgrab{\PackageError{foobar}}%|\\
%     |Some error message text\MessageBreak%|\\
%     |with several lines|\\
%     |\@ehc|
%   \end{quote}
%   In the first two lines of the example, the line end is suppressed
%   by the comment character (percent), thus the argument is delimited
%   by the line end of the third line. The result is:
%   \begin{quote}
%     |\PackageError{foobar}{Some error message text\MessageBreak|\\
%     |with several lines}\@ehc|
%   \end{quote}
% \item
%   The original request for macro \cs{eolsection}, see above,
%   can be implemented easily with the help of \cs{eolgrab}.
%   Example for \hologo{LaTeX}:
%   \begin{quote}
%     |\usepackage{eolgrab}|\\
%     |\newcommand*{\eolsection}{\eolgrab\section}|
%   \end{quote}
%   Example for \hologo{plainTeX}:
%   \begin{quote}
%     |\input eolgrab.sty\relax|\\
%     |\def\eolsection{\eolgrab\section}|
%   \end{quote}
%   And a sophisticated variant for \hologo{LaTeX} that also supports
%   the star syntax and the optional argument:
%    \begin{macrocode}
%<*example-sec>
\documentclass{article}
\usepackage{eolgrab}
\makeatletter
\newcommand*{\eolsection}{%
  \@ifstar{%
    \eolgrab{\section*}%
  }{%
    \@ifnextchar[{%
      \eoloptsection
    }{%
      \eolgrab\section
    }%
  }%
}
\newcommand*{\eoloptsection}[1][]{%
  \eolgrab{\section[{#1}]}%
}
\makeatother
\begin{document}
\tableofcontents
\eolsection Section without star and optional argument
\eolsection*Section with star
\eolsection[Short section title]Long section title
\end{document}
%</example-sec>
%    \end{macrocode}
% \end{itemize}
%
% \subsubsection{Small \hologo{LaTeX} document as example}
%
%    \begin{macrocode}
%<*example-ltx>
\RequirePackage{eolgrab}
\eolgrab\documentclass article
\eolgrab\begin document
\eolgrab\section Hello World
\eolgrab\emph Some text
\eolgrab\end document
%</example-ltx>
%    \end{macrocode}
%
% \subsubsection{\hologo{LaTeX} document with environments}
%
%    \begin{macrocode}
%<*example-env>
\documentclass{article}
\usepackage{eolgrab}
\newcommand*{\Begin}{\eolgrab   \begin}
\newcommand*{\End  }{\eolgrab   \end  }
\newcommand*{\Item }{\eolgrabopt\item }
\Begin document
  \Begin itemize
    \Item
      first item
    \Item
      second item
  \End itemize
  \Begin description
    \Item foo
      is the first syllable of foobar.
    \Item bar
      is the second syllable of foobar.
  \End description
\End document
%</example-env>
%    \end{macrocode}
%
% \subsection{Limitations}
%
% Macro \cs{eolgrab} needs to catch the line end.
% If \hologo{TeX} reads a line, then it throws away the line end
% characters (carriage return, line feed) and removes spaces at the
% end of the line. Then it adds the character with the character code
% that is given by \cs{endlinechar} at the end of the line. The category
% code of the inserted character is given by the current value of
% its \cs{catcode}. If \cs{endlinechar} is not a valid character code
% (especially if it is negative), then no character is added.
%
% In \hologo{plainTeX} and \hologo{LaTeX} the standard settings
% of the inserted endline character is the character with code 13
% (or |^||^||M| in \hologo{TeX} notation) with catcode 5 (end of line).
% That means the inserted end of line character behaves like a space
% token. For example, it is removed after macro names. Therefore
% \cs{eolgrab} changes the catcode.
%
% Therefore \cs{eolgrab} has some limitations:
% \begin{itemize}
% \item
%   Like other verbatim stuff, the macro \cs{eolgrab} cannot be
%   used in the argument of other macros. \cs{eolgrab} want to
%   change the catcode of the end of line character. If this
%   character is read before, because it is processed as argument
%   of another macro, the catcode is already set and is not reassigned
%   later if \cs{eolgrab} changes the category code for this character
%   code.
% \item
%   The argument must not contain the end of line character.
%   Otherwise the first end of line character is already taken
%   as delimiter, leaving the rest of the line outside the argument.
% \item
%   Because \cs{eolgrab} is probably mostly used in the line with
%   the delimited argument. Therefore changes of \cs{endlinechar}
%   will not affect the current line.
% \end{itemize}
%
% \StopEventually{
% }
%
% \section{Implementation}
%
%    \begin{macrocode}
%<*package>
%    \end{macrocode}
%
% \subsection{Reload check and package identification}
%    Reload check, especially if the package is not used with \LaTeX.
%    \begin{macrocode}
\begingroup\catcode61\catcode48\catcode32=10\relax%
  \catcode13=5 % ^^M
  \endlinechar=13 %
  \catcode35=6 % #
  \catcode39=12 % '
  \catcode44=12 % ,
  \catcode45=12 % -
  \catcode46=12 % .
  \catcode58=12 % :
  \catcode64=11 % @
  \catcode123=1 % {
  \catcode125=2 % }
  \expandafter\let\expandafter\x\csname ver@eolgrab.sty\endcsname
  \ifx\x\relax % plain-TeX, first loading
  \else
    \def\empty{}%
    \ifx\x\empty % LaTeX, first loading,
      % variable is initialized, but \ProvidesPackage not yet seen
    \else
      \expandafter\ifx\csname PackageInfo\endcsname\relax
        \def\x#1#2{%
          \immediate\write-1{Package #1 Info: #2.}%
        }%
      \else
        \def\x#1#2{\PackageInfo{#1}{#2, stopped}}%
      \fi
      \x{eolgrab}{The package is already loaded}%
      \aftergroup\endinput
    \fi
  \fi
\endgroup%
%    \end{macrocode}
%    Package identification:
%    \begin{macrocode}
\begingroup\catcode61\catcode48\catcode32=10\relax%
  \catcode13=5 % ^^M
  \endlinechar=13 %
  \catcode35=6 % #
  \catcode39=12 % '
  \catcode40=12 % (
  \catcode41=12 % )
  \catcode44=12 % ,
  \catcode45=12 % -
  \catcode46=12 % .
  \catcode47=12 % /
  \catcode58=12 % :
  \catcode64=11 % @
  \catcode91=12 % [
  \catcode93=12 % ]
  \catcode123=1 % {
  \catcode125=2 % }
  \expandafter\ifx\csname ProvidesPackage\endcsname\relax
    \def\x#1#2#3[#4]{\endgroup
      \immediate\write-1{Package: #3 #4}%
      \xdef#1{#4}%
    }%
  \else
    \def\x#1#2[#3]{\endgroup
      #2[{#3}]%
      \ifx#1\@undefined
        \xdef#1{#3}%
      \fi
      \ifx#1\relax
        \xdef#1{#3}%
      \fi
    }%
  \fi
\expandafter\x\csname ver@eolgrab.sty\endcsname
\ProvidesPackage{eolgrab}%
  [2016/05/16 v1.1 Catch arguments delimited by end of line (HO)]%
%    \end{macrocode}
%
% \subsection{Catcodes}
%
%    \begin{macrocode}
\begingroup\catcode61\catcode48\catcode32=10\relax%
  \catcode13=5 % ^^M
  \endlinechar=13 %
  \catcode123=1 % {
  \catcode125=2 % }
  \catcode64=11 % @
  \def\x{\endgroup
    \expandafter\edef\csname eolgrab@AtEnd\endcsname{%
      \endlinechar=\the\endlinechar\relax
      \catcode13=\the\catcode13\relax
      \catcode32=\the\catcode32\relax
      \catcode35=\the\catcode35\relax
      \catcode61=\the\catcode61\relax
      \catcode64=\the\catcode64\relax
      \catcode123=\the\catcode123\relax
      \catcode125=\the\catcode125\relax
    }%
  }%
\x\catcode61\catcode48\catcode32=10\relax%
\catcode13=5 % ^^M
\endlinechar=13 %
\catcode35=6 % #
\catcode64=11 % @
\catcode123=1 % {
\catcode125=2 % }
\def\TMP@EnsureCode#1#2{%
  \edef\eolgrab@AtEnd{%
    \eolgrab@AtEnd
    \catcode#1=\the\catcode#1\relax
  }%
  \catcode#1=#2\relax
}
\TMP@EnsureCode{40}{12}% (
\TMP@EnsureCode{41}{12}% )
\TMP@EnsureCode{42}{12}% *
\TMP@EnsureCode{46}{12}% .
\TMP@EnsureCode{47}{12}% /
\TMP@EnsureCode{91}{12}% [
\TMP@EnsureCode{93}{12}% ]
\TMP@EnsureCode{94}{7}% ^
\edef\eolgrab@AtEnd{\eolgrab@AtEnd\noexpand\endinput}
%    \end{macrocode}
%
% \subsection{Resources}
%
%    \begin{macrocode}
\begingroup\expandafter\expandafter\expandafter\endgroup
\expandafter\ifx\csname RequirePackage\endcsname\relax
  \input ltxcmds.sty\relax
  \input infwarerr.sty\relax
\else
  \RequirePackage{ltxcmds}[2010/12/04]%
  \RequirePackage{infwarerr}[2010/04/08]%
\fi
%    \end{macrocode}
%
%    \begin{macro}{\eolgrab@ifdefinable}
%    \begin{macrocode}
\ltx@IfUndefined{@ifdefinable}{%
  \def\eolgrab@ifdefinable#1#2{%
    \ltx@ifundefined{#1}{#2}{%
      \@PackageError{eolgrab}{%
        Command \ltx@backslashchar#1 already defined%
      }\@ehc
    }%
  }%
}{%
  \def\eolgrab@ifdefinable#1{%
    \expandafter\@ifdefinable\csname#1\endcsname
  }%
}
%    \end{macrocode}
%    \end{macro}
%
% \subsection{Macro \cs{eolgrab}}
%
%    \begin{macro}{\eolgrab}
%    \begin{macrocode}
\eolgrab@ifdefinable{eolgrab}{%
  \ltx@IfUndefined{protected}{%
    \ltx@IfUndefined{DeclareRobustCommand}{%
      \def\eolgrab#1%
    }{%
      \newcommand\eolgrab{}%
      \DeclareRobustCommand*\eolgrab
    }%
  }{%
    \protected\def\eolgrab#1%
  }{%
    \begingroup
    \endlinechar=13 %
    \catcode13=\ltx@active
    \eolgrab@{#1}%
  }%
}
%    \end{macrocode}
%    \end{macro}
%
%    \begin{macro}{\eolgrabopt}
%    \begin{macrocode}
\eolgrab@ifdefinable{eolgrabopt}{%
  \ltx@IfUndefined{protected}{%
    \ltx@IfUndefined{DeclareRobustCommand}{%
      \def\eolgrabopt#1%
    }{%
      \newcommand\eolgrabopt{}%
      \DeclareRobustCommand*\eolgrabopt
    }%
  }{%
    \protected\def\eolgrabopt#1%
  }{%
    \begingroup
    \endlinechar=13 %
    \catcode13=\ltx@active
    \eolgrab@opt{#1}%
  }%
}
%    \end{macrocode}
%    \end{macro}
%
%    \begin{macrocode}
\begingroup
  \catcode13=\ltx@active %
\ltx@firstofone{\endgroup %
%    \end{macrocode}
%    \begin{macro}{\eolgrab@}
%    \begin{macrocode}
  \def\eolgrab@#1#2^^M{%
    \endgroup %
    #1{#2}%
  }%
%    \end{macrocode}
%    \end{macro}
%    \begin{macro}{\eolgrab@opt}
%    \begin{macrocode}
  \def\eolgrab@opt#1#2^^M{%
    \endgroup %
    \ltx@ifempty{#2}{%
      #1%
    }{%
      #1[{#2}]%
    }%
  }%
%    \end{macrocode}
%    \end{macro}
%    \begin{macrocode}
}%
%    \end{macrocode}
%
%    \begin{macrocode}
\eolgrab@AtEnd%
%    \end{macrocode}
%    \begin{macrocode}
%</package>
%    \end{macrocode}
%% \section{Installation}
%
% \subsection{Download}
%
% \paragraph{Package.} This package is available on
% CTAN\footnote{\CTANpkg{eolgrab}}:
% \begin{description}
% \item[\CTAN{macros/latex/contrib/oberdiek/eolgrab.dtx}] The source file.
% \item[\CTAN{macros/latex/contrib/oberdiek/eolgrab.pdf}] Documentation.
% \end{description}
%
%
% \paragraph{Bundle.} All the packages of the bundle `oberdiek'
% are also available in a TDS compliant ZIP archive. There
% the packages are already unpacked and the documentation files
% are generated. The files and directories obey the TDS standard.
% \begin{description}
% \item[\CTANinstall{install/macros/latex/contrib/oberdiek.tds.zip}]
% \end{description}
% \emph{TDS} refers to the standard ``A Directory Structure
% for \TeX\ Files'' (\CTANpkg{tds}). Directories
% with \xfile{texmf} in their name are usually organized this way.
%
% \subsection{Bundle installation}
%
% \paragraph{Unpacking.} Unpack the \xfile{oberdiek.tds.zip} in the
% TDS tree (also known as \xfile{texmf} tree) of your choice.
% Example (linux):
% \begin{quote}
%   |unzip oberdiek.tds.zip -d ~/texmf|
% \end{quote}
%
% \subsection{Package installation}
%
% \paragraph{Unpacking.} The \xfile{.dtx} file is a self-extracting
% \docstrip\ archive. The files are extracted by running the
% \xfile{.dtx} through \plainTeX:
% \begin{quote}
%   \verb|tex eolgrab.dtx|
% \end{quote}
%
% \paragraph{TDS.} Now the different files must be moved into
% the different directories in your installation TDS tree
% (also known as \xfile{texmf} tree):
% \begin{quote}
% \def\t{^^A
% \begin{tabular}{@{}>{\ttfamily}l@{ $\rightarrow$ }>{\ttfamily}l@{}}
%   eolgrab.sty & tex/generic/oberdiek/eolgrab.sty\\
%   eolgrab.pdf & doc/latex/oberdiek/eolgrab.pdf\\
%   example/eolgrab-example-ltx.tex & doc/latex/oberdiek/example/eolgrab-example-ltx.tex\\
%   example/eolgrab-example-env.tex & doc/latex/oberdiek/example/eolgrab-example-env.tex\\
%   example/eolgrab-example-sec.tex & doc/latex/oberdiek/example/eolgrab-example-sec.tex\\
%   eolgrab.dtx & source/latex/oberdiek/eolgrab.dtx\\
% \end{tabular}^^A
% }^^A
% \sbox0{\t}^^A
% \ifdim\wd0>\linewidth
%   \begingroup
%     \advance\linewidth by\leftmargin
%     \advance\linewidth by\rightmargin
%   \edef\x{\endgroup
%     \def\noexpand\lw{\the\linewidth}^^A
%   }\x
%   \def\lwbox{^^A
%     \leavevmode
%     \hbox to \linewidth{^^A
%       \kern-\leftmargin\relax
%       \hss
%       \usebox0
%       \hss
%       \kern-\rightmargin\relax
%     }^^A
%   }^^A
%   \ifdim\wd0>\lw
%     \sbox0{\small\t}^^A
%     \ifdim\wd0>\linewidth
%       \ifdim\wd0>\lw
%         \sbox0{\footnotesize\t}^^A
%         \ifdim\wd0>\linewidth
%           \ifdim\wd0>\lw
%             \sbox0{\scriptsize\t}^^A
%             \ifdim\wd0>\linewidth
%               \ifdim\wd0>\lw
%                 \sbox0{\tiny\t}^^A
%                 \ifdim\wd0>\linewidth
%                   \lwbox
%                 \else
%                   \usebox0
%                 \fi
%               \else
%                 \lwbox
%               \fi
%             \else
%               \usebox0
%             \fi
%           \else
%             \lwbox
%           \fi
%         \else
%           \usebox0
%         \fi
%       \else
%         \lwbox
%       \fi
%     \else
%       \usebox0
%     \fi
%   \else
%     \lwbox
%   \fi
% \else
%   \usebox0
% \fi
% \end{quote}
% If you have a \xfile{docstrip.cfg} that configures and enables \docstrip's
% TDS installing feature, then some files can already be in the right
% place, see the documentation of \docstrip.
%
% \subsection{Refresh file name databases}
%
% If your \TeX~distribution
% (\TeX\,Live, \mikTeX, \dots) relies on file name databases, you must refresh
% these. For example, \TeX\,Live\ users run \verb|texhash| or
% \verb|mktexlsr|.
%
% \subsection{Some details for the interested}
%
% \paragraph{Unpacking with \LaTeX.}
% The \xfile{.dtx} chooses its action depending on the format:
% \begin{description}
% \item[\plainTeX:] Run \docstrip\ and extract the files.
% \item[\LaTeX:] Generate the documentation.
% \end{description}
% If you insist on using \LaTeX\ for \docstrip\ (really,
% \docstrip\ does not need \LaTeX), then inform the autodetect routine
% about your intention:
% \begin{quote}
%   \verb|latex \let\install=y\input{eolgrab.dtx}|
% \end{quote}
% Do not forget to quote the argument according to the demands
% of your shell.
%
% \paragraph{Generating the documentation.}
% You can use both the \xfile{.dtx} or the \xfile{.drv} to generate
% the documentation. The process can be configured by the
% configuration file \xfile{ltxdoc.cfg}. For instance, put this
% line into this file, if you want to have A4 as paper format:
% \begin{quote}
%   \verb|\PassOptionsToClass{a4paper}{article}|
% \end{quote}
% An example follows how to generate the
% documentation with pdf\LaTeX:
% \begin{quote}
%\begin{verbatim}
%pdflatex eolgrab.dtx
%makeindex -s gind.ist eolgrab.idx
%pdflatex eolgrab.dtx
%makeindex -s gind.ist eolgrab.idx
%pdflatex eolgrab.dtx
%\end{verbatim}
% \end{quote}
%
% \begin{thebibliography}{9}
% \raggedright
% \bibitem{trautmann-request}
% Arno Trautmann,
% \textit{[texhax] read argument until EOL};
% mailing list \texttt{texthax@tug.org},
% 2011-01-06;
% \url{https://tug.org/pipermail/texhax/2011-January/016517.html}.
%
% \bibitem{taylor}
% Philip Taylor,
% \textit{Re: [texhax] read argument until EOL};
% mailing list \texttt{texhax@tug.org},
% 2011-01-06;
% \url{https://tug.org/pipermail/texhax/2011-January/016519.html}.
%
% \bibitem{oberdiek-eolgrab}
% Heiko Oberdiek,
% \textit{Re: [texhax] read argument until EOL};
% mailing list \texttt{texhax@tug.org},
% 2011-01-06;
% \url{https://tug.org/pipermail/texhax/2011-January/016526.html}.
%
% \end{thebibliography}
%
% \begin{History}
%   \begin{Version}{2011/01/12 v1.0}
%   \item
%     First public version.
%   \end{Version}
%   \begin{Version}{2016/05/16 v1.1}
%   \item
%     Documentation updates.
%   \end{Version}
% \end{History}
%
% \PrintIndex
%
% \Finale
\endinput

%        (quote the arguments according to the demands of your shell)
%
% Documentation:
%    (a) If eolgrab.drv is present:
%           latex eolgrab.drv
%    (b) Without eolgrab.drv:
%           latex eolgrab.dtx; ...
%    The class ltxdoc loads the configuration file ltxdoc.cfg
%    if available. Here you can specify further options, e.g.
%    use A4 as paper format:
%       \PassOptionsToClass{a4paper}{article}
%
%    Programm calls to get the documentation (example):
%       pdflatex eolgrab.dtx
%       makeindex -s gind.ist eolgrab.idx
%       pdflatex eolgrab.dtx
%       makeindex -s gind.ist eolgrab.idx
%       pdflatex eolgrab.dtx
%
% Installation:
%    TDS:tex/generic/oberdiek/eolgrab.sty
%    TDS:doc/latex/oberdiek/eolgrab.pdf
%    TDS:doc/latex/oberdiek/example/eolgrab-example-ltx.tex
%    TDS:doc/latex/oberdiek/example/eolgrab-example-env.tex
%    TDS:doc/latex/oberdiek/example/eolgrab-example-sec.tex
%    TDS:source/latex/oberdiek/eolgrab.dtx
%
%<*ignore>
\begingroup
  \catcode123=1 %
  \catcode125=2 %
  \def\x{LaTeX2e}%
\expandafter\endgroup
\ifcase 0\ifx\install y1\fi\expandafter
         \ifx\csname processbatchFile\endcsname\relax\else1\fi
         \ifx\fmtname\x\else 1\fi\relax
\else\csname fi\endcsname
%</ignore>
%<*install>
\input docstrip.tex
\Msg{************************************************************************}
\Msg{* Installation}
\Msg{* Package: eolgrab 2016/05/16 v1.1 Catch arguments delimited by end of line (HO)}
\Msg{************************************************************************}

\keepsilent
\askforoverwritefalse

\let\MetaPrefix\relax
\preamble

This is a generated file.

Project: eolgrab
Version: 2016/05/16 v1.1

Copyright (C)
   2011 Heiko Oberdiek
   2016-2019 Oberdiek Package Support Group

This work may be distributed and/or modified under the
conditions of the LaTeX Project Public License, either
version 1.3c of this license or (at your option) any later
version. This version of this license is in
   https://www.latex-project.org/lppl/lppl-1-3c.txt
and the latest version of this license is in
   https://www.latex-project.org/lppl.txt
and version 1.3 or later is part of all distributions of
LaTeX version 2005/12/01 or later.

This work has the LPPL maintenance status "maintained".

The Current Maintainers of this work are
Heiko Oberdiek and the Oberdiek Package Support Group
https://github.com/ho-tex/oberdiek/issues


The Base Interpreter refers to any `TeX-Format',
because some files are installed in TDS:tex/generic//.

This work consists of the main source file eolgrab.dtx
and the derived files
   eolgrab.sty, eolgrab.pdf, eolgrab.ins, eolgrab.drv,
   eolgrab-example-ltx.tex, eolgrab-example-env.tex,
   eolgrab-example-sec.tex, eolgrab-test1.tex,
   eolgrab-test2.tex.

\endpreamble
\let\MetaPrefix\DoubleperCent

\generate{%
  \file{eolgrab.ins}{\from{eolgrab.dtx}{install}}%
  \file{eolgrab.drv}{\from{eolgrab.dtx}{driver}}%
  \usedir{tex/generic/oberdiek}%
  \file{eolgrab.sty}{\from{eolgrab.dtx}{package}}%
  \usedir{doc/latex/oberdiek/example}%
  \file{eolgrab-example-ltx.tex}{\from{eolgrab.dtx}{example-ltx}}%
  \file{eolgrab-example-env.tex}{\from{eolgrab.dtx}{example-env}}%
  \file{eolgrab-example-sec.tex}{\from{eolgrab.dtx}{example-sec}}%
%  \usedir{doc/latex/oberdiek/test}%
%  \file{eolgrab-test1.tex}{\from{eolgrab.dtx}{test1}}%
%  \file{eolgrab-test2.tex}{\from{eolgrab.dtx}{test2}}%
}

\catcode32=13\relax% active space
\let =\space%
\Msg{************************************************************************}
\Msg{*}
\Msg{* To finish the installation you have to move the following}
\Msg{* file into a directory searched by TeX:}
\Msg{*}
\Msg{*     eolgrab.sty}
\Msg{*}
\Msg{* To produce the documentation run the file `eolgrab.drv'}
\Msg{* through LaTeX.}
\Msg{*}
\Msg{* Happy TeXing!}
\Msg{*}
\Msg{************************************************************************}

\endbatchfile
%</install>
%<*ignore>
\fi
%</ignore>
%<*driver>
\NeedsTeXFormat{LaTeX2e}
\ProvidesFile{eolgrab.drv}%
  [2016/05/16 v1.1 Catch arguments delimited by end of line (HO)]%
\documentclass{ltxdoc}
\usepackage{holtxdoc}[2011/11/22]
\begin{document}
  \DocInput{eolgrab.dtx}%
\end{document}
%</driver>
% \fi
%
%
%
% \GetFileInfo{eolgrab.drv}
%
% \title{The \xpackage{eolgrab} package}
% \date{2016/05/16 v1.1}
% \author{Heiko Oberdiek\thanks
% {Please report any issues at \url{https://github.com/ho-tex/oberdiek/issues}}}
%
% \maketitle
%
% \begin{abstract}
% This package implements a generic argument grabber
% to catch an argument that is delimited by the line end.
% \end{abstract}
%
% \tableofcontents
%
% \urldef\urltexhax\url{https://tug.org/mailman/listinfo/texhax}
% \def\eol/{^^A
%   \meta{\texttt{EOL}}^^A
% }
%
% \section{Documentation}
%
% The starting point for this package was a feature
% request of Arno Trautmann in the mailing list
% \texttt{texhax}\footnote{Info page for mailing list
% \texttt{texhax}: \urltexhax} \cite{trautmann-request}.
% A macro \cs{eolsection} should behave like \cs{section},
% but the argument should be delimited by the line end
% instead of given in curly braces:
% \begin{quote}
%   |\eolsection My Title|
% \end{quote}
% Phil Taylor answered this with an implementation for \cs{eolsection}.
% Because this feature could be useful for other macros as well,
% I answered with an implementation of \cs{eolgrab} as general solution
% \cite{oberdiek-eolgrab}.
%
% Both formats \hologo{plainTeX} and \hologo{LaTeX} are supported
% by the package, see the example for \cs{eolsection} below.
%
% \begin{declcs}{eolgrab} \M{code} \meta{argument} \eol/
% \end{declcs}
% Macro \cs{eolgrab} takes two arguments. The first argument
% is \meta{code}, a classical undelimited \hologo{TeX} macro argument.
% The second argument is delimited by the line end \eol/.
% The macro calls \meta{code} with \meta{argument} as argument
% in curly braces. Because the catcode of the line end is changed,
% \cs{eolgrab} will not work in the argument of other macros.
% Macro \cs{eolgrab} is made robust if either \hologo{eTeX}'s
% \cs{protected} or \hologo{LaTeX}'s \cs{DeclareRobustCommand}
% is available.
%
% \begin{declcs}{eolgrabopt} \M{code} \meta{argument} \eol/
% \end{declcs}
% Macro \cs{eolgrabopt} passes \meta{argument} as optional
% argument to \meta{code} if \meta{argument} is not empty.
% \begin{quote}
%   |\eolgrabopt\item foo|
% \end{quote}
% becomes to
% \begin{quote}
%   |\item[{foo}]|
% \end{quote}
% The curly argument braces are added to support square brackets
% inside \meta{argument}.
% If the \meta{argument} is empty:
% \begin{quote}
%   |\eolgrabopt\item|
% \end{quote}
% then
% \begin{quote}
%   |\item|
% \end{quote}
% is called without optional argument.
%
% \subsection{Examples}
%
% \begin{itemize}
% \item
%   The line
%   \begin{quote}
%     |\eolgrab\section My Title|
%   \end{quote}
%   is equivalent to
%   \begin{quote}
%     |\section{My Title}|
%   \end{quote}
% \item
%   The next example uses the star form of \cs{section}. Then the
%   command to be called consists of two tokens. Therefore the
%   first argument of \cs{eolgrab} needs curly braces:
%   \begin{quote}
%     |\eolgrab{\section*}My Title|
%   \end{quote}
%   becomes
%   \begin{quote}
%     |\section*{My Title}|
%   \end{quote}
% \item
%   Now \hologo{LaTeX}'s \cs{PackageError} is used. This macro
%   has three arguments, the package or class name, the message
%   text and the help text. A standard help text of \hologo{LaTeX}
%   is used as given in macro \cs{@ehc}. The second argument,
%   the message text is used as argument, delimited by line end:
%   \begin{quote}
%     |\eolgrab{\PackageError{foobar}}%|\\
%     |Some error message text\MessageBreak%|\\
%     |with several lines|\\
%     |\@ehc|
%   \end{quote}
%   In the first two lines of the example, the line end is suppressed
%   by the comment character (percent), thus the argument is delimited
%   by the line end of the third line. The result is:
%   \begin{quote}
%     |\PackageError{foobar}{Some error message text\MessageBreak|\\
%     |with several lines}\@ehc|
%   \end{quote}
% \item
%   The original request for macro \cs{eolsection}, see above,
%   can be implemented easily with the help of \cs{eolgrab}.
%   Example for \hologo{LaTeX}:
%   \begin{quote}
%     |\usepackage{eolgrab}|\\
%     |\newcommand*{\eolsection}{\eolgrab\section}|
%   \end{quote}
%   Example for \hologo{plainTeX}:
%   \begin{quote}
%     |\input eolgrab.sty\relax|\\
%     |\def\eolsection{\eolgrab\section}|
%   \end{quote}
%   And a sophisticated variant for \hologo{LaTeX} that also supports
%   the star syntax and the optional argument:
%    \begin{macrocode}
%<*example-sec>
\documentclass{article}
\usepackage{eolgrab}
\makeatletter
\newcommand*{\eolsection}{%
  \@ifstar{%
    \eolgrab{\section*}%
  }{%
    \@ifnextchar[{%
      \eoloptsection
    }{%
      \eolgrab\section
    }%
  }%
}
\newcommand*{\eoloptsection}[1][]{%
  \eolgrab{\section[{#1}]}%
}
\makeatother
\begin{document}
\tableofcontents
\eolsection Section without star and optional argument
\eolsection*Section with star
\eolsection[Short section title]Long section title
\end{document}
%</example-sec>
%    \end{macrocode}
% \end{itemize}
%
% \subsubsection{Small \hologo{LaTeX} document as example}
%
%    \begin{macrocode}
%<*example-ltx>
\RequirePackage{eolgrab}
\eolgrab\documentclass article
\eolgrab\begin document
\eolgrab\section Hello World
\eolgrab\emph Some text
\eolgrab\end document
%</example-ltx>
%    \end{macrocode}
%
% \subsubsection{\hologo{LaTeX} document with environments}
%
%    \begin{macrocode}
%<*example-env>
\documentclass{article}
\usepackage{eolgrab}
\newcommand*{\Begin}{\eolgrab   \begin}
\newcommand*{\End  }{\eolgrab   \end  }
\newcommand*{\Item }{\eolgrabopt\item }
\Begin document
  \Begin itemize
    \Item
      first item
    \Item
      second item
  \End itemize
  \Begin description
    \Item foo
      is the first syllable of foobar.
    \Item bar
      is the second syllable of foobar.
  \End description
\End document
%</example-env>
%    \end{macrocode}
%
% \subsection{Limitations}
%
% Macro \cs{eolgrab} needs to catch the line end.
% If \hologo{TeX} reads a line, then it throws away the line end
% characters (carriage return, line feed) and removes spaces at the
% end of the line. Then it adds the character with the character code
% that is given by \cs{endlinechar} at the end of the line. The category
% code of the inserted character is given by the current value of
% its \cs{catcode}. If \cs{endlinechar} is not a valid character code
% (especially if it is negative), then no character is added.
%
% In \hologo{plainTeX} and \hologo{LaTeX} the standard settings
% of the inserted endline character is the character with code 13
% (or |^||^||M| in \hologo{TeX} notation) with catcode 5 (end of line).
% That means the inserted end of line character behaves like a space
% token. For example, it is removed after macro names. Therefore
% \cs{eolgrab} changes the catcode.
%
% Therefore \cs{eolgrab} has some limitations:
% \begin{itemize}
% \item
%   Like other verbatim stuff, the macro \cs{eolgrab} cannot be
%   used in the argument of other macros. \cs{eolgrab} want to
%   change the catcode of the end of line character. If this
%   character is read before, because it is processed as argument
%   of another macro, the catcode is already set and is not reassigned
%   later if \cs{eolgrab} changes the category code for this character
%   code.
% \item
%   The argument must not contain the end of line character.
%   Otherwise the first end of line character is already taken
%   as delimiter, leaving the rest of the line outside the argument.
% \item
%   Because \cs{eolgrab} is probably mostly used in the line with
%   the delimited argument. Therefore changes of \cs{endlinechar}
%   will not affect the current line.
% \end{itemize}
%
% \StopEventually{
% }
%
% \section{Implementation}
%
%    \begin{macrocode}
%<*package>
%    \end{macrocode}
%
% \subsection{Reload check and package identification}
%    Reload check, especially if the package is not used with \LaTeX.
%    \begin{macrocode}
\begingroup\catcode61\catcode48\catcode32=10\relax%
  \catcode13=5 % ^^M
  \endlinechar=13 %
  \catcode35=6 % #
  \catcode39=12 % '
  \catcode44=12 % ,
  \catcode45=12 % -
  \catcode46=12 % .
  \catcode58=12 % :
  \catcode64=11 % @
  \catcode123=1 % {
  \catcode125=2 % }
  \expandafter\let\expandafter\x\csname ver@eolgrab.sty\endcsname
  \ifx\x\relax % plain-TeX, first loading
  \else
    \def\empty{}%
    \ifx\x\empty % LaTeX, first loading,
      % variable is initialized, but \ProvidesPackage not yet seen
    \else
      \expandafter\ifx\csname PackageInfo\endcsname\relax
        \def\x#1#2{%
          \immediate\write-1{Package #1 Info: #2.}%
        }%
      \else
        \def\x#1#2{\PackageInfo{#1}{#2, stopped}}%
      \fi
      \x{eolgrab}{The package is already loaded}%
      \aftergroup\endinput
    \fi
  \fi
\endgroup%
%    \end{macrocode}
%    Package identification:
%    \begin{macrocode}
\begingroup\catcode61\catcode48\catcode32=10\relax%
  \catcode13=5 % ^^M
  \endlinechar=13 %
  \catcode35=6 % #
  \catcode39=12 % '
  \catcode40=12 % (
  \catcode41=12 % )
  \catcode44=12 % ,
  \catcode45=12 % -
  \catcode46=12 % .
  \catcode47=12 % /
  \catcode58=12 % :
  \catcode64=11 % @
  \catcode91=12 % [
  \catcode93=12 % ]
  \catcode123=1 % {
  \catcode125=2 % }
  \expandafter\ifx\csname ProvidesPackage\endcsname\relax
    \def\x#1#2#3[#4]{\endgroup
      \immediate\write-1{Package: #3 #4}%
      \xdef#1{#4}%
    }%
  \else
    \def\x#1#2[#3]{\endgroup
      #2[{#3}]%
      \ifx#1\@undefined
        \xdef#1{#3}%
      \fi
      \ifx#1\relax
        \xdef#1{#3}%
      \fi
    }%
  \fi
\expandafter\x\csname ver@eolgrab.sty\endcsname
\ProvidesPackage{eolgrab}%
  [2016/05/16 v1.1 Catch arguments delimited by end of line (HO)]%
%    \end{macrocode}
%
% \subsection{Catcodes}
%
%    \begin{macrocode}
\begingroup\catcode61\catcode48\catcode32=10\relax%
  \catcode13=5 % ^^M
  \endlinechar=13 %
  \catcode123=1 % {
  \catcode125=2 % }
  \catcode64=11 % @
  \def\x{\endgroup
    \expandafter\edef\csname eolgrab@AtEnd\endcsname{%
      \endlinechar=\the\endlinechar\relax
      \catcode13=\the\catcode13\relax
      \catcode32=\the\catcode32\relax
      \catcode35=\the\catcode35\relax
      \catcode61=\the\catcode61\relax
      \catcode64=\the\catcode64\relax
      \catcode123=\the\catcode123\relax
      \catcode125=\the\catcode125\relax
    }%
  }%
\x\catcode61\catcode48\catcode32=10\relax%
\catcode13=5 % ^^M
\endlinechar=13 %
\catcode35=6 % #
\catcode64=11 % @
\catcode123=1 % {
\catcode125=2 % }
\def\TMP@EnsureCode#1#2{%
  \edef\eolgrab@AtEnd{%
    \eolgrab@AtEnd
    \catcode#1=\the\catcode#1\relax
  }%
  \catcode#1=#2\relax
}
\TMP@EnsureCode{40}{12}% (
\TMP@EnsureCode{41}{12}% )
\TMP@EnsureCode{42}{12}% *
\TMP@EnsureCode{46}{12}% .
\TMP@EnsureCode{47}{12}% /
\TMP@EnsureCode{91}{12}% [
\TMP@EnsureCode{93}{12}% ]
\TMP@EnsureCode{94}{7}% ^
\edef\eolgrab@AtEnd{\eolgrab@AtEnd\noexpand\endinput}
%    \end{macrocode}
%
% \subsection{Resources}
%
%    \begin{macrocode}
\begingroup\expandafter\expandafter\expandafter\endgroup
\expandafter\ifx\csname RequirePackage\endcsname\relax
  \input ltxcmds.sty\relax
  \input infwarerr.sty\relax
\else
  \RequirePackage{ltxcmds}[2010/12/04]%
  \RequirePackage{infwarerr}[2010/04/08]%
\fi
%    \end{macrocode}
%
%    \begin{macro}{\eolgrab@ifdefinable}
%    \begin{macrocode}
\ltx@IfUndefined{@ifdefinable}{%
  \def\eolgrab@ifdefinable#1#2{%
    \ltx@ifundefined{#1}{#2}{%
      \@PackageError{eolgrab}{%
        Command \ltx@backslashchar#1 already defined%
      }\@ehc
    }%
  }%
}{%
  \def\eolgrab@ifdefinable#1{%
    \expandafter\@ifdefinable\csname#1\endcsname
  }%
}
%    \end{macrocode}
%    \end{macro}
%
% \subsection{Macro \cs{eolgrab}}
%
%    \begin{macro}{\eolgrab}
%    \begin{macrocode}
\eolgrab@ifdefinable{eolgrab}{%
  \ltx@IfUndefined{protected}{%
    \ltx@IfUndefined{DeclareRobustCommand}{%
      \def\eolgrab#1%
    }{%
      \newcommand\eolgrab{}%
      \DeclareRobustCommand*\eolgrab
    }%
  }{%
    \protected\def\eolgrab#1%
  }{%
    \begingroup
    \endlinechar=13 %
    \catcode13=\ltx@active
    \eolgrab@{#1}%
  }%
}
%    \end{macrocode}
%    \end{macro}
%
%    \begin{macro}{\eolgrabopt}
%    \begin{macrocode}
\eolgrab@ifdefinable{eolgrabopt}{%
  \ltx@IfUndefined{protected}{%
    \ltx@IfUndefined{DeclareRobustCommand}{%
      \def\eolgrabopt#1%
    }{%
      \newcommand\eolgrabopt{}%
      \DeclareRobustCommand*\eolgrabopt
    }%
  }{%
    \protected\def\eolgrabopt#1%
  }{%
    \begingroup
    \endlinechar=13 %
    \catcode13=\ltx@active
    \eolgrab@opt{#1}%
  }%
}
%    \end{macrocode}
%    \end{macro}
%
%    \begin{macrocode}
\begingroup
  \catcode13=\ltx@active %
\ltx@firstofone{\endgroup %
%    \end{macrocode}
%    \begin{macro}{\eolgrab@}
%    \begin{macrocode}
  \def\eolgrab@#1#2^^M{%
    \endgroup %
    #1{#2}%
  }%
%    \end{macrocode}
%    \end{macro}
%    \begin{macro}{\eolgrab@opt}
%    \begin{macrocode}
  \def\eolgrab@opt#1#2^^M{%
    \endgroup %
    \ltx@ifempty{#2}{%
      #1%
    }{%
      #1[{#2}]%
    }%
  }%
%    \end{macrocode}
%    \end{macro}
%    \begin{macrocode}
}%
%    \end{macrocode}
%
%    \begin{macrocode}
\eolgrab@AtEnd%
%    \end{macrocode}
%    \begin{macrocode}
%</package>
%    \end{macrocode}
%% \section{Installation}
%
% \subsection{Download}
%
% \paragraph{Package.} This package is available on
% CTAN\footnote{\CTANpkg{eolgrab}}:
% \begin{description}
% \item[\CTAN{macros/latex/contrib/oberdiek/eolgrab.dtx}] The source file.
% \item[\CTAN{macros/latex/contrib/oberdiek/eolgrab.pdf}] Documentation.
% \end{description}
%
%
% \paragraph{Bundle.} All the packages of the bundle `oberdiek'
% are also available in a TDS compliant ZIP archive. There
% the packages are already unpacked and the documentation files
% are generated. The files and directories obey the TDS standard.
% \begin{description}
% \item[\CTANinstall{install/macros/latex/contrib/oberdiek.tds.zip}]
% \end{description}
% \emph{TDS} refers to the standard ``A Directory Structure
% for \TeX\ Files'' (\CTANpkg{tds}). Directories
% with \xfile{texmf} in their name are usually organized this way.
%
% \subsection{Bundle installation}
%
% \paragraph{Unpacking.} Unpack the \xfile{oberdiek.tds.zip} in the
% TDS tree (also known as \xfile{texmf} tree) of your choice.
% Example (linux):
% \begin{quote}
%   |unzip oberdiek.tds.zip -d ~/texmf|
% \end{quote}
%
% \subsection{Package installation}
%
% \paragraph{Unpacking.} The \xfile{.dtx} file is a self-extracting
% \docstrip\ archive. The files are extracted by running the
% \xfile{.dtx} through \plainTeX:
% \begin{quote}
%   \verb|tex eolgrab.dtx|
% \end{quote}
%
% \paragraph{TDS.} Now the different files must be moved into
% the different directories in your installation TDS tree
% (also known as \xfile{texmf} tree):
% \begin{quote}
% \def\t{^^A
% \begin{tabular}{@{}>{\ttfamily}l@{ $\rightarrow$ }>{\ttfamily}l@{}}
%   eolgrab.sty & tex/generic/oberdiek/eolgrab.sty\\
%   eolgrab.pdf & doc/latex/oberdiek/eolgrab.pdf\\
%   example/eolgrab-example-ltx.tex & doc/latex/oberdiek/example/eolgrab-example-ltx.tex\\
%   example/eolgrab-example-env.tex & doc/latex/oberdiek/example/eolgrab-example-env.tex\\
%   example/eolgrab-example-sec.tex & doc/latex/oberdiek/example/eolgrab-example-sec.tex\\
%   eolgrab.dtx & source/latex/oberdiek/eolgrab.dtx\\
% \end{tabular}^^A
% }^^A
% \sbox0{\t}^^A
% \ifdim\wd0>\linewidth
%   \begingroup
%     \advance\linewidth by\leftmargin
%     \advance\linewidth by\rightmargin
%   \edef\x{\endgroup
%     \def\noexpand\lw{\the\linewidth}^^A
%   }\x
%   \def\lwbox{^^A
%     \leavevmode
%     \hbox to \linewidth{^^A
%       \kern-\leftmargin\relax
%       \hss
%       \usebox0
%       \hss
%       \kern-\rightmargin\relax
%     }^^A
%   }^^A
%   \ifdim\wd0>\lw
%     \sbox0{\small\t}^^A
%     \ifdim\wd0>\linewidth
%       \ifdim\wd0>\lw
%         \sbox0{\footnotesize\t}^^A
%         \ifdim\wd0>\linewidth
%           \ifdim\wd0>\lw
%             \sbox0{\scriptsize\t}^^A
%             \ifdim\wd0>\linewidth
%               \ifdim\wd0>\lw
%                 \sbox0{\tiny\t}^^A
%                 \ifdim\wd0>\linewidth
%                   \lwbox
%                 \else
%                   \usebox0
%                 \fi
%               \else
%                 \lwbox
%               \fi
%             \else
%               \usebox0
%             \fi
%           \else
%             \lwbox
%           \fi
%         \else
%           \usebox0
%         \fi
%       \else
%         \lwbox
%       \fi
%     \else
%       \usebox0
%     \fi
%   \else
%     \lwbox
%   \fi
% \else
%   \usebox0
% \fi
% \end{quote}
% If you have a \xfile{docstrip.cfg} that configures and enables \docstrip's
% TDS installing feature, then some files can already be in the right
% place, see the documentation of \docstrip.
%
% \subsection{Refresh file name databases}
%
% If your \TeX~distribution
% (\TeX\,Live, \mikTeX, \dots) relies on file name databases, you must refresh
% these. For example, \TeX\,Live\ users run \verb|texhash| or
% \verb|mktexlsr|.
%
% \subsection{Some details for the interested}
%
% \paragraph{Unpacking with \LaTeX.}
% The \xfile{.dtx} chooses its action depending on the format:
% \begin{description}
% \item[\plainTeX:] Run \docstrip\ and extract the files.
% \item[\LaTeX:] Generate the documentation.
% \end{description}
% If you insist on using \LaTeX\ for \docstrip\ (really,
% \docstrip\ does not need \LaTeX), then inform the autodetect routine
% about your intention:
% \begin{quote}
%   \verb|latex \let\install=y% \iffalse meta-comment
%
% File: eolgrab.dtx
% Version: 2016/05/16 v1.1
% Info: Catch arguments delimited by end of line
%
% Copyright (C)
%    2011 Heiko Oberdiek
%    2016-2019 Oberdiek Package Support Group
%    https://github.com/ho-tex/oberdiek/issues
%
% This work may be distributed and/or modified under the
% conditions of the LaTeX Project Public License, either
% version 1.3c of this license or (at your option) any later
% version. This version of this license is in
%    https://www.latex-project.org/lppl/lppl-1-3c.txt
% and the latest version of this license is in
%    https://www.latex-project.org/lppl.txt
% and version 1.3 or later is part of all distributions of
% LaTeX version 2005/12/01 or later.
%
% This work has the LPPL maintenance status "maintained".
%
% The Current Maintainers of this work are
% Heiko Oberdiek and the Oberdiek Package Support Group
% https://github.com/ho-tex/oberdiek/issues
%
% The Base Interpreter refers to any `TeX-Format',
% because some files are installed in TDS:tex/generic//.
%
% This work consists of the main source file eolgrab.dtx
% and the derived files
%    eolgrab.sty, eolgrab.pdf, eolgrab.ins, eolgrab.drv,
%    eolgrab-example-ltx.tex, eolgrab-example-env.tex,
%    eolgrab-example-sec.tex, eolgrab-test1.tex,
%    eolgrab-test2.tex.
%
% Distribution:
%    CTAN:macros/latex/contrib/oberdiek/eolgrab.dtx
%    CTAN:macros/latex/contrib/oberdiek/eolgrab.pdf
%
% Unpacking:
%    (a) If eolgrab.ins is present:
%           tex eolgrab.ins
%    (b) Without eolgrab.ins:
%           tex eolgrab.dtx
%    (c) If you insist on using LaTeX
%           latex \let\install=y\input{eolgrab.dtx}
%        (quote the arguments according to the demands of your shell)
%
% Documentation:
%    (a) If eolgrab.drv is present:
%           latex eolgrab.drv
%    (b) Without eolgrab.drv:
%           latex eolgrab.dtx; ...
%    The class ltxdoc loads the configuration file ltxdoc.cfg
%    if available. Here you can specify further options, e.g.
%    use A4 as paper format:
%       \PassOptionsToClass{a4paper}{article}
%
%    Programm calls to get the documentation (example):
%       pdflatex eolgrab.dtx
%       makeindex -s gind.ist eolgrab.idx
%       pdflatex eolgrab.dtx
%       makeindex -s gind.ist eolgrab.idx
%       pdflatex eolgrab.dtx
%
% Installation:
%    TDS:tex/generic/oberdiek/eolgrab.sty
%    TDS:doc/latex/oberdiek/eolgrab.pdf
%    TDS:doc/latex/oberdiek/example/eolgrab-example-ltx.tex
%    TDS:doc/latex/oberdiek/example/eolgrab-example-env.tex
%    TDS:doc/latex/oberdiek/example/eolgrab-example-sec.tex
%    TDS:source/latex/oberdiek/eolgrab.dtx
%
%<*ignore>
\begingroup
  \catcode123=1 %
  \catcode125=2 %
  \def\x{LaTeX2e}%
\expandafter\endgroup
\ifcase 0\ifx\install y1\fi\expandafter
         \ifx\csname processbatchFile\endcsname\relax\else1\fi
         \ifx\fmtname\x\else 1\fi\relax
\else\csname fi\endcsname
%</ignore>
%<*install>
\input docstrip.tex
\Msg{************************************************************************}
\Msg{* Installation}
\Msg{* Package: eolgrab 2016/05/16 v1.1 Catch arguments delimited by end of line (HO)}
\Msg{************************************************************************}

\keepsilent
\askforoverwritefalse

\let\MetaPrefix\relax
\preamble

This is a generated file.

Project: eolgrab
Version: 2016/05/16 v1.1

Copyright (C)
   2011 Heiko Oberdiek
   2016-2019 Oberdiek Package Support Group

This work may be distributed and/or modified under the
conditions of the LaTeX Project Public License, either
version 1.3c of this license or (at your option) any later
version. This version of this license is in
   https://www.latex-project.org/lppl/lppl-1-3c.txt
and the latest version of this license is in
   https://www.latex-project.org/lppl.txt
and version 1.3 or later is part of all distributions of
LaTeX version 2005/12/01 or later.

This work has the LPPL maintenance status "maintained".

The Current Maintainers of this work are
Heiko Oberdiek and the Oberdiek Package Support Group
https://github.com/ho-tex/oberdiek/issues


The Base Interpreter refers to any `TeX-Format',
because some files are installed in TDS:tex/generic//.

This work consists of the main source file eolgrab.dtx
and the derived files
   eolgrab.sty, eolgrab.pdf, eolgrab.ins, eolgrab.drv,
   eolgrab-example-ltx.tex, eolgrab-example-env.tex,
   eolgrab-example-sec.tex, eolgrab-test1.tex,
   eolgrab-test2.tex.

\endpreamble
\let\MetaPrefix\DoubleperCent

\generate{%
  \file{eolgrab.ins}{\from{eolgrab.dtx}{install}}%
  \file{eolgrab.drv}{\from{eolgrab.dtx}{driver}}%
  \usedir{tex/generic/oberdiek}%
  \file{eolgrab.sty}{\from{eolgrab.dtx}{package}}%
  \usedir{doc/latex/oberdiek/example}%
  \file{eolgrab-example-ltx.tex}{\from{eolgrab.dtx}{example-ltx}}%
  \file{eolgrab-example-env.tex}{\from{eolgrab.dtx}{example-env}}%
  \file{eolgrab-example-sec.tex}{\from{eolgrab.dtx}{example-sec}}%
%  \usedir{doc/latex/oberdiek/test}%
%  \file{eolgrab-test1.tex}{\from{eolgrab.dtx}{test1}}%
%  \file{eolgrab-test2.tex}{\from{eolgrab.dtx}{test2}}%
}

\catcode32=13\relax% active space
\let =\space%
\Msg{************************************************************************}
\Msg{*}
\Msg{* To finish the installation you have to move the following}
\Msg{* file into a directory searched by TeX:}
\Msg{*}
\Msg{*     eolgrab.sty}
\Msg{*}
\Msg{* To produce the documentation run the file `eolgrab.drv'}
\Msg{* through LaTeX.}
\Msg{*}
\Msg{* Happy TeXing!}
\Msg{*}
\Msg{************************************************************************}

\endbatchfile
%</install>
%<*ignore>
\fi
%</ignore>
%<*driver>
\NeedsTeXFormat{LaTeX2e}
\ProvidesFile{eolgrab.drv}%
  [2016/05/16 v1.1 Catch arguments delimited by end of line (HO)]%
\documentclass{ltxdoc}
\usepackage{holtxdoc}[2011/11/22]
\begin{document}
  \DocInput{eolgrab.dtx}%
\end{document}
%</driver>
% \fi
%
%
%
% \GetFileInfo{eolgrab.drv}
%
% \title{The \xpackage{eolgrab} package}
% \date{2016/05/16 v1.1}
% \author{Heiko Oberdiek\thanks
% {Please report any issues at \url{https://github.com/ho-tex/oberdiek/issues}}}
%
% \maketitle
%
% \begin{abstract}
% This package implements a generic argument grabber
% to catch an argument that is delimited by the line end.
% \end{abstract}
%
% \tableofcontents
%
% \urldef\urltexhax\url{https://tug.org/mailman/listinfo/texhax}
% \def\eol/{^^A
%   \meta{\texttt{EOL}}^^A
% }
%
% \section{Documentation}
%
% The starting point for this package was a feature
% request of Arno Trautmann in the mailing list
% \texttt{texhax}\footnote{Info page for mailing list
% \texttt{texhax}: \urltexhax} \cite{trautmann-request}.
% A macro \cs{eolsection} should behave like \cs{section},
% but the argument should be delimited by the line end
% instead of given in curly braces:
% \begin{quote}
%   |\eolsection My Title|
% \end{quote}
% Phil Taylor answered this with an implementation for \cs{eolsection}.
% Because this feature could be useful for other macros as well,
% I answered with an implementation of \cs{eolgrab} as general solution
% \cite{oberdiek-eolgrab}.
%
% Both formats \hologo{plainTeX} and \hologo{LaTeX} are supported
% by the package, see the example for \cs{eolsection} below.
%
% \begin{declcs}{eolgrab} \M{code} \meta{argument} \eol/
% \end{declcs}
% Macro \cs{eolgrab} takes two arguments. The first argument
% is \meta{code}, a classical undelimited \hologo{TeX} macro argument.
% The second argument is delimited by the line end \eol/.
% The macro calls \meta{code} with \meta{argument} as argument
% in curly braces. Because the catcode of the line end is changed,
% \cs{eolgrab} will not work in the argument of other macros.
% Macro \cs{eolgrab} is made robust if either \hologo{eTeX}'s
% \cs{protected} or \hologo{LaTeX}'s \cs{DeclareRobustCommand}
% is available.
%
% \begin{declcs}{eolgrabopt} \M{code} \meta{argument} \eol/
% \end{declcs}
% Macro \cs{eolgrabopt} passes \meta{argument} as optional
% argument to \meta{code} if \meta{argument} is not empty.
% \begin{quote}
%   |\eolgrabopt\item foo|
% \end{quote}
% becomes to
% \begin{quote}
%   |\item[{foo}]|
% \end{quote}
% The curly argument braces are added to support square brackets
% inside \meta{argument}.
% If the \meta{argument} is empty:
% \begin{quote}
%   |\eolgrabopt\item|
% \end{quote}
% then
% \begin{quote}
%   |\item|
% \end{quote}
% is called without optional argument.
%
% \subsection{Examples}
%
% \begin{itemize}
% \item
%   The line
%   \begin{quote}
%     |\eolgrab\section My Title|
%   \end{quote}
%   is equivalent to
%   \begin{quote}
%     |\section{My Title}|
%   \end{quote}
% \item
%   The next example uses the star form of \cs{section}. Then the
%   command to be called consists of two tokens. Therefore the
%   first argument of \cs{eolgrab} needs curly braces:
%   \begin{quote}
%     |\eolgrab{\section*}My Title|
%   \end{quote}
%   becomes
%   \begin{quote}
%     |\section*{My Title}|
%   \end{quote}
% \item
%   Now \hologo{LaTeX}'s \cs{PackageError} is used. This macro
%   has three arguments, the package or class name, the message
%   text and the help text. A standard help text of \hologo{LaTeX}
%   is used as given in macro \cs{@ehc}. The second argument,
%   the message text is used as argument, delimited by line end:
%   \begin{quote}
%     |\eolgrab{\PackageError{foobar}}%|\\
%     |Some error message text\MessageBreak%|\\
%     |with several lines|\\
%     |\@ehc|
%   \end{quote}
%   In the first two lines of the example, the line end is suppressed
%   by the comment character (percent), thus the argument is delimited
%   by the line end of the third line. The result is:
%   \begin{quote}
%     |\PackageError{foobar}{Some error message text\MessageBreak|\\
%     |with several lines}\@ehc|
%   \end{quote}
% \item
%   The original request for macro \cs{eolsection}, see above,
%   can be implemented easily with the help of \cs{eolgrab}.
%   Example for \hologo{LaTeX}:
%   \begin{quote}
%     |\usepackage{eolgrab}|\\
%     |\newcommand*{\eolsection}{\eolgrab\section}|
%   \end{quote}
%   Example for \hologo{plainTeX}:
%   \begin{quote}
%     |\input eolgrab.sty\relax|\\
%     |\def\eolsection{\eolgrab\section}|
%   \end{quote}
%   And a sophisticated variant for \hologo{LaTeX} that also supports
%   the star syntax and the optional argument:
%    \begin{macrocode}
%<*example-sec>
\documentclass{article}
\usepackage{eolgrab}
\makeatletter
\newcommand*{\eolsection}{%
  \@ifstar{%
    \eolgrab{\section*}%
  }{%
    \@ifnextchar[{%
      \eoloptsection
    }{%
      \eolgrab\section
    }%
  }%
}
\newcommand*{\eoloptsection}[1][]{%
  \eolgrab{\section[{#1}]}%
}
\makeatother
\begin{document}
\tableofcontents
\eolsection Section without star and optional argument
\eolsection*Section with star
\eolsection[Short section title]Long section title
\end{document}
%</example-sec>
%    \end{macrocode}
% \end{itemize}
%
% \subsubsection{Small \hologo{LaTeX} document as example}
%
%    \begin{macrocode}
%<*example-ltx>
\RequirePackage{eolgrab}
\eolgrab\documentclass article
\eolgrab\begin document
\eolgrab\section Hello World
\eolgrab\emph Some text
\eolgrab\end document
%</example-ltx>
%    \end{macrocode}
%
% \subsubsection{\hologo{LaTeX} document with environments}
%
%    \begin{macrocode}
%<*example-env>
\documentclass{article}
\usepackage{eolgrab}
\newcommand*{\Begin}{\eolgrab   \begin}
\newcommand*{\End  }{\eolgrab   \end  }
\newcommand*{\Item }{\eolgrabopt\item }
\Begin document
  \Begin itemize
    \Item
      first item
    \Item
      second item
  \End itemize
  \Begin description
    \Item foo
      is the first syllable of foobar.
    \Item bar
      is the second syllable of foobar.
  \End description
\End document
%</example-env>
%    \end{macrocode}
%
% \subsection{Limitations}
%
% Macro \cs{eolgrab} needs to catch the line end.
% If \hologo{TeX} reads a line, then it throws away the line end
% characters (carriage return, line feed) and removes spaces at the
% end of the line. Then it adds the character with the character code
% that is given by \cs{endlinechar} at the end of the line. The category
% code of the inserted character is given by the current value of
% its \cs{catcode}. If \cs{endlinechar} is not a valid character code
% (especially if it is negative), then no character is added.
%
% In \hologo{plainTeX} and \hologo{LaTeX} the standard settings
% of the inserted endline character is the character with code 13
% (or |^||^||M| in \hologo{TeX} notation) with catcode 5 (end of line).
% That means the inserted end of line character behaves like a space
% token. For example, it is removed after macro names. Therefore
% \cs{eolgrab} changes the catcode.
%
% Therefore \cs{eolgrab} has some limitations:
% \begin{itemize}
% \item
%   Like other verbatim stuff, the macro \cs{eolgrab} cannot be
%   used in the argument of other macros. \cs{eolgrab} want to
%   change the catcode of the end of line character. If this
%   character is read before, because it is processed as argument
%   of another macro, the catcode is already set and is not reassigned
%   later if \cs{eolgrab} changes the category code for this character
%   code.
% \item
%   The argument must not contain the end of line character.
%   Otherwise the first end of line character is already taken
%   as delimiter, leaving the rest of the line outside the argument.
% \item
%   Because \cs{eolgrab} is probably mostly used in the line with
%   the delimited argument. Therefore changes of \cs{endlinechar}
%   will not affect the current line.
% \end{itemize}
%
% \StopEventually{
% }
%
% \section{Implementation}
%
%    \begin{macrocode}
%<*package>
%    \end{macrocode}
%
% \subsection{Reload check and package identification}
%    Reload check, especially if the package is not used with \LaTeX.
%    \begin{macrocode}
\begingroup\catcode61\catcode48\catcode32=10\relax%
  \catcode13=5 % ^^M
  \endlinechar=13 %
  \catcode35=6 % #
  \catcode39=12 % '
  \catcode44=12 % ,
  \catcode45=12 % -
  \catcode46=12 % .
  \catcode58=12 % :
  \catcode64=11 % @
  \catcode123=1 % {
  \catcode125=2 % }
  \expandafter\let\expandafter\x\csname ver@eolgrab.sty\endcsname
  \ifx\x\relax % plain-TeX, first loading
  \else
    \def\empty{}%
    \ifx\x\empty % LaTeX, first loading,
      % variable is initialized, but \ProvidesPackage not yet seen
    \else
      \expandafter\ifx\csname PackageInfo\endcsname\relax
        \def\x#1#2{%
          \immediate\write-1{Package #1 Info: #2.}%
        }%
      \else
        \def\x#1#2{\PackageInfo{#1}{#2, stopped}}%
      \fi
      \x{eolgrab}{The package is already loaded}%
      \aftergroup\endinput
    \fi
  \fi
\endgroup%
%    \end{macrocode}
%    Package identification:
%    \begin{macrocode}
\begingroup\catcode61\catcode48\catcode32=10\relax%
  \catcode13=5 % ^^M
  \endlinechar=13 %
  \catcode35=6 % #
  \catcode39=12 % '
  \catcode40=12 % (
  \catcode41=12 % )
  \catcode44=12 % ,
  \catcode45=12 % -
  \catcode46=12 % .
  \catcode47=12 % /
  \catcode58=12 % :
  \catcode64=11 % @
  \catcode91=12 % [
  \catcode93=12 % ]
  \catcode123=1 % {
  \catcode125=2 % }
  \expandafter\ifx\csname ProvidesPackage\endcsname\relax
    \def\x#1#2#3[#4]{\endgroup
      \immediate\write-1{Package: #3 #4}%
      \xdef#1{#4}%
    }%
  \else
    \def\x#1#2[#3]{\endgroup
      #2[{#3}]%
      \ifx#1\@undefined
        \xdef#1{#3}%
      \fi
      \ifx#1\relax
        \xdef#1{#3}%
      \fi
    }%
  \fi
\expandafter\x\csname ver@eolgrab.sty\endcsname
\ProvidesPackage{eolgrab}%
  [2016/05/16 v1.1 Catch arguments delimited by end of line (HO)]%
%    \end{macrocode}
%
% \subsection{Catcodes}
%
%    \begin{macrocode}
\begingroup\catcode61\catcode48\catcode32=10\relax%
  \catcode13=5 % ^^M
  \endlinechar=13 %
  \catcode123=1 % {
  \catcode125=2 % }
  \catcode64=11 % @
  \def\x{\endgroup
    \expandafter\edef\csname eolgrab@AtEnd\endcsname{%
      \endlinechar=\the\endlinechar\relax
      \catcode13=\the\catcode13\relax
      \catcode32=\the\catcode32\relax
      \catcode35=\the\catcode35\relax
      \catcode61=\the\catcode61\relax
      \catcode64=\the\catcode64\relax
      \catcode123=\the\catcode123\relax
      \catcode125=\the\catcode125\relax
    }%
  }%
\x\catcode61\catcode48\catcode32=10\relax%
\catcode13=5 % ^^M
\endlinechar=13 %
\catcode35=6 % #
\catcode64=11 % @
\catcode123=1 % {
\catcode125=2 % }
\def\TMP@EnsureCode#1#2{%
  \edef\eolgrab@AtEnd{%
    \eolgrab@AtEnd
    \catcode#1=\the\catcode#1\relax
  }%
  \catcode#1=#2\relax
}
\TMP@EnsureCode{40}{12}% (
\TMP@EnsureCode{41}{12}% )
\TMP@EnsureCode{42}{12}% *
\TMP@EnsureCode{46}{12}% .
\TMP@EnsureCode{47}{12}% /
\TMP@EnsureCode{91}{12}% [
\TMP@EnsureCode{93}{12}% ]
\TMP@EnsureCode{94}{7}% ^
\edef\eolgrab@AtEnd{\eolgrab@AtEnd\noexpand\endinput}
%    \end{macrocode}
%
% \subsection{Resources}
%
%    \begin{macrocode}
\begingroup\expandafter\expandafter\expandafter\endgroup
\expandafter\ifx\csname RequirePackage\endcsname\relax
  \input ltxcmds.sty\relax
  \input infwarerr.sty\relax
\else
  \RequirePackage{ltxcmds}[2010/12/04]%
  \RequirePackage{infwarerr}[2010/04/08]%
\fi
%    \end{macrocode}
%
%    \begin{macro}{\eolgrab@ifdefinable}
%    \begin{macrocode}
\ltx@IfUndefined{@ifdefinable}{%
  \def\eolgrab@ifdefinable#1#2{%
    \ltx@ifundefined{#1}{#2}{%
      \@PackageError{eolgrab}{%
        Command \ltx@backslashchar#1 already defined%
      }\@ehc
    }%
  }%
}{%
  \def\eolgrab@ifdefinable#1{%
    \expandafter\@ifdefinable\csname#1\endcsname
  }%
}
%    \end{macrocode}
%    \end{macro}
%
% \subsection{Macro \cs{eolgrab}}
%
%    \begin{macro}{\eolgrab}
%    \begin{macrocode}
\eolgrab@ifdefinable{eolgrab}{%
  \ltx@IfUndefined{protected}{%
    \ltx@IfUndefined{DeclareRobustCommand}{%
      \def\eolgrab#1%
    }{%
      \newcommand\eolgrab{}%
      \DeclareRobustCommand*\eolgrab
    }%
  }{%
    \protected\def\eolgrab#1%
  }{%
    \begingroup
    \endlinechar=13 %
    \catcode13=\ltx@active
    \eolgrab@{#1}%
  }%
}
%    \end{macrocode}
%    \end{macro}
%
%    \begin{macro}{\eolgrabopt}
%    \begin{macrocode}
\eolgrab@ifdefinable{eolgrabopt}{%
  \ltx@IfUndefined{protected}{%
    \ltx@IfUndefined{DeclareRobustCommand}{%
      \def\eolgrabopt#1%
    }{%
      \newcommand\eolgrabopt{}%
      \DeclareRobustCommand*\eolgrabopt
    }%
  }{%
    \protected\def\eolgrabopt#1%
  }{%
    \begingroup
    \endlinechar=13 %
    \catcode13=\ltx@active
    \eolgrab@opt{#1}%
  }%
}
%    \end{macrocode}
%    \end{macro}
%
%    \begin{macrocode}
\begingroup
  \catcode13=\ltx@active %
\ltx@firstofone{\endgroup %
%    \end{macrocode}
%    \begin{macro}{\eolgrab@}
%    \begin{macrocode}
  \def\eolgrab@#1#2^^M{%
    \endgroup %
    #1{#2}%
  }%
%    \end{macrocode}
%    \end{macro}
%    \begin{macro}{\eolgrab@opt}
%    \begin{macrocode}
  \def\eolgrab@opt#1#2^^M{%
    \endgroup %
    \ltx@ifempty{#2}{%
      #1%
    }{%
      #1[{#2}]%
    }%
  }%
%    \end{macrocode}
%    \end{macro}
%    \begin{macrocode}
}%
%    \end{macrocode}
%
%    \begin{macrocode}
\eolgrab@AtEnd%
%    \end{macrocode}
%    \begin{macrocode}
%</package>
%    \end{macrocode}
%% \section{Installation}
%
% \subsection{Download}
%
% \paragraph{Package.} This package is available on
% CTAN\footnote{\CTANpkg{eolgrab}}:
% \begin{description}
% \item[\CTAN{macros/latex/contrib/oberdiek/eolgrab.dtx}] The source file.
% \item[\CTAN{macros/latex/contrib/oberdiek/eolgrab.pdf}] Documentation.
% \end{description}
%
%
% \paragraph{Bundle.} All the packages of the bundle `oberdiek'
% are also available in a TDS compliant ZIP archive. There
% the packages are already unpacked and the documentation files
% are generated. The files and directories obey the TDS standard.
% \begin{description}
% \item[\CTANinstall{install/macros/latex/contrib/oberdiek.tds.zip}]
% \end{description}
% \emph{TDS} refers to the standard ``A Directory Structure
% for \TeX\ Files'' (\CTANpkg{tds}). Directories
% with \xfile{texmf} in their name are usually organized this way.
%
% \subsection{Bundle installation}
%
% \paragraph{Unpacking.} Unpack the \xfile{oberdiek.tds.zip} in the
% TDS tree (also known as \xfile{texmf} tree) of your choice.
% Example (linux):
% \begin{quote}
%   |unzip oberdiek.tds.zip -d ~/texmf|
% \end{quote}
%
% \subsection{Package installation}
%
% \paragraph{Unpacking.} The \xfile{.dtx} file is a self-extracting
% \docstrip\ archive. The files are extracted by running the
% \xfile{.dtx} through \plainTeX:
% \begin{quote}
%   \verb|tex eolgrab.dtx|
% \end{quote}
%
% \paragraph{TDS.} Now the different files must be moved into
% the different directories in your installation TDS tree
% (also known as \xfile{texmf} tree):
% \begin{quote}
% \def\t{^^A
% \begin{tabular}{@{}>{\ttfamily}l@{ $\rightarrow$ }>{\ttfamily}l@{}}
%   eolgrab.sty & tex/generic/oberdiek/eolgrab.sty\\
%   eolgrab.pdf & doc/latex/oberdiek/eolgrab.pdf\\
%   example/eolgrab-example-ltx.tex & doc/latex/oberdiek/example/eolgrab-example-ltx.tex\\
%   example/eolgrab-example-env.tex & doc/latex/oberdiek/example/eolgrab-example-env.tex\\
%   example/eolgrab-example-sec.tex & doc/latex/oberdiek/example/eolgrab-example-sec.tex\\
%   eolgrab.dtx & source/latex/oberdiek/eolgrab.dtx\\
% \end{tabular}^^A
% }^^A
% \sbox0{\t}^^A
% \ifdim\wd0>\linewidth
%   \begingroup
%     \advance\linewidth by\leftmargin
%     \advance\linewidth by\rightmargin
%   \edef\x{\endgroup
%     \def\noexpand\lw{\the\linewidth}^^A
%   }\x
%   \def\lwbox{^^A
%     \leavevmode
%     \hbox to \linewidth{^^A
%       \kern-\leftmargin\relax
%       \hss
%       \usebox0
%       \hss
%       \kern-\rightmargin\relax
%     }^^A
%   }^^A
%   \ifdim\wd0>\lw
%     \sbox0{\small\t}^^A
%     \ifdim\wd0>\linewidth
%       \ifdim\wd0>\lw
%         \sbox0{\footnotesize\t}^^A
%         \ifdim\wd0>\linewidth
%           \ifdim\wd0>\lw
%             \sbox0{\scriptsize\t}^^A
%             \ifdim\wd0>\linewidth
%               \ifdim\wd0>\lw
%                 \sbox0{\tiny\t}^^A
%                 \ifdim\wd0>\linewidth
%                   \lwbox
%                 \else
%                   \usebox0
%                 \fi
%               \else
%                 \lwbox
%               \fi
%             \else
%               \usebox0
%             \fi
%           \else
%             \lwbox
%           \fi
%         \else
%           \usebox0
%         \fi
%       \else
%         \lwbox
%       \fi
%     \else
%       \usebox0
%     \fi
%   \else
%     \lwbox
%   \fi
% \else
%   \usebox0
% \fi
% \end{quote}
% If you have a \xfile{docstrip.cfg} that configures and enables \docstrip's
% TDS installing feature, then some files can already be in the right
% place, see the documentation of \docstrip.
%
% \subsection{Refresh file name databases}
%
% If your \TeX~distribution
% (\TeX\,Live, \mikTeX, \dots) relies on file name databases, you must refresh
% these. For example, \TeX\,Live\ users run \verb|texhash| or
% \verb|mktexlsr|.
%
% \subsection{Some details for the interested}
%
% \paragraph{Unpacking with \LaTeX.}
% The \xfile{.dtx} chooses its action depending on the format:
% \begin{description}
% \item[\plainTeX:] Run \docstrip\ and extract the files.
% \item[\LaTeX:] Generate the documentation.
% \end{description}
% If you insist on using \LaTeX\ for \docstrip\ (really,
% \docstrip\ does not need \LaTeX), then inform the autodetect routine
% about your intention:
% \begin{quote}
%   \verb|latex \let\install=y\input{eolgrab.dtx}|
% \end{quote}
% Do not forget to quote the argument according to the demands
% of your shell.
%
% \paragraph{Generating the documentation.}
% You can use both the \xfile{.dtx} or the \xfile{.drv} to generate
% the documentation. The process can be configured by the
% configuration file \xfile{ltxdoc.cfg}. For instance, put this
% line into this file, if you want to have A4 as paper format:
% \begin{quote}
%   \verb|\PassOptionsToClass{a4paper}{article}|
% \end{quote}
% An example follows how to generate the
% documentation with pdf\LaTeX:
% \begin{quote}
%\begin{verbatim}
%pdflatex eolgrab.dtx
%makeindex -s gind.ist eolgrab.idx
%pdflatex eolgrab.dtx
%makeindex -s gind.ist eolgrab.idx
%pdflatex eolgrab.dtx
%\end{verbatim}
% \end{quote}
%
% \begin{thebibliography}{9}
% \raggedright
% \bibitem{trautmann-request}
% Arno Trautmann,
% \textit{[texhax] read argument until EOL};
% mailing list \texttt{texthax@tug.org},
% 2011-01-06;
% \url{https://tug.org/pipermail/texhax/2011-January/016517.html}.
%
% \bibitem{taylor}
% Philip Taylor,
% \textit{Re: [texhax] read argument until EOL};
% mailing list \texttt{texhax@tug.org},
% 2011-01-06;
% \url{https://tug.org/pipermail/texhax/2011-January/016519.html}.
%
% \bibitem{oberdiek-eolgrab}
% Heiko Oberdiek,
% \textit{Re: [texhax] read argument until EOL};
% mailing list \texttt{texhax@tug.org},
% 2011-01-06;
% \url{https://tug.org/pipermail/texhax/2011-January/016526.html}.
%
% \end{thebibliography}
%
% \begin{History}
%   \begin{Version}{2011/01/12 v1.0}
%   \item
%     First public version.
%   \end{Version}
%   \begin{Version}{2016/05/16 v1.1}
%   \item
%     Documentation updates.
%   \end{Version}
% \end{History}
%
% \PrintIndex
%
% \Finale
\endinput
|
% \end{quote}
% Do not forget to quote the argument according to the demands
% of your shell.
%
% \paragraph{Generating the documentation.}
% You can use both the \xfile{.dtx} or the \xfile{.drv} to generate
% the documentation. The process can be configured by the
% configuration file \xfile{ltxdoc.cfg}. For instance, put this
% line into this file, if you want to have A4 as paper format:
% \begin{quote}
%   \verb|\PassOptionsToClass{a4paper}{article}|
% \end{quote}
% An example follows how to generate the
% documentation with pdf\LaTeX:
% \begin{quote}
%\begin{verbatim}
%pdflatex eolgrab.dtx
%makeindex -s gind.ist eolgrab.idx
%pdflatex eolgrab.dtx
%makeindex -s gind.ist eolgrab.idx
%pdflatex eolgrab.dtx
%\end{verbatim}
% \end{quote}
%
% \begin{thebibliography}{9}
% \raggedright
% \bibitem{trautmann-request}
% Arno Trautmann,
% \textit{[texhax] read argument until EOL};
% mailing list \texttt{texthax@tug.org},
% 2011-01-06;
% \url{https://tug.org/pipermail/texhax/2011-January/016517.html}.
%
% \bibitem{taylor}
% Philip Taylor,
% \textit{Re: [texhax] read argument until EOL};
% mailing list \texttt{texhax@tug.org},
% 2011-01-06;
% \url{https://tug.org/pipermail/texhax/2011-January/016519.html}.
%
% \bibitem{oberdiek-eolgrab}
% Heiko Oberdiek,
% \textit{Re: [texhax] read argument until EOL};
% mailing list \texttt{texhax@tug.org},
% 2011-01-06;
% \url{https://tug.org/pipermail/texhax/2011-January/016526.html}.
%
% \end{thebibliography}
%
% \begin{History}
%   \begin{Version}{2011/01/12 v1.0}
%   \item
%     First public version.
%   \end{Version}
%   \begin{Version}{2016/05/16 v1.1}
%   \item
%     Documentation updates.
%   \end{Version}
% \end{History}
%
% \PrintIndex
%
% \Finale
\endinput
|
% \end{quote}
% Do not forget to quote the argument according to the demands
% of your shell.
%
% \paragraph{Generating the documentation.}
% You can use both the \xfile{.dtx} or the \xfile{.drv} to generate
% the documentation. The process can be configured by the
% configuration file \xfile{ltxdoc.cfg}. For instance, put this
% line into this file, if you want to have A4 as paper format:
% \begin{quote}
%   \verb|\PassOptionsToClass{a4paper}{article}|
% \end{quote}
% An example follows how to generate the
% documentation with pdf\LaTeX:
% \begin{quote}
%\begin{verbatim}
%pdflatex eolgrab.dtx
%makeindex -s gind.ist eolgrab.idx
%pdflatex eolgrab.dtx
%makeindex -s gind.ist eolgrab.idx
%pdflatex eolgrab.dtx
%\end{verbatim}
% \end{quote}
%
% \begin{thebibliography}{9}
% \raggedright
% \bibitem{trautmann-request}
% Arno Trautmann,
% \textit{[texhax] read argument until EOL};
% mailing list \texttt{texthax@tug.org},
% 2011-01-06;
% \url{https://tug.org/pipermail/texhax/2011-January/016517.html}.
%
% \bibitem{taylor}
% Philip Taylor,
% \textit{Re: [texhax] read argument until EOL};
% mailing list \texttt{texhax@tug.org},
% 2011-01-06;
% \url{https://tug.org/pipermail/texhax/2011-January/016519.html}.
%
% \bibitem{oberdiek-eolgrab}
% Heiko Oberdiek,
% \textit{Re: [texhax] read argument until EOL};
% mailing list \texttt{texhax@tug.org},
% 2011-01-06;
% \url{https://tug.org/pipermail/texhax/2011-January/016526.html}.
%
% \end{thebibliography}
%
% \begin{History}
%   \begin{Version}{2011/01/12 v1.0}
%   \item
%     First public version.
%   \end{Version}
%   \begin{Version}{2016/05/16 v1.1}
%   \item
%     Documentation updates.
%   \end{Version}
% \end{History}
%
% \PrintIndex
%
% \Finale
\endinput

%        (quote the arguments according to the demands of your shell)
%
% Documentation:
%    (a) If eolgrab.drv is present:
%           latex eolgrab.drv
%    (b) Without eolgrab.drv:
%           latex eolgrab.dtx; ...
%    The class ltxdoc loads the configuration file ltxdoc.cfg
%    if available. Here you can specify further options, e.g.
%    use A4 as paper format:
%       \PassOptionsToClass{a4paper}{article}
%
%    Programm calls to get the documentation (example):
%       pdflatex eolgrab.dtx
%       makeindex -s gind.ist eolgrab.idx
%       pdflatex eolgrab.dtx
%       makeindex -s gind.ist eolgrab.idx
%       pdflatex eolgrab.dtx
%
% Installation:
%    TDS:tex/generic/oberdiek/eolgrab.sty
%    TDS:doc/latex/oberdiek/eolgrab.pdf
%    TDS:doc/latex/oberdiek/example/eolgrab-example-ltx.tex
%    TDS:doc/latex/oberdiek/example/eolgrab-example-env.tex
%    TDS:doc/latex/oberdiek/example/eolgrab-example-sec.tex
%    TDS:source/latex/oberdiek/eolgrab.dtx
%
%<*ignore>
\begingroup
  \catcode123=1 %
  \catcode125=2 %
  \def\x{LaTeX2e}%
\expandafter\endgroup
\ifcase 0\ifx\install y1\fi\expandafter
         \ifx\csname processbatchFile\endcsname\relax\else1\fi
         \ifx\fmtname\x\else 1\fi\relax
\else\csname fi\endcsname
%</ignore>
%<*install>
\input docstrip.tex
\Msg{************************************************************************}
\Msg{* Installation}
\Msg{* Package: eolgrab 2016/05/16 v1.1 Catch arguments delimited by end of line (HO)}
\Msg{************************************************************************}

\keepsilent
\askforoverwritefalse

\let\MetaPrefix\relax
\preamble

This is a generated file.

Project: eolgrab
Version: 2016/05/16 v1.1

Copyright (C)
   2011 Heiko Oberdiek
   2016-2019 Oberdiek Package Support Group

This work may be distributed and/or modified under the
conditions of the LaTeX Project Public License, either
version 1.3c of this license or (at your option) any later
version. This version of this license is in
   https://www.latex-project.org/lppl/lppl-1-3c.txt
and the latest version of this license is in
   https://www.latex-project.org/lppl.txt
and version 1.3 or later is part of all distributions of
LaTeX version 2005/12/01 or later.

This work has the LPPL maintenance status "maintained".

The Current Maintainers of this work are
Heiko Oberdiek and the Oberdiek Package Support Group
https://github.com/ho-tex/oberdiek/issues


The Base Interpreter refers to any `TeX-Format',
because some files are installed in TDS:tex/generic//.

This work consists of the main source file eolgrab.dtx
and the derived files
   eolgrab.sty, eolgrab.pdf, eolgrab.ins, eolgrab.drv,
   eolgrab-example-ltx.tex, eolgrab-example-env.tex,
   eolgrab-example-sec.tex, eolgrab-test1.tex,
   eolgrab-test2.tex.

\endpreamble
\let\MetaPrefix\DoubleperCent

\generate{%
  \file{eolgrab.ins}{\from{eolgrab.dtx}{install}}%
  \file{eolgrab.drv}{\from{eolgrab.dtx}{driver}}%
  \usedir{tex/generic/oberdiek}%
  \file{eolgrab.sty}{\from{eolgrab.dtx}{package}}%
  \usedir{doc/latex/oberdiek/example}%
  \file{eolgrab-example-ltx.tex}{\from{eolgrab.dtx}{example-ltx}}%
  \file{eolgrab-example-env.tex}{\from{eolgrab.dtx}{example-env}}%
  \file{eolgrab-example-sec.tex}{\from{eolgrab.dtx}{example-sec}}%
%  \usedir{doc/latex/oberdiek/test}%
%  \file{eolgrab-test1.tex}{\from{eolgrab.dtx}{test1}}%
%  \file{eolgrab-test2.tex}{\from{eolgrab.dtx}{test2}}%
}

\catcode32=13\relax% active space
\let =\space%
\Msg{************************************************************************}
\Msg{*}
\Msg{* To finish the installation you have to move the following}
\Msg{* file into a directory searched by TeX:}
\Msg{*}
\Msg{*     eolgrab.sty}
\Msg{*}
\Msg{* To produce the documentation run the file `eolgrab.drv'}
\Msg{* through LaTeX.}
\Msg{*}
\Msg{* Happy TeXing!}
\Msg{*}
\Msg{************************************************************************}

\endbatchfile
%</install>
%<*ignore>
\fi
%</ignore>
%<*driver>
\NeedsTeXFormat{LaTeX2e}
\ProvidesFile{eolgrab.drv}%
  [2016/05/16 v1.1 Catch arguments delimited by end of line (HO)]%
\documentclass{ltxdoc}
\usepackage{holtxdoc}[2011/11/22]
\begin{document}
  \DocInput{eolgrab.dtx}%
\end{document}
%</driver>
% \fi
%
%
%
% \GetFileInfo{eolgrab.drv}
%
% \title{The \xpackage{eolgrab} package}
% \date{2016/05/16 v1.1}
% \author{Heiko Oberdiek\thanks
% {Please report any issues at \url{https://github.com/ho-tex/oberdiek/issues}}}
%
% \maketitle
%
% \begin{abstract}
% This package implements a generic argument grabber
% to catch an argument that is delimited by the line end.
% \end{abstract}
%
% \tableofcontents
%
% \urldef\urltexhax\url{https://tug.org/mailman/listinfo/texhax}
% \def\eol/{^^A
%   \meta{\texttt{EOL}}^^A
% }
%
% \section{Documentation}
%
% The starting point for this package was a feature
% request of Arno Trautmann in the mailing list
% \texttt{texhax}\footnote{Info page for mailing list
% \texttt{texhax}: \urltexhax} \cite{trautmann-request}.
% A macro \cs{eolsection} should behave like \cs{section},
% but the argument should be delimited by the line end
% instead of given in curly braces:
% \begin{quote}
%   |\eolsection My Title|
% \end{quote}
% Phil Taylor answered this with an implementation for \cs{eolsection}.
% Because this feature could be useful for other macros as well,
% I answered with an implementation of \cs{eolgrab} as general solution
% \cite{oberdiek-eolgrab}.
%
% Both formats \hologo{plainTeX} and \hologo{LaTeX} are supported
% by the package, see the example for \cs{eolsection} below.
%
% \begin{declcs}{eolgrab} \M{code} \meta{argument} \eol/
% \end{declcs}
% Macro \cs{eolgrab} takes two arguments. The first argument
% is \meta{code}, a classical undelimited \hologo{TeX} macro argument.
% The second argument is delimited by the line end \eol/.
% The macro calls \meta{code} with \meta{argument} as argument
% in curly braces. Because the catcode of the line end is changed,
% \cs{eolgrab} will not work in the argument of other macros.
% Macro \cs{eolgrab} is made robust if either \hologo{eTeX}'s
% \cs{protected} or \hologo{LaTeX}'s \cs{DeclareRobustCommand}
% is available.
%
% \begin{declcs}{eolgrabopt} \M{code} \meta{argument} \eol/
% \end{declcs}
% Macro \cs{eolgrabopt} passes \meta{argument} as optional
% argument to \meta{code} if \meta{argument} is not empty.
% \begin{quote}
%   |\eolgrabopt\item foo|
% \end{quote}
% becomes to
% \begin{quote}
%   |\item[{foo}]|
% \end{quote}
% The curly argument braces are added to support square brackets
% inside \meta{argument}.
% If the \meta{argument} is empty:
% \begin{quote}
%   |\eolgrabopt\item|
% \end{quote}
% then
% \begin{quote}
%   |\item|
% \end{quote}
% is called without optional argument.
%
% \subsection{Examples}
%
% \begin{itemize}
% \item
%   The line
%   \begin{quote}
%     |\eolgrab\section My Title|
%   \end{quote}
%   is equivalent to
%   \begin{quote}
%     |\section{My Title}|
%   \end{quote}
% \item
%   The next example uses the star form of \cs{section}. Then the
%   command to be called consists of two tokens. Therefore the
%   first argument of \cs{eolgrab} needs curly braces:
%   \begin{quote}
%     |\eolgrab{\section*}My Title|
%   \end{quote}
%   becomes
%   \begin{quote}
%     |\section*{My Title}|
%   \end{quote}
% \item
%   Now \hologo{LaTeX}'s \cs{PackageError} is used. This macro
%   has three arguments, the package or class name, the message
%   text and the help text. A standard help text of \hologo{LaTeX}
%   is used as given in macro \cs{@ehc}. The second argument,
%   the message text is used as argument, delimited by line end:
%   \begin{quote}
%     |\eolgrab{\PackageError{foobar}}%|\\
%     |Some error message text\MessageBreak%|\\
%     |with several lines|\\
%     |\@ehc|
%   \end{quote}
%   In the first two lines of the example, the line end is suppressed
%   by the comment character (percent), thus the argument is delimited
%   by the line end of the third line. The result is:
%   \begin{quote}
%     |\PackageError{foobar}{Some error message text\MessageBreak|\\
%     |with several lines}\@ehc|
%   \end{quote}
% \item
%   The original request for macro \cs{eolsection}, see above,
%   can be implemented easily with the help of \cs{eolgrab}.
%   Example for \hologo{LaTeX}:
%   \begin{quote}
%     |\usepackage{eolgrab}|\\
%     |\newcommand*{\eolsection}{\eolgrab\section}|
%   \end{quote}
%   Example for \hologo{plainTeX}:
%   \begin{quote}
%     |\input eolgrab.sty\relax|\\
%     |\def\eolsection{\eolgrab\section}|
%   \end{quote}
%   And a sophisticated variant for \hologo{LaTeX} that also supports
%   the star syntax and the optional argument:
%    \begin{macrocode}
%<*example-sec>
\documentclass{article}
\usepackage{eolgrab}
\makeatletter
\newcommand*{\eolsection}{%
  \@ifstar{%
    \eolgrab{\section*}%
  }{%
    \@ifnextchar[{%
      \eoloptsection
    }{%
      \eolgrab\section
    }%
  }%
}
\newcommand*{\eoloptsection}[1][]{%
  \eolgrab{\section[{#1}]}%
}
\makeatother
\begin{document}
\tableofcontents
\eolsection Section without star and optional argument
\eolsection*Section with star
\eolsection[Short section title]Long section title
\end{document}
%</example-sec>
%    \end{macrocode}
% \end{itemize}
%
% \subsubsection{Small \hologo{LaTeX} document as example}
%
%    \begin{macrocode}
%<*example-ltx>
\RequirePackage{eolgrab}
\eolgrab\documentclass article
\eolgrab\begin document
\eolgrab\section Hello World
\eolgrab\emph Some text
\eolgrab\end document
%</example-ltx>
%    \end{macrocode}
%
% \subsubsection{\hologo{LaTeX} document with environments}
%
%    \begin{macrocode}
%<*example-env>
\documentclass{article}
\usepackage{eolgrab}
\newcommand*{\Begin}{\eolgrab   \begin}
\newcommand*{\End  }{\eolgrab   \end  }
\newcommand*{\Item }{\eolgrabopt\item }
\Begin document
  \Begin itemize
    \Item
      first item
    \Item
      second item
  \End itemize
  \Begin description
    \Item foo
      is the first syllable of foobar.
    \Item bar
      is the second syllable of foobar.
  \End description
\End document
%</example-env>
%    \end{macrocode}
%
% \subsection{Limitations}
%
% Macro \cs{eolgrab} needs to catch the line end.
% If \hologo{TeX} reads a line, then it throws away the line end
% characters (carriage return, line feed) and removes spaces at the
% end of the line. Then it adds the character with the character code
% that is given by \cs{endlinechar} at the end of the line. The category
% code of the inserted character is given by the current value of
% its \cs{catcode}. If \cs{endlinechar} is not a valid character code
% (especially if it is negative), then no character is added.
%
% In \hologo{plainTeX} and \hologo{LaTeX} the standard settings
% of the inserted endline character is the character with code 13
% (or |^||^||M| in \hologo{TeX} notation) with catcode 5 (end of line).
% That means the inserted end of line character behaves like a space
% token. For example, it is removed after macro names. Therefore
% \cs{eolgrab} changes the catcode.
%
% Therefore \cs{eolgrab} has some limitations:
% \begin{itemize}
% \item
%   Like other verbatim stuff, the macro \cs{eolgrab} cannot be
%   used in the argument of other macros. \cs{eolgrab} want to
%   change the catcode of the end of line character. If this
%   character is read before, because it is processed as argument
%   of another macro, the catcode is already set and is not reassigned
%   later if \cs{eolgrab} changes the category code for this character
%   code.
% \item
%   The argument must not contain the end of line character.
%   Otherwise the first end of line character is already taken
%   as delimiter, leaving the rest of the line outside the argument.
% \item
%   Because \cs{eolgrab} is probably mostly used in the line with
%   the delimited argument. Therefore changes of \cs{endlinechar}
%   will not affect the current line.
% \end{itemize}
%
% \StopEventually{
% }
%
% \section{Implementation}
%
%    \begin{macrocode}
%<*package>
%    \end{macrocode}
%
% \subsection{Reload check and package identification}
%    Reload check, especially if the package is not used with \LaTeX.
%    \begin{macrocode}
\begingroup\catcode61\catcode48\catcode32=10\relax%
  \catcode13=5 % ^^M
  \endlinechar=13 %
  \catcode35=6 % #
  \catcode39=12 % '
  \catcode44=12 % ,
  \catcode45=12 % -
  \catcode46=12 % .
  \catcode58=12 % :
  \catcode64=11 % @
  \catcode123=1 % {
  \catcode125=2 % }
  \expandafter\let\expandafter\x\csname ver@eolgrab.sty\endcsname
  \ifx\x\relax % plain-TeX, first loading
  \else
    \def\empty{}%
    \ifx\x\empty % LaTeX, first loading,
      % variable is initialized, but \ProvidesPackage not yet seen
    \else
      \expandafter\ifx\csname PackageInfo\endcsname\relax
        \def\x#1#2{%
          \immediate\write-1{Package #1 Info: #2.}%
        }%
      \else
        \def\x#1#2{\PackageInfo{#1}{#2, stopped}}%
      \fi
      \x{eolgrab}{The package is already loaded}%
      \aftergroup\endinput
    \fi
  \fi
\endgroup%
%    \end{macrocode}
%    Package identification:
%    \begin{macrocode}
\begingroup\catcode61\catcode48\catcode32=10\relax%
  \catcode13=5 % ^^M
  \endlinechar=13 %
  \catcode35=6 % #
  \catcode39=12 % '
  \catcode40=12 % (
  \catcode41=12 % )
  \catcode44=12 % ,
  \catcode45=12 % -
  \catcode46=12 % .
  \catcode47=12 % /
  \catcode58=12 % :
  \catcode64=11 % @
  \catcode91=12 % [
  \catcode93=12 % ]
  \catcode123=1 % {
  \catcode125=2 % }
  \expandafter\ifx\csname ProvidesPackage\endcsname\relax
    \def\x#1#2#3[#4]{\endgroup
      \immediate\write-1{Package: #3 #4}%
      \xdef#1{#4}%
    }%
  \else
    \def\x#1#2[#3]{\endgroup
      #2[{#3}]%
      \ifx#1\@undefined
        \xdef#1{#3}%
      \fi
      \ifx#1\relax
        \xdef#1{#3}%
      \fi
    }%
  \fi
\expandafter\x\csname ver@eolgrab.sty\endcsname
\ProvidesPackage{eolgrab}%
  [2016/05/16 v1.1 Catch arguments delimited by end of line (HO)]%
%    \end{macrocode}
%
% \subsection{Catcodes}
%
%    \begin{macrocode}
\begingroup\catcode61\catcode48\catcode32=10\relax%
  \catcode13=5 % ^^M
  \endlinechar=13 %
  \catcode123=1 % {
  \catcode125=2 % }
  \catcode64=11 % @
  \def\x{\endgroup
    \expandafter\edef\csname eolgrab@AtEnd\endcsname{%
      \endlinechar=\the\endlinechar\relax
      \catcode13=\the\catcode13\relax
      \catcode32=\the\catcode32\relax
      \catcode35=\the\catcode35\relax
      \catcode61=\the\catcode61\relax
      \catcode64=\the\catcode64\relax
      \catcode123=\the\catcode123\relax
      \catcode125=\the\catcode125\relax
    }%
  }%
\x\catcode61\catcode48\catcode32=10\relax%
\catcode13=5 % ^^M
\endlinechar=13 %
\catcode35=6 % #
\catcode64=11 % @
\catcode123=1 % {
\catcode125=2 % }
\def\TMP@EnsureCode#1#2{%
  \edef\eolgrab@AtEnd{%
    \eolgrab@AtEnd
    \catcode#1=\the\catcode#1\relax
  }%
  \catcode#1=#2\relax
}
\TMP@EnsureCode{40}{12}% (
\TMP@EnsureCode{41}{12}% )
\TMP@EnsureCode{42}{12}% *
\TMP@EnsureCode{46}{12}% .
\TMP@EnsureCode{47}{12}% /
\TMP@EnsureCode{91}{12}% [
\TMP@EnsureCode{93}{12}% ]
\TMP@EnsureCode{94}{7}% ^
\edef\eolgrab@AtEnd{\eolgrab@AtEnd\noexpand\endinput}
%    \end{macrocode}
%
% \subsection{Resources}
%
%    \begin{macrocode}
\begingroup\expandafter\expandafter\expandafter\endgroup
\expandafter\ifx\csname RequirePackage\endcsname\relax
  \input ltxcmds.sty\relax
  \input infwarerr.sty\relax
\else
  \RequirePackage{ltxcmds}[2010/12/04]%
  \RequirePackage{infwarerr}[2010/04/08]%
\fi
%    \end{macrocode}
%
%    \begin{macro}{\eolgrab@ifdefinable}
%    \begin{macrocode}
\ltx@IfUndefined{@ifdefinable}{%
  \def\eolgrab@ifdefinable#1#2{%
    \ltx@ifundefined{#1}{#2}{%
      \@PackageError{eolgrab}{%
        Command \ltx@backslashchar#1 already defined%
      }\@ehc
    }%
  }%
}{%
  \def\eolgrab@ifdefinable#1{%
    \expandafter\@ifdefinable\csname#1\endcsname
  }%
}
%    \end{macrocode}
%    \end{macro}
%
% \subsection{Macro \cs{eolgrab}}
%
%    \begin{macro}{\eolgrab}
%    \begin{macrocode}
\eolgrab@ifdefinable{eolgrab}{%
  \ltx@IfUndefined{protected}{%
    \ltx@IfUndefined{DeclareRobustCommand}{%
      \def\eolgrab#1%
    }{%
      \newcommand\eolgrab{}%
      \DeclareRobustCommand*\eolgrab
    }%
  }{%
    \protected\def\eolgrab#1%
  }{%
    \begingroup
    \endlinechar=13 %
    \catcode13=\ltx@active
    \eolgrab@{#1}%
  }%
}
%    \end{macrocode}
%    \end{macro}
%
%    \begin{macro}{\eolgrabopt}
%    \begin{macrocode}
\eolgrab@ifdefinable{eolgrabopt}{%
  \ltx@IfUndefined{protected}{%
    \ltx@IfUndefined{DeclareRobustCommand}{%
      \def\eolgrabopt#1%
    }{%
      \newcommand\eolgrabopt{}%
      \DeclareRobustCommand*\eolgrabopt
    }%
  }{%
    \protected\def\eolgrabopt#1%
  }{%
    \begingroup
    \endlinechar=13 %
    \catcode13=\ltx@active
    \eolgrab@opt{#1}%
  }%
}
%    \end{macrocode}
%    \end{macro}
%
%    \begin{macrocode}
\begingroup
  \catcode13=\ltx@active %
\ltx@firstofone{\endgroup %
%    \end{macrocode}
%    \begin{macro}{\eolgrab@}
%    \begin{macrocode}
  \def\eolgrab@#1#2^^M{%
    \endgroup %
    #1{#2}%
  }%
%    \end{macrocode}
%    \end{macro}
%    \begin{macro}{\eolgrab@opt}
%    \begin{macrocode}
  \def\eolgrab@opt#1#2^^M{%
    \endgroup %
    \ltx@ifempty{#2}{%
      #1%
    }{%
      #1[{#2}]%
    }%
  }%
%    \end{macrocode}
%    \end{macro}
%    \begin{macrocode}
}%
%    \end{macrocode}
%
%    \begin{macrocode}
\eolgrab@AtEnd%
%    \end{macrocode}
%    \begin{macrocode}
%</package>
%    \end{macrocode}
%% \section{Installation}
%
% \subsection{Download}
%
% \paragraph{Package.} This package is available on
% CTAN\footnote{\CTANpkg{eolgrab}}:
% \begin{description}
% \item[\CTAN{macros/latex/contrib/oberdiek/eolgrab.dtx}] The source file.
% \item[\CTAN{macros/latex/contrib/oberdiek/eolgrab.pdf}] Documentation.
% \end{description}
%
%
% \paragraph{Bundle.} All the packages of the bundle `oberdiek'
% are also available in a TDS compliant ZIP archive. There
% the packages are already unpacked and the documentation files
% are generated. The files and directories obey the TDS standard.
% \begin{description}
% \item[\CTANinstall{install/macros/latex/contrib/oberdiek.tds.zip}]
% \end{description}
% \emph{TDS} refers to the standard ``A Directory Structure
% for \TeX\ Files'' (\CTANpkg{tds}). Directories
% with \xfile{texmf} in their name are usually organized this way.
%
% \subsection{Bundle installation}
%
% \paragraph{Unpacking.} Unpack the \xfile{oberdiek.tds.zip} in the
% TDS tree (also known as \xfile{texmf} tree) of your choice.
% Example (linux):
% \begin{quote}
%   |unzip oberdiek.tds.zip -d ~/texmf|
% \end{quote}
%
% \subsection{Package installation}
%
% \paragraph{Unpacking.} The \xfile{.dtx} file is a self-extracting
% \docstrip\ archive. The files are extracted by running the
% \xfile{.dtx} through \plainTeX:
% \begin{quote}
%   \verb|tex eolgrab.dtx|
% \end{quote}
%
% \paragraph{TDS.} Now the different files must be moved into
% the different directories in your installation TDS tree
% (also known as \xfile{texmf} tree):
% \begin{quote}
% \def\t{^^A
% \begin{tabular}{@{}>{\ttfamily}l@{ $\rightarrow$ }>{\ttfamily}l@{}}
%   eolgrab.sty & tex/generic/oberdiek/eolgrab.sty\\
%   eolgrab.pdf & doc/latex/oberdiek/eolgrab.pdf\\
%   example/eolgrab-example-ltx.tex & doc/latex/oberdiek/example/eolgrab-example-ltx.tex\\
%   example/eolgrab-example-env.tex & doc/latex/oberdiek/example/eolgrab-example-env.tex\\
%   example/eolgrab-example-sec.tex & doc/latex/oberdiek/example/eolgrab-example-sec.tex\\
%   eolgrab.dtx & source/latex/oberdiek/eolgrab.dtx\\
% \end{tabular}^^A
% }^^A
% \sbox0{\t}^^A
% \ifdim\wd0>\linewidth
%   \begingroup
%     \advance\linewidth by\leftmargin
%     \advance\linewidth by\rightmargin
%   \edef\x{\endgroup
%     \def\noexpand\lw{\the\linewidth}^^A
%   }\x
%   \def\lwbox{^^A
%     \leavevmode
%     \hbox to \linewidth{^^A
%       \kern-\leftmargin\relax
%       \hss
%       \usebox0
%       \hss
%       \kern-\rightmargin\relax
%     }^^A
%   }^^A
%   \ifdim\wd0>\lw
%     \sbox0{\small\t}^^A
%     \ifdim\wd0>\linewidth
%       \ifdim\wd0>\lw
%         \sbox0{\footnotesize\t}^^A
%         \ifdim\wd0>\linewidth
%           \ifdim\wd0>\lw
%             \sbox0{\scriptsize\t}^^A
%             \ifdim\wd0>\linewidth
%               \ifdim\wd0>\lw
%                 \sbox0{\tiny\t}^^A
%                 \ifdim\wd0>\linewidth
%                   \lwbox
%                 \else
%                   \usebox0
%                 \fi
%               \else
%                 \lwbox
%               \fi
%             \else
%               \usebox0
%             \fi
%           \else
%             \lwbox
%           \fi
%         \else
%           \usebox0
%         \fi
%       \else
%         \lwbox
%       \fi
%     \else
%       \usebox0
%     \fi
%   \else
%     \lwbox
%   \fi
% \else
%   \usebox0
% \fi
% \end{quote}
% If you have a \xfile{docstrip.cfg} that configures and enables \docstrip's
% TDS installing feature, then some files can already be in the right
% place, see the documentation of \docstrip.
%
% \subsection{Refresh file name databases}
%
% If your \TeX~distribution
% (\TeX\,Live, \mikTeX, \dots) relies on file name databases, you must refresh
% these. For example, \TeX\,Live\ users run \verb|texhash| or
% \verb|mktexlsr|.
%
% \subsection{Some details for the interested}
%
% \paragraph{Unpacking with \LaTeX.}
% The \xfile{.dtx} chooses its action depending on the format:
% \begin{description}
% \item[\plainTeX:] Run \docstrip\ and extract the files.
% \item[\LaTeX:] Generate the documentation.
% \end{description}
% If you insist on using \LaTeX\ for \docstrip\ (really,
% \docstrip\ does not need \LaTeX), then inform the autodetect routine
% about your intention:
% \begin{quote}
%   \verb|latex \let\install=y% \iffalse meta-comment
%
% File: eolgrab.dtx
% Version: 2016/05/16 v1.1
% Info: Catch arguments delimited by end of line
%
% Copyright (C)
%    2011 Heiko Oberdiek
%    2016-2019 Oberdiek Package Support Group
%    https://github.com/ho-tex/oberdiek/issues
%
% This work may be distributed and/or modified under the
% conditions of the LaTeX Project Public License, either
% version 1.3c of this license or (at your option) any later
% version. This version of this license is in
%    https://www.latex-project.org/lppl/lppl-1-3c.txt
% and the latest version of this license is in
%    https://www.latex-project.org/lppl.txt
% and version 1.3 or later is part of all distributions of
% LaTeX version 2005/12/01 or later.
%
% This work has the LPPL maintenance status "maintained".
%
% The Current Maintainers of this work are
% Heiko Oberdiek and the Oberdiek Package Support Group
% https://github.com/ho-tex/oberdiek/issues
%
% The Base Interpreter refers to any `TeX-Format',
% because some files are installed in TDS:tex/generic//.
%
% This work consists of the main source file eolgrab.dtx
% and the derived files
%    eolgrab.sty, eolgrab.pdf, eolgrab.ins, eolgrab.drv,
%    eolgrab-example-ltx.tex, eolgrab-example-env.tex,
%    eolgrab-example-sec.tex, eolgrab-test1.tex,
%    eolgrab-test2.tex.
%
% Distribution:
%    CTAN:macros/latex/contrib/oberdiek/eolgrab.dtx
%    CTAN:macros/latex/contrib/oberdiek/eolgrab.pdf
%
% Unpacking:
%    (a) If eolgrab.ins is present:
%           tex eolgrab.ins
%    (b) Without eolgrab.ins:
%           tex eolgrab.dtx
%    (c) If you insist on using LaTeX
%           latex \let\install=y% \iffalse meta-comment
%
% File: eolgrab.dtx
% Version: 2016/05/16 v1.1
% Info: Catch arguments delimited by end of line
%
% Copyright (C)
%    2011 Heiko Oberdiek
%    2016-2019 Oberdiek Package Support Group
%    https://github.com/ho-tex/oberdiek/issues
%
% This work may be distributed and/or modified under the
% conditions of the LaTeX Project Public License, either
% version 1.3c of this license or (at your option) any later
% version. This version of this license is in
%    https://www.latex-project.org/lppl/lppl-1-3c.txt
% and the latest version of this license is in
%    https://www.latex-project.org/lppl.txt
% and version 1.3 or later is part of all distributions of
% LaTeX version 2005/12/01 or later.
%
% This work has the LPPL maintenance status "maintained".
%
% The Current Maintainers of this work are
% Heiko Oberdiek and the Oberdiek Package Support Group
% https://github.com/ho-tex/oberdiek/issues
%
% The Base Interpreter refers to any `TeX-Format',
% because some files are installed in TDS:tex/generic//.
%
% This work consists of the main source file eolgrab.dtx
% and the derived files
%    eolgrab.sty, eolgrab.pdf, eolgrab.ins, eolgrab.drv,
%    eolgrab-example-ltx.tex, eolgrab-example-env.tex,
%    eolgrab-example-sec.tex, eolgrab-test1.tex,
%    eolgrab-test2.tex.
%
% Distribution:
%    CTAN:macros/latex/contrib/oberdiek/eolgrab.dtx
%    CTAN:macros/latex/contrib/oberdiek/eolgrab.pdf
%
% Unpacking:
%    (a) If eolgrab.ins is present:
%           tex eolgrab.ins
%    (b) Without eolgrab.ins:
%           tex eolgrab.dtx
%    (c) If you insist on using LaTeX
%           latex \let\install=y% \iffalse meta-comment
%
% File: eolgrab.dtx
% Version: 2016/05/16 v1.1
% Info: Catch arguments delimited by end of line
%
% Copyright (C)
%    2011 Heiko Oberdiek
%    2016-2019 Oberdiek Package Support Group
%    https://github.com/ho-tex/oberdiek/issues
%
% This work may be distributed and/or modified under the
% conditions of the LaTeX Project Public License, either
% version 1.3c of this license or (at your option) any later
% version. This version of this license is in
%    https://www.latex-project.org/lppl/lppl-1-3c.txt
% and the latest version of this license is in
%    https://www.latex-project.org/lppl.txt
% and version 1.3 or later is part of all distributions of
% LaTeX version 2005/12/01 or later.
%
% This work has the LPPL maintenance status "maintained".
%
% The Current Maintainers of this work are
% Heiko Oberdiek and the Oberdiek Package Support Group
% https://github.com/ho-tex/oberdiek/issues
%
% The Base Interpreter refers to any `TeX-Format',
% because some files are installed in TDS:tex/generic//.
%
% This work consists of the main source file eolgrab.dtx
% and the derived files
%    eolgrab.sty, eolgrab.pdf, eolgrab.ins, eolgrab.drv,
%    eolgrab-example-ltx.tex, eolgrab-example-env.tex,
%    eolgrab-example-sec.tex, eolgrab-test1.tex,
%    eolgrab-test2.tex.
%
% Distribution:
%    CTAN:macros/latex/contrib/oberdiek/eolgrab.dtx
%    CTAN:macros/latex/contrib/oberdiek/eolgrab.pdf
%
% Unpacking:
%    (a) If eolgrab.ins is present:
%           tex eolgrab.ins
%    (b) Without eolgrab.ins:
%           tex eolgrab.dtx
%    (c) If you insist on using LaTeX
%           latex \let\install=y\input{eolgrab.dtx}
%        (quote the arguments according to the demands of your shell)
%
% Documentation:
%    (a) If eolgrab.drv is present:
%           latex eolgrab.drv
%    (b) Without eolgrab.drv:
%           latex eolgrab.dtx; ...
%    The class ltxdoc loads the configuration file ltxdoc.cfg
%    if available. Here you can specify further options, e.g.
%    use A4 as paper format:
%       \PassOptionsToClass{a4paper}{article}
%
%    Programm calls to get the documentation (example):
%       pdflatex eolgrab.dtx
%       makeindex -s gind.ist eolgrab.idx
%       pdflatex eolgrab.dtx
%       makeindex -s gind.ist eolgrab.idx
%       pdflatex eolgrab.dtx
%
% Installation:
%    TDS:tex/generic/oberdiek/eolgrab.sty
%    TDS:doc/latex/oberdiek/eolgrab.pdf
%    TDS:doc/latex/oberdiek/example/eolgrab-example-ltx.tex
%    TDS:doc/latex/oberdiek/example/eolgrab-example-env.tex
%    TDS:doc/latex/oberdiek/example/eolgrab-example-sec.tex
%    TDS:source/latex/oberdiek/eolgrab.dtx
%
%<*ignore>
\begingroup
  \catcode123=1 %
  \catcode125=2 %
  \def\x{LaTeX2e}%
\expandafter\endgroup
\ifcase 0\ifx\install y1\fi\expandafter
         \ifx\csname processbatchFile\endcsname\relax\else1\fi
         \ifx\fmtname\x\else 1\fi\relax
\else\csname fi\endcsname
%</ignore>
%<*install>
\input docstrip.tex
\Msg{************************************************************************}
\Msg{* Installation}
\Msg{* Package: eolgrab 2016/05/16 v1.1 Catch arguments delimited by end of line (HO)}
\Msg{************************************************************************}

\keepsilent
\askforoverwritefalse

\let\MetaPrefix\relax
\preamble

This is a generated file.

Project: eolgrab
Version: 2016/05/16 v1.1

Copyright (C)
   2011 Heiko Oberdiek
   2016-2019 Oberdiek Package Support Group

This work may be distributed and/or modified under the
conditions of the LaTeX Project Public License, either
version 1.3c of this license or (at your option) any later
version. This version of this license is in
   https://www.latex-project.org/lppl/lppl-1-3c.txt
and the latest version of this license is in
   https://www.latex-project.org/lppl.txt
and version 1.3 or later is part of all distributions of
LaTeX version 2005/12/01 or later.

This work has the LPPL maintenance status "maintained".

The Current Maintainers of this work are
Heiko Oberdiek and the Oberdiek Package Support Group
https://github.com/ho-tex/oberdiek/issues


The Base Interpreter refers to any `TeX-Format',
because some files are installed in TDS:tex/generic//.

This work consists of the main source file eolgrab.dtx
and the derived files
   eolgrab.sty, eolgrab.pdf, eolgrab.ins, eolgrab.drv,
   eolgrab-example-ltx.tex, eolgrab-example-env.tex,
   eolgrab-example-sec.tex, eolgrab-test1.tex,
   eolgrab-test2.tex.

\endpreamble
\let\MetaPrefix\DoubleperCent

\generate{%
  \file{eolgrab.ins}{\from{eolgrab.dtx}{install}}%
  \file{eolgrab.drv}{\from{eolgrab.dtx}{driver}}%
  \usedir{tex/generic/oberdiek}%
  \file{eolgrab.sty}{\from{eolgrab.dtx}{package}}%
  \usedir{doc/latex/oberdiek/example}%
  \file{eolgrab-example-ltx.tex}{\from{eolgrab.dtx}{example-ltx}}%
  \file{eolgrab-example-env.tex}{\from{eolgrab.dtx}{example-env}}%
  \file{eolgrab-example-sec.tex}{\from{eolgrab.dtx}{example-sec}}%
%  \usedir{doc/latex/oberdiek/test}%
%  \file{eolgrab-test1.tex}{\from{eolgrab.dtx}{test1}}%
%  \file{eolgrab-test2.tex}{\from{eolgrab.dtx}{test2}}%
}

\catcode32=13\relax% active space
\let =\space%
\Msg{************************************************************************}
\Msg{*}
\Msg{* To finish the installation you have to move the following}
\Msg{* file into a directory searched by TeX:}
\Msg{*}
\Msg{*     eolgrab.sty}
\Msg{*}
\Msg{* To produce the documentation run the file `eolgrab.drv'}
\Msg{* through LaTeX.}
\Msg{*}
\Msg{* Happy TeXing!}
\Msg{*}
\Msg{************************************************************************}

\endbatchfile
%</install>
%<*ignore>
\fi
%</ignore>
%<*driver>
\NeedsTeXFormat{LaTeX2e}
\ProvidesFile{eolgrab.drv}%
  [2016/05/16 v1.1 Catch arguments delimited by end of line (HO)]%
\documentclass{ltxdoc}
\usepackage{holtxdoc}[2011/11/22]
\begin{document}
  \DocInput{eolgrab.dtx}%
\end{document}
%</driver>
% \fi
%
%
%
% \GetFileInfo{eolgrab.drv}
%
% \title{The \xpackage{eolgrab} package}
% \date{2016/05/16 v1.1}
% \author{Heiko Oberdiek\thanks
% {Please report any issues at \url{https://github.com/ho-tex/oberdiek/issues}}}
%
% \maketitle
%
% \begin{abstract}
% This package implements a generic argument grabber
% to catch an argument that is delimited by the line end.
% \end{abstract}
%
% \tableofcontents
%
% \urldef\urltexhax\url{https://tug.org/mailman/listinfo/texhax}
% \def\eol/{^^A
%   \meta{\texttt{EOL}}^^A
% }
%
% \section{Documentation}
%
% The starting point for this package was a feature
% request of Arno Trautmann in the mailing list
% \texttt{texhax}\footnote{Info page for mailing list
% \texttt{texhax}: \urltexhax} \cite{trautmann-request}.
% A macro \cs{eolsection} should behave like \cs{section},
% but the argument should be delimited by the line end
% instead of given in curly braces:
% \begin{quote}
%   |\eolsection My Title|
% \end{quote}
% Phil Taylor answered this with an implementation for \cs{eolsection}.
% Because this feature could be useful for other macros as well,
% I answered with an implementation of \cs{eolgrab} as general solution
% \cite{oberdiek-eolgrab}.
%
% Both formats \hologo{plainTeX} and \hologo{LaTeX} are supported
% by the package, see the example for \cs{eolsection} below.
%
% \begin{declcs}{eolgrab} \M{code} \meta{argument} \eol/
% \end{declcs}
% Macro \cs{eolgrab} takes two arguments. The first argument
% is \meta{code}, a classical undelimited \hologo{TeX} macro argument.
% The second argument is delimited by the line end \eol/.
% The macro calls \meta{code} with \meta{argument} as argument
% in curly braces. Because the catcode of the line end is changed,
% \cs{eolgrab} will not work in the argument of other macros.
% Macro \cs{eolgrab} is made robust if either \hologo{eTeX}'s
% \cs{protected} or \hologo{LaTeX}'s \cs{DeclareRobustCommand}
% is available.
%
% \begin{declcs}{eolgrabopt} \M{code} \meta{argument} \eol/
% \end{declcs}
% Macro \cs{eolgrabopt} passes \meta{argument} as optional
% argument to \meta{code} if \meta{argument} is not empty.
% \begin{quote}
%   |\eolgrabopt\item foo|
% \end{quote}
% becomes to
% \begin{quote}
%   |\item[{foo}]|
% \end{quote}
% The curly argument braces are added to support square brackets
% inside \meta{argument}.
% If the \meta{argument} is empty:
% \begin{quote}
%   |\eolgrabopt\item|
% \end{quote}
% then
% \begin{quote}
%   |\item|
% \end{quote}
% is called without optional argument.
%
% \subsection{Examples}
%
% \begin{itemize}
% \item
%   The line
%   \begin{quote}
%     |\eolgrab\section My Title|
%   \end{quote}
%   is equivalent to
%   \begin{quote}
%     |\section{My Title}|
%   \end{quote}
% \item
%   The next example uses the star form of \cs{section}. Then the
%   command to be called consists of two tokens. Therefore the
%   first argument of \cs{eolgrab} needs curly braces:
%   \begin{quote}
%     |\eolgrab{\section*}My Title|
%   \end{quote}
%   becomes
%   \begin{quote}
%     |\section*{My Title}|
%   \end{quote}
% \item
%   Now \hologo{LaTeX}'s \cs{PackageError} is used. This macro
%   has three arguments, the package or class name, the message
%   text and the help text. A standard help text of \hologo{LaTeX}
%   is used as given in macro \cs{@ehc}. The second argument,
%   the message text is used as argument, delimited by line end:
%   \begin{quote}
%     |\eolgrab{\PackageError{foobar}}%|\\
%     |Some error message text\MessageBreak%|\\
%     |with several lines|\\
%     |\@ehc|
%   \end{quote}
%   In the first two lines of the example, the line end is suppressed
%   by the comment character (percent), thus the argument is delimited
%   by the line end of the third line. The result is:
%   \begin{quote}
%     |\PackageError{foobar}{Some error message text\MessageBreak|\\
%     |with several lines}\@ehc|
%   \end{quote}
% \item
%   The original request for macro \cs{eolsection}, see above,
%   can be implemented easily with the help of \cs{eolgrab}.
%   Example for \hologo{LaTeX}:
%   \begin{quote}
%     |\usepackage{eolgrab}|\\
%     |\newcommand*{\eolsection}{\eolgrab\section}|
%   \end{quote}
%   Example for \hologo{plainTeX}:
%   \begin{quote}
%     |\input eolgrab.sty\relax|\\
%     |\def\eolsection{\eolgrab\section}|
%   \end{quote}
%   And a sophisticated variant for \hologo{LaTeX} that also supports
%   the star syntax and the optional argument:
%    \begin{macrocode}
%<*example-sec>
\documentclass{article}
\usepackage{eolgrab}
\makeatletter
\newcommand*{\eolsection}{%
  \@ifstar{%
    \eolgrab{\section*}%
  }{%
    \@ifnextchar[{%
      \eoloptsection
    }{%
      \eolgrab\section
    }%
  }%
}
\newcommand*{\eoloptsection}[1][]{%
  \eolgrab{\section[{#1}]}%
}
\makeatother
\begin{document}
\tableofcontents
\eolsection Section without star and optional argument
\eolsection*Section with star
\eolsection[Short section title]Long section title
\end{document}
%</example-sec>
%    \end{macrocode}
% \end{itemize}
%
% \subsubsection{Small \hologo{LaTeX} document as example}
%
%    \begin{macrocode}
%<*example-ltx>
\RequirePackage{eolgrab}
\eolgrab\documentclass article
\eolgrab\begin document
\eolgrab\section Hello World
\eolgrab\emph Some text
\eolgrab\end document
%</example-ltx>
%    \end{macrocode}
%
% \subsubsection{\hologo{LaTeX} document with environments}
%
%    \begin{macrocode}
%<*example-env>
\documentclass{article}
\usepackage{eolgrab}
\newcommand*{\Begin}{\eolgrab   \begin}
\newcommand*{\End  }{\eolgrab   \end  }
\newcommand*{\Item }{\eolgrabopt\item }
\Begin document
  \Begin itemize
    \Item
      first item
    \Item
      second item
  \End itemize
  \Begin description
    \Item foo
      is the first syllable of foobar.
    \Item bar
      is the second syllable of foobar.
  \End description
\End document
%</example-env>
%    \end{macrocode}
%
% \subsection{Limitations}
%
% Macro \cs{eolgrab} needs to catch the line end.
% If \hologo{TeX} reads a line, then it throws away the line end
% characters (carriage return, line feed) and removes spaces at the
% end of the line. Then it adds the character with the character code
% that is given by \cs{endlinechar} at the end of the line. The category
% code of the inserted character is given by the current value of
% its \cs{catcode}. If \cs{endlinechar} is not a valid character code
% (especially if it is negative), then no character is added.
%
% In \hologo{plainTeX} and \hologo{LaTeX} the standard settings
% of the inserted endline character is the character with code 13
% (or |^||^||M| in \hologo{TeX} notation) with catcode 5 (end of line).
% That means the inserted end of line character behaves like a space
% token. For example, it is removed after macro names. Therefore
% \cs{eolgrab} changes the catcode.
%
% Therefore \cs{eolgrab} has some limitations:
% \begin{itemize}
% \item
%   Like other verbatim stuff, the macro \cs{eolgrab} cannot be
%   used in the argument of other macros. \cs{eolgrab} want to
%   change the catcode of the end of line character. If this
%   character is read before, because it is processed as argument
%   of another macro, the catcode is already set and is not reassigned
%   later if \cs{eolgrab} changes the category code for this character
%   code.
% \item
%   The argument must not contain the end of line character.
%   Otherwise the first end of line character is already taken
%   as delimiter, leaving the rest of the line outside the argument.
% \item
%   Because \cs{eolgrab} is probably mostly used in the line with
%   the delimited argument. Therefore changes of \cs{endlinechar}
%   will not affect the current line.
% \end{itemize}
%
% \StopEventually{
% }
%
% \section{Implementation}
%
%    \begin{macrocode}
%<*package>
%    \end{macrocode}
%
% \subsection{Reload check and package identification}
%    Reload check, especially if the package is not used with \LaTeX.
%    \begin{macrocode}
\begingroup\catcode61\catcode48\catcode32=10\relax%
  \catcode13=5 % ^^M
  \endlinechar=13 %
  \catcode35=6 % #
  \catcode39=12 % '
  \catcode44=12 % ,
  \catcode45=12 % -
  \catcode46=12 % .
  \catcode58=12 % :
  \catcode64=11 % @
  \catcode123=1 % {
  \catcode125=2 % }
  \expandafter\let\expandafter\x\csname ver@eolgrab.sty\endcsname
  \ifx\x\relax % plain-TeX, first loading
  \else
    \def\empty{}%
    \ifx\x\empty % LaTeX, first loading,
      % variable is initialized, but \ProvidesPackage not yet seen
    \else
      \expandafter\ifx\csname PackageInfo\endcsname\relax
        \def\x#1#2{%
          \immediate\write-1{Package #1 Info: #2.}%
        }%
      \else
        \def\x#1#2{\PackageInfo{#1}{#2, stopped}}%
      \fi
      \x{eolgrab}{The package is already loaded}%
      \aftergroup\endinput
    \fi
  \fi
\endgroup%
%    \end{macrocode}
%    Package identification:
%    \begin{macrocode}
\begingroup\catcode61\catcode48\catcode32=10\relax%
  \catcode13=5 % ^^M
  \endlinechar=13 %
  \catcode35=6 % #
  \catcode39=12 % '
  \catcode40=12 % (
  \catcode41=12 % )
  \catcode44=12 % ,
  \catcode45=12 % -
  \catcode46=12 % .
  \catcode47=12 % /
  \catcode58=12 % :
  \catcode64=11 % @
  \catcode91=12 % [
  \catcode93=12 % ]
  \catcode123=1 % {
  \catcode125=2 % }
  \expandafter\ifx\csname ProvidesPackage\endcsname\relax
    \def\x#1#2#3[#4]{\endgroup
      \immediate\write-1{Package: #3 #4}%
      \xdef#1{#4}%
    }%
  \else
    \def\x#1#2[#3]{\endgroup
      #2[{#3}]%
      \ifx#1\@undefined
        \xdef#1{#3}%
      \fi
      \ifx#1\relax
        \xdef#1{#3}%
      \fi
    }%
  \fi
\expandafter\x\csname ver@eolgrab.sty\endcsname
\ProvidesPackage{eolgrab}%
  [2016/05/16 v1.1 Catch arguments delimited by end of line (HO)]%
%    \end{macrocode}
%
% \subsection{Catcodes}
%
%    \begin{macrocode}
\begingroup\catcode61\catcode48\catcode32=10\relax%
  \catcode13=5 % ^^M
  \endlinechar=13 %
  \catcode123=1 % {
  \catcode125=2 % }
  \catcode64=11 % @
  \def\x{\endgroup
    \expandafter\edef\csname eolgrab@AtEnd\endcsname{%
      \endlinechar=\the\endlinechar\relax
      \catcode13=\the\catcode13\relax
      \catcode32=\the\catcode32\relax
      \catcode35=\the\catcode35\relax
      \catcode61=\the\catcode61\relax
      \catcode64=\the\catcode64\relax
      \catcode123=\the\catcode123\relax
      \catcode125=\the\catcode125\relax
    }%
  }%
\x\catcode61\catcode48\catcode32=10\relax%
\catcode13=5 % ^^M
\endlinechar=13 %
\catcode35=6 % #
\catcode64=11 % @
\catcode123=1 % {
\catcode125=2 % }
\def\TMP@EnsureCode#1#2{%
  \edef\eolgrab@AtEnd{%
    \eolgrab@AtEnd
    \catcode#1=\the\catcode#1\relax
  }%
  \catcode#1=#2\relax
}
\TMP@EnsureCode{40}{12}% (
\TMP@EnsureCode{41}{12}% )
\TMP@EnsureCode{42}{12}% *
\TMP@EnsureCode{46}{12}% .
\TMP@EnsureCode{47}{12}% /
\TMP@EnsureCode{91}{12}% [
\TMP@EnsureCode{93}{12}% ]
\TMP@EnsureCode{94}{7}% ^
\edef\eolgrab@AtEnd{\eolgrab@AtEnd\noexpand\endinput}
%    \end{macrocode}
%
% \subsection{Resources}
%
%    \begin{macrocode}
\begingroup\expandafter\expandafter\expandafter\endgroup
\expandafter\ifx\csname RequirePackage\endcsname\relax
  \input ltxcmds.sty\relax
  \input infwarerr.sty\relax
\else
  \RequirePackage{ltxcmds}[2010/12/04]%
  \RequirePackage{infwarerr}[2010/04/08]%
\fi
%    \end{macrocode}
%
%    \begin{macro}{\eolgrab@ifdefinable}
%    \begin{macrocode}
\ltx@IfUndefined{@ifdefinable}{%
  \def\eolgrab@ifdefinable#1#2{%
    \ltx@ifundefined{#1}{#2}{%
      \@PackageError{eolgrab}{%
        Command \ltx@backslashchar#1 already defined%
      }\@ehc
    }%
  }%
}{%
  \def\eolgrab@ifdefinable#1{%
    \expandafter\@ifdefinable\csname#1\endcsname
  }%
}
%    \end{macrocode}
%    \end{macro}
%
% \subsection{Macro \cs{eolgrab}}
%
%    \begin{macro}{\eolgrab}
%    \begin{macrocode}
\eolgrab@ifdefinable{eolgrab}{%
  \ltx@IfUndefined{protected}{%
    \ltx@IfUndefined{DeclareRobustCommand}{%
      \def\eolgrab#1%
    }{%
      \newcommand\eolgrab{}%
      \DeclareRobustCommand*\eolgrab
    }%
  }{%
    \protected\def\eolgrab#1%
  }{%
    \begingroup
    \endlinechar=13 %
    \catcode13=\ltx@active
    \eolgrab@{#1}%
  }%
}
%    \end{macrocode}
%    \end{macro}
%
%    \begin{macro}{\eolgrabopt}
%    \begin{macrocode}
\eolgrab@ifdefinable{eolgrabopt}{%
  \ltx@IfUndefined{protected}{%
    \ltx@IfUndefined{DeclareRobustCommand}{%
      \def\eolgrabopt#1%
    }{%
      \newcommand\eolgrabopt{}%
      \DeclareRobustCommand*\eolgrabopt
    }%
  }{%
    \protected\def\eolgrabopt#1%
  }{%
    \begingroup
    \endlinechar=13 %
    \catcode13=\ltx@active
    \eolgrab@opt{#1}%
  }%
}
%    \end{macrocode}
%    \end{macro}
%
%    \begin{macrocode}
\begingroup
  \catcode13=\ltx@active %
\ltx@firstofone{\endgroup %
%    \end{macrocode}
%    \begin{macro}{\eolgrab@}
%    \begin{macrocode}
  \def\eolgrab@#1#2^^M{%
    \endgroup %
    #1{#2}%
  }%
%    \end{macrocode}
%    \end{macro}
%    \begin{macro}{\eolgrab@opt}
%    \begin{macrocode}
  \def\eolgrab@opt#1#2^^M{%
    \endgroup %
    \ltx@ifempty{#2}{%
      #1%
    }{%
      #1[{#2}]%
    }%
  }%
%    \end{macrocode}
%    \end{macro}
%    \begin{macrocode}
}%
%    \end{macrocode}
%
%    \begin{macrocode}
\eolgrab@AtEnd%
%    \end{macrocode}
%    \begin{macrocode}
%</package>
%    \end{macrocode}
%% \section{Installation}
%
% \subsection{Download}
%
% \paragraph{Package.} This package is available on
% CTAN\footnote{\CTANpkg{eolgrab}}:
% \begin{description}
% \item[\CTAN{macros/latex/contrib/oberdiek/eolgrab.dtx}] The source file.
% \item[\CTAN{macros/latex/contrib/oberdiek/eolgrab.pdf}] Documentation.
% \end{description}
%
%
% \paragraph{Bundle.} All the packages of the bundle `oberdiek'
% are also available in a TDS compliant ZIP archive. There
% the packages are already unpacked and the documentation files
% are generated. The files and directories obey the TDS standard.
% \begin{description}
% \item[\CTANinstall{install/macros/latex/contrib/oberdiek.tds.zip}]
% \end{description}
% \emph{TDS} refers to the standard ``A Directory Structure
% for \TeX\ Files'' (\CTANpkg{tds}). Directories
% with \xfile{texmf} in their name are usually organized this way.
%
% \subsection{Bundle installation}
%
% \paragraph{Unpacking.} Unpack the \xfile{oberdiek.tds.zip} in the
% TDS tree (also known as \xfile{texmf} tree) of your choice.
% Example (linux):
% \begin{quote}
%   |unzip oberdiek.tds.zip -d ~/texmf|
% \end{quote}
%
% \subsection{Package installation}
%
% \paragraph{Unpacking.} The \xfile{.dtx} file is a self-extracting
% \docstrip\ archive. The files are extracted by running the
% \xfile{.dtx} through \plainTeX:
% \begin{quote}
%   \verb|tex eolgrab.dtx|
% \end{quote}
%
% \paragraph{TDS.} Now the different files must be moved into
% the different directories in your installation TDS tree
% (also known as \xfile{texmf} tree):
% \begin{quote}
% \def\t{^^A
% \begin{tabular}{@{}>{\ttfamily}l@{ $\rightarrow$ }>{\ttfamily}l@{}}
%   eolgrab.sty & tex/generic/oberdiek/eolgrab.sty\\
%   eolgrab.pdf & doc/latex/oberdiek/eolgrab.pdf\\
%   example/eolgrab-example-ltx.tex & doc/latex/oberdiek/example/eolgrab-example-ltx.tex\\
%   example/eolgrab-example-env.tex & doc/latex/oberdiek/example/eolgrab-example-env.tex\\
%   example/eolgrab-example-sec.tex & doc/latex/oberdiek/example/eolgrab-example-sec.tex\\
%   eolgrab.dtx & source/latex/oberdiek/eolgrab.dtx\\
% \end{tabular}^^A
% }^^A
% \sbox0{\t}^^A
% \ifdim\wd0>\linewidth
%   \begingroup
%     \advance\linewidth by\leftmargin
%     \advance\linewidth by\rightmargin
%   \edef\x{\endgroup
%     \def\noexpand\lw{\the\linewidth}^^A
%   }\x
%   \def\lwbox{^^A
%     \leavevmode
%     \hbox to \linewidth{^^A
%       \kern-\leftmargin\relax
%       \hss
%       \usebox0
%       \hss
%       \kern-\rightmargin\relax
%     }^^A
%   }^^A
%   \ifdim\wd0>\lw
%     \sbox0{\small\t}^^A
%     \ifdim\wd0>\linewidth
%       \ifdim\wd0>\lw
%         \sbox0{\footnotesize\t}^^A
%         \ifdim\wd0>\linewidth
%           \ifdim\wd0>\lw
%             \sbox0{\scriptsize\t}^^A
%             \ifdim\wd0>\linewidth
%               \ifdim\wd0>\lw
%                 \sbox0{\tiny\t}^^A
%                 \ifdim\wd0>\linewidth
%                   \lwbox
%                 \else
%                   \usebox0
%                 \fi
%               \else
%                 \lwbox
%               \fi
%             \else
%               \usebox0
%             \fi
%           \else
%             \lwbox
%           \fi
%         \else
%           \usebox0
%         \fi
%       \else
%         \lwbox
%       \fi
%     \else
%       \usebox0
%     \fi
%   \else
%     \lwbox
%   \fi
% \else
%   \usebox0
% \fi
% \end{quote}
% If you have a \xfile{docstrip.cfg} that configures and enables \docstrip's
% TDS installing feature, then some files can already be in the right
% place, see the documentation of \docstrip.
%
% \subsection{Refresh file name databases}
%
% If your \TeX~distribution
% (\TeX\,Live, \mikTeX, \dots) relies on file name databases, you must refresh
% these. For example, \TeX\,Live\ users run \verb|texhash| or
% \verb|mktexlsr|.
%
% \subsection{Some details for the interested}
%
% \paragraph{Unpacking with \LaTeX.}
% The \xfile{.dtx} chooses its action depending on the format:
% \begin{description}
% \item[\plainTeX:] Run \docstrip\ and extract the files.
% \item[\LaTeX:] Generate the documentation.
% \end{description}
% If you insist on using \LaTeX\ for \docstrip\ (really,
% \docstrip\ does not need \LaTeX), then inform the autodetect routine
% about your intention:
% \begin{quote}
%   \verb|latex \let\install=y\input{eolgrab.dtx}|
% \end{quote}
% Do not forget to quote the argument according to the demands
% of your shell.
%
% \paragraph{Generating the documentation.}
% You can use both the \xfile{.dtx} or the \xfile{.drv} to generate
% the documentation. The process can be configured by the
% configuration file \xfile{ltxdoc.cfg}. For instance, put this
% line into this file, if you want to have A4 as paper format:
% \begin{quote}
%   \verb|\PassOptionsToClass{a4paper}{article}|
% \end{quote}
% An example follows how to generate the
% documentation with pdf\LaTeX:
% \begin{quote}
%\begin{verbatim}
%pdflatex eolgrab.dtx
%makeindex -s gind.ist eolgrab.idx
%pdflatex eolgrab.dtx
%makeindex -s gind.ist eolgrab.idx
%pdflatex eolgrab.dtx
%\end{verbatim}
% \end{quote}
%
% \begin{thebibliography}{9}
% \raggedright
% \bibitem{trautmann-request}
% Arno Trautmann,
% \textit{[texhax] read argument until EOL};
% mailing list \texttt{texthax@tug.org},
% 2011-01-06;
% \url{https://tug.org/pipermail/texhax/2011-January/016517.html}.
%
% \bibitem{taylor}
% Philip Taylor,
% \textit{Re: [texhax] read argument until EOL};
% mailing list \texttt{texhax@tug.org},
% 2011-01-06;
% \url{https://tug.org/pipermail/texhax/2011-January/016519.html}.
%
% \bibitem{oberdiek-eolgrab}
% Heiko Oberdiek,
% \textit{Re: [texhax] read argument until EOL};
% mailing list \texttt{texhax@tug.org},
% 2011-01-06;
% \url{https://tug.org/pipermail/texhax/2011-January/016526.html}.
%
% \end{thebibliography}
%
% \begin{History}
%   \begin{Version}{2011/01/12 v1.0}
%   \item
%     First public version.
%   \end{Version}
%   \begin{Version}{2016/05/16 v1.1}
%   \item
%     Documentation updates.
%   \end{Version}
% \end{History}
%
% \PrintIndex
%
% \Finale
\endinput

%        (quote the arguments according to the demands of your shell)
%
% Documentation:
%    (a) If eolgrab.drv is present:
%           latex eolgrab.drv
%    (b) Without eolgrab.drv:
%           latex eolgrab.dtx; ...
%    The class ltxdoc loads the configuration file ltxdoc.cfg
%    if available. Here you can specify further options, e.g.
%    use A4 as paper format:
%       \PassOptionsToClass{a4paper}{article}
%
%    Programm calls to get the documentation (example):
%       pdflatex eolgrab.dtx
%       makeindex -s gind.ist eolgrab.idx
%       pdflatex eolgrab.dtx
%       makeindex -s gind.ist eolgrab.idx
%       pdflatex eolgrab.dtx
%
% Installation:
%    TDS:tex/generic/oberdiek/eolgrab.sty
%    TDS:doc/latex/oberdiek/eolgrab.pdf
%    TDS:doc/latex/oberdiek/example/eolgrab-example-ltx.tex
%    TDS:doc/latex/oberdiek/example/eolgrab-example-env.tex
%    TDS:doc/latex/oberdiek/example/eolgrab-example-sec.tex
%    TDS:source/latex/oberdiek/eolgrab.dtx
%
%<*ignore>
\begingroup
  \catcode123=1 %
  \catcode125=2 %
  \def\x{LaTeX2e}%
\expandafter\endgroup
\ifcase 0\ifx\install y1\fi\expandafter
         \ifx\csname processbatchFile\endcsname\relax\else1\fi
         \ifx\fmtname\x\else 1\fi\relax
\else\csname fi\endcsname
%</ignore>
%<*install>
\input docstrip.tex
\Msg{************************************************************************}
\Msg{* Installation}
\Msg{* Package: eolgrab 2016/05/16 v1.1 Catch arguments delimited by end of line (HO)}
\Msg{************************************************************************}

\keepsilent
\askforoverwritefalse

\let\MetaPrefix\relax
\preamble

This is a generated file.

Project: eolgrab
Version: 2016/05/16 v1.1

Copyright (C)
   2011 Heiko Oberdiek
   2016-2019 Oberdiek Package Support Group

This work may be distributed and/or modified under the
conditions of the LaTeX Project Public License, either
version 1.3c of this license or (at your option) any later
version. This version of this license is in
   https://www.latex-project.org/lppl/lppl-1-3c.txt
and the latest version of this license is in
   https://www.latex-project.org/lppl.txt
and version 1.3 or later is part of all distributions of
LaTeX version 2005/12/01 or later.

This work has the LPPL maintenance status "maintained".

The Current Maintainers of this work are
Heiko Oberdiek and the Oberdiek Package Support Group
https://github.com/ho-tex/oberdiek/issues


The Base Interpreter refers to any `TeX-Format',
because some files are installed in TDS:tex/generic//.

This work consists of the main source file eolgrab.dtx
and the derived files
   eolgrab.sty, eolgrab.pdf, eolgrab.ins, eolgrab.drv,
   eolgrab-example-ltx.tex, eolgrab-example-env.tex,
   eolgrab-example-sec.tex, eolgrab-test1.tex,
   eolgrab-test2.tex.

\endpreamble
\let\MetaPrefix\DoubleperCent

\generate{%
  \file{eolgrab.ins}{\from{eolgrab.dtx}{install}}%
  \file{eolgrab.drv}{\from{eolgrab.dtx}{driver}}%
  \usedir{tex/generic/oberdiek}%
  \file{eolgrab.sty}{\from{eolgrab.dtx}{package}}%
  \usedir{doc/latex/oberdiek/example}%
  \file{eolgrab-example-ltx.tex}{\from{eolgrab.dtx}{example-ltx}}%
  \file{eolgrab-example-env.tex}{\from{eolgrab.dtx}{example-env}}%
  \file{eolgrab-example-sec.tex}{\from{eolgrab.dtx}{example-sec}}%
%  \usedir{doc/latex/oberdiek/test}%
%  \file{eolgrab-test1.tex}{\from{eolgrab.dtx}{test1}}%
%  \file{eolgrab-test2.tex}{\from{eolgrab.dtx}{test2}}%
}

\catcode32=13\relax% active space
\let =\space%
\Msg{************************************************************************}
\Msg{*}
\Msg{* To finish the installation you have to move the following}
\Msg{* file into a directory searched by TeX:}
\Msg{*}
\Msg{*     eolgrab.sty}
\Msg{*}
\Msg{* To produce the documentation run the file `eolgrab.drv'}
\Msg{* through LaTeX.}
\Msg{*}
\Msg{* Happy TeXing!}
\Msg{*}
\Msg{************************************************************************}

\endbatchfile
%</install>
%<*ignore>
\fi
%</ignore>
%<*driver>
\NeedsTeXFormat{LaTeX2e}
\ProvidesFile{eolgrab.drv}%
  [2016/05/16 v1.1 Catch arguments delimited by end of line (HO)]%
\documentclass{ltxdoc}
\usepackage{holtxdoc}[2011/11/22]
\begin{document}
  \DocInput{eolgrab.dtx}%
\end{document}
%</driver>
% \fi
%
%
%
% \GetFileInfo{eolgrab.drv}
%
% \title{The \xpackage{eolgrab} package}
% \date{2016/05/16 v1.1}
% \author{Heiko Oberdiek\thanks
% {Please report any issues at \url{https://github.com/ho-tex/oberdiek/issues}}}
%
% \maketitle
%
% \begin{abstract}
% This package implements a generic argument grabber
% to catch an argument that is delimited by the line end.
% \end{abstract}
%
% \tableofcontents
%
% \urldef\urltexhax\url{https://tug.org/mailman/listinfo/texhax}
% \def\eol/{^^A
%   \meta{\texttt{EOL}}^^A
% }
%
% \section{Documentation}
%
% The starting point for this package was a feature
% request of Arno Trautmann in the mailing list
% \texttt{texhax}\footnote{Info page for mailing list
% \texttt{texhax}: \urltexhax} \cite{trautmann-request}.
% A macro \cs{eolsection} should behave like \cs{section},
% but the argument should be delimited by the line end
% instead of given in curly braces:
% \begin{quote}
%   |\eolsection My Title|
% \end{quote}
% Phil Taylor answered this with an implementation for \cs{eolsection}.
% Because this feature could be useful for other macros as well,
% I answered with an implementation of \cs{eolgrab} as general solution
% \cite{oberdiek-eolgrab}.
%
% Both formats \hologo{plainTeX} and \hologo{LaTeX} are supported
% by the package, see the example for \cs{eolsection} below.
%
% \begin{declcs}{eolgrab} \M{code} \meta{argument} \eol/
% \end{declcs}
% Macro \cs{eolgrab} takes two arguments. The first argument
% is \meta{code}, a classical undelimited \hologo{TeX} macro argument.
% The second argument is delimited by the line end \eol/.
% The macro calls \meta{code} with \meta{argument} as argument
% in curly braces. Because the catcode of the line end is changed,
% \cs{eolgrab} will not work in the argument of other macros.
% Macro \cs{eolgrab} is made robust if either \hologo{eTeX}'s
% \cs{protected} or \hologo{LaTeX}'s \cs{DeclareRobustCommand}
% is available.
%
% \begin{declcs}{eolgrabopt} \M{code} \meta{argument} \eol/
% \end{declcs}
% Macro \cs{eolgrabopt} passes \meta{argument} as optional
% argument to \meta{code} if \meta{argument} is not empty.
% \begin{quote}
%   |\eolgrabopt\item foo|
% \end{quote}
% becomes to
% \begin{quote}
%   |\item[{foo}]|
% \end{quote}
% The curly argument braces are added to support square brackets
% inside \meta{argument}.
% If the \meta{argument} is empty:
% \begin{quote}
%   |\eolgrabopt\item|
% \end{quote}
% then
% \begin{quote}
%   |\item|
% \end{quote}
% is called without optional argument.
%
% \subsection{Examples}
%
% \begin{itemize}
% \item
%   The line
%   \begin{quote}
%     |\eolgrab\section My Title|
%   \end{quote}
%   is equivalent to
%   \begin{quote}
%     |\section{My Title}|
%   \end{quote}
% \item
%   The next example uses the star form of \cs{section}. Then the
%   command to be called consists of two tokens. Therefore the
%   first argument of \cs{eolgrab} needs curly braces:
%   \begin{quote}
%     |\eolgrab{\section*}My Title|
%   \end{quote}
%   becomes
%   \begin{quote}
%     |\section*{My Title}|
%   \end{quote}
% \item
%   Now \hologo{LaTeX}'s \cs{PackageError} is used. This macro
%   has three arguments, the package or class name, the message
%   text and the help text. A standard help text of \hologo{LaTeX}
%   is used as given in macro \cs{@ehc}. The second argument,
%   the message text is used as argument, delimited by line end:
%   \begin{quote}
%     |\eolgrab{\PackageError{foobar}}%|\\
%     |Some error message text\MessageBreak%|\\
%     |with several lines|\\
%     |\@ehc|
%   \end{quote}
%   In the first two lines of the example, the line end is suppressed
%   by the comment character (percent), thus the argument is delimited
%   by the line end of the third line. The result is:
%   \begin{quote}
%     |\PackageError{foobar}{Some error message text\MessageBreak|\\
%     |with several lines}\@ehc|
%   \end{quote}
% \item
%   The original request for macro \cs{eolsection}, see above,
%   can be implemented easily with the help of \cs{eolgrab}.
%   Example for \hologo{LaTeX}:
%   \begin{quote}
%     |\usepackage{eolgrab}|\\
%     |\newcommand*{\eolsection}{\eolgrab\section}|
%   \end{quote}
%   Example for \hologo{plainTeX}:
%   \begin{quote}
%     |\input eolgrab.sty\relax|\\
%     |\def\eolsection{\eolgrab\section}|
%   \end{quote}
%   And a sophisticated variant for \hologo{LaTeX} that also supports
%   the star syntax and the optional argument:
%    \begin{macrocode}
%<*example-sec>
\documentclass{article}
\usepackage{eolgrab}
\makeatletter
\newcommand*{\eolsection}{%
  \@ifstar{%
    \eolgrab{\section*}%
  }{%
    \@ifnextchar[{%
      \eoloptsection
    }{%
      \eolgrab\section
    }%
  }%
}
\newcommand*{\eoloptsection}[1][]{%
  \eolgrab{\section[{#1}]}%
}
\makeatother
\begin{document}
\tableofcontents
\eolsection Section without star and optional argument
\eolsection*Section with star
\eolsection[Short section title]Long section title
\end{document}
%</example-sec>
%    \end{macrocode}
% \end{itemize}
%
% \subsubsection{Small \hologo{LaTeX} document as example}
%
%    \begin{macrocode}
%<*example-ltx>
\RequirePackage{eolgrab}
\eolgrab\documentclass article
\eolgrab\begin document
\eolgrab\section Hello World
\eolgrab\emph Some text
\eolgrab\end document
%</example-ltx>
%    \end{macrocode}
%
% \subsubsection{\hologo{LaTeX} document with environments}
%
%    \begin{macrocode}
%<*example-env>
\documentclass{article}
\usepackage{eolgrab}
\newcommand*{\Begin}{\eolgrab   \begin}
\newcommand*{\End  }{\eolgrab   \end  }
\newcommand*{\Item }{\eolgrabopt\item }
\Begin document
  \Begin itemize
    \Item
      first item
    \Item
      second item
  \End itemize
  \Begin description
    \Item foo
      is the first syllable of foobar.
    \Item bar
      is the second syllable of foobar.
  \End description
\End document
%</example-env>
%    \end{macrocode}
%
% \subsection{Limitations}
%
% Macro \cs{eolgrab} needs to catch the line end.
% If \hologo{TeX} reads a line, then it throws away the line end
% characters (carriage return, line feed) and removes spaces at the
% end of the line. Then it adds the character with the character code
% that is given by \cs{endlinechar} at the end of the line. The category
% code of the inserted character is given by the current value of
% its \cs{catcode}. If \cs{endlinechar} is not a valid character code
% (especially if it is negative), then no character is added.
%
% In \hologo{plainTeX} and \hologo{LaTeX} the standard settings
% of the inserted endline character is the character with code 13
% (or |^||^||M| in \hologo{TeX} notation) with catcode 5 (end of line).
% That means the inserted end of line character behaves like a space
% token. For example, it is removed after macro names. Therefore
% \cs{eolgrab} changes the catcode.
%
% Therefore \cs{eolgrab} has some limitations:
% \begin{itemize}
% \item
%   Like other verbatim stuff, the macro \cs{eolgrab} cannot be
%   used in the argument of other macros. \cs{eolgrab} want to
%   change the catcode of the end of line character. If this
%   character is read before, because it is processed as argument
%   of another macro, the catcode is already set and is not reassigned
%   later if \cs{eolgrab} changes the category code for this character
%   code.
% \item
%   The argument must not contain the end of line character.
%   Otherwise the first end of line character is already taken
%   as delimiter, leaving the rest of the line outside the argument.
% \item
%   Because \cs{eolgrab} is probably mostly used in the line with
%   the delimited argument. Therefore changes of \cs{endlinechar}
%   will not affect the current line.
% \end{itemize}
%
% \StopEventually{
% }
%
% \section{Implementation}
%
%    \begin{macrocode}
%<*package>
%    \end{macrocode}
%
% \subsection{Reload check and package identification}
%    Reload check, especially if the package is not used with \LaTeX.
%    \begin{macrocode}
\begingroup\catcode61\catcode48\catcode32=10\relax%
  \catcode13=5 % ^^M
  \endlinechar=13 %
  \catcode35=6 % #
  \catcode39=12 % '
  \catcode44=12 % ,
  \catcode45=12 % -
  \catcode46=12 % .
  \catcode58=12 % :
  \catcode64=11 % @
  \catcode123=1 % {
  \catcode125=2 % }
  \expandafter\let\expandafter\x\csname ver@eolgrab.sty\endcsname
  \ifx\x\relax % plain-TeX, first loading
  \else
    \def\empty{}%
    \ifx\x\empty % LaTeX, first loading,
      % variable is initialized, but \ProvidesPackage not yet seen
    \else
      \expandafter\ifx\csname PackageInfo\endcsname\relax
        \def\x#1#2{%
          \immediate\write-1{Package #1 Info: #2.}%
        }%
      \else
        \def\x#1#2{\PackageInfo{#1}{#2, stopped}}%
      \fi
      \x{eolgrab}{The package is already loaded}%
      \aftergroup\endinput
    \fi
  \fi
\endgroup%
%    \end{macrocode}
%    Package identification:
%    \begin{macrocode}
\begingroup\catcode61\catcode48\catcode32=10\relax%
  \catcode13=5 % ^^M
  \endlinechar=13 %
  \catcode35=6 % #
  \catcode39=12 % '
  \catcode40=12 % (
  \catcode41=12 % )
  \catcode44=12 % ,
  \catcode45=12 % -
  \catcode46=12 % .
  \catcode47=12 % /
  \catcode58=12 % :
  \catcode64=11 % @
  \catcode91=12 % [
  \catcode93=12 % ]
  \catcode123=1 % {
  \catcode125=2 % }
  \expandafter\ifx\csname ProvidesPackage\endcsname\relax
    \def\x#1#2#3[#4]{\endgroup
      \immediate\write-1{Package: #3 #4}%
      \xdef#1{#4}%
    }%
  \else
    \def\x#1#2[#3]{\endgroup
      #2[{#3}]%
      \ifx#1\@undefined
        \xdef#1{#3}%
      \fi
      \ifx#1\relax
        \xdef#1{#3}%
      \fi
    }%
  \fi
\expandafter\x\csname ver@eolgrab.sty\endcsname
\ProvidesPackage{eolgrab}%
  [2016/05/16 v1.1 Catch arguments delimited by end of line (HO)]%
%    \end{macrocode}
%
% \subsection{Catcodes}
%
%    \begin{macrocode}
\begingroup\catcode61\catcode48\catcode32=10\relax%
  \catcode13=5 % ^^M
  \endlinechar=13 %
  \catcode123=1 % {
  \catcode125=2 % }
  \catcode64=11 % @
  \def\x{\endgroup
    \expandafter\edef\csname eolgrab@AtEnd\endcsname{%
      \endlinechar=\the\endlinechar\relax
      \catcode13=\the\catcode13\relax
      \catcode32=\the\catcode32\relax
      \catcode35=\the\catcode35\relax
      \catcode61=\the\catcode61\relax
      \catcode64=\the\catcode64\relax
      \catcode123=\the\catcode123\relax
      \catcode125=\the\catcode125\relax
    }%
  }%
\x\catcode61\catcode48\catcode32=10\relax%
\catcode13=5 % ^^M
\endlinechar=13 %
\catcode35=6 % #
\catcode64=11 % @
\catcode123=1 % {
\catcode125=2 % }
\def\TMP@EnsureCode#1#2{%
  \edef\eolgrab@AtEnd{%
    \eolgrab@AtEnd
    \catcode#1=\the\catcode#1\relax
  }%
  \catcode#1=#2\relax
}
\TMP@EnsureCode{40}{12}% (
\TMP@EnsureCode{41}{12}% )
\TMP@EnsureCode{42}{12}% *
\TMP@EnsureCode{46}{12}% .
\TMP@EnsureCode{47}{12}% /
\TMP@EnsureCode{91}{12}% [
\TMP@EnsureCode{93}{12}% ]
\TMP@EnsureCode{94}{7}% ^
\edef\eolgrab@AtEnd{\eolgrab@AtEnd\noexpand\endinput}
%    \end{macrocode}
%
% \subsection{Resources}
%
%    \begin{macrocode}
\begingroup\expandafter\expandafter\expandafter\endgroup
\expandafter\ifx\csname RequirePackage\endcsname\relax
  \input ltxcmds.sty\relax
  \input infwarerr.sty\relax
\else
  \RequirePackage{ltxcmds}[2010/12/04]%
  \RequirePackage{infwarerr}[2010/04/08]%
\fi
%    \end{macrocode}
%
%    \begin{macro}{\eolgrab@ifdefinable}
%    \begin{macrocode}
\ltx@IfUndefined{@ifdefinable}{%
  \def\eolgrab@ifdefinable#1#2{%
    \ltx@ifundefined{#1}{#2}{%
      \@PackageError{eolgrab}{%
        Command \ltx@backslashchar#1 already defined%
      }\@ehc
    }%
  }%
}{%
  \def\eolgrab@ifdefinable#1{%
    \expandafter\@ifdefinable\csname#1\endcsname
  }%
}
%    \end{macrocode}
%    \end{macro}
%
% \subsection{Macro \cs{eolgrab}}
%
%    \begin{macro}{\eolgrab}
%    \begin{macrocode}
\eolgrab@ifdefinable{eolgrab}{%
  \ltx@IfUndefined{protected}{%
    \ltx@IfUndefined{DeclareRobustCommand}{%
      \def\eolgrab#1%
    }{%
      \newcommand\eolgrab{}%
      \DeclareRobustCommand*\eolgrab
    }%
  }{%
    \protected\def\eolgrab#1%
  }{%
    \begingroup
    \endlinechar=13 %
    \catcode13=\ltx@active
    \eolgrab@{#1}%
  }%
}
%    \end{macrocode}
%    \end{macro}
%
%    \begin{macro}{\eolgrabopt}
%    \begin{macrocode}
\eolgrab@ifdefinable{eolgrabopt}{%
  \ltx@IfUndefined{protected}{%
    \ltx@IfUndefined{DeclareRobustCommand}{%
      \def\eolgrabopt#1%
    }{%
      \newcommand\eolgrabopt{}%
      \DeclareRobustCommand*\eolgrabopt
    }%
  }{%
    \protected\def\eolgrabopt#1%
  }{%
    \begingroup
    \endlinechar=13 %
    \catcode13=\ltx@active
    \eolgrab@opt{#1}%
  }%
}
%    \end{macrocode}
%    \end{macro}
%
%    \begin{macrocode}
\begingroup
  \catcode13=\ltx@active %
\ltx@firstofone{\endgroup %
%    \end{macrocode}
%    \begin{macro}{\eolgrab@}
%    \begin{macrocode}
  \def\eolgrab@#1#2^^M{%
    \endgroup %
    #1{#2}%
  }%
%    \end{macrocode}
%    \end{macro}
%    \begin{macro}{\eolgrab@opt}
%    \begin{macrocode}
  \def\eolgrab@opt#1#2^^M{%
    \endgroup %
    \ltx@ifempty{#2}{%
      #1%
    }{%
      #1[{#2}]%
    }%
  }%
%    \end{macrocode}
%    \end{macro}
%    \begin{macrocode}
}%
%    \end{macrocode}
%
%    \begin{macrocode}
\eolgrab@AtEnd%
%    \end{macrocode}
%    \begin{macrocode}
%</package>
%    \end{macrocode}
%% \section{Installation}
%
% \subsection{Download}
%
% \paragraph{Package.} This package is available on
% CTAN\footnote{\CTANpkg{eolgrab}}:
% \begin{description}
% \item[\CTAN{macros/latex/contrib/oberdiek/eolgrab.dtx}] The source file.
% \item[\CTAN{macros/latex/contrib/oberdiek/eolgrab.pdf}] Documentation.
% \end{description}
%
%
% \paragraph{Bundle.} All the packages of the bundle `oberdiek'
% are also available in a TDS compliant ZIP archive. There
% the packages are already unpacked and the documentation files
% are generated. The files and directories obey the TDS standard.
% \begin{description}
% \item[\CTANinstall{install/macros/latex/contrib/oberdiek.tds.zip}]
% \end{description}
% \emph{TDS} refers to the standard ``A Directory Structure
% for \TeX\ Files'' (\CTANpkg{tds}). Directories
% with \xfile{texmf} in their name are usually organized this way.
%
% \subsection{Bundle installation}
%
% \paragraph{Unpacking.} Unpack the \xfile{oberdiek.tds.zip} in the
% TDS tree (also known as \xfile{texmf} tree) of your choice.
% Example (linux):
% \begin{quote}
%   |unzip oberdiek.tds.zip -d ~/texmf|
% \end{quote}
%
% \subsection{Package installation}
%
% \paragraph{Unpacking.} The \xfile{.dtx} file is a self-extracting
% \docstrip\ archive. The files are extracted by running the
% \xfile{.dtx} through \plainTeX:
% \begin{quote}
%   \verb|tex eolgrab.dtx|
% \end{quote}
%
% \paragraph{TDS.} Now the different files must be moved into
% the different directories in your installation TDS tree
% (also known as \xfile{texmf} tree):
% \begin{quote}
% \def\t{^^A
% \begin{tabular}{@{}>{\ttfamily}l@{ $\rightarrow$ }>{\ttfamily}l@{}}
%   eolgrab.sty & tex/generic/oberdiek/eolgrab.sty\\
%   eolgrab.pdf & doc/latex/oberdiek/eolgrab.pdf\\
%   example/eolgrab-example-ltx.tex & doc/latex/oberdiek/example/eolgrab-example-ltx.tex\\
%   example/eolgrab-example-env.tex & doc/latex/oberdiek/example/eolgrab-example-env.tex\\
%   example/eolgrab-example-sec.tex & doc/latex/oberdiek/example/eolgrab-example-sec.tex\\
%   eolgrab.dtx & source/latex/oberdiek/eolgrab.dtx\\
% \end{tabular}^^A
% }^^A
% \sbox0{\t}^^A
% \ifdim\wd0>\linewidth
%   \begingroup
%     \advance\linewidth by\leftmargin
%     \advance\linewidth by\rightmargin
%   \edef\x{\endgroup
%     \def\noexpand\lw{\the\linewidth}^^A
%   }\x
%   \def\lwbox{^^A
%     \leavevmode
%     \hbox to \linewidth{^^A
%       \kern-\leftmargin\relax
%       \hss
%       \usebox0
%       \hss
%       \kern-\rightmargin\relax
%     }^^A
%   }^^A
%   \ifdim\wd0>\lw
%     \sbox0{\small\t}^^A
%     \ifdim\wd0>\linewidth
%       \ifdim\wd0>\lw
%         \sbox0{\footnotesize\t}^^A
%         \ifdim\wd0>\linewidth
%           \ifdim\wd0>\lw
%             \sbox0{\scriptsize\t}^^A
%             \ifdim\wd0>\linewidth
%               \ifdim\wd0>\lw
%                 \sbox0{\tiny\t}^^A
%                 \ifdim\wd0>\linewidth
%                   \lwbox
%                 \else
%                   \usebox0
%                 \fi
%               \else
%                 \lwbox
%               \fi
%             \else
%               \usebox0
%             \fi
%           \else
%             \lwbox
%           \fi
%         \else
%           \usebox0
%         \fi
%       \else
%         \lwbox
%       \fi
%     \else
%       \usebox0
%     \fi
%   \else
%     \lwbox
%   \fi
% \else
%   \usebox0
% \fi
% \end{quote}
% If you have a \xfile{docstrip.cfg} that configures and enables \docstrip's
% TDS installing feature, then some files can already be in the right
% place, see the documentation of \docstrip.
%
% \subsection{Refresh file name databases}
%
% If your \TeX~distribution
% (\TeX\,Live, \mikTeX, \dots) relies on file name databases, you must refresh
% these. For example, \TeX\,Live\ users run \verb|texhash| or
% \verb|mktexlsr|.
%
% \subsection{Some details for the interested}
%
% \paragraph{Unpacking with \LaTeX.}
% The \xfile{.dtx} chooses its action depending on the format:
% \begin{description}
% \item[\plainTeX:] Run \docstrip\ and extract the files.
% \item[\LaTeX:] Generate the documentation.
% \end{description}
% If you insist on using \LaTeX\ for \docstrip\ (really,
% \docstrip\ does not need \LaTeX), then inform the autodetect routine
% about your intention:
% \begin{quote}
%   \verb|latex \let\install=y% \iffalse meta-comment
%
% File: eolgrab.dtx
% Version: 2016/05/16 v1.1
% Info: Catch arguments delimited by end of line
%
% Copyright (C)
%    2011 Heiko Oberdiek
%    2016-2019 Oberdiek Package Support Group
%    https://github.com/ho-tex/oberdiek/issues
%
% This work may be distributed and/or modified under the
% conditions of the LaTeX Project Public License, either
% version 1.3c of this license or (at your option) any later
% version. This version of this license is in
%    https://www.latex-project.org/lppl/lppl-1-3c.txt
% and the latest version of this license is in
%    https://www.latex-project.org/lppl.txt
% and version 1.3 or later is part of all distributions of
% LaTeX version 2005/12/01 or later.
%
% This work has the LPPL maintenance status "maintained".
%
% The Current Maintainers of this work are
% Heiko Oberdiek and the Oberdiek Package Support Group
% https://github.com/ho-tex/oberdiek/issues
%
% The Base Interpreter refers to any `TeX-Format',
% because some files are installed in TDS:tex/generic//.
%
% This work consists of the main source file eolgrab.dtx
% and the derived files
%    eolgrab.sty, eolgrab.pdf, eolgrab.ins, eolgrab.drv,
%    eolgrab-example-ltx.tex, eolgrab-example-env.tex,
%    eolgrab-example-sec.tex, eolgrab-test1.tex,
%    eolgrab-test2.tex.
%
% Distribution:
%    CTAN:macros/latex/contrib/oberdiek/eolgrab.dtx
%    CTAN:macros/latex/contrib/oberdiek/eolgrab.pdf
%
% Unpacking:
%    (a) If eolgrab.ins is present:
%           tex eolgrab.ins
%    (b) Without eolgrab.ins:
%           tex eolgrab.dtx
%    (c) If you insist on using LaTeX
%           latex \let\install=y\input{eolgrab.dtx}
%        (quote the arguments according to the demands of your shell)
%
% Documentation:
%    (a) If eolgrab.drv is present:
%           latex eolgrab.drv
%    (b) Without eolgrab.drv:
%           latex eolgrab.dtx; ...
%    The class ltxdoc loads the configuration file ltxdoc.cfg
%    if available. Here you can specify further options, e.g.
%    use A4 as paper format:
%       \PassOptionsToClass{a4paper}{article}
%
%    Programm calls to get the documentation (example):
%       pdflatex eolgrab.dtx
%       makeindex -s gind.ist eolgrab.idx
%       pdflatex eolgrab.dtx
%       makeindex -s gind.ist eolgrab.idx
%       pdflatex eolgrab.dtx
%
% Installation:
%    TDS:tex/generic/oberdiek/eolgrab.sty
%    TDS:doc/latex/oberdiek/eolgrab.pdf
%    TDS:doc/latex/oberdiek/example/eolgrab-example-ltx.tex
%    TDS:doc/latex/oberdiek/example/eolgrab-example-env.tex
%    TDS:doc/latex/oberdiek/example/eolgrab-example-sec.tex
%    TDS:source/latex/oberdiek/eolgrab.dtx
%
%<*ignore>
\begingroup
  \catcode123=1 %
  \catcode125=2 %
  \def\x{LaTeX2e}%
\expandafter\endgroup
\ifcase 0\ifx\install y1\fi\expandafter
         \ifx\csname processbatchFile\endcsname\relax\else1\fi
         \ifx\fmtname\x\else 1\fi\relax
\else\csname fi\endcsname
%</ignore>
%<*install>
\input docstrip.tex
\Msg{************************************************************************}
\Msg{* Installation}
\Msg{* Package: eolgrab 2016/05/16 v1.1 Catch arguments delimited by end of line (HO)}
\Msg{************************************************************************}

\keepsilent
\askforoverwritefalse

\let\MetaPrefix\relax
\preamble

This is a generated file.

Project: eolgrab
Version: 2016/05/16 v1.1

Copyright (C)
   2011 Heiko Oberdiek
   2016-2019 Oberdiek Package Support Group

This work may be distributed and/or modified under the
conditions of the LaTeX Project Public License, either
version 1.3c of this license or (at your option) any later
version. This version of this license is in
   https://www.latex-project.org/lppl/lppl-1-3c.txt
and the latest version of this license is in
   https://www.latex-project.org/lppl.txt
and version 1.3 or later is part of all distributions of
LaTeX version 2005/12/01 or later.

This work has the LPPL maintenance status "maintained".

The Current Maintainers of this work are
Heiko Oberdiek and the Oberdiek Package Support Group
https://github.com/ho-tex/oberdiek/issues


The Base Interpreter refers to any `TeX-Format',
because some files are installed in TDS:tex/generic//.

This work consists of the main source file eolgrab.dtx
and the derived files
   eolgrab.sty, eolgrab.pdf, eolgrab.ins, eolgrab.drv,
   eolgrab-example-ltx.tex, eolgrab-example-env.tex,
   eolgrab-example-sec.tex, eolgrab-test1.tex,
   eolgrab-test2.tex.

\endpreamble
\let\MetaPrefix\DoubleperCent

\generate{%
  \file{eolgrab.ins}{\from{eolgrab.dtx}{install}}%
  \file{eolgrab.drv}{\from{eolgrab.dtx}{driver}}%
  \usedir{tex/generic/oberdiek}%
  \file{eolgrab.sty}{\from{eolgrab.dtx}{package}}%
  \usedir{doc/latex/oberdiek/example}%
  \file{eolgrab-example-ltx.tex}{\from{eolgrab.dtx}{example-ltx}}%
  \file{eolgrab-example-env.tex}{\from{eolgrab.dtx}{example-env}}%
  \file{eolgrab-example-sec.tex}{\from{eolgrab.dtx}{example-sec}}%
%  \usedir{doc/latex/oberdiek/test}%
%  \file{eolgrab-test1.tex}{\from{eolgrab.dtx}{test1}}%
%  \file{eolgrab-test2.tex}{\from{eolgrab.dtx}{test2}}%
}

\catcode32=13\relax% active space
\let =\space%
\Msg{************************************************************************}
\Msg{*}
\Msg{* To finish the installation you have to move the following}
\Msg{* file into a directory searched by TeX:}
\Msg{*}
\Msg{*     eolgrab.sty}
\Msg{*}
\Msg{* To produce the documentation run the file `eolgrab.drv'}
\Msg{* through LaTeX.}
\Msg{*}
\Msg{* Happy TeXing!}
\Msg{*}
\Msg{************************************************************************}

\endbatchfile
%</install>
%<*ignore>
\fi
%</ignore>
%<*driver>
\NeedsTeXFormat{LaTeX2e}
\ProvidesFile{eolgrab.drv}%
  [2016/05/16 v1.1 Catch arguments delimited by end of line (HO)]%
\documentclass{ltxdoc}
\usepackage{holtxdoc}[2011/11/22]
\begin{document}
  \DocInput{eolgrab.dtx}%
\end{document}
%</driver>
% \fi
%
%
%
% \GetFileInfo{eolgrab.drv}
%
% \title{The \xpackage{eolgrab} package}
% \date{2016/05/16 v1.1}
% \author{Heiko Oberdiek\thanks
% {Please report any issues at \url{https://github.com/ho-tex/oberdiek/issues}}}
%
% \maketitle
%
% \begin{abstract}
% This package implements a generic argument grabber
% to catch an argument that is delimited by the line end.
% \end{abstract}
%
% \tableofcontents
%
% \urldef\urltexhax\url{https://tug.org/mailman/listinfo/texhax}
% \def\eol/{^^A
%   \meta{\texttt{EOL}}^^A
% }
%
% \section{Documentation}
%
% The starting point for this package was a feature
% request of Arno Trautmann in the mailing list
% \texttt{texhax}\footnote{Info page for mailing list
% \texttt{texhax}: \urltexhax} \cite{trautmann-request}.
% A macro \cs{eolsection} should behave like \cs{section},
% but the argument should be delimited by the line end
% instead of given in curly braces:
% \begin{quote}
%   |\eolsection My Title|
% \end{quote}
% Phil Taylor answered this with an implementation for \cs{eolsection}.
% Because this feature could be useful for other macros as well,
% I answered with an implementation of \cs{eolgrab} as general solution
% \cite{oberdiek-eolgrab}.
%
% Both formats \hologo{plainTeX} and \hologo{LaTeX} are supported
% by the package, see the example for \cs{eolsection} below.
%
% \begin{declcs}{eolgrab} \M{code} \meta{argument} \eol/
% \end{declcs}
% Macro \cs{eolgrab} takes two arguments. The first argument
% is \meta{code}, a classical undelimited \hologo{TeX} macro argument.
% The second argument is delimited by the line end \eol/.
% The macro calls \meta{code} with \meta{argument} as argument
% in curly braces. Because the catcode of the line end is changed,
% \cs{eolgrab} will not work in the argument of other macros.
% Macro \cs{eolgrab} is made robust if either \hologo{eTeX}'s
% \cs{protected} or \hologo{LaTeX}'s \cs{DeclareRobustCommand}
% is available.
%
% \begin{declcs}{eolgrabopt} \M{code} \meta{argument} \eol/
% \end{declcs}
% Macro \cs{eolgrabopt} passes \meta{argument} as optional
% argument to \meta{code} if \meta{argument} is not empty.
% \begin{quote}
%   |\eolgrabopt\item foo|
% \end{quote}
% becomes to
% \begin{quote}
%   |\item[{foo}]|
% \end{quote}
% The curly argument braces are added to support square brackets
% inside \meta{argument}.
% If the \meta{argument} is empty:
% \begin{quote}
%   |\eolgrabopt\item|
% \end{quote}
% then
% \begin{quote}
%   |\item|
% \end{quote}
% is called without optional argument.
%
% \subsection{Examples}
%
% \begin{itemize}
% \item
%   The line
%   \begin{quote}
%     |\eolgrab\section My Title|
%   \end{quote}
%   is equivalent to
%   \begin{quote}
%     |\section{My Title}|
%   \end{quote}
% \item
%   The next example uses the star form of \cs{section}. Then the
%   command to be called consists of two tokens. Therefore the
%   first argument of \cs{eolgrab} needs curly braces:
%   \begin{quote}
%     |\eolgrab{\section*}My Title|
%   \end{quote}
%   becomes
%   \begin{quote}
%     |\section*{My Title}|
%   \end{quote}
% \item
%   Now \hologo{LaTeX}'s \cs{PackageError} is used. This macro
%   has three arguments, the package or class name, the message
%   text and the help text. A standard help text of \hologo{LaTeX}
%   is used as given in macro \cs{@ehc}. The second argument,
%   the message text is used as argument, delimited by line end:
%   \begin{quote}
%     |\eolgrab{\PackageError{foobar}}%|\\
%     |Some error message text\MessageBreak%|\\
%     |with several lines|\\
%     |\@ehc|
%   \end{quote}
%   In the first two lines of the example, the line end is suppressed
%   by the comment character (percent), thus the argument is delimited
%   by the line end of the third line. The result is:
%   \begin{quote}
%     |\PackageError{foobar}{Some error message text\MessageBreak|\\
%     |with several lines}\@ehc|
%   \end{quote}
% \item
%   The original request for macro \cs{eolsection}, see above,
%   can be implemented easily with the help of \cs{eolgrab}.
%   Example for \hologo{LaTeX}:
%   \begin{quote}
%     |\usepackage{eolgrab}|\\
%     |\newcommand*{\eolsection}{\eolgrab\section}|
%   \end{quote}
%   Example for \hologo{plainTeX}:
%   \begin{quote}
%     |\input eolgrab.sty\relax|\\
%     |\def\eolsection{\eolgrab\section}|
%   \end{quote}
%   And a sophisticated variant for \hologo{LaTeX} that also supports
%   the star syntax and the optional argument:
%    \begin{macrocode}
%<*example-sec>
\documentclass{article}
\usepackage{eolgrab}
\makeatletter
\newcommand*{\eolsection}{%
  \@ifstar{%
    \eolgrab{\section*}%
  }{%
    \@ifnextchar[{%
      \eoloptsection
    }{%
      \eolgrab\section
    }%
  }%
}
\newcommand*{\eoloptsection}[1][]{%
  \eolgrab{\section[{#1}]}%
}
\makeatother
\begin{document}
\tableofcontents
\eolsection Section without star and optional argument
\eolsection*Section with star
\eolsection[Short section title]Long section title
\end{document}
%</example-sec>
%    \end{macrocode}
% \end{itemize}
%
% \subsubsection{Small \hologo{LaTeX} document as example}
%
%    \begin{macrocode}
%<*example-ltx>
\RequirePackage{eolgrab}
\eolgrab\documentclass article
\eolgrab\begin document
\eolgrab\section Hello World
\eolgrab\emph Some text
\eolgrab\end document
%</example-ltx>
%    \end{macrocode}
%
% \subsubsection{\hologo{LaTeX} document with environments}
%
%    \begin{macrocode}
%<*example-env>
\documentclass{article}
\usepackage{eolgrab}
\newcommand*{\Begin}{\eolgrab   \begin}
\newcommand*{\End  }{\eolgrab   \end  }
\newcommand*{\Item }{\eolgrabopt\item }
\Begin document
  \Begin itemize
    \Item
      first item
    \Item
      second item
  \End itemize
  \Begin description
    \Item foo
      is the first syllable of foobar.
    \Item bar
      is the second syllable of foobar.
  \End description
\End document
%</example-env>
%    \end{macrocode}
%
% \subsection{Limitations}
%
% Macro \cs{eolgrab} needs to catch the line end.
% If \hologo{TeX} reads a line, then it throws away the line end
% characters (carriage return, line feed) and removes spaces at the
% end of the line. Then it adds the character with the character code
% that is given by \cs{endlinechar} at the end of the line. The category
% code of the inserted character is given by the current value of
% its \cs{catcode}. If \cs{endlinechar} is not a valid character code
% (especially if it is negative), then no character is added.
%
% In \hologo{plainTeX} and \hologo{LaTeX} the standard settings
% of the inserted endline character is the character with code 13
% (or |^||^||M| in \hologo{TeX} notation) with catcode 5 (end of line).
% That means the inserted end of line character behaves like a space
% token. For example, it is removed after macro names. Therefore
% \cs{eolgrab} changes the catcode.
%
% Therefore \cs{eolgrab} has some limitations:
% \begin{itemize}
% \item
%   Like other verbatim stuff, the macro \cs{eolgrab} cannot be
%   used in the argument of other macros. \cs{eolgrab} want to
%   change the catcode of the end of line character. If this
%   character is read before, because it is processed as argument
%   of another macro, the catcode is already set and is not reassigned
%   later if \cs{eolgrab} changes the category code for this character
%   code.
% \item
%   The argument must not contain the end of line character.
%   Otherwise the first end of line character is already taken
%   as delimiter, leaving the rest of the line outside the argument.
% \item
%   Because \cs{eolgrab} is probably mostly used in the line with
%   the delimited argument. Therefore changes of \cs{endlinechar}
%   will not affect the current line.
% \end{itemize}
%
% \StopEventually{
% }
%
% \section{Implementation}
%
%    \begin{macrocode}
%<*package>
%    \end{macrocode}
%
% \subsection{Reload check and package identification}
%    Reload check, especially if the package is not used with \LaTeX.
%    \begin{macrocode}
\begingroup\catcode61\catcode48\catcode32=10\relax%
  \catcode13=5 % ^^M
  \endlinechar=13 %
  \catcode35=6 % #
  \catcode39=12 % '
  \catcode44=12 % ,
  \catcode45=12 % -
  \catcode46=12 % .
  \catcode58=12 % :
  \catcode64=11 % @
  \catcode123=1 % {
  \catcode125=2 % }
  \expandafter\let\expandafter\x\csname ver@eolgrab.sty\endcsname
  \ifx\x\relax % plain-TeX, first loading
  \else
    \def\empty{}%
    \ifx\x\empty % LaTeX, first loading,
      % variable is initialized, but \ProvidesPackage not yet seen
    \else
      \expandafter\ifx\csname PackageInfo\endcsname\relax
        \def\x#1#2{%
          \immediate\write-1{Package #1 Info: #2.}%
        }%
      \else
        \def\x#1#2{\PackageInfo{#1}{#2, stopped}}%
      \fi
      \x{eolgrab}{The package is already loaded}%
      \aftergroup\endinput
    \fi
  \fi
\endgroup%
%    \end{macrocode}
%    Package identification:
%    \begin{macrocode}
\begingroup\catcode61\catcode48\catcode32=10\relax%
  \catcode13=5 % ^^M
  \endlinechar=13 %
  \catcode35=6 % #
  \catcode39=12 % '
  \catcode40=12 % (
  \catcode41=12 % )
  \catcode44=12 % ,
  \catcode45=12 % -
  \catcode46=12 % .
  \catcode47=12 % /
  \catcode58=12 % :
  \catcode64=11 % @
  \catcode91=12 % [
  \catcode93=12 % ]
  \catcode123=1 % {
  \catcode125=2 % }
  \expandafter\ifx\csname ProvidesPackage\endcsname\relax
    \def\x#1#2#3[#4]{\endgroup
      \immediate\write-1{Package: #3 #4}%
      \xdef#1{#4}%
    }%
  \else
    \def\x#1#2[#3]{\endgroup
      #2[{#3}]%
      \ifx#1\@undefined
        \xdef#1{#3}%
      \fi
      \ifx#1\relax
        \xdef#1{#3}%
      \fi
    }%
  \fi
\expandafter\x\csname ver@eolgrab.sty\endcsname
\ProvidesPackage{eolgrab}%
  [2016/05/16 v1.1 Catch arguments delimited by end of line (HO)]%
%    \end{macrocode}
%
% \subsection{Catcodes}
%
%    \begin{macrocode}
\begingroup\catcode61\catcode48\catcode32=10\relax%
  \catcode13=5 % ^^M
  \endlinechar=13 %
  \catcode123=1 % {
  \catcode125=2 % }
  \catcode64=11 % @
  \def\x{\endgroup
    \expandafter\edef\csname eolgrab@AtEnd\endcsname{%
      \endlinechar=\the\endlinechar\relax
      \catcode13=\the\catcode13\relax
      \catcode32=\the\catcode32\relax
      \catcode35=\the\catcode35\relax
      \catcode61=\the\catcode61\relax
      \catcode64=\the\catcode64\relax
      \catcode123=\the\catcode123\relax
      \catcode125=\the\catcode125\relax
    }%
  }%
\x\catcode61\catcode48\catcode32=10\relax%
\catcode13=5 % ^^M
\endlinechar=13 %
\catcode35=6 % #
\catcode64=11 % @
\catcode123=1 % {
\catcode125=2 % }
\def\TMP@EnsureCode#1#2{%
  \edef\eolgrab@AtEnd{%
    \eolgrab@AtEnd
    \catcode#1=\the\catcode#1\relax
  }%
  \catcode#1=#2\relax
}
\TMP@EnsureCode{40}{12}% (
\TMP@EnsureCode{41}{12}% )
\TMP@EnsureCode{42}{12}% *
\TMP@EnsureCode{46}{12}% .
\TMP@EnsureCode{47}{12}% /
\TMP@EnsureCode{91}{12}% [
\TMP@EnsureCode{93}{12}% ]
\TMP@EnsureCode{94}{7}% ^
\edef\eolgrab@AtEnd{\eolgrab@AtEnd\noexpand\endinput}
%    \end{macrocode}
%
% \subsection{Resources}
%
%    \begin{macrocode}
\begingroup\expandafter\expandafter\expandafter\endgroup
\expandafter\ifx\csname RequirePackage\endcsname\relax
  \input ltxcmds.sty\relax
  \input infwarerr.sty\relax
\else
  \RequirePackage{ltxcmds}[2010/12/04]%
  \RequirePackage{infwarerr}[2010/04/08]%
\fi
%    \end{macrocode}
%
%    \begin{macro}{\eolgrab@ifdefinable}
%    \begin{macrocode}
\ltx@IfUndefined{@ifdefinable}{%
  \def\eolgrab@ifdefinable#1#2{%
    \ltx@ifundefined{#1}{#2}{%
      \@PackageError{eolgrab}{%
        Command \ltx@backslashchar#1 already defined%
      }\@ehc
    }%
  }%
}{%
  \def\eolgrab@ifdefinable#1{%
    \expandafter\@ifdefinable\csname#1\endcsname
  }%
}
%    \end{macrocode}
%    \end{macro}
%
% \subsection{Macro \cs{eolgrab}}
%
%    \begin{macro}{\eolgrab}
%    \begin{macrocode}
\eolgrab@ifdefinable{eolgrab}{%
  \ltx@IfUndefined{protected}{%
    \ltx@IfUndefined{DeclareRobustCommand}{%
      \def\eolgrab#1%
    }{%
      \newcommand\eolgrab{}%
      \DeclareRobustCommand*\eolgrab
    }%
  }{%
    \protected\def\eolgrab#1%
  }{%
    \begingroup
    \endlinechar=13 %
    \catcode13=\ltx@active
    \eolgrab@{#1}%
  }%
}
%    \end{macrocode}
%    \end{macro}
%
%    \begin{macro}{\eolgrabopt}
%    \begin{macrocode}
\eolgrab@ifdefinable{eolgrabopt}{%
  \ltx@IfUndefined{protected}{%
    \ltx@IfUndefined{DeclareRobustCommand}{%
      \def\eolgrabopt#1%
    }{%
      \newcommand\eolgrabopt{}%
      \DeclareRobustCommand*\eolgrabopt
    }%
  }{%
    \protected\def\eolgrabopt#1%
  }{%
    \begingroup
    \endlinechar=13 %
    \catcode13=\ltx@active
    \eolgrab@opt{#1}%
  }%
}
%    \end{macrocode}
%    \end{macro}
%
%    \begin{macrocode}
\begingroup
  \catcode13=\ltx@active %
\ltx@firstofone{\endgroup %
%    \end{macrocode}
%    \begin{macro}{\eolgrab@}
%    \begin{macrocode}
  \def\eolgrab@#1#2^^M{%
    \endgroup %
    #1{#2}%
  }%
%    \end{macrocode}
%    \end{macro}
%    \begin{macro}{\eolgrab@opt}
%    \begin{macrocode}
  \def\eolgrab@opt#1#2^^M{%
    \endgroup %
    \ltx@ifempty{#2}{%
      #1%
    }{%
      #1[{#2}]%
    }%
  }%
%    \end{macrocode}
%    \end{macro}
%    \begin{macrocode}
}%
%    \end{macrocode}
%
%    \begin{macrocode}
\eolgrab@AtEnd%
%    \end{macrocode}
%    \begin{macrocode}
%</package>
%    \end{macrocode}
%% \section{Installation}
%
% \subsection{Download}
%
% \paragraph{Package.} This package is available on
% CTAN\footnote{\CTANpkg{eolgrab}}:
% \begin{description}
% \item[\CTAN{macros/latex/contrib/oberdiek/eolgrab.dtx}] The source file.
% \item[\CTAN{macros/latex/contrib/oberdiek/eolgrab.pdf}] Documentation.
% \end{description}
%
%
% \paragraph{Bundle.} All the packages of the bundle `oberdiek'
% are also available in a TDS compliant ZIP archive. There
% the packages are already unpacked and the documentation files
% are generated. The files and directories obey the TDS standard.
% \begin{description}
% \item[\CTANinstall{install/macros/latex/contrib/oberdiek.tds.zip}]
% \end{description}
% \emph{TDS} refers to the standard ``A Directory Structure
% for \TeX\ Files'' (\CTANpkg{tds}). Directories
% with \xfile{texmf} in their name are usually organized this way.
%
% \subsection{Bundle installation}
%
% \paragraph{Unpacking.} Unpack the \xfile{oberdiek.tds.zip} in the
% TDS tree (also known as \xfile{texmf} tree) of your choice.
% Example (linux):
% \begin{quote}
%   |unzip oberdiek.tds.zip -d ~/texmf|
% \end{quote}
%
% \subsection{Package installation}
%
% \paragraph{Unpacking.} The \xfile{.dtx} file is a self-extracting
% \docstrip\ archive. The files are extracted by running the
% \xfile{.dtx} through \plainTeX:
% \begin{quote}
%   \verb|tex eolgrab.dtx|
% \end{quote}
%
% \paragraph{TDS.} Now the different files must be moved into
% the different directories in your installation TDS tree
% (also known as \xfile{texmf} tree):
% \begin{quote}
% \def\t{^^A
% \begin{tabular}{@{}>{\ttfamily}l@{ $\rightarrow$ }>{\ttfamily}l@{}}
%   eolgrab.sty & tex/generic/oberdiek/eolgrab.sty\\
%   eolgrab.pdf & doc/latex/oberdiek/eolgrab.pdf\\
%   example/eolgrab-example-ltx.tex & doc/latex/oberdiek/example/eolgrab-example-ltx.tex\\
%   example/eolgrab-example-env.tex & doc/latex/oberdiek/example/eolgrab-example-env.tex\\
%   example/eolgrab-example-sec.tex & doc/latex/oberdiek/example/eolgrab-example-sec.tex\\
%   eolgrab.dtx & source/latex/oberdiek/eolgrab.dtx\\
% \end{tabular}^^A
% }^^A
% \sbox0{\t}^^A
% \ifdim\wd0>\linewidth
%   \begingroup
%     \advance\linewidth by\leftmargin
%     \advance\linewidth by\rightmargin
%   \edef\x{\endgroup
%     \def\noexpand\lw{\the\linewidth}^^A
%   }\x
%   \def\lwbox{^^A
%     \leavevmode
%     \hbox to \linewidth{^^A
%       \kern-\leftmargin\relax
%       \hss
%       \usebox0
%       \hss
%       \kern-\rightmargin\relax
%     }^^A
%   }^^A
%   \ifdim\wd0>\lw
%     \sbox0{\small\t}^^A
%     \ifdim\wd0>\linewidth
%       \ifdim\wd0>\lw
%         \sbox0{\footnotesize\t}^^A
%         \ifdim\wd0>\linewidth
%           \ifdim\wd0>\lw
%             \sbox0{\scriptsize\t}^^A
%             \ifdim\wd0>\linewidth
%               \ifdim\wd0>\lw
%                 \sbox0{\tiny\t}^^A
%                 \ifdim\wd0>\linewidth
%                   \lwbox
%                 \else
%                   \usebox0
%                 \fi
%               \else
%                 \lwbox
%               \fi
%             \else
%               \usebox0
%             \fi
%           \else
%             \lwbox
%           \fi
%         \else
%           \usebox0
%         \fi
%       \else
%         \lwbox
%       \fi
%     \else
%       \usebox0
%     \fi
%   \else
%     \lwbox
%   \fi
% \else
%   \usebox0
% \fi
% \end{quote}
% If you have a \xfile{docstrip.cfg} that configures and enables \docstrip's
% TDS installing feature, then some files can already be in the right
% place, see the documentation of \docstrip.
%
% \subsection{Refresh file name databases}
%
% If your \TeX~distribution
% (\TeX\,Live, \mikTeX, \dots) relies on file name databases, you must refresh
% these. For example, \TeX\,Live\ users run \verb|texhash| or
% \verb|mktexlsr|.
%
% \subsection{Some details for the interested}
%
% \paragraph{Unpacking with \LaTeX.}
% The \xfile{.dtx} chooses its action depending on the format:
% \begin{description}
% \item[\plainTeX:] Run \docstrip\ and extract the files.
% \item[\LaTeX:] Generate the documentation.
% \end{description}
% If you insist on using \LaTeX\ for \docstrip\ (really,
% \docstrip\ does not need \LaTeX), then inform the autodetect routine
% about your intention:
% \begin{quote}
%   \verb|latex \let\install=y\input{eolgrab.dtx}|
% \end{quote}
% Do not forget to quote the argument according to the demands
% of your shell.
%
% \paragraph{Generating the documentation.}
% You can use both the \xfile{.dtx} or the \xfile{.drv} to generate
% the documentation. The process can be configured by the
% configuration file \xfile{ltxdoc.cfg}. For instance, put this
% line into this file, if you want to have A4 as paper format:
% \begin{quote}
%   \verb|\PassOptionsToClass{a4paper}{article}|
% \end{quote}
% An example follows how to generate the
% documentation with pdf\LaTeX:
% \begin{quote}
%\begin{verbatim}
%pdflatex eolgrab.dtx
%makeindex -s gind.ist eolgrab.idx
%pdflatex eolgrab.dtx
%makeindex -s gind.ist eolgrab.idx
%pdflatex eolgrab.dtx
%\end{verbatim}
% \end{quote}
%
% \begin{thebibliography}{9}
% \raggedright
% \bibitem{trautmann-request}
% Arno Trautmann,
% \textit{[texhax] read argument until EOL};
% mailing list \texttt{texthax@tug.org},
% 2011-01-06;
% \url{https://tug.org/pipermail/texhax/2011-January/016517.html}.
%
% \bibitem{taylor}
% Philip Taylor,
% \textit{Re: [texhax] read argument until EOL};
% mailing list \texttt{texhax@tug.org},
% 2011-01-06;
% \url{https://tug.org/pipermail/texhax/2011-January/016519.html}.
%
% \bibitem{oberdiek-eolgrab}
% Heiko Oberdiek,
% \textit{Re: [texhax] read argument until EOL};
% mailing list \texttt{texhax@tug.org},
% 2011-01-06;
% \url{https://tug.org/pipermail/texhax/2011-January/016526.html}.
%
% \end{thebibliography}
%
% \begin{History}
%   \begin{Version}{2011/01/12 v1.0}
%   \item
%     First public version.
%   \end{Version}
%   \begin{Version}{2016/05/16 v1.1}
%   \item
%     Documentation updates.
%   \end{Version}
% \end{History}
%
% \PrintIndex
%
% \Finale
\endinput
|
% \end{quote}
% Do not forget to quote the argument according to the demands
% of your shell.
%
% \paragraph{Generating the documentation.}
% You can use both the \xfile{.dtx} or the \xfile{.drv} to generate
% the documentation. The process can be configured by the
% configuration file \xfile{ltxdoc.cfg}. For instance, put this
% line into this file, if you want to have A4 as paper format:
% \begin{quote}
%   \verb|\PassOptionsToClass{a4paper}{article}|
% \end{quote}
% An example follows how to generate the
% documentation with pdf\LaTeX:
% \begin{quote}
%\begin{verbatim}
%pdflatex eolgrab.dtx
%makeindex -s gind.ist eolgrab.idx
%pdflatex eolgrab.dtx
%makeindex -s gind.ist eolgrab.idx
%pdflatex eolgrab.dtx
%\end{verbatim}
% \end{quote}
%
% \begin{thebibliography}{9}
% \raggedright
% \bibitem{trautmann-request}
% Arno Trautmann,
% \textit{[texhax] read argument until EOL};
% mailing list \texttt{texthax@tug.org},
% 2011-01-06;
% \url{https://tug.org/pipermail/texhax/2011-January/016517.html}.
%
% \bibitem{taylor}
% Philip Taylor,
% \textit{Re: [texhax] read argument until EOL};
% mailing list \texttt{texhax@tug.org},
% 2011-01-06;
% \url{https://tug.org/pipermail/texhax/2011-January/016519.html}.
%
% \bibitem{oberdiek-eolgrab}
% Heiko Oberdiek,
% \textit{Re: [texhax] read argument until EOL};
% mailing list \texttt{texhax@tug.org},
% 2011-01-06;
% \url{https://tug.org/pipermail/texhax/2011-January/016526.html}.
%
% \end{thebibliography}
%
% \begin{History}
%   \begin{Version}{2011/01/12 v1.0}
%   \item
%     First public version.
%   \end{Version}
%   \begin{Version}{2016/05/16 v1.1}
%   \item
%     Documentation updates.
%   \end{Version}
% \end{History}
%
% \PrintIndex
%
% \Finale
\endinput

%        (quote the arguments according to the demands of your shell)
%
% Documentation:
%    (a) If eolgrab.drv is present:
%           latex eolgrab.drv
%    (b) Without eolgrab.drv:
%           latex eolgrab.dtx; ...
%    The class ltxdoc loads the configuration file ltxdoc.cfg
%    if available. Here you can specify further options, e.g.
%    use A4 as paper format:
%       \PassOptionsToClass{a4paper}{article}
%
%    Programm calls to get the documentation (example):
%       pdflatex eolgrab.dtx
%       makeindex -s gind.ist eolgrab.idx
%       pdflatex eolgrab.dtx
%       makeindex -s gind.ist eolgrab.idx
%       pdflatex eolgrab.dtx
%
% Installation:
%    TDS:tex/generic/oberdiek/eolgrab.sty
%    TDS:doc/latex/oberdiek/eolgrab.pdf
%    TDS:doc/latex/oberdiek/example/eolgrab-example-ltx.tex
%    TDS:doc/latex/oberdiek/example/eolgrab-example-env.tex
%    TDS:doc/latex/oberdiek/example/eolgrab-example-sec.tex
%    TDS:source/latex/oberdiek/eolgrab.dtx
%
%<*ignore>
\begingroup
  \catcode123=1 %
  \catcode125=2 %
  \def\x{LaTeX2e}%
\expandafter\endgroup
\ifcase 0\ifx\install y1\fi\expandafter
         \ifx\csname processbatchFile\endcsname\relax\else1\fi
         \ifx\fmtname\x\else 1\fi\relax
\else\csname fi\endcsname
%</ignore>
%<*install>
\input docstrip.tex
\Msg{************************************************************************}
\Msg{* Installation}
\Msg{* Package: eolgrab 2016/05/16 v1.1 Catch arguments delimited by end of line (HO)}
\Msg{************************************************************************}

\keepsilent
\askforoverwritefalse

\let\MetaPrefix\relax
\preamble

This is a generated file.

Project: eolgrab
Version: 2016/05/16 v1.1

Copyright (C)
   2011 Heiko Oberdiek
   2016-2019 Oberdiek Package Support Group

This work may be distributed and/or modified under the
conditions of the LaTeX Project Public License, either
version 1.3c of this license or (at your option) any later
version. This version of this license is in
   https://www.latex-project.org/lppl/lppl-1-3c.txt
and the latest version of this license is in
   https://www.latex-project.org/lppl.txt
and version 1.3 or later is part of all distributions of
LaTeX version 2005/12/01 or later.

This work has the LPPL maintenance status "maintained".

The Current Maintainers of this work are
Heiko Oberdiek and the Oberdiek Package Support Group
https://github.com/ho-tex/oberdiek/issues


The Base Interpreter refers to any `TeX-Format',
because some files are installed in TDS:tex/generic//.

This work consists of the main source file eolgrab.dtx
and the derived files
   eolgrab.sty, eolgrab.pdf, eolgrab.ins, eolgrab.drv,
   eolgrab-example-ltx.tex, eolgrab-example-env.tex,
   eolgrab-example-sec.tex, eolgrab-test1.tex,
   eolgrab-test2.tex.

\endpreamble
\let\MetaPrefix\DoubleperCent

\generate{%
  \file{eolgrab.ins}{\from{eolgrab.dtx}{install}}%
  \file{eolgrab.drv}{\from{eolgrab.dtx}{driver}}%
  \usedir{tex/generic/oberdiek}%
  \file{eolgrab.sty}{\from{eolgrab.dtx}{package}}%
  \usedir{doc/latex/oberdiek/example}%
  \file{eolgrab-example-ltx.tex}{\from{eolgrab.dtx}{example-ltx}}%
  \file{eolgrab-example-env.tex}{\from{eolgrab.dtx}{example-env}}%
  \file{eolgrab-example-sec.tex}{\from{eolgrab.dtx}{example-sec}}%
%  \usedir{doc/latex/oberdiek/test}%
%  \file{eolgrab-test1.tex}{\from{eolgrab.dtx}{test1}}%
%  \file{eolgrab-test2.tex}{\from{eolgrab.dtx}{test2}}%
}

\catcode32=13\relax% active space
\let =\space%
\Msg{************************************************************************}
\Msg{*}
\Msg{* To finish the installation you have to move the following}
\Msg{* file into a directory searched by TeX:}
\Msg{*}
\Msg{*     eolgrab.sty}
\Msg{*}
\Msg{* To produce the documentation run the file `eolgrab.drv'}
\Msg{* through LaTeX.}
\Msg{*}
\Msg{* Happy TeXing!}
\Msg{*}
\Msg{************************************************************************}

\endbatchfile
%</install>
%<*ignore>
\fi
%</ignore>
%<*driver>
\NeedsTeXFormat{LaTeX2e}
\ProvidesFile{eolgrab.drv}%
  [2016/05/16 v1.1 Catch arguments delimited by end of line (HO)]%
\documentclass{ltxdoc}
\usepackage{holtxdoc}[2011/11/22]
\begin{document}
  \DocInput{eolgrab.dtx}%
\end{document}
%</driver>
% \fi
%
%
%
% \GetFileInfo{eolgrab.drv}
%
% \title{The \xpackage{eolgrab} package}
% \date{2016/05/16 v1.1}
% \author{Heiko Oberdiek\thanks
% {Please report any issues at \url{https://github.com/ho-tex/oberdiek/issues}}}
%
% \maketitle
%
% \begin{abstract}
% This package implements a generic argument grabber
% to catch an argument that is delimited by the line end.
% \end{abstract}
%
% \tableofcontents
%
% \urldef\urltexhax\url{https://tug.org/mailman/listinfo/texhax}
% \def\eol/{^^A
%   \meta{\texttt{EOL}}^^A
% }
%
% \section{Documentation}
%
% The starting point for this package was a feature
% request of Arno Trautmann in the mailing list
% \texttt{texhax}\footnote{Info page for mailing list
% \texttt{texhax}: \urltexhax} \cite{trautmann-request}.
% A macro \cs{eolsection} should behave like \cs{section},
% but the argument should be delimited by the line end
% instead of given in curly braces:
% \begin{quote}
%   |\eolsection My Title|
% \end{quote}
% Phil Taylor answered this with an implementation for \cs{eolsection}.
% Because this feature could be useful for other macros as well,
% I answered with an implementation of \cs{eolgrab} as general solution
% \cite{oberdiek-eolgrab}.
%
% Both formats \hologo{plainTeX} and \hologo{LaTeX} are supported
% by the package, see the example for \cs{eolsection} below.
%
% \begin{declcs}{eolgrab} \M{code} \meta{argument} \eol/
% \end{declcs}
% Macro \cs{eolgrab} takes two arguments. The first argument
% is \meta{code}, a classical undelimited \hologo{TeX} macro argument.
% The second argument is delimited by the line end \eol/.
% The macro calls \meta{code} with \meta{argument} as argument
% in curly braces. Because the catcode of the line end is changed,
% \cs{eolgrab} will not work in the argument of other macros.
% Macro \cs{eolgrab} is made robust if either \hologo{eTeX}'s
% \cs{protected} or \hologo{LaTeX}'s \cs{DeclareRobustCommand}
% is available.
%
% \begin{declcs}{eolgrabopt} \M{code} \meta{argument} \eol/
% \end{declcs}
% Macro \cs{eolgrabopt} passes \meta{argument} as optional
% argument to \meta{code} if \meta{argument} is not empty.
% \begin{quote}
%   |\eolgrabopt\item foo|
% \end{quote}
% becomes to
% \begin{quote}
%   |\item[{foo}]|
% \end{quote}
% The curly argument braces are added to support square brackets
% inside \meta{argument}.
% If the \meta{argument} is empty:
% \begin{quote}
%   |\eolgrabopt\item|
% \end{quote}
% then
% \begin{quote}
%   |\item|
% \end{quote}
% is called without optional argument.
%
% \subsection{Examples}
%
% \begin{itemize}
% \item
%   The line
%   \begin{quote}
%     |\eolgrab\section My Title|
%   \end{quote}
%   is equivalent to
%   \begin{quote}
%     |\section{My Title}|
%   \end{quote}
% \item
%   The next example uses the star form of \cs{section}. Then the
%   command to be called consists of two tokens. Therefore the
%   first argument of \cs{eolgrab} needs curly braces:
%   \begin{quote}
%     |\eolgrab{\section*}My Title|
%   \end{quote}
%   becomes
%   \begin{quote}
%     |\section*{My Title}|
%   \end{quote}
% \item
%   Now \hologo{LaTeX}'s \cs{PackageError} is used. This macro
%   has three arguments, the package or class name, the message
%   text and the help text. A standard help text of \hologo{LaTeX}
%   is used as given in macro \cs{@ehc}. The second argument,
%   the message text is used as argument, delimited by line end:
%   \begin{quote}
%     |\eolgrab{\PackageError{foobar}}%|\\
%     |Some error message text\MessageBreak%|\\
%     |with several lines|\\
%     |\@ehc|
%   \end{quote}
%   In the first two lines of the example, the line end is suppressed
%   by the comment character (percent), thus the argument is delimited
%   by the line end of the third line. The result is:
%   \begin{quote}
%     |\PackageError{foobar}{Some error message text\MessageBreak|\\
%     |with several lines}\@ehc|
%   \end{quote}
% \item
%   The original request for macro \cs{eolsection}, see above,
%   can be implemented easily with the help of \cs{eolgrab}.
%   Example for \hologo{LaTeX}:
%   \begin{quote}
%     |\usepackage{eolgrab}|\\
%     |\newcommand*{\eolsection}{\eolgrab\section}|
%   \end{quote}
%   Example for \hologo{plainTeX}:
%   \begin{quote}
%     |\input eolgrab.sty\relax|\\
%     |\def\eolsection{\eolgrab\section}|
%   \end{quote}
%   And a sophisticated variant for \hologo{LaTeX} that also supports
%   the star syntax and the optional argument:
%    \begin{macrocode}
%<*example-sec>
\documentclass{article}
\usepackage{eolgrab}
\makeatletter
\newcommand*{\eolsection}{%
  \@ifstar{%
    \eolgrab{\section*}%
  }{%
    \@ifnextchar[{%
      \eoloptsection
    }{%
      \eolgrab\section
    }%
  }%
}
\newcommand*{\eoloptsection}[1][]{%
  \eolgrab{\section[{#1}]}%
}
\makeatother
\begin{document}
\tableofcontents
\eolsection Section without star and optional argument
\eolsection*Section with star
\eolsection[Short section title]Long section title
\end{document}
%</example-sec>
%    \end{macrocode}
% \end{itemize}
%
% \subsubsection{Small \hologo{LaTeX} document as example}
%
%    \begin{macrocode}
%<*example-ltx>
\RequirePackage{eolgrab}
\eolgrab\documentclass article
\eolgrab\begin document
\eolgrab\section Hello World
\eolgrab\emph Some text
\eolgrab\end document
%</example-ltx>
%    \end{macrocode}
%
% \subsubsection{\hologo{LaTeX} document with environments}
%
%    \begin{macrocode}
%<*example-env>
\documentclass{article}
\usepackage{eolgrab}
\newcommand*{\Begin}{\eolgrab   \begin}
\newcommand*{\End  }{\eolgrab   \end  }
\newcommand*{\Item }{\eolgrabopt\item }
\Begin document
  \Begin itemize
    \Item
      first item
    \Item
      second item
  \End itemize
  \Begin description
    \Item foo
      is the first syllable of foobar.
    \Item bar
      is the second syllable of foobar.
  \End description
\End document
%</example-env>
%    \end{macrocode}
%
% \subsection{Limitations}
%
% Macro \cs{eolgrab} needs to catch the line end.
% If \hologo{TeX} reads a line, then it throws away the line end
% characters (carriage return, line feed) and removes spaces at the
% end of the line. Then it adds the character with the character code
% that is given by \cs{endlinechar} at the end of the line. The category
% code of the inserted character is given by the current value of
% its \cs{catcode}. If \cs{endlinechar} is not a valid character code
% (especially if it is negative), then no character is added.
%
% In \hologo{plainTeX} and \hologo{LaTeX} the standard settings
% of the inserted endline character is the character with code 13
% (or |^||^||M| in \hologo{TeX} notation) with catcode 5 (end of line).
% That means the inserted end of line character behaves like a space
% token. For example, it is removed after macro names. Therefore
% \cs{eolgrab} changes the catcode.
%
% Therefore \cs{eolgrab} has some limitations:
% \begin{itemize}
% \item
%   Like other verbatim stuff, the macro \cs{eolgrab} cannot be
%   used in the argument of other macros. \cs{eolgrab} want to
%   change the catcode of the end of line character. If this
%   character is read before, because it is processed as argument
%   of another macro, the catcode is already set and is not reassigned
%   later if \cs{eolgrab} changes the category code for this character
%   code.
% \item
%   The argument must not contain the end of line character.
%   Otherwise the first end of line character is already taken
%   as delimiter, leaving the rest of the line outside the argument.
% \item
%   Because \cs{eolgrab} is probably mostly used in the line with
%   the delimited argument. Therefore changes of \cs{endlinechar}
%   will not affect the current line.
% \end{itemize}
%
% \StopEventually{
% }
%
% \section{Implementation}
%
%    \begin{macrocode}
%<*package>
%    \end{macrocode}
%
% \subsection{Reload check and package identification}
%    Reload check, especially if the package is not used with \LaTeX.
%    \begin{macrocode}
\begingroup\catcode61\catcode48\catcode32=10\relax%
  \catcode13=5 % ^^M
  \endlinechar=13 %
  \catcode35=6 % #
  \catcode39=12 % '
  \catcode44=12 % ,
  \catcode45=12 % -
  \catcode46=12 % .
  \catcode58=12 % :
  \catcode64=11 % @
  \catcode123=1 % {
  \catcode125=2 % }
  \expandafter\let\expandafter\x\csname ver@eolgrab.sty\endcsname
  \ifx\x\relax % plain-TeX, first loading
  \else
    \def\empty{}%
    \ifx\x\empty % LaTeX, first loading,
      % variable is initialized, but \ProvidesPackage not yet seen
    \else
      \expandafter\ifx\csname PackageInfo\endcsname\relax
        \def\x#1#2{%
          \immediate\write-1{Package #1 Info: #2.}%
        }%
      \else
        \def\x#1#2{\PackageInfo{#1}{#2, stopped}}%
      \fi
      \x{eolgrab}{The package is already loaded}%
      \aftergroup\endinput
    \fi
  \fi
\endgroup%
%    \end{macrocode}
%    Package identification:
%    \begin{macrocode}
\begingroup\catcode61\catcode48\catcode32=10\relax%
  \catcode13=5 % ^^M
  \endlinechar=13 %
  \catcode35=6 % #
  \catcode39=12 % '
  \catcode40=12 % (
  \catcode41=12 % )
  \catcode44=12 % ,
  \catcode45=12 % -
  \catcode46=12 % .
  \catcode47=12 % /
  \catcode58=12 % :
  \catcode64=11 % @
  \catcode91=12 % [
  \catcode93=12 % ]
  \catcode123=1 % {
  \catcode125=2 % }
  \expandafter\ifx\csname ProvidesPackage\endcsname\relax
    \def\x#1#2#3[#4]{\endgroup
      \immediate\write-1{Package: #3 #4}%
      \xdef#1{#4}%
    }%
  \else
    \def\x#1#2[#3]{\endgroup
      #2[{#3}]%
      \ifx#1\@undefined
        \xdef#1{#3}%
      \fi
      \ifx#1\relax
        \xdef#1{#3}%
      \fi
    }%
  \fi
\expandafter\x\csname ver@eolgrab.sty\endcsname
\ProvidesPackage{eolgrab}%
  [2016/05/16 v1.1 Catch arguments delimited by end of line (HO)]%
%    \end{macrocode}
%
% \subsection{Catcodes}
%
%    \begin{macrocode}
\begingroup\catcode61\catcode48\catcode32=10\relax%
  \catcode13=5 % ^^M
  \endlinechar=13 %
  \catcode123=1 % {
  \catcode125=2 % }
  \catcode64=11 % @
  \def\x{\endgroup
    \expandafter\edef\csname eolgrab@AtEnd\endcsname{%
      \endlinechar=\the\endlinechar\relax
      \catcode13=\the\catcode13\relax
      \catcode32=\the\catcode32\relax
      \catcode35=\the\catcode35\relax
      \catcode61=\the\catcode61\relax
      \catcode64=\the\catcode64\relax
      \catcode123=\the\catcode123\relax
      \catcode125=\the\catcode125\relax
    }%
  }%
\x\catcode61\catcode48\catcode32=10\relax%
\catcode13=5 % ^^M
\endlinechar=13 %
\catcode35=6 % #
\catcode64=11 % @
\catcode123=1 % {
\catcode125=2 % }
\def\TMP@EnsureCode#1#2{%
  \edef\eolgrab@AtEnd{%
    \eolgrab@AtEnd
    \catcode#1=\the\catcode#1\relax
  }%
  \catcode#1=#2\relax
}
\TMP@EnsureCode{40}{12}% (
\TMP@EnsureCode{41}{12}% )
\TMP@EnsureCode{42}{12}% *
\TMP@EnsureCode{46}{12}% .
\TMP@EnsureCode{47}{12}% /
\TMP@EnsureCode{91}{12}% [
\TMP@EnsureCode{93}{12}% ]
\TMP@EnsureCode{94}{7}% ^
\edef\eolgrab@AtEnd{\eolgrab@AtEnd\noexpand\endinput}
%    \end{macrocode}
%
% \subsection{Resources}
%
%    \begin{macrocode}
\begingroup\expandafter\expandafter\expandafter\endgroup
\expandafter\ifx\csname RequirePackage\endcsname\relax
  \input ltxcmds.sty\relax
  \input infwarerr.sty\relax
\else
  \RequirePackage{ltxcmds}[2010/12/04]%
  \RequirePackage{infwarerr}[2010/04/08]%
\fi
%    \end{macrocode}
%
%    \begin{macro}{\eolgrab@ifdefinable}
%    \begin{macrocode}
\ltx@IfUndefined{@ifdefinable}{%
  \def\eolgrab@ifdefinable#1#2{%
    \ltx@ifundefined{#1}{#2}{%
      \@PackageError{eolgrab}{%
        Command \ltx@backslashchar#1 already defined%
      }\@ehc
    }%
  }%
}{%
  \def\eolgrab@ifdefinable#1{%
    \expandafter\@ifdefinable\csname#1\endcsname
  }%
}
%    \end{macrocode}
%    \end{macro}
%
% \subsection{Macro \cs{eolgrab}}
%
%    \begin{macro}{\eolgrab}
%    \begin{macrocode}
\eolgrab@ifdefinable{eolgrab}{%
  \ltx@IfUndefined{protected}{%
    \ltx@IfUndefined{DeclareRobustCommand}{%
      \def\eolgrab#1%
    }{%
      \newcommand\eolgrab{}%
      \DeclareRobustCommand*\eolgrab
    }%
  }{%
    \protected\def\eolgrab#1%
  }{%
    \begingroup
    \endlinechar=13 %
    \catcode13=\ltx@active
    \eolgrab@{#1}%
  }%
}
%    \end{macrocode}
%    \end{macro}
%
%    \begin{macro}{\eolgrabopt}
%    \begin{macrocode}
\eolgrab@ifdefinable{eolgrabopt}{%
  \ltx@IfUndefined{protected}{%
    \ltx@IfUndefined{DeclareRobustCommand}{%
      \def\eolgrabopt#1%
    }{%
      \newcommand\eolgrabopt{}%
      \DeclareRobustCommand*\eolgrabopt
    }%
  }{%
    \protected\def\eolgrabopt#1%
  }{%
    \begingroup
    \endlinechar=13 %
    \catcode13=\ltx@active
    \eolgrab@opt{#1}%
  }%
}
%    \end{macrocode}
%    \end{macro}
%
%    \begin{macrocode}
\begingroup
  \catcode13=\ltx@active %
\ltx@firstofone{\endgroup %
%    \end{macrocode}
%    \begin{macro}{\eolgrab@}
%    \begin{macrocode}
  \def\eolgrab@#1#2^^M{%
    \endgroup %
    #1{#2}%
  }%
%    \end{macrocode}
%    \end{macro}
%    \begin{macro}{\eolgrab@opt}
%    \begin{macrocode}
  \def\eolgrab@opt#1#2^^M{%
    \endgroup %
    \ltx@ifempty{#2}{%
      #1%
    }{%
      #1[{#2}]%
    }%
  }%
%    \end{macrocode}
%    \end{macro}
%    \begin{macrocode}
}%
%    \end{macrocode}
%
%    \begin{macrocode}
\eolgrab@AtEnd%
%    \end{macrocode}
%    \begin{macrocode}
%</package>
%    \end{macrocode}
%% \section{Installation}
%
% \subsection{Download}
%
% \paragraph{Package.} This package is available on
% CTAN\footnote{\CTANpkg{eolgrab}}:
% \begin{description}
% \item[\CTAN{macros/latex/contrib/oberdiek/eolgrab.dtx}] The source file.
% \item[\CTAN{macros/latex/contrib/oberdiek/eolgrab.pdf}] Documentation.
% \end{description}
%
%
% \paragraph{Bundle.} All the packages of the bundle `oberdiek'
% are also available in a TDS compliant ZIP archive. There
% the packages are already unpacked and the documentation files
% are generated. The files and directories obey the TDS standard.
% \begin{description}
% \item[\CTANinstall{install/macros/latex/contrib/oberdiek.tds.zip}]
% \end{description}
% \emph{TDS} refers to the standard ``A Directory Structure
% for \TeX\ Files'' (\CTANpkg{tds}). Directories
% with \xfile{texmf} in their name are usually organized this way.
%
% \subsection{Bundle installation}
%
% \paragraph{Unpacking.} Unpack the \xfile{oberdiek.tds.zip} in the
% TDS tree (also known as \xfile{texmf} tree) of your choice.
% Example (linux):
% \begin{quote}
%   |unzip oberdiek.tds.zip -d ~/texmf|
% \end{quote}
%
% \subsection{Package installation}
%
% \paragraph{Unpacking.} The \xfile{.dtx} file is a self-extracting
% \docstrip\ archive. The files are extracted by running the
% \xfile{.dtx} through \plainTeX:
% \begin{quote}
%   \verb|tex eolgrab.dtx|
% \end{quote}
%
% \paragraph{TDS.} Now the different files must be moved into
% the different directories in your installation TDS tree
% (also known as \xfile{texmf} tree):
% \begin{quote}
% \def\t{^^A
% \begin{tabular}{@{}>{\ttfamily}l@{ $\rightarrow$ }>{\ttfamily}l@{}}
%   eolgrab.sty & tex/generic/oberdiek/eolgrab.sty\\
%   eolgrab.pdf & doc/latex/oberdiek/eolgrab.pdf\\
%   example/eolgrab-example-ltx.tex & doc/latex/oberdiek/example/eolgrab-example-ltx.tex\\
%   example/eolgrab-example-env.tex & doc/latex/oberdiek/example/eolgrab-example-env.tex\\
%   example/eolgrab-example-sec.tex & doc/latex/oberdiek/example/eolgrab-example-sec.tex\\
%   eolgrab.dtx & source/latex/oberdiek/eolgrab.dtx\\
% \end{tabular}^^A
% }^^A
% \sbox0{\t}^^A
% \ifdim\wd0>\linewidth
%   \begingroup
%     \advance\linewidth by\leftmargin
%     \advance\linewidth by\rightmargin
%   \edef\x{\endgroup
%     \def\noexpand\lw{\the\linewidth}^^A
%   }\x
%   \def\lwbox{^^A
%     \leavevmode
%     \hbox to \linewidth{^^A
%       \kern-\leftmargin\relax
%       \hss
%       \usebox0
%       \hss
%       \kern-\rightmargin\relax
%     }^^A
%   }^^A
%   \ifdim\wd0>\lw
%     \sbox0{\small\t}^^A
%     \ifdim\wd0>\linewidth
%       \ifdim\wd0>\lw
%         \sbox0{\footnotesize\t}^^A
%         \ifdim\wd0>\linewidth
%           \ifdim\wd0>\lw
%             \sbox0{\scriptsize\t}^^A
%             \ifdim\wd0>\linewidth
%               \ifdim\wd0>\lw
%                 \sbox0{\tiny\t}^^A
%                 \ifdim\wd0>\linewidth
%                   \lwbox
%                 \else
%                   \usebox0
%                 \fi
%               \else
%                 \lwbox
%               \fi
%             \else
%               \usebox0
%             \fi
%           \else
%             \lwbox
%           \fi
%         \else
%           \usebox0
%         \fi
%       \else
%         \lwbox
%       \fi
%     \else
%       \usebox0
%     \fi
%   \else
%     \lwbox
%   \fi
% \else
%   \usebox0
% \fi
% \end{quote}
% If you have a \xfile{docstrip.cfg} that configures and enables \docstrip's
% TDS installing feature, then some files can already be in the right
% place, see the documentation of \docstrip.
%
% \subsection{Refresh file name databases}
%
% If your \TeX~distribution
% (\TeX\,Live, \mikTeX, \dots) relies on file name databases, you must refresh
% these. For example, \TeX\,Live\ users run \verb|texhash| or
% \verb|mktexlsr|.
%
% \subsection{Some details for the interested}
%
% \paragraph{Unpacking with \LaTeX.}
% The \xfile{.dtx} chooses its action depending on the format:
% \begin{description}
% \item[\plainTeX:] Run \docstrip\ and extract the files.
% \item[\LaTeX:] Generate the documentation.
% \end{description}
% If you insist on using \LaTeX\ for \docstrip\ (really,
% \docstrip\ does not need \LaTeX), then inform the autodetect routine
% about your intention:
% \begin{quote}
%   \verb|latex \let\install=y% \iffalse meta-comment
%
% File: eolgrab.dtx
% Version: 2016/05/16 v1.1
% Info: Catch arguments delimited by end of line
%
% Copyright (C)
%    2011 Heiko Oberdiek
%    2016-2019 Oberdiek Package Support Group
%    https://github.com/ho-tex/oberdiek/issues
%
% This work may be distributed and/or modified under the
% conditions of the LaTeX Project Public License, either
% version 1.3c of this license or (at your option) any later
% version. This version of this license is in
%    https://www.latex-project.org/lppl/lppl-1-3c.txt
% and the latest version of this license is in
%    https://www.latex-project.org/lppl.txt
% and version 1.3 or later is part of all distributions of
% LaTeX version 2005/12/01 or later.
%
% This work has the LPPL maintenance status "maintained".
%
% The Current Maintainers of this work are
% Heiko Oberdiek and the Oberdiek Package Support Group
% https://github.com/ho-tex/oberdiek/issues
%
% The Base Interpreter refers to any `TeX-Format',
% because some files are installed in TDS:tex/generic//.
%
% This work consists of the main source file eolgrab.dtx
% and the derived files
%    eolgrab.sty, eolgrab.pdf, eolgrab.ins, eolgrab.drv,
%    eolgrab-example-ltx.tex, eolgrab-example-env.tex,
%    eolgrab-example-sec.tex, eolgrab-test1.tex,
%    eolgrab-test2.tex.
%
% Distribution:
%    CTAN:macros/latex/contrib/oberdiek/eolgrab.dtx
%    CTAN:macros/latex/contrib/oberdiek/eolgrab.pdf
%
% Unpacking:
%    (a) If eolgrab.ins is present:
%           tex eolgrab.ins
%    (b) Without eolgrab.ins:
%           tex eolgrab.dtx
%    (c) If you insist on using LaTeX
%           latex \let\install=y% \iffalse meta-comment
%
% File: eolgrab.dtx
% Version: 2016/05/16 v1.1
% Info: Catch arguments delimited by end of line
%
% Copyright (C)
%    2011 Heiko Oberdiek
%    2016-2019 Oberdiek Package Support Group
%    https://github.com/ho-tex/oberdiek/issues
%
% This work may be distributed and/or modified under the
% conditions of the LaTeX Project Public License, either
% version 1.3c of this license or (at your option) any later
% version. This version of this license is in
%    https://www.latex-project.org/lppl/lppl-1-3c.txt
% and the latest version of this license is in
%    https://www.latex-project.org/lppl.txt
% and version 1.3 or later is part of all distributions of
% LaTeX version 2005/12/01 or later.
%
% This work has the LPPL maintenance status "maintained".
%
% The Current Maintainers of this work are
% Heiko Oberdiek and the Oberdiek Package Support Group
% https://github.com/ho-tex/oberdiek/issues
%
% The Base Interpreter refers to any `TeX-Format',
% because some files are installed in TDS:tex/generic//.
%
% This work consists of the main source file eolgrab.dtx
% and the derived files
%    eolgrab.sty, eolgrab.pdf, eolgrab.ins, eolgrab.drv,
%    eolgrab-example-ltx.tex, eolgrab-example-env.tex,
%    eolgrab-example-sec.tex, eolgrab-test1.tex,
%    eolgrab-test2.tex.
%
% Distribution:
%    CTAN:macros/latex/contrib/oberdiek/eolgrab.dtx
%    CTAN:macros/latex/contrib/oberdiek/eolgrab.pdf
%
% Unpacking:
%    (a) If eolgrab.ins is present:
%           tex eolgrab.ins
%    (b) Without eolgrab.ins:
%           tex eolgrab.dtx
%    (c) If you insist on using LaTeX
%           latex \let\install=y\input{eolgrab.dtx}
%        (quote the arguments according to the demands of your shell)
%
% Documentation:
%    (a) If eolgrab.drv is present:
%           latex eolgrab.drv
%    (b) Without eolgrab.drv:
%           latex eolgrab.dtx; ...
%    The class ltxdoc loads the configuration file ltxdoc.cfg
%    if available. Here you can specify further options, e.g.
%    use A4 as paper format:
%       \PassOptionsToClass{a4paper}{article}
%
%    Programm calls to get the documentation (example):
%       pdflatex eolgrab.dtx
%       makeindex -s gind.ist eolgrab.idx
%       pdflatex eolgrab.dtx
%       makeindex -s gind.ist eolgrab.idx
%       pdflatex eolgrab.dtx
%
% Installation:
%    TDS:tex/generic/oberdiek/eolgrab.sty
%    TDS:doc/latex/oberdiek/eolgrab.pdf
%    TDS:doc/latex/oberdiek/example/eolgrab-example-ltx.tex
%    TDS:doc/latex/oberdiek/example/eolgrab-example-env.tex
%    TDS:doc/latex/oberdiek/example/eolgrab-example-sec.tex
%    TDS:source/latex/oberdiek/eolgrab.dtx
%
%<*ignore>
\begingroup
  \catcode123=1 %
  \catcode125=2 %
  \def\x{LaTeX2e}%
\expandafter\endgroup
\ifcase 0\ifx\install y1\fi\expandafter
         \ifx\csname processbatchFile\endcsname\relax\else1\fi
         \ifx\fmtname\x\else 1\fi\relax
\else\csname fi\endcsname
%</ignore>
%<*install>
\input docstrip.tex
\Msg{************************************************************************}
\Msg{* Installation}
\Msg{* Package: eolgrab 2016/05/16 v1.1 Catch arguments delimited by end of line (HO)}
\Msg{************************************************************************}

\keepsilent
\askforoverwritefalse

\let\MetaPrefix\relax
\preamble

This is a generated file.

Project: eolgrab
Version: 2016/05/16 v1.1

Copyright (C)
   2011 Heiko Oberdiek
   2016-2019 Oberdiek Package Support Group

This work may be distributed and/or modified under the
conditions of the LaTeX Project Public License, either
version 1.3c of this license or (at your option) any later
version. This version of this license is in
   https://www.latex-project.org/lppl/lppl-1-3c.txt
and the latest version of this license is in
   https://www.latex-project.org/lppl.txt
and version 1.3 or later is part of all distributions of
LaTeX version 2005/12/01 or later.

This work has the LPPL maintenance status "maintained".

The Current Maintainers of this work are
Heiko Oberdiek and the Oberdiek Package Support Group
https://github.com/ho-tex/oberdiek/issues


The Base Interpreter refers to any `TeX-Format',
because some files are installed in TDS:tex/generic//.

This work consists of the main source file eolgrab.dtx
and the derived files
   eolgrab.sty, eolgrab.pdf, eolgrab.ins, eolgrab.drv,
   eolgrab-example-ltx.tex, eolgrab-example-env.tex,
   eolgrab-example-sec.tex, eolgrab-test1.tex,
   eolgrab-test2.tex.

\endpreamble
\let\MetaPrefix\DoubleperCent

\generate{%
  \file{eolgrab.ins}{\from{eolgrab.dtx}{install}}%
  \file{eolgrab.drv}{\from{eolgrab.dtx}{driver}}%
  \usedir{tex/generic/oberdiek}%
  \file{eolgrab.sty}{\from{eolgrab.dtx}{package}}%
  \usedir{doc/latex/oberdiek/example}%
  \file{eolgrab-example-ltx.tex}{\from{eolgrab.dtx}{example-ltx}}%
  \file{eolgrab-example-env.tex}{\from{eolgrab.dtx}{example-env}}%
  \file{eolgrab-example-sec.tex}{\from{eolgrab.dtx}{example-sec}}%
%  \usedir{doc/latex/oberdiek/test}%
%  \file{eolgrab-test1.tex}{\from{eolgrab.dtx}{test1}}%
%  \file{eolgrab-test2.tex}{\from{eolgrab.dtx}{test2}}%
}

\catcode32=13\relax% active space
\let =\space%
\Msg{************************************************************************}
\Msg{*}
\Msg{* To finish the installation you have to move the following}
\Msg{* file into a directory searched by TeX:}
\Msg{*}
\Msg{*     eolgrab.sty}
\Msg{*}
\Msg{* To produce the documentation run the file `eolgrab.drv'}
\Msg{* through LaTeX.}
\Msg{*}
\Msg{* Happy TeXing!}
\Msg{*}
\Msg{************************************************************************}

\endbatchfile
%</install>
%<*ignore>
\fi
%</ignore>
%<*driver>
\NeedsTeXFormat{LaTeX2e}
\ProvidesFile{eolgrab.drv}%
  [2016/05/16 v1.1 Catch arguments delimited by end of line (HO)]%
\documentclass{ltxdoc}
\usepackage{holtxdoc}[2011/11/22]
\begin{document}
  \DocInput{eolgrab.dtx}%
\end{document}
%</driver>
% \fi
%
%
%
% \GetFileInfo{eolgrab.drv}
%
% \title{The \xpackage{eolgrab} package}
% \date{2016/05/16 v1.1}
% \author{Heiko Oberdiek\thanks
% {Please report any issues at \url{https://github.com/ho-tex/oberdiek/issues}}}
%
% \maketitle
%
% \begin{abstract}
% This package implements a generic argument grabber
% to catch an argument that is delimited by the line end.
% \end{abstract}
%
% \tableofcontents
%
% \urldef\urltexhax\url{https://tug.org/mailman/listinfo/texhax}
% \def\eol/{^^A
%   \meta{\texttt{EOL}}^^A
% }
%
% \section{Documentation}
%
% The starting point for this package was a feature
% request of Arno Trautmann in the mailing list
% \texttt{texhax}\footnote{Info page for mailing list
% \texttt{texhax}: \urltexhax} \cite{trautmann-request}.
% A macro \cs{eolsection} should behave like \cs{section},
% but the argument should be delimited by the line end
% instead of given in curly braces:
% \begin{quote}
%   |\eolsection My Title|
% \end{quote}
% Phil Taylor answered this with an implementation for \cs{eolsection}.
% Because this feature could be useful for other macros as well,
% I answered with an implementation of \cs{eolgrab} as general solution
% \cite{oberdiek-eolgrab}.
%
% Both formats \hologo{plainTeX} and \hologo{LaTeX} are supported
% by the package, see the example for \cs{eolsection} below.
%
% \begin{declcs}{eolgrab} \M{code} \meta{argument} \eol/
% \end{declcs}
% Macro \cs{eolgrab} takes two arguments. The first argument
% is \meta{code}, a classical undelimited \hologo{TeX} macro argument.
% The second argument is delimited by the line end \eol/.
% The macro calls \meta{code} with \meta{argument} as argument
% in curly braces. Because the catcode of the line end is changed,
% \cs{eolgrab} will not work in the argument of other macros.
% Macro \cs{eolgrab} is made robust if either \hologo{eTeX}'s
% \cs{protected} or \hologo{LaTeX}'s \cs{DeclareRobustCommand}
% is available.
%
% \begin{declcs}{eolgrabopt} \M{code} \meta{argument} \eol/
% \end{declcs}
% Macro \cs{eolgrabopt} passes \meta{argument} as optional
% argument to \meta{code} if \meta{argument} is not empty.
% \begin{quote}
%   |\eolgrabopt\item foo|
% \end{quote}
% becomes to
% \begin{quote}
%   |\item[{foo}]|
% \end{quote}
% The curly argument braces are added to support square brackets
% inside \meta{argument}.
% If the \meta{argument} is empty:
% \begin{quote}
%   |\eolgrabopt\item|
% \end{quote}
% then
% \begin{quote}
%   |\item|
% \end{quote}
% is called without optional argument.
%
% \subsection{Examples}
%
% \begin{itemize}
% \item
%   The line
%   \begin{quote}
%     |\eolgrab\section My Title|
%   \end{quote}
%   is equivalent to
%   \begin{quote}
%     |\section{My Title}|
%   \end{quote}
% \item
%   The next example uses the star form of \cs{section}. Then the
%   command to be called consists of two tokens. Therefore the
%   first argument of \cs{eolgrab} needs curly braces:
%   \begin{quote}
%     |\eolgrab{\section*}My Title|
%   \end{quote}
%   becomes
%   \begin{quote}
%     |\section*{My Title}|
%   \end{quote}
% \item
%   Now \hologo{LaTeX}'s \cs{PackageError} is used. This macro
%   has three arguments, the package or class name, the message
%   text and the help text. A standard help text of \hologo{LaTeX}
%   is used as given in macro \cs{@ehc}. The second argument,
%   the message text is used as argument, delimited by line end:
%   \begin{quote}
%     |\eolgrab{\PackageError{foobar}}%|\\
%     |Some error message text\MessageBreak%|\\
%     |with several lines|\\
%     |\@ehc|
%   \end{quote}
%   In the first two lines of the example, the line end is suppressed
%   by the comment character (percent), thus the argument is delimited
%   by the line end of the third line. The result is:
%   \begin{quote}
%     |\PackageError{foobar}{Some error message text\MessageBreak|\\
%     |with several lines}\@ehc|
%   \end{quote}
% \item
%   The original request for macro \cs{eolsection}, see above,
%   can be implemented easily with the help of \cs{eolgrab}.
%   Example for \hologo{LaTeX}:
%   \begin{quote}
%     |\usepackage{eolgrab}|\\
%     |\newcommand*{\eolsection}{\eolgrab\section}|
%   \end{quote}
%   Example for \hologo{plainTeX}:
%   \begin{quote}
%     |\input eolgrab.sty\relax|\\
%     |\def\eolsection{\eolgrab\section}|
%   \end{quote}
%   And a sophisticated variant for \hologo{LaTeX} that also supports
%   the star syntax and the optional argument:
%    \begin{macrocode}
%<*example-sec>
\documentclass{article}
\usepackage{eolgrab}
\makeatletter
\newcommand*{\eolsection}{%
  \@ifstar{%
    \eolgrab{\section*}%
  }{%
    \@ifnextchar[{%
      \eoloptsection
    }{%
      \eolgrab\section
    }%
  }%
}
\newcommand*{\eoloptsection}[1][]{%
  \eolgrab{\section[{#1}]}%
}
\makeatother
\begin{document}
\tableofcontents
\eolsection Section without star and optional argument
\eolsection*Section with star
\eolsection[Short section title]Long section title
\end{document}
%</example-sec>
%    \end{macrocode}
% \end{itemize}
%
% \subsubsection{Small \hologo{LaTeX} document as example}
%
%    \begin{macrocode}
%<*example-ltx>
\RequirePackage{eolgrab}
\eolgrab\documentclass article
\eolgrab\begin document
\eolgrab\section Hello World
\eolgrab\emph Some text
\eolgrab\end document
%</example-ltx>
%    \end{macrocode}
%
% \subsubsection{\hologo{LaTeX} document with environments}
%
%    \begin{macrocode}
%<*example-env>
\documentclass{article}
\usepackage{eolgrab}
\newcommand*{\Begin}{\eolgrab   \begin}
\newcommand*{\End  }{\eolgrab   \end  }
\newcommand*{\Item }{\eolgrabopt\item }
\Begin document
  \Begin itemize
    \Item
      first item
    \Item
      second item
  \End itemize
  \Begin description
    \Item foo
      is the first syllable of foobar.
    \Item bar
      is the second syllable of foobar.
  \End description
\End document
%</example-env>
%    \end{macrocode}
%
% \subsection{Limitations}
%
% Macro \cs{eolgrab} needs to catch the line end.
% If \hologo{TeX} reads a line, then it throws away the line end
% characters (carriage return, line feed) and removes spaces at the
% end of the line. Then it adds the character with the character code
% that is given by \cs{endlinechar} at the end of the line. The category
% code of the inserted character is given by the current value of
% its \cs{catcode}. If \cs{endlinechar} is not a valid character code
% (especially if it is negative), then no character is added.
%
% In \hologo{plainTeX} and \hologo{LaTeX} the standard settings
% of the inserted endline character is the character with code 13
% (or |^||^||M| in \hologo{TeX} notation) with catcode 5 (end of line).
% That means the inserted end of line character behaves like a space
% token. For example, it is removed after macro names. Therefore
% \cs{eolgrab} changes the catcode.
%
% Therefore \cs{eolgrab} has some limitations:
% \begin{itemize}
% \item
%   Like other verbatim stuff, the macro \cs{eolgrab} cannot be
%   used in the argument of other macros. \cs{eolgrab} want to
%   change the catcode of the end of line character. If this
%   character is read before, because it is processed as argument
%   of another macro, the catcode is already set and is not reassigned
%   later if \cs{eolgrab} changes the category code for this character
%   code.
% \item
%   The argument must not contain the end of line character.
%   Otherwise the first end of line character is already taken
%   as delimiter, leaving the rest of the line outside the argument.
% \item
%   Because \cs{eolgrab} is probably mostly used in the line with
%   the delimited argument. Therefore changes of \cs{endlinechar}
%   will not affect the current line.
% \end{itemize}
%
% \StopEventually{
% }
%
% \section{Implementation}
%
%    \begin{macrocode}
%<*package>
%    \end{macrocode}
%
% \subsection{Reload check and package identification}
%    Reload check, especially if the package is not used with \LaTeX.
%    \begin{macrocode}
\begingroup\catcode61\catcode48\catcode32=10\relax%
  \catcode13=5 % ^^M
  \endlinechar=13 %
  \catcode35=6 % #
  \catcode39=12 % '
  \catcode44=12 % ,
  \catcode45=12 % -
  \catcode46=12 % .
  \catcode58=12 % :
  \catcode64=11 % @
  \catcode123=1 % {
  \catcode125=2 % }
  \expandafter\let\expandafter\x\csname ver@eolgrab.sty\endcsname
  \ifx\x\relax % plain-TeX, first loading
  \else
    \def\empty{}%
    \ifx\x\empty % LaTeX, first loading,
      % variable is initialized, but \ProvidesPackage not yet seen
    \else
      \expandafter\ifx\csname PackageInfo\endcsname\relax
        \def\x#1#2{%
          \immediate\write-1{Package #1 Info: #2.}%
        }%
      \else
        \def\x#1#2{\PackageInfo{#1}{#2, stopped}}%
      \fi
      \x{eolgrab}{The package is already loaded}%
      \aftergroup\endinput
    \fi
  \fi
\endgroup%
%    \end{macrocode}
%    Package identification:
%    \begin{macrocode}
\begingroup\catcode61\catcode48\catcode32=10\relax%
  \catcode13=5 % ^^M
  \endlinechar=13 %
  \catcode35=6 % #
  \catcode39=12 % '
  \catcode40=12 % (
  \catcode41=12 % )
  \catcode44=12 % ,
  \catcode45=12 % -
  \catcode46=12 % .
  \catcode47=12 % /
  \catcode58=12 % :
  \catcode64=11 % @
  \catcode91=12 % [
  \catcode93=12 % ]
  \catcode123=1 % {
  \catcode125=2 % }
  \expandafter\ifx\csname ProvidesPackage\endcsname\relax
    \def\x#1#2#3[#4]{\endgroup
      \immediate\write-1{Package: #3 #4}%
      \xdef#1{#4}%
    }%
  \else
    \def\x#1#2[#3]{\endgroup
      #2[{#3}]%
      \ifx#1\@undefined
        \xdef#1{#3}%
      \fi
      \ifx#1\relax
        \xdef#1{#3}%
      \fi
    }%
  \fi
\expandafter\x\csname ver@eolgrab.sty\endcsname
\ProvidesPackage{eolgrab}%
  [2016/05/16 v1.1 Catch arguments delimited by end of line (HO)]%
%    \end{macrocode}
%
% \subsection{Catcodes}
%
%    \begin{macrocode}
\begingroup\catcode61\catcode48\catcode32=10\relax%
  \catcode13=5 % ^^M
  \endlinechar=13 %
  \catcode123=1 % {
  \catcode125=2 % }
  \catcode64=11 % @
  \def\x{\endgroup
    \expandafter\edef\csname eolgrab@AtEnd\endcsname{%
      \endlinechar=\the\endlinechar\relax
      \catcode13=\the\catcode13\relax
      \catcode32=\the\catcode32\relax
      \catcode35=\the\catcode35\relax
      \catcode61=\the\catcode61\relax
      \catcode64=\the\catcode64\relax
      \catcode123=\the\catcode123\relax
      \catcode125=\the\catcode125\relax
    }%
  }%
\x\catcode61\catcode48\catcode32=10\relax%
\catcode13=5 % ^^M
\endlinechar=13 %
\catcode35=6 % #
\catcode64=11 % @
\catcode123=1 % {
\catcode125=2 % }
\def\TMP@EnsureCode#1#2{%
  \edef\eolgrab@AtEnd{%
    \eolgrab@AtEnd
    \catcode#1=\the\catcode#1\relax
  }%
  \catcode#1=#2\relax
}
\TMP@EnsureCode{40}{12}% (
\TMP@EnsureCode{41}{12}% )
\TMP@EnsureCode{42}{12}% *
\TMP@EnsureCode{46}{12}% .
\TMP@EnsureCode{47}{12}% /
\TMP@EnsureCode{91}{12}% [
\TMP@EnsureCode{93}{12}% ]
\TMP@EnsureCode{94}{7}% ^
\edef\eolgrab@AtEnd{\eolgrab@AtEnd\noexpand\endinput}
%    \end{macrocode}
%
% \subsection{Resources}
%
%    \begin{macrocode}
\begingroup\expandafter\expandafter\expandafter\endgroup
\expandafter\ifx\csname RequirePackage\endcsname\relax
  \input ltxcmds.sty\relax
  \input infwarerr.sty\relax
\else
  \RequirePackage{ltxcmds}[2010/12/04]%
  \RequirePackage{infwarerr}[2010/04/08]%
\fi
%    \end{macrocode}
%
%    \begin{macro}{\eolgrab@ifdefinable}
%    \begin{macrocode}
\ltx@IfUndefined{@ifdefinable}{%
  \def\eolgrab@ifdefinable#1#2{%
    \ltx@ifundefined{#1}{#2}{%
      \@PackageError{eolgrab}{%
        Command \ltx@backslashchar#1 already defined%
      }\@ehc
    }%
  }%
}{%
  \def\eolgrab@ifdefinable#1{%
    \expandafter\@ifdefinable\csname#1\endcsname
  }%
}
%    \end{macrocode}
%    \end{macro}
%
% \subsection{Macro \cs{eolgrab}}
%
%    \begin{macro}{\eolgrab}
%    \begin{macrocode}
\eolgrab@ifdefinable{eolgrab}{%
  \ltx@IfUndefined{protected}{%
    \ltx@IfUndefined{DeclareRobustCommand}{%
      \def\eolgrab#1%
    }{%
      \newcommand\eolgrab{}%
      \DeclareRobustCommand*\eolgrab
    }%
  }{%
    \protected\def\eolgrab#1%
  }{%
    \begingroup
    \endlinechar=13 %
    \catcode13=\ltx@active
    \eolgrab@{#1}%
  }%
}
%    \end{macrocode}
%    \end{macro}
%
%    \begin{macro}{\eolgrabopt}
%    \begin{macrocode}
\eolgrab@ifdefinable{eolgrabopt}{%
  \ltx@IfUndefined{protected}{%
    \ltx@IfUndefined{DeclareRobustCommand}{%
      \def\eolgrabopt#1%
    }{%
      \newcommand\eolgrabopt{}%
      \DeclareRobustCommand*\eolgrabopt
    }%
  }{%
    \protected\def\eolgrabopt#1%
  }{%
    \begingroup
    \endlinechar=13 %
    \catcode13=\ltx@active
    \eolgrab@opt{#1}%
  }%
}
%    \end{macrocode}
%    \end{macro}
%
%    \begin{macrocode}
\begingroup
  \catcode13=\ltx@active %
\ltx@firstofone{\endgroup %
%    \end{macrocode}
%    \begin{macro}{\eolgrab@}
%    \begin{macrocode}
  \def\eolgrab@#1#2^^M{%
    \endgroup %
    #1{#2}%
  }%
%    \end{macrocode}
%    \end{macro}
%    \begin{macro}{\eolgrab@opt}
%    \begin{macrocode}
  \def\eolgrab@opt#1#2^^M{%
    \endgroup %
    \ltx@ifempty{#2}{%
      #1%
    }{%
      #1[{#2}]%
    }%
  }%
%    \end{macrocode}
%    \end{macro}
%    \begin{macrocode}
}%
%    \end{macrocode}
%
%    \begin{macrocode}
\eolgrab@AtEnd%
%    \end{macrocode}
%    \begin{macrocode}
%</package>
%    \end{macrocode}
%% \section{Installation}
%
% \subsection{Download}
%
% \paragraph{Package.} This package is available on
% CTAN\footnote{\CTANpkg{eolgrab}}:
% \begin{description}
% \item[\CTAN{macros/latex/contrib/oberdiek/eolgrab.dtx}] The source file.
% \item[\CTAN{macros/latex/contrib/oberdiek/eolgrab.pdf}] Documentation.
% \end{description}
%
%
% \paragraph{Bundle.} All the packages of the bundle `oberdiek'
% are also available in a TDS compliant ZIP archive. There
% the packages are already unpacked and the documentation files
% are generated. The files and directories obey the TDS standard.
% \begin{description}
% \item[\CTANinstall{install/macros/latex/contrib/oberdiek.tds.zip}]
% \end{description}
% \emph{TDS} refers to the standard ``A Directory Structure
% for \TeX\ Files'' (\CTANpkg{tds}). Directories
% with \xfile{texmf} in their name are usually organized this way.
%
% \subsection{Bundle installation}
%
% \paragraph{Unpacking.} Unpack the \xfile{oberdiek.tds.zip} in the
% TDS tree (also known as \xfile{texmf} tree) of your choice.
% Example (linux):
% \begin{quote}
%   |unzip oberdiek.tds.zip -d ~/texmf|
% \end{quote}
%
% \subsection{Package installation}
%
% \paragraph{Unpacking.} The \xfile{.dtx} file is a self-extracting
% \docstrip\ archive. The files are extracted by running the
% \xfile{.dtx} through \plainTeX:
% \begin{quote}
%   \verb|tex eolgrab.dtx|
% \end{quote}
%
% \paragraph{TDS.} Now the different files must be moved into
% the different directories in your installation TDS tree
% (also known as \xfile{texmf} tree):
% \begin{quote}
% \def\t{^^A
% \begin{tabular}{@{}>{\ttfamily}l@{ $\rightarrow$ }>{\ttfamily}l@{}}
%   eolgrab.sty & tex/generic/oberdiek/eolgrab.sty\\
%   eolgrab.pdf & doc/latex/oberdiek/eolgrab.pdf\\
%   example/eolgrab-example-ltx.tex & doc/latex/oberdiek/example/eolgrab-example-ltx.tex\\
%   example/eolgrab-example-env.tex & doc/latex/oberdiek/example/eolgrab-example-env.tex\\
%   example/eolgrab-example-sec.tex & doc/latex/oberdiek/example/eolgrab-example-sec.tex\\
%   eolgrab.dtx & source/latex/oberdiek/eolgrab.dtx\\
% \end{tabular}^^A
% }^^A
% \sbox0{\t}^^A
% \ifdim\wd0>\linewidth
%   \begingroup
%     \advance\linewidth by\leftmargin
%     \advance\linewidth by\rightmargin
%   \edef\x{\endgroup
%     \def\noexpand\lw{\the\linewidth}^^A
%   }\x
%   \def\lwbox{^^A
%     \leavevmode
%     \hbox to \linewidth{^^A
%       \kern-\leftmargin\relax
%       \hss
%       \usebox0
%       \hss
%       \kern-\rightmargin\relax
%     }^^A
%   }^^A
%   \ifdim\wd0>\lw
%     \sbox0{\small\t}^^A
%     \ifdim\wd0>\linewidth
%       \ifdim\wd0>\lw
%         \sbox0{\footnotesize\t}^^A
%         \ifdim\wd0>\linewidth
%           \ifdim\wd0>\lw
%             \sbox0{\scriptsize\t}^^A
%             \ifdim\wd0>\linewidth
%               \ifdim\wd0>\lw
%                 \sbox0{\tiny\t}^^A
%                 \ifdim\wd0>\linewidth
%                   \lwbox
%                 \else
%                   \usebox0
%                 \fi
%               \else
%                 \lwbox
%               \fi
%             \else
%               \usebox0
%             \fi
%           \else
%             \lwbox
%           \fi
%         \else
%           \usebox0
%         \fi
%       \else
%         \lwbox
%       \fi
%     \else
%       \usebox0
%     \fi
%   \else
%     \lwbox
%   \fi
% \else
%   \usebox0
% \fi
% \end{quote}
% If you have a \xfile{docstrip.cfg} that configures and enables \docstrip's
% TDS installing feature, then some files can already be in the right
% place, see the documentation of \docstrip.
%
% \subsection{Refresh file name databases}
%
% If your \TeX~distribution
% (\TeX\,Live, \mikTeX, \dots) relies on file name databases, you must refresh
% these. For example, \TeX\,Live\ users run \verb|texhash| or
% \verb|mktexlsr|.
%
% \subsection{Some details for the interested}
%
% \paragraph{Unpacking with \LaTeX.}
% The \xfile{.dtx} chooses its action depending on the format:
% \begin{description}
% \item[\plainTeX:] Run \docstrip\ and extract the files.
% \item[\LaTeX:] Generate the documentation.
% \end{description}
% If you insist on using \LaTeX\ for \docstrip\ (really,
% \docstrip\ does not need \LaTeX), then inform the autodetect routine
% about your intention:
% \begin{quote}
%   \verb|latex \let\install=y\input{eolgrab.dtx}|
% \end{quote}
% Do not forget to quote the argument according to the demands
% of your shell.
%
% \paragraph{Generating the documentation.}
% You can use both the \xfile{.dtx} or the \xfile{.drv} to generate
% the documentation. The process can be configured by the
% configuration file \xfile{ltxdoc.cfg}. For instance, put this
% line into this file, if you want to have A4 as paper format:
% \begin{quote}
%   \verb|\PassOptionsToClass{a4paper}{article}|
% \end{quote}
% An example follows how to generate the
% documentation with pdf\LaTeX:
% \begin{quote}
%\begin{verbatim}
%pdflatex eolgrab.dtx
%makeindex -s gind.ist eolgrab.idx
%pdflatex eolgrab.dtx
%makeindex -s gind.ist eolgrab.idx
%pdflatex eolgrab.dtx
%\end{verbatim}
% \end{quote}
%
% \begin{thebibliography}{9}
% \raggedright
% \bibitem{trautmann-request}
% Arno Trautmann,
% \textit{[texhax] read argument until EOL};
% mailing list \texttt{texthax@tug.org},
% 2011-01-06;
% \url{https://tug.org/pipermail/texhax/2011-January/016517.html}.
%
% \bibitem{taylor}
% Philip Taylor,
% \textit{Re: [texhax] read argument until EOL};
% mailing list \texttt{texhax@tug.org},
% 2011-01-06;
% \url{https://tug.org/pipermail/texhax/2011-January/016519.html}.
%
% \bibitem{oberdiek-eolgrab}
% Heiko Oberdiek,
% \textit{Re: [texhax] read argument until EOL};
% mailing list \texttt{texhax@tug.org},
% 2011-01-06;
% \url{https://tug.org/pipermail/texhax/2011-January/016526.html}.
%
% \end{thebibliography}
%
% \begin{History}
%   \begin{Version}{2011/01/12 v1.0}
%   \item
%     First public version.
%   \end{Version}
%   \begin{Version}{2016/05/16 v1.1}
%   \item
%     Documentation updates.
%   \end{Version}
% \end{History}
%
% \PrintIndex
%
% \Finale
\endinput

%        (quote the arguments according to the demands of your shell)
%
% Documentation:
%    (a) If eolgrab.drv is present:
%           latex eolgrab.drv
%    (b) Without eolgrab.drv:
%           latex eolgrab.dtx; ...
%    The class ltxdoc loads the configuration file ltxdoc.cfg
%    if available. Here you can specify further options, e.g.
%    use A4 as paper format:
%       \PassOptionsToClass{a4paper}{article}
%
%    Programm calls to get the documentation (example):
%       pdflatex eolgrab.dtx
%       makeindex -s gind.ist eolgrab.idx
%       pdflatex eolgrab.dtx
%       makeindex -s gind.ist eolgrab.idx
%       pdflatex eolgrab.dtx
%
% Installation:
%    TDS:tex/generic/oberdiek/eolgrab.sty
%    TDS:doc/latex/oberdiek/eolgrab.pdf
%    TDS:doc/latex/oberdiek/example/eolgrab-example-ltx.tex
%    TDS:doc/latex/oberdiek/example/eolgrab-example-env.tex
%    TDS:doc/latex/oberdiek/example/eolgrab-example-sec.tex
%    TDS:source/latex/oberdiek/eolgrab.dtx
%
%<*ignore>
\begingroup
  \catcode123=1 %
  \catcode125=2 %
  \def\x{LaTeX2e}%
\expandafter\endgroup
\ifcase 0\ifx\install y1\fi\expandafter
         \ifx\csname processbatchFile\endcsname\relax\else1\fi
         \ifx\fmtname\x\else 1\fi\relax
\else\csname fi\endcsname
%</ignore>
%<*install>
\input docstrip.tex
\Msg{************************************************************************}
\Msg{* Installation}
\Msg{* Package: eolgrab 2016/05/16 v1.1 Catch arguments delimited by end of line (HO)}
\Msg{************************************************************************}

\keepsilent
\askforoverwritefalse

\let\MetaPrefix\relax
\preamble

This is a generated file.

Project: eolgrab
Version: 2016/05/16 v1.1

Copyright (C)
   2011 Heiko Oberdiek
   2016-2019 Oberdiek Package Support Group

This work may be distributed and/or modified under the
conditions of the LaTeX Project Public License, either
version 1.3c of this license or (at your option) any later
version. This version of this license is in
   https://www.latex-project.org/lppl/lppl-1-3c.txt
and the latest version of this license is in
   https://www.latex-project.org/lppl.txt
and version 1.3 or later is part of all distributions of
LaTeX version 2005/12/01 or later.

This work has the LPPL maintenance status "maintained".

The Current Maintainers of this work are
Heiko Oberdiek and the Oberdiek Package Support Group
https://github.com/ho-tex/oberdiek/issues


The Base Interpreter refers to any `TeX-Format',
because some files are installed in TDS:tex/generic//.

This work consists of the main source file eolgrab.dtx
and the derived files
   eolgrab.sty, eolgrab.pdf, eolgrab.ins, eolgrab.drv,
   eolgrab-example-ltx.tex, eolgrab-example-env.tex,
   eolgrab-example-sec.tex, eolgrab-test1.tex,
   eolgrab-test2.tex.

\endpreamble
\let\MetaPrefix\DoubleperCent

\generate{%
  \file{eolgrab.ins}{\from{eolgrab.dtx}{install}}%
  \file{eolgrab.drv}{\from{eolgrab.dtx}{driver}}%
  \usedir{tex/generic/oberdiek}%
  \file{eolgrab.sty}{\from{eolgrab.dtx}{package}}%
  \usedir{doc/latex/oberdiek/example}%
  \file{eolgrab-example-ltx.tex}{\from{eolgrab.dtx}{example-ltx}}%
  \file{eolgrab-example-env.tex}{\from{eolgrab.dtx}{example-env}}%
  \file{eolgrab-example-sec.tex}{\from{eolgrab.dtx}{example-sec}}%
%  \usedir{doc/latex/oberdiek/test}%
%  \file{eolgrab-test1.tex}{\from{eolgrab.dtx}{test1}}%
%  \file{eolgrab-test2.tex}{\from{eolgrab.dtx}{test2}}%
}

\catcode32=13\relax% active space
\let =\space%
\Msg{************************************************************************}
\Msg{*}
\Msg{* To finish the installation you have to move the following}
\Msg{* file into a directory searched by TeX:}
\Msg{*}
\Msg{*     eolgrab.sty}
\Msg{*}
\Msg{* To produce the documentation run the file `eolgrab.drv'}
\Msg{* through LaTeX.}
\Msg{*}
\Msg{* Happy TeXing!}
\Msg{*}
\Msg{************************************************************************}

\endbatchfile
%</install>
%<*ignore>
\fi
%</ignore>
%<*driver>
\NeedsTeXFormat{LaTeX2e}
\ProvidesFile{eolgrab.drv}%
  [2016/05/16 v1.1 Catch arguments delimited by end of line (HO)]%
\documentclass{ltxdoc}
\usepackage{holtxdoc}[2011/11/22]
\begin{document}
  \DocInput{eolgrab.dtx}%
\end{document}
%</driver>
% \fi
%
%
%
% \GetFileInfo{eolgrab.drv}
%
% \title{The \xpackage{eolgrab} package}
% \date{2016/05/16 v1.1}
% \author{Heiko Oberdiek\thanks
% {Please report any issues at \url{https://github.com/ho-tex/oberdiek/issues}}}
%
% \maketitle
%
% \begin{abstract}
% This package implements a generic argument grabber
% to catch an argument that is delimited by the line end.
% \end{abstract}
%
% \tableofcontents
%
% \urldef\urltexhax\url{https://tug.org/mailman/listinfo/texhax}
% \def\eol/{^^A
%   \meta{\texttt{EOL}}^^A
% }
%
% \section{Documentation}
%
% The starting point for this package was a feature
% request of Arno Trautmann in the mailing list
% \texttt{texhax}\footnote{Info page for mailing list
% \texttt{texhax}: \urltexhax} \cite{trautmann-request}.
% A macro \cs{eolsection} should behave like \cs{section},
% but the argument should be delimited by the line end
% instead of given in curly braces:
% \begin{quote}
%   |\eolsection My Title|
% \end{quote}
% Phil Taylor answered this with an implementation for \cs{eolsection}.
% Because this feature could be useful for other macros as well,
% I answered with an implementation of \cs{eolgrab} as general solution
% \cite{oberdiek-eolgrab}.
%
% Both formats \hologo{plainTeX} and \hologo{LaTeX} are supported
% by the package, see the example for \cs{eolsection} below.
%
% \begin{declcs}{eolgrab} \M{code} \meta{argument} \eol/
% \end{declcs}
% Macro \cs{eolgrab} takes two arguments. The first argument
% is \meta{code}, a classical undelimited \hologo{TeX} macro argument.
% The second argument is delimited by the line end \eol/.
% The macro calls \meta{code} with \meta{argument} as argument
% in curly braces. Because the catcode of the line end is changed,
% \cs{eolgrab} will not work in the argument of other macros.
% Macro \cs{eolgrab} is made robust if either \hologo{eTeX}'s
% \cs{protected} or \hologo{LaTeX}'s \cs{DeclareRobustCommand}
% is available.
%
% \begin{declcs}{eolgrabopt} \M{code} \meta{argument} \eol/
% \end{declcs}
% Macro \cs{eolgrabopt} passes \meta{argument} as optional
% argument to \meta{code} if \meta{argument} is not empty.
% \begin{quote}
%   |\eolgrabopt\item foo|
% \end{quote}
% becomes to
% \begin{quote}
%   |\item[{foo}]|
% \end{quote}
% The curly argument braces are added to support square brackets
% inside \meta{argument}.
% If the \meta{argument} is empty:
% \begin{quote}
%   |\eolgrabopt\item|
% \end{quote}
% then
% \begin{quote}
%   |\item|
% \end{quote}
% is called without optional argument.
%
% \subsection{Examples}
%
% \begin{itemize}
% \item
%   The line
%   \begin{quote}
%     |\eolgrab\section My Title|
%   \end{quote}
%   is equivalent to
%   \begin{quote}
%     |\section{My Title}|
%   \end{quote}
% \item
%   The next example uses the star form of \cs{section}. Then the
%   command to be called consists of two tokens. Therefore the
%   first argument of \cs{eolgrab} needs curly braces:
%   \begin{quote}
%     |\eolgrab{\section*}My Title|
%   \end{quote}
%   becomes
%   \begin{quote}
%     |\section*{My Title}|
%   \end{quote}
% \item
%   Now \hologo{LaTeX}'s \cs{PackageError} is used. This macro
%   has three arguments, the package or class name, the message
%   text and the help text. A standard help text of \hologo{LaTeX}
%   is used as given in macro \cs{@ehc}. The second argument,
%   the message text is used as argument, delimited by line end:
%   \begin{quote}
%     |\eolgrab{\PackageError{foobar}}%|\\
%     |Some error message text\MessageBreak%|\\
%     |with several lines|\\
%     |\@ehc|
%   \end{quote}
%   In the first two lines of the example, the line end is suppressed
%   by the comment character (percent), thus the argument is delimited
%   by the line end of the third line. The result is:
%   \begin{quote}
%     |\PackageError{foobar}{Some error message text\MessageBreak|\\
%     |with several lines}\@ehc|
%   \end{quote}
% \item
%   The original request for macro \cs{eolsection}, see above,
%   can be implemented easily with the help of \cs{eolgrab}.
%   Example for \hologo{LaTeX}:
%   \begin{quote}
%     |\usepackage{eolgrab}|\\
%     |\newcommand*{\eolsection}{\eolgrab\section}|
%   \end{quote}
%   Example for \hologo{plainTeX}:
%   \begin{quote}
%     |\input eolgrab.sty\relax|\\
%     |\def\eolsection{\eolgrab\section}|
%   \end{quote}
%   And a sophisticated variant for \hologo{LaTeX} that also supports
%   the star syntax and the optional argument:
%    \begin{macrocode}
%<*example-sec>
\documentclass{article}
\usepackage{eolgrab}
\makeatletter
\newcommand*{\eolsection}{%
  \@ifstar{%
    \eolgrab{\section*}%
  }{%
    \@ifnextchar[{%
      \eoloptsection
    }{%
      \eolgrab\section
    }%
  }%
}
\newcommand*{\eoloptsection}[1][]{%
  \eolgrab{\section[{#1}]}%
}
\makeatother
\begin{document}
\tableofcontents
\eolsection Section without star and optional argument
\eolsection*Section with star
\eolsection[Short section title]Long section title
\end{document}
%</example-sec>
%    \end{macrocode}
% \end{itemize}
%
% \subsubsection{Small \hologo{LaTeX} document as example}
%
%    \begin{macrocode}
%<*example-ltx>
\RequirePackage{eolgrab}
\eolgrab\documentclass article
\eolgrab\begin document
\eolgrab\section Hello World
\eolgrab\emph Some text
\eolgrab\end document
%</example-ltx>
%    \end{macrocode}
%
% \subsubsection{\hologo{LaTeX} document with environments}
%
%    \begin{macrocode}
%<*example-env>
\documentclass{article}
\usepackage{eolgrab}
\newcommand*{\Begin}{\eolgrab   \begin}
\newcommand*{\End  }{\eolgrab   \end  }
\newcommand*{\Item }{\eolgrabopt\item }
\Begin document
  \Begin itemize
    \Item
      first item
    \Item
      second item
  \End itemize
  \Begin description
    \Item foo
      is the first syllable of foobar.
    \Item bar
      is the second syllable of foobar.
  \End description
\End document
%</example-env>
%    \end{macrocode}
%
% \subsection{Limitations}
%
% Macro \cs{eolgrab} needs to catch the line end.
% If \hologo{TeX} reads a line, then it throws away the line end
% characters (carriage return, line feed) and removes spaces at the
% end of the line. Then it adds the character with the character code
% that is given by \cs{endlinechar} at the end of the line. The category
% code of the inserted character is given by the current value of
% its \cs{catcode}. If \cs{endlinechar} is not a valid character code
% (especially if it is negative), then no character is added.
%
% In \hologo{plainTeX} and \hologo{LaTeX} the standard settings
% of the inserted endline character is the character with code 13
% (or |^||^||M| in \hologo{TeX} notation) with catcode 5 (end of line).
% That means the inserted end of line character behaves like a space
% token. For example, it is removed after macro names. Therefore
% \cs{eolgrab} changes the catcode.
%
% Therefore \cs{eolgrab} has some limitations:
% \begin{itemize}
% \item
%   Like other verbatim stuff, the macro \cs{eolgrab} cannot be
%   used in the argument of other macros. \cs{eolgrab} want to
%   change the catcode of the end of line character. If this
%   character is read before, because it is processed as argument
%   of another macro, the catcode is already set and is not reassigned
%   later if \cs{eolgrab} changes the category code for this character
%   code.
% \item
%   The argument must not contain the end of line character.
%   Otherwise the first end of line character is already taken
%   as delimiter, leaving the rest of the line outside the argument.
% \item
%   Because \cs{eolgrab} is probably mostly used in the line with
%   the delimited argument. Therefore changes of \cs{endlinechar}
%   will not affect the current line.
% \end{itemize}
%
% \StopEventually{
% }
%
% \section{Implementation}
%
%    \begin{macrocode}
%<*package>
%    \end{macrocode}
%
% \subsection{Reload check and package identification}
%    Reload check, especially if the package is not used with \LaTeX.
%    \begin{macrocode}
\begingroup\catcode61\catcode48\catcode32=10\relax%
  \catcode13=5 % ^^M
  \endlinechar=13 %
  \catcode35=6 % #
  \catcode39=12 % '
  \catcode44=12 % ,
  \catcode45=12 % -
  \catcode46=12 % .
  \catcode58=12 % :
  \catcode64=11 % @
  \catcode123=1 % {
  \catcode125=2 % }
  \expandafter\let\expandafter\x\csname ver@eolgrab.sty\endcsname
  \ifx\x\relax % plain-TeX, first loading
  \else
    \def\empty{}%
    \ifx\x\empty % LaTeX, first loading,
      % variable is initialized, but \ProvidesPackage not yet seen
    \else
      \expandafter\ifx\csname PackageInfo\endcsname\relax
        \def\x#1#2{%
          \immediate\write-1{Package #1 Info: #2.}%
        }%
      \else
        \def\x#1#2{\PackageInfo{#1}{#2, stopped}}%
      \fi
      \x{eolgrab}{The package is already loaded}%
      \aftergroup\endinput
    \fi
  \fi
\endgroup%
%    \end{macrocode}
%    Package identification:
%    \begin{macrocode}
\begingroup\catcode61\catcode48\catcode32=10\relax%
  \catcode13=5 % ^^M
  \endlinechar=13 %
  \catcode35=6 % #
  \catcode39=12 % '
  \catcode40=12 % (
  \catcode41=12 % )
  \catcode44=12 % ,
  \catcode45=12 % -
  \catcode46=12 % .
  \catcode47=12 % /
  \catcode58=12 % :
  \catcode64=11 % @
  \catcode91=12 % [
  \catcode93=12 % ]
  \catcode123=1 % {
  \catcode125=2 % }
  \expandafter\ifx\csname ProvidesPackage\endcsname\relax
    \def\x#1#2#3[#4]{\endgroup
      \immediate\write-1{Package: #3 #4}%
      \xdef#1{#4}%
    }%
  \else
    \def\x#1#2[#3]{\endgroup
      #2[{#3}]%
      \ifx#1\@undefined
        \xdef#1{#3}%
      \fi
      \ifx#1\relax
        \xdef#1{#3}%
      \fi
    }%
  \fi
\expandafter\x\csname ver@eolgrab.sty\endcsname
\ProvidesPackage{eolgrab}%
  [2016/05/16 v1.1 Catch arguments delimited by end of line (HO)]%
%    \end{macrocode}
%
% \subsection{Catcodes}
%
%    \begin{macrocode}
\begingroup\catcode61\catcode48\catcode32=10\relax%
  \catcode13=5 % ^^M
  \endlinechar=13 %
  \catcode123=1 % {
  \catcode125=2 % }
  \catcode64=11 % @
  \def\x{\endgroup
    \expandafter\edef\csname eolgrab@AtEnd\endcsname{%
      \endlinechar=\the\endlinechar\relax
      \catcode13=\the\catcode13\relax
      \catcode32=\the\catcode32\relax
      \catcode35=\the\catcode35\relax
      \catcode61=\the\catcode61\relax
      \catcode64=\the\catcode64\relax
      \catcode123=\the\catcode123\relax
      \catcode125=\the\catcode125\relax
    }%
  }%
\x\catcode61\catcode48\catcode32=10\relax%
\catcode13=5 % ^^M
\endlinechar=13 %
\catcode35=6 % #
\catcode64=11 % @
\catcode123=1 % {
\catcode125=2 % }
\def\TMP@EnsureCode#1#2{%
  \edef\eolgrab@AtEnd{%
    \eolgrab@AtEnd
    \catcode#1=\the\catcode#1\relax
  }%
  \catcode#1=#2\relax
}
\TMP@EnsureCode{40}{12}% (
\TMP@EnsureCode{41}{12}% )
\TMP@EnsureCode{42}{12}% *
\TMP@EnsureCode{46}{12}% .
\TMP@EnsureCode{47}{12}% /
\TMP@EnsureCode{91}{12}% [
\TMP@EnsureCode{93}{12}% ]
\TMP@EnsureCode{94}{7}% ^
\edef\eolgrab@AtEnd{\eolgrab@AtEnd\noexpand\endinput}
%    \end{macrocode}
%
% \subsection{Resources}
%
%    \begin{macrocode}
\begingroup\expandafter\expandafter\expandafter\endgroup
\expandafter\ifx\csname RequirePackage\endcsname\relax
  \input ltxcmds.sty\relax
  \input infwarerr.sty\relax
\else
  \RequirePackage{ltxcmds}[2010/12/04]%
  \RequirePackage{infwarerr}[2010/04/08]%
\fi
%    \end{macrocode}
%
%    \begin{macro}{\eolgrab@ifdefinable}
%    \begin{macrocode}
\ltx@IfUndefined{@ifdefinable}{%
  \def\eolgrab@ifdefinable#1#2{%
    \ltx@ifundefined{#1}{#2}{%
      \@PackageError{eolgrab}{%
        Command \ltx@backslashchar#1 already defined%
      }\@ehc
    }%
  }%
}{%
  \def\eolgrab@ifdefinable#1{%
    \expandafter\@ifdefinable\csname#1\endcsname
  }%
}
%    \end{macrocode}
%    \end{macro}
%
% \subsection{Macro \cs{eolgrab}}
%
%    \begin{macro}{\eolgrab}
%    \begin{macrocode}
\eolgrab@ifdefinable{eolgrab}{%
  \ltx@IfUndefined{protected}{%
    \ltx@IfUndefined{DeclareRobustCommand}{%
      \def\eolgrab#1%
    }{%
      \newcommand\eolgrab{}%
      \DeclareRobustCommand*\eolgrab
    }%
  }{%
    \protected\def\eolgrab#1%
  }{%
    \begingroup
    \endlinechar=13 %
    \catcode13=\ltx@active
    \eolgrab@{#1}%
  }%
}
%    \end{macrocode}
%    \end{macro}
%
%    \begin{macro}{\eolgrabopt}
%    \begin{macrocode}
\eolgrab@ifdefinable{eolgrabopt}{%
  \ltx@IfUndefined{protected}{%
    \ltx@IfUndefined{DeclareRobustCommand}{%
      \def\eolgrabopt#1%
    }{%
      \newcommand\eolgrabopt{}%
      \DeclareRobustCommand*\eolgrabopt
    }%
  }{%
    \protected\def\eolgrabopt#1%
  }{%
    \begingroup
    \endlinechar=13 %
    \catcode13=\ltx@active
    \eolgrab@opt{#1}%
  }%
}
%    \end{macrocode}
%    \end{macro}
%
%    \begin{macrocode}
\begingroup
  \catcode13=\ltx@active %
\ltx@firstofone{\endgroup %
%    \end{macrocode}
%    \begin{macro}{\eolgrab@}
%    \begin{macrocode}
  \def\eolgrab@#1#2^^M{%
    \endgroup %
    #1{#2}%
  }%
%    \end{macrocode}
%    \end{macro}
%    \begin{macro}{\eolgrab@opt}
%    \begin{macrocode}
  \def\eolgrab@opt#1#2^^M{%
    \endgroup %
    \ltx@ifempty{#2}{%
      #1%
    }{%
      #1[{#2}]%
    }%
  }%
%    \end{macrocode}
%    \end{macro}
%    \begin{macrocode}
}%
%    \end{macrocode}
%
%    \begin{macrocode}
\eolgrab@AtEnd%
%    \end{macrocode}
%    \begin{macrocode}
%</package>
%    \end{macrocode}
%% \section{Installation}
%
% \subsection{Download}
%
% \paragraph{Package.} This package is available on
% CTAN\footnote{\CTANpkg{eolgrab}}:
% \begin{description}
% \item[\CTAN{macros/latex/contrib/oberdiek/eolgrab.dtx}] The source file.
% \item[\CTAN{macros/latex/contrib/oberdiek/eolgrab.pdf}] Documentation.
% \end{description}
%
%
% \paragraph{Bundle.} All the packages of the bundle `oberdiek'
% are also available in a TDS compliant ZIP archive. There
% the packages are already unpacked and the documentation files
% are generated. The files and directories obey the TDS standard.
% \begin{description}
% \item[\CTANinstall{install/macros/latex/contrib/oberdiek.tds.zip}]
% \end{description}
% \emph{TDS} refers to the standard ``A Directory Structure
% for \TeX\ Files'' (\CTANpkg{tds}). Directories
% with \xfile{texmf} in their name are usually organized this way.
%
% \subsection{Bundle installation}
%
% \paragraph{Unpacking.} Unpack the \xfile{oberdiek.tds.zip} in the
% TDS tree (also known as \xfile{texmf} tree) of your choice.
% Example (linux):
% \begin{quote}
%   |unzip oberdiek.tds.zip -d ~/texmf|
% \end{quote}
%
% \subsection{Package installation}
%
% \paragraph{Unpacking.} The \xfile{.dtx} file is a self-extracting
% \docstrip\ archive. The files are extracted by running the
% \xfile{.dtx} through \plainTeX:
% \begin{quote}
%   \verb|tex eolgrab.dtx|
% \end{quote}
%
% \paragraph{TDS.} Now the different files must be moved into
% the different directories in your installation TDS tree
% (also known as \xfile{texmf} tree):
% \begin{quote}
% \def\t{^^A
% \begin{tabular}{@{}>{\ttfamily}l@{ $\rightarrow$ }>{\ttfamily}l@{}}
%   eolgrab.sty & tex/generic/oberdiek/eolgrab.sty\\
%   eolgrab.pdf & doc/latex/oberdiek/eolgrab.pdf\\
%   example/eolgrab-example-ltx.tex & doc/latex/oberdiek/example/eolgrab-example-ltx.tex\\
%   example/eolgrab-example-env.tex & doc/latex/oberdiek/example/eolgrab-example-env.tex\\
%   example/eolgrab-example-sec.tex & doc/latex/oberdiek/example/eolgrab-example-sec.tex\\
%   eolgrab.dtx & source/latex/oberdiek/eolgrab.dtx\\
% \end{tabular}^^A
% }^^A
% \sbox0{\t}^^A
% \ifdim\wd0>\linewidth
%   \begingroup
%     \advance\linewidth by\leftmargin
%     \advance\linewidth by\rightmargin
%   \edef\x{\endgroup
%     \def\noexpand\lw{\the\linewidth}^^A
%   }\x
%   \def\lwbox{^^A
%     \leavevmode
%     \hbox to \linewidth{^^A
%       \kern-\leftmargin\relax
%       \hss
%       \usebox0
%       \hss
%       \kern-\rightmargin\relax
%     }^^A
%   }^^A
%   \ifdim\wd0>\lw
%     \sbox0{\small\t}^^A
%     \ifdim\wd0>\linewidth
%       \ifdim\wd0>\lw
%         \sbox0{\footnotesize\t}^^A
%         \ifdim\wd0>\linewidth
%           \ifdim\wd0>\lw
%             \sbox0{\scriptsize\t}^^A
%             \ifdim\wd0>\linewidth
%               \ifdim\wd0>\lw
%                 \sbox0{\tiny\t}^^A
%                 \ifdim\wd0>\linewidth
%                   \lwbox
%                 \else
%                   \usebox0
%                 \fi
%               \else
%                 \lwbox
%               \fi
%             \else
%               \usebox0
%             \fi
%           \else
%             \lwbox
%           \fi
%         \else
%           \usebox0
%         \fi
%       \else
%         \lwbox
%       \fi
%     \else
%       \usebox0
%     \fi
%   \else
%     \lwbox
%   \fi
% \else
%   \usebox0
% \fi
% \end{quote}
% If you have a \xfile{docstrip.cfg} that configures and enables \docstrip's
% TDS installing feature, then some files can already be in the right
% place, see the documentation of \docstrip.
%
% \subsection{Refresh file name databases}
%
% If your \TeX~distribution
% (\TeX\,Live, \mikTeX, \dots) relies on file name databases, you must refresh
% these. For example, \TeX\,Live\ users run \verb|texhash| or
% \verb|mktexlsr|.
%
% \subsection{Some details for the interested}
%
% \paragraph{Unpacking with \LaTeX.}
% The \xfile{.dtx} chooses its action depending on the format:
% \begin{description}
% \item[\plainTeX:] Run \docstrip\ and extract the files.
% \item[\LaTeX:] Generate the documentation.
% \end{description}
% If you insist on using \LaTeX\ for \docstrip\ (really,
% \docstrip\ does not need \LaTeX), then inform the autodetect routine
% about your intention:
% \begin{quote}
%   \verb|latex \let\install=y% \iffalse meta-comment
%
% File: eolgrab.dtx
% Version: 2016/05/16 v1.1
% Info: Catch arguments delimited by end of line
%
% Copyright (C)
%    2011 Heiko Oberdiek
%    2016-2019 Oberdiek Package Support Group
%    https://github.com/ho-tex/oberdiek/issues
%
% This work may be distributed and/or modified under the
% conditions of the LaTeX Project Public License, either
% version 1.3c of this license or (at your option) any later
% version. This version of this license is in
%    https://www.latex-project.org/lppl/lppl-1-3c.txt
% and the latest version of this license is in
%    https://www.latex-project.org/lppl.txt
% and version 1.3 or later is part of all distributions of
% LaTeX version 2005/12/01 or later.
%
% This work has the LPPL maintenance status "maintained".
%
% The Current Maintainers of this work are
% Heiko Oberdiek and the Oberdiek Package Support Group
% https://github.com/ho-tex/oberdiek/issues
%
% The Base Interpreter refers to any `TeX-Format',
% because some files are installed in TDS:tex/generic//.
%
% This work consists of the main source file eolgrab.dtx
% and the derived files
%    eolgrab.sty, eolgrab.pdf, eolgrab.ins, eolgrab.drv,
%    eolgrab-example-ltx.tex, eolgrab-example-env.tex,
%    eolgrab-example-sec.tex, eolgrab-test1.tex,
%    eolgrab-test2.tex.
%
% Distribution:
%    CTAN:macros/latex/contrib/oberdiek/eolgrab.dtx
%    CTAN:macros/latex/contrib/oberdiek/eolgrab.pdf
%
% Unpacking:
%    (a) If eolgrab.ins is present:
%           tex eolgrab.ins
%    (b) Without eolgrab.ins:
%           tex eolgrab.dtx
%    (c) If you insist on using LaTeX
%           latex \let\install=y\input{eolgrab.dtx}
%        (quote the arguments according to the demands of your shell)
%
% Documentation:
%    (a) If eolgrab.drv is present:
%           latex eolgrab.drv
%    (b) Without eolgrab.drv:
%           latex eolgrab.dtx; ...
%    The class ltxdoc loads the configuration file ltxdoc.cfg
%    if available. Here you can specify further options, e.g.
%    use A4 as paper format:
%       \PassOptionsToClass{a4paper}{article}
%
%    Programm calls to get the documentation (example):
%       pdflatex eolgrab.dtx
%       makeindex -s gind.ist eolgrab.idx
%       pdflatex eolgrab.dtx
%       makeindex -s gind.ist eolgrab.idx
%       pdflatex eolgrab.dtx
%
% Installation:
%    TDS:tex/generic/oberdiek/eolgrab.sty
%    TDS:doc/latex/oberdiek/eolgrab.pdf
%    TDS:doc/latex/oberdiek/example/eolgrab-example-ltx.tex
%    TDS:doc/latex/oberdiek/example/eolgrab-example-env.tex
%    TDS:doc/latex/oberdiek/example/eolgrab-example-sec.tex
%    TDS:source/latex/oberdiek/eolgrab.dtx
%
%<*ignore>
\begingroup
  \catcode123=1 %
  \catcode125=2 %
  \def\x{LaTeX2e}%
\expandafter\endgroup
\ifcase 0\ifx\install y1\fi\expandafter
         \ifx\csname processbatchFile\endcsname\relax\else1\fi
         \ifx\fmtname\x\else 1\fi\relax
\else\csname fi\endcsname
%</ignore>
%<*install>
\input docstrip.tex
\Msg{************************************************************************}
\Msg{* Installation}
\Msg{* Package: eolgrab 2016/05/16 v1.1 Catch arguments delimited by end of line (HO)}
\Msg{************************************************************************}

\keepsilent
\askforoverwritefalse

\let\MetaPrefix\relax
\preamble

This is a generated file.

Project: eolgrab
Version: 2016/05/16 v1.1

Copyright (C)
   2011 Heiko Oberdiek
   2016-2019 Oberdiek Package Support Group

This work may be distributed and/or modified under the
conditions of the LaTeX Project Public License, either
version 1.3c of this license or (at your option) any later
version. This version of this license is in
   https://www.latex-project.org/lppl/lppl-1-3c.txt
and the latest version of this license is in
   https://www.latex-project.org/lppl.txt
and version 1.3 or later is part of all distributions of
LaTeX version 2005/12/01 or later.

This work has the LPPL maintenance status "maintained".

The Current Maintainers of this work are
Heiko Oberdiek and the Oberdiek Package Support Group
https://github.com/ho-tex/oberdiek/issues


The Base Interpreter refers to any `TeX-Format',
because some files are installed in TDS:tex/generic//.

This work consists of the main source file eolgrab.dtx
and the derived files
   eolgrab.sty, eolgrab.pdf, eolgrab.ins, eolgrab.drv,
   eolgrab-example-ltx.tex, eolgrab-example-env.tex,
   eolgrab-example-sec.tex, eolgrab-test1.tex,
   eolgrab-test2.tex.

\endpreamble
\let\MetaPrefix\DoubleperCent

\generate{%
  \file{eolgrab.ins}{\from{eolgrab.dtx}{install}}%
  \file{eolgrab.drv}{\from{eolgrab.dtx}{driver}}%
  \usedir{tex/generic/oberdiek}%
  \file{eolgrab.sty}{\from{eolgrab.dtx}{package}}%
  \usedir{doc/latex/oberdiek/example}%
  \file{eolgrab-example-ltx.tex}{\from{eolgrab.dtx}{example-ltx}}%
  \file{eolgrab-example-env.tex}{\from{eolgrab.dtx}{example-env}}%
  \file{eolgrab-example-sec.tex}{\from{eolgrab.dtx}{example-sec}}%
%  \usedir{doc/latex/oberdiek/test}%
%  \file{eolgrab-test1.tex}{\from{eolgrab.dtx}{test1}}%
%  \file{eolgrab-test2.tex}{\from{eolgrab.dtx}{test2}}%
}

\catcode32=13\relax% active space
\let =\space%
\Msg{************************************************************************}
\Msg{*}
\Msg{* To finish the installation you have to move the following}
\Msg{* file into a directory searched by TeX:}
\Msg{*}
\Msg{*     eolgrab.sty}
\Msg{*}
\Msg{* To produce the documentation run the file `eolgrab.drv'}
\Msg{* through LaTeX.}
\Msg{*}
\Msg{* Happy TeXing!}
\Msg{*}
\Msg{************************************************************************}

\endbatchfile
%</install>
%<*ignore>
\fi
%</ignore>
%<*driver>
\NeedsTeXFormat{LaTeX2e}
\ProvidesFile{eolgrab.drv}%
  [2016/05/16 v1.1 Catch arguments delimited by end of line (HO)]%
\documentclass{ltxdoc}
\usepackage{holtxdoc}[2011/11/22]
\begin{document}
  \DocInput{eolgrab.dtx}%
\end{document}
%</driver>
% \fi
%
%
%
% \GetFileInfo{eolgrab.drv}
%
% \title{The \xpackage{eolgrab} package}
% \date{2016/05/16 v1.1}
% \author{Heiko Oberdiek\thanks
% {Please report any issues at \url{https://github.com/ho-tex/oberdiek/issues}}}
%
% \maketitle
%
% \begin{abstract}
% This package implements a generic argument grabber
% to catch an argument that is delimited by the line end.
% \end{abstract}
%
% \tableofcontents
%
% \urldef\urltexhax\url{https://tug.org/mailman/listinfo/texhax}
% \def\eol/{^^A
%   \meta{\texttt{EOL}}^^A
% }
%
% \section{Documentation}
%
% The starting point for this package was a feature
% request of Arno Trautmann in the mailing list
% \texttt{texhax}\footnote{Info page for mailing list
% \texttt{texhax}: \urltexhax} \cite{trautmann-request}.
% A macro \cs{eolsection} should behave like \cs{section},
% but the argument should be delimited by the line end
% instead of given in curly braces:
% \begin{quote}
%   |\eolsection My Title|
% \end{quote}
% Phil Taylor answered this with an implementation for \cs{eolsection}.
% Because this feature could be useful for other macros as well,
% I answered with an implementation of \cs{eolgrab} as general solution
% \cite{oberdiek-eolgrab}.
%
% Both formats \hologo{plainTeX} and \hologo{LaTeX} are supported
% by the package, see the example for \cs{eolsection} below.
%
% \begin{declcs}{eolgrab} \M{code} \meta{argument} \eol/
% \end{declcs}
% Macro \cs{eolgrab} takes two arguments. The first argument
% is \meta{code}, a classical undelimited \hologo{TeX} macro argument.
% The second argument is delimited by the line end \eol/.
% The macro calls \meta{code} with \meta{argument} as argument
% in curly braces. Because the catcode of the line end is changed,
% \cs{eolgrab} will not work in the argument of other macros.
% Macro \cs{eolgrab} is made robust if either \hologo{eTeX}'s
% \cs{protected} or \hologo{LaTeX}'s \cs{DeclareRobustCommand}
% is available.
%
% \begin{declcs}{eolgrabopt} \M{code} \meta{argument} \eol/
% \end{declcs}
% Macro \cs{eolgrabopt} passes \meta{argument} as optional
% argument to \meta{code} if \meta{argument} is not empty.
% \begin{quote}
%   |\eolgrabopt\item foo|
% \end{quote}
% becomes to
% \begin{quote}
%   |\item[{foo}]|
% \end{quote}
% The curly argument braces are added to support square brackets
% inside \meta{argument}.
% If the \meta{argument} is empty:
% \begin{quote}
%   |\eolgrabopt\item|
% \end{quote}
% then
% \begin{quote}
%   |\item|
% \end{quote}
% is called without optional argument.
%
% \subsection{Examples}
%
% \begin{itemize}
% \item
%   The line
%   \begin{quote}
%     |\eolgrab\section My Title|
%   \end{quote}
%   is equivalent to
%   \begin{quote}
%     |\section{My Title}|
%   \end{quote}
% \item
%   The next example uses the star form of \cs{section}. Then the
%   command to be called consists of two tokens. Therefore the
%   first argument of \cs{eolgrab} needs curly braces:
%   \begin{quote}
%     |\eolgrab{\section*}My Title|
%   \end{quote}
%   becomes
%   \begin{quote}
%     |\section*{My Title}|
%   \end{quote}
% \item
%   Now \hologo{LaTeX}'s \cs{PackageError} is used. This macro
%   has three arguments, the package or class name, the message
%   text and the help text. A standard help text of \hologo{LaTeX}
%   is used as given in macro \cs{@ehc}. The second argument,
%   the message text is used as argument, delimited by line end:
%   \begin{quote}
%     |\eolgrab{\PackageError{foobar}}%|\\
%     |Some error message text\MessageBreak%|\\
%     |with several lines|\\
%     |\@ehc|
%   \end{quote}
%   In the first two lines of the example, the line end is suppressed
%   by the comment character (percent), thus the argument is delimited
%   by the line end of the third line. The result is:
%   \begin{quote}
%     |\PackageError{foobar}{Some error message text\MessageBreak|\\
%     |with several lines}\@ehc|
%   \end{quote}
% \item
%   The original request for macro \cs{eolsection}, see above,
%   can be implemented easily with the help of \cs{eolgrab}.
%   Example for \hologo{LaTeX}:
%   \begin{quote}
%     |\usepackage{eolgrab}|\\
%     |\newcommand*{\eolsection}{\eolgrab\section}|
%   \end{quote}
%   Example for \hologo{plainTeX}:
%   \begin{quote}
%     |\input eolgrab.sty\relax|\\
%     |\def\eolsection{\eolgrab\section}|
%   \end{quote}
%   And a sophisticated variant for \hologo{LaTeX} that also supports
%   the star syntax and the optional argument:
%    \begin{macrocode}
%<*example-sec>
\documentclass{article}
\usepackage{eolgrab}
\makeatletter
\newcommand*{\eolsection}{%
  \@ifstar{%
    \eolgrab{\section*}%
  }{%
    \@ifnextchar[{%
      \eoloptsection
    }{%
      \eolgrab\section
    }%
  }%
}
\newcommand*{\eoloptsection}[1][]{%
  \eolgrab{\section[{#1}]}%
}
\makeatother
\begin{document}
\tableofcontents
\eolsection Section without star and optional argument
\eolsection*Section with star
\eolsection[Short section title]Long section title
\end{document}
%</example-sec>
%    \end{macrocode}
% \end{itemize}
%
% \subsubsection{Small \hologo{LaTeX} document as example}
%
%    \begin{macrocode}
%<*example-ltx>
\RequirePackage{eolgrab}
\eolgrab\documentclass article
\eolgrab\begin document
\eolgrab\section Hello World
\eolgrab\emph Some text
\eolgrab\end document
%</example-ltx>
%    \end{macrocode}
%
% \subsubsection{\hologo{LaTeX} document with environments}
%
%    \begin{macrocode}
%<*example-env>
\documentclass{article}
\usepackage{eolgrab}
\newcommand*{\Begin}{\eolgrab   \begin}
\newcommand*{\End  }{\eolgrab   \end  }
\newcommand*{\Item }{\eolgrabopt\item }
\Begin document
  \Begin itemize
    \Item
      first item
    \Item
      second item
  \End itemize
  \Begin description
    \Item foo
      is the first syllable of foobar.
    \Item bar
      is the second syllable of foobar.
  \End description
\End document
%</example-env>
%    \end{macrocode}
%
% \subsection{Limitations}
%
% Macro \cs{eolgrab} needs to catch the line end.
% If \hologo{TeX} reads a line, then it throws away the line end
% characters (carriage return, line feed) and removes spaces at the
% end of the line. Then it adds the character with the character code
% that is given by \cs{endlinechar} at the end of the line. The category
% code of the inserted character is given by the current value of
% its \cs{catcode}. If \cs{endlinechar} is not a valid character code
% (especially if it is negative), then no character is added.
%
% In \hologo{plainTeX} and \hologo{LaTeX} the standard settings
% of the inserted endline character is the character with code 13
% (or |^||^||M| in \hologo{TeX} notation) with catcode 5 (end of line).
% That means the inserted end of line character behaves like a space
% token. For example, it is removed after macro names. Therefore
% \cs{eolgrab} changes the catcode.
%
% Therefore \cs{eolgrab} has some limitations:
% \begin{itemize}
% \item
%   Like other verbatim stuff, the macro \cs{eolgrab} cannot be
%   used in the argument of other macros. \cs{eolgrab} want to
%   change the catcode of the end of line character. If this
%   character is read before, because it is processed as argument
%   of another macro, the catcode is already set and is not reassigned
%   later if \cs{eolgrab} changes the category code for this character
%   code.
% \item
%   The argument must not contain the end of line character.
%   Otherwise the first end of line character is already taken
%   as delimiter, leaving the rest of the line outside the argument.
% \item
%   Because \cs{eolgrab} is probably mostly used in the line with
%   the delimited argument. Therefore changes of \cs{endlinechar}
%   will not affect the current line.
% \end{itemize}
%
% \StopEventually{
% }
%
% \section{Implementation}
%
%    \begin{macrocode}
%<*package>
%    \end{macrocode}
%
% \subsection{Reload check and package identification}
%    Reload check, especially if the package is not used with \LaTeX.
%    \begin{macrocode}
\begingroup\catcode61\catcode48\catcode32=10\relax%
  \catcode13=5 % ^^M
  \endlinechar=13 %
  \catcode35=6 % #
  \catcode39=12 % '
  \catcode44=12 % ,
  \catcode45=12 % -
  \catcode46=12 % .
  \catcode58=12 % :
  \catcode64=11 % @
  \catcode123=1 % {
  \catcode125=2 % }
  \expandafter\let\expandafter\x\csname ver@eolgrab.sty\endcsname
  \ifx\x\relax % plain-TeX, first loading
  \else
    \def\empty{}%
    \ifx\x\empty % LaTeX, first loading,
      % variable is initialized, but \ProvidesPackage not yet seen
    \else
      \expandafter\ifx\csname PackageInfo\endcsname\relax
        \def\x#1#2{%
          \immediate\write-1{Package #1 Info: #2.}%
        }%
      \else
        \def\x#1#2{\PackageInfo{#1}{#2, stopped}}%
      \fi
      \x{eolgrab}{The package is already loaded}%
      \aftergroup\endinput
    \fi
  \fi
\endgroup%
%    \end{macrocode}
%    Package identification:
%    \begin{macrocode}
\begingroup\catcode61\catcode48\catcode32=10\relax%
  \catcode13=5 % ^^M
  \endlinechar=13 %
  \catcode35=6 % #
  \catcode39=12 % '
  \catcode40=12 % (
  \catcode41=12 % )
  \catcode44=12 % ,
  \catcode45=12 % -
  \catcode46=12 % .
  \catcode47=12 % /
  \catcode58=12 % :
  \catcode64=11 % @
  \catcode91=12 % [
  \catcode93=12 % ]
  \catcode123=1 % {
  \catcode125=2 % }
  \expandafter\ifx\csname ProvidesPackage\endcsname\relax
    \def\x#1#2#3[#4]{\endgroup
      \immediate\write-1{Package: #3 #4}%
      \xdef#1{#4}%
    }%
  \else
    \def\x#1#2[#3]{\endgroup
      #2[{#3}]%
      \ifx#1\@undefined
        \xdef#1{#3}%
      \fi
      \ifx#1\relax
        \xdef#1{#3}%
      \fi
    }%
  \fi
\expandafter\x\csname ver@eolgrab.sty\endcsname
\ProvidesPackage{eolgrab}%
  [2016/05/16 v1.1 Catch arguments delimited by end of line (HO)]%
%    \end{macrocode}
%
% \subsection{Catcodes}
%
%    \begin{macrocode}
\begingroup\catcode61\catcode48\catcode32=10\relax%
  \catcode13=5 % ^^M
  \endlinechar=13 %
  \catcode123=1 % {
  \catcode125=2 % }
  \catcode64=11 % @
  \def\x{\endgroup
    \expandafter\edef\csname eolgrab@AtEnd\endcsname{%
      \endlinechar=\the\endlinechar\relax
      \catcode13=\the\catcode13\relax
      \catcode32=\the\catcode32\relax
      \catcode35=\the\catcode35\relax
      \catcode61=\the\catcode61\relax
      \catcode64=\the\catcode64\relax
      \catcode123=\the\catcode123\relax
      \catcode125=\the\catcode125\relax
    }%
  }%
\x\catcode61\catcode48\catcode32=10\relax%
\catcode13=5 % ^^M
\endlinechar=13 %
\catcode35=6 % #
\catcode64=11 % @
\catcode123=1 % {
\catcode125=2 % }
\def\TMP@EnsureCode#1#2{%
  \edef\eolgrab@AtEnd{%
    \eolgrab@AtEnd
    \catcode#1=\the\catcode#1\relax
  }%
  \catcode#1=#2\relax
}
\TMP@EnsureCode{40}{12}% (
\TMP@EnsureCode{41}{12}% )
\TMP@EnsureCode{42}{12}% *
\TMP@EnsureCode{46}{12}% .
\TMP@EnsureCode{47}{12}% /
\TMP@EnsureCode{91}{12}% [
\TMP@EnsureCode{93}{12}% ]
\TMP@EnsureCode{94}{7}% ^
\edef\eolgrab@AtEnd{\eolgrab@AtEnd\noexpand\endinput}
%    \end{macrocode}
%
% \subsection{Resources}
%
%    \begin{macrocode}
\begingroup\expandafter\expandafter\expandafter\endgroup
\expandafter\ifx\csname RequirePackage\endcsname\relax
  \input ltxcmds.sty\relax
  \input infwarerr.sty\relax
\else
  \RequirePackage{ltxcmds}[2010/12/04]%
  \RequirePackage{infwarerr}[2010/04/08]%
\fi
%    \end{macrocode}
%
%    \begin{macro}{\eolgrab@ifdefinable}
%    \begin{macrocode}
\ltx@IfUndefined{@ifdefinable}{%
  \def\eolgrab@ifdefinable#1#2{%
    \ltx@ifundefined{#1}{#2}{%
      \@PackageError{eolgrab}{%
        Command \ltx@backslashchar#1 already defined%
      }\@ehc
    }%
  }%
}{%
  \def\eolgrab@ifdefinable#1{%
    \expandafter\@ifdefinable\csname#1\endcsname
  }%
}
%    \end{macrocode}
%    \end{macro}
%
% \subsection{Macro \cs{eolgrab}}
%
%    \begin{macro}{\eolgrab}
%    \begin{macrocode}
\eolgrab@ifdefinable{eolgrab}{%
  \ltx@IfUndefined{protected}{%
    \ltx@IfUndefined{DeclareRobustCommand}{%
      \def\eolgrab#1%
    }{%
      \newcommand\eolgrab{}%
      \DeclareRobustCommand*\eolgrab
    }%
  }{%
    \protected\def\eolgrab#1%
  }{%
    \begingroup
    \endlinechar=13 %
    \catcode13=\ltx@active
    \eolgrab@{#1}%
  }%
}
%    \end{macrocode}
%    \end{macro}
%
%    \begin{macro}{\eolgrabopt}
%    \begin{macrocode}
\eolgrab@ifdefinable{eolgrabopt}{%
  \ltx@IfUndefined{protected}{%
    \ltx@IfUndefined{DeclareRobustCommand}{%
      \def\eolgrabopt#1%
    }{%
      \newcommand\eolgrabopt{}%
      \DeclareRobustCommand*\eolgrabopt
    }%
  }{%
    \protected\def\eolgrabopt#1%
  }{%
    \begingroup
    \endlinechar=13 %
    \catcode13=\ltx@active
    \eolgrab@opt{#1}%
  }%
}
%    \end{macrocode}
%    \end{macro}
%
%    \begin{macrocode}
\begingroup
  \catcode13=\ltx@active %
\ltx@firstofone{\endgroup %
%    \end{macrocode}
%    \begin{macro}{\eolgrab@}
%    \begin{macrocode}
  \def\eolgrab@#1#2^^M{%
    \endgroup %
    #1{#2}%
  }%
%    \end{macrocode}
%    \end{macro}
%    \begin{macro}{\eolgrab@opt}
%    \begin{macrocode}
  \def\eolgrab@opt#1#2^^M{%
    \endgroup %
    \ltx@ifempty{#2}{%
      #1%
    }{%
      #1[{#2}]%
    }%
  }%
%    \end{macrocode}
%    \end{macro}
%    \begin{macrocode}
}%
%    \end{macrocode}
%
%    \begin{macrocode}
\eolgrab@AtEnd%
%    \end{macrocode}
%    \begin{macrocode}
%</package>
%    \end{macrocode}
%% \section{Installation}
%
% \subsection{Download}
%
% \paragraph{Package.} This package is available on
% CTAN\footnote{\CTANpkg{eolgrab}}:
% \begin{description}
% \item[\CTAN{macros/latex/contrib/oberdiek/eolgrab.dtx}] The source file.
% \item[\CTAN{macros/latex/contrib/oberdiek/eolgrab.pdf}] Documentation.
% \end{description}
%
%
% \paragraph{Bundle.} All the packages of the bundle `oberdiek'
% are also available in a TDS compliant ZIP archive. There
% the packages are already unpacked and the documentation files
% are generated. The files and directories obey the TDS standard.
% \begin{description}
% \item[\CTANinstall{install/macros/latex/contrib/oberdiek.tds.zip}]
% \end{description}
% \emph{TDS} refers to the standard ``A Directory Structure
% for \TeX\ Files'' (\CTANpkg{tds}). Directories
% with \xfile{texmf} in their name are usually organized this way.
%
% \subsection{Bundle installation}
%
% \paragraph{Unpacking.} Unpack the \xfile{oberdiek.tds.zip} in the
% TDS tree (also known as \xfile{texmf} tree) of your choice.
% Example (linux):
% \begin{quote}
%   |unzip oberdiek.tds.zip -d ~/texmf|
% \end{quote}
%
% \subsection{Package installation}
%
% \paragraph{Unpacking.} The \xfile{.dtx} file is a self-extracting
% \docstrip\ archive. The files are extracted by running the
% \xfile{.dtx} through \plainTeX:
% \begin{quote}
%   \verb|tex eolgrab.dtx|
% \end{quote}
%
% \paragraph{TDS.} Now the different files must be moved into
% the different directories in your installation TDS tree
% (also known as \xfile{texmf} tree):
% \begin{quote}
% \def\t{^^A
% \begin{tabular}{@{}>{\ttfamily}l@{ $\rightarrow$ }>{\ttfamily}l@{}}
%   eolgrab.sty & tex/generic/oberdiek/eolgrab.sty\\
%   eolgrab.pdf & doc/latex/oberdiek/eolgrab.pdf\\
%   example/eolgrab-example-ltx.tex & doc/latex/oberdiek/example/eolgrab-example-ltx.tex\\
%   example/eolgrab-example-env.tex & doc/latex/oberdiek/example/eolgrab-example-env.tex\\
%   example/eolgrab-example-sec.tex & doc/latex/oberdiek/example/eolgrab-example-sec.tex\\
%   eolgrab.dtx & source/latex/oberdiek/eolgrab.dtx\\
% \end{tabular}^^A
% }^^A
% \sbox0{\t}^^A
% \ifdim\wd0>\linewidth
%   \begingroup
%     \advance\linewidth by\leftmargin
%     \advance\linewidth by\rightmargin
%   \edef\x{\endgroup
%     \def\noexpand\lw{\the\linewidth}^^A
%   }\x
%   \def\lwbox{^^A
%     \leavevmode
%     \hbox to \linewidth{^^A
%       \kern-\leftmargin\relax
%       \hss
%       \usebox0
%       \hss
%       \kern-\rightmargin\relax
%     }^^A
%   }^^A
%   \ifdim\wd0>\lw
%     \sbox0{\small\t}^^A
%     \ifdim\wd0>\linewidth
%       \ifdim\wd0>\lw
%         \sbox0{\footnotesize\t}^^A
%         \ifdim\wd0>\linewidth
%           \ifdim\wd0>\lw
%             \sbox0{\scriptsize\t}^^A
%             \ifdim\wd0>\linewidth
%               \ifdim\wd0>\lw
%                 \sbox0{\tiny\t}^^A
%                 \ifdim\wd0>\linewidth
%                   \lwbox
%                 \else
%                   \usebox0
%                 \fi
%               \else
%                 \lwbox
%               \fi
%             \else
%               \usebox0
%             \fi
%           \else
%             \lwbox
%           \fi
%         \else
%           \usebox0
%         \fi
%       \else
%         \lwbox
%       \fi
%     \else
%       \usebox0
%     \fi
%   \else
%     \lwbox
%   \fi
% \else
%   \usebox0
% \fi
% \end{quote}
% If you have a \xfile{docstrip.cfg} that configures and enables \docstrip's
% TDS installing feature, then some files can already be in the right
% place, see the documentation of \docstrip.
%
% \subsection{Refresh file name databases}
%
% If your \TeX~distribution
% (\TeX\,Live, \mikTeX, \dots) relies on file name databases, you must refresh
% these. For example, \TeX\,Live\ users run \verb|texhash| or
% \verb|mktexlsr|.
%
% \subsection{Some details for the interested}
%
% \paragraph{Unpacking with \LaTeX.}
% The \xfile{.dtx} chooses its action depending on the format:
% \begin{description}
% \item[\plainTeX:] Run \docstrip\ and extract the files.
% \item[\LaTeX:] Generate the documentation.
% \end{description}
% If you insist on using \LaTeX\ for \docstrip\ (really,
% \docstrip\ does not need \LaTeX), then inform the autodetect routine
% about your intention:
% \begin{quote}
%   \verb|latex \let\install=y\input{eolgrab.dtx}|
% \end{quote}
% Do not forget to quote the argument according to the demands
% of your shell.
%
% \paragraph{Generating the documentation.}
% You can use both the \xfile{.dtx} or the \xfile{.drv} to generate
% the documentation. The process can be configured by the
% configuration file \xfile{ltxdoc.cfg}. For instance, put this
% line into this file, if you want to have A4 as paper format:
% \begin{quote}
%   \verb|\PassOptionsToClass{a4paper}{article}|
% \end{quote}
% An example follows how to generate the
% documentation with pdf\LaTeX:
% \begin{quote}
%\begin{verbatim}
%pdflatex eolgrab.dtx
%makeindex -s gind.ist eolgrab.idx
%pdflatex eolgrab.dtx
%makeindex -s gind.ist eolgrab.idx
%pdflatex eolgrab.dtx
%\end{verbatim}
% \end{quote}
%
% \begin{thebibliography}{9}
% \raggedright
% \bibitem{trautmann-request}
% Arno Trautmann,
% \textit{[texhax] read argument until EOL};
% mailing list \texttt{texthax@tug.org},
% 2011-01-06;
% \url{https://tug.org/pipermail/texhax/2011-January/016517.html}.
%
% \bibitem{taylor}
% Philip Taylor,
% \textit{Re: [texhax] read argument until EOL};
% mailing list \texttt{texhax@tug.org},
% 2011-01-06;
% \url{https://tug.org/pipermail/texhax/2011-January/016519.html}.
%
% \bibitem{oberdiek-eolgrab}
% Heiko Oberdiek,
% \textit{Re: [texhax] read argument until EOL};
% mailing list \texttt{texhax@tug.org},
% 2011-01-06;
% \url{https://tug.org/pipermail/texhax/2011-January/016526.html}.
%
% \end{thebibliography}
%
% \begin{History}
%   \begin{Version}{2011/01/12 v1.0}
%   \item
%     First public version.
%   \end{Version}
%   \begin{Version}{2016/05/16 v1.1}
%   \item
%     Documentation updates.
%   \end{Version}
% \end{History}
%
% \PrintIndex
%
% \Finale
\endinput
|
% \end{quote}
% Do not forget to quote the argument according to the demands
% of your shell.
%
% \paragraph{Generating the documentation.}
% You can use both the \xfile{.dtx} or the \xfile{.drv} to generate
% the documentation. The process can be configured by the
% configuration file \xfile{ltxdoc.cfg}. For instance, put this
% line into this file, if you want to have A4 as paper format:
% \begin{quote}
%   \verb|\PassOptionsToClass{a4paper}{article}|
% \end{quote}
% An example follows how to generate the
% documentation with pdf\LaTeX:
% \begin{quote}
%\begin{verbatim}
%pdflatex eolgrab.dtx
%makeindex -s gind.ist eolgrab.idx
%pdflatex eolgrab.dtx
%makeindex -s gind.ist eolgrab.idx
%pdflatex eolgrab.dtx
%\end{verbatim}
% \end{quote}
%
% \begin{thebibliography}{9}
% \raggedright
% \bibitem{trautmann-request}
% Arno Trautmann,
% \textit{[texhax] read argument until EOL};
% mailing list \texttt{texthax@tug.org},
% 2011-01-06;
% \url{https://tug.org/pipermail/texhax/2011-January/016517.html}.
%
% \bibitem{taylor}
% Philip Taylor,
% \textit{Re: [texhax] read argument until EOL};
% mailing list \texttt{texhax@tug.org},
% 2011-01-06;
% \url{https://tug.org/pipermail/texhax/2011-January/016519.html}.
%
% \bibitem{oberdiek-eolgrab}
% Heiko Oberdiek,
% \textit{Re: [texhax] read argument until EOL};
% mailing list \texttt{texhax@tug.org},
% 2011-01-06;
% \url{https://tug.org/pipermail/texhax/2011-January/016526.html}.
%
% \end{thebibliography}
%
% \begin{History}
%   \begin{Version}{2011/01/12 v1.0}
%   \item
%     First public version.
%   \end{Version}
%   \begin{Version}{2016/05/16 v1.1}
%   \item
%     Documentation updates.
%   \end{Version}
% \end{History}
%
% \PrintIndex
%
% \Finale
\endinput
|
% \end{quote}
% Do not forget to quote the argument according to the demands
% of your shell.
%
% \paragraph{Generating the documentation.}
% You can use both the \xfile{.dtx} or the \xfile{.drv} to generate
% the documentation. The process can be configured by the
% configuration file \xfile{ltxdoc.cfg}. For instance, put this
% line into this file, if you want to have A4 as paper format:
% \begin{quote}
%   \verb|\PassOptionsToClass{a4paper}{article}|
% \end{quote}
% An example follows how to generate the
% documentation with pdf\LaTeX:
% \begin{quote}
%\begin{verbatim}
%pdflatex eolgrab.dtx
%makeindex -s gind.ist eolgrab.idx
%pdflatex eolgrab.dtx
%makeindex -s gind.ist eolgrab.idx
%pdflatex eolgrab.dtx
%\end{verbatim}
% \end{quote}
%
% \begin{thebibliography}{9}
% \raggedright
% \bibitem{trautmann-request}
% Arno Trautmann,
% \textit{[texhax] read argument until EOL};
% mailing list \texttt{texthax@tug.org},
% 2011-01-06;
% \url{https://tug.org/pipermail/texhax/2011-January/016517.html}.
%
% \bibitem{taylor}
% Philip Taylor,
% \textit{Re: [texhax] read argument until EOL};
% mailing list \texttt{texhax@tug.org},
% 2011-01-06;
% \url{https://tug.org/pipermail/texhax/2011-January/016519.html}.
%
% \bibitem{oberdiek-eolgrab}
% Heiko Oberdiek,
% \textit{Re: [texhax] read argument until EOL};
% mailing list \texttt{texhax@tug.org},
% 2011-01-06;
% \url{https://tug.org/pipermail/texhax/2011-January/016526.html}.
%
% \end{thebibliography}
%
% \begin{History}
%   \begin{Version}{2011/01/12 v1.0}
%   \item
%     First public version.
%   \end{Version}
%   \begin{Version}{2016/05/16 v1.1}
%   \item
%     Documentation updates.
%   \end{Version}
% \end{History}
%
% \PrintIndex
%
% \Finale
\endinput
|
% \end{quote}
% Do not forget to quote the argument according to the demands
% of your shell.
%
% \paragraph{Generating the documentation.}
% You can use both the \xfile{.dtx} or the \xfile{.drv} to generate
% the documentation. The process can be configured by the
% configuration file \xfile{ltxdoc.cfg}. For instance, put this
% line into this file, if you want to have A4 as paper format:
% \begin{quote}
%   \verb|\PassOptionsToClass{a4paper}{article}|
% \end{quote}
% An example follows how to generate the
% documentation with pdf\LaTeX:
% \begin{quote}
%\begin{verbatim}
%pdflatex eolgrab.dtx
%makeindex -s gind.ist eolgrab.idx
%pdflatex eolgrab.dtx
%makeindex -s gind.ist eolgrab.idx
%pdflatex eolgrab.dtx
%\end{verbatim}
% \end{quote}
%
% \begin{thebibliography}{9}
% \raggedright
% \bibitem{trautmann-request}
% Arno Trautmann,
% \textit{[texhax] read argument until EOL};
% mailing list \texttt{texthax@tug.org},
% 2011-01-06;
% \url{https://tug.org/pipermail/texhax/2011-January/016517.html}.
%
% \bibitem{taylor}
% Philip Taylor,
% \textit{Re: [texhax] read argument until EOL};
% mailing list \texttt{texhax@tug.org},
% 2011-01-06;
% \url{https://tug.org/pipermail/texhax/2011-January/016519.html}.
%
% \bibitem{oberdiek-eolgrab}
% Heiko Oberdiek,
% \textit{Re: [texhax] read argument until EOL};
% mailing list \texttt{texhax@tug.org},
% 2011-01-06;
% \url{https://tug.org/pipermail/texhax/2011-January/016526.html}.
%
% \end{thebibliography}
%
% \begin{History}
%   \begin{Version}{2011/01/12 v1.0}
%   \item
%     First public version.
%   \end{Version}
%   \begin{Version}{2016/05/16 v1.1}
%   \item
%     Documentation updates.
%   \end{Version}
% \end{History}
%
% \PrintIndex
%
% \Finale
\endinput
