% ======================================================================
% tocbasic.tex
% Copyright (c) Markus Kohm, 2002-2021
%
% This file is part of the LaTeX2e KOMA-Script bundle.
%
% This work may be distributed and/or modified under the conditions of
% the LaTeX Project Public License, version 1.3c of the license.
% The latest version of this license is in
%   http://www.latex-project.org/lppl.txt
% and version 1.3c or later is part of all distributions of LaTeX 
% version 2005/12/01 or later and of this work.
%
% This work has the LPPL maintenance status "author-maintained".
%
% The Current Maintainer and author of this work is Markus Kohm.
%
% This work consists of all files listed in manifest.txt.
% ----------------------------------------------------------------------
% tocbasic.tex
% Copyright (c) Markus Kohm, 2002-2021
%
% Dieses Werk darf nach den Bedingungen der LaTeX Project Public Lizenz,
% Version 1.3c, verteilt und/oder veraendert werden.
% Die neuste Version dieser Lizenz ist
%   http://www.latex-project.org/lppl.txt
% und Version 1.3c ist Teil aller Verteilungen von LaTeX
% Version 2005/12/01 oder spaeter und dieses Werks.
%
% Dieses Werk hat den LPPL-Verwaltungs-Status "author-maintained"
% (allein durch den Autor verwaltet).
%
% Der Aktuelle Verwalter und Autor dieses Werkes ist Markus Kohm.
% 
% Dieses Werk besteht aus den in manifest.txt aufgefuehrten Dateien.
% ======================================================================
%
% Package tocbasic for Package and Class Authors
% Maintained by Markus Kohm
%
% ----------------------------------------------------------------------
%
% Paket tocbasic für Paket- und Klassenautoren
% Verwaltet von Markus Kohm
%
% ======================================================================

\KOMAProvidesFile{tocbasic.tex}
                 [$Date: 2021-10-15 13:47:06 +0200 (Fri, 15 Oct 2021) $
                  KOMA-Script guide (package tocbasic)]

\chapter{Verzeichnisse verwalten mit Hilfe von \Package{tocbasic}}
\labelbase{tocbasic}

\BeginIndexGroup
\BeginIndex{Package}{tocbasic}%
\BeginIndex{}{Verzeichnis}%
\BeginIndex{}{Dateierweiterung}%
\index{Dateiendung|see{Dateierweiterung}}%|
Der Hauptzweck des Pakets \Package{tocbasic} besteht darin, Paket- und
Klassenautoren die Möglichkeit zu geben, eigene Verzeichnisse vergleichbar mit
dem Abbildungs- und Tabellenverzeichnis zu erstellen und dabei Klassen und
anderen Paketen einen Teil der Kontrolle über diese Verzeichnisse zu
erlauben. Dabei sorgt das Paket \Package{tocbasic} auch dafür, dass diese
Verzeichnisse von \Package{babel}\IndexPackage{babel} (siehe
\cite{package:babel}) bei der Sprachumschaltung mit berücksichtigt
werden. Durch Verwendung von \Package{tocbasic} soll dem Paketautor die Mühe
genommen werden, selbst solche Anpassungen an andere Pakete oder an Klassen
vornehmen zu müssen.

Als kleiner Nebeneffekt kann das Paket auch verwendet werden, um neue
Gleitumgebungen oder den Gleitumgebungen ähnliche nicht gleitende Umgebungen
für Konsultationsobjekte zu definieren. Näheres dazu wird nach der Erklärung
der grundlegenden Anweisungen in den folgenden vier Abschnitten durch ein
Beispiel in \autoref{sec:tocbasic.example} verdeutlicht, das in kompakter Form
noch einmal in \autoref{sec:tocbasic.declarenewtoc} aufgegriffen wird.

\KOMAScript{} verwendet \Package{tocbasic} sowohl für das Inhaltsverzeichnis
als auch für die bereits erwähnten Verzeichnisse für Abbildungen und Tabellen.


\section{Grundlegende Anweisungen}
\seclabel{basics}

Die grundlegenden Anweisungen dienen in erster Linie dazu, eine Liste aller
bekannten Dateierweiterungen\textnote{Dateierweiterung, Verzeichnis}, die für
Verzeichnisse stehen, zu verwalten. Einträge in Dateien mit solchen
Dateierweiterungen werden typischerweise mit
\Macro{addtocontents}\important{\Macro{addtocontents},
  \DescRef{\LabelBase.cmd.addxcontentsline}} oder
\DescRef{\LabelBase.cmd.addxcontentsline} vorgenommen. Darüber hinaus gibt es
Anweisungen, mit denen Aktionen für all diese Dateierweiterungen durchgeführt
werden können. Außerdem gibt es Anweisungen, um Einstellungen für die Datei
vorzunehmen, die zu einer gegebenen Dateierweiterung gehört. Typischerweise
hat so eine Dateierweiterung auch einen
Besitzer\textnote{Dateibesitzer}. Dieser Besitzer kann eine Klasse oder ein
Paket oder die Bezeichnung einer Kategorie sein, die der Autor der Klasse oder
des Pakets, das \Package{tocbasic} verwendet, frei gewählt hat. \KOMAScript{}
selbst verwendet beispielsweise die Kategorie \PValue{float} für die
Dateierweiterungen \File{lof} und \File{lot}, die für das Abbildungs- und das
Tabellenverzeichnis stehen. Für das Inhaltsverzeichnis verwendet \KOMAScript{}
als Besitzer den Dateinamen der Klasse.

\begin{Declaration}
  \Macro{Ifattoclist}\Parameter{Dateierweiterung}
                     \Parameter{Dann-Teil}\Parameter{Sonst-Teil}
\end{Declaration}
Mit\ChangedAt{v3.28}{\Package{tocbasic}} dieser Anweisung wird überprüft, ob
die \PName{Dateierweiterung} bereits in der Liste der bekannten
Dateierweiterungen vorhanden ist oder nicht. Ist die \PName{Dateierweiterung}
bereits über diese Liste bekannt, so wird der \PName{Dann-Teil}
ausgeführt. Anderenfalls wird der \PName{Sonst-Teil} ausgeführt.
\begin{Example}
  Angenommen, Sie wollen wissen, ob die Dateierweiterung »\File{foo}« bereits
  verwendet wird, um in diesem Fall eine Fehlermeldung auszugeben, weil diese
  damit nicht mehr verwendet werden kann:
\begin{lstcode}
  \Ifattoclist{foo}{%
    \PackageError{bar}{%
      extension `foo' already in use%
    }{%
      Each extension may be used only
      once.\MessageBreak
      The class or another package already
      uses extension `foo'.\MessageBreak
      This error is fatal!\MessageBreak
      You should not continue!}%
  }{%
    \PackageInfo{bar}{using extension `foo'}%
  }
\end{lstcode}
\end{Example}
\EndIndexGroup%
\ExampleEndFix


\begin{Declaration}
  \Macro{addtotoclist}\OParameter{Besitzer}\Parameter{Dateierweiterung}
\end{Declaration}
Diese Anweisung fügt die \PName{Dateierweiterung} der Liste der bekannten
Dateierweiterungen hinzu. Ist die \PName{Dateierweiterung} bereits bekannt, so
wird hingegen ein Fehler gemeldet, um die doppelte Verwendung derselben
\PName{Dateierweiterung} zu verhindern.

Wenn das optionale Argument \OParameter{Besitzer} angegeben wurde, wird der
angegebene \PName{Besitzer} für diese Dateierweiterung mit gespeichert. Wurde
das optionale Argument weggelassen, dann versucht \Package{tocbasic} den
Dateinamen der aktuell abgearbeiteten Klasse oder des Pakets herauszufinden
und als \PName{Besitzer} zu speichern. Dies\textnote{Achtung!} funktioniert
nur, wenn \Macro{addtotoclist} während des Ladens der Klasse oder des Pakets
aufgerufen wird. Es funktioniert nicht, wenn \Macro{addtotoclist} erst später
aufgrund der Verwendung einer Anweisung durch den Benutzer aufgerufen wird. In
diesem Fall wird ein leerer \PName{Besitzer} eingetragen. 

Beachten\textnote{Achtung!} Sie, dass ein leeres Argument \PName{Besitzer}
nicht immer das Gleiche ist wie das Weglassen des kompletten optionalen
Arguments einschließlich der eckigen Klammern. Ein leeres Argument würde immer
einen leeren \PName{Besitzer} ergeben.
\begin{Example}
  Angenommen, Sie wollen die Dateierweiterung »\File{foo}« der Liste der
  bekannten Dateierweiterungen hinzufügen, während Ihr Paket mit dem
  Dateinamen »\File{bar.sty}« geladen wird:
\begin{lstcode}
  \addtotoclist{foo}
\end{lstcode}
  Dies fügt die Dateierweiterung »\PValue{foo}« mit dem Besitzer
  »\PValue{bar.sty}« der Liste der bekannten Dateierweiterung hinzu, wenn diese
  Erweiterung nicht bereits in der Liste ist. Wenn die verwendete
  Klasse oder ein anderes Paket diese Dateierweiterung schon angemeldet hat,
  erhalten Sie den Fehler:
\begin{lstoutput}[breakatwhitespace]
  Package tocbasic Error: file extension `foo' cannot be used twice

  See the tocbasic package documentation for explanation.
  Type H <return> for immediate help.
\end{lstoutput}
  Wenn Sie dann tatsächlich die Taste »\texttt{H}«, gefolgt von der
  Eingabe-Taste drücken, erhalten Sie als Hilfe:
\begin{lstoutput}[breakatwhitespace]
  File extension `foo' is already used by a toc-file, while bar.sty
  tried to use it again for a toc-file.
  This may be either an incompatibility of packages, an error at a package,
  or a mistake by the user.
\end{lstoutput}

  Vielleicht stellt Ihr Paket auch eine Anweisung bereit, die ein
  Verzeichnis dynamisch erzeugt. In diesem Fall sollten Sie das
  optionale Argument von \Macro{addtotoclist} verwenden, um den
  \PName{Besitzer} anzugeben:
\begin{lstcode}
  \newcommand*{\createnewlistofsomething}[1]{%
    \addtotoclist[bar.sty]{#1}%
    % Weitere Aktionen, um dieses Verzeichnis 
    % verfügbar zu machen
  }
\end{lstcode}
  Wenn jetzt der Anwender diese Anweisung aufruft, beispielsweise mit
\begin{lstcode}
  \createnewlistofsomething{foo}
\end{lstcode}
  dann wird die Dateierweiterung »\PValue{foo}« ebenfalls mit dem Besitzer
  »\PValue{bar.sty}« zur Liste der bekannten Dateierweiterungen hinzugefügt
  oder aber ein Fehler gemeldet, wenn diese Dateierweiterung bereits verwendet
  wird. 
\end{Example}

Sie können als \PName{Besitzer} angeben, was immer Sie wollen, aber es sollte
eindeutig sein! Wenn Sie beispielsweise der Autor des Pakets \Package{float}
wären, könnten Sie als \PName{Besitzer} auch die Kategorie »\PValue{float}«
anstelle von »\PValue{float.sty}« angeben. In diesem Fall würden die
\KOMAScript-Optionen\important{\DescRef{maincls.option.listof}}%
\IndexOption{listof~=\PName{Einstellung}} für das Verzeichnis der Abbildungen
und das Verzeichnis der Tabellen auch Ihre Verzeichnisse betreffen. Das liegt
daran, dass \KOMAScript{} die Dateierweiterungen »\PValue{lof}« für das
Abbildungsverzeichnis und »\PValue{lot}« für das Tabellenverzeichnis mit der
Kategorie »\PValue{float}« als \PName{Besitzer} anmeldet und die Optionen für
diesen Besitzer setzt.

Das Paket \hyperref[cha:scrhack]{\Package{scrhack}}\IndexPackage{scrhack}%
\important{ \hyperref[cha:scrhack]{\Package{scrhack}}} enthält übrigens
Patches für mehrere Pakete wie \Package{float} oder \Package{listings}, die
eigene Verzeichnisse bereitstellen. Bei Verwendung von
\hyperref[cha:scrhack]{\Package{scrhack}} wird unter anderem die jeweilige
Dateierweiterung der Liste der bekannten Dateierweiterungen hinzugefügt. Dabei
wird als \PName{Besitzer} »\PValue{float}« verwendet. Dies ist sozusagen der
grundlegende Baustein, um die Möglichkeiten von \Package{tocbasic} und der
\KOMAScript-Klassen auch für diese Verzeichnisse automatisch nutzen zu
können.%
\EndIndexGroup


\begin{Declaration}
  \Macro{AtAddToTocList}\OParameter{Besitzer}\Parameter{Anweisungen}
\end{Declaration}
Auf diese Weise können die \PName{Anweisungen} zu einer internen Liste von
Anweisungen hinzugefügt werden, die immer dann auszuführen sind, wenn eine
Dateierweiterung mit dem angegebenen \PName{Besitzer} zur Liste der bekannten
Dateierweiterungen hinzugefügt wird. Bezüglich des optionalen Arguments wird
wie in der Erklärung von \DescRef{\LabelBase.cmd.addtotoclist} beschrieben
verfahren. Wird das optionale Argument leer gelassen, werden in diesem Fall
die Aktionen unabhängig vom Besitzer immer ausgeführt, wenn die
Dateierweiterung zu der Liste der bekannten Dateierweiterungen hinzugefügt
wird. Während der Ausführung der \PName{Anweisungen} ist außerdem
\Macro{@currext}\IndexCmd{@currext}\important{\Macro{@currext}} die
Dateierweiterung, die gerade hinzugefügt wird.
\begin{Example}
  \Package{tocbasic} selbst verwendet
  % Umbruchkorrektur (schließende Klammer verschoben)
\begin{lstcode}
  \AtAddToTocList[]{%
    \expandafter\tocbasic@extend@babel
    \expandafter{\@currext}}
\end{lstcode}
  um jede Dateierweiterung zu der in \Package{tocbasic} vorhandenen
  Erweiterung für das Paket \Package{babel} hinzuzufügen. 
\end{Example}

Die\textnote{Achtung!} zweimalige Verwendung von \Macro{expandafter} ist im
Beispiel erforderlich, weil das Argument von
\DescRef{\LabelBase.cmd.tocbasic@extend@babel} zwingend bereits expandiert
sein muss. Siehe dazu auch die Erklärung zu
\DescRef{\LabelBase.cmd.tocbasic@extend@babel} in
\autoref{sec:tocbasic.internals},
\DescPageRef{\LabelBase.cmd.tocbasic@extend@babel}.
%
\EndIndexGroup


\begin{Declaration}
  \Macro{removefromtoclist}\OParameter{Besitzer}\Parameter{Dateierweiterung}
\end{Declaration}
Man kann eine \PName{Dateierweiterung} auch wieder aus der Liste der bekannten
Dateierweiterungen entfernen. Ist das optionale Argument \OParameter{Besitzer}
angegeben, so wird die Dateierweiterung nur entfernt, wenn sie für den
angegebenen \PName{Besitzer} angemeldet wurde. Das gilt auch für den leeren
\PName{Besitzer}. Wird dagegen gar kein \OParameter{Besitzer} angegeben,
entfallen also auch die eckigen Klammern, findet kein Besitzertest statt,
sondern die \PName{Dateierweiterung} wird unabhängig vom Besitzer entfernt.%
\EndIndexGroup


\begin{Declaration}
  \Macro{doforeachtocfile}\OParameter{Besitzer}\Parameter{Anweisungen}
\end{Declaration}
Bisher haben Sie nur Anweisungen kennengelernt, die für Klassen- und
Paketautoren zwar zusätzliche Sicherheit, aber auch eher zusätzlichen Aufwand
bedeuten. Mit \Macro{doforeachtocfile} kann man die erste Ernte dafür
einfahren. Diese Anweisung erlaubt es, die angegebenen \PName{Anweisungen} für
jede mit dem \PName{Besitzer} angemeldete Dateierweiterung
auszuführen. Während der Ausführung der \PName{Anweisungen} ist
\Macro{@currext}\IndexCmd{@currext}\important{\Macro{@currext}} die aktuell
verarbeitete Dateierweiterung. Wird das optionale Argument
\OParameter{Besitzer} weggelassen, so werden alle Dateierweiterungen
unabhängig vom Besitzer abgearbeitet. Ein leeres optionales Argument würde
hingegen nur die Dateierweiterungen mit leerem Besitzer verarbeiten.
\begin{Example}
  Wenn Sie die Liste aller bekannten Dateierweiterungen auf das Terminal und
  in die \File{log}-Datei ausgeben wollen, ist dies einfach mit
\begin{lstcode}
  \doforeachtocfile{\typeout{\@currext}}
\end{lstcode}
  möglich. Sollen hingegen nur die Dateierweiterungen des Besitzers
  »\PValue{foo}« ausgegeben werden, geht das einfach mit:
\begin{lstcode}
  \doforeachtocfile[foo]{\typeout{\@currext}}
\end{lstcode}
\end{Example}
Die \KOMAScript-Klassen \Class{scrbook} und \Class{scrreprt} verwenden die
Anweisung, um für Verzeichnisse, für die Eigenschaft \PValue{chapteratlist}
gesetzt ist, optional einen vertikalen Abstand oder die Kapitelüberschrift in
das Verzeichnis einzutragen. Wie Sie diese Eigenschaft setzen können, ist
in \autoref{sec:tocbasic.toc} ab \DescPageRef{\LabelBase.cmd.setuptoc} zu
finden.%
\EndIndexGroup


\begin{Declaration}
  \Macro{tocbasicautomode}
\end{Declaration}
Diese Anweisung definiert das vom \LaTeX-Kern für Klassen- und Paketautoren
bereitgestellte
\Macro{@starttoc}\IndexCmd{@starttoc}\important{\Macro{\@starttoc}} so um,
dass bei jedem Aufruf von \Macro{@starttoc} die dabei angegebene
Dateierweiterung in die Liste der bekannten Dateierweiterungen eingefügt wird,
soweit sie dort noch nicht vorhanden ist. Außerdem wird dann
\DescRef{\LabelBase.cmd.tocbasic@starttoc} anstelle von \Macro{@starttoc}
verwendet. Näheres zu \DescRef{\LabelBase.cmd.tocbasic@starttoc} und
\Macro{@starttoc} ist \autoref{sec:tocbasic.internals},
\DescPageRef{\LabelBase.cmd.tocbasic@starttoc} zu entnehmen.

Mit Hilfe von \Macro{tocbasicautomode} wird also jedes Verzeichnis, das mit
Hilfe von \Macro{@starttoc} erstellt wird, automatisch unter die Kontrolle von
\Package{tocbasic} gestellt. Ob das zum gewünschten Ergebnis führt, hängt
jedoch sehr von den jeweiligen Verzeichnissen ab. Immerhin funktioniert damit
schon einmal die Erweiterung für das \Package{babel}-Paket für alle
Verzeichnisse. Es ist jedoch vorzuziehen, wenn der Paketautor selbst
\Package{tocbasic} explizit verwendet. Er kann dann auch die weiteren Vorteile
nutzen, die ihm das Paket bietet und die in den nachfolgenden Abschnitten
beschrieben werden.%
\EndIndexGroup


\section{Erzeugen eines Verzeichnisses}
\seclabel{toc}

Im vorherigen Abschnitt haben Sie erfahren, wie eine Liste bekannter
Dateierweiterungen verwaltet werden kann\iffree{ und wie automatisch Anweisungen
  beim Hinzufügen von Dateierweiterungen zu dieser Liste ausgeführt werden
  können}{}. Des Weiteren haben Sie eine Anweisung kennengelernt, mit der man
für jede einzelne bekannte Dateierweiterung oder einen spezifischen Teil davon
Anweisungen ausführen kann. In diesem Abschnitt werden Sie Anweisungen
kennenlernen, die sich auf die Datei beziehen, die mit dieser Dateierweiterung
verbunden ist.


\begin{Declaration}
  \Macro{addtoeachtocfile}\OParameter{Besitzer}\Parameter{Inhalt}
\end{Declaration}
Die Anweisung \Macro{addtoeachtocfile} schreibt \PName{Inhalt} mit Hilfe von
\Macro{addtocontents}\IndexCmd{addtocontents}\important{\Macro{addtocontents}}%
\iffree{ aus dem \LaTeX-Kern}{} in jede Datei, die mit dem angegebenen
\PName{Besitzer} in der Liste der bekannten Dateierweiterungen \iffree{zu
  finden ist}{steht}. Wird das optionale Argument weggelassen, wird in jede
Datei aus der Liste der bekannten Dateierweiterungen geschrieben.\iffree{ Der
  konkrete Dateiname setzt sich dabei übrigens aus \Macro{jobname} und der
  Dateierweiterung zusammen.}{} Während des Schreibens von \PName{Inhalt} ist
\Macro{@currext}\IndexCmd{@currext}\important{\Macro{@currext}} die
Dateierweiterung der Datei, in die aktuell geschrieben wird.
\begin{Example}
  Sie wollen einen vertikalen Abstand von einer Zeile in alle Dateien aus der
  Liste der bekannten Dateierweiterungen schreiben.
\begin{lstcode}
  \addtoeachtocfile{%
    \protect\addvspace{\protect\baselineskip}%
  }%
\end{lstcode}
  Wenn Sie das hingegen nur für die Dateien mit dem definierten Besitzer
  »\PValue{foo}« machen wollen, verwenden Sie:
\begin{lstcode}
  \addtoeachtocfile[foo]{%
    \protect\addvspace{\protect\baselineskip}%
  }
\end{lstcode}%
\end{Example}%
\iffree{Anweisungen, die nicht bereits beim Schreiben expandiert werden
  sollen, sind wie bei \Macro{addtocontents} mit \Macro{protect} zu
  schützen.}{\ExampleEndFix}%
\EndIndexGroup


\begin{Declaration}
  \Macro{addxcontentsline}\Parameter{Dateierweiterung}\Parameter{Ebene}
                          \OParameter{Gliederungsnummer}\Parameter{Inhalt}
\end{Declaration}
Diese\ChangedAt{v3.12}{\Package{tocbasic}} Anweisung ähnelt sehr der Anweisung
\Macro{addcontentsline}\IndexCmd{addcontentsline} aus dem
\LaTeX-Kern. Allerdings besitzt sie ein zusätzliches optionales Argument für
die \PName{Gliederungsnummer} des Eintrags, während diese bei
\Macro{addcontentsline} im Argument \PName{Inhalt} mit angegeben wird. Sie
wird verwendet, um nummerierte oder nicht nummerierte Einträge in das über die
\PName{Dateierweiterung} spezifizierte Verzeichnis aufzunehmen. Dabei ist
\PName{Ebene} der symbolische Name der Gliederungsebene und \PName{Inhalt} der
entsprechende Eintrag. Die Seitenzahl wird automatisch bestimmt.

Im Unterschied zu \Macro{addcontentsline} testet \Macro{addxcontentsline}
zunächst, ob Anweisung \Macro{add\PName{Ebene}\PName{Dateierweiterung}entry}
definiert ist. In diesem Fall wird sie für den Eintrag verwendet, wobei
\PName{Gliederungsnummer} als optionales Argument und \PName{Inhalt} als
obligatorisches Argument übergeben wird. Ein Beispiel für eine solche
Anweisung, die von den \KOMAScript-Klassen bereitgestellt wird, wäre
\DescRef{maincls-experts.cmd.addparttocentry} (siehe
\autoref{sec:maincls-experts.toc},
\DescPageRef{maincls-experts.cmd.addparttocentry}). Ist die entsprechende
Anweisung nicht definiert, wird stattdessen die interne Anweisung
\DescRef{\LabelBase.cmd.tocbasic@addxcontentsline} verwendet. Diese erhält
alle vier Argumente als obligatorische Argumente und verwendet dann
ihrerseits \Macro{addcontentsline}, um den gewünschten Eintrag
vorzunehmen. Näheres zu \DescRef{\LabelBase.cmd.tocbasic@addxcontentsline} ist
\autoref{sec:tocbasic.internals},
\DescPageRef{\LabelBase.cmd.tocbasic@addxcontentsline} zu entnehmen.

Ein Vorteil der Verwendung von \Macro{addxcontentsline} gegenüber
\Macro{addcontentsline} ist zum einen, dass die Eigenschaft
\PValue{numberline} (siehe \DescPageRef{\LabelBase.cmd.setuptoc}) beachtet
wird. Zum anderen kann die Form der Einträge über die Definition
entsprechender, für die \PName{Ebene} und \PName{Dateierweiterung}
spezifischer Anweisungen konfiguriert werden.%
%
\EndIndexGroup


\begin{Declaration}
  \Macro{addxcontentslinetoeachtocfile}\OParameter{Besitzer}
                                       \Parameter{Ebene}
                                       \OParameter{Gliederungsnummer}%
                                       \Parameter{Inhalt}
  \Macro{addcontentslinetoeachtocfile}\OParameter{Besitzer}
                                      \Parameter{Ebene}\Parameter{Inhalt}%
\end{Declaration}
Diese beiden Anweisungen stehen in direkter Beziehung zu dem oben erklärten
\DescRef{\LabelBase.cmd.addxcontentsline}\ChangedAt{v3.12}{\Package{tocbasic}}
beziehungsweise zum im \LaTeX-Kern definierten \Macro{addcontentsline}. Der
Unterschied besteht darin, dass diese Anweisungen \PName{Inhalt} nicht nur in
eine einzelne Datei, sondern in alle Dateien eines angegebenen
\PName{Besitzers} und bei Verzicht auf das erste optionale Argument in alle
Dateien aus der Liste der bekannten Dateierweiterungen schreibt.
\begin{Example}
  Angenommen, Sie sind Klassen-Autor und wollen den Kapiteleintrag nicht nur in
  das Inhaltsverzeichnis, sondern in alle Verzeichnisdateien schreiben. Nehmen
  wir weiter an, dass aktuell \PValue{\#1} den Titel enthält, der geschrieben
  werden soll.
\begin{lstcode}
  \addxcontentslinetoeachtocfile{chapter}%
                                [\thechapter]{#1}
\end{lstcode}
  In diesem Fall soll natürlich die aktuelle Kapitelnummer direkt beim
  Schreiben in die Verzeichnisdatei expandiert werden, weshalb sie nicht mit
  \Macro{protect} vor der Expansion geschützt wurde.
\end{Example}
Während des Schreibens von \PName{Inhalt} ist auch hier, wie schon bei
\DescRef{\LabelBase.cmd.addtoeachtocfile},
\Macro{@currext}\IndexCmd{@currext}\important{\Macro{@currext}} die
Dateierweiterung der Datei, in die aktuell geschrieben wird.%

Die Anweisung\ChangedAt{v3.12}{\Package{tocbasic}}
\Macro{addxcontentslinetoeachtocfile} ist gegenüber
\Macro{addcontentslinetoeachtocfile} möglichst vorzuziehen, da die
Erweiterungen von \DescRef{\LabelBase.cmd.addxcontentsline} nur damit
Anwendung finden. Näheres zu diesen Erweiterungen und Vorteilen ist in der
vorausgehenden Erklärung von \DescRef{\LabelBase.cmd.addxcontentsline} zu
finden.%
%
\EndIndexGroup


\begin{Declaration}
  \Macro{listoftoc}\OParameter{Titel}\Parameter{Dateierweiterung}
  \Macro{listoftoc*}\Parameter{Dateierweiterung}
  \Macro{listofeachtoc}\OParameter{Besitzer}
  \Macro{listof\PName{Dateierweiterung}name}
\end{Declaration}
Mit diesen Anweisungen werden die Verzeichnisse
ausgegeben. Die\important{\Macro{listoftoc*}} Sternvariante \Macro{listoftoc*}
benötigt als einziges Argument die \PName{Dateierweiterung} der Datei mit den
Daten zu dem Verzeichnis. Die Anweisung setzt zunächst die vertikalen und
horizontalen Abstände, die innerhalb von Verzeichnissen gelten sollen, führt
die Anweisungen aus, die vor dem Einlesen der Datei ausgeführt werden sollen,
liest dann die Datei und führt zum Schluss die Anweisungen aus, die nach dem
Einlesen der Datei ausgeführt werden sollen. Damit kann \Macro{listoftoc*} als
direkter Ersatz der \LaTeX-Kern-Anweisung
\Macro{@starttoc}\IndexCmd{@starttoc}\important{\Macro{@starttoc}} verstanden
werden.

Die\important{\Macro{listoftoc}} Version von \Macro{listoftoc} ohne Stern
setzt das komplette Verzeichnis und veranlasst auch einen optionalen Eintrag
in das Inhaltsverzeichnis und den Kolumnentitel. Ist das optionale Argument
\OParameter{Titel} gegeben, so wird diese Angabe sowohl als Überschrift als
auch als optionaler Eintrag in das Inhaltsverzeichnis und den Kolumnentitel
verwendet. Ist das Argument \PName{Titel} lediglich leer, so wird auch eine
leere Angabe verwendet. Wird\textnote{Achtung!} hingegen das komplette
Argument einschließlich der eckigen Klammern weggelassen, so wird die
Anweisung \Macro{listof\PName{Dateierweiterung}name} verwendet, wenn diese
definiert ist. Ist sie nicht definiert, wird ein Standard-Ersatzname verwendet
und eine Warnung ausgegeben.

Die\important{\Macro{listofeachtoc}} Anweisung \Macro{listofeachtoc} gibt alle
Verzeichnisse mit dem angegebenen Besitzer oder alle Verzeichnisse aller
bekannten Dateinamenerweiterungen aus. Damit\textnote{Achtung!} dabei der
korrekte Titel ausgegeben werden kann, sollte
\Macro{listof\PName{Dateierweiterung}name} passend definiert sein.
Da\textnote{Tipp!} eventuell auch der Anwender selbst \Macro{listoftoc} ohne
optionales Argument oder \Macro{listofeachtoc} verwenden könnte, wird dies
ohnehin empfohlen.
\begin{Example}
  Angenommen, Sie haben ein neues »Verzeichnis der Algorithmen« mit der
  Dateierweiterung »\PValue{loa}« und wollen dieses anzeigen lassen:
\begin{lstcode}
  \listoftoc[Verzeichnis der Algorithmen]{loa}
\end{lstcode}
  erledigt das für Sie. Wollen Sie das Verzeichnis hingegen ohne Überschrift
  ausgegeben haben, dann genügt:
\begin{lstcode}
  \listoftoc*{loa}
\end{lstcode}
  Im zweiten Fall würde natürlich auch ein optional aktivierter Eintrag in das
  Inhaltsverzeichnis nicht gesetzt. Näheres zur Eigenschaft des Eintrags in
  das Inhaltsverzeichnis ist bei der Anweisung
  \DescRef{\LabelBase.cmd.setuptoc}, \DescPageRef{\LabelBase.cmd.setuptoc} zu
  finden.

  Wenn Sie zuvor
\begin{lstcode}
  \newcommand*{\listofloaname}{%
    Verzeichnis der Algorithmen%
  }
\end{lstcode}
  definiert haben, genügt auch:
\begin{lstcode}
  \listoftoc{loa}
\end{lstcode}
  um ein Verzeichnis mit der gewünschten Überschrift zu erzeugen. Für den
  Anwender ist es eventuell einprägsamer, wenn Sie dann außerdem noch
\begin{lstcode}
  \newcommand*{\listofalgorithms}{\listoftoc{loa}}
\end{lstcode}
  als einfache Verzeichnisanweisung definieren.
\end{Example}

Da\textnote{Achtung!} \LaTeX{} bei der Ausgabe eines Verzeichnisses auch
gleich eine neue Verzeichnisdatei zum Schreiben öffnet, kann der Aufruf jeder
dieser Anweisungen zu einer Fehlermeldung der Art
\begin{lstoutput}
  ! No room for a new \write .
  \ch@ck ...\else \errmessage {No room for a new #3}
                                                    \fi   
\end{lstoutput}
führen, wenn keine Schreibdateien mehr zur Verfügung stehen. Abhilfe kann in
diesem Fall das Laden des in \autoref{cha:scrwfile} beschriebenen Pakets
\Package{scrwfile}\important{\Package{scrwfile}}\IndexPackage{scrwfile}
oder die Verwendung von Lua\LaTeX{} bieten.

Das Paket \hyperref[cha:scrhack]{\Package{scrhack}}\IndexPackage{scrhack}%
\important{ \hyperref[cha:scrhack]{\Package{scrhack}}} enthält übrigens
Patches für mehrere Pakete wie \Package{float} oder \Package{listings}, damit
deren Verzeichnisbefehle \Macro{listoftoc} verwenden. Dadurch stehen viele
Möglichkeiten von \Package{tocbasic} und den \KOMAScript-Klassen auch für
deren Verzeichnisse zur Verfügung.%
\EndIndexGroup


\begin{Declaration}
  \Macro{BeforeStartingTOC}\OParameter{Dateierweiterung}
                           \Parameter{Anweisungen}
  \Macro{AfterStartingTOC}\OParameter{Dateierweiterung}\Parameter{Anweisungen}
\end{Declaration}
Manchmal ist es nützlich, wenn unmittelbar vor dem Einlesen der Datei mit den
Verzeichnisdaten \PName{Anweisungen} ausgeführt werden können. Mit Hilfe von
\Macro{BeforeStartingTOC} können Sie eine solche Ausführung
wahlweise für eine einzelne \PName{Dateierweiterung} oder alle Dateien, die
mit Hilfe von \DescRef{\LabelBase.cmd.listoftoc*},
\DescRef{\LabelBase.cmd.listoftoc} oder \DescRef{\LabelBase.cmd.listofeachtoc}
eingelesen werden, erreichen. Ebenso können Sie \PName{Anweisungen} nach dem
Einlesen der Datei ausführen, wenn Sie diese mit \Macro{AfterStartingTOC}
definieren. Während der Ausführung der \PName{Anweisungen} ist
\Macro{@currext}\IndexCmd{@currext}\important{\Macro{@currext}} die
Dateierweiterung der Datei, die eingelesen wird bzw. gerade eingelesen wurde.

Ein Beispiel\textnote{Beispiel} zur Verwendung von \Macro{BeforeStartingTOC}
ist in \autoref{sec:scrwfile.clonefilefeature} auf
\PageRefxmpl{scrwfile.cmd.BeforeStartingTOC} zu finden.
%
\EndIndexGroup


\begin{Declaration}
  \Macro{BeforeTOCHead}\OParameter{Dateierweiterung}\Parameter{Anweisungen}
  \Macro{AfterTOCHead}\OParameter{Dateierweiterung}\Parameter{Anweisungen}
\end{Declaration}
Es können auch \PName{Anweisungen} definiert werden, die unmittelbar vor oder
nach dem Setzen der Überschrift bei Verwendung von
\DescRef{\LabelBase.cmd.listoftoc} oder \DescRef{\LabelBase.cmd.listofeachtoc}
ausgeführt werden. Bezüglich des optionalen Arguments und der Bedeutung von
\Macro{@currext}\IndexCmd{@currext}\important{\Macro{@currext}} gilt, was
bereits bei \DescRef{\LabelBase.cmd.BeforeStartingTOC} und
\DescRef{\LabelBase.cmd.AfterStartingTOC} oben erklärt wurde.%
\EndIndexGroup


\begin{Declaration}
  \Macro{MakeMarkcase}
\end{Declaration}
Wann immer \Package{tocbasic} eine Marke für einen Kolumnentitel setzt,
erfolgt dies als Argument der Anweisung \Macro{MakeMarkcase}. Diese Anweisung
ist dazu gedacht, bei Bedarf die Groß-/Kleinschreibung des Kolumnentitels zu
ändern. In der Voreinstellung ist diese Anweisung bei Verwendung einer
\KOMAScript-Klasse
\Macro{@firstofone}\IndexCmd{@firstofone}\important{\Macro{@firstofone}}, also
das unveränderte Argument selbst. Bei Verwendung einer anderen Klasse ist
\Macro{MakeMarkcase} im Gegensatz dazu
\Macro{MakeUppercase}\IndexCmd{MakeUppercase}\important{\Macro{MakeUppercase}}.
Die Anweisung wird von \Package{tocbasic} jedoch nur definiert, wenn sie
nicht bereits definiert ist. Sie kann also in einer Klasse in der gewünschten
Weise vorbelegt werden und wird dann von \Package{tocbasic} nicht umdefiniert,
sondern wie vorgefunden verwendet.
\begin{Example}
  Sie wollen aus unerfindlichen Gründen, dass die Kolumnentitel in Ihrer
  Klasse in Kleinbuchstaben ausgegeben werden. Damit dies auch für die
  Kolumnentitel gilt, die von \Package{tocbasic} gesetzt werden, definieren
  Sie:
\begin{lstcode}
  \let\MakeMarkcase\MakeLowercase
\end{lstcode}
\end{Example}

Erlauben\textnote{Tipp!} Sie mir einen Hinweis zu \Macro{MakeUppercase}. Diese
Anweisung ist zum einen nicht voll expandierbar. Das bedeutet, dass sie im
Zusammenspiel mit anderen Anweisungen zu Problemen führen kann. Zum anderen
sind sich alle Typografen einig, dass beim Versalsatz, also beim Satz
kompletter Wörter oder Passagen in Großbuchstaben, Sperrung unbedingt
notwendig ist. Dabei darf jedoch kein fester Abstand zwischen den Buchstaben
verwendet werden. Vielmehr muss zwischen unterschiedlichen Buchstaben auch ein
unterschiedlicher Abstand gesetzt werden, weil sich unterschiedliche
Buchstabenkombinationen unterschiedlich verhalten. Gleichzeitig bilden einige
Buchstaben von sich aus bereits Löcher, was bei der Sperrung ebenfalls zu
berücksichtigen ist. Pakete wie \Package{ulem} oder \Package{soul} können das
ebenso wenig leisten wie der Befehl \Macro{MakeUppercase} selbst. Auch die
automatische Sperrung mit Hilfe des \Package{microtype}-Pakets ist
diesbezüglich nur eine näherungsweise Notlösung, da die von der konkreten
Schrift abhängige Form der Buchstaben auch hier nicht näher betrachtet
wird. Da\textnote{Empfehlung} Versalsatz also eher etwas für absolute
Experten ist und fast immer Handarbeit bedeutet, wird Laien empfohlen, darauf
zu verzichten oder ihn nur vorsichtig und nicht an so exponierter Stelle
wie dem Kolumnentitel zu verwenden.%
\EndIndexGroup


\begin{Declaration}
  \Macro{deftocheading}\Parameter{Dateierweiterung}\Parameter{Definition}
\end{Declaration}
Das Paket \Package{tocbasic} enthält eine Standarddefinition für das Setzen
von Überschriften von Verzeichnissen. Diese Standarddefinition ist durch
verschiedene Eigenschaften, die bei der Anweisung
\DescRef{\LabelBase.cmd.setuptoc} erläutert werden, konfigurierbar. Sollte
diese Möglichkeit einmal nicht ausreichen, so besteht die Möglichkeit, mit
\Macro{deftocheading} eine alternative Überschriftenanweisung für ein
Verzeichnis mit einer bestimmten \PName{Dateierweiterung} zu definieren. Die
Definition kann als einzigen Parameter \PValue{\#1} enthalten. Beim Aufruf der
Anweisung innerhalb von \DescRef{\LabelBase.cmd.listoftoc} oder
\DescRef{\LabelBase.cmd.listofeachtoc} wird für dieses Argument der Titel
des Verzeichnisses übergeben.

Die \PName{Definition} ist dann selbstverständlich auch für die Auswertung
weiterer Eigenschaften, die sich auf die Überschrift beziehen,
verantwortlich. Das gilt insbesondere für die nachfolgend erklärten
Eigenschaften \PValue{leveldown}, \PValue{numbered} und \PValue{totoc}.%
\EndIndexGroup


\begin{Declaration}
  \Macro{setuptoc}\Parameter{Dateierweiterung}
                  \Parameter{Liste von Eigenschaften}
  \Macro{unsettoc}\Parameter{Dateierweiterung}
                  \Parameter{Liste von Eigenschaften}
\end{Declaration}
Mit diesen beiden Anweisungen können \PName{Eigenschaften} für eine
\PName{Dateierweiterung} bzw. das Verzeichnis, das dazu gehört, gesetzt und
gelöscht werden. Die \PName{Liste von Eigenschaften} ist dabei eine durch
Komma getrennte Folge von \PName{Eigenschaften}. Das Paket
\Package{tocbasic} wertet folgende Eigenschaften aus:
\begin{description}\setkomafont{descriptionlabel}{}%
\item[\PValue{leveldown}] bedeutet, dass das Verzeichnis nicht mit der
  obersten Gliederungsebene unterhalb von \DescRef{maincls.cmd.part} -- wenn
  vorhanden \DescRef{maincls.cmd.chapter}, sonst \DescRef{maincls.cmd.section}
  -- erstellt wird, sondern mit einer Überschrift der nächsttieferen
  Gliederungsebene. Diese Eigenschaft wird von der internen
  Überschriftenanweisung ausgewertet. Wird\textnote{Achtung!}  hingegen eine
  eigene Überschriftenanweisung mit \DescRef{\LabelBase.cmd.deftocheading}
  definiert, liegt die Auswertung der Eigenschaft in der Verantwortung dessen,
  der die Definition vornimmt. Die \KOMAScript-Klassen setzen diese
  Eigenschaft bei Verwendung der Option
  \OptionValueRef{maincls}{listof}{leveldown}%
  \important{\OptionValueRef{maincls}{listof}{leveldown}}%
  \IndexOption{listof~=\textKValue{leveldown}} für alle Dateierweiterungen mit
  dem Besitzer \PValue{float}.
\item[\PValue{nobabel}] bedeutet, dass die normalerweise automatisch
  verwendete Erweiterung für die Sprachumschaltung mit
  \Package{babel}\IndexPackage{babel} für diese Dateierweiterung nicht
  verwendet wird. Diese Eigenschaft sollte nur für Verzeichnisse verwendet
  werden, die nur in einer festen Sprache erstellt werden, in denen also
  Sprachumschaltungen im Dokument nicht zu berücksichtigen sind. Sie wird
  außerdem vom Paket
  \Package{scrwfile}\important{\Package{scrwfile}}\IndexPackage{scrwfile} für
  Klonziele verwendet, da die Erweiterungen dort bereits durch das Klonen
  selbst aus der Klonquelle übernommen werden.

  Es ist zu beachten\textnote{Achtung!}, dass die Eigenschaft bereits vor dem
  Hinzufügen der Dateierweiterung zu der Liste der bekannten
  Dateierweiterungen gesetzt sein muss, damit sie eine Wirkung hat.
\item[\PValue{noindent}]\ChangedAt{v3.27}{\Package{tocbasic}} veranlasst alle
  von \KOMAScript{} bereitgestellten Verzeichniseintragsstile ihre Eigenschaft
  \PValue{indent} (siehe \autoref{tab:tocbasic.tocstyle.attributes},
  \autopageref{tab:tocbasic.tocstyle.attributes.indent}) zu ignorieren und
  stattdessen den Einzug zu deaktivieren.
\item[\PValue{noparskipfake}] verhindert\ChangedAt{v3.17}{\Package{tocbasic}},
  dass vor dem Abschalten des Absatzabstandes für die Verzeichnisse ein
  letztes Mal ein expliziter Absatzabstand eingefügt wird. Dies führt in der
  Regel dazu, dass bei Dokumenten mit Absatzabstand der Abstand zwischen
  Überschrift und erstem Verzeichniseintrag\index{Verzeichnis>Eintrag} kleiner
  wird als zwischen Überschriften und normalem Text. %
  \iffalse% Umbruchkorrektur
  Normalerweise erhält man daher ohne diese Eigenschaft eine einheitlichere
  Formatierung.%
  \else%
  Ohne diese Eigenschaft wirkt die Formatierung daher meist einheitlicher.%
  \fi%
\item[\PValue{noprotrusion}] verhindert\ChangedAt{v3.10}{\Package{tocbasic}}
  das Abschalten des optischen Randausgleichs in den Verzeichnissen. Optischer
  Randausgleich wird standardmäßig abgeschaltet, wenn das Paket
  \Package{microtype}\IndexPackage{microtype} oder ein anderes Paket, das die
  Anweisung \Macro{microtypesetup}\IndexCmd{microtypesetup} bereitstellt,
  geladen ist. Wenn also optischer Randausgleich in den Verzeichnissen
  gewünscht wird, dann muss diese Eigenschaft aktiviert
  werden. Es\textnote{Achtung!} ist jedoch zu beachten, dass der optische
  Randausgleich in Verzeichnissen häufig zu einem falschen Ergebnis
  führt. Dies ist ein bekanntes Problem des optischen Randausgleichs.
\item[\PValue{numbered}] bedeutet, dass das Verzeichnis nummeriert und damit
  ebenfalls in das Inhaltsverzeichnis aufgenommen werden soll. Diese
  Eigenschaft wird von der internen Überschriftenanweisung ausgewertet. Wird
  hingegen eine eigene Überschriftenanweisung mit
  \DescRef{\LabelBase.cmd.deftocheading} definiert, liegt die Auswertung der
  Eigenschaft in der Verantwortung dessen, der die Definition vornimmt.  Die
  \KOMAScript-Klassen setzen diese Eigenschaft bei Verwendung der Option
  \OptionValueRef{maincls}{listof}{numbered}%
  \important{\OptionValueRef{maincls}{listof}{numbered}}%
  \IndexOption{listof~=\textKValue{numbered}} für alle Dateierweiterungen mit 
  dem Besitzer \Package{float}.
\item[\PValue{numberline}] \ChangedAt{v3.12}{\Package{tocbasic}}%
  bedeutet, dass all diejenigen Einträge, die mit Hilfe der Anweisung
  \DescRef{\LabelBase.cmd.addxcontentsline} oder der Anweisung
  \DescRef{\LabelBase.cmd.addxcontentslinetoeachtocfile} vorgenommen werden,
  wobei das optionale Argument für die Nummer fehlt oder leer ist, mit einer
  leeren \DescRef{\LabelBase.cmd.numberline}-Anweisung versehen werden. Das
  führt in der Regel dazu, dass diese Einträge nicht linksbündig mit der
  Nummer, sondern mit dem Text der nummerierten Einträge gleicher Ebene
  gesetzt werden. Bei\ChangedAt{v3.20}{\Package{tocbasic}} Verwendung des
  Verzeichniseintragsstils \PValue{tocline} kann die Eigenschaft weitere
  Auswirkungen haben. Siehe dazu die Stil-Eigenschaften
  \Option{breakafternumber} und \Option{entrynumberformat} in
  \autoref{tab:tocbasic.tocstyle.attributes} ab
  \autopageref{tab:tocbasic.tocstyle.attributes}.

  Die \KOMAScript-Klassen setzen diese Eigenschaft bei Verwendung der Option
  \OptionValueRef{maincls}{listof}{numberline}%
  \important{\OptionValueRef{maincls}{listof}{numberline}\\
    \OptionValueRef{maincls}{toc}{numberline}%
  }%
  \IndexOption{listof~=\textKValue{numberline}} für die Dateierweiterungen mit
  dem Besitzer \PValue{float} und bei Verwendung der Option
  \OptionValueRef{maincls}{toc}{numberline}%
  \IndexOption{toc~=\textKValue{numberline}} für die Dateierweiterung
  \PValue{toc}. Entsprechend
  wird die Eigenschaft bei Verwendung von Option
  \OptionValueRef{maincls}{listof}{nonumberline}%
  \important{%
    \OptionValueRef{maincls}{listof}{nonumberline}\\
    \OptionValueRef{maincls}{toc}{nonumberline}%
  }\IndexOption{listof~=\textKValue{nonumberline}} oder
  \OptionValueRef{maincls}{toc}{nonumberline}%
  \IndexOption{toc~=\textKValue{nonumberline}} wieder zurückgesetzt.
\item[\PValue{onecolumn}] \leavevmode\ChangedAt{v3.01}{\Package{tocbasic}}%
  bedeutet, dass für dieses Verzeichnis automatisch der \LaTeX-interne
  Einspaltenmodus mit \Macro{onecolumn}\IndexCmd{onecolumn} verwendet
  wird. Das\textnote{Achtung!} gilt jedoch nur, falls dieses Verzeichnis
  nicht mit der oben beschriebenen Eigenschaft
  \PValue{leveldown}\important{\PValue{leveldown}} um eine Gliederungsebene
  nach unten verschoben wurde. Die \KOMAScript-Klassen \Class{scrbook} und
  \Class{scrreprt} setzen die Eigenschaft per
  \DescRef{\LabelBase.cmd.AtAddToTocList} (siehe
  \DescPageRef{\LabelBase.cmd.AtAddToTocList}) für alle Verzeichnisse mit dem
  Besitzer \PValue{float} oder mit sich selbst als Besitzer. Damit werden
  beispielsweise das Inhaltsverzeichnis, das Abbildungsverzeichnis und das
  Tabellenverzeichnis bei diesen beiden Klassen automatisch einspaltig
  gesetzt. Der Mehrspaltenmodus des
  \Package{multicol}-Pakets\IndexPackage{multicol} ist von der Eigenschaft
  ausdrücklich nicht betroffen.
\item[\PValue{totoc}] bedeutet, dass der Titel des Verzeichnisses in das
  Inhaltsverzeichnis aufgenommen werden soll. Diese Eigenschaft wird von der
  internen Überschriftenanweisung ausgewertet. Wird\textnote{Achtung!} mit
  \DescRef{\LabelBase.cmd.deftocheading}\important{\DescRef{\LabelBase.cmd.deftocheading}}%
  \IndexCmd{deftocheading} hingegen eine eigene Überschriftenanweisung
  definiert, liegt die Auswertung der Eigenschaft in der Verantwortung dessen,
  der die Definition vornimmt. Die \KOMAScript-Klassen setzen diese
  Eigenschaft bei Verwendung der Option
  \OptionValueRef{maincls}{listof}{totoc}%
  \important{\OptionValueRef{maincls}{listof}{totoc}}%
  \IndexOption{listof~=\textKValue{totoc}} für alle Dateierweiterungen mit dem
  Besitzer \Package{float}.
\end{description}
Die \KOMAScript-Klassen kennen eine weitere Eigenschaft:
\begin{description}\setkomafont{descriptionlabel}{}
\item[\PValue{chapteratlist}] sorgt dafür, dass in dieses Verzeichnis bei
  jedem neuen Kapitel eine optionale Gliederung eingefügt wird. In der
  Voreinstellung ist diese Untergliederung dann ein vertikaler
  Abstand. Näheres zu den Möglichkeiten ist Option
  \DescRef{maincls.option.listof}%
  \IndexOption{listof~=\PName{Einstellung}}%
  \important{\DescRef{maincls.option.listof}} in
  \autoref{sec:maincls.floats}, \DescPageRef{maincls.option.listof} zu
  entnehmen.
\end{description}
\begin{Example}
  \iffalse% Umbruchkorrektur
  Da \KOMAScript{} für das Abbildungs- und das Tabellenverzeichnis auf
  \Package{tocbasic} aufbaut, gibt es nun eine weitere Möglichkeit, jegliche
  Kapiteluntergliederung dieser Verzeichnisse zu verhindern:
\begin{lstcode}
  \unsettoc{lof}{chapteratlist}
  \unsettoc{lot}{chapteratlist}
\end{lstcode}

  \fi%
  Wollen Sie\iffalse hingegen\fi, dass das von Ihnen definierte Verzeichnis
  mit der Dateierweiterung »\PValue{loa}« ebenfalls von der
  Kapiteluntergliederung der \KOMAScript-Klassen betroffen ist, so verwenden
  Sie
\begin{lstcode}
  \setuptoc{loa}{chapteratlist}
\end{lstcode}
Wollen Sie außerdem, dass bei Klassen, die \DescRef{maincls.cmd.chapter} als
oberste Gliederungsebene verwenden, das Verzeichnis automatisch einspaltig
gesetzt wird, so verwenden Sie zusätzlich
\begin{lstcode}
  \Ifundefinedorrelax{chapter}{}{%
    \setuptoc{loa}{onecolumn}%
  }
\end{lstcode}
  Die Verwendung von \DescRef{scrbase.cmd.Ifundefinedorrelax}
  setzt das Paket \Package{scrbase} voraus (siehe
  \autoref{sec:scrbase.if},
  \DescPageRef{scrbase.cmd.Ifundefinedorrelax}).

  Sollte\textnote{Tipp!} Ihr Paket mit einer anderen Klasse verwendet werden,
  so schadet es trotzdem nicht, dass Sie diese Eigenschaften setzen, im
  Gegenteil: Wertet eine andere Klasse diese Eigenschaften ebenfalls aus, so
  nutzt Ihr Paket automatisch die Möglichkeiten jener Klasse.
\end{Example}
Wie Sie hier sehen, unterstützt ein Paket, das \Package{tocbasic} verwendet,
bereits ohne nennenswerten Aufwand diverse Möglichkeiten für die dadurch
realisierten Verzeichnisse, die sonst einigen Implementierungsaufwand
bedeuten würden und deshalb in vielen Paketen leider fehlen.%
\EndIndexGroup


\begin{Declaration}
  \Macro{Iftocfeature}\Parameter{Dateierweiterung}\Parameter{Eigenschaft}
                      \Parameter{Dann-Teil}\Parameter{Sonst-Teil}
\end{Declaration}
Hiermit\ChangedAt{v3.28}{\Package{tocbasic}} kann man für jede
\PName{Eigenschaft} feststellen, ob sie für eine \PName{Dateierweiterung}
gesetzt ist. Ist dies der Fall, wird der \PName{Dann-Teil} ausgeführt,
anderenfalls der \PName{Sonst-Teil}. Das kann beispielsweise nützlich sein,
wenn Sie eigene Überschriftenanweisungen mit
\DescRef{\LabelBase.cmd.deftocheading} definieren, aber die oben beschriebenen
Eigenschaften \PValue{totoc}, \PValue{numbered} oder \PValue{leveldown}
unterstützen wollen.
%
\EndIndexGroup


\section{Konfiguration von Verzeichniseinträgen}
\seclabel{tocstyle}
\BeginIndexGroup
\BeginIndex{}{Verzeichnis>Eintrag}

Neben\ChangedAt{v3.20}{\Package{tocbasic}} den eigentlichen
Verzeichnissen und den zugehörigen Hilfsdateien kann man mit dem Paket
\Package{tocbasic} ab Version~3.20 auch Einfluss auf die Verzeichniseinträge
nehmen. Dazu können neue Stile definiert werden. Es stehen aber auch mehrere
vordefinierte Stile zur Verfügung. Damit soll \Package{tocbasic} auch das
nie offiziell gewordene \KOMAScript-Paket \Package{tocstyle} ablösen. Die
\KOMAScript-Klassen selbst bauen seit Version~3.20 ebenfalls vollständig auf
die von \Package{tocbasic} bereitgestellten Stile für Verzeichniseinträge.

\begin{Declaration}
  \Counter{tocdepth}
\end{Declaration}
Verzeichniseinträge sind normalerweise hierarchisch geordnet. Dazu wird jeder
Eintragsebene ein numerischer Wert zugeordnet. Je höher dieser Wert, desto
tiefer in der Hierarchie liegt die Ebene. Bei den Standardklassen hat die
Ebene für Teile beispielsweise den Wert -1 und die Ebene für Kapitel den Wert
0. Über den \LaTeX-Zähler \Counter{tocdepth} wird bestimmt, bis zu welcher
Ebene Einträge im Verzeichnis ausgegeben werden. 

Bei \Class{book} ist \Counter{tocdepth} beispielsweise mit 2 voreingestellt,
es werden also die Einträge der Ebenen \PValue{part}, \PValue{chapter},
\PValue{section} und \PValue{subsection} ausgeben. Tiefere Ebenen wie
\PValue{subsubsection}, deren numerischer Wert 3 ist, werden nicht
ausgegeben. Trotzdem sind die Einträge in der Hilfsdatei für das
Inhaltsverzeichnis vorhanden.

Die von \Package{tocbasic} definierten Eintragsstile beachten, abgesehen von
\PValue{gobble} (siehe
\DescRef{\LabelBase.cmd.DeclareTOCStyleEntry}\iffree{}{, weiter unten in
  diesem Abschnitt}), ebenfalls \Counter{tocdepth}.%
\EndIndexGroup


\begin{Declaration}
  \Macro{numberline}\Parameter{Gliederungsnummer}
  \Macro{usetocbasicnumberline}\OParameter{Code}
\end{Declaration}
Zwar\ChangedAt{v3.20}{\Package{tocbasic}} definiert bereits der
\LaTeX-Kern eine Anweisung \Macro{numberline}, diese ist allerdings für die
Anforderungen von \Package{tocbasic} nicht ausreichend. Deshalb definiert
\Package{tocbasic} eigene Anweisungen und setzt \Macro{numberline} bei Bedarf
mit Hilfe von \Macro{usetocbasicnumberline} für die einzelnen
Verzeichniseinträge entsprechend. Eine Umdefinierung von \Macro{numberline}
ist daher bei Verwendung von \Package{tocbasic} oftmals wirkungslos und führt
teilweise auch zu Warnungen.

Man kann auch die Definition von \Package{tocbasic} generell nutzen, indem man
bereits in der Dokumentpräambel \Macro{usetocbasicnumberline} aufruft. Die
Anweisung versucht zunächst zu ermitteln, ob in der aktuellen Definition
wichtige interne Anweisungen von \Package{tocbasic} verwendet werden. Ist dies
nicht der Fall, wird \Macro{numberline} entsprechend umdefiniert und zusätzlich
\PName{Code} ausgeführt. Ist kein optionales Arguments angegeben, wird
stattdessen via \Macro{PackageInfo} eine Meldung über die erfolgte
Umdefinierung ausgegeben. Diese Meldung kann man einfach unterdrücken, indem
ein leeres optionales Argument angegeben wird.

Es\textnote{Achtung!} ist zu beachten, dass \Macro{usetocbasicnumberline} den
internen Schalter \Macro{if@tempswa} global verändern kann!%
\EndIndexGroup


\begin{Declaration}
  \Macro{DeclareTOCStyleEntry}\OParameter{Optionenliste}\Parameter{Stil}
                              \Parameter{Eintragsebene}
  \Macro{DeclareTOCStyleEntries}\OParameter{Optionenliste}\Parameter{Stil}
                                \Parameter{Liste von Eintragsebenen}
\end{Declaration}
Über\ChangedAt{v3.20}{\Package{tocbasic}} diese Anweisungen werden
die Verzeichniseinträge für bestimmte \PName{Eintragsebenen} deklariert
oder konfiguriert. Dabei ist die \PName{Eintragsebene} der Name der jeweiligen
Eintragsebene, beispielsweise \PValue{section} für einen zur gleichnamigen
Gliederungsebene gehörenden Eintrag ins Inhaltsverzeichnis oder
\PValue{figure} für den Eintrag einer Abbildung ins
Abbildungsverzeichnis. Jeder \PName{Eintragsebene} wird ein bestimmter
\PName{Stil} zugeordnet, der zum Zeitpunkt der Deklaration bereits definiert
sein muss. Über die \PName{Optionenliste} können die verschiedenen, meist vom
\PName{Stil} abhängenden Eigenschaften des Eintrags festgelegt werden.

Derzeit werden von \Package{tocbasic} die folgenden Eintragsstile definiert:
\begin{description}\setkomafont{descriptionlabel}{}
\item[\PValue{default}] ist in der Voreinstellung ein Klon von Stil
  \PValue{dottedtocline}. Klassenautoren, die \Package{tocbasic} verwenden,
  wird empfohlen, diesen Stil mit Hilfe von
  \DescRef{\LabelBase.cmd.CloneTOCEntryStyle} auf den
  Standardverzeichniseintragsstil der Klasse zu ändern. Bei den
  \KOMAScript-Klassen wird \PValue{default} beispielsweise zu einem Klon von
  Stil \PValue{tocline}.
\item[\PValue{dottedtocline}] entspricht dem Stil, der von den Standardklassen
  \Class{book} und \Class{report} für die Inhaltsverzeichniseinträge der
  Ebenen \PValue{section} bis \PValue{subparagraph} und bei allen
  Standardklassen für die Einträge in das Abbildungs- und das
  Tabellenverzeichnis bekannt ist. Er kennt nur drei\important{\PValue{level},
    \PValue{indent}, \PValue{numwidth}} Eigenschaften. Die Einträge werden
  um \PValue{indent} von links eingezogen in der aktuellen Schrift
  ausgegeben. \DescRef{\LabelBase.cmd.numberline} wird nicht umdefiniert. Die
  Breite der Nummer wird von \PValue{numwidth} bestimmt. Bei mehrzeiligen
  Einträgen wird der Einzug ab der zweiten Zeile um \PValue{numwidth}
  erhöht. Die Seitenzahl wird mit \Macro{normalfont} ausgegeben. Eintragstext
  und Seitenzahl werden durch eine punktierte Linie
  verbunden. \autoref{fig:tocbasic.dottedtocline} illustriert die
  Eigenschaften des Stils.
  \begin{figure}
    \centering
    \resizebox{.8\linewidth}{!}{%
      \begin{tikzpicture}
        \draw[color=lightgray] (0,2\baselineskip) -- (0,-2.5\baselineskip);
        \draw[color=lightgray] (\linewidth,2\baselineskip) --
                               (\linewidth,-2.5\baselineskip);
        \node (dottedtocline) at (0,0) [anchor=west,inner sep=0,outer sep=0]
        {%
          \hspace*{7em}%
          \parbox[t]{\dimexpr\linewidth-9.55em}{%
            \setlength{\parindent}{-3.2em}%
            \addtolength{\parfillskip}{-2.55em}%
            \makebox[3.2em][l]{1.1.10}%
            Überschrift im Verzeichniseintragsstil \PValue{dottedtocline} mit
            mehr als einer Zeile%
            \leaders\hbox{$\csname m@th\endcsname
              \mkern 4.5mu\hbox{.}\mkern 4.5mu$}\hfill\nobreak
            \makebox[1.55em][r]{12}%
          }%
        };
        \draw[|-|,color=gray,overlay] (0,0) --
                              node [anchor=north,font=\small] {
                                \PValue{indent}
                              }
                              (3.8em,0);
        \draw[|-|,color=gray,overlay] (3.8em,\baselineskip) -- 
                              node [anchor=south,font=\small] {
                                \PValue{numwidth}
                              }
                              (7em,\baselineskip);
        \draw[|-|,color=gray,overlay] (\linewidth,\ht\strutbox) -- 
                              node [anchor=south,font=\small] { 
                                \Macro{@tocrmarg} 
                              }
                              (\linewidth-2.55em,\ht\strutbox);
        \draw[|-|,color=gray,overlay] (\linewidth,-\baselineskip) -- 
                              node [anchor=north,font=\small] { 
                                \Macro{@pnumwidth} 
                              } 
                              (\linewidth-1.55em,-\baselineskip);
      \end{tikzpicture}%
    }
    \caption{Illustration einiger Attribute des Verzeichniseintragsstils
      \PValue{dottedtocline}}
    \label{fig:tocbasic.dottedtocline}
  \end{figure}
\item[\PValue{gobble}] ist der denkbar einfachste Stil. Einträge in diesem
  Stil werden unabhängig von allen Einstellungen für
  \DescRef{\LabelBase.counter.tocdepth}%
  \IndexCounter{tocdepth}\important{\DescRef{\LabelBase.counter.tocdepth}}
  nicht ausgegeben, sondern sozusagen verschluckt. Dennoch verfügt er über die
  Standardeigenschaft \PValue{level}, die jedoch nie ausgewertet wird.
\item[\PValue{largetocline}] entspricht dem Stil, der von den Standardklassen
  für die Ebene \PValue{part} bekannt ist. Er kennt
  nur\important{\PValue{level}, \PValue{indent}} die Eigenschaften
  \PName{level} und \PName{indent}. Letzteres ist gleichsam eine Abweichung
  von den Standardklassen, die selbst keinen Einzug der \PValue{part}-Einträge
  unterstützen.

  Vor dem Eintrag wird ein Seitenumbruch erleichtert. Die Einträge werden um
  \PValue{indent} von links eingezogen und mit den Schrifteinstellungen
  \Macro{large}\Macro{bfseries} ausgegeben. Sollte
  \DescRef{\LabelBase.cmd.numberline} verwendet werden, so ist die
  Nummernbreite fest auf 3\Unit{em}
  eingestellt. \DescRef{\LabelBase.cmd.numberline} wird nicht umdefiniert. Die
  Standardklassen verwenden für \PName{part}-Einträge kein
  \DescRef{\LabelBase.cmd.numberline}. Der Wert hat bei mehrzeiligen Einträgen
  auch keine Auswirkung auf den Einzug ab der zweiten Zeile.

  \autoref{fig:tocbasic.largetocline} illustriert die
  Eigenschaften
  des Stils. Dabei wird auch auf"|fällig, dass der Stil einige Ungereimtheiten
  der Standardklassen übernommen hat, beispielsweise den fehlenden Einzug ab
  der zweiten Zeile bei mehrzeiligen Einträgen und zwei unterschiedliche Werte
  für \Macro{@pnumwidth}, die aus einer Abhängigkeit von der Schriftgröße
  resultieren. Daraus resultiert auch der Umstand, dass im Extremfall der Text
  der Überschrift der Seitenzahl zu nahe kommen kann.  Es ist zu beachten, dass
  die in der Abbildung gezeigte Breite der Gliederungsnummer nur dann
  Verwendung findet, wenn auch tatsächlich \DescRef{\LabelBase.cmd.numberline}
  verwendet wird. Die Standardklassen setzen hingegen einen festen Abstand von
  1\Unit{em} nach der Nummer.
  \begin{figure}
    \centering
    \resizebox{.8\linewidth}{!}{%
      \begin{tikzpicture}
        \draw[color=lightgray] (0,2\baselineskip) -- (0,-2.5\baselineskip);
        \draw[color=lightgray] (\linewidth,2\baselineskip) --
                               (\linewidth,-2.5\baselineskip);
        \node (largetocline) at (0,0) [anchor=west,inner sep=0,outer sep=0] {%
          \parbox[t]{\dimexpr \linewidth-1.55em\relax}{%
            \makebox[3em][l]{\large\bfseries I}%
            \large\bfseries
            Überschrift im Verzeichniseintragsstil \PValue{largetocline} mit
            mehr als einer Zeile\hfill 
            \makebox[0pt][l]{\normalsize\normalfont
              \makebox[1.55em][r]{\large\bfseries 1}}%
          }%
        };
        \draw[|-|,color=gray] (0,\baselineskip) -- 
                              node [anchor=south] { 3\,em } 
                              (3em,\baselineskip);
        \draw[|-|,color=gray,overlay] (\linewidth,\ht\strutbox) -- 
                              node [anchor=south] { \Macro{@pnumwidth} }
                              (\linewidth-1.55em,\ht\strutbox);
        \large\bfseries
        \draw[|-|,color=gray,overlay] (\linewidth,-\baselineskip) -- 
                              node [anchor=north,font=\normalfont\normalsize] { 
                                \Macro{large}\Macro{@pnumwidth} 
                              }
                              (\linewidth-1.55em,-\baselineskip);
      \end{tikzpicture}%
    }
    \caption{Illustration einiger Attribute des Verzeichniseintragsstils
      \PValue{largetocline}}
    \label{fig:tocbasic.largetocline}
  \end{figure}
\item[\PValue{tocline}] ist ein flexibler Stil, der in der Voreinstellung für
  alle Einträge der \KOMAScript-Klassen verwendet wird. Entsprechend
  definieren diese Klassen auch die Klone \PValue{part}, \PValue{chapter} und
  \PValue{section} beziehungsweise \PValue{section} und \PValue{subsection}
  mit Hilfe dieses Stils, ändern dann jedoch den \PName{Initialisierungscode}
  der Klone so ab, dass sie unterschiedliche Voreinstellungen besitzen.

  Der Stil kennt neben der Standardeigenschaft \PValue{level} noch
  20\important{\PValue{level}, \PValue{beforeskip}, \PValue{breakafternumber},
    \PValue{dynindent},
    \PValue{dynnumwidth}, \PValue{entryformat}, \PValue{entrynumberformat},
    \PValue{indent}, \PValue{indentfollows},
    \PValue{linefill}, \PValue{numsep}, \PValue{numwidth},
    \PValue{onstarthigherlevel}, \PValue{onstartlowerlevel},
    \PValue{onstartsamelevel}, \PValue{pagenumberbox},
    \PValue{pagenumberformat}, \PValue{pagenumberwidth},
    \PValue{raggedentrytext}, \PValue{raggedpagenumber}, \PValue{rightindent}}
  weitere Eigenschaften. Die Voreinstellungen all dieser Eigenschaften werden
  abhängig vom Namen der \PName{Eintragsebene} bestimmt und orientieren sich
  dann an den Ergebnissen der Standardklassen. Es ist daher möglich, nach
  Laden von \Package{tocbasic} den Stil der Inhaltsverzeichniseinträge der
  Standardklassen mit \DescRef{\LabelBase.cmd.DeclareTOCEntryStyle} in
  \PValue{tocline} zu ändern, ohne dass dies unmittelbar zu größeren
  Veränderungen im Aussehen der Inhaltsverzeichniseinträge führt. So kann man
  gezielt nur die Eigenschaften ändern, die für erwünschte Änderungen
  notwendig sind.  Dasselbe gilt für Abbildungs- und Tabellenverzeichnis der
  Standardklassen.%

  Aufgrund der hohen Flexibilität kann dieser Stil prinzipiell die Stile
  \PValue{dottedtocline}, \PValue{undottedtocline} und \PValue{largetocline}
  ersetzen, bedarf dann jedoch teilweise eines höheren Aufwands bei der
  Konfiguration. 

  \autoref{fig:tocbasic.tocline} illustriert einige Längen-Eigenschaften des
  Stils. Die weiteren sind in \autoref{tab:tocbasic.tocstyle.attributes}, ab
  \autopageref{tab:tocbasic.tocstyle.attributes} erklärt.
  \begin{figure}
    \centering
    \resizebox{.8\linewidth}{!}{\hspace{-2em}% Leicht nach links schieben
      \begin{tikzpicture}
        \coordinate (subsection) at (0,0);
        \coordinate (section) at ($(subsection)+(0,2\baselineskip)$);
        \coordinate (chapter) at ($(section)+(0,2\baselineskip)$);
        \coordinate (part)    at ($(chapter)+(0,2.4\baselineskip+1em)$);

        \draw[color=lightgray] 
          ($(part)+(0,2\baselineskip)$) -- 
          (0,-2.5\baselineskip);
        \draw[color=lightgray] 
          ($(part)+(\linewidth,2\baselineskip)$) --
          (\linewidth,-2.5\baselineskip);

        \coordinate (subsection) at (0,0);

        \node at (part) [anchor=west,inner sep=0,outer sep=0]
        {%
          \hspace*{3em}%
          \parbox[t]{\dimexpr\linewidth-5.55em}{%
            \setlength{\parindent}{-3em}%
            \addtolength{\parfillskip}{-2.55em}%
            \makebox[3em][l]{\large\bfseries I.}%
            \textbf{\large Überschrift eines Teil-Eintrags im Stil
            \PValue{tocline} zur Demonstration mit mehr als einer Zeile}%
            \hfill
            \makebox[1.55em][r]{\bfseries 12}\large
          }%
        };
        \draw[|-|,color=gray,overlay] 
          (part) --
          ($(part)+(3em,0)$)
          node [anchor=north east,font=\small] {
            \PValue{numwidth}
          };
        \draw[|-|,color=gray,overlay] 
          ($(part)+(\linewidth,\ht\strutbox)$)
          node [anchor=north,font=\small] { 
            ~\PValue{rightindent} 
          } --
          ($(part)+(\linewidth-2.55em,\ht\strutbox)$);
        \draw[|-|,color=gray,overlay] 
          ($(part)+(\linewidth,-\baselineskip)$) -- 
          node [anchor=north,font=\small] { 
            \PValue{pagenumberwidth}
          } 
          ($(part)+(\linewidth-1.55em,-\baselineskip)$);
        \node at (chapter) [anchor=west,inner sep=0,outer sep=0]
        {%
          \hspace*{1.5em}%
          \parbox[t]{\dimexpr\linewidth-4.05em}{%
            \setlength{\parindent}{-1.5em}%
            \addtolength{\parfillskip}{-2.55em}%
            \makebox[1.5em][l]{\bfseries 1.}%
            \textbf{Überschrift eines Kapiteleintrags im Stil
            \PValue{tocline} zur Demonstration mit mehr als einer Zeile}%
            \hfill
            \makebox[1.55em][r]{\bfseries 12}%
          }%
        };
        \draw[|-|,color=gray,overlay]
          ($(chapter)+(3em,\baselineskip)$) --
          node [anchor=west,font=\small] {
            \PValue{beforeskip}
          }
          ($(part)+(3em,-\baselineskip)$);
        \draw[|-|,color=gray,overlay] 
          (chapter) --
          ($(chapter)+(1.5em,0)$)
          node [anchor=north east,font=\small] {
            \PValue{numwidth}
          };
        \draw[|-|,color=gray,overlay] 
          ($(chapter)+(\linewidth,\ht\strutbox)$)
          node [anchor=north,font=\small] { 
            ~\PValue{rightindent}
          } --
          ($(chapter)+(\linewidth-2.55em,\ht\strutbox)$);
        \draw[|-|,color=gray,overlay] 
          ($(chapter)+(\linewidth,-\baselineskip)$)
          node [anchor=north west,font=\small] {
            \hspace*{-3.1em}\PValue{pagenumberwidth}
          } --
          ($(chapter)+(\linewidth-1.55em,-\baselineskip)$);
        \node at (section) [anchor=west,inner sep=0,outer sep=0]
        {
          \hspace*{3.8em}%
          \parbox[t]{\dimexpr\linewidth-6.35em}{%
            \setlength{\parindent}{-2.3em}%
            \addtolength{\parfillskip}{-2.55em}%
            \makebox[2.3em][l]{1.1.}%
            Überschrift eines Abschnitteintrags im Stil
            \PValue{tocline} zur Demonstration mit mehr als einer Zeile%
            \leaders\hbox{$\csname m@th\endcsname
              \mkern 4.5mu\hbox{.}\mkern 4.5mu$}\hfill\nobreak
            \makebox[1.55em][r]{3}%
          }%
        };
        \node at (subsection) [anchor=west,inner sep=0,outer sep=0]
        {%
          \hspace*{7em}%
          \parbox[t]{\dimexpr\linewidth-9.55em}{%
            \setlength{\parindent}{-3.2em}%
            \addtolength{\parfillskip}{-2.55em}%
            \makebox[3.2em][l]{1.1.10.}%
            Überschrift eines Unterabschnitteintrags im Stil
            \PValue{tocline} mit mehr als einer Zeile%
            \leaders\hbox{$\csname m@th\endcsname
              \mkern 4.5mu\hbox{.}\mkern 4.5mu$}\hfill\nobreak
            \makebox[1.55em][r]{12}%
          }%
        };
        \draw[|-|,color=gray,overlay] 
          ($(subsection)+(0,\ht\strutbox)$) -- 
          node [anchor=north,font=\small] {
            \PValue{indent}
          }
          ($(subsection)+(3.8em,\ht\strutbox)$);
        \draw[|-|,color=gray,overlay] 
          ($(subsection)+(3.8em,0)$) --
          ($(subsection)+(7em,0)$)
          node [anchor=north east,font=\small] {
            \PValue{numwidth}
          };
        \draw[|-|,color=gray,overlay] 
          ($(subsection)+(\linewidth,\ht\strutbox)$)
          node [anchor=north,font=\small] { 
            ~\PValue{rightindent}
          } --
          ($(subsection)+(\linewidth-2.55em,\ht\strutbox)$);
        \draw[|-|,color=gray,overlay] 
          ($(subsection)+(\linewidth,-\baselineskip)$) -- 
          node [anchor=north,font=\small] { 
            \PValue{pagenumberwidth}
          } 
          ($(subsection)+(\linewidth-1.55em,-\baselineskip)$);
      \end{tikzpicture}%
    }
    \caption{Illustration einiger Attribute des Verzeichniseintragsstils
      \PValue{tocline}}
    \label{fig:tocbasic.tocline}
  \end{figure}
\item[\PValue{toctext}]\ChangedAt{v3.27}{\Package{tocbasic}}%
  stellt eine Besonderheit dar. Während alle anderen Stile je Eintrag einen
  Absatz\important{\PValue{level}, \PValue{afterpar}, \PValue{beforeskip},
    \PValue{entryformat},
    \PValue{entrynumberformat}, \PValue{indent}, \PValue{numsep},
    \PValue{onendentry}, \PValue{onendlastentry}, \PValue{onstartentry},
    \PValue{onstartfirstentry}, \PValue{pagenumberformat},
    \PValue{prepagenumber}, \PValue{raggedright}, \PValue{rightindent}}
  erzeugen, wird hier für alle aufeinanderfolgenden Einträge dieses Stils nur
  ein einziger Absatz erzeugt.  Für die Konfigurierung stehen neben der
  Standardeigenschaft \PValue{level} mit 14 weiteren Eigenschaften fast so
  viele Möglichkeiten wie bei \PValue{tocline} zur Verfügung. Dieser Stil ist
  aber zwingend darauf angewiesen, dass am Anfang aller anderen Stile und auch
  am Ende des Verzeichnisses ein noch nicht beendeter Absatz tasächlich
  beendet wird. Daher sollten Einträge dieses Stils nicht mit Einträgen oder
  Verzeichnissen kombiniert werden, die an \Package{tocbasic} vorbei erzeugt
  werden.
\item[\PValue{undottedtocline}] entspricht dem Stil, der von den
  Standardklassen \Class{book} und \Class{report} für die Ebene
  \PValue{chapter} und von \Class{article} für die Ebene \PValue{section}
  bekannt ist. Er kennt\important{\PValue{level}, \PValue{indent},
    \PValue{numwidth}} nur drei Eigenschaften.
  Vor dem Eintrag wird ein Seitenumbruch erleichtert und ein vertikaler
  Abstand eingefügt. Die Einträge werden um \PValue{indent} von links
  eingezogen in \Macro{bfseries} ausgegeben. Dies ist gleichsam eine
  Abweichung von den Standardklassen, die selbst keinen Einzug für Einträge
  der genannten Ebenen bieten. \DescRef{\LabelBase.cmd.numberline} wird nicht
  umdefiniert. Die Breite der Nummer wird von \PValue{numwidth} bestimmt. Bei
  mehrzeiligen Einträgen wird der Einzug ab der zweiten Zeile um
  \PValue{numwidth} erhöht. \autoref{fig:tocbasic.undottedtocline} illustriert
  die Eigenschaften des Stils.
  \begin{figure}
    \centering
    \resizebox{.8\linewidth}{!}{%
      \begin{tikzpicture}
        \draw[color=lightgray] (0,2\baselineskip) -- (0,-2.5\baselineskip);
        \draw[color=lightgray] (\linewidth,2\baselineskip) --
                               (\linewidth,-2.5\baselineskip);
        \node (undottedtocline) at (0,0) [anchor=west,inner sep=0,outer sep=0]
        {%
          \makebox[1.5em][l]{\bfseries 1}%
          \parbox[t]{\dimexpr \linewidth-4.05em\relax}{%
            \bfseries
            Überschrift im Verzeichniseintragsstil \PValue{undottedtocline} mit
            mehr als einer Zeile%
          }%
          \raisebox{-\baselineskip}{\makebox[2.55em][r]{\bfseries 3}}%
        };
        \draw[|-|,color=gray,overlay] (0,\baselineskip) -- 
                              node [anchor=south,font=\small] {
                                \PValue{numwidth}
                              }
                              (1.5em,\baselineskip);
        \draw[|-|,color=gray,overlay] (\linewidth,\ht\strutbox) -- 
                              node [anchor=south,font=\small] { 
                                \Macro{@tocrmarg} 
                              }
                              (\linewidth-2.55em,\ht\strutbox);
        \draw[|-|,color=gray,overlay] (\linewidth,-\baselineskip) -- 
                              node [anchor=north,font=\small] { 
                                \Macro{@pnumwidth} 
                              } 
                              (\linewidth-1.55em,-\baselineskip);
      \end{tikzpicture}%
    }
    \caption{Illustration einiger Attribute des Verzeichniseintragsstils
      \PValue{undottedtocline} am Beispiel einer Kapitelüberschrift}
    \label{fig:tocbasic.undottedtocline}
  \end{figure}
\end{description}
Eine Erklärung zu den Eigenschaften der von \Package{tocbasic} definierten
Stile findet sich in \autoref{tab:tocbasic.tocstyle.attributes}.
Neben\ChangedAt{v3.27}{\Package{tocbasic}} der normalen Zuweisung eines Wertes
in der Form \Option{\PName{Schlüssel}=\PName{Wert}} verstehen beide Befehle
für alle Optionen der durch \KOMAScript{} definierten Stile auch die
Zuweisungen in der Form \Option{\PName{Schlüssel}:=\PName{Eintragsebene}}. In
diesem Fall wird die aktuell gültige Einstellung für \PName{Eintragsebene}
kopiert, soweit diese verfügbar ist. Es kann also beispielsweise mit
\OptionValue{indent:}{figure} der Einzug für \PValue{figure}-Einträge kopiert
werden. Für Optionen, die eine Länge oder einen Integer als \PName{Wert}
erwarten, gibt es außerdem die Möglichkeit mit
\Option{\PName{Schlüssel}+=\PName{Wert}} den \PName{Wert} zur aktuell gültigen
Einstellung zu addieren. Für eine Substraktion kann ein negativer \PName{Wert}
verwendet werden. Mit \OptionValue{indent+}{1cm} könnte so beispielsweise der
Einzug um 1\Unit{cm} erhöht werden. Für Optionen, die eine
Liste\ChangedAt{v3.31}{\Package{tocbasic}} als \PName{Wert} erwarten, kann mit
\Option{\PName{Schlüssel}+=\PName{Wert}} der neue \PName{Wert} an den bereits
vorhandenen angehängt werden.

Bei\ChangedAt{v3.21}{\Package{tocbasic}} Verwendung der Eigenschaften als
Optionen für Anweisung
\DescRef{\LabelBase.cmd.DeclareNewTOC}\IndexCmd{DeclareNewTOC} (siehe
\DescPageRef{\LabelBase.cmd.DeclareNewTOC}) sind die Namen der Eigenschaften
mit dem Präfix \PValue{tocentry} zu verwenden, also beispielsweise
\Option{tocentrylevel} anstelle von \PValue{level}. Die zuvor beschriebene
Kopiermöglichkeit mit \Option{:=} ist hierbei ebenfalls verfügbar. Die
Addition mit Hilfe von \Option{+=} wird derzeit jedoch nicht unterstützt.

Bei\ChangedAt{v3.20}{\Package{tocbasic}} Verwendung als Optionen
für \DescRef{maincls-experts.cmd.DeclareSectionCommand}%
\IndexCmd{DeclareSectionCommand}\IndexCmd{DeclareNewSectionCommand}%
\IndexCmd{RedeclareSectionCommand}\IndexCmd{ProvideSectionCommand} (siehe
\DescPageRef{maincls-experts.cmd.DeclareSectionCommand}) und verwandten
Anweisungen sind die Namen der Eigenschaften mit dem Präfix \PValue{toc} zu
versehen, also beispielsweise \Option{toclevel} anstelle von \PValue{level}. Hierbei existiert derzeit weder die Kopiermöglichkeit mit
\Option{:=} noch die Addition mit \Option{+=}.

Letztlich führt der Aufruf von \Macro{DeclareTOCStyleEntry} zur Definition der
Anweisung \Macro{l@\PName{Eintragsebene}}.

Während \Macro{DeclareTOCStyleEntry} nur eine \PName{Eintragsebene}
definiert, kann über
\Macro{DeclareTOCStyeEntries}\ChangedAt{v3.26}{\Package{tocbasic}} auf einen
Schlag eine ganze \PName{Liste von Eintragsebenen} definiert werden. Die durch
Komma voneinander getrennt angegebenen Eintragsebenen der Liste werden dabei
alle mit demselben \PName{Stil} und den über \PName{Optionenliste}
angegebenen Einstellungen definiert.%
%
\begin{desclist}
  \desccaption{%
    Attribute für die vordefinierten Verzeichniseintragsstile von
    \Package{tocbasic}%
    \label{tab:tocbasic.tocstyle.attributes}%
  }{%
    Attribute der Verzeichniseintragsstile (\emph{Fortsetzung})%
  }%
  \entry{\OptionVName{afterpar}{Code}}{%
    \ChangedAt{v3.27}{\Package{tocbasic}}%
    Der angegebene \PName{Code} wird nach dem Ende des Absatzes ausgeführt, in
    dem ein Eintrag mit dem Stil \PValue{toctext} ausgegeben wird. Verfügen
    mehrere Einträge über solche Einstellungen, so werden diese in der
    Reihenfolge der Einträge ausgeführt.%
  }%
  \entry{\OptionVName{beforeskip}{Länge}}{%
    Vertikaler Abstand, der vor einem Eintrag dieser Ebene im Stil
    \PValue{tocline} eingefügt wird (siehe
    \autoref{fig:tocbasic.tocline}). Der Abstand wird je nach Ebene mit
    \Macro{vskip} oder \Macro{addvspace} eingefügt, so dass diesbezüglich
    möglichst Kompatibilität zu den Standardklassen und früheren Versionen von
    \KOMAScript{} besteht.

    Bei der \PName{Eintragsebene} \PValue{part} wird das Attribut mit
    \texttt{2.25em plus 1pt} initialisiert, bei \PValue{chapter} mit
    \texttt{1em plus 1pt}. Ist noch keine \PName{Eintragsebene}
    \PValue{chapter} bekannt, wird stattdessen für \PValue{section}
    \texttt{1em plus 1pt} verwendet. Ansonsten wird es für \PValue{section}
    wie für alle
    anderen Ebenen mit \texttt{0pt plus .2pt} initialisiert.
    
    Im\ChangedAt{v3.31}{\Package{tocbasic}} Stil \PValue{toctext} wird der
    vertikale Abstand vor dem Absatz eingefügt, wenn es sich um den ersten
    Eintrag im Absatz handelt. Bei allen weiteren Einträgen des Absatzes wird
    er ignoriert. Findet die Initialisierung über diesen Stil statt, so wird
    als Voreinstellung \texttt{0pt} verwendet.%
  }%
  \entry{\OptionVName{breakafternumber}{Schalter}}{%
    \PName{Schalter} ist einer der Werte für einfache Schalter aus
    \autoref{tab:truefalseswitch}, \autopageref{tab:truefalseswitch}.  Ist der
    Schalter beim Stil \PValue{tocline} aktiviert, so wird nach der mit
    \DescRef{\LabelBase.cmd.numberline}\IndexCmd{numberline} gesetzten Nummer
    eine neue Zeile begonnen, die erneut linksbündig mit der Nummer beginnt.

    In der Voreinstellung ist die Eigenschaft im Stil \PValue{tocline} nicht
    gesetzt.

    Ist %\textnote{Achtung!}
    für ein Verzeichnis mit \DescRef{\LabelBase.cmd.setuptoc} die Eigenschaft
    \Option{numberline} gesetzt (siehe \autoref{sec:tocbasic.toc},
    \DescPageRef{\LabelBase.cmd.setuptoc}), wie das bei den
    \KOMAScript-Klassen und Verwendung deren Option
    \OptionValueRef{maincls}{toc}{numberline}%
    \IndexOption{toc~=\textKValue{numberline}} der Fall ist, führt dies auch
    dazu, dass bei nicht nummerierten Einträgen dennoch die Zeile mit der dann
    leeren Nummer in der Formatierung von \Option{entrynumberformat} gesetzt
    wird.%
  }%
  \entry{\OptionVName{dynindent}{Schalter}}{%
    \ChangedAt{v3.31}{\Package{tocbasic}}%
    \PName{Schalter} ist einer der Werte für einfache Schalter aus
    \autoref{tab:truefalseswitch}, \autopageref{tab:truefalseswitch}. Ist der
    Schalter beim Stil \PValue{tocline} aktiviert, gibt die Eigenschaft
    \PValue{indent} nur noch einen Minimalwert an. Der Maximalwert wird durch
    die Nummernbreite und den Einzug der via \Option{indentfollows}
    vorgegebenen Ebenen bestimmt.%
  }%
  \entry{\OptionVName{dynnumwidth}{Schalter}}{%
    \PName{Schalter} ist einer der Werte für einfache Schalter aus
    \autoref{tab:truefalseswitch}, \autopageref{tab:truefalseswitch}. Ist der
    Schalter beim Stil \PValue{tocline} aktiviert, gibt die Eigenschaft
    \PValue{numwidth} nur noch einen Minimalwert an. Übertrifft die beim
    letzten \LaTeX-Lauf ermittelte maximale Breite der Eintragsnummern
    gleicher Ebene zuzüglich des Wertes von \PValue{numsep} diesen
    Minimalwert, so wird stattdessen der ermittelte Wert verwendet.%
  }%
  \entry{\OptionVName{entryformat}{Befehl}}{%
    Über diese Eigenschaft kann die Formatierung des gesamten Eintrags
    verändert werden. Der dabei als Wert angegebene \PName{Befehl} hat genau
    ein Argument zu erwarten. Dieses Argument ist nicht zwingend voll
    expandierbar. Befehle wie \Macro{MakeUppercase}, die ein voll
    expandierbares Argument erwarten, dürfen an dieser Stelle also nicht
    verwendet werden. Font-Änderungen über \Option{entryformat} erfolgen
    ausgehend von \Macro{normalfont}\Macro{normalsize}. Es wird darauf
    hingewiesen, dass die Ausgabe von \Option{linefill} und der Seitenzahl
    unabhängig von \Option{entryformat} ist. Siehe dazu auch die Eigenschaft
    \Option{pagenumberformat}.

    Initialisiert wird die Eigenschaft für die \PName{Eintragsebene}
    \PValue{part} mit der Ausgabe des übergebenen Argument in
    \Macro{large}\Macro{bfseries} und für \PValue{chapter} in
    \Macro{bfseries}. Falls bei der Initialisierung von \PValue{section} noch
    keine Ebene \PValue{chapter} existiert, wird auch für diese Ebene
    \Macro{bfseries} verwendet. Bei allen anderen Ebenen wird das Argument
    unverändert ausgegeben.%
  }%
  \entry{\OptionVName{entrynumberformat}{Befehl}}{%
    Über diese Eigenschaft kann die Formatierung der mit
    \DescRef{\LabelBase.cmd.numberline} gesetzten Eintragsnummer verändert
    werden. Der dabei als Wert angegebene \PName{Befehl} hat genau ein
    Argument zu erwarten. Font-Änderungen erfolgen ausgehend von der
    Eigenschaft \Option{entryformat}.

    Initialisiert wird die Eigenschaft mit der Ausgabe des übergebenen
    Arguments. Die Eintragsnummer wird also unverändert ausgegeben.

    Ist %\textnote{Achtung!}
    für ein Verzeichnis mit \DescRef{\LabelBase.cmd.setuptoc} die Eigenschaft
    \Option{numberline} gesetzt (siehe \autoref{sec:tocbasic.toc},
    \DescPageRef{\LabelBase.cmd.setuptoc}), wie dies bei den
    \KOMAScript-Klassen und Verwendung deren Option
    \OptionValueRef{maincls}{toc}{numberline}%
    \IndexOption{toc~=\textKValue{numberline}} der Fall ist, führt dies auch
    dazu, dass bei nicht nummerierten Einträgen \PName{Befehl} dennoch
    ausgeführt wird.%
  }%
  \entry{%
    \OptionVName{indent}{Länge}%
    {\phantomsection\label{tab:tocbasic.tocstyle.attributes.indent}}%
  }{%
    Beim\ChangedAt{v3.27}{\Package{tocbasic}} Stil \PValue{toctext} ist
    \PName{Länge} der horizontale Abstand des Absatzes vom linken Rand. Haben
    die unterschiedlichen Einträge des Absatzes unterschiedliche
    Einstellungen, so gewinnt der letzte Eintrag. Bei den übrigen Stilen ist
    \PValue{Länge} entsprechend der horizontale Abstand des Eintrags vom
    linken Rand (siehe \autoref{fig:tocbasic.dottedtocline} und
    \autoref{fig:tocbasic.tocline}).

    Bei den Stilen \PValue{tocline} und \PValue{toctext} wird für alle
    Eintragsebenen, deren Name mit »\texttt{sub}« beginnt, eine
    Initialisierung mit \PValue{indent}+\PValue{numwidth} der gleichnamigen
    Eintragsebene ohne diesen Präfix vorgenommen, falls eine solche Ebene mit
    entsprechenden Eigenschaften existiert. Bei den Stilen
    \PValue{dottedtocline}, \PValue{undottedtocline} und \PValue{tocline}
    findet für die Eintragsebenen \PValue{part} bis \PValue{subparagraph}
    sowie \PValue{figure} und \PValue{table} eine Initialisierung mit Werten
    entsprechend der Standardklassen statt. Alle anderen Ebenen erhalten keine
    Initialisierung. Für sie ist eine explizite Angabe daher bei der ersten
    Verwendung zwingend.

    Ist für ein Verzeichnis die Eigenschaft \PValue{noindent} via
    \DescRef{\LabelBase.cmd.setuptoc} gesetzt, so ignorieren die Einträge bei
    allen von \KOMAScript{} bereitgestellten Stilen diesen Wert und verwenden
    stattdessen 0\Unit{pt}. Der Einzug wird also deaktiviert.%
  }%
  \entry{\OptionVName{indentfollows}{Ebenenliste}}{%
    \ChangedAt{v3.31}{\Package{tocbasic}}%
    Ist \Option{dynindent} beim Stil \PValue{tocline} gesetzt, so dient die
    hier angegebene durch Komma separierte Liste an Ebenennamen dazu, den
    tatsächlichen Einzug zu ermitteln. Dabei findet bei Ebenen, deren Name mit
    »\texttt{sub}« beginnt, eine Initialisierung mit dem Namen ohne diesen
    Präfix statt. Die \KOMAScript-Klassen setzen außerdem automatisch passende
    Werte für die Ebenen \PValue{section} und \PValue{paragraph}.%
  }%
  \entry{\OptionVName{level}{Integer}}{%
    Numerischer Wert der \PName{Eintragsebene}. Tatsächlich angezeigt werden
    nur Einträge, deren numerische Ebene nicht größer als Zähler
    \DescRef{\LabelBase.counter.tocdepth}%
    %\important{\DescRef{\LabelBase.counter.tocdepth}}
    \IndexCounter{tocdepth}
    ist.

    Diese Eigenschaft ist für alle Stile zwingend und wird bei der
    Definition eines Stils automatisch definiert.
    
    Bei den Stilen \PValue{tocline} und \PValue{toctext} findet für alle
    Eintragsebenen, deren Name mit »\texttt{sub}« beginnt, eine
    Initialisierung entsprechend dem um eins erhöhten Wert einer gleichnamigen
    Eintragsebene ohne diesen Präfix statt, falls eine solche Ebene
    existiert. Bei den Stilen \PValue{dottedtocline}, \PValue{largetocline},
    \PValue{tocline}, \PValue{toctext} und \PValue{undottedtocline} findet für
    die \PName{Eintragsebene} \PValue{part}, \PValue{chapter},
    \PValue{section}, \PValue{subsection}, \PValue{subsubsection},
    \PValue{paragraph}, \PValue{subparagraph}, \PValue{figure} und
    \PValue{table} automatisch eine Initialisierung aufgrund des Namens
    statt. Für andere Ebenen findet eine Initialisierung mit dem Wert der
    Gliederungsebene statt, falls kompatibel zu den \KOMAScript-Klassen
    \Macro{\PName{Eintragsebene}numdepth} definiert ist.%
  }%
  \entry{\OptionVName{linefill}{Code}}{%
    Beim Stil \PValue{tocline} kann zwischen dem Ende des Eintragstextes und
    der Seitenzahl die Art der Füllung verändert werden. Die Eigenschaft
    \PName{linefill} erhält als Wert direkt den gewünschten \PName{Code}.  Für
    \PName{Eintragsebene} \PValue{part} und \PValue{chapter} wird die
    Eigenschaft mit \Macro{hfill} initialisiert. Dadurch rückt die Seitenzahl
    an den rechten Rand. Ist bisher keine \PName{Eintragsebene}
    \PValue{chapter} definiert, so gilt dies auch für \PValue{section}. Alle
    anderen Ebenen werden mit \DescRef{\LabelBase.cmd.TOCLineLeaderFill}
    (siehe \DescPageRef{\LabelBase.cmd.TOCLineLeaderFill}) initialisiert.

    Wird \PName{Code} angegeben, der nicht automatisch zu einer Füllung des
    Abstandes führt, sollte übrigens auch die Eigenschaft
    \PValue{raggedpagenumber} gesetzt werden, damit es nicht zu
    »\texttt{underfull \Macro{hbox}}«-Meldungen kommt.%
  }%
  \entry{\OptionVName{numsep}{Länge}}{%
    Der Stil \PValue{tocline} versucht sicherzustellen, dass zwischen der
    Nummer und dem Text eines Eintrags mindestens ein Abstand von
    \PName{Länge} eingehalten wird. Bei aktiviertem \PValue{dynnumwidth} kann
    die für die Nummer reservierte Breite \PValue{numwidth} entsprechend
    korrigiert werden. Bei nicht aktiviertem \PValue{dynnumwidth} wird
    hingegen lediglich eine Warnung ausgegeben, wenn diese Bedingung nicht
    eingehalten wird.

    Der\ChangedAt{v3.27}{\Package{tocbasic}} Stil \PValue{toctext} fügt
    dagegen immer einen Abstand dieser \PName{Länge} nach der Nummer des
    Eintrags ein.

    Die Eigenschaft wird mit einem Wert von 0,4\Unit{em} initialisiert.%
  }%
  \entry{\OptionVName{numwidth}{Länge}}{%
    Für die Nummer eines Eintrags reservierte Breite (siehe
    \autoref{fig:tocbasic.dottedtocline} bis
    \autoref{fig:tocbasic.undottedtocline}). Dieser Wert wird bei den
    Stilen \PValue{dottedtocline}, \PValue{tocline} und
    \PValue{undottedtocline} ab der zweiten Zeile eines Eintrags zum linken
    Einzug hinzugerechnet.

    Beim Stil \PValue{tocline} wird für alle Eintragsebenen, deren Name mit
    »\texttt{sub}« beginnt, eine Initialisierung mit dem Wert der
    gleichnamigen Eintragsebene ohne diesen Präfix zuzüglich 0,9\Unit{em}
    vorgenommen, falls eine solche Ebene mit entsprechender Eigenschaft
    existiert. Bei den Stilen \PValue{dottedtocline}, \PValue{undottedtocline}
    und \PValue{tocline} findet für die Eintragsebenen \PValue{part} bis
    \PValue{subparagraph} sowie \PValue{figure} und \PValue{table} eine
    Initialisierung mit Werten entsprechend der Standardklassen statt. Alle
    anderen Ebenen erhalten keine Initialisierung. Für sie ist eine explizite
    Angabe daher bei der ersten Verwendung zwingend.%
  }%
  \entry{\OptionVName{onendentry}{Code}}{%
    \ChangedAt{v3.27}{\Package{tocbasic}}%
    Führt den angegebenen \PName{Code} unmittelbar nach einem Eintrag im Stil
    \PValue{toctext} aus, sofern es nicht der letzte Eintrag im Absatz
    ist. Der Anwender muss unbedingt sicherstellen, dass \PName{Code} auf
    keinen Fall zum Beenden des Absatzes führt.

    Hinweis: Tatsächlich wird der \PName{Code} gar nicht am Ende des Eintrags,
    sondern vor dem nächsten Eintrag im Stil \PValue{toctext} ausgeführt.%
  }%
  \entry{\OptionVName{onendlastentry}{Code}}{%
    \ChangedAt{v3.27}{\Package{tocbasic}}%
    Führt den angegebenen \PName{Code} unmittelbar vor dem Ende des Absatzes
    mit dem Eintrag im Stil \PValue{toctext} aus, sofern es sich um den
    letzten Eintrag im Absatz handelt. Der Anwender sollte sicherstellen, dass
    \PName{Code} nicht zum Beenden des Absatzes führt.%
  }%
  \entry{\OptionVName{onstartentry}{Code}}{%
    \ChangedAt{v3.27}{\Package{tocbasic}}%
    Führt den angegebenen \PName{Code} unmittelbar vor dem Eintrag im Stil
    \PValue{toctext} aus, sofern es sich nicht um den ersten Eintrag im Absatz
    handelt. Der Anwender muss unbedingt sicherstellen, dass \PName{Code} auf
    keinen Fall zum Beenden des Absatzes führt.%
  }%
  \entry{\OptionVName{onstartfirstentry}{Code}}{%
    \ChangedAt{v3.27}{\Package{tocbasic}}%
    Führt den angegebenen \PName{Code} unmittelbar vor dem Eintrag im Stil
    \PValue{toctext} aus, sofern es sich um den ersten Eintrag im Absatz
    handelt. Der Anwender muss unbedingt sicherstellen, dass \PName{Code} auf
    keinen Fall zum Beenden des bereits begonnen Absatzes führt.%
  }%
  \entry{\OptionVName{onstarthigherlevel}{Code}}{%
    Der Stil \PValue{tocline} kann zu Beginn eines Eintrags eine Aktion in
    Abhängigkeit davon ausführen, ob der zuletzt gesetzte Eintrag einen
    höheren, denselben oder einen niedrigeren \PName{level}-Wert hatte. Im
    Falle, dass der aktuelle Eintrag einen größeren \PName{level}-Wert
    besitzt, in der Hierarchie der Einträge also tiefer steht, wird der über
    diese Eigenschaft angegebene \PName{Code} ausgeführt.

    Die Erkennung funktioniert übrigens nur, solange sich \Macro{lastpenalty}
    seit dem letzten Eintrag nicht geändert hat.

    Initialisiert wird die Eigenschaft mit
    \DescRef{\LabelBase.cmd.LastTOCLevelWasLower} (siehe
    \DescPageRef{\LabelBase.cmd.LastTOCLevelWasLower}).%
  }%
  \entry{\OptionVName{onstartlowerlevel}{Code}}{%
    Der Stil \PValue{tocline} kann zu Beginn eines Eintrags eine Aktion in
    Abhängigkeit davon ausführen, ob der zuletzt gesetzte Eintrag einen
    höheren, denselben oder einen niedrigeren \PName{level}-Wert hatte. Im
    Falle, dass der aktuelle Eintrag einen kleineren \PName{level}-Wert
    besitzt, in der Hierarchie der Einträge also höher steht, wird der über
    diese Eigenschaft angegebene \PName{Code} ausgeführt.

    Die Erkennung funktioniert übrigens nur, solange sich \Macro{lastpenalty}
    seit dem letzten Eintrag nicht geändert hat.

    Initialisiert wird die Eigenschaft mit
    \DescRef{\LabelBase.cmd.LastTOCLevelWasHigher} (siehe
    \DescPageRef{\LabelBase.cmd.LastTOCLevelWasHigher}), was normalerweise
    dazu führt, dass ein Umbruch vor dem Eintrag begünstigt wird.%
  }%
  \entry{\OptionVName{onstartsamelevel}{Code}}{%
    Der Stil \PValue{tocline} kann zu Beginn eines Eintrags eine Aktion in
    Abhängigkeit davon ausführen, ob der zuletzt gesetzte Eintrag einen
    höheren, denselben oder einen niedrigeren \PName{level}-Wert hatte. Im
    Falle, dass der aktuelle Eintrag denselben \PName{level}-Wert besitzt, in
    der Hierarchie der Einträge also gleich gestellt ist, wird der über diese
    Eigenschaft angegebene \PName{Code} ausgeführt.

    Die Erkennung funktioniert übrigens nur, solange sich \Macro{lastpenalty}
    seit dem letzten Eintrag nicht geändert hat.

    Initialisiert wird die Eigenschaft mit
    \DescRef{\LabelBase.cmd.LastTOCLevelWasSame} (siehe
    \DescPageRef{\LabelBase.cmd.LastTOCLevelWasSame}), was normalerweise dazu
    führt, dass ein Umbruch vor dem Eintrag begünstigt wird.%
  }%
  \entry{\OptionVName{pagenumberbox}{Befehl}}{%
    Normalerweise wird die zu einem Eintrag gehörende Seitenzahl rechtsbündig
    in eine Box der Breite \Macro{@pnumwidth} gesetzt. Beim Stil
    \PValue{tocline} kann der Befehl, der dazu verwendet wird, über diese
    Eigenschaft konfiguriert werden. Der dabei anzugebende \PName{Befehl} hat
    genau ein Argument zu erwarten.

    Initialisiert wird die Eigenschaft mit der bereits erwähnten Box.%
  }%
  \entry{\OptionVName{pagenumberformat}{Befehl}}{%
    Über diese Eigenschaft kann die Formatierung der Seitenzahl des Eintrags
    verändert werden. Der dabei als Wert angegebene \PName{Befehl} hat genau
    ein Argument zu erwarten. Font-Änderungen über \Option{entryformat}
    erfolgen ausgehend von \Option{entryformat}, gefolgt von
    \Macro{normalfont}\Macro{normalsize}.

    Initialisiert wird die Eigenschaft für die \PName{Eintragsebene}
    \PValue{part} mit der Ausgabe des übergebenen Arguments in
    \Macro{large}\Macro{bfseries}. Für die \PName{Eintragsebene}
    \PValue{chapter} wird nur \Macro{bfseries} verwendet. Bei Klassen
    ohne vordefiniertes \Macro{l@chapter} geschieht dies auch für die
    \PName{Eintragsebene} \PValue{section}. Für alle anderen Ebenen erfolgt
    die Ausgabe in \Macro{normalfont}\Macro{normalcolor}.%
  }%
  \entry{\OptionVName{pagenumberwidth}{Länge}}{%
    \ChangedAt{v3.27}{\Package{tocbasic}}%
    Mit dieser Eigenschaft kann die Breite der Standardbox für die Seitenzahl
    eines Eintrags im Stil \PValue{tocline} von \Macro{@pnumwidth} in die
    angegebene \PName{Länge} geändert werden. Es ist zu beachten, dass bei
    Änderung des Befehls für die Box über die Eigenschaft
    \Option{pagenumberbox} die angegebene \PName{Länge} nicht mehr automatisch
    Anwendung findet.%
  }%
  \entry{\OptionVName{prepagenumber}{Code}}{%
    \ChangedAt{v3.27}{\Package{tocbasic}}%
    Im Stil \PValue{toctext} wird zwischen dem Eintragstext und der Seitenzahl
    \PName{Code} ausgeführt. Dies dient in erster Linie dazu, Abstand oder
    Trennzeichen zwischen Text und Seitenzahl einzufügen.%

    Voreingestellt ist mit \Macro{nonbreakspace} ein nicht umbrechbares
    Leerzeichen.%
  }%
  \entry{\OptionVName{raggedentrytext}{Schalter}}{%
    \ChangedAt{v3.21}{\Package{tocbasic}}%
    \PName{Schalter} ist einer der Werte für einfache Schalter aus
    \autoref{tab:truefalseswitch}, \autopageref{tab:truefalseswitch}.
    Ist der Schalter beim Stil \PValue{tocline} aktiviert, so wird der Text
    des Eintrags nicht im Blocksatz, sondern im Flattersatz gesetzt. Dabei
    werden nur noch Wörter getrennt, die länger als eine Zeile sind.

    In der Voreinstellung ist dieser Schalter nicht gesetzt.%
  }%
  \entry{\OptionVName{raggedpagenumber}{Schalter}}{%
    \PName{Schalter} ist einer der Werte für einfache Schalter aus
    \autoref{tab:truefalseswitch}, \autopageref{tab:truefalseswitch}.
    Ist der Schalter beim Stil \PValue{tocline} aktiviert, so wird die
    Seitenzahl nicht zwingend rechtsbündig gesetzt.

    Je nach Wert der Eigenschaft \PValue{linefill} kann sich das
    Setzen dieses Schalters nur im Erscheinen oder Verschwinden einer Warnung
    oder auch konkret in der Formatierung der Einträge auswirken. Es ist also
    wichtig, diese beiden Eigenschaften zueinander passend zu setzen.

    In der Voreinstellung ist dieser Schalter nicht gesetzt und passt damit
    zur Initialisierung von \PValue{linefill} sowohl mit \Macro{hfill} als
    auch mit \DescRef{\LabelBase.cmd.TOCLineLeaderFill}.%
  }%
  \entry{\OptionVName{raggedright}{Schalter}}{%
    \ChangedAt{v3.27}{\Package{tocbasic}}%
    \PName{Schalter} ist einer der Werte für einfache Schalter aus
    \autoref{tab:truefalseswitch}, \autopageref{tab:truefalseswitch}.
    Ist der Schalter innerhalb eines Absatzes bei irgendeinem Eintrag im Stil
    \PValue{toctext} gesetzt, so wird der komplette Absatz in linksbündigem
    Flattersatz gesetzt.%
  }%
  \entry{\OptionVName{rightindent}{Länge}}{%
    \ChangedAt{v3.27}{\Package{tocbasic}}%
    Mit dieser Eigenschaft kann der rechte Rand für den Text eines Eintrags im
    Stil \PValue{tocline} von \Macro{@tocrmarg} in die angegebene
    \PName{Länge} geändert werden. Beim Stil \PValue{toctext} wird
    entsprechend der rechte Rand für den kompletten Absatz eingestellt.%
  }%
\end{desclist}
\EndIndexGroup

\begin{Declaration}
  \Macro{DeclareTOCEntryStyle}\Parameter{Stil}
                              \OParameter{Initialisierungscode}
                              \Parameter{Befehlscode}
  \Macro{DefineTOCEntryOption}\Parameter{Option}
                              \OParameter{Säumniswert}
                              \Parameter{Code}
  \Macro{DefineTOCEntryBooleanOption}\Parameter{Option}
                                     \OParameter{Säumniswert}
                                     \Parameter{Präfix}
                                     \Parameter{Postfix}
                                     \Parameter{Erklärung}
                                     %\OParameter{Initialisierungscode}
  \Macro{DefineTOCEntryCommandOption}\Parameter{Option}
                                     \OParameter{Säumniswert}
                                     \Parameter{Präfix}
                                     \Parameter{Postfix}
                                     \Parameter{Erklärung}
                                     %\OParameter{Initialisierungscode}
  \Macro{DefineTOCEntryIfOption}\Parameter{Option}
                                \OParameter{Säumniswert}
                                \Parameter{Präfix}%
                                \Parameter{Postfix}
                                \Parameter{Erklärung}
                                % \OParameter{Initialisierungscode}
  \Macro{DefineTOCEntryLengthOption}\Parameter{Option}
                                    \OParameter{Säumniswert}
                                    \Parameter{Präfix}
                                    \Parameter{Postfix}
                                    \Parameter{Erklärung}
                                    % \OParameter{Initialisierungscode}
  \Macro{DefineTOCEntryNumberOption}\Parameter{Option}
                                    \OParameter{Säumniswert}
                                    \Parameter{Präfix}
                                    \Parameter{Postfix}
                                    \Parameter{Erklärung}
                                    % \OParameter{Initialisierungscode}
\end{Declaration}
\Macro{DeclareTOCEntryStyle}\ChangedAt{v3.20}{\Package{tocbasic}} ist
eine der komplexesten Anweisungen in \KOMAScript. Sie richtet sich daher
ausdrücklich an \LaTeX-Entwickler und nicht an \LaTeX-Anwender. Mit ihrer
Hilfe ist es möglich, einen neuen \PName{Stil} für Verzeichniseinträge zu
definieren. Üblicherweise werden Verzeichniseinträge mit
\Macro{addcontentsline}\IndexCmd{addcontentsline} oder bei Verwendung von
\Package{tocbasic} vorzugsweise mit
\DescRef{\LabelBase.cmd.addxcontentsline}\IndexCmd{addxcontentsline} (siehe
\autoref{sec:tocbasic.basics}, \DescPageRef{\LabelBase.cmd.addxcontentsline})
erzeugt. Dabei schreibt \LaTeX{} eine zugehörige Anweisung
\Macro{contentsline}\IndexCmd{contentsline} in die jeweilige Hilfsdatei. Beim
Einlesen dieser Hilfsdatei führt \LaTeX{} dann für jedes \Macro{contentsline}
eine Anweisung \Macro{l@\PName{Eintragsebene}} aus.

Wird später einer \PName{Eintragsebene} über
\DescRef{\LabelBase.cmd.DeclareTOCStyleEntry} ein \PName{Stil} zugewiesen, so
wird zunächst \PName{Initialisierungscode} ausgeführt, falls angegeben, und
dann \PName{Befehlscode} für die Definition von
\Macro{l@\PName{Eintragsebene}} verwendet. \PName{Befehlscode} ist also
letztlich der Code, der bei \Macro{l@\PName{Eintragsebene}} ausgeführt
wird. Dabei ist \texttt{\#1} der Name der Eintragsebene, während
\texttt{\#\#1} und \texttt{\#\#2} Platzhalter für die beiden Argumente von
\Macro{l@\PName{Eintragsebene}} sind.

Der \PName{Initialisierungscode} dient einerseits dazu, die Einstellungen
eines Stils zu initialisieren. Entwickler\textnote{Empfehlung!} sollten darauf
achten, dass wirklich alle Einstellungen hier bereits einen Wert erhalten. Nur
dann funktioniert \DescRef{\LabelBase.cmd.DeclareTOCStyleEntry} auch ohne
Angabe einer \PName{Optionenliste} fehlerfrei. Darüber hinaus hat der
\PName{Initialisierungscode} auch alle Optionen, die der jeweilige Stil
versteht, zu definieren. Zwingend vordefiniert wird lediglich
\Option{level}. Der eingestellte Wert für \Option{level} kann in
\PName{Befehlscode} mit \Macro{@nameuse}\PParameter{\#1tocdepth}%
\important{\Macro{\PName{Eintragsebene}tocdepth}} abgefragt werden, um ihn
beispielsweise mit dem Wert des Zählers
\DescRef{\LabelBase.counter.tocdepth}\IndexCounter{tocdepth} zu vergleichen.

Zur Definition neuer Optionen für die Eigenschaften einer
Eintragsebene existieren nur innerhalb von \PName{Initialisierungscode} die
Anweisungen \iffree{\Macro{DefineTOCEntryBooleanOption},
\Macro{DefineTOCEntryCommandOption}, \Macro{DefineTOCEntryIfOption},
\Macro{DefineTOCEntryLengthOption} und
\Macro{DefineTOCEntryNumberOption}}{\Macro{DefineTOCEntry\dots Option}}.
Diese Anweisungen definieren jeweils eine \PName{Option}, die bei ihrem
Aufruf eine Anweisung
\Macro{\PName{Präfix}\PName{Eintragsebene}\PName{Postfix}} mit dem übergebenen
Wert oder bei Fehlen einer Wertzuweisung mit dem \PName{Säumniswert}
definieren. Eine Besonderheit stellt \Macro{DefineTOCEntryIfOption} dar. Diese
definiert \Macro{\PName{Präfix}\PName{Eintragsebene}\PName{Postfix}} immer als
Anweisung mit zwei Argumenten. Ist der an die Option übergebene Wert einer der
Aktivierungswerte aus \autoref{tab:truefalseswitch},
\autopageref{tab:truefalseswitch}, so expandiert die Anweisung zum ersten
Argument. Ist der an die Option übergebene Wert hingegen ein
Deaktivierungswert, so expandiert die Anweisung zum zweiten Argument.

Neben\ChangedAt{v3.27}{\Package{tocbasic}} den normalen Optionen der Form
\Option{\PName{Schlüssel}=\PName{Wert}} werden von allen fünf
\Macro{DefineTOCEntry\dots Option}-Anweisungen automatisch Optionen der Form
\Option{\PName{Schlüssel}:=\PName{Eintragsebene}} definiert. Diese dienen
dazu, den Wert einer anderen \PName{Eintragsebene} zu kopieren, sofern der
Wert in einem Makro mit gleichem \PName{Präfix} und \PName{Postfix}
gespeichert ist. Bei den von \Package{tocbasic} vordefinierten Stilen ist das
für gleichnamige Optionen über Stilgrenzen hinweg der Fall.

Vergleichbar dazu werden von \Macro{DefineTOCEntryLengthOption} und
\Macro{DefineTOCEntryNumberOption} jeweils zusätzliche Optionen der Form
\Option{\PName{Schlüssel}+=\PName{Wert}} mitdefiniert, die dazu dienen, zu
dem in \Macro{\PName{Präfix}\PName{Eintragsebene}\PName{Postfix}} bereits
gespeicherten Wert den neuen \PName{Wert} zu addieren.

Die \PName{Erklärung} sollte ein möglichst kurzer Text sein, der den Sinn der
Option mit wenigen Schlagworten beschreibt. Er wird von \Package{tocbasic} bei
Fehlermeldungen, Warnungen und Informationen auf dem Terminal und in der
\File{log}-Datei ausgegeben.
\begin{Example}
Der einfachste Stil von \Package{tocbasic}, \PValue{gobble}, wurde mit
\begin{lstcode}
  \DeclareTOCEntryStyle{gobble}{}%
\end{lstcode}
definiert. Würde man nun mit
\begin{lstcode}
  \DeclareTOCStyleEntry[level=1]{gobble}{dummy}
\end{lstcode}
eine Eintragsebene \PValue{dummy} in diesem Stil definieren, so würde das%
\iffalse unter anderem\fi% Umbruchkorrektur
\begin{lstcode}
  \def\dummytocdepth{1}
  \def\l@dummy#1#2{}
\end{lstcode}
entsprechen.

Innerhalb von Stil \PValue{tocline} wird beispielsweise
\begin{lstcode}
  \DefineTOCEntryCommandOption{linefill}%
    [\TOCLineLeaderFill]{scr@tso@}{@linefill}%
    {filling between text and page number}%
\end{lstcode}
verwendet, um Option \Option{linefill} zu definieren. Durch die Angabe von
\DescRef{\LabelBase.cmd.TOCLineLeaderFill} als \PName{Säumniswert} würde ein
Aufruf wie
\begin{lstcode}
  \DeclareTOCStyleEntry[linefill]{tocline}{part}
\end{lstcode}
unter anderem die Definition
\begin{lstcode}
  \def\scr@tso@part@linefill{\TOCLineLeaderFill}
\end{lstcode}
vornehmen.
\end{Example}
Wer\textnote{Tipp!} sich selbst einen Stil definieren möchte, dem sei
empfohlen, zunächst die Definition des Stils \PValue{dottedtocline} zu
studieren. Nachdem dessen Definition verstanden wurde, gibt dann die deutlich
komplexere Definition von Stil \PValue{tocline} viele Hinweise darauf, wie die
Anweisungen sinnvoll zu verwenden sind.

In\textnote{Tipp!} vielen Fällen wird es jedoch auch ausreichen, einen der
vorhandenen Stile mit \DescRef{\LabelBase.cmd.CloneTOCEntryStyle} zu klonen
und dann dessen Initialisierungscode mit Anweisung
\DescRef{\LabelBase.cmd.TOCEntryStyleInitCode} oder
\DescRef{\LabelBase.cmd.TOCEntryStyleStartInitCode} abzuändern.

\Macro{DefineTOCEntryOption} dient eher der Definition der übrigen
Anweisungen und sollte in der Regel nicht direkt verwendet
werden. Normalerweise besteht dafür auch keine Notwendigkeit. Sie sei hier
nur der Vollständigkeit halber erwähnt.%
\EndIndexGroup


\begin{Declaration}
  \Macro{CloneTOCEntryStyle}\Parameter{Stil}\Parameter{neuer Stil}%
\end{Declaration}
Mit\ChangedAt{v3.20}{\Package{tocbasic}} dieser Anweisung kann ein
existierender \PName{Stil} geklont werden. Dabei wird ein \PName{neuer Stil}
mit denselben Eigenschaften und Voreinstellungen wie der existierende
\PName{Stil} deklariert. Das Paket selbst verwendet
\Macro{CloneTOCEntryStyle}, um den Stil \PValue{default} als Klon von
\PValue{dottedtocline} zu deklarieren. Die \KOMAScript-Klassen verwenden die
Anweisung um die Stile \PValue{part}, \PValue{section} und \PValue{chapter}
oder \PValue{subsection} als Klon von \PValue{tocline} zu deklarieren und dann
mit \DescRef{\LabelBase.cmd.TOCEntryStyleInitCode} und
\DescRef{\LabelBase.cmd.TOCEntryStyleStartInitCode} abzuändern. Der Stil
\PValue{default} wird von \Class{scrbook} und \Class{scrreprt} neu als Klon
von \PValue{section} und von \Class{scrartcl} als Klon von \PValue{subsection}
deklariert.%
\EndIndexGroup


\begin{Declaration}
  \Macro{TOCEntryStyleInitCode}\Parameter{Stil}%
                               \Parameter{Initialisierungscode}
  \Macro{TOCEntryStyleStartInitCode}\Parameter{Stil}%
                                    \Parameter{Initialisierungscode}
\end{Declaration}
Jeder\ChangedAt{v3.20}{\Package{tocbasic}} Verzeichniseintragsstil
verfügt über einen Initialisierungscode. Dieser wird immer dann aufgerufen,
wenn einer Verzeichnisebene der entsprechende \PName{Stil} mit
\DescRef{\LabelBase.cmd.DeclareTOCEntryStyle}\IndexCmd{DeclareTOCEntryStyle}
zugewiesen wird. Dieser\textnote{Achtung!} \PName{Initialisierungscode} sollte
keine globalen Seiteneffekte aufweisen, da er auch für lokale
Initialisierungen innerhalb anderer Anweisungen wie
\DescRef{\LabelBase.cmd.DeclareNewTOC}\IndexCmd{DeclareNewTOC} verwendet
wird. Der \PName{Initialisierungscode} dient einerseits dazu, Eigenschaften
für den jeweiligen \PName{Stil} zu definieren. Er setzt aber auch die
Standardeinstellungen für diese Eigenschaften.

Mit Hilfe der Anweisungen \Macro{TOCEntryStyleStartInitCode} und
\Macro{TOCEntryStyleInitCode} kann der für einen \PName{Stil} bereits
definierte Initialisierungscode um weiteren \PName{Initialisierungscode}
erweitert werden. Dabei fügt \Macro{TOCEntryStyleStartInitCode} den neuen
\PName{Intialisierungscode} vorn an, während \Macro{TOCEntryStyleInitCode} den
\PName{Initialisierungscode} hinten an den vorhandenen Code anfügt. Dies wird
beispielsweise von den \KOMAScript-Klassen verwendet, um für den von
\PValue{tocline} geklonten Stil \PValue{part} Füllung, Schrift und vertikalen
Abstand passend zu initialisieren.
\begin{Example}
Die Klassen \Class{scrbook} und
\Class{scrreprt} verwenden
\begin{lstcode}
  \CloneTOCEntryStyle{tocline}{section}
  \TOCEntryStyleStartInitCode{section}{%
    \expandafter\providecommand%
    \csname scr@tso@#1@linefill\endcsname
    {\TOCLineLeaderFill\relax}%
  }
\end{lstcode}
um den Stil \PValue{section} als abgewandelten Klon von \PValue{tocline} zu
definieren.%
\end{Example}%
\EndIndexGroup
\ExampleEndFix


\begin{Declaration}
  \Macro{LastTOCLevelWasHigher}
  \Macro{LastTOCLevelWasSame}
  \Macro{LastTOCLevelWasLower}
\end{Declaration}
Bei\ChangedAt{v3.20}{\Package{tocbasic}} Einträgen im Stil
\PValue{tocline} wird am Anfang abhängig vom Wert von
\Macro{lastpenalty}\IndexCmd{lastpenalty} eine dieser drei Anweisungen
ausgeführt. Dabei fügen \Macro{LastTOCLevelWasHigher} und
\Macro{LastTOCLevelWasSame} im vertikalen Modus
\Macro{addpenalty}\PParameter{\Macro{@lowpenalty}} ein und ermöglichen so
einen Umbruch vor Einträgen gleicher oder übergeordneter
Ebene. \Macro{LastTOCLevelWasLower} ist hingegen bisher leer definiert, so
dass zwischen einem Eintrag und seinem ersten Untereintrag normalerweise ein
Umbruch untersagt ist.

Anwender sollten diese Anweisungen nicht umdefinieren. Stattdessen können und
sollten Änderungen bei Zuweisung des Stils an eine Eintragsebene gezielt über
die Eigenschaften \PValue{onstartlowerlevel}, \PValue{onstartsamelevel} und
\PValue{onstarthigherlevel} vorgenommen werden.%
\EndIndexGroup


\begin{Declaration}
  \Macro{TOCLineLeaderFill}\OParameter{Füllzeichen}
\end{Declaration}
Die\ChangedAt{v3.20}{\Package{tocbasic}} Anweisung ist dazu gedacht,
als Wert für Eigenschaft \Option{linefill} des Verzeichniseintragsstils
\PName{tocline} verwendet zu werden. Sie erzeugt dann eine Verbindung zwischen
dem Ende des Textes eines Eintrags und der zugehörigen Seitenzahl. Das
\PName{Füllzeichen}, das dazu in regelmäßigem Abstand wiederholt wird, kann
als optionales Argument angegeben werden. Voreinstellung ist ein
Punkt.\textnote{Voreinstellung}

Wie der Name schon vermuten lässt, werden die Füllzeichen mit Hilfe von
\Macro{leaders} gesetzt. Als Abstand wird wie bei der \LaTeX-Kern-Anweisung
\Macro{@dottedtocline} vor und nach dem Füllzeichen
\Macro{mkern}\Macro{@dotsep}\Unit{\texttt{mu}} verwendet.%
\EndIndexGroup
%
\EndIndexGroup


\section{Interne Anweisungen für Klassen- und Paketautoren}
\seclabel{internals}

Das Paket \Package{tocbasic} bietet einige interne Anweisungen, deren
Benutzung durch Klassen- und Paketautoren freigegeben ist. Diese Anweisungen
beginnen alle mit \Macro{tocbasic@}. Aber\textnote{Achtung!} auch Klassen- und
Paketautoren sollten diese Anweisungen \iffalse nur verwenden und \fi nicht
etwa umdefinieren!\iffalse Ihre interne Funktion kann jederzeit geändert oder
erweitert werden, so dass jede Umdefinierung der Anweisungen die Funktion von
\Package{tocbasic} erheblich beschädigen könnte!\fi% Umbruchkorrektur


\begin{Declaration}
  \Macro{tocbasic@extend@babel}\Parameter{Dateierweiterung}
\end{Declaration}
Das Paket \Package{babel}\IndexPackage{babel} (siehe \cite{package:babel})
bzw. ein \LaTeX-Kern, der um die Sprachverwaltung von \Package{babel}
erweitert wurde, schreibt bei jeder Sprachumschaltung am Anfang oder innerhalb
eines Dokuments in die Dateien mit den Dateierweiterungen \File{toc},
\File{lof} und \File{lot} Anweisungen, um diese Sprachumschaltung in diesen
Dateien mit zu führen. \Package{tocbasic} erweitert diesen Mechanismus so,
dass mit Hilfe von \Macro{tocbasic@extend@babel} auch andere
\PName{Dateierweiterungen} davon profitieren. Das Argument
\PName{Dateierweiterung} sollte dabei vollständig expandiert sein!
Anderenfalls besteht die Gefahr, dass etwa die Bedeutung eines Makros zum
Zeitpunkt der tatsächlichen Auswertung bereits geändert wurde.

In der Voreinstellung wird diese Anweisung normalerweise für alle
\PName{Dateierweiterungen}, die mit \DescRef{\LabelBase.cmd.addtotoclist} zur
Liste der bekannten Dateierweiterungen hinzugefügt werden, aufgerufen. Über
die Eigenschaft \PValue{nobabel}\important{\PValue{nobabel}} (siehe
\DescRef{\LabelBase.cmd.setuptoc}, \autoref{sec:tocbasic.toc},
\DescPageRef{\LabelBase.cmd.setuptoc}) kann das unterdrückt werden. Für die
Dateinamenerweiterungen \File{toc}, \File{lof} und \File{lot} unterdrückt
\Package{tocbasic} dies bereits selbst, %
\iftrue% Umbruchvarianten
da \Package{babel} sie von sich aus vornimmt.%
\else%
damit die Umschaltung nicht mehrfach in die zugehörigen Dateien
eingetragen wird.%
\fi
\iffalse% Umbruchkorrektur
Normalerweise gibt es keinen Grund, diese Anweisung selbst aufzurufen. Es sind
allerdings Verzeichnisse denkbar, die nicht unter der Kontrolle von
\Package{tocbasic} stehen, also nicht in der Liste der bekannten
Dateierweiterungen geführt werden, aber trotzdem die Spracherweiterung für
\Package{babel} nutzen sollen. Für derartige Verzeichnisse wäre die Anweisung
explizit aufzurufen. Bitte\textnote{Achtung!} achten Sie jedoch darauf, dass
dies für jede Dateierweiterung nur einmal geschieht!%
\fi
\EndIndexGroup


\begin{Declaration}
  \Macro{tocbasic@starttoc}\Parameter{Dateierweiterung}
\end{Declaration}
Diese Anweisung ist der eigentliche Ersatz der Anweisung
\Macro{@starttoc}\IndexCmd{@starttoc}\important{\Macro{@starttoc}} aus dem
\LaTeX-Kern. Es ist die Anweisung, die sich hinter
\DescRef{\LabelBase.cmd.listoftoc*} (siehe \autoref{sec:tocbasic.toc},
\DescPageRef{\LabelBase.cmd.listoftoc*}) verbirgt. Klassen- oder Paketautoren,
die Vorteile von \Package{tocbasic} nutzen wollen, sollten zumindest diese
Anweisung, besser jedoch \DescRef{\LabelBase.cmd.listoftoc} verwenden. Die
Anweisung baut selbst auf \Macro{@starttoc} auf, setzt allerdings zuvor
\Length{parskip}\IndexLength{parskip}\important[i]{\Length{parskip},
  \Length{parindent}, \Length{parfillskip}} und
\Length{parindent}\IndexLength{parindent} auf 0 und
\Length{parfillskip}\IndexLength{parfillskip} auf 0 bis unendlich. Außerdem
wird \Macro{@currext}\important{\Macro{@currext}}\IndexCmd{@currext} auf die
aktuelle Dateierweiterung gesetzt, damit diese in den nachfolgend ausgeführten
Haken\Index{Haken} ausgewertet werden kann. Die Erklärungen der Haken finden
Sie im Anschluss.

Da\textnote{Achtung!} \LaTeX{} bei der Ausgabe eines Verzeichnisses auch
gleich eine neue Verzeichnisdatei zum Schreiben öffnet, kann der Aufruf dieser
Anweisung zu einer Fehlermeldung der Art
\begin{lstoutput}
  ! No room for a new \write .
  \ch@ck ...\else \errmessage {No room for a new #3}
                                                    \fi   
\end{lstoutput}
führen, wenn keine Schreibdateien mehr zur Verfügung stehen. Abhilfe kann in
diesem Fall das Laden des in \autoref{cha:scrwfile} beschriebenen Pakets
\Package{scrwfile}\important{\Package{scrwfile}}\IndexPackage{scrwfile}
oder die Verwendung von Lua\LaTeX{} bieten.%
\EndIndexGroup


\begin{Declaration}
  \Macro{tocbasic@@before@hook}
  \Macro{tocbasic@@after@hook}
\end{Declaration}
Der Haken\Index{Haken} \Macro{tocbasic@@before@hook} wird unmittelbar vor dem
Einlesen der Verzeichnisdatei, noch vor den mit
\DescRef{\LabelBase.cmd.BeforeStartingTOC} definierten Anweisungen
ausgeführt. Es ist erlaubt, diesen Haken mit Hilfe von
\Macro{g@addto@macro}\IndexCmd{g@addto@macro} zu erweitern.

Ebenso wird \Macro{tocbasic@@after@hook} unmittelbar nach der
Verzeichnisdatei, aber noch vor den mit
\DescRef{\LabelBase.cmd.AfterStartingTOC} definierten Anweisungen
ausgeführt. Es ist erlaubt, diesen Haken mit Hilfe von
\Macro{g@addto@macro}\IndexCmd{g@addto@macro} zu erweitern.

\KOMAScript{} nutzt diese Haken, um Verzeichnisse mit dynamischer Anpassung an
die Breite der Gliederungsnummern zu ermöglichen. Ihre Verwendung ist Klassen
und Paketen vorbehalten. Anwender\textnote{Achtung!} sollten sich auf
\DescRef{\LabelBase.cmd.BeforeStartingTOC} und
\DescRef{\LabelBase.cmd.AfterStartingTOC} beschränken. Paketautoren sollten
ebenfalls vorzugsweise diese beiden Anwenderanweisungen verwenden!  Ausgaben
innerhalb der beiden Haken sind nicht gestattet!

Wird keine\textnote{Achtung!} der Anweisungen
\DescRef{\LabelBase.cmd.listofeachtoc}, \DescRef{\LabelBase.cmd.listoftoc} und
\DescRef{\LabelBase.cmd.listoftoc*} für die Ausgabe der Verzeichnisse
verwendet, sollten die Anweisungen für die Haken\Index{Haken} trotzdem
aufgerufen werden.%
\EndIndexGroup


\begin{Declaration}
  \Macro{tb@\PName{Dateierweiterung}@before@hook}
  \Macro{tb@\PName{Dateierweiterung}@after@hook}
\end{Declaration}
Diese Anweisungen werden direkt nach
\DescRef{\LabelBase.cmd.tocbasic@@before@hook} bzw. vor
\DescRef{\LabelBase.cmd.tocbasic@@after@hook} für das jeweilige Verzeichnis
mit der entsprechenden \PName{Dateierweiterung}
ausgeführt. Sie\textnote{Achtung!}  dürfen keinesfalls von Klassen- und
Paketautoren verändert werden. Werden für die Ausgabe der Verzeichnisse die
Anweisungen \DescRef{\LabelBase.cmd.listoftoc},
\DescRef{\LabelBase.cmd.listoftoc*} und \DescRef{\LabelBase.cmd.listofeachtoc}
nicht verwendet, sollten die beiden Anweisungen für die Haken\Index{Haken}
trotzdem aufgerufen werden, soweit sie definiert sind. Die Anweisungen können
auch undefiniert sein.%
\iffalse% Umbruchkorrektur / inzwischen passt auch \@ifundefined
\ Für einen entsprechenden Test siehe
\DescRef{scrbase.cmd.scr@ifundefinedorrelax}\IndexCmd{scr@ifundefinedorrelax}
in \autoref{sec:scrbase.if},
\DescPageRef{scrbase.cmd.scr@ifundefinedorrelax}.%
\fi%
\EndIndexGroup


\begin{Declaration}
  \Macro{tocbasic@listhead}\Parameter{Titel}
\end{Declaration}
Diese Anweisung wird von \DescRef{\LabelBase.cmd.listoftoc} und
\DescRef{\LabelBase.cmd.listofeachtoc} verwendet, um die Anweisung zum Setzen
der Überschrift eines Verzeichnisses aufzurufen. Das kann entweder die
vordefinierte Anweisung des Pakets \Package{tocbasic} oder eine individuelle
Anweisung sein. Wenn Sie Ihre eigene Anweisung für die Überschrift definieren,
können Sie ebenfalls \Macro{tocbasic@listhead} verwenden. In diesem Fall
sollte vor dem Aufruf von \Macro{tocbasic@listhead} die Anweisung
\Macro{@currext}\important{\Macro{@currext}}\IndexCmd{@currext} auf die
Dateinamenerweiterung, die zu diesem Verzeichnis gehört, gesetzt werden.%
\EndIndexGroup


\begin{Declaration}
  \Macro{tocbasic@listhead@\PName{Dateierweiterung}}\Parameter{Titel}
\end{Declaration}
Ist diese individuelle Anweisung für das Setzen einer Verzeichnisüberschrift
definiert, so verwendet \DescRef{\LabelBase.cmd.tocbasic@listhead}
sie. Anderenfalls definiert \DescRef{\LabelBase.cmd.tocbasic@listhead} diese
vor der Verwendung.%
\EndIndexGroup


\begin{Declaration}
  \Macro{tocbasic@addxcontentsline}\Parameter{Dateierweiterung}
                                   \Parameter{Ebene}
                                   \Parameter{Gliederungsnummer}
                                   \Parameter{Eintrag}
  \Macro{nonumberline}
\end{Declaration}
Anweisung\ChangedAt{v3.12}{\Package{tocbasic}}
\Macro{tocbasic@addxcontentsline} nimmt einen \PName{Eintrag} der angegebenen
\PName{Ebene} in das über die \PName{Dateierweiterung} spezifizierte
Verzeichnis vor. Ob der Eintrag nummeriert wird\iffree{ oder
  nicht}{}% Umbruchvarianten
, hängt davon ab, ob das Argument \PName{Gliederungsnummer} leer ist oder
nicht. Im Falle eines leeren Argument wird dem \PName{Eintrag} ein
\Macro{nonumberline} ohne Argument vorangestellt. Anderenfalls wird wie
gewohnt \DescRef{\LabelBase.cmd.numberline} mit \PName{Gliederungsnummer} als
Argument verwendet.

Die Anweisung \Macro{nonumberline} wird innerhalb
\DescRef{\LabelBase.cmd.listoftoc} (siehe \autoref{sec:tocbasic.toc},
\DescPageRef{\LabelBase.cmd.listoftoc}) entsprechend der Eigenschaft
\PValue{numberline} (siehe \autoref{sec:tocbasic.toc},
\DescPageRef{\LabelBase.cmd.setuptoc}) umdefiniert. Dadurch wirkt sich das
Setzen oder Löschen dieser Eigenschaft \iffree{bereits beim nächsten
  \LaTeX-Lauf}{unmittelbar} % Umbruchvarianten
aus.%
%
\EndIndexGroup


\begin{Declaration}
  \Macro{tocbasic@DependOnPenaltyAndTOCLevel}\Parameter{Eintragsebene}
  \Macro{tocbasic@SetPenaltyByTOCLevel}\Parameter{Eintragsebene}
\end{Declaration}
Der\ChangedAt{v3.20}{\Package{tocbasic}} Verzeichniseintragsstil
\PValue{tocline} (siehe \autoref{sec:tocbasic.tocstyle}) setzt am Ende jedes
Eintrags über \Macro{tocbasic@SetPenaltyByTOCLevel} \Macro{penalty} so, dass
nach einem Eintrag kein Seitenumbruch erfolgen darf. Der genaue Wert wird
dabei abhängig von der \PName{Eintragsebene} gewählt.

% \begin{sloppypar}
\begingroup%
\emergencystretch=2em\relax%
Über \Macro{tocbasic@DependOnPenaltyAndTOCLevel} wird am Anfang eines
Eintrags, abhängig von \Macro{lastpenalty} und \PName{Eintragsebene}, die über
die Eigenschaft \Option{onstartlowerlevel} im internen Makro
\Macro{scr@tso@\PName{Eintragsebene}@LastTOCLevelWasHigher}, über die
Eigenschaft \Option{onstartsamelevel} im zugehörigen, internen Makro
\Macro{scr@tso@\PName{Eintragsebene}@LastTOCLevelWasSame} oder über die
Eigenschaft \Option{onstarthigherlevel} im zugehörigen, internen Makro
\Macro{scr@tso@\PName{Eintragsebene}@LastTOCLevelWasLower} gespeicherte Aktion
ausgeführt. In der Voreinstellung\textnote{Voreinstellung} erlauben die ersten
beiden einen Umbruch, wenn sie im vertikalen Modus ausgeführt werden.\par%
\endgroup
%\end{sloppypar}

Entwicklern, die eigene Stile kompatibel mit \PValue{tocline} erstellen
wollen, sei empfohlen, dieses Verhalten zu kopieren. Zu diesem Zweck dürfen
sie auf diese eigentlich internen Makros zurückgreifen.%
\EndIndexGroup


\section{Ein komplettes Beispiel}
\seclabel{example}

In diesem Abschnitt finden Sie ein komplettes Beispiel, wie eine eigene
Gleitumgebung einschließlich Verzeichnis und \KOMAScript-Integration mit Hilfe
von \Package{tocbasic} definiert werden kann. In diesem Beispiel werden
interne Anweisungen, also solche mit »\texttt{@}« im Namen
verwendet. Das\textnote{Achtung!}  bedeutet, dass die Anweisungen entweder in
einem eigenen Paket, einer Klasse oder zwischen
\Macro{makeatletter}\important[i]{\Macro{makeatletter}\\\Macro{makeatother}}%
\IndexCmd{makeatletter} und \Macro{makeatother}\IndexCmd{makeatother}
verwendet werden müssen.

Als\textnote{Umgebung} erstes wird eine Umgebung benötigt, die diese neue
Gleitumgebung \iftrue% Umbruchkorrektur (siehe auch nach dem Code)
namens \Environment{remarkbox} bereitstellt:%
\else%
bereitstellt. Das geht ganz einfach mit:%
\fi%
\begin{lstcode}
  \newenvironment{remarkbox}%
    {\@float{remarkbox}}%
    {\end@float}
\end{lstcode}
\iffalse Die neue Umgebung heißt also \Environment{remarkbox}.\fi % siehe oben

Jede\textnote{Platzierung} Gleitumgebung hat eine Standardplatzierung. Diese
setzt sich aus einer oder mehreren der bekannten Platzierungsoptionen
\PValue{b}, \PValue{h}, \PValue{p} und \PValue{t} zusammen:
\begin{lstcode}
  \newcommand*{\fps@remarkbox}{tbp}
\end{lstcode}
Die neue Gleitumgebung soll also in der Voreinstellung nur oben, unten oder
auf einer eigenen Seite platziert werden dürfen.

Gleitumgebungen\textnote{Gleittyp} haben außerdem einen numerischen Gleittyp
zwischen 1 und 31. Umgebungen, bei denen das gleiche Bit
im Gleittyp gesetzt ist, dürfen sich nicht gegenseitig überholen. Abbildungen
und Tabellen haben normalerweise den Typ~1 und 2. Abbildungen dürfen also
Tabellen überholen und umgekehrt.
\begin{lstcode}
  \newcommand*{\ftype@remarkbox}{4}
\end{lstcode}
Die neue Umgebung hat den Typ~4, darf also Tabellen und Abbildungen überholen
und von diesen überholt werden.

Gleitumgebungen\textnote{Nummer} haben außerdem eine Nummer.
\begin{lstcode}
  \newcounter{remarkbox}
  \newcommand*{\remarkboxformat}{%
    Merksatz~\theremarkbox\csname autodot\endcsname
  }
  \newcommand*{\fnum@remarkbox}{\remarkboxformat}
\end{lstcode}
Hier wird zunächst ein neuer Zähler definiert, der unabhängig von Kapiteln
oder sonstigen Gliederungszählern ist. Dabei definiert \LaTeX{} auch gleich
\Macro{theremarkbox} mit der Standardausgabe als arabische Zahl. Diese wird
dann in der Definition der formatierten Ausgabe verwendet. Die formatierte
Ausgabe wird wiederum als Gleitumgebungsnummer für die Verwendung in
\DescRef{maincls.cmd.caption} definiert.

Gleitumgebungen\textnote{Dateierweiterung} haben Verzeichnisse und diese haben
eine Datei mit dem Namen \Macro{jobname} und einer Dateierweiterung.
\begin{lstcode}
  \newcommand*{\ext@remarkbox}{lor}
\end{lstcode}
Als Dateierweiterung verwenden wir also »\File{lor}«.

Die\important{\Package{tocbasic}} Umgebung selbst steht damit. Es fehlt
allerdings das Verzeichnis. Damit wir dabei möglichst wenig selbst machen
müssen, verwenden wir das Paket \Package{tocbasic}. Dieses wird in Dokumenten
mit
\begin{lstcode}
  \usepackage{tocbasic}
\end{lstcode}
geladen. Ein Klassen- oder Paketautor würde hingegen
\begin{lstcode}
  \RequirePackage{tocbasic}
\end{lstcode}
verwenden.

Nun\textnote{Dateierweiterung} machen wir die neue Dateierweiterung dem Paket
\Package{tocbasic} bekannt.
\begin{lstcode}
  \addtotoclist[float]{lor}
\end{lstcode}
Dabei verwenden wir als Besitzer \PValue{float}. Damit beziehen sich
automatisch alle Optionen, die von den \KOMAScript-Klassen für Verzeichnisse
von Gleitumgebungen angeboten werden, auch auf das neue Verzeichnis.

Jetzt\textnote{Verzeichnistitel} definieren wir noch einen Titel für dieses
Verzeichnis:
\begin{lstcode}
  \newcommand*{\listoflorname}{Verzeichnis der Merksätze}
\end{lstcode}
Normalerweise würde man in einem Paket übrigens zunächst einen englischen
Titel definieren und dann beispielsweise mit Hilfe des Pakets
\Package{scrbase} Titel für alle weiteren Sprachen, die man unterstützen
will. Siehe dazu \autoref{sec:scrbase.languageSupport}, ab
\autopageref{sec:scrbase.languageSupport}.

Jetzt\textnote{Verzeichniseintrag}\index{Verzeichnis>Eintrag} müssen wir nur
noch definieren, wie ein einzelner Eintrag in dem Verzeichnis aussehen soll:
\begin{lstcode}
  \newcommand*{\l@remarkbox}{\l@figure}
\end{lstcode}
Weil das die einfachste Lösung ist, wurde hier festgelegt, dass Einträge in
das Verzeichnis der Merksätze genau wie Einträge in das Abbildungsverzeichnis
aussehen sollen. Man hätte stattdessen auch die Einstellungen selbst kopieren
können:
\begin{lstcode}
  \DeclareTOCStyleEntry[level:=figure,%
                        indent:=figure,%
                        numwidth:=figure]%
                       {tocline}{remarkbox}
\end{lstcode}
Selbstverständlich wären auch explizite Festlegungen möglich:
\begin{lstcode}
  \DeclareTOCStyleEntry[level=1,%
                        indent=1.5em,%
                        numwidth=2.3em]%
                       {tocline}{remarkbox}
\end{lstcode}

Außerdem\textnote{Kapiteleintrag} wollen Sie, dass sich Kapiteleinträge auf das
Verzeichnis auswirken.
\begin{lstcode}
  \setuptoc{lor}{chapteratlist}
\end{lstcode}
Das Setzen dieser Eigenschaft ermöglicht dies bei Verwendung einer
\KOMAScript-Klasse und jeder anderen Klasse, die diese Eigenschaft
unterstützt. Leider gehören die Standardklassen nicht dazu.

Das\textnote{Verzeichnis} genügt schon. Der Anwender kann nun bereits
wahlweise mit Hilfe der Optionen einer \KOMAScript-Klasse oder
\DescRef{\LabelBase.cmd.setuptoc} verschiedene Formen der Überschrift (ohne
Inhaltsverzeichniseintrag, mit Inhaltsverzeichniseintrag, mit Nummerierung)
wählen und das Verzeichnis mit
\DescRef{\LabelBase.cmd.listoftoc}\PParameter{lor} ausgeben. Mit einem
schlichten
\begin{lstcode}
  \newcommand*{\listofremarkboxes}{\listoftoc{lor}}
\end{lstcode}
kann man die Anwendung noch etwas vereinfachen.

Wie Sie gesehen haben, beziehen sich gerade einmal fünf einzeilige
Anweisungen, von denen nur drei bis vier wirklich notwendig sind, auf das
Verzeichnis selbst. Trotzdem bietet dieses Verzeichnis bereits die
Möglichkeit, es zu nummerieren oder auch nicht nummeriert in das
Inhaltsverzeichnis aufzunehmen. Es kann sogar per Eigenschaft bereits eine
tiefere Gliederungsebene gewählt werden. Kolumnentitel werden für \KOMAScript,
die Standardklassen und alle Klassen, die \Package{tocbasic} explizit
unterstützen, angepasst gesetzt. Unterstützende Klassen beachten das neue
Verzeichnis sogar beim Wechsel zu einem neuen Kapitel. Sprachumschaltungen
durch \Package{babel} werden in dem Verzeichnis ebenfalls berücksichtigt.

Natürlich\textnote{Verzeichniseigenschaften} kann ein Paketautor weiteres
hinzufügen. So könnte er explizit Optionen anbieten, um die Verwendung von
\DescRef{\LabelBase.cmd.setuptoc} vor dem Anwender zu verbergen. Andererseits
kann er auch auf diese Anleitung zu \Package{tocbasic} verweisen, wenn es
darum geht, die entsprechenden Möglichkeiten zu erklären. Vorteil ist dann,
dass der Anwender automatisch von etwaigen zukünftigen Erweiterungen von
\Package{tocbasic} profitiert. Soll der Anwender aber nicht mit der Tatsache
belastet werden, dass für die Merksätze die Dateierweiterung \File{lor}
verwendet wird, so genügt
\begin{lstcode}
  \newcommand*{\setupremarkboxes}{\setuptoc{lor}}
\end{lstcode}
\iftrue% Umbruchkorrektur
um eine als Argument an \Macro{setupremarkboxes} übergebene Liste von
Eigenschaften direkt als Liste von %
\else%
um über \Macro{setupremarkboxes}
\fi
Eigenschaften für \File{lor} zu setzen.


\section{Alles mit einer Anweisung}
\seclabel{declarenewtoc}

Das Beispiel aus dem vorherigen Abschnitt hat gezeigt, dass es mit mit
\Package{tocbasic} recht einfach ist, eigene Gleitumgebungen mit eigenen
Verzeichnissen zu definieren. In diesem Abschnitt wird gezeigt, dass es sogar
noch einfacher gehen kann.

\begin{Declaration}
  \Macro{DeclareNewTOC}\OParameter{Optionenliste}
                       \Parameter{Dateierweiterung}
\end{Declaration}%
Mit\ChangedAt{v3.06}{\Package{tocbasic}} dieser Anweisung wird in einem
einzigen Schritt ein neues Verzeichnis, dessen Überschrift und die Bezeichnung
für die Einträge unter Kontrolle von \Package{tocbasic} definiert. Optional
können dabei gleichzeitig gleitende oder nicht gleitende Umgebungen definiert
werden, innerhalb derer \DescRef{maincls.cmd.caption}%
\important{\DescRef{maincls.cmd.caption}}\IndexCmd{caption} Einträge für
dieses neue Verzeichnis erzeugt. Auch die Erweiterungen
\DescRef{maincls.cmd.captionabove}\important[i]{%
  \DescRef{maincls.cmd.captionabove}\\
  \DescRef{maincls.cmd.captionbelow}}, \DescRef{maincls.cmd.captionbelow} und
\DescRef{maincls.env.captionbeside} aus den \KOMAScript-Klassen (siehe
\autoref{sec:maincls.floats}, ab \DescPageRef{maincls.cmd.captionabove})
können dann verwendet werden.

\PName{Dateierweiterung} definiert dabei die Dateiendung der Hilfsdatei, die
das Verzeichnis repräsentiert, wie dies in \autoref{sec:tocbasic.basics}, ab
\autopageref{sec:tocbasic.basics} bereits erläutert
wurde. Dieser\textnote{Achtung!} Parameter muss angegeben werden und darf
nicht leer sein!

\PName{Optionenliste} ist eine durch Komma getrennte Liste, wie dies auch von
\DescRef{maincls.cmd.KOMAoptions} (siehe \autoref{sec:typearea.options},
\DescPageRef{typearea.cmd.KOMAoptions}) bekannt ist. Diese
Optionen\textnote{Achtung!} können jedoch \emph{nicht} mit
\DescRef{maincls.cmd.KOMAoptions}\IndexCmd{KOMAoptions} gesetzt werden! Eine
Übersicht über die möglichen Optionen bietet
\autoref{tab:tocbasic.DeclareNewTOC-options}\iffalse ab
\autopageref{tab:tocbasic.DeclareNewTOC-options}\fi.

Wird\ChangedAt{v3.20}{\Package{tocbasic}} Option
\Option{tocentrystyle} nicht gesetzt, so wird bei Bedarf der Stil
\PValue{default} verwendet. Näheres zu diesem Stil ist
\autoref{sec:tocbasic.tocstyle} zu entnehmen. Soll kein Befehl für
Verzeichniseinträge definiert werden, so kann ein leeres Argument, also
wahlweise \OptionValue{tocentrystyle}{}\iffree{}{\unskip} oder
\OptionValue{tocentrystyle}{\PParameter{}} verwendet werden.

Abhängig\ChangedAt{v3.21}{\Package{tocbasic}} vom Stil der
Verzeichniseinträge können auch alle für diesen Stil gültigen Eigenschaften
gesetzt werden, indem die entsprechenden in
\autoref{tab:tocbasic.tocstyle.attributes} ab
\autopageref{tab:tocbasic.tocstyle.attributes} aufgeführten Namen, mit dem
Präfix \PValue{tocentry} versehen, in der \PName{Optionenliste} angegeben
werden. Nachträgliche Änderungen am Stil sind mit
\DescRef{\LabelBase.cmd.DeclareTOCStyleEntry}%
\IndexCmd{DeclareTOCStyleEntry}%
\important{\DescRef{\LabelBase.cmd.DeclareTOCStyleEntry}} jederzeit
möglich. Siehe dazu \autoref{sec:tocbasic.tocstyle},
\DescPageRef{\LabelBase.cmd.DeclareTOCStyleEntry}.

% Umbruchoptimierung: Tabelle verschoben.
\begin{desclist}
  \renewcommand*{\abovecaptionskipcorrection}{-\normalbaselineskip}%
  \desccaption[{%
    Optionen für die Anweisung \Macro{DeclareNewTOC}%
    }]{%
    Optionen für die Anweisung
    \Macro{DeclareNewTOC}\ChangedAt{v3.06}{\Package{tocbasic}}%
    \label{tab:tocbasic.DeclareNewTOC-options}%
  }{%
    Optionen für die Anweisung \Macro{DeclareNewTOC} (\emph{Fortsetzung})%
  }%
  \entry{\OptionVName{atbegin}{Code}\ChangedAt{v3.09}{\Package{tocbasic}}}{%
    Falls eine neue Gleitumgebung oder nicht gleitende Umgebung definiert
    wird, so wird \PName{Code} jeweils am Anfang dieser Umgebung ausgeführt.%
  }%
  \entry{\OptionVName{atend}{Code}\ChangedAt{v3.09}{\Package{tocbasic}}}{%
    Falls eine neue Gleitumgebung oder nicht gleitende Umgebung definiert
    wird, so wird \PName{Code} jeweils am Ende dieser Umgebung ausgeführt.%
  }%
  \entry{\OptionVName{category}{Kategorie}}{%
    \ChangedAt{v3.27}{\Package{tocbasic}}%
    Diese Option kann als Synonym für \OptionVName{owner}{Besitzer} verwendet
    werden.}%
  \entry{\OptionVName{counterwithin}{\LaTeX-Zähler}}{%
    Falls eine neue Gleitumgebung oder eine nicht gleitende Umgebung definiert
    wird, so wird für diese auch ein neuer Zähler
    \Counter{\PName{Eintragstyp}} (siehe Option \Option{type})
    angelegt. Dieser Zähler kann, in gleicher Weise wie beispielsweise der
    Zähler \Counter{figure} bei \Class{book}-Klassen von Zähler
    \Counter{chapter} abhängt, von einem \PName{\LaTeX-Zähler} abhängig gemacht
    werden. Eine\ChangedAt{v3.35}{\Package{tocbasic}} Einstellung
    \OptionValue{counterwithin}{chapter} wird bei Klassen mit
    \DescRef{maincls.cmd.chapter}\IndexCmd{chapter} jedoch nur im Hauptteil
    (siehe \DescRef{maincls.cmd.frontmatter}\IndexCmd{frontmatter},
    \DescRef{maincls.cmd.mainmatter} und
    \DescRef{maincls.cmd.backmatter}\IndexCmd{backmatter} in
    \autoref{sec:maincls.separation}, \DescPageRef{maincls.cmd.frontmatter})
    und nur dann beachtet, wenn der Zähler
    \Counter{chapter}\IndexCounter{chapter} bei der Ausgabe größer als Null
    ist. Bei Klassen ohne \DescRef{maincls.cmd.chapter} gilt dies entsprechend
    für die Einstellung \OptionValue{counterwithin}{section} und Zähler
    \Counter{section}\IndexCounter{section}\IndexCmd{section}.%
  }%
  \entry{\Option{float}}{%
    Es wird nicht nur ein neuer Verzeichnistyp definiert, sondern auch
    Gleitumgebungen \Environment{\PName{Eintragstyp}} (siehe Option
    \Option{type}) und \Environment{\PName{Eintragstyp}*}
    (vgl. \Environment{figure} und \Environment{figure*}).}%
  \entry{\OptionVName{floatpos}{Gleitverhalten}}{%
    Jede Gleitumgebung hat ein voreingestelltes \PName{Gleitverhalten}, das
    über das optionale Argument der Gleitumgebung geändert werden kann. Mit
    dieser Option wird das \PName{Gleitverhalten} für die optional erstellbare
    Gleitumgebung (siehe Option \Option{float}) festgelegt. Die Syntax und
    Semantik sind dabei mit der des optionalen Arguments für die Gleitumgebung
    identisch. Wird die Option nicht verwendet, so ist das voreingestellte
    Gleitverhalten \texttt{tbp}, also \emph{top}, \emph{bottom},
    \emph{page}.}%
  \entry{\OptionVName{floattype}{Gleittyp}}{%
    Jede Gleitumgebung hat einen numerischen Typ. Gleitumgebungen, bei denen
    in diesem \PName{Gleittyp} nur unterschiedliche Bits gesetzt sind, können
    sich gegenseitig überholen. Die Gleitumgebungen \Environment{figure} und
    \Environment{table} haben normalerweise die Typen 1 und 2, können sich
    also gegenseitig überholen. Es sind Typen von 1 bis 31 (alle Bits gesetzt,
    kann also keinen anderen Typ überholen und von keinem anderen Typen
    überholt werden) zulässig. Wird kein Typ angegeben, so wird mit 16 der
    höchst mögliche Ein-Bit-Typ verwendet.}%
  \entry{\Option{forcenames}}{%
    Siehe Option \Option{name} und \Option{listname}.}%
  \entry{\OptionVName{hang}{Einzug}}{%
    \ChangedAt{v3.20}{\Package{tocbasic}}%
    \ChangedAt{v3.21}{\Package{tocbasic}}%
    Diese Option gilt seit \KOMAScript~3.20 als überholt. Die Breite der
    Nummer des Verzeichniseintrags\index{Verzeichnis>Eintrag} ist nun
    stattdessen als Eigenschaft in Abhängigkeit des Verzeichniseintragsstils
    von Option \Option{tocentrystyle} anzugeben. Bei den Stilen von
    \KOMAScript{} wäre das beispielsweise die Eigenschaft \PValue{numwidth}
    und damit Option \Option{tocentrynumwidth}. Besitzt ein Stil diese
    Eigenschaft, so wird sie von \Macro{DeclareNewTOC} mit 1,5\Unit{em}
    voreingestellt. Diese Voreinstellung kann durch explizite Angabe von
    \OptionVName{tocentrynumwidth}{Wert} leicht mit einem anderen \PName{Wert}
    überschrieben werden. Für Abbildungen verwenden die \KOMAScript-Klassen
    beispielsweise den \PName{Wert} \PValue{2.3em}.}%
  \entry{\OptionVName{indent}{Einzug}}{%
    \ChangedAt{v3.20}{\Package{tocbasic}}%
    \ChangedAt{v3.21}{\Package{tocbasic}}%
    Diese Option gilt seit \KOMAScript~3.20 als überholt. Der Einzug des
    Verzeichniseintrags\index{Verzeichnis>Eintrag} von links ist nun
    stattdessen als Eigenschaft in Abhängigkeit des Verzeichniseintragsstils
    von Option \Option{tocentrystyle} anzugeben. Bei den Stilen von
    \KOMAScript{} wäre das beispielsweise die Eigenschaft \PValue{indent} und
    damit Option \Option{tocentryindent}. Besitzt ein Stil diese Eigenschaft,
    so wird sie von \Macro{DeclareNewTOC} mit 1\Unit{em} voreingestellt. Diese
    Voreinstellung kann durch explizite Angabe von
    \OptionVName{tocentryindent}{Wert} leicht mit einem anderen \PName{Wert}
    überschrieben werden. Für Abbildungen verwenden die \KOMAScript-Klassen
    beispielsweise den \PName{Wert} \PValue{1.5em}.}%
  \entry{\OptionVName{level}{Gliederungsebene}}{%
    \ChangedAt{v3.20}{\Package{tocbasic}}%
    \ChangedAt{v3.21}{\Package{tocbasic}}%
    Diese Option gilt seit \KOMAScript~3.20 als überholt. Der numerische Wert
    der Ebene des Verzeichniseintrags\index{Verzeichnis>Eintrag} ist nun
    stattdessen als Eigenschaft in Abhängigkeit des Verzeichniseintragsstils
    von Option \Option{tocentrystyle} anzugeben. Nichtsdestotrotz haben alle
    Stile die Eigenschaft \PValue{level} und damit Option
    \Option{tocentrylevel}. Die Eigenschaft wird von \Macro{DeclareNewTOC} mit
    1 voreingestellt. Diese Voreinstellung kann durch explizite Angabe von
    \OptionVName{tocentrylevel}{Wert} leicht mit einem anderen \PName{Wert}
    überschrieben werden.}%
  \entry{\OptionVName{listname}{Verzeichnistitel}}{%
    Jedes Verzeichnis hat eine Überschrift, die durch diese Option bestimmt
    werden kann. Ist die Option nicht angegeben, so wird als Verzeichnistitel
    »\texttt{List of \PName{Mehrzahl des Eintragstyps}}« (siehe Option
    \Option{types}) verwendet, wobei das erste Zeichen der \PName{Mehrzahl des
      Eintragstyps} in einen Großbuchstaben gewandelt wird. Es wird auch ein
    Makro \Macro{list\PName{Eintragstyp}name} mit diesem Wert definiert, der
    jederzeit geändert werden kann. Dieses Makro wird jedoch nur definiert,
    wenn es nicht bereits definiert ist oder zusätzlich Option
    \Option{forcenames} gesetzt ist.}%
  \entry{\OptionVName{name}{Eintragsname}}{%
    Sowohl als optionaler Präfix für die Einträge im Verzeichnis als auch für
    die Beschriftung in einer Gleitumgebung (siehe Option \Option{float}) oder
    einer nicht gleitenden Umgebung (siehe Option \Option{nonfloat}) wird ein
    Name benötigt. Ohne diese Option wird als \PName{Eintragsname} der
    \PName{Eintragstyp} (siehe Option \Option{type}) verwendet, bei dem das
    erste Zeichen in einen Großbuchstaben gewandelt wird. Es wird auch ein
    Makro \Macro{\PName{Eintragstyp}name} mit diesem Wert definiert, der
    jederzeit geändert werden kann. Dieses Makro wird jedoch nur definiert,
    wenn es nicht bereits definiert ist oder zusätzlich Option
    \Option{forcenames} gesetzt ist.}%
  \entry{\Option{nonfloat}}{%
    Es wird nicht nur ein neuer Verzeichnistyp definiert, sondern auch eine
    nicht gleitende Umgebungen \Environment{\PName{Eintragstyp}-} (siehe
    Option \Option{type}), die ähnlich wie eine Gleitumgebung verwendet werden
    kann, jedoch nicht gleitet und auch nicht die Grenzen der Umgebung, in der
    sie verwendet wird, durchbricht.}%
  \entry{\OptionVName{owner}{Besitzer}}{%
    Jedes neue Verzeichnis hat bei \Package{tocbasic} einen Besitzer (siehe
    \autoref{sec:tocbasic.basics}). Dieser kann hier angegeben werden. Ist
    kein \PName{Besitzer} angegeben, so wird als \PName{Besitzer} die
    Kategorie »\PValue{float}« verwendet, die auch von den \KOMAScript-Klassen
    für das Abbildungs- und das Tabellenverzeichnis verwendet wird.}%
  \entry{\OptionVName{setup}{Liste von Eigenschaften}}{%
    \ChangedAt{v3.25}{\Package{tocbasic}}%
    Die \PName{Liste von Eigenschaften} wird via
    \DescRef{\LabelBase.cmd.setuptoc} gesetzt. Es wird darauf hingewiesen,
    dass für die Angabe mehrerer durch Komma getrennter Eigenschaften die
    \PName{Liste von Eigenschaften} in geschweifte Klammern gesetzt werden
    muss.%
  }%
  \entry{\OptionVName{tocentrystyle}{Eintragsstil}}{%
    \ChangedAt{v3.20}{\Package{tocbasic}}%
    \PName{Eintragsstil} gibt den Stil an, den Einträge in das entsprechende
    Verzeichnis haben sollen. Der Name der Eintragsebene wird dabei über
    Option \Option{type} bestimmt. Zusätzlich zu den Optionen dieser Tabelle
    können auch alle Eigenschaften des Stils als Optionen angegeben werden,
    indem die Namen der Eigenschaften mit dem Präfix \PValue{tocentry} ergänzt
    werden. So kann der numerische Wert der Ebene beispielsweise als
    \Option{tocentrylevel} angegeben werden. Näheres zu den Stilen ist
    \autoref{sec:tocbasic.tocstyle} ab \autopageref{sec:tocbasic.tocstyle} zu
    entnehmen.%
  }%
  \entry{\OptionVName{tocentry\PName{Stiloption}}{Wert}}{%
    \ChangedAt{v3.20}{\Package{tocbasic}}%
    Weitere Optionen in Abhängigkeit vom via \Option{tocentrystyle} gewählten
    \PName{Eintragsstil}. Siehe dazu \autoref{sec:tocbasic.tocstyle} ab
    \autopageref{sec:tocbasic.tocstyle}. Für die von \Package{tocbasic}
    vordefinierten Verzeichniseintragsstile finden sich die als
    \PName{Stiloption} verwendbaren Attribute in
    \autoref{tab:tocbasic.tocstyle.attributes}, ab
    \autopageref{tab:tocbasic.tocstyle.attributes}.%
  }%
  \entry{\OptionVName{type}{Eintragstyp}}{%
    \PName{Eintragstyp} gibt den Typ der Einträge in das entsprechende
    Verzeichnis an. Der Typ wird auch als Basisname für verschiedene Makros
    und gegebenenfalls Umgebungen und Zähler verwendet. Er sollte daher nur
    aus Buchstaben bestehen. Wird diese Option nicht verwendet, so wird für
    \PName{Eintragstyp} die \PName{Dateierweiterung} aus dem obligatorischen
    Argument verwendet.}%
  \entry{\OptionVName{types}{Mehrzahl des Eintragstyps}}{%
    An verschiedenen Stellen wird auch die Mehrzahlform des Eintragstyps
    verwendet, beispielsweise um eine Anweisung \Macro{listof\PName{Mehrzahl
        des Eintragstyps}} zu definieren. Wird diese Option nicht verwendet,
    so wird als \PName{Mehrzahl des Eintragstyps} der Wert von \Option{type}
    mit angehängtem »s« verwendet.}%
  \entry{\OptionVName{unset}{Liste von Eigenschaften}}{%
    \ChangedAt{v3.25}{\Package{tocbasic}}%
    Die \PName{Liste von Eigenschaften} wird via
    \DescRef{\LabelBase.cmd.unsettoc} aufgehoben. Es wird darauf hingewiesen,
    dass für die Angabe mehrerer durch Komma getrennter Eigenschaften die
    \PName{Liste von Eigenschaften} in geschweifte Klammern gesetzt werden
    muss.%
  }%
\end{desclist}

\begin{Example}
  Das Beispiel aus \autoref{sec:tocbasic.example} kann mit Hilfe der neuen
  Anweisung deutlich verkürzt werden:
\begin{lstcode}
  \DeclareNewTOC[%
    type=remarkbox,%
    types=remarkboxes,%
    float,% Gleitumgebungen sollen definiert werden.
    floattype=4,%
    name=Merksatz,%
    listname={Verzeichnis der Merks\"atze}%
  ]{lor}
  \setuptoc{lor}{chapteratlist}
\end{lstcode}
  Neben den Umgebungen \Environment{remarkbox} und \Environment{remarkbox*}
  sind damit auch der Zähler \Counter{remarkbox}, die zur Ausgabe gehörenden
  Anweisungen \Macro{theremarkbox}, \Macro{remarkboxname} und
  \Macro{remarkboxformat}, die für das Verzeichnis benötigten
  \Macro{listremarkboxname} und \Macro{listofremarkboxes} %
  \iffalse % Umbruchkorrektur
  sowie einige intern verwendete Anweisungen mit Bezug auf %
  \else%
  und einige interne Anweisunen für %
  \fi%
  die Dateiendung \File{lor} definiert. Soll der Gleitumgebungstyp dem Paket
  überlassen werden, so kann Option \Option{floattype} \iffalse im Beispiel \fi
  entfallen. Wird zusätzlich die Option \Option{nonfloat} angegeben, wird
  außerdem eine nicht gleitende Umgebung \Environment{remarkbox-} definiert,
  in der ebenfalls \DescRef{maincls.cmd.caption}\IndexCmd{caption} verwendet
  werden kann.

  Zum besseren Verständnis zeigt \autoref{tab:tocbasic.comparison} eine
  Gegenüberstellung der Anweisungen und Umgebungen für die neu erstellte
  Beispielumgebung \Environment{remarkbox} mit den entsprechenden Befehlen und
  Umgebungen für Abbildungen.%
  \begin{table}
    \centering
    \caption{Gegenüberstellung von Beispielumgebung \Environment{remarkbox}
      und Umgebung \Environment{figure}}
    \label{tab:tocbasic.comparison}
    \begin{tabularx}{\textwidth}{l@{\hskip\tabcolsep}l@{\hskip\tabcolsep}>{\raggedright}p{7em}@{\hskip\tabcolsep}X}
      \toprule
      \begin{tabular}[t]{@{}l@{}}
        Umgebung\\
        \Environment{remarkbox}
      \end{tabular}
      & \begin{tabular}[t]{@{}l@{}}
          Umgebung\\
          \Environment{figure}
        \end{tabular}
      & Optionen von \Macro{DeclareNewTOC} & Kurzbeschreibung \\[1ex]
      \midrule
      \Environment{remarkbox} & \Environment{figure} 
      & \Option{type}, \Option{float} 
      & Gleitumgebung des jeweiligen Typs.\\[1ex]
      \Environment{remarkbox*} & \Environment{figure*} 
      & \Option{type}, \Option{float} 
      & spaltenübergreifende Gleitumgebung des jeweiligen Typs \\[1ex]
      \Counter{remarkbox} & \Counter{figure} 
      & \Option{type}, \Option{float} 
      & Zähler, der von \DescRef{maincls.cmd.caption} verwendet wird \\[1ex]
      \Macro{theremarkbox} & \Macro{thefigure} 
      & \Option{type}, \Option{float} 
      & Anweisung zur Ausgabe des jeweiligen Zählers \\[1ex]
      \Macro{remarkboxformat} & \DescRef{maincls.cmd.figureformat}
      & \Option{type}, \Option{float}
      & Anweisung zur Formatierung des jeweiligen Zählers in der Ausgabe von
        \DescRef{maincls.cmd.caption}\\[1ex]
      \Macro{remarkboxname} & \Macro{figurename}
      & \Option{type}, \Option{float}, \Option{name}
      & Name, der im Label von \DescRef{maincls.cmd.caption} verwendet
        wird\\[1ex]
      \Macro{listofremarkboxes} & \DescRef{maincls.cmd.listoffigures}
      & \Option{types}, \Option{float}
      & Anweisung zur Ausgabe des jeweiligen Verzeichnisses \\[1ex]
      \Macro{listremarkboxname} & \Macro{listfigurename}
      & \Option{type}, \Option{float}, \Option{listname}
      & Überschrift des jeweiligen Verzeichnisses \\[1ex]
      \Macro{fps@remarkbox} & \Macro{fps@figure}
      & \Option{type}, \Option{float}, \Option{floattype}
      & numerischer Gleitumgebungstyp zwecks
        Reihen\-fol\-gen\-erhalts\\[1ex]
      \File{lor} & \File{lof}
      &
      & Dateiendung der Hilfsdatei für das jeweilige Verzeichnis \\
      \bottomrule
    \end{tabularx}
  \end{table}

  Und hier nun eine mögliche Verwendung der Umgebung:
\begin{lstcode}
  \begin{remarkbox}
    \centering
    Gleiches sollte immer auf gleiche Weise und
    mit gleichem Aussehen gesetzt werden.
    \caption{Erster Hauptsatz der Typografie}
    \label{rem:typo1}
  \end{remarkbox}
\end{lstcode}
  Ein Ausschnitt aus einer Beispielseite mit dieser Umgebung könnte dann so
  aussehen:
  \begin{center}\footnotesize
    \begin{tabular}
      {|!{\hspace{.1\linewidth}}p{.55\linewidth}!{\hspace{.1\linewidth}}|}
      \\
      \centering
      Gleiches sollte immer auf gleiche Weise und
      mit gleichem Aussehen gesetzt werden.\\[\abovecaptionskip]
      {%
        \usekomafont{caption}\footnotesize
        \begin{tabularx}{\hsize}[t]{@{}l@{ }X@{}}
          \usekomafont{captionlabel}{Merksatz 1:} &
          Erster Hauptsatz der Typografie%\\
        \end{tabularx}%
      }%
      \\
    \end{tabular}%
  \end{center}%
\end{Example}%

Benutzer von
\Package{hyperref}\IndexPackage{hyperref}\important{\Package{hyperref}}
sollten Option \Option{listname} übrigens immer angeben. Anderenfalls kommt es
in der Regel zu einer Fehlermeldung, weil \Package{hyperref} nicht mit dem
\Macro{MakeUppercase}\IndexCmd{MakeUppercase} im Namen des Verzeichnisses
zurecht kommt, das benötigt wird, um den ersten Buchstaben des Wertes von
\Option{types} in Großbuchstaben zu wandeln.%
\EndIndexGroup%

\section{Obsolete Befehle}
\seclabel{obsolete}

Frühere Versionen von \Package{tocbasic} verfügten über Befehle,
die aufgrund von Äußerungen von Mitgliedern des \LaTeX-Project-Teams umbenannt
wurden. Diese veralteten Befehle sollten nicht mehr verwendet werden.

\LoadNonFree{tocbasic}{0}%
% 
\EndIndexGroup

\endinput

%%% Local Variables:
%%% mode: latex
%%% mode: flyspell
%%% coding: utf-8
%%% ispell-local-dictionary: "de_DE"
%%% TeX-master: "../guide"
%%% End:

%  LocalWords:  Verzeichnisdatei Klonquelle Absatzabstandes Absatzabstand
%  LocalWords:  Verzeichniseinträge Verzeichniseintrag Eintragsstile
%  LocalWords:  Inhaltsverzeichniseinträge Standardklassen Nummernbreite
%  LocalWords:  Standardeigenschaft Eintragsebene Optionenliste expandierbar
%  LocalWords:  Standardverzeichniseintragsstil Verzeichniseigenschaften
%  LocalWords:  Verzeichniseintragsstil Verzeichniseintragsstils Eintragsstil
%  LocalWords:  Kapitelüberschrift Gliederungsebene expandierbares
%  LocalWords:  Eintragsnummer Initialisierungscode Aktivierungswerte
%  LocalWords:  Deaktivierungswert Verzeichnistitel Eintragstyp Eintragstyps
%  LocalWords:  Verzeichniseintrags Säumniswert Überschriftenanweisung
%  LocalWords:  Eintragstext Dateierweiterung Dateierweiterungen Paketautoren
%  LocalWords:  Verzeichnisbefehle Verzeichnisdaten Schrifteinstellungen
%  LocalWords:  Eintragsebenen Platzierungsoptionen Kapiteleinträge
