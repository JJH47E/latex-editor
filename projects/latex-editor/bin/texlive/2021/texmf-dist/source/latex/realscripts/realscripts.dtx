% \iffalse
%
%<*internal>
\begingroup
%</internal>
%<*batchfile>
\input docstrip.tex
\keepsilent
\preamble
  _____________________________
  The REALSCRIPTS package
  (C) 2010--2016 Will Robertson
  License information appended

\endpreamble
\postamble

Copyright (C) 2010--2016 by Will Robertson <will.robertson@latex-project.org>

Distributable under the LaTeX Project Public License,
version 1.3c or higher (your choice). The latest version of
this license is at: http://www.latex-project.org/lppl.txt

This work is "maintained" (as per LPPL maintenance status)
by Will Robertson.

This work consists of the file  realscripts.dtx
          and the derived files realscripts.sty,
                                realscripts.ins, and
                                realscripts.pdf.

\endpostamble
\askforoverwritefalse
\generate{\file{realscripts.sty}{\from{realscripts.dtx}{package}}}
%</batchfile>
%<batchfile>\endbatchfile
%<*internal>
\generate{\file{realscripts.ins}{\from{realscripts.dtx}{batchfile}}}
\def\tmpa{plain}
\ifx\tmpa\fmtname\endgroup\expandafter\bye\fi
\endgroup
%</internal>
%
%    \begin{macrocode}
%<*driver>
\ProvidesFile{realscripts.dtx}
%</driver>
%<package>\ProvidesPackage{realscripts}
%<*package>
  [2016/02/13 v0.3d Access OpenType subscripts and superscripts]
%</package>
%    \end{macrocode}
%
%<*driver>
\documentclass{ltxdoc}
\usepackage{color,graphicx,metalogo,hologo,fontspec,realscripts}
\linespread{1.05}      % A bit more space between lines
\frenchspacing         % Remove extra space after punctuation
\definecolor{niceblue}{rgb}{0.1,0.2,1}
\def\theCodelineNo{\textcolor{niceblue}{\sffamily\tiny\arabic{CodelineNo}}}
\newcommand*\pkg[1]{\texttt{#1}}
\begin{document}
  \DocInput{\jobname.dtx}
\end{document}
%</driver>
%
% \fi
%
% \errorcontextlines=999
% \makeatletter
%
% \GetFileInfo{\jobname.sty}
%
% \title{The \pkg{\jobname} package}
% \author{Will Robertson}
% \date{\filedate \qquad \fileversion}
%
% \maketitle
%
% \section{Introduction}
%
% OpenType fonts provide the possiblity of using specially-drawn glyphs for
% subscript and superscript text. \LaTeX\ by default simply uses a smaller
% font size, which is acceptable if the font has optical sizes. Most fonts
% don't, however.
%
% If you are using the \pkg{fontspec} package\footnote{The \pkg{fontspec} package requires \XeLaTeX\ or \hologo{LuaLaTeX}.} to select OpenType fonts
% (or other sorts of fonts with the necessary font features), then loading
% this package will redefine the \cmd\textsuperscript\ and
% \cmd\textsubscript\ commands that take advantage of the OpenType font
% features.
%
% The \cs{textsubsuperscript}\marg{sub}\marg{super} command is
% defined for typesetting superscripts above subscripts using the same
% approach. For symmetry, \cs{textsupersubscript}\marg{super}\marg{sub}
% is also provided. Options are available to specify the alignment of the sub- and super-scripts; see below.
%
% \section{Technical details}
%
% The old commands are defined in starred form (e.g., \cs{textsubscript*}) for cases where the new commands are not appropriate.
% The functionality of the starred and non-starred commands can
% also be accessed using the macros
% \cmd\realsubscript,
% \cmd\realsuperscript,
% \cmd\fakesubscript, and
% \cmd\fakesuperscript, in case another package (or you)
% redefine the original \cmd\text\dots\ commands.
%
% This package will also patch the default \LaTeX\ footnote mechanism to
% use \cs{textsuperscript} automatically. You may change the font used
% to typeset the footnote numbers (by default it is the \cs{normalfont})
% by redefining \cs{footnotemarkfont}; e.g.:
% {\par\qquad\verb|\renewcommand\footnotemarkfont|\marg{font switch}\par}
% \noindent and users of \pkg{KOMA-Script} may use instead
% {\par\qquad\verb|\setkomafont{footnotelabel}|\marg{font switch}\par}
% \noindent where \meta{font switch} is a command such as \cs{sffamily} or a \pkg{fontspec} font defined with \cs{newfontfamily}.
%
% Beware of other packages, however, that change the footnote mechanism. (Usually by redefining \cs{@makefnmark}.) I can often work around or work with such packages so \pkg{realscripts} cooperates gracefully with them; please report any conflicts to me.
%
% \section{Examples}
%
% Here is an example using the `{\fontspec{Skia} Skia}' font of Mac\,OS\,X: (surrounded by `A' and `Z' for visual context)
%
% \begin{quotation}\color{niceblue}
%	\fontspec{Skia}
%	  |\textsuperscript{…}     | A \textsuperscript{abcdefghijklmnopqrstuvwxyz1234567890} Z\par
%	  |\textsubscript{…}       | A \textsubscript{abcdefghijklmnopqrstuvwxyz1234567890} Z
% \end{quotation}
% The original definitions are available in starred verions of the commands:
% (compare this example to that above to see why using these features is often desirable)
% \begin{quotation}\color{niceblue}
%	\fontspec{Skia}
%	  |\textsuperscript*{…}    | A \textsuperscript*{abcdefghijklmnopqrstuvwxyz1234567890} Z\par
%	  |\textsubscript*{…}      | A \textsubscript*{abcdefghijklmnopqrstuvwxyz1234567890} Z
% \end{quotation}
% When the font features are not available the commands will fall back on the standard technique of scaling down the text font:
% (this is Mac\,OS\,X's `{\fontspec{Didot} Didot}')
% \begin{quotation}\color{niceblue}
%	\fontspec{Didot}
%	  |\textsuperscript{…}     | A \textsuperscript{abcdefghijklmnopqrstuvwxyz1234567890} Z\par
%	  |\textsubscript{…}       | A \textsubscript{abcdefghijklmnopqrstuvwxyz1234567890} Z
% \end{quotation}
% But beware fonts that contain the necessary font features but lack the full repertoire of glyphs: (this is `{\fontspec{Adobe Jenson Pro}Adobe Jenson Pro}')
% \begin{quotation}\color{niceblue}
%	\fontspec{Adobe Jenson Pro}
%	  |\textsuperscript{…}     | A {\textsuperscript{abcdefghijklmnopqrstuvwxyz1234567890}} Z\par
%	  |\textsubscript{…}       | A {\textsubscript{abcdefghijklmnopqrstuvwxyz1234567890}} Z
% \end{quotation}
% Finally, the `combined' commands. Note the dimension available
% to add some extra space between the sub- and super-scripts, and
% the optional argument for specifying alignment.
% \begin{quotation}\color{niceblue}
%	\fontspec{Skia}
%	  |\textsubsuperscript{…}{…}       |
%    A {\textsubsuperscript{123456}{789}} Z\par\smallskip
%	  |\setlength\subsupersep{2pt}     | \subsupersep=2pt
%    A {\textsubsuperscript{123456}{789}} Z\par\smallskip
%	  |\textsubsuperscript[c]{…}{…}    |
%    A {\textsubsuperscript[c]{123}{456789}} Z\par\smallskip
%	  |\textsubsuperscript[r]{…}{…}    |
%    A {\textsubsuperscript[r]{123456}{789}} Z\par\smallskip
% \end{quotation}
%
% \newpage
% \part{Implementation of \textsf{\jobname}}
%\iffalse
%<*package>
%\fi
% This is the package implementation.
% If you're only interested in the footnote redefinition,
% skip ahead to Section~\ref{sec:fn} on page~\pageref{sec:fn}.
%
%    \begin{macrocode}
\RequirePackage{fontspec}
\ExplSyntaxOn
%    \end{macrocode}
%
% \begin{macro}{\fakesubscript}
% \begin{macro}{\fakesuperscript}
% The old (`fake') methods.
% Because \cs{textsubscript} is not defined in \LaTeXe,
% we either define it from scratch along with \cs{textsuperscript}
% (for consistency).
%    \begin{macrocode}
\DeclareDocumentCommand \fakesubscript {m} {
  \@textsubscript{\selectfont#1}
}
\DeclareDocumentCommand \fakesuperscript {m} {
  \@textsuperscript{\selectfont#1}
}
%    \end{macrocode}
% \end{macro}
% \end{macro}
%
% \begin{macro}{\textsubscript}
% \begin{macro}{\textsubscript*}
% \begin{macro}{\textsuperscript}
% \begin{macro}{\textsuperscript*}
% These commands are either defined to create fake or real sub-/super-scripts if they are starred or not, respectively.
%    \begin{macrocode}
\RenewDocumentCommand \textsubscript {s} {
    \IfBooleanTF #1 \fakesubscript \realsubscript
}
\RenewDocumentCommand \textsuperscript {s} {
    \IfBooleanTF #1 \fakesuperscript \realsuperscript
}
%    \end{macrocode}
% \end{macro}
% \end{macro}
% \end{macro}
% \end{macro}
%
% \begin{macro}{\realsubscript}
% The new subscript command to use rich font features if possible.
%    \begin{macrocode}
\DeclareDocumentCommand \realsubscript {m} {
  \fontspec_if_fontspec_font:TF {
%    \end{macrocode}
% For OpenType fonts, the subscript feature (|subs|) is used, but if that doesn't
% exist then the scientific inferior feature (|sinf|) is used on the assumption
% that something's better than nothing.
%    \begin{macrocode}
    \fontspec_if_opentype:TF
    { \fontspec_if_feature:nTF {+subs}
        { {\addfontfeature{VerticalPosition=Inferior}#1} }
        { \fontspec_if_feature:nTF {+sinf}
            { {\addfontfeature{VerticalPosition=ScientificInferior}#1} }
            { \fakesubscript{#1} }
        }
    }
%    \end{macrocode}
% ATSUI fonts:
%    \begin{macrocode}
    { \fontspec_if_aat_feature:nnTF {10} {2}
        { {\addfontfeature{VerticalPosition=Inferior}#1} }
        { \fakesubscript{#1} }
    }
  }
%    \end{macrocode}
% Non-fontspec fonts:
%    \begin{macrocode}
  { \fakesubscript{#1} }
}
%    \end{macrocode}
% \end{macro}
%
% \begin{macro}{\realsuperscript}
% The new superscript command to use rich font features if possible.
%    \begin{macrocode}
\DeclareDocumentCommand \realsuperscript {m} {
  \fontspec_if_fontspec_font:TF
  {
%    \end{macrocode}
% OpenType fonts:
%    \begin{macrocode}
    \fontspec_if_opentype:TF
    { \fontspec_if_feature:nTF {+sups}
      { {\addfontfeature{VerticalPosition=Superior}#1} }
      { \fakesuperscript{#1} }
    }
%    \end{macrocode}
% ATSUI fonts:
%    \begin{macrocode}
    { \fontspec_if_aat_feature:nnTF {10} {1}
      { {\addfontfeature{VerticalPosition=Superior}#1} }
      { \fakesuperscript{#1} }
    }
  }
%    \end{macrocode}
% Non-fontspec fonts:
%    \begin{macrocode}
  { \fakesuperscript{#1} }
}
%    \end{macrocode}
% \end{macro}
%
% \section{Patching footnotes}
% \label{sec:fn}
%
% \begin{macro}{\@makefnmark}
% This is the command used to typeset the `footnote mark'.
% Feel free to redefine it as necessary for your own purposes.
%    \begin{macrocode}
\cs_set:Npn \@makefnmark {
  \mbox{\footnotemarkfont\textsuperscript{\@thefnmark}}
}
%    \end{macrocode}
% We define a \cs{footnotemarkfont} that is used to style
% the number used for the footnote, which by default is
% simply \cs{normalfont} (following \LaTeXe's default).
% If \pkg{KOMA-Script} is
% being used, we use their hook for the footnotemark font instead.
%    \begin{macrocode}
\cs_if_exist:NTF \ftntm@font
{
  \cs_new:Npn \footnotemarkfont {\ftntm@font}
}
{
  \cs_new:Npn \footnotemarkfont {\normalfont}
}
%    \end{macrocode}
% \end{macro}
%
% \section{sub+super}
%
% \begin{macro}{\textsubsuperscript}
% \begin{macro}{\textsupersubscript}
% Although the implementation below would be fine outside of this package too,
% no point writing yet another small package.
%    \begin{macrocode}
\DeclareDocumentCommand \textsubsuperscript {s O{l} mm} {
  \leavevmode
  \group_begin:
  \IfBooleanTF #1
  {
    \hbox_set:Nn \l_tmpa_box {\textsubscript*{#3}}
    \hbox_set:Nn \l_tmpb_box {\textsuperscript*{#4}}
  }
  {
    \hbox_set:Nn \l_tmpa_box {\textsubscript{#3}}
    \hbox_set:Nn \l_tmpb_box {\textsuperscript{#4}}
  }
  \hbox_set:Nn \l_tmpa_box
    { \box_move_down:nn \subsupersep {\box_use:N \l_tmpa_box} }
  \hbox_set:Nn \l_tmpb_box
    { \box_move_up:nn \subsupersep {\box_use:N \l_tmpb_box} }
  \str_case:nnF {#2}
  {
    {l}{\use_i:nnn}
    {c}{\use_ii:nnn}
    {r}{\use_iii:nnn}
  }
  {
    \PackageWarning{realscripts}{
      Unknown~alignment~option~`#2'. \MessageBreak
      One~ of~ `l',~ `c',~ `r',~ only
    }
    \use_i:nnn
  }
%    \end{macrocode}
% Left aligned:
%    \begin{macrocode}
  {
    \hbox_overlap_right:n { \box_use:N \l_tmpa_box }
    \hbox_overlap_right:n { \box_use:N \l_tmpb_box }
    \skip_horizontal:n {
      \dim_max:nn {\box_wd:N \l_tmpa_box} {\box_wd:N \l_tmpb_box}
    }
  }
%    \end{macrocode}
% Center aligned: (for completeness)
%    \begin{macrocode}
  {
    \dim_compare:nTF { \box_wd:N \l_tmpa_box > \box_wd:N \l_tmpb_box }
    {
      \skip_horizontal:n {
        0.5\box_wd:N \l_tmpa_box-0.5\box_wd:N \l_tmpb_box
      }
      \box_use:N \l_tmpb_box
      \skip_horizontal:n {
        -0.5\box_wd:N \l_tmpa_box-0.5\box_wd:N \l_tmpb_box
      }
      \box_use:N \l_tmpa_box
    }
    {
      \skip_horizontal:n {
        0.5\box_wd:N \l_tmpb_box-0.5\box_wd:N \l_tmpa_box
      }
      \box_use:N \l_tmpa_box
      \skip_horizontal:n {
        -0.5\box_wd:N \l_tmpb_box-0.5\box_wd:N \l_tmpa_box
      }
      \box_use:N \l_tmpb_box
    }
  }
%    \end{macrocode}
% Right aligned:
%    \begin{macrocode}
  {
    \skip_horizontal:n {
      \dim_max:nn {\box_wd:N \l_tmpa_box} {\box_wd:N \l_tmpb_box}
    }
    \hbox_overlap_left:n { \box_use:N \l_tmpa_box }
    \hbox_overlap_left:n { \box_use:N \l_tmpb_box }
  }
  \group_end:
}
\DeclareDocumentCommand \textsupersubscript {s O{l} mm} {
  \IfBooleanTF #1
   { \textsubsuperscript * [#2] {#4} {#3} }
   { \textsubsuperscript   [#2] {#4} {#3} }
}
%    \end{macrocode}
%
%    \begin{macrocode}
\dim_new:N \subsupersep
%    \end{macrocode}
% \end{macro}
% \end{macro}
%
% Fin.
%
%\iffalse
%</package>
%\fi
%
% \Finale
%
% \typeout{*************************************************************}
% \typeout{*}
% \typeout{* To finish the installation you have to move the following}
% \typeout{* file into a directory searched by XeTeX:}
% \typeout{*}
% \typeout{* \space\space\space realscripts.sty}
% \typeout{*}
% \typeout{*************************************************************}
%
\endinput

