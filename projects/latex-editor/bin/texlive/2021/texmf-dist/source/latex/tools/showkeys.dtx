% \iffalse meta-comment
%
% Copyright (C) 1993-2021
%
% The LaTeX Project and any individual authors listed elsewhere
% in this file.
%
% This file is part of the Standard LaTeX `Tools Bundle'.
% -------------------------------------------------------
%
% It may be distributed and/or modified under the
% conditions of the LaTeX Project Public License, either version 1.3c
% of this license or (at your option) any later version.
% The latest version of this license is in
%    https://www.latex-project.org/lppl.txt
% and version 1.3c or later is part of all distributions of LaTeX
% version 2005/12/01 or later.
%
% The list of all files belonging to the LaTeX `Tools Bundle' is
% given in the file `manifest.txt'.
%
% \fi
% \iffalse
%% File: showkeys.dtx Copyright (C) 1992-1997 David Carlisle
%% File: showkeys.dtx Copyright (C) 2006-2019 David Carlisle, LaTeX Project
%
%<*dtx>
          \ProvidesFile{showkeys.dtx}
%</dtx>
%<package>\NeedsTeXFormat{LaTeX2e}
%<package>\ProvidesPackage{showkeys}
%<driver> \ProvidesFile{showkeys.drv}
% \fi
%         \ProvidesFile{showkeys.dtx}
          [2014/10/28 v3.17 Show cite and label keys (DPC, MH)]
%
% \iffalse
%<*driver>
\documentclass{ltxdoc}
\usepackage
%     [notcite,notref,color]
            {showkeys}
\setlength{\belowcaptionskip}{\baselineskip}
\setlength{\abovecaptionskip}{0pt}
\begin{document}
\DocInput{showkeys.dtx}
\end{document}
%</driver>
% \fi
%
% \GetFileInfo{showkeys.dtx}
% \title{The \textsf{showkeys} package\thanks{This file
%         has version number \fileversion, last
%         revised \filedate.}}
% \author{David Carlisle \and Morten H\o gholm}
% \date{\filedate}
% \MaintainedByLaTeXTeam{tools}
% \maketitle
%
% %%%%%%%%%%%%%%%%%%%%%%%%%%%%%%%%%%%%%%%%%%%%%%%%%%%%%%%%%%%%%%%%%%%%
%
%
%
% \changes{v1.01}{1992/08/25}{Initial version}
% \changes{v1.02}{1994/01/05}
%         {Fix incorrect initialisation (FGBDA19@CC1.KULEUVEN.AC.BE)}
% \changes{v2.00}{1994/01/31}{new version for LaTeX2e}
% \changes{v2.01}{1994/06/30}{Fix \cs{ProvidesPackage} usage.}
% \changes{v3.00}{1994/09/07}
%         {Support harvard, varioref and natbib packages,}
% \changes{v3.02}{1995/03/17}
%         {Support new AMS files}
%
% \section{Introduction}\label{sec:intro}
%
% |showkeys.sty| modifies the |\label|, |\ref|, |\pageref|, |\cite|, and
% |\bibitem| commands so that the `internal' key is printed. The package
% tries hard to position these labels so that the formatting of the rest
% of the document is unchanged. |\label| and |\bibitem| cause the
% key to appear in a box either in the margin, or in a \TeX\ box of zero
% width, which may possibly over-print other text. The |\ref|,
% |\pageref| and |\cite| commands print their arguments in small type,
% raised just above the line, like this: \ref{sec:intro}.
% This package works with the \textsf{fleqn} option, the packages
% in the AMS-\LaTeX\ collection, and the \textsf{varioref},
% \textsf{natbib} and \textsf{harvard} packages.
%
% \changes{v2.00}{1992/01/31}
%         {leqno or ams* may now be loaded after showkeys}
%
% \section{Package Options}\label{options}
% Some people have commented that the printing of the |\ref| and
% |\cite| keys is less useful than the printing of the |\label| keys
% and so \textsf{showkeys} now supports two options that can be given
% in the |\usepackage| command:
% \begin{description}
% \item[notref] to stop the redefinition of |\ref| and |\pageref|,
% and related commands from the \textsf{varioref} package.
% \item[notcite] to stop the redefinition of |\cite| and related
% commands from the \textsf{harvard} and \textsf{natbib} packages.
% \end{description}
% So if the package is loaded with |\usepackage[notref]{showkeys}|
% then |\ref| will have its standard definition, but |\label| will
% print its key argument (usually in the margin).
%
% If you find the printed keys distracting, but don't want to use the
% above options to stop them altogether you may use:
% \begin{description}
% \item[color] Print the keys in a distinguishing colour. The default
%  value is a light grey.
% \end{description}
% The colours may be changed by redefining the following two colours
% after the package is loaded.
% |refkey| (also used for |\cite|) and
% |labelkey| (also used for |\bibitem|).
% The defaults are:
%\begin{verbatim}
%  \definecolor{refkey}{gray}{.75}
%  \definecolor{labelkey}{gray}{.75}
%\end{verbatim}
%
% If this option is used the \textsf{color} package will be loaded.
%
% The package accepts two further options.
% \begin{description}
% \item[final] to suporess the action of this package, for `final'
% versions.
% \item[draft] the normal behaviour of this package.
% \end{description}
% Clearly there is not much point in entering the |final| option
% directly in the |\usepackage| command, as just not loading this
% package would have the same effect, and execute more quickly,
% however the |final| option may be useful as it may be used once in
% the |documentclass| command to affect any number of packages that
% may be loaded. The |draft| option does not do anything, but is there
% to honour an informal convention that packages have these options in
% pairs.
%
% You can also control the appearance of the typeset label with the
% command |\showkeyslabelformat|, which takes one argument. The default is
% \begin{verbatim}
% \providecommand*\showkeyslabelformat[1]{%
%   \fbox{\normalfont\small\ttfamily#1}}
% \end{verbatim}
% The command is called inside a group so you can put in local
% modifications of |\fboxsep|, for instance, without them leaking to
% the rest of the document.
%
%
% \section{More Examples}\label{examples}
% The only other similar package that I could find in the macro index,
% \cite{DMJ:mi}, was |showlabels.sty|, \cite{GN:sl}. After the first
% draft of this package was written, I found \cite{anon:sk} on my local
% installation! I think the current package is more robust than
% \cite{anon:sk}, but I thought that |showkeys| was rather a good name,
% so I have stolen it for this file.
%
% \begin{enumerate}
% \item \label{e^1}This has |\label| immediately after |\item|.
% \item This has the |\label| at the end.\label{e^2}
% \end{enumerate}
%
% \[
% \mbox{A minipage :- }\left\{
% \begin{minipage}{3in}
% Within environments like this |minipage|, we cannot use
% |\marginpar|\footnotemark,
% so the appearance is slightly different. Here is that |enumerate|
% environment again:
%
% \begin{enumerate}
% \item \label{m&e^1}This has |\label| immediately after |\item|.
% \item This has the |\label| at the end.\label{m&e^2}
% \end{enumerate}
% \end{minipage}
% \right.
% \]
%
% Displayed math (without |equation| counter).
% \[0=0\label{disp}\]
%
% Some text referring to the maths on page~\pageref{disp},  and the
% item~\ref{e^1}.\footnotetext{Actually \texttt{\string\marginpar} is
% not used at all in this package now.}
%
% If |showkeys| thinks that the current environment is going to produce
% an ``equation number'', then it does not show the label where the
% |\label| command occurs, but tries to put it in the margin, as shown
% with equation~\ref{eq:xx}.
% The package `knows' about the standard |equation| and |eqnarray|
% environments, and also all the numbered alignment environments offered
% by the AMS\LaTeX\ package, |amsmath|.
%
% ^^A (|amstex.sty| must be loaded before |showkeys.sty|
% ^^A for this to work.)
% \changes{v2.00}{1992/01/31}
%         {leqno or ams* may now be loaded after showkeys}
%
% \begin{equation}
% 1=1\label{eq:xx}
% \end{equation}
%
% \begin{eqnarray}
% 2&=&2\label{eqnar:a}\\
% 3&=&3\nonumber\\
% 4&=&4\label{eqnar:b}
% \end{eqnarray}
%
%
% \begin{figure}[ht]
% Within a |figure| environment, the |\label| must not come  before the
% |\caption| command. If you place |\label| inside the argument of
% |\caption| the label will be shown like this:
%
% \caption{Within the caption argument.\label{cap:a}}
%
% If you place |\label| immediately after the |\caption| command it will
% be shown like this:
%
% \caption{Immediately after the caption argument.}\label{cap:b}
%
% If you place the |\label| command at some random point after the
% |\caption| command, it may be shown like:
%
% \caption{In vertical mode not immediately after a box.}
% \vspace{2pt}
%
% \label{cap:c}
% \end{figure}
%
%
% \begin{thebibliography}{9}
%
% \bibitem{GN:sl}
% Gil Neiger, \emph{showlabels.sty},
% Undated package, similar to this one, but shows labels
% inline, affecting the formatting of the document.
%
% \bibitem{anon:sk}
% Anonymous, \emph{showkeys.sty},
% Package, dated 14 May 1988. Very similar to this one,
% also uses |\marginpar| in outer vertical mode.
%
% \bibitem{DMJ:mi}
% David M. Jones, \emph{\TeX\ Macro Index},
% A catalogue of \TeX\ macros, including \LaTeX\ packages,
% available from all good \TeX\ archives.
%
% \end{thebibliography}
%
% \StopEventually{}
%
% \section{The Macros}
%
%    \begin{macrocode}
%<*package>
%    \end{macrocode}
%
%
% First we handle the options. Normally all related commands are
% defined to show their `keys'. But since v3.03 one can specify:
%
% \texttt{notref} to stop the redefinition of |\ref| (and |\pageref|,
% and related commands from \textsf{varioref} package),
%
% \texttt{notcite} to stop the redefinition of |\cite| and related
% commands from the \textsf{harvard} and \textsf{natbib} packages.
%
% \changes{v3.03}{1995/04/25}
%      {Add option handling.}
%    \begin{macrocode}
\DeclareOption{notref}{\let\SK@ref\@empty}
\DeclareOption{notcite}{\let\SK@cite\@empty}
%    \end{macrocode}
%
% \begin{macro}{\SK@refcolor}
% \begin{macro}{\SK@labelcolor}
% Colour commands. Normally no-op.
%    \begin{macrocode}
\let\SK@refcolor\relax
\let\SK@labelcolor\relax
%    \end{macrocode}
% \end{macro}
% \end{macro}
%
% \changes{v3.11}{1996/11/01}
%         {Colour support added, inspired by tools/2297}
% |color| option loads the \textsf{color} package and defines the
% colours. Delayed to the end of the  package as package loading not
% allowed in this option section.
%    \begin{macrocode}
\DeclareOption{color}{\AtEndOfPackage{%
  \RequirePackage{color}%
  \definecolor{refkey}{gray}{.75}%
  \definecolor{labelkey}{gray}{.75}%
  \def\SK@refcolor{\color{refkey}}%
  \def\SK@labelcolor{\color{labelkey}}}}
%    \end{macrocode}
%
% \changes{v3.04}{1995/10/30}
%      {final and draft options handling.}
% \changes{v3.15}{2007/08/07}
%      {Fix \cs{showkeyslabelformat} for final option PR/3918.}
% Allow |final| to be specified in the document class options
% to suppress the loading of this package.
%    \begin{macrocode}
\DeclareOption{final}{%
  \providecommand*\showkeyslabelformat[1]{}%
  \endinput}
\DeclareOption{draft}{}
%    \end{macrocode}
%
%    \begin{macrocode}
\ProcessOptions
%    \end{macrocode}
%
% \changes{v2.00}{1992/01/31}
%         {\cmd{reset@font} is now standard}
%
% \begin{macro}{\SK@label}
% \begin{macro}{\SK@bibitem}
% \begin{macro}{\SK@lbibitem}
% The saved original definitions
%    \begin{macrocode}
\let\SK@label\label
\let\SK@bibitem\@bibitem
\let\SK@lbibitem\@lbibitem
%    \end{macrocode}
% \end{macro}
% \end{macro}
% \end{macro}
%
%
% \begin{macro}{\label}
% \changes{v3.09}{1996/08/30}
%      {Add extra group so brace hack works. Donald Arseneau tools/2147}
% The new definition, print the argument, and then do the old
% definition.
%    \begin{macrocode}
\def\label#1{%
  \@bsphack
  \SK@\SK@@label{#1}%
  \begingroup
    \SK@label{#1}%
  \endgroup
  \@esphack}
%    \end{macrocode}
% \end{macro}
%
% \begin{macro}{\@bibitem}
% \begin{macro}{\@lbibitem}
% \changes{v3.02}{1995/03/17}
%         {New label code.}
% For |\bibitem|, position the \textsf{showkeys} code as for a standard
% list with |\item| and |\label|.
%    \begin{macrocode}
\def\@bibitem#1{%
  \SK@bibitem{#1}\SK@\SK@@label{#1}\ignorespaces}
%    \end{macrocode}
%
%    \begin{macrocode}
\def\@lbibitem[#1]#2{%
  \SK@lbibitem[{#1}]{#2}\SK@\SK@@label{#2}\ignorespaces}
%    \end{macrocode}
% \end{macro}
% \end{macro}
%
% \begin{macro}{\SK@}
% \changes{v3.07}{1996/05/17}
%      {use \cs{protected@edef} for tools/2147}
% Grab hold of |#2| via |\meaning| so characters like |&| and
% |^| do not cause problems later, and pass the result on to the command
% |#1|.
%    \begin{macrocode}
\def\SK@#1#2{%
  \protected@edef\@tempa{#2}%
  \expandafter#1\meaning\@tempa\SK@}
%    \end{macrocode}
% \end{macro}
%
%
%
% \begin{macro}{\showkeyslabelformat}
% \changes{v3.13}{2006/01/09}{Added command}
%    \begin{macrocode}
\providecommand*\showkeyslabelformat[1]{%
  \fbox{\normalfont\small\ttfamily#1}}
%    \end{macrocode}
% \end{macro}
% \begin{macro}{\SK@@label}
% \changes{v3.13}{2006/01/09}{Do not hardwire label format}
% Strip off the initial segment of the |\meaning| output, and then put
% the rest either in a |\marginpar| or in a box of size 0pt,
% hopefully not disturbing the surrounding text.
%    \begin{macrocode}
\def\SK@@label#1>#2\SK@{%
%    \end{macrocode}
% Need to work globally as in some cases like alignments, and |fleqn|,
% the counter will be printed in a different group to the |\label|
% command.
%    \begin{macrocode}
  \gdef\SK@lab{\smash{\SK@labelcolor\showkeyslabelformat{#2}}}%
  \ifvmode
    \if@inlabel
%    \end{macrocode}
% \changes{v3.02}{1995/03/17}
%         {New code for `in label' case.}
% If the |\label| is straight after |\item| (|\bibitem| is handled by
% this case as well) then the item label has not been added to the page
% yet. It is hanging around in the box |\@labels| waiting for the
% paragraph to start. So just need to attach the label to this box.
%    \begin{macrocode}
      \global\setbox\@labels\hbox{%
        \llap{\SK@lab\SK@lab@relax
              \kern\@totalleftmargin\kern\marginparsep}%
        \box\@labels}%
%    \end{macrocode}
%
%    \begin{macrocode}
    \else
%    \end{macrocode}
% \changes{v3.10}{1996/09/06}
%      {Save prevdepth and restore later}
% If we insert a box into the main vertical list, do not want to
% change |\prevdepth| as that would affect vertical spacing in the
% document. (The box itself should not cause any difference in break
% points as there is a node there anyway coming from the |\write| to
% the aux file.
%    \begin{macrocode}
      \dimen@\prevdepth
      \nointerlineskip
%    \end{macrocode}
% The inner vertical mode cases are mainly designed to do the right
% thing with float captions, but seem to work OK in other cases as well.
%    \begin{macrocode}
      \ifinner
        \skip@\lastskip\unskip
%    \end{macrocode}
% In inner vertical mode, attach the label to the right of the
% immediately preceding box, if it is a box before the current point.
% Otherwise just put it in a box of zero dimensions, with no interline
% skip. (This may slightly move the surrounding text (but perhaps not
% now that |\prevdepth| is restored.)
% \changes{v3.00}{1994/09/07}
%      {Back up over a previous skip because of the new
%       \cs{belowcaptionskip}}
% \changes{v3.04}{1995/10/30}
%      {\cs{advance} added, to total two successive skips.}
% \changes{v3.04}{1995/10/30}
%      {\cs{nointerlineskip} called before \cs{ifvoid} test, not just
%      void case}
% \changes{v3.04}{1995/10/30}
%      {\cs{marginparskip} added in inner vmode case}
%    \begin{macrocode}
        \advance\skip@\lastskip\unskip
        \setbox\z@\lastbox
%    \end{macrocode}
% \changes{v3.10}{1996/09/06}
%      {Inner vertical mode case, put it in the margin.}
% \changes{v3.14}{2006/09/25}
% {The label put in the margin shouldn't make box 255 wider (PR/3884).}
%    \begin{macrocode}
        \ifvoid\z@
          \llap{\SK@lab\SK@lab@relax\kern\marginparsep}%
        \else
          \hbox{\box\z@\rlap{\kern\marginparsep\SK@labx}}%
        \fi
        \vskip\skip@
      \else
%    \end{macrocode}
% In outer vertical mode, previously used a |\vadjust| at the start of
% the next  paragraph (and before that used |\marginpar|). These
% methods sometimes cause extra space, eg if paragraph starts with a
% math display, so now just insert the box directly, taking care not
% to change |\prevdepth|.
% \changes{v3.02}{1995/03/17}
%         {Use \cs{vadjust} instead of \cs{marginpar}}
% \changes{v3.10}{1996/09/06}
%      {Insert the box directly}
%    \begin{macrocode}
        \llap{\SK@lab\SK@lab@relax\kern\marginparsep}%
      \fi
%    \end{macrocode}
% Restore |\prevdepth|.
%    \begin{macrocode}
      \prevdepth\dimen@
%    \end{macrocode}
%
%    \begin{macrocode}
    \fi
  \else
%    \end{macrocode}
% If we are in an numbered equation-style environment, do nothing as the
% code to print the number will also print the label, otherwise just
% stick the label at the current point, in a box of zero dimensions.
% \changes{v3.02}{1995/03/17}
%         {Add \cs{ifmmode} test}
%    \begin{macrocode}
    \csname SK@\@currenvir\endcsname
    \ifSK@equation\else
      \ifmmode
        \SK@labx
      \else
%    \end{macrocode}
% Inner horizontal mode. Not much we can do, just stick it here.
% \changes{v3.03}{1995/04/25}
%         {Fix inner horizontal mode case (broken in 3.02)}
%    \begin{macrocode}
        \ifinner
          \rlap\SK@lab
      \else
%    \end{macrocode}
% In outer horizontal mode use |\vadjust| to get to the margin.
% \changes{v3.02}{1995/03/17}
%         {Use \cs{vadjust} in horizontal mode}
%    \begin{macrocode}
          \vadjust{\llap{\SK@lab\kern\marginparsep}}%
        \fi
        \SK@lab@relax
      \fi
    \fi
  \fi}
%    \end{macrocode}
% \end{macro}
%
% \begin{macro}{\tagform@}
% \changes{v2.00}{1992/01/31}
%         {Defer tests to begin document}
% \changes{v3.02}{1995/03/17}
%         {Support new AMS files}
% \begin{macro}{\@eqnnum}
% \changes{v3.13}{2006/01/09}
%         {Change to also work correctly for \texttt{leqno} option.}
% \begin{macro}{\maketag@@@}
% \changes{v3.13}{2006/01/09}
%         {Support for \cs{tag*} tools/3693.}
% Firstly we grab |\@eqnnum|.
%    \begin{macrocode}
\AtBeginDocument{%
  \let\SK@eqnnum\@eqnnum
%    \end{macrocode}
% Then check for \textsf{amsmath} where we grab the internal commands
% |\tagform@| and |\maketag@@@|. Redefine them and redefine |\@eqnnum|
% as well.
%    \begin{macrocode}
  \@ifpackageloaded{amsmath}{%
    \let\SK@tagform@\tagform@
    \let\SK@maketag@@@\maketag@@@
    \iftagsleft@
      \def\tagform@#1{%
        \ifx\df@label\@empty
          \SK@lab@relax
        \else
          \expandafter\SK@@label\meaning\df@label\SK@
        \fi
        \llap{\SK@lab\kern\marginparsep}%
        \SK@lab@relax\SK@tagform@{#1}}%
      \def\maketag@@@#1{%
        \ifx\df@label\@empty
          \SK@lab@relax
        \else
          \expandafter\SK@@label\meaning\df@label\SK@
        \fi
        \llap{\SK@lab\kern\marginparsep}\SK@lab@relax
        \SK@maketag@@@{#1}%
      }%
      \def\@eqnnum{%
        \llap{\SK@lab\kern\displaywidth\kern\marginparsep}%
        \SK@lab@relax\SK@eqnnum}%
    \else
%    \end{macrocode}
% Almost the same for tags on the right, except we use |\rlap| and
% typeset it after the tag.
%    \begin{macrocode}
      \def\tagform@#1{%
        \ifx\df@label\@empty
          \SK@lab@relax
        \else
          \expandafter\SK@@label\meaning\df@label\SK@
        \fi
%    \end{macrocode}
% \changes{v3.08}{1996/07/10}{Missing percent added. /2215}
%    \begin{macrocode}
        \SK@tagform@{#1}%
        \rlap{\kern\marginparsep\SK@lab}\SK@lab@relax}%
      \def\maketag@@@#1{%
        \ifx\df@label\@empty
          \SK@lab@relax
        \else
          \expandafter\SK@@label\meaning\df@label\SK@
        \fi
        \SK@maketag@@@{#1}%
        \rlap{\kern\marginparsep\SK@lab}\SK@lab@relax
      }%
      \def\@eqnnum{\SK@eqnnum\rlap{\kern\marginparsep\SK@lab}%
        \SK@lab@relax}%
    \fi
  }%
%    \end{macrocode}
% If \textsf{amsmath} wasn't loaded we check explicitly if the
% \texttt{leqno} option was used in |\documentclass| and redefine
% accordingly.
%    \begin{macrocode}
  {%
    \@ifundefined{ver@leqno.clo}{%
      \def\@eqnnum{\SK@eqnnum\rlap{\kern\marginparsep\SK@lab}%
        \SK@lab@relax}%
    }{%
      \def\@eqnnum{%
        \llap{\SK@lab\kern\displaywidth\kern\marginparsep}%
        \SK@lab@relax\SK@eqnnum}%
    }%
  }%
}
%    \end{macrocode}
% \end{macro}
% \end{macro}
% \end{macro}
%
% \begin{macro}{\SK@labx}
% Print the label, and then globally reset the print command to
% |\relax|.
%    \begin{macrocode}
\def\SK@labx{\rlap\SK@lab\global\let\SK@lab\relax}
%    \end{macrocode}
% \end{macro}
%
% \begin{macro}{\SK@lab@relax}
% Clear the label.
% \changes{v1.02}{1994/01/05}{Execute to initialise correctly}
%    \begin{macrocode}
\def\SK@lab@relax{\global\let\SK@lab\relax}\SK@lab@relax
%    \end{macrocode}
% \end{macro}
%
% \begin{macro}{\SK@equation}
% \begin{macro}{\SK@eqnarray}
% The following environments print an equation number, so |\label|
% should not print its argument at the point where it appears.
% Note this will fail to show the label if you are in an |eqnarray|
% environment, and use |\label| together with |\nonumber| This might
% just about make sense if you are going to use |\pageref|, but that is
% too bad\ldots
%    \begin{macrocode}
\newif\ifSK@equation
\let\SK@equation\SK@equationtrue
\let\SK@eqnarray\SK@equationtrue
%    \end{macrocode}
% \end{macro}
% \end{macro}
%
% \begin{macro}{\eqnarray}
% \changes{v3.09}{1996/08/30}
%      {Fix eqnarray AMS incompatibility. tools/2252}
% When the AMS packages are loaded |showkeys| assumes environments
% work `The AMS way' However |eqnarray| (unlike |equation|) is not
% redefined, so here we need to remove some of the AMS hacks.
%    \begin{macrocode}
\toks@\expandafter{\eqnarray}
\edef\eqnarray{\let\noexpand\tagform@\noexpand\SK@tagform@\the\toks@}
%    \end{macrocode}
% \end{macro}
%
% \begin{macro}{\SK@align}
% \begin{macro}{\SK@alignat}
% \begin{macro}{\SK@xalignat}
% \begin{macro}{\SK@xxalignat}
% \begin{macro}{\SK@gather}
% \begin{macro}{\SK@multline}
% \begin{macro}{\SK@flalign}
% \changes{v3.02}{1995/03/17}
%         {Add \cs{SK@flalign}}
% The AMS environments
%    \begin{macrocode}
\let\SK@align\SK@equationtrue
\let\SK@alignat\SK@equationtrue
\let\SK@xalignat\SK@equationtrue
\let\SK@xxalignat\SK@equationtrue
\let\SK@gather\SK@equationtrue
\let\SK@multline\SK@equationtrue
\let\SK@flalign\SK@equationtrue
%    \end{macrocode}
% \end{macro}
% \end{macro}
% \end{macro}
% \end{macro}
% \end{macro}
% \end{macro}
% \end{macro}
%
% \begin{macro}{\SK@align*}
% \begin{macro}{\SK@alignat*}
% \begin{macro}{\SK@flalign*}
% \begin{macro}{\SK@gather*}
% \begin{macro}{\SK@multline*}
% \begin{macro}{\SK@equation*}
% \changes{v3.13}{2006/01/09}
%         {Add the starred AMS environments for tools/3697.}
% Starred versions of the AMS environments.
%    \begin{macrocode}
\expandafter\let\csname SK@align*\endcsname\SK@equationtrue
\expandafter\let\csname SK@alignat*\endcsname\SK@equationtrue
\expandafter\let\csname SK@flalign*\endcsname\SK@equationtrue
\expandafter\let\csname SK@gather*\endcsname\SK@equationtrue
\expandafter\let\csname SK@multline*\endcsname\SK@equationtrue
\expandafter\let\csname SK@equation*\endcsname\SK@equationtrue
%    \end{macrocode}
% \end{macro}
% \end{macro}
% \end{macro}
% \end{macro}
% \end{macro}
% \end{macro}
%
%  \begin{macro}{\SK@def}
% \changes{v3.05}{1995/11/09}
%      {Macro added}
% This macro redefines a command |#1|. The new definition can make use
% of the old definition as |\SK@|\emph{old name}. If |#1| is really a
% |\protect|'ed command with the real definition in a `\emph{space}'
% command then the `space' version is used as the old definition.
% Need to test this for each command as some package may have changed
% the status of a command to being `protected'.
% The new definition is made as if with |\DeclareRobustCommand|, but
% with |\def| syntax for the argument specification.
%    \begin{macrocode}
\def\SK@def#1{%
  \edef\@tempa{\expandafter\@gobble\string#1}%
  \@ifundefined{\@tempa\space}%
    {\expandafter\let\csname SK@\@tempa\endcsname#1}%
    {\expandafter\let\csname SK@\@tempa\expandafter\endcsname
                         \csname\@tempa\space\endcsname}%
  \expandafter\def\expandafter#1\expandafter{%
        \expandafter\protect\csname\@tempa\space\endcsname}%
  \expandafter\def\csname\@tempa\space\endcsname}
%    \end{macrocode}
%  \end{macro}
%
% The next section redefines |\ref| and |\pageref| (unless the
% \texttt{notref} option was given).
%    \begin{macrocode}
\ifx\SK@ref\@empty
%    \end{macrocode}
% Even if \texttt{notref} option is used, need to fudge the
% \textsf{varioref} commands as they use |\label| internally.
% \changes{v3.04}{1995/10/30}
%      {improve varioref support in notref option case, for
%      tools/1744}
% \changes{v3.13}{2006/01/09}{Updated varioref support, for latex/3373}
% \changes{v3.15}{2006/06/19}
%         {Prevent loss of brackets}
%    \begin{macrocode}
\AtBeginDocument{%
  \@ifpackageloaded{varioref}{%
    \SK@def\@@vpageref#1[#2]#3{{%
      \let\label\SK@label
      \SK@@@vpageref{#1}[{#2}]{#3}}}%
    \def\vr@f#1{%
      \leavevmode\unskip\vref@space
      \ref{#1}%
      {\let\label\SK@label
      \vpageref[\unskip]{#1}}}%
  }{}}
\else
%    \end{macrocode}
%
% \begin{macro}{\ref}
% \begin{macro}{\pageref}
% Save the redefinition to |\begin{document}| so that this package can
% work with packages that redefine |\cite|. Tested with harvard and
% natbib packages. Also add code at this point to support varioref.
% \changes{v3.00}{1994/09/07}
%      {Delay \cs{ref} redefinition.}
% \changes{v3.03}{1995/04/25}
%      {Make redefinition conditional on notref option}
% \changes{v3.15}{2006/06/19}
%         {Prevent loss of brackets}
%    \begin{macrocode}
\AtBeginDocument{%
  \SK@def\ref#1{\SK@\SK@@ref{#1}\SK@ref{#1}}%
  \SK@def\pageref#1{\SK@\SK@@ref{#1}\SK@pageref{#1}}%
%    \end{macrocode}
% varioref support.
%    \begin{macrocode}
  \@ifpackageloaded{varioref}{%
    \SK@def\@@vpageref#1[#2]#3{{%
      \let\label\SK@label\let\ref\SK@ref\let\pageref\SK@pageref
      \leavevmode\unskip\SK@\SK@@ref{#3}\SK@@@vpageref{#1}[{#2}]{#3}}}%
    \def\vr@f#1{%
      \leavevmode\unskip\vref@space
      \ref{#1}%
      \let\label\SK@label\let\ref\SK@ref\let\pageref\SK@pageref
      \vpageref[\unskip]{#1}}%
  }{}}
%    \end{macrocode}
%
%    \begin{macrocode}
\fi
%    \end{macrocode}
% \end{macro}
% \end{macro}
%
% Now redefine |\cite| unless \texttt{notcite} option given.
%    \begin{macrocode}
\ifx\SK@cite\@empty
%    \end{macrocode}
% \changes{v3.06}{1995/11/22}
%         {Fix \cs{harvarditem} support}
%    \begin{macrocode}
\AtBeginDocument{%
  \ifx\HAR@checkdef\@undefined\else
      \expandafter\let\expandafter
         \SK@HAR@bi\csname\string\harvarditem\endcsname
      \expandafter\def\csname\string\harvarditem\endcsname[#1]#2#3#4{%
        \SK@HAR@bi[{#1}]{#2}{#3}{#4}\SK@\SK@@label{#4}}%
  \fi}
\else
%    \end{macrocode}
%
% \begin{macro}{\cite}
% \changes{v3.00}{1994/09/07}
%      {Delay \cs{cite} redefinition.}
% \changes{v3.03}{1995/04/25}
%      {Make redefinition conditional on notcite option}
%    \begin{macrocode}
\AtBeginDocument{%
  \ifx\HAR@checkdef\@undefined
%    \end{macrocode}
% Standard (non-harvard) support, including extra cite commands from
% \textsf{natbib} and \textsf{cite}.
% \changes{v3.01}{1994/09/09}
%         {Add \cs{citefullauthor}}
% \changes{v3.12}{1997/06/12}
%         {Support cite package. tools/2490}
% \changes{v3.17}{2014/04/24}
%         {redefine \cs{@citex} rather than \cs{cite} for tools/4162}
%
% If \textsf{cite} or \textsf{overcite} is being used, redefine |\citen|
% rather than |\cite| so as not to spoil the space and punctuation
% calculations done by those packages.
%    \begin{macrocode}
    \ifx\citen\@undefined
      \SK@def\@citex[#1]#2{\SK@citex[{#1}]{#2}}%
    \else
      \SK@def\citen#1{\SK@\SK@@ref{#1}\SK@citen{#1}}%
    \fi
    \SK@def\citeauthor#1{\SK@\SK@@ref{#1}\SK@citeauthor{#1}}%
    \SK@def\citefullauthor#1{\SK@\SK@@ref{#1}\SK@citefullauthor{#1}}%
    \SK@def\citeyear#1{\SK@\SK@@ref{#1}\SK@citeyear{#1}}%
  \else
%    \end{macrocode}
% In the \textsf{harvard} style do \emph{not} redefine individual cite
% commands. Just redefine one internal command that is used in all the
% citation forms.
%    \begin{macrocode}
    \SK@def\HAR@checkdef#1#2{%
      \expandafter\SK@\expandafter\SK@@ref\expandafter{#1}%
      \SK@HAR@checkdef{#1}{#2}}%
      \expandafter\let\expandafter
         \SK@HAR@bi\csname\string\harvarditem\endcsname
%    \end{macrocode}
% \changes{v3.06}{1995/11/22}
%         {Fix \cs{harvarditem} support}
%    \begin{macrocode}
      \expandafter\def\csname\string\harvarditem\endcsname[#1]#2#3#4{%
        \SK@HAR@bi[{#1}]{#2}{#3}{#4}\SK@\SK@@label{#4}}%
  \fi}
%    \end{macrocode}
%
% \changes{v3.17}{2014/04/24}
%         {Change name, redefine \cs{@citex} rather than \cs{cite} for tools/4162}
%    \begin{macrocode}
\def\SK@citex[#1]#2{%
  \SK@\SK@@ref{#2}\SK@@citex[{#1}]{#2}}
%    \end{macrocode}
%
%    \begin{macrocode}
\fi
%    \end{macrocode}
% \end{macro}
%
% \begin{macro}{\SK@@ref}
%   This is much simpler than the printing of the label, as we know
%   that we can be in horizontal mode.
%   Note extra group for colour safety.
% \changes{v3.14}{2006/09/25}
%                {Add fix for pdf\TeX.}
% \changes{v3.16}{2011/11/24}
%                {revert change made at v3.14 because of tools/4173}
%    \begin{macrocode}
\def\SK@@ref#1>#2\SK@{%
  \leavevmode\vbox to\z@{{%
    \vss
    \SK@refcolor
    \rlap{\vrule\raise .75em%
       \hbox{\underbar{\normalfont\footnotesize\ttfamily#2}}}}}}
%    \end{macrocode}
% \end{macro}
%
%    \begin{macrocode}
%</package>
%    \end{macrocode}
%
% \Finale
%
