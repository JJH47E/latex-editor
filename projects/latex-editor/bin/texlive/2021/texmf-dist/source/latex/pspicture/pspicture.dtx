%
% \iffalse
%%
%% Source File `pspicture.dtx'.
%% Copyright (C) 1992 1999 David Carlisle
%% This file may be distributed under the terms of the LPPL.
%% See 00readme.txt for details.
%%
%
%<*dtx>
          \ProvidesFile{pspicture.dtx}
%</dtx>
%<package>\NeedsTeXFormat{LaTeX2e}
%<package>\ProvidesPackage{pspicture}
%<driver>\ProvidesFile{pspicture.drv}
%<*package,driver>
% \fi
%         \ProvidesFile{pspicture.dtx}
          [1999/04/11/ v2.02 Picture mode via PS specials (DPC)]
%
% \iffalse
%</package,driver>
%<*driver>
\documentclass{ltxdoc}
\begin{document}
\DocInput{pspicture.dtx}
\end{document}
%</driver>
% \fi
%
% \GetFileInfo{pspicture.dtx}
% \CheckSum{110}
%
% \changes{1}   {1989/05/10}{(Not made widely available)}
% \changes{2}   {1992/01/02}{support for dvips added}
% \changes{2.01}{1992/06/16}{First public release}
% \changes{2.02}{1999/04/11}{Fix for `new' docstrip, and LPPL}
%
% %%%%%%%%%%%%%%%%%%%%%%%%%%%%%%%%%%%%%%%%%%%%%%%%%%%%%%%%%%%%%%%%%%%%%%
%
% \title{The \textsf{pspicture} package\thanks{This file
%        has version number \fileversion, last
%        revised \filedate.}}
%
% \author{D. P. Carlisle}
% \date{16 June 1992}
% \maketitle
%
%
% \section{Introduction}
% {\tt pspicture} is a re-implementation, and extension of, \LaTeX's
% {\tt picture} environment, using PostScript |\special|'s. This has
% several advantages, mainly that lines of arbitrary slope and thickness
% may be specified, and there is no limit on the size of the circles
% that may be drawn\footnote{^^A
% There is a certain amount of overlap between this style option and the
% widely available {\tt eepic} option. However when I wrote the first
% version of this, in 1989, I was not aware of {\tt eepic}, and {\tt
% pspicture} has been reasonably popular in Manchester, even
% though {\tt epic} and {\tt eepic} have been installed.}.
%
% One disadvantage is that the picture can no longer be previewed on a
% |dvi| previewer, such as |xdvi|. To help with this problem, a
% companion style option, {\tt texpicture}, may be used while developing
% a document, this uses the standard picture commands as much as
% possible, and silently omits any picture objects that can not be drawn
% with standard \LaTeX.
%
% A second disadvantage, is that a {\tt dvi} file produced with {\tt
% pspicture} will contain embedded |\special| commands. These commands
% will only work with the driver program for which they were intended.
% This makes the {\tt dvi} file less portable. {\tt pspicture} will by
% default use |\special|'s set up for Rokicki's {\tt dvips} program,
% although it should be easy to modify the code to work with other
% PostScript drivers. A {\sc DocStrip} option for a version of {\tt
% dvi2ps} is included with this distribution.
%
% \subsection{Commands Available}{\parskip=\baselineskip\parindent=0pt
% \DescribeMacro{\circle}\DescribeMacro{\circle*}
%   Use as described in the \LaTeX\ book but with no maximum diameter.
%   The thickness of the circle is altered by the |\linethickness|
%   command. The size of the circle produced by |\circle*| is not
%   affected by |\linethickness|, so it is not the same as `filling in'
%   the circle drawn by |\circle|.
%
% \DescribeMacro{\oval}
%   Use as described in the \LaTeX\ book, but as there is no maximum
%   diameter for the circular arcs, the oval (in the absence of the
%   optional |[tr]| etc) always consists of two semi-circular arcs
%   joined by a pair of parallel lines. To obtain a `rectangle with
%   rounded corners' the oval command has a second optional argument
%   (given first !).\\
%   |\oval[20](100,200)[t]|\\
%   Produces the top half of an oval with quarter circles of radius
%   20*unitlength.
%   If unitlength = 1pt then this is equivalent to the standard oval
%   command. In general |\oval[R](x,y)| uses circular arcs of radius
%   $\min(R,x/2,y/2)$.
%
% \DescribeMacro{\line}\DescribeMacro{\vector}
%  Use as described in the \LaTeX\ book but with no restriction on the
%  available slopes. The thickness of a sloping line is altered by the
%  |\linethickness| command.
%
% \DescribeMacro{\Line}\DescribeMacro{\Vector}
%  New forms of the line and vector commands.\\
%   |\put(x1,y1){\Line(x2,y2)}|\\
%   produces a line from (x1,y1) to (x1+x2,y1+y2) and similarly for
%   |\Vector|.
%
% \DescribeMacro{\Curve}
%  Like |\Line| except that it produce a curve!\\
%   |\put(x1,y1){\Curve(x2,y2){m}}|\\
%   produces a curve from (x1,y1) to (x1+x2,y1+y2). the amount of
%   curvature is controlled by m but try $1$ or $-1$ first. m does not
%   have to be an integer. Negative numbers curve the opposite way to
%   positive numbers.
%
% \DescribeMacro{\thinlines}\DescribeMacro{\thicklines}%
% \DescribeMacro{\linethickness}
%  These commands alter the thickness of {\bf all} lines including
%  slanted lines and circular arcs.\vspace{\baselineskip}
%
% \DescribeMacro{\arrowlength}
%  A new command which specifies the size of the arrowhead drawn by the
%  |\vector| and |\Vector| commands. Like |\linethickness| it does not
%  get multiplied by |\unitlength|. At present the arrowhead is
%  triangular. If a head with curved sides more like the standard
%  \LaTeX\ head is required the definition of |!A| in pspicture.ps
%  should be altered.
%
% Other {\tt picture} mode commands are not altered by this style, and
% so may be used, just as described in the \LaTeX\ book. These include:
% |\put|, |\multiput|, |\makebox|, |\framebox|, |dashbox| and
% |\shortstack|.
%
% }\newpage\section{Examples}
%
% \noindent A picture built with \LaTeX's line and circle fonts.
%
% \noindent\begin{picture}(300,130)
% \put(0,0){\line(1,2){50}}
% \put(50,50){\vector(1,1){30}}
% \put(50,50){\circle{30}}
% \put(100,50){\circle*{10}}
% \put(150,50){\oval(50,20)[t]}
% \put(150,100){\oval(30,30)[bl]}
%
% ^^A LaTeX does not do these so well
% \put(250,60){\circle{60}}
% \put(250,60){\circle{50}}
%
% \end{picture}
%
% \bigskip
% \def\eatmodule#1>{}

%\eatmodule
%<*x>
 \ifcat a\noexpand @\else
 \MakePercentComment
 \def\ProvidesPackage#1[#2]{}
 \def\NeedsTeXFormat#1{}
 \makeatletter         ^^A This is NOT the usual way to access this
 %%
%% This is file `pspicture.sty',
%% generated with the docstrip utility.
%%
%% The original source files were:
%%
%% pspicture.dtx  (with options: `package,dvips')
%% 
%%
%% Source File `pspicture.dtx'.
%% Copyright (C) 1992 1999 David Carlisle
%% This file may be distributed under the terms of the LPPL.
%% See 00readme.txt for details.
%%
\NeedsTeXFormat{LaTeX2e}
\ProvidesPackage{pspicture}
          [1999/04/11/ v2.02 Picture mode via PS specials (DPC)]

       \def\PS@header#1{\special{header=#1}}
\PS@header{pspicture.ps}
       \def\PS@special#1{\special{"#1}}
{\catcode`t=12\catcode`p=12\gdef\noPT#1pt{#1}}
\def\strippt#1{\expandafter\noPT\the#1\space}
\def\@circle#1{%
  \@tempdimb #1\unitlength
  \PS@special{%
       \strippt\@wholewidth
       \strippt\@tempdimb
       !C}}
\def\@dot#1{%
  \@tempdimb #1\unitlength
  \PS@special{%
       \strippt\@tempdimb
       !D}}
\def\line(#1,#2)#3{%
  \@linelen=#3\unitlength
  \PS@special{%
       \strippt\@wholewidth
       #1
       #2
       \strippt\@linelen
       !L}}
\def\vector(#1,#2)#3{%
  \@linelen=#3\unitlength
  \PS@special{%
       \strippt\@arrowlength
       \strippt\@wholewidth
       #1
       #2
       \strippt\@linelen
       !V}}
\def\oval{%
  \@ifnextchar[%
   {\@ov@l}%
   {\count@=\maxdimen \divide\count@ by \unitlength \@ov@l[\count@]}}
\def\@ov@l[#1](#2,#3){%
  \@ifnextchar[{\@oval[#1](#2,#3)}{\@oval[#1](#2,#3)[]}}%
\def\@oval[#1](#2,#3)[#4]{\begingroup
  \@tempdimb #1\unitlength
  \@ovxx #2\unitlength
  \@ovyy #3\unitlength
  \def\r{\def\TL{0 }\def\BL{0 }}%
  \def\l{\def\TR{0 }\def\BR{0 }}%
  \def\t{\def\BL{0 }\def\BR{0 }}%
  \def\b{\def\TL{0 }\def\TR{0 }}%
  \def\TL{1 }\def\BL{1 }\def\TR{1 }\def\BR{1 }%
  \@tfor\@tempa :=#4\do{\csname\@tempa\endcsname}%
  \PS@special{%
       \BR\BL\TR\TL
       \strippt\@wholewidth
       \strippt\@tempdimb
       \strippt\@ovxx
       \strippt\@ovyy
       !O}%
  \endgroup}
\def\Line(#1,#2){%
  \@ovxx #1\unitlength
  \@ovyy #2\unitlength
  \PS@special{%
       \strippt\@wholewidth
       \strippt\@ovxx
       \strippt\@ovyy
       !L2}}
\def\Curve(#1,#2)#3{%
  \@ovxx #1\unitlength
  \@ovyy #2\unitlength
  \PS@special{%
       \strippt\@wholewidth
       \strippt\@ovxx
       \strippt\@ovyy
       #3
       !C2}}
\def\Vector(#1,#2){%
  \@ovxx #1\unitlength
  \@ovyy #2\unitlength
  \PS@special{%
       \strippt\@arrowlength
       \strippt\@wholewidth
       \strippt\@ovxx
       \strippt\@ovyy
       !V2}}
\newdimen\@arrowlength
\def\arrowlength#1{\@arrowlength #1}
\arrowlength{8pt}
     \endinput



\endinput
%%
%% End of file `pspicture.sty'.
 ^^A style. It is intended as a style option:
 \makeatother          ^^A \documentstyle[a4,pspicture]{article}
 \MakePercentIgnore
 \fi
%\eatmodule
%</x>
%
% \noindent The same picture built with PostScript |\special|'s.
%
% \noindent\begin{picture}(300,130)
% \put(0,0){\line(1,2){50}}
% \put(50,50){\vector(1,1){30}}
% \put(50,50){\circle{30}}
% \put(100,50){\circle*{10}}
% \put(150,50){\oval(50,20)[t]}
% \put(150,100){\oval(30,30)[bl]}
%
% ^^A LaTeX does not do these so well
% \put(250,60){\circle{60}}
% \put(250,60){\circle{50}}
%
% \end{picture}
%
% \bigskip
% \noindent Some extra features not available using the standard picture
% mode.
%
% \noindent\begin{picture}(300,150)
% \put(250,60){\vector(-8,1){50}} ^^A Vector of any slope.
% \put(200,0){\Line(100,100)}     ^^A Line command.
% \put(210,100){\Vector(90,10)}   ^^A Vector command.
% \put(70,90){\oval(100,70)}      ^^A Control over the radius of
% \put(70,90){\oval[10](100,70)}  ^^A   `corners' of an oval.
% \linethickness{3pt}
% \put(20,20){\circle{20}}        ^^A Circles of any thickness and size.
% \put(20,20){\line(10,1){30}}    ^^A Sloping lines of any thickness.
%
% \end{picture}
%
% \StopEventually{}
%
% \section{pspicture.sty}
%    \begin{macrocode}
%<*package>
%    \end{macrocode}
%
% First we set up the code that is specific to the driver program that
% is being used. If the driver can incorporate a header file, define
% |\PS@header| appropriately, |\PS@special| should expand to the format
% for inline PostScript code. The driver should protect this code with a
% (g)save (g)restore pair. dvips is treated specially so that it will be
% the default driver if this file is used without being stripped. If you
% find definitions of these macros which work for the driver you use,
% email me, and I will include them in the next release.
% \begin{macro}{\PS@header}
% Code to include the PostScript header.
%    \begin{macrocode}
%<*dvips>
       \def\PS@header#1{\special{header=#1}}
%</dvips>
%<dvi2ps>\def\PS@header#1{\typeout{Print with the option -i #1}}
\PS@header{pspicture.ps}
%    \end{macrocode}
% \end{macro}
%
% \begin{macro}{\PS@special}
% The format of the |\special| command for inline PostScript.
%    \begin{macrocode}
%<*dvips>
       \def\PS@special#1{\special{"#1}}
%</dvips>
%<dvi2ps>\def\PS@special#1{\special{pstext="#1"}}
%    \end{macrocode}
% \end{macro}
%
% \begin{macro}{\strippt}
% Strip the final `pt' off the string returned by |\the|.
%    \begin{macrocode}
{\catcode`t=12\catcode`p=12\gdef\noPT#1pt{#1}}
\def\strippt#1{\expandafter\noPT\the#1\space}
%    \end{macrocode}
% \end{macro}
%
% \begin{macro}{\@circle}
% Internal name for |\circle|.
%    \begin{macrocode}
\def\@circle#1{%
  \@tempdimb #1\unitlength
  \PS@special{%
       \strippt\@wholewidth
       \strippt\@tempdimb
       !C}}
%    \end{macrocode}
% \end{macro}
%
% \begin{macro}{\@dot}
% Internal name for |\circle*|.
%    \begin{macrocode}
\def\@dot#1{%
  \@tempdimb #1\unitlength
  \PS@special{%
       \strippt\@tempdimb
       !D}}
%    \end{macrocode}
% \end{macro}
%
% \begin{macro}{\line}
% Line with a \LaTeX\ style slope specification.
%    \begin{macrocode}
\def\line(#1,#2)#3{%
  \@linelen=#3\unitlength
  \PS@special{%
       \strippt\@wholewidth
       #1
       #2
       \strippt\@linelen
       !L}}
%    \end{macrocode}
% \end{macro}
%
% \begin{macro}{\vector}
% Line and arrow head with a \LaTeX\ style slope specification.
%    \begin{macrocode}
\def\vector(#1,#2)#3{%
  \@linelen=#3\unitlength
  \PS@special{%
       \strippt\@arrowlength
       \strippt\@wholewidth
       #1
       #2
       \strippt\@linelen
       !V}}
%    \end{macrocode}
% \end{macro}
%
% \begin{macro}{\oval}
% If no optional argument appears, use a default of maximum radius of
% \TeX's maximum length.
%    \begin{macrocode}
\def\oval{%
  \@ifnextchar[%
   {\@ov@l}%
   {\count@=\maxdimen \divide\count@ by \unitlength \@ov@l[\count@]}}
%    \end{macrocode}
% \end{macro}
%
% \begin{macro}{\@ov@l}
% Look for an optional |tlbr| argument.
%    \begin{macrocode}
\def\@ov@l[#1](#2,#3){%
  \@ifnextchar[{\@oval[#1](#2,#3)}{\@oval[#1](#2,#3)[]}}%
%    \end{macrocode}
% \end{macro}
%
% \begin{macro}{\@oval}
% The PostScript version of the |\oval| command will print each quarter
% of the oval separately, each quarter will only be printed if the
% appropriate argument is $1$. An optional argument of |t| causes the
% arguments for the two bottom quarters to be set to $0$, similarly, |r|
% causes the two left quarters to be set to $0$. Thus an argument [tr]
% will set the bottom and left quarters to $0$, resulting in only the
% top right quarter being printed.
%    \begin{macrocode}
\def\@oval[#1](#2,#3)[#4]{\begingroup
  \@tempdimb #1\unitlength
  \@ovxx #2\unitlength
  \@ovyy #3\unitlength
  \def\r{\def\TL{0 }\def\BL{0 }}%
  \def\l{\def\TR{0 }\def\BR{0 }}%
  \def\t{\def\BL{0 }\def\BR{0 }}%
  \def\b{\def\TL{0 }\def\TR{0 }}%
  \def\TL{1 }\def\BL{1 }\def\TR{1 }\def\BR{1 }%
  \@tfor\@tempa :=#4\do{\csname\@tempa\endcsname}%
  \PS@special{%
       \BR\BL\TR\TL
       \strippt\@wholewidth
       \strippt\@tempdimb
       \strippt\@ovxx
       \strippt\@ovyy
       !O}%
  \endgroup}
%    \end{macrocode}
% \end{macro}
%
% \begin{macro}{\Line}
% New style |\Line| command.
%    \begin{macrocode}
\def\Line(#1,#2){%
  \@ovxx #1\unitlength
  \@ovyy #2\unitlength
  \PS@special{%
       \strippt\@wholewidth
       \strippt\@ovxx
       \strippt\@ovyy
       !L2}}
%    \end{macrocode}
% \end{macro}
%
% \begin{macro}{\Curve}
% Not particularly good, but it will do for now.
%    \begin{macrocode}
\def\Curve(#1,#2)#3{%
  \@ovxx #1\unitlength
  \@ovyy #2\unitlength
  \PS@special{%
       \strippt\@wholewidth
       \strippt\@ovxx
       \strippt\@ovyy
       #3
       !C2}}
%    \end{macrocode}
% \end{macro}
%
% \begin{macro}{\Vector}
% New style |\Vector| command.
%    \begin{macrocode}
\def\Vector(#1,#2){%
  \@ovxx #1\unitlength
  \@ovyy #2\unitlength
  \PS@special{%
       \strippt\@arrowlength
       \strippt\@wholewidth
       \strippt\@ovxx
       \strippt\@ovyy
       !V2}}
%    \end{macrocode}
% \end{macro}
%
% \begin{macro}{\@arrowlength}
% Length of an arrow head.
%    \begin{macrocode}
\newdimen\@arrowlength
%    \end{macrocode}
% \end{macro}
%
% \begin{macro}{\arrowlength}
% Set the length of an arrow head.
%    \begin{macrocode}
\def\arrowlength#1{\@arrowlength #1}
\arrowlength{8pt}
%    \end{macrocode}
% \end{macro}
% If this file is used as a {\tt .sty} file without being stripped, we
% want to stop here. The |\endinput| must not be at the beginning of the
% line, or {\sf DocStrip} will stop here as well!.
%    \begin{macrocode}
     \endinput
%    \end{macrocode}
%
%    \begin{macrocode}
%</package>
%    \end{macrocode}

% \section{texpicture.sty}
% A dummy style file so that documents using {\tt pspicture.sty}
% can be previewed or printed (as much as possible) using a
% dvi (not PostScript) previewer or printer driver.

% Just change `{\tt pspicture}' to `{\tt texpicture}' in the
% |\documentstyle| options list.

%    \begin{macrocode}
%<*texsty>
%    \end{macrocode}

%    \begin{macrocode}
\@warning{texpicture.sty in operation:^^J\@spaces
LaTeX document with pspicture.sty before printing}
%    \end{macrocode}
% \begin{macro}{\Line}
% \begin{macro}{\Vector}
% \begin{macro}{\arrowlength}
% \begin{macro}{\Curve}
% Define all these new commands to silently gobble their arguments.
%    \begin{macrocode}
\def\Line(#1,#2){}
\def\Vector(#1,#2){}
\def\arrowlength#1{}
\def\Curve(#1,#2)#3{}
%    \end{macrocode}
% \end{macro}
% \end{macro}
% \end{macro}
% \end{macro}
%
% \begin{macro}{\@badlinearg}
% If a vector or line is called with a slope specification that is not
% allowed by standard \LaTeX, |\@badlinearg| is called to produce the
% error message. We do not want to see these errors, so:
%    \begin{macrocode}
\def\@badlinearg{}
%    \end{macrocode}
% \end{macro}
%
% \begin{macro}{\oval}
% \begin{macro}{\@@v@l}
% \begin{macro}{\@@vv@l}
% Give the standard |\oval| command another optional argument (which
% will be ignored), to match the extra argument defined in {\tt
% pspicture.sty}.
%    \begin{macrocode}
\let\@@v@l\oval
\def\@@@v@l[#1]{\@@v@l}
\def\oval{\@ifnextchar[{\@@@v@l}{\@@v@l}}
%    \end{macrocode}
% \end{macro}
% \end{macro}
% \end{macro}
%
%    \begin{macrocode}
%</texsty>
%    \end{macrocode}
%
% \section{pspicture.ps}
% The PostScript header file for use with {\tt pspicture.sty}.
% Probably this should use the PostScript dictionary mechanism, to keep
% identifiers local to this package, but for now, just give them names
% begining with {\tt!}.
%    \begin{macrocode}
%<*ps>
%    \end{macrocode}
%
% \begin{macro}{!BP}
% PostScript uses \TeX's |bp|, that is $1/72$ of an inch, not \TeX's
% |pt|, $1/72.27$ of an inch, but it is inconvenient to get \TeX\ to
% output in |bp|, so we need to scale the PostScript.
%    \begin{macrocode}
/!BP{
  72 72.27 div dup scale
  }def
%    \end{macrocode}
% \end{macro}
%
% \begin{macro}{!A}
% Arrow head:\\
% \meta{arrow length} |!A|
%    \begin{macrocode}
/!A{
  newpath
  0 0 moveto
  dup neg dup .4 mul rlineto
  .8 mul 0 exch  rlineto
  closepath
  fill
  } def
%    \end{macrocode}
% \end{macro}
%
% \begin{macro}{!V}
% |\vector(|\meta{x}|,|\meta{y}|)| \\
% \meta{arrow length} \meta{line width} \meta{x} \meta{y}
%             \meta{len*unitlength} |!V|
%    \begin{macrocode}
/!V{
  !BP
  /!X exch def
  /!y exch def
  /!x exch def
  newpath
  0 0 moveto
  !x 0 eq {0  !y 0 lt {!X neg}{!X} ifelse}
         {!x 0 lt {!X neg}{!X}ifelse  !X !y mul !x abs div} ifelse
  lineto
  setlinewidth  % @wholewidth
  currentpoint
  stroke
  translate
  !y !x atan
  rotate
  !A            % @arrowlength
  }def
%    \end{macrocode}
% \end{macro}
%
% \begin{macro}{!L}
% |\line(|\meta{x}|,|\meta{y}|)| \\
% \meta{arrow length} \meta{line width} \meta{x} \meta{y}
%             \meta{len*unitlength} |!L|
%    \begin{macrocode}
/!L{
  !BP
  /!X exch def
  /!y exch def
  /!x exch def
  newpath
  0 0 moveto
  !x 0 eq {0  !y 0 lt {!X neg}{!X} ifelse}
         {!x 0 lt {!X neg}{!X}ifelse  !X !y mul !x abs div} ifelse
  lineto
  setlinewidth  % @wholewidth
  stroke
  }def
%    \end{macrocode}
% \end{macro}
%
% \begin{macro}{!C}
% |\circle{|\meta{diam}|}| \\
% \meta{line width} \meta{diam*unitlength} |!C|
%    \begin{macrocode}
/!C{
  !BP
  0 0 3 2 roll
  2 div 0 360 arc
  setlinewidth  % @wholewidth
  stroke
  }def
%    \end{macrocode}
% \end{macro}
%
% \begin{macro}{!D}
% |\circle*{|\meta{diam}|}| \\
% \meta{diam*unitlength} |!D|
%    \begin{macrocode}
/!D{
  !BP
  0 0 3 2 roll
  2 div 0 360 arc fill
  }def
%    \end{macrocode}
% \end{macro}
%
% \begin{macro}{!O}
% |\oval[|\meta{max-radius}|](|\meta{x}|,|\meta{y}|)[|\meta{tlbr}|]| \\
% \meta{br}\meta{bl}\meta{tr}\meta{tl}\\
%       \meta{line width} \meta{max-radius*unitlength}
%          \meta{x*unitlength} \meta{y*unitlength} |!O|\\
% The first four arguments should be either $0$, denoting that that
% quarter should not be drawn, or $1$, to draw a quarter oval.
%    \begin{macrocode}
/!O{
  !BP
  /!y exch 2 div def
  /!x exch 2 div def
  /!r exch !x !y
%    \end{macrocode}
% Ghostscript appears to have a {\tt min} operator, so the following 2
% lines could be coded as {\tt min min}, but it's not in the
% Ref.~Manual, and it doesn't work on my printer!
%    \begin{macrocode}
    2 copy gt {exch} if pop
    2 copy gt {exch} if pop
      def
  setlinewidth  % @wholewidth
  1 eq
  {newpath
   !x neg 0 moveto
   !x neg !y 0 !y !r arcto 4 {pop} repeat
   0 !y lineto
   stroke}if
  1 eq
  {newpath
   !x  0 moveto
   !x  !y 0 !y !r arcto 4 {pop} repeat
   0 !y lineto
   stroke}if
  1 eq
  {newpath
   !x neg 0 moveto
   !x neg !y neg 0 !y neg  !r arcto 4 {pop} repeat
   0 !y neg lineto
   stroke}if
  1 eq
  {newpath
   !x  0 moveto
   !x  !y neg 0 !y neg !r arcto 4 {pop} repeat
   0 !y neg lineto
   stroke}if
  }def
%    \end{macrocode}
% \end{macro}
%
% \begin{macro}{!V2}
% |\Vector(|\meta{x}|,|\meta{y}|)| \\
% \meta{arrow length} \meta{line width} \meta{x*unitlength}
% \meta{y*unitlength} |!V2|
%    \begin{macrocode}
/!V2{
  !BP
  2 copy exch
  atan
  /a exch def
  2 copy
  newpath
  0 0 moveto
  lineto          % <x*unitlength> <y*unitlength>
  3 2 roll
  setlinewidth  % @wholewidth
  stroke
  translate       % <x*unitlength> <y*unitlength>
  a rotate
  !A                    % @arrowlength
  }def
%    \end{macrocode}
% \end{macro}
%
% \begin{macro}{!L2}
% |\Line(|\meta{x}|,|\meta{y}|)| \\
% \meta{line width} \meta{x*unitlength} \meta{y*unitlength} |!L2|
%    \begin{macrocode}
/!L2{
  !BP
  newpath
  0 0 moveto
  lineto          % <x*unitlength> <y*unitlength>
  setlinewidth  % @wholewidth
  stroke
  }def
%    \end{macrocode}
% \end{macro}
%
% \begin{macro}{!C2}
% |\Curve(|\meta{x}|,|\meta{y}|){|\meta{$\pm$}|}| \\
% \meta{line width} \meta{x*unitlength} \meta{y*unitlength} \meta{$\pm$}
%             |!C2|
%    \begin{macrocode}
/!C2{
  !BP
  /!s exch def
  /!y exch def
  /!x exch def
  newpath
  0 0 moveto
  0 0
  !x 2 div !y 10 div !s mul add
  !y 2 div  !x 10 div  !s mul sub
  !x !y
  curveto
  setlinewidth  % @wholewidth
  stroke
  }def
%    \end{macrocode}
% \end{macro}
%
%    \begin{macrocode}
%</ps>
%    \end{macrocode}
%
% \end{document}
% \Finale