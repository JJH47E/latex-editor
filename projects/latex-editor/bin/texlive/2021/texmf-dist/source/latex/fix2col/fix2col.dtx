% \iffalse
%% Source File: fix2col.dtx Copyright 1997 1998 David Carlisle
%%
%% This file may be distributed under the terms of the LPPL.
%% See 00readme.txt for details.
%
%<*dtx>
          \ProvidesFile{fix2col.dtx}
%</dtx>
%<package>\NeedsTeXFormat{LaTeX2e}
%<package>\ProvidesPackage{fix2col}
%<driver> \ProvidesFile{fix2col.drv}
% \fi
%         \ProvidesFile{fix2col.dtx}
       [2015/11/13 v0.04 Output Routine fixes for two column mode (DPC,FMi)]
%
% \iffalse
%<*driver>
\documentclass{ltxdoc}
\begin{document}
\DocInput{fix2col.dtx}
\end{document}
%</driver>
% \fi
%
% \GetFileInfo{fix2col.dtx}
% \CheckSum{583}
%
% \changes{v0.01}{1997/11/07}
%    {Initial version}
%
% \changes{v0.02}{1997/11/11}
%    {\cs{@ztryfc}}
%
% \title{The \textsf{fix2col} package\thanks{This file
%        has version number \fileversion, last
%        revised \filedate.}}
% \author{David Carlisle\thanks{%^^A
% Part one is essentially a copy of the fixmarks package by
% Piet van Oostrum, itself based on earlier work by Joe Pallas.
% Part two is loosely based on the fixfloats package, originally
% by Ed Sznyter, with some modifications by Bil Kleb.}}
% 
% \date{\filedate}
%
% \let\package\textsf
% \let\env\textsf
% \let\url\texttt
%
%  \maketitle
%
% \section{Introduction}
%
% \textbf{Note:} This package is obsolete, the fixes described here are 
% incorporated into \LaTeX\ formats from 2015/01/01 onwards.
%
% This package makes two independent changes to \LaTeX's two column
% output routine to fix the following two longstanding `features'.
%
% \begin{itemize}
% \item
% If the \TeX\ mark system is used (for example using the 
% `headings' page style in the standard \LaTeX\ classes) then any marks
% that originate on the first column are `lost' as \LaTeX\ constructs
% the second column. An example document showing how this can result
% in incorrect page headings may be found in the latex bug database:\\
% \url{http://www.uni-mainz.de/cgi-bin/ltxbugs2html?pr=latex/2613}
%
% \item
% The second feature is documented in the \LaTeX\ book. By default
% \LaTeX\ does not attempt to keep double and single column floats in
% sequence, so if `Figure 1' is a double column float produced with
% |figure*|, then it may float 
% after `Figure 2' if that is a single column, |figure|, float.
% Further correspondence about this may also be found in the
% bug database:\\
% \url{http://www.uni-mainz.de/cgi-bin/ltxbugs2html?pr=latex/2346}
% \end{itemize}
%
% \StopEventually{}
%
% \section{Notes on the Implementation Strategies}
%
% \subsection{Preserving Marks}
%
% The standard \LaTeX\ twocolumn system works internally by making
% each column a separate `page' that is passed independently to \TeX's
% pagebreaker. (Unlike say the \package{multicol} package, where all
% columns are gathered together and then split into columns later,
% using |\vsplit|.) This means that the primitive \TeX\ marks that are
% normally used for header information, are globally reset after the
% first column. By default \LaTeX\ does nothing about this.
% A good solution is provided by Piet van Oostrum (building on earlier
% work of Joe Pallas) in his \package{fixmarks} package.
%
% After the first column box has been collected the mark information
% for that box is saved, so that any |\firstmark| can be
% `artificially' used to set the page-level marks after the second
% column has been collected. (The second column |\firstmark| is not
% normally required.) Unfortunately \TeX\ does not provide a direct
% way of knowing if any marks are in the page, |\firstmark| always has a
% value from previous pages, even if there is no mark in this page.
% The solution is to make a copy of the box and then |\vsplit| it
% so that any marks show up as |\splitfirstmark|.
%
% The use of |\vsplit| does mean that the output routine will globally
% change the value of |\splitfirstmark| and
% |\splitbotmark|. The \package{fixmarks} package goes to some trouble
% to save and restore these values so that the output routine does
% \emph{not} change the values. This part of \package{fixmarks} is not
% copied here as it is quite costly (having to be run on every page) and
% there is no reason why anyone writing code using |\vsplit| should
% allow the output routine to be triggered before the split marks have
% been accessed.
%
% \subsection{Preserving Float Order}
%
% The standard output routine maintains two lists of floats that have
% been `deferred' for later consideration. One list for single column
% floats, and one for double column floats (which are always
% immediately put onto their deferred list). This mechanism means
% that \LaTeX\ `knows' which type of float is contained in each box
% by the list that it is processing, but having two lists means
% that there is no mechanism for preserving the order between the
% floats in each list.
%
% The solution to this problem consists of two small changes to
% the output routine.
%
% Firstly, abandon the `double column float list' |\@dbldeferlist|
% and change every command where it is used so that instead the
% same |\@deferlist| is used as for single column floats.
% That one change ensures that double and single column floats
% stay in the same sequence, but as \LaTeX\ no longer `knows'
% whether a float is double or single column, it will happily
% insert a double float into a single column, overprinting the
% other column, or the margin.
%
% The second change is to provide an alternative mechanism for
% recording the two column floats. \LaTeX\ already has a compact
% mechanism for recording float information, an integer count register
% assigned to each float records information about the `type' of float
% `figure', `table' and the position information `htp' etc.
%
% The type information is stored in the `high' bits, one bit position
% (above `32') allocated to each float type. The `low' bits store
% information about the allowed positions, one bit each allocated for
% |h t b p|.  In the \LaTeX2.09 system, the bit corresponding to `16'
% formed a `boundary' between these two sets of information, and it
% was never actually used by the system. Ed Sznyter's
% \package{fixfloats} package not unreasonably used this position to
% store the double column information, setting the bit for double
% column floats. Then at each point in the output routine at which a
% float is committed to a certain region, an additional check must be
% made to check that the float is (or is not) double column. If it
% spans the wrong number of columns it is deferred rather than being
% added. 
%
% Unfortunately the bit `16' is not available in \LaTeXe. It is used
% to encode the extra float position possibility `|!|' that was added
% in that system. It would be possible to use position `32' and to
% move the flags for `table', `figure',\ldots\ up one position, to
% start at 64, but this would mean that in principle one less float
% type would be supported, and more importantly is likely to break
% any other packages that assume anything about the output routine
% internals. So here I instead use another mechanism for flagging
% double column floats: By default all floats have depth 0pt.
% This package arranges that double column ones have depth 1sp.
% This information may then be used in the same manner as in
% the \package{fixfloats} package, to defer any floats that are not of
% the correct column spanning type.
%
% Use of the package showed that one also has to change the way
% \LaTeX{} handles star-form floats: if they are immediately deferred
% (as done normally) certain situations can still result in the float
% sequence getting out of order. This happens when a floats are placed
% in the middle of a paragraph.  In that case the wide float is
% deferred immediately while a column wide float early on in the same
% paragraph might not be handled until the end of the paragraph when
% it is finally seen by the output routine. Since by that time the
% wide float is already on the |\@deferlist| the column float will
% also end up there (which is not only incorrect because it may have
% fitted onto the page but also because it is then placed at the end
% of this list). Version v0.03 now fixes this problem.
%
%
% \section{Implementation}
%    \begin{macrocode}
%<*package>
%    \end{macrocode}
%
% \subsection{Do nothing on current releases}
%
% \changes{v0.04}{2015/11/03}
%    {Abort on new formats}
%    \begin{macrocode}
\@ifl@t@r\fmtversion{2014/12/31}
{\PackageWarningNoLine{fix2col}{Obsolete package: ignored}\endinput}
{}
%    \end{macrocode}
%
% \subsection{Preserving Marks}
%
% This is just a change to the single command |\@outputdblcol|
% so that it saves mark information for the first column and restores
% it in the second column.
%    \begin{macrocode}
\def\@outputdblcol{%
  \if@firstcolumn
    \global\@firstcolumnfalse
%    \end{macrocode}
% Save the left column
%    \begin{macrocode}
    \global\setbox\@leftcolumn\copy\@outputbox
%    \end{macrocode}
%
% Remember the marks from the first column
%    \begin{macrocode}
    \splitmaxdepth\maxdimen
    \vbadness\maxdimen
    \setbox\@outputbox\vsplit\@outputbox to\maxdimen
%    \end{macrocode}
%
% One minor difference from the current \package{fixmarks}, pass the 
% marks through a token register to stop any |#| tokens causing an
% error in a |\def|.
%    \begin{macrocode}
    \toks@\expandafter{\topmark}%
    \xdef\@firstcoltopmark{\the\toks@}%
    \toks@\expandafter{\splitfirstmark}%
    \xdef\@firstcolfirstmark{\the\toks@}%
%    \end{macrocode}
%
% This test does not work if truly empty marks have been inserted, but
% \LaTeX\ marks should always have (at least) two brace groups.
% (Except before the first mark is used, when the marks are empty,
% but that is OK here.)
%    \begin{macrocode}
    \ifx\@firstcolfirstmark\@empty
      \global\let\@setmarks\relax
    \else
      \gdef\@setmarks{%
        \let\firstmark\@firstcolfirstmark
        \let\topmark\@firstcoltopmark}%
    \fi
%    \end{macrocode}
%
% End of change
%    \begin{macrocode}
  \else
    \global\@firstcolumntrue
    \setbox\@outputbox\vbox{%
     \hb@xt@\textwidth{%
        \hb@xt@\columnwidth{\box\@leftcolumn \hss}%
        \hfil
        \vrule \@width\columnseprule
        \hfil
       \hb@xt@\columnwidth{\box\@outputbox \hss}}}%
  \@combinedblfloats
%    \end{macrocode}
% Override current first and top with those of first column if necessary
%    \begin{macrocode}
    \@setmarks
%    \end{macrocode}
% End of change
%    \begin{macrocode}
    \@outputpage
    \begingroup
      \@dblfloatplacement
      \@startdblcolumn
      \@whilesw\if@fcolmade \fi{\@outputpage\@startdblcolumn}%
    \endgroup
  \fi}
%    \end{macrocode}
%
% \subsection{Preserving Float Order}
%
% Changes |\@dbldeferlist| to |\@deferlist| are not explicitly noted
% but are flagged by blank comment lines around the changed line.
%
%
%    \begin{macrocode}
\def\end@dblfloat{%
\if@twocolumn
  \@endfloatbox
  \ifnum\@floatpenalty <\z@
    \@largefloatcheck
%    \end{macrocode}
%
%  Force the depth of two column float boxes.
%    \begin{macrocode}
    \global\dp\@currbox1sp %
%    \end{macrocode}
%
% \changes{v0.03}{1998/08/17}{FMi: use output routine to
%    defer float}
%    Next line assumes that first token of |\end@float| is
%    |\@endfloatbox| so we gobble that.
%    \begin{macrocode}
%    \@cons\@deferlist\@currbox
     \expandafter\@gobble\end@float
%    \end{macrocode}
%    |\@Esphack| is then added by |\@endfloat| above.
%    \begin{macrocode}
  \fi
%    \ifnum \@floatpenalty =-\@Mii \@Esphack\fi
\else
  \end@float
\fi
}
%    \end{macrocode}
%
% Test if the float box has the wrong width. (Actually as noted above
% the test is for a conventional depth setting rather than for the
% width of the float).
%    \begin{macrocode}
\def\@testwrongwidth #1{%
  \ifdim\dp#1=\f@depth 
  \else
    \global\@testtrue
  \fi}
%    \end{macrocode}
%
% Normally looking for single column floats, which have zero depth.
%    \begin{macrocode}
\let\f@depth\z@
%    \end{macrocode}
%
% but when making two column float area, look for floats with 1sp
% depth.
%    \begin{macrocode}
\def\@dblfloatplacement{\global\@dbltopnum\c@dbltopnumber
   \global\@dbltoproom \dbltopfraction\@colht
   \@textmin \@colht
   \advance \@textmin -\@dbltoproom
   \@fpmin \dblfloatpagefraction\textheight
   \@fptop \@dblfptop
   \@fpsep \@dblfpsep
   \@fpbot \@dblfpbot
%    \end{macrocode}
%
%    \begin{macrocode}
   \def\f@depth{1sp}}
%    \end{macrocode}
%
% All the remaining changes are replacing the double column defer list
% or insering the extra test |\@testwrongwidth|\marg{box} at suitable
% places. That is at plces where a box is taken off the deferlist.
%    \begin{macrocode}
\def \@doclearpage {%
     \ifvoid\footins
       \setbox\@tempboxa\vsplit\@cclv to\z@ \unvbox\@tempboxa
       \setbox\@tempboxa\box\@cclv
       \xdef\@deferlist{\@toplist\@botlist\@deferlist}%
       \global \let \@toplist \@empty
       \global \let \@botlist \@empty
       \global \@colroom \@colht
       \ifx \@currlist\@empty
       \else
          \@latexerr{Float(s) lost}\@ehb
          \global \let \@currlist \@empty
       \fi
       \@makefcolumn\@deferlist
       \@whilesw\if@fcolmade \fi{\@opcol\@makefcolumn\@deferlist}%
       \if@twocolumn
         \if@firstcolumn
%    \end{macrocode}
%
%    \begin{macrocode}
           \xdef\@deferlist{\@dbltoplist\@deferlist}%
%    \end{macrocode}
%
%    \begin{macrocode}
           \global \let \@dbltoplist \@empty
           \global \@colht \textheight
           \begingroup
              \@dblfloatplacement
%    \end{macrocode}
%
%    \begin{macrocode}
              \@makefcolumn\@deferlist
              \@whilesw\if@fcolmade \fi{\@outputpage
                                        \@makefcolumn\@deferlist}%
%    \end{macrocode}
%
%    \begin{macrocode}
           \endgroup
         \else
           \vbox{}\clearpage
         \fi
       \fi
%    \end{macrocode}
%    the next line is needed to avoid losing floats in certain 
%    circumstances a single call to the original |\doclearpage|
%    will now no longer output all floats.
% \changes{v0.03}{1998/08/17}{FMi: ensure \cs{doclearpage} is called again
%     until all floats are output}
%    \begin{macrocode}
       \ifx\@deferlist\@empty \else\clearpage \fi
     \else
       \setbox\@cclv\vbox{\box\@cclv\vfil}%
       \@makecol\@opcol
       \clearpage
     \fi
}
%    \end{macrocode}
%
%    \begin{macrocode}
\def \@startdblcolumn {%
  \@tryfcolumn \@deferlist
  \if@fcolmade
  \else
    \begingroup
      \let \reserved@b \@deferlist
      \global \let \@deferlist \@empty
      \let \@elt \@sdblcolelt
      \reserved@b
    \endgroup
  \fi
}
%    \end{macrocode}
%
%    \begin{macrocode}
\def\@addtonextcol{%
  \begingroup
   \@insertfalse
   \@setfloattypecounts
   \ifnum \@fpstype=8
   \else
     \ifnum \@fpstype=24
     \else
       \@flsettextmin
       \@reqcolroom \ht\@currbox
       \advance \@reqcolroom \@textmin
       \ifdim \@colroom>\@reqcolroom
         \@flsetnum \@colnum
         \ifnum\@colnum>\z@
            \@bitor\@currtype\@deferlist
            \@testwrongwidth\@currbox
            \if@test
            \else
              \@addtotoporbot
            \fi
         \fi
       \fi
     \fi
   \fi
   \if@insert
   \else
     \@cons\@deferlist\@currbox
   \fi
  \endgroup
}
%    \end{macrocode}
%
%    \begin{macrocode}
\def\@addtodblcol{%
  \begingroup
   \@insertfalse
   \@setfloattypecounts
   \@getfpsbit \tw@
   \ifodd\@tempcnta
     \@flsetnum \@dbltopnum
     \ifnum \@dbltopnum>\z@
       \@tempswafalse
       \ifdim \@dbltoproom>\ht\@currbox
         \@tempswatrue
       \else
         \ifnum \@fpstype<\sixt@@n
           \advance \@dbltoproom \@textmin
           \ifdim \@dbltoproom>\ht\@currbox
             \@tempswatrue
           \fi
           \advance \@dbltoproom -\@textmin
         \fi
       \fi
       \if@tempswa
           \@bitor \@currtype \@deferlist
%    \end{macrocode}
%
% not in fixfloats?
%    \begin{macrocode}
          \@testwrongwidth\@currbox
%    \end{macrocode}
%
%    \begin{macrocode}
           \if@test
           \else
              \@tempdima -\ht\@currbox
              \advance\@tempdima
                -\ifx \@dbltoplist\@empty \dbltextfloatsep \else
                                          \dblfloatsep \fi
              \global \advance \@dbltoproom \@tempdima
              \global \advance \@colht \@tempdima
              \global \advance \@dbltopnum \m@ne
              \@cons \@dbltoplist \@currbox
              \@inserttrue
           \fi
       \fi
     \fi
   \fi
   \if@insert
   \else
     \@cons\@deferlist\@currbox
   \fi
  \endgroup
}
%    \end{macrocode}
%
%    \begin{macrocode}
\def \@addtocurcol {%
   \@insertfalse
   \@setfloattypecounts
   \ifnum \@fpstype=8
   \else
     \ifnum \@fpstype=24
     \else
       \@flsettextmin
       \advance \@textmin \@textfloatsheight
       \@reqcolroom \@pageht
       \ifdim \@textmin>\@reqcolroom
         \@reqcolroom \@textmin
       \fi
       \advance \@reqcolroom \ht\@currbox
       \ifdim \@colroom>\@reqcolroom
         \@flsetnum \@colnum
         \ifnum \@colnum>\z@
           \@bitor\@currtype\@deferlist
%    \end{macrocode}
%    We need to defer the float also if its width
%    doesn't fit.
% \changes{v0.03}{1998/08/17}{FMi: test for wide float was
%    in wrong place}
%    \begin{macrocode}
          \@testwrongwidth\@currbox
%    \end{macrocode}
%
%    \begin{macrocode}
           \if@test
           \else
             \@bitor\@currtype\@botlist
             \if@test
               \@addtobot
             \else
               \ifodd \count\@currbox
                 \advance \@reqcolroom \intextsep
                 \ifdim \@colroom>\@reqcolroom
                   \global \advance \@colnum \m@ne
                   \global \advance \@textfloatsheight \ht\@currbox
                   \global \advance \@textfloatsheight 2\intextsep
                   \@cons \@midlist \@currbox
                   \if@nobreak
                     \nobreak
                     \@nobreakfalse
                     \everypar{}%
                   \else
                     \addpenalty \interlinepenalty
                   \fi
                   \vskip \intextsep
                   \box\@currbox
                   \penalty\interlinepenalty
                   \vskip\intextsep
                   \ifnum\outputpenalty <-\@Mii \vskip -\parskip\fi
                   \outputpenalty \z@
                   \@inserttrue
                 \fi
               \fi
               \if@insert
               \else
                 \@addtotoporbot
               \fi
             \fi
           \fi
         \fi
       \fi
     \fi
   \fi
   \if@insert
   \else
     \@resethfps
     \@cons\@deferlist\@currbox
   \fi
}
%    \end{macrocode}
%
%    \begin{macrocode}
\def\@xtryfc #1{%
  \@next\reserved@a\@trylist{}{}%
  \@currtype \count #1%
  \divide\@currtype\@xxxii
  \multiply\@currtype\@xxxii
  \@bitor \@currtype \@failedlist
  \@testfp #1%
%    \end{macrocode}
%
%    \begin{macrocode}
  \@testwrongwidth #1%
%    \end{macrocode}
%
%    \begin{macrocode}
  \ifdim \ht #1>\@colht
     \@testtrue
  \fi
  \if@test 
    \@cons\@failedlist #1%
  \else
    \@ytryfc #1%
  \fi}
%    \end{macrocode}
%
%    \begin{macrocode}
\def\@ztryfc #1{%
  \@tempcnta\count #1%
  \divide\@tempcnta\@xxxii
  \multiply\@tempcnta\@xxxii 
  \@bitor \@tempcnta {\@failedlist \@flfail}%
  \@testfp #1%
%    \end{macrocode}
%
% not in fixfloats?
%    \begin{macrocode}
  \@testwrongwidth #1%
%    \end{macrocode}
%
%    \begin{macrocode}
  \@tempdimb\@tempdima
  \advance\@tempdimb\ht #1%
  \advance\@tempdimb\@fpsep
  \ifdim \@tempdimb >\@colht
    \@testtrue
  \fi
  \if@test
    \@cons\@flfail #1%
  \else
    \@cons\@flsucceed #1%
    \@tempdima\@tempdimb
  \fi}
%    \end{macrocode}
%
%    \begin{macrocode}
%</package>
%    \end{macrocode}
%
% \Finale
%
