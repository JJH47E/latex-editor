% \begin{meta-comment}
%
% $Id: footnote.dtx,v 1.13 1997/01/28 19:45:16 mdw Exp $
%
% Save footnotes around boxing environments and things
%
% (c) 1996 Mark Wooding
%
%----- Revision history -----------------------------------------------------
%
% $Log: footnote.dtx,v $
% Revision 1.13  1997/01/28 19:45:16  mdw
% Fixed stupid bug in AMS environment handling which stops the thing from
% working properly if you haven't included amsmath.  Doh.
%
% Revision 1.12  1997/01/18 00:45:37  mdw
% Fix problems with duplicated footnotes in broken AMS environments which
% typeset things multiple times.  This is a nasty kludge.
%
% Revision 1.11  1996/11/19 20:50:05  mdw
% Entered into RCS
%
%
% \end{meta-comment}
%
% \begin{meta-comment} <general public licence>
%%
%% footnote package -- Save footnotes around boxing environments
%% Copyright (c) 1996 Mark Wooding
%<*package>
%%
%% This program is free software; you can redistribute it and/or modify
%% it under the terms of the GNU General Public License as published by
%% the Free Software Foundation; either version 2 of the License, or
%% (at your option) any later version.
%%
%% This program is distributed in the hope that it will be useful,
%% but WITHOUT ANY WARRANTY; without even the implied warranty of
%% MERCHANTABILITY or FITNESS FOR A PARTICULAR PURPOSE.  See the
%% GNU General Public License for more details.
%%
%% You should have received a copy of the GNU General Public License
%% along with this program; if not, write to the Free Software
%% Foundation, Inc., 675 Mass Ave, Cambridge, MA 02139, USA.
%</package>
%%
% \end{meta-comment}
%
% \begin{meta-comment} <Package preamble>
%<+package>\NeedsTeXFormat{LaTeX2e}
%<+package>\ProvidesPackage{footnote}
%<+package>                [1997/01/28 1.13 Save footnotes around boxes]
% \end{meta-comment}
%
% \CheckSum{327}
%\iffalse
%<*package>
%\fi
%% \CharacterTable
%%  {Upper-case    \A\B\C\D\E\F\G\H\I\J\K\L\M\N\O\P\Q\R\S\T\U\V\W\X\Y\Z
%%   Lower-case    \a\b\c\d\e\f\g\h\i\j\k\l\m\n\o\p\q\r\s\t\u\v\w\x\y\z
%%   Digits        \0\1\2\3\4\5\6\7\8\9
%%   Exclamation   \!     Double quote  \"     Hash (number) \#
%%   Dollar        \$     Percent       \%     Ampersand     \&
%%   Acute accent  \'     Left paren    \(     Right paren   \)
%%   Asterisk      \*     Plus          \+     Comma         \,
%%   Minus         \-     Point         \.     Solidus       \/
%%   Colon         \:     Semicolon     \;     Less than     \<
%%   Equals        \=     Greater than  \>     Question mark \?
%%   Commercial at \@     Left bracket  \[     Backslash     \\
%%   Right bracket \]     Circumflex    \^     Underscore    \_
%%   Grave accent  \`     Left brace    \{     Vertical bar  \|
%%   Right brace   \}     Tilde         \~}
%%
%\iffalse
%</package>
%\fi
%
% \begin{meta-comment} <driver>
%
%<*driver>
% \begin{meta-comment}
%
% $Id: mdwtools.tex,v 1.4 1996/11/19 20:55:55 mdw Exp $
%
% Common declarations for mdwtools.dtx files
%
% (c) 1996 Mark Wooding
%
%----- Revision history -----------------------------------------------------
%
% $Log: mdwtools.tex,v $
% Revision 1.4  1996/11/19 20:55:55  mdw
% Entered into RCS
%
%
% \end{meta-comment}
%
% \begin{meta-comment} <general public licence>
%%
%% mdwtools common declarations
%% Copyright (c) 1996 Mark Wooding
%%
%% This program is free software; you can redistribute it and/or modify
%% it under the terms of the GNU General Public License as published by
%% the Free Software Foundation; either version 2 of the License, or
%% (at your option) any later version.
%%
%% This program is distributed in the hope that it will be useful,
%% but WITHOUT ANY WARRANTY; without even the implied warranty of
%% MERCHANTABILITY or FITNESS FOR A PARTICULAR PURPOSE.  See the
%% GNU General Public License for more details.
%%
%% You should have received a copy of the GNU General Public License
%% along with this program; if not, write to the Free Software
%% Foundation, Inc., 675 Mass Ave, Cambridge, MA 02139, USA.
%%
% \end{meta-comment}
%
% \begin{meta-comment} <file preamble>
%<*mdwtools>
\ProvidesFile{mdwtools.tex}
             [1996/05/10 1.4 Shared definitions for mdwtools .dtx files]
%</mdwtools>
% \end{meta-comment}
%
% \CheckSum{668}
%% \CharacterTable
%%  {Upper-case    \A\B\C\D\E\F\G\H\I\J\K\L\M\N\O\P\Q\R\S\T\U\V\W\X\Y\Z
%%   Lower-case    \a\b\c\d\e\f\g\h\i\j\k\l\m\n\o\p\q\r\s\t\u\v\w\x\y\z
%%   Digits        \0\1\2\3\4\5\6\7\8\9
%%   Exclamation   \!     Double quote  \"     Hash (number) \#
%%   Dollar        \$     Percent       \%     Ampersand     \&
%%   Acute accent  \'     Left paren    \(     Right paren   \)
%%   Asterisk      \*     Plus          \+     Comma         \,
%%   Minus         \-     Point         \.     Solidus       \/
%%   Colon         \:     Semicolon     \;     Less than     \<
%%   Equals        \=     Greater than  \>     Question mark \?
%%   Commercial at \@     Left bracket  \[     Backslash     \\
%%   Right bracket \]     Circumflex    \^     Underscore    \_
%%   Grave accent  \`     Left brace    \{     Vertical bar  \|
%%   Right brace   \}     Tilde         \~}
%%
%
% \section{Introduction and user guide}
%
% This file is really rather strange; it gets |\input| by other package
% documentation files to set up most of the environmental gubbins for them.
% It handles almost everything, like loading a document class, finding any
% packages, and building and formatting the title.
%
% It also offers an opportunity for users to customise my nice documentation,
% by using a |mdwtools.cfg| file (not included).
%
%
% \subsection{Declarations}
%
% A typical documentation file contains something like
% \begin{listinglist} \listingsize \obeylines
% |% \begin{meta-comment}
%
% $Id: mdwtools.tex,v 1.4 1996/11/19 20:55:55 mdw Exp $
%
% Common declarations for mdwtools.dtx files
%
% (c) 1996 Mark Wooding
%
%----- Revision history -----------------------------------------------------
%
% $Log: mdwtools.tex,v $
% Revision 1.4  1996/11/19 20:55:55  mdw
% Entered into RCS
%
%
% \end{meta-comment}
%
% \begin{meta-comment} <general public licence>
%%
%% mdwtools common declarations
%% Copyright (c) 1996 Mark Wooding
%%
%% This program is free software; you can redistribute it and/or modify
%% it under the terms of the GNU General Public License as published by
%% the Free Software Foundation; either version 2 of the License, or
%% (at your option) any later version.
%%
%% This program is distributed in the hope that it will be useful,
%% but WITHOUT ANY WARRANTY; without even the implied warranty of
%% MERCHANTABILITY or FITNESS FOR A PARTICULAR PURPOSE.  See the
%% GNU General Public License for more details.
%%
%% You should have received a copy of the GNU General Public License
%% along with this program; if not, write to the Free Software
%% Foundation, Inc., 675 Mass Ave, Cambridge, MA 02139, USA.
%%
% \end{meta-comment}
%
% \begin{meta-comment} <file preamble>
%<*mdwtools>
\ProvidesFile{mdwtools.tex}
             [1996/05/10 1.4 Shared definitions for mdwtools .dtx files]
%</mdwtools>
% \end{meta-comment}
%
% \CheckSum{668}
%% \CharacterTable
%%  {Upper-case    \A\B\C\D\E\F\G\H\I\J\K\L\M\N\O\P\Q\R\S\T\U\V\W\X\Y\Z
%%   Lower-case    \a\b\c\d\e\f\g\h\i\j\k\l\m\n\o\p\q\r\s\t\u\v\w\x\y\z
%%   Digits        \0\1\2\3\4\5\6\7\8\9
%%   Exclamation   \!     Double quote  \"     Hash (number) \#
%%   Dollar        \$     Percent       \%     Ampersand     \&
%%   Acute accent  \'     Left paren    \(     Right paren   \)
%%   Asterisk      \*     Plus          \+     Comma         \,
%%   Minus         \-     Point         \.     Solidus       \/
%%   Colon         \:     Semicolon     \;     Less than     \<
%%   Equals        \=     Greater than  \>     Question mark \?
%%   Commercial at \@     Left bracket  \[     Backslash     \\
%%   Right bracket \]     Circumflex    \^     Underscore    \_
%%   Grave accent  \`     Left brace    \{     Vertical bar  \|
%%   Right brace   \}     Tilde         \~}
%%
%
% \section{Introduction and user guide}
%
% This file is really rather strange; it gets |\input| by other package
% documentation files to set up most of the environmental gubbins for them.
% It handles almost everything, like loading a document class, finding any
% packages, and building and formatting the title.
%
% It also offers an opportunity for users to customise my nice documentation,
% by using a |mdwtools.cfg| file (not included).
%
%
% \subsection{Declarations}
%
% A typical documentation file contains something like
% \begin{listinglist} \listingsize \obeylines
% |% \begin{meta-comment}
%
% $Id: mdwtools.tex,v 1.4 1996/11/19 20:55:55 mdw Exp $
%
% Common declarations for mdwtools.dtx files
%
% (c) 1996 Mark Wooding
%
%----- Revision history -----------------------------------------------------
%
% $Log: mdwtools.tex,v $
% Revision 1.4  1996/11/19 20:55:55  mdw
% Entered into RCS
%
%
% \end{meta-comment}
%
% \begin{meta-comment} <general public licence>
%%
%% mdwtools common declarations
%% Copyright (c) 1996 Mark Wooding
%%
%% This program is free software; you can redistribute it and/or modify
%% it under the terms of the GNU General Public License as published by
%% the Free Software Foundation; either version 2 of the License, or
%% (at your option) any later version.
%%
%% This program is distributed in the hope that it will be useful,
%% but WITHOUT ANY WARRANTY; without even the implied warranty of
%% MERCHANTABILITY or FITNESS FOR A PARTICULAR PURPOSE.  See the
%% GNU General Public License for more details.
%%
%% You should have received a copy of the GNU General Public License
%% along with this program; if not, write to the Free Software
%% Foundation, Inc., 675 Mass Ave, Cambridge, MA 02139, USA.
%%
% \end{meta-comment}
%
% \begin{meta-comment} <file preamble>
%<*mdwtools>
\ProvidesFile{mdwtools.tex}
             [1996/05/10 1.4 Shared definitions for mdwtools .dtx files]
%</mdwtools>
% \end{meta-comment}
%
% \CheckSum{668}
%% \CharacterTable
%%  {Upper-case    \A\B\C\D\E\F\G\H\I\J\K\L\M\N\O\P\Q\R\S\T\U\V\W\X\Y\Z
%%   Lower-case    \a\b\c\d\e\f\g\h\i\j\k\l\m\n\o\p\q\r\s\t\u\v\w\x\y\z
%%   Digits        \0\1\2\3\4\5\6\7\8\9
%%   Exclamation   \!     Double quote  \"     Hash (number) \#
%%   Dollar        \$     Percent       \%     Ampersand     \&
%%   Acute accent  \'     Left paren    \(     Right paren   \)
%%   Asterisk      \*     Plus          \+     Comma         \,
%%   Minus         \-     Point         \.     Solidus       \/
%%   Colon         \:     Semicolon     \;     Less than     \<
%%   Equals        \=     Greater than  \>     Question mark \?
%%   Commercial at \@     Left bracket  \[     Backslash     \\
%%   Right bracket \]     Circumflex    \^     Underscore    \_
%%   Grave accent  \`     Left brace    \{     Vertical bar  \|
%%   Right brace   \}     Tilde         \~}
%%
%
% \section{Introduction and user guide}
%
% This file is really rather strange; it gets |\input| by other package
% documentation files to set up most of the environmental gubbins for them.
% It handles almost everything, like loading a document class, finding any
% packages, and building and formatting the title.
%
% It also offers an opportunity for users to customise my nice documentation,
% by using a |mdwtools.cfg| file (not included).
%
%
% \subsection{Declarations}
%
% A typical documentation file contains something like
% \begin{listinglist} \listingsize \obeylines
% |\input{mdwtools}|
% \<declarations>
% |\mdwdoc|
% \end{listinglist}
% The initial |\input| reads in this file and sets up the various commands
% which may be needed.  The final |\mdwdoc| actually starts the document,
% inserting a title (which is automatically generated), a table of
% contents etc., and reads the documentation file in (using the |\DocInput|
% command from the \package{doc} package.
%
% \subsubsection{Describing packages}
%
% \DescribeMacro{\describespackage}
% \DescribeMacro{\describesclass}
% \DescribeMacro{\describesfile}
% \DescribeMacro{\describesfile*}
% The most important declarations are those which declare what the
% documentation describes.  Saying \syntax{"\\describespackage{<package>}"}
% loads the \<package> (if necessary) and adds it to the auto-generated
% title, along with a footnote containing version information.  Similarly,
% |\describesclass| adds a document class name to the title (without loading
% it -- the document itself must do this, with the |\documentclass| command).
% For files which aren't packages or classes, use the |\describesfile| or
% |\describesfile*| command (the $*$-version won't |\input| the file, which
% is handy for files like |mdwtools.tex|, which are already input).
%
% \DescribeMacro{\author}
% \DescribeMacro{\date}
% \DescribeMacro{\title}
% The |\author|, |\date| and |\title| declarations work slightly differently
% to normal -- they ensure that only the \emph{first} declaration has an
% effect.  (Don't you play with |\author|, please, unless you're using this
% program to document your own packages.)  Using |\title| suppresses the
% automatic title generation.
%
% \DescribeMacro{\docdate}
% The default date is worked out from the version string of the package or
% document class whose name is the same as that of the documentation file.
% You can choose a different `main' file by saying
% \syntax{"\\docdate{"<file>"}"}.
%
% \subsubsection{Contents handling}
%
% \DescribeMacro{\addcontents}
% A documentation file always has a table of contents.  Other
% contents-like lists can be added by saying
% \syntax{"\\addcontents{"<extension>"}{"<command>"}"}.  The \<extension>
% is the file extension of the contents file (e.g., \lit{lot} for the
% list of tables); the \<command> is the command to actually typeset the
% contents file (e.g., |\listoftables|).
%
% \subsubsection{Other declarations}
%
% \DescribeMacro{\implementation}
% The \package{doc} package wants you to say
% \syntax{"\\StopEventually{"<stuff>"}"}' before describing the package
% implementation.  Using |mdwtools.tex|, you just say |\implementation|, and
% everything works.  It will automatically read in the licence text (from
% |gpl.tex|, and wraps some other things up.
%
% 
% \subsection{Other commands}
%
% The |mdwtools.tex| file includes the \package{syntax} and \package{sverb}
% packages so that they can be used in documentation files.  It also defines
% some trivial commands of its own.
%
% \DescribeMacro{\<}
% Saying \syntax{"\\<"<text>">" is the same as "\\synt{"<text>"}"}; this
% is a simple abbreviation.
%
% \DescribeMacro{\smallf}
% Saying \syntax{"\\smallf" <number>"/"<number>} typesets a little fraction,
% like this: \smallf 3/4.  It's useful when you want to say that the default
% value of a length is 2 \smallf 1/2\,pt, or something like that.
%
%
% \subsection{Customisation}
%
% You can customise the way that the package documentation looks by writing
% a file called |mdwtools.cfg|.  You can redefine various commands (before
% they're defined here, even; |mdwtools.tex| checks most of the commands that
% it defines to make sure they haven't been defined already.
%
% \DescribeMacro{\indexing}
% If you don't want the prompt about whether to generate index files, you
% can define the |\indexing| command to either \lit{y} or \lit{n}.  I'd
% recommend that you use |\providecommand| for this, to allow further
% customisation from the command line.
%
% \DescribeMacro{\mdwdateformat}
% If you don't like my date format (maybe you're American or something),
% you can redefine the |\mdwdateformat| command.  It takes three arguments:
% the year, month and date, as numbers; it should expand to something which
% typesets the date nicely.  The default format gives something like
% `10 May 1996'.  You can produce something rather more exotic, like
% `10\textsuperscript{th} May \textsc{\romannumeral 1996}' by saying
%\begin{listing}
%\newcommand{\mdwdateformat}[3]{%
%  \number#3\textsuperscript{\numsuffix{#3}}\ %
%  \monthname{#2}\ %
%  \textsc{\romannumeral #1}%
%}
%\end{listing}
% \DescribeMacro{\monthname}
% \DescribeMacro{\numsuffix}
% Saying \syntax{"\\monthname{"<number>"}"} expands to the name of the
% numbered month (which can be useful when doing date formats).  Saying
% \syntax{"\\numsuffix{"<number>"}"} will expand to the appropriate suffix
% (`th' or `rd' or whatever) for the \<number>.  You'll have to superscript
% it yourself, if this is what you want to do.  Putting the year number
% in roman numerals is just pretentious |;-)|.
%
% \DescribeMacro{\mdwhook}
% After all the declarations in |mdwtools.tex|, the command |\mdwhook| is
% executed, if it exists.  This can be set up by the configuration file
% to do whatever you want.
%
% There are lots of other things you can play with; you should look at the
% implementation section to see what's possible.
%
% \implementation
%
% \section{Implementation}
%
%    \begin{macrocode}
%<*mdwtools>
%    \end{macrocode}
%
% The first thing is that I'm not a \LaTeX\ package or anything official
% like that, so I must enable `|@|' as a letter by hand.
%
%    \begin{macrocode}
\makeatletter
%    \end{macrocode}
%
% Now input the user's configuration file, if it exists.  This is fairly
% simple stuff.
%
%    \begin{macrocode}
\@input{mdwtools.cfg}
%    \end{macrocode}
%
% Well, that's the easy bit done.
%
%
% \subsection{Initialisation}
%
% Obviously the first thing to do is to obtain a document class.  Obviously,
% it would be silly to do this if a document class has already been loaded,
% either by the package documentation or by the configuration file.
%
% The only way I can think of for finding out if a document class is already
% loaded is by seeing if the |\documentclass| command has been redefined
% to raise an error.  This isn't too hard, really.
%
%    \begin{macrocode}
\ifx\documentclass\@twoclasseserror\else
  \documentclass[a4paper]{ltxdoc}
  \ifx\doneclasses\mdw@undefined\else\doneclasses\fi
\fi
%    \end{macrocode}
%
% As part of my standard environment, I'll load some of my more useful
% packages.  If they're already loaded (possibly with different options),
% I'll not try to load them again.
%
%    \begin{macrocode}
\@ifpackageloaded{doc}{}{\usepackage{doc}}
\@ifpackageloaded{syntax}{}{\usepackage[rounded]{syntax}}
\@ifpackageloaded{sverb}{}{\usepackage{sverb}}
%    \end{macrocode}
%
%
% \subsection{Some macros for interaction}
%
% I like the \LaTeX\ star-boxes, although it's a pain having to cope with
% \TeX's space-handling rules.  I'll define a new typing-out macro which
% makes spaces more significant, and has a $*$-version which doesn't put
% a newline on the end, and interacts prettily with |\read|.
%
% First of all, I need to make spaces active, so I can define things about
% active spaces.
%
%    \begin{macrocode}
\begingroup\obeyspaces
%    \end{macrocode}
%
% Now to define the main macro.  This is easy stuff.  Spaces must be
% carefully rationed here, though.
%
% I'll start a group, make spaces active, and make spaces expand to ordinary
% space-like spaces.  Then I'll look for a star, and pass either |\message|
% (which doesn't start a newline, and interacts with |\read| well) or
% |\immediate\write 16| which does a normal write well.
%
%    \begin{macrocode}
\gdef\mdwtype{%
\begingroup\catcode`\ \active\let \space%
\@ifstar{\mdwtype@i{\message}}{\mdwtype@i{\immediate\write\sixt@@n}}%
}
\endgroup
%    \end{macrocode}
%
% Now for the easy bit.  I have the thing to do, and the thing to do it to,
% so do that and end the group.
%
%    \begin{macrocode}
\def\mdwtype@i#1#2{#1{#2}\endgroup}
%    \end{macrocode}
%
%
% \subsection{Decide on indexing}
%
% A configuration file can decide on indexing by defining the |\indexing|
% macro to either \lit{y} or \lit{n}.  If it's not set, then I'll prompt
% the user.
%
% First of all, I want a switch to say whether I'm indexing.
%
%    \begin{macrocode}
\newif\ifcreateindex
%    \end{macrocode}
%
% Right: now I need to decide how to make progress.  If the macro's not set,
% then I want to set it, and start a row of stars.
%
%    \begin{macrocode}
\ifx\indexing\@@undefined
  \mdwtype{*****************************}
  \def\indexing{?}
\fi
%    \end{macrocode}
%
% Now enter a loop, asking the user whether to do indexing, until I get
% a sensible answer.
%
%    \begin{macrocode}
\loop
  \@tempswafalse
  \if y\indexing\@tempswatrue\createindextrue\fi
  \if Y\indexing\@tempswatrue\createindextrue\fi
  \if n\indexing\@tempswatrue\createindexfalse\fi
  \if N\indexing\@tempswatrue\createindexfalse\fi
  \if@tempswa\else
  \mdwtype*{* Create index files? (y/n) *}
  \read\sixt@@n to\indexing%
\repeat
%    \end{macrocode}
%
% Now, based on the results of that, display a message about the indexing.
%
%    \begin{macrocode}
\mdwtype{*****************************}
\ifcreateindex
  \mdwtype{* Creating index files      *}
  \mdwtype{* This may take some time   *}
\else
  \mdwtype{* Not creating index files  *}
\fi
\mdwtype{*****************************}
%    \end{macrocode}
%
% Now I can play with the indexing commands of the \package{doc} package
% to do whatever it is that the user wants.
%
%    \begin{macrocode}
\ifcreateindex
  \CodelineIndex
  \EnableCrossrefs
\else
  \CodelineNumbered
  \DisableCrossrefs
\fi
%    \end{macrocode}
%
% And register lots of plain \TeX\ things which shouldn't be indexed.
% This contains lots of |\if|\dots\ things which don't fit nicely in
% conditionals, which is a shame.  Still, it doesn't matter that much,
% really.
%
%    \begin{macrocode}
\DoNotIndex{\def,\long,\edef,\xdef,\gdef,\let,\global}
\DoNotIndex{\if,\ifnum,\ifdim,\ifcat,\ifmmode,\ifvmode,\ifhmode,%
            \iftrue,\iffalse,\ifvoid,\ifx,\ifeof,\ifcase,\else,\or,\fi}
\DoNotIndex{\box,\copy,\setbox,\unvbox,\unhbox,\hbox,%
            \vbox,\vtop,\vcenter}
\DoNotIndex{\@empty,\immediate,\write}
\DoNotIndex{\egroup,\bgroup,\expandafter,\begingroup,\endgroup}
\DoNotIndex{\divide,\advance,\multiply,\count,\dimen}
\DoNotIndex{\relax,\space,\string}
\DoNotIndex{\csname,\endcsname,\@spaces,\openin,\openout,%
            \closein,\closeout}
\DoNotIndex{\catcode,\endinput}
\DoNotIndex{\jobname,\message,\read,\the,\m@ne,\noexpand}
\DoNotIndex{\hsize,\vsize,\hskip,\vskip,\kern,\hfil,\hfill,\hss}
\DoNotIndex{\m@ne,\z@,\z@skip,\@ne,\tw@,\p@}
\DoNotIndex{\dp,\wd,\ht,\vss,\unskip}
%    \end{macrocode}
%
% Last bit of indexing stuff, for now: I'll typeset the index in two columns
% (the default is three, which makes them too narrow for my tastes).
%
%    \begin{macrocode}
\setcounter{IndexColumns}{2}
%    \end{macrocode}
%
%
% \subsection{Selectively defining things}
%
% I don't want to tread on anyone's toes if they redefine any of these
% commands and things in a configuration file.  The following definitions
% are fairly evil, but should do the job OK.
%
% \begin{macro}{\@gobbledef}
%
% This macro eats the following |\def|inition, leaving not a trace behind.
%
%    \begin{macrocode}
\def\@gobbledef#1#{\@gobble}
%    \end{macrocode}
%
% \end{macro}
%
% \begin{macro}{\tdef}
% \begin{macro}{\tlet}
%
% The |\tdef| command is a sort of `tentative' definition -- it's like
% |\def| if the control sequence named doesn't already have a definition.
% |\tlet| does the same thing with |\let|.
%
%    \begin{macrocode}
\def\tdef#1{
  \ifx#1\@@undefined%
    \expandafter\def\expandafter#1%
  \else%
    \expandafter\@gobbledef%
  \fi%
}
\def\tlet#1#2{\ifx#1\@@undefined\let#1=#2\fi}
%    \end{macrocode}
%
% \end{macro}
% \end{macro}
%
%
% \subsection{General markup things}
%
% Now for some really simple things.  I'll define how to typeset package
% names and environment names (both in the sans serif font, for now).
%
%    \begin{macrocode}
\tlet\package\textsf
\tlet\env\textsf
%    \end{macrocode}
%
% I'll define the |\<|\dots|>| shortcut for syntax items suggested in the
% \package{syntax} package.
%
%    \begin{macrocode}
\tdef\<#1>{\synt{#1}}
%    \end{macrocode}
%
% And because it's used in a few places (mainly for typesetting lengths),
% here's a command for typesetting fractions in text.
%
%    \begin{macrocode}
\tdef\smallf#1/#2{\ensuremath{^{#1}\!/\!_{#2}}}
%    \end{macrocode}
%
%
% \subsection{A table environment}
%
% \begin{environment}{tab}
%
% Most of the packages don't use the (obviously perfect) \package{mdwtab}
% package, because it's big, and takes a while to load.  Here's an
% environment for typesetting centred tables.  The first (optional) argument
% is some declarations to perform.  The mandatory argument is the table
% preamble (obviously).
%
%    \begin{macrocode}
\@ifundefined{tab}{%
  \newenvironment{tab}[2][\relax]{%
    \par\vskip2ex%
    \centering%
    #1%
    \begin{tabular}{#2}%
  }{%
    \end{tabular}%
    \par\vskip2ex%
  }
}{}
%    \end{macrocode}
%
% \end{environment}
%
%
% \subsection{Commenting out of stuff}
%
% \begin{environment}{meta-comment}
%
% Using |\iffalse|\dots|\fi| isn't much fun.  I'll define a gobbling
% environment using the \package{sverb} stuff.
%
%    \begin{macrocode}
\ignoreenv{meta-comment}
%    \end{macrocode}
%
% \end{environment}
%
%
% \subsection{Float handling}
%
% This gubbins will try to avoid float pages as much as possible, and (with
% any luck) encourage floats to be put on the same pages as text.
%
%    \begin{macrocode}
\def\textfraction{0.1}
\def\topfraction{0.9}
\def\bottomfraction{0.9}
\def\floatpagefraction{0.7}
%    \end{macrocode}
%
% Now redefine the default float-placement parameters to allow `here' floats.
%
%    \begin{macrocode}
\def\fps@figure{htbp}
\def\fps@table{htbp}
%    \end{macrocode}
%
%
% \subsection{Other bits of parameter tweaking}
%
% Make \env{grammar} environments look pretty, by indenting the left hand
% sides by a large amount.
%
%    \begin{macrocode}
\grammarindent1in
%    \end{macrocode}
%
% I don't like being told by \TeX\ that my paragraphs are hard to linebreak:
% I know this already.  This lot should shut \TeX\ up about most problems.
%
%    \begin{macrocode}
\sloppy
\hbadness\@M
\hfuzz10\p@
%    \end{macrocode}
%
% Also make \TeX\ shut up in the index.  The \package{multicol} package
% irritatingly plays with |\hbadness|.  This is the best hook I could find
% for playing with this setting.
%
%    \begin{macrocode}
\expandafter\def\expandafter\IndexParms\expandafter{%
  \IndexParms%
  \hbadness\@M%
}
%    \end{macrocode}
%
% The other thing I really don't like is `Marginpar moved' warnings.  This
% will get rid of them, and lots of other \LaTeX\ warnings at the same time.
%
%    \begin{macrocode}
\let\@latex@warning@no@line\@gobble
%    \end{macrocode}
%
% Put some extra space between table rows, please.
%
%    \begin{macrocode}
\def\arraystretch{1.2}
%    \end{macrocode}
%
% Most of the code is at guard level one, so typeset that in upright text.
%
%    \begin{macrocode}
\setcounter{StandardModuleDepth}{1}
%    \end{macrocode}
%
%
% \subsection{Contents handling}
%
% I use at least one contents file (the main table of contents) although
% I may want more.  I'll keep a list of contents files which I need to
% handle.
%
% There are two things I need to do to contents files here:
% \begin{itemize}
% \item I must typeset the table of contents at the beginning of the
%       document; and
% \item I want to typeset tables of contents in two columns (using the
%       \package{multicol} package).
% \end{itemize}
%
% The list consists of items of the form
% \syntax{"\\do{"<extension>"}{"<command>"}"}, where \<extension> is the
% file extension of the contents file, and \<command> is the command to
% typeset it.
%
% \begin{macro}{\docontents}
%
% This is where I keep the list of contents files.  I'll initialise it to
% just do the standard contents table.
%
%    \begin{macrocode}
\def\docontents{\do{toc}{\tableofcontents}}
%    \end{macrocode}
%
% \end{macro}
%
% \begin{macro}{\addcontents}
%
% By saying \syntax{"\\addcontents{"<extension>"}{"<command>"}"}, a document
% can register a new table of contents which gets given the two-column
% treatment properly.  This is really easy to implement.
%
%    \begin{macrocode}
\def\addcontents#1#2{%
  \toks@\expandafter{\docontents\do{#1}{#2}}%
  \edef\docontents{\the\toks@}%
}
%    \end{macrocode}
%
% \end{macro}
%
%
% \subsection{Finishing it all off}
%
% \begin{macro}{\finalstuff}
%
% The |\finalstuff| macro is a hook for doing things at the end of the
% document.  Currently, it inputs the licence agreement as an appendix.
%
%    \begin{macrocode}
\tdef\finalstuff{\appendix\part*{Appendix}\input{gpl}}
%    \end{macrocode}
%
% \end{macro}
%
% \begin{macro}{\implementation}
%
% The |\implementation| macro starts typesetting the implementation of
% the package(s).  If we're not doing the implementation, it just does
% this lot and ends the input file.
%
% I define a macro with arguments inside the |\StopEventually|, which causes
% problems, since the code gets put through an extra level of |\def|fing
% depending on whether the implementation stuff gets typeset or not.  I'll
% store the code I want to do in a separate macro.
%
%    \begin{macrocode}
\def\implementation{\StopEventually{\attheend}}
%    \end{macrocode}
%
% Now for the actual activity.  First, I'll do the |\finalstuff|.  Then, if
% \package{doc}'s managed to find the \package{multicol} package, I'll add
% the end of the environment to the end of each contents file in the list.
% Finally, I'll read the index in from its formatted |.ind| file.
%
%    \begin{macrocode}
\tdef\attheend{%
  \finalstuff%
  \ifhave@multicol%
    \def\do##1##2{\addtocontents{##1}{\protect\end{multicols}}}%
    \docontents%
  \fi%
  \PrintIndex%
}
%    \end{macrocode}
%
% \end{macro}
%
%
% \subsection{File version information}
%
% \begin{macro}{\mdwpkginfo}
%
% For setting up the automatic titles, I'll need to be able to work out
% file versions and things.  This macro will, given a file name, extract
% from \LaTeX\ the version information and format it into a sensible string.
%
% First of all, I'll put the original string (direct from the
% |\Provides|\dots\ command).  Then I'll pass it to another macro which can
% parse up the string into its various bits, along with the original
% filename.
%
%    \begin{macrocode}
\def\mdwpkginfo#1{%
  \edef\@tempa{\csname ver@#1\endcsname}%
  \expandafter\mdwpkginfo@i\@tempa\@@#1\@@%
}
%    \end{macrocode}
%
% Now for the real business.  I'll store the string I build in macros called
% \syntax{"\\"<filename>"date", "\\"<filename>"version" and
% "\\"<filename>"info"}, which store the file's date, version and
% `information string' respectively.  (Note that the file extension isn't
% included in the name.)
%
% This is mainly just tedious playing with |\expandafter|.  The date format
% is defined by a separate macro, which can be modified from the
% configuration file.
%
%    \begin{macrocode}
\def\mdwpkginfo@i#1/#2/#3 #4 #5\@@#6.#7\@@{%
  \expandafter\def\csname #6date\endcsname%
    {\protect\mdwdateformat{#1}{#2}{#3}}%
  \expandafter\def\csname #6version\endcsname{#4}%
  \expandafter\def\csname #6info\endcsname{#5}%
}
%    \end{macrocode}
%
% \end{macro}
%
% \begin{macro}{\mdwdateformat}
%
% Given three arguments, a year, a month and a date (all numeric), build a
% pretty date string.  This is fairly simple really.
%
%    \begin{macrocode}
\tdef\mdwdateformat#1#2#3{\number#3\ \monthname{#2}\ \number#1}
\def\monthname#1{%
  \ifcase#1\or%
     January\or February\or March\or April\or May\or June\or%
     July\or August\or September\or October\or November\or December%
  \fi%
}
\def\numsuffix#1{%
  \ifnum#1=1 st\else%
  \ifnum#1=2 nd\else%
  \ifnum#1=3 rd\else%
  \ifnum#1=21 st\else%
  \ifnum#1=22 nd\else%
  \ifnum#1=23 rd\else%
  \ifnum#1=31 st\else%
  th%
  \fi\fi\fi\fi\fi\fi\fi%
}
%    \end{macrocode}
%
% \end{macro}
%
% \begin{macro}{\mdwfileinfo}
%
% Saying \syntax{"\\mdwfileinfo{"<file-name>"}{"<info>"}"} extracts the
% wanted item of \<info> from the version information for file \<file-name>.
%
%    \begin{macrocode}
\def\mdwfileinfo#1#2{\mdwfileinfo@i{#2}#1.\@@}
\def\mdwfileinfo@i#1#2.#3\@@{\csname#2#1\endcsname}
%    \end{macrocode}
%
% \end{macro}
%
%
% \subsection{List handling}
%
% There are several other lists I need to build.  These macros will do
% the necessary stuff.
%
% \begin{macro}{\mdw@ifitem}
%
% The macro \syntax{"\\mdw@ifitem"<item>"\\in"<list>"{"<true-text>"}"^^A
% "{"<false-text>"}"} does \<true-text> if the \<item> matches any item in
% the \<list>; otherwise it does \<false-text>.
%
%    \begin{macrocode}
\def\mdw@ifitem#1\in#2{%
  \@tempswafalse%
  \def\@tempa{#1}%
  \def\do##1{\def\@tempb{##1}\ifx\@tempa\@tempb\@tempswatrue\fi}%
  #2%
  \if@tempswa\expandafter\@firstoftwo\else\expandafter\@secondoftwo\fi%
}
%    \end{macrocode}
%
% \end{macro}
%
% \begin{macro}{\mdw@append}
%
% Saying \syntax{"\\mdw@append"<item>"\\to"<list>} adds the given \<item>
% to the end of the given \<list>.
%
%    \begin{macrocode}
\def\mdw@append#1\to#2{%
  \toks@{\do{#1}}%
  \toks\tw@\expandafter{#2}%
  \edef#2{\the\toks\tw@\the\toks@}%
}
%    \end{macrocode}
%
% \end{macro}
%
% \begin{macro}{\mdw@prepend}
%
% Saying \syntax{"\\mdw@prepend"<item>"\\to"<list>} adds the \<item> to the
% beginning of the \<list>.
%
%    \begin{macrocode}
\def\mdw@prepend#1\to#2{%
  \toks@{\do{#1}}%
  \toks\tw@\expandafter{#2}%
  \edef#2{\the\toks@\the\toks\tw@}%
}
%    \end{macrocode}
%
% \end{macro}
%
% \begin{macro}{\mdw@add}
%
% Finally, saying \syntax{"\\mdw@add"<item>"\\to"<list>} adds the \<item>
% to the list only if it isn't there already.
%
%    \begin{macrocode}
\def\mdw@add#1\to#2{\mdw@ifitem#1\in#2{}{\mdw@append#1\to#2}}
%    \end{macrocode}
%
% \end{macro}
%
%
% \subsection{Described file handling}
%
% I'l maintain lists of packages, document classes, and other files
% described by the current documentation file.
%
% First of all, I'll declare the various list macros.
%
%    \begin{macrocode}
\def\dopackages{}
\def\doclasses{}
\def\dootherfiles{}
%    \end{macrocode}
%
% \begin{macro}{\describespackage}
%
% A document file can declare that it describes a package by saying
% \syntax{"\\describespackage{"<package-name>"}"}.  I add the package to
% my list, read the package into memory (so that the documentation can
% offer demonstrations of it) and read the version information.
%
%    \begin{macrocode}
\def\describespackage#1{%
  \mdw@ifitem#1\in\dopackages{}{%
    \mdw@append#1\to\dopackages%
    \usepackage{#1}%
    \mdwpkginfo{#1.sty}%
  }%
}
%    \end{macrocode}
%
% \end{macro}
%
% \begin{macro}{\describesclass}
%
% By saying \syntax{"\\describesclass{"<class-name>"}"}, a document file
% can declare that it describes a document class.  I'll assume that the
% document class is already loaded, because it's much too late to load
% it now.
%
%    \begin{macrocode}
\def\describesclass#1{\mdw@add#1\to\doclasses\mdwpkginfo{#1.cls}}
%    \end{macrocode}
%
% \end{macro}
%
% \begin{macro}{\describesfile}
%
% Finally, other `random' files, which don't have the status of real \LaTeX\
% packages or document classes, can be described by saying \syntax{^^A
% "\\describesfile{"<file-name>"}" or "\\describesfile*{"<file-name>"}"}.
% The difference is that the starred version will not |\input| the file.
%
%    \begin{macrocode}
\def\describesfile{%
  \@ifstar{\describesfile@i\@gobble}{\describesfile@i\input}%
}
\def\describesfile@i#1#2{%
  \mdw@ifitem#2\in\dootherfiles{}{%
    \mdw@add#2\to\dootherfiles%
    #1{#2}%
    \mdwpkginfo{#2}%
  }%
}
%    \end{macrocode}
%
% \end{macro}
%
%
% \subsection{Author and title handling}
%
% I'll redefine the |\author| and |\title| commands so that I get told
% whether I need to do it myself.
%
% \begin{macro}{\author}
%
% This is easy: I'll save the old meaning, and then redefine |\author| to
% do the old thing and redefine itself to then do nothing.
%
%    \begin{macrocode}
\let\mdw@author\author
\def\author{\let\author\@gobble\mdw@author}
%    \end{macrocode}
%
% \end{macro}
%
% \begin{macro}{\title}
%
% And oddly enough, I'll do exactly the same thing for the title, except
% that I'll also disable the |\mdw@buildtitle| command, which constructs
% the title automatically.
%
%    \begin{macrocode}
\let\mdw@title\title
\def\title{\let\title\@gobble\let\mdw@buildtitle\relax\mdw@title}
%    \end{macrocode}
%
% \end{macro}
%
% \begin{macro}{\date}
%
% This works in a very similar sort of way.
%
%    \begin{macrocode}
\def\date#1{\let\date\@gobble\def\today{#1}}
%    \end{macrocode}
%
% \end{macro}
%
% \begin{macro}{\datefrom}
%
% Saying \syntax{"\\datefrom{"<file-name>"}"} sets the document date from
% the given filename.
%
%    \begin{macrocode}
\def\datefrom#1{%
  \protected@edef\@tempa{\noexpand\date{\csname #1date\endcsname}}%
  \@tempa%
}
%    \end{macrocode}
%
% \end{macro}
%
% \begin{macro}{\docfile}
%
% Saying \syntax{"\\docfile{"<file-name>"}"} sets up the file name from which
% documentation will be read.
%
%    \begin{macrocode}
\def\docfile#1{%
  \def\@tempa##1.##2\@@{\def\@basefile{##1.##2}\def\@basename{##1}}%
  \edef\@tempb{\noexpand\@tempa#1\noexpand\@@}%
  \@tempb%
}
%    \end{macrocode}
%
% I'll set up a default value as well.
%
%    \begin{macrocode}
\docfile{\jobname.dtx}
%    \end{macrocode}
%
% \end{macro}
%
%
% \subsection{Building title strings}
%
% This is rather tricky.  For each list, I need to build a legible looking
% string.
%
% \begin{macro}{\mdw@addtotitle}
%
% By saying
%\syntax{"\\mdw@addtotitle{"<list>"}{"<command>"}{"<singular>"}{"<plural>"}"}
% I can add the contents of a list to the current title string in the
% |\mdw@title| macro.
%
%    \begin{macrocode}
\tdef\mdw@addtotitle#1#2#3#4{%
%    \end{macrocode}
%
% Now to get to work.  I need to keep one `lookahead' list item, and a count
% of the number of items read so far.  I'll keep the lookahead item in
% |\@nextitem| and the counter in |\count@|.
%
%    \begin{macrocode}
  \count@\z@%
%    \end{macrocode}
%
% Now I'll define what to do for each list item.  The |\protect| command is
% already set up appropriately for playing with |\edef| commands.
%
%    \begin{macrocode}
  \def\do##1{%
%    \end{macrocode}
%
% The first job is to add the previous item to the title string.  If this
% is the first item, though, I'll just add the appropriate \lit{The } or
% \lit{ and the } string to the title (this is stored in the |\@prefix|
% macro).
%
%    \begin{macrocode}
    \edef\mdw@title{%
      \mdw@title%
      \ifcase\count@\@prefix%
      \or\@nextitem%
      \else, \@nextitem%
      \fi%
    }%
%    \end{macrocode}
%
% That was rather easy.  Now I'll set up the |\@nextitem| macro for the
% next time around the loop.
%
%    \begin{macrocode}
    \edef\@nextitem{%
      \protect#2{##1}%
      \protect\footnote{%
        The \protect#2{##1} #3 is currently at version %
        \mdwfileinfo{##1}{version}, dated \mdwfileinfo{##1}{date}.%
      }\space%
    }%
%    \end{macrocode}
%
% Finally, I need to increment the counter.
%
%    \begin{macrocode}
    \advance\count@\@ne%
  }%
%    \end{macrocode}
%
% Now execute the list.
%
%    \begin{macrocode}
  #1%
%    \end{macrocode}
%
% I still have one item left over, unless the list was empty.  I'll add
% that now.
%
%    \begin{macrocode}
  \edef\mdw@title{%
    \mdw@title%
    \ifcase\count@%
    \or\@nextitem\space#3%
    \or\ and \@nextitem\space#4%
    \fi%
  }%
%    \end{macrocode}
%
% Finally, if $|\count@| \ne 0$, I must set |\@prefix| to \lit{ and the }.
%
%    \begin{macrocode}
  \ifnum\count@>\z@\def\@prefix{ and the }\fi%
}
%    \end{macrocode}
%
% \end{macro}
%
% \begin{macro}{\mdw@buildtitle}
%
% This macro will actually do the job of building the title string.
%
%    \begin{macrocode}
\tdef\mdw@buildtitle{%
%    \end{macrocode}
%
% First of all, I'll open a group to avoid polluting the namespace with
% my gubbins (although the code is now much tidier than it has been in
% earlier releases).
%
%    \begin{macrocode}
  \begingroup%
%    \end{macrocode}
%
% The title building stuff makes extensive use of |\edef|.  I'll set
% |\protect| appropriately.  (For those not in the know,
% |\@unexpandable@protect| expands to `|\noexpand\protect\noexpand|',
% which prevents expansion of the following macro, and inserts a |\protect|
% in front of it ready for the next |\edef|.)
%
%    \begin{macrocode}
  \let\@@protect\protect\let\protect\@unexpandable@protect%
%    \end{macrocode}
%
% Set up some simple macros ready for the main code.
%
%    \begin{macrocode}
  \def\mdw@title{}%
  \def\@prefix{The }%
%    \end{macrocode}
%
% Now build the title.  This is fun.
%
%    \begin{macrocode}
  \mdw@addtotitle\dopackages\package{package}{packages}%
  \mdw@addtotitle\doclasses\package{document class}{document classes}%
  \mdw@addtotitle\dootherfiles\texttt{file}{files}%
%    \end{macrocode}
%
% Now I want to end the group and set the title from my string.  The
% following hacking will do this.
%
%    \begin{macrocode}
  \edef\next{\endgroup\noexpand\title{\mdw@title}}%
  \next%
}
%    \end{macrocode}
%
% \end{macro}
%
%
% \subsection{Starting the main document}
%
% \begin{macro}{\mdwdoc}
%
% Once the document preamble has done all of its stuff, it calls the
% |\mdwdoc| command, which takes over and really starts the documentation
% going.
%
%    \begin{macrocode}
\def\mdwdoc{%
%    \end{macrocode}
%
% First, I'll construct the title string.
%
%    \begin{macrocode}
  \mdw@buildtitle%
  \author{Mark Wooding}%
%    \end{macrocode}
%
% Set up the date string based on the date of the package which shares
% the same name as the current file.
%
%    \begin{macrocode}
  \datefrom\@basename%
%    \end{macrocode}
%
% Set up verbatim characters after all the packages have started.
%
%    \begin{macrocode}
  \shortverb\|%
  \shortverb\"%
%    \end{macrocode}
%
% Start the document, and put the title in.
%
%    \begin{macrocode}
  \begin{document}
  \maketitle%
%    \end{macrocode}
%
% This is nasty.  It makes maths displays work properly in demo environments.
% \emph{The \LaTeX\ Companion} exhibits the bug which this hack fixes.  So
% ner.
%
%    \begin{macrocode}
  \abovedisplayskip\z@%
%    \end{macrocode}
%
% Now start the contents tables.  After starting each one, I'll make it
% be multicolumnar.
%
%    \begin{macrocode}
  \def\do##1##2{%
    ##2%
    \ifhave@multicol\addtocontents{##1}{%
      \protect\begin{multicols}{2}%
      \hbadness\@M%
    }\fi%
  }%
  \docontents%
%    \end{macrocode}
%
% Input the main file now.
%
%    \begin{macrocode}
  \DocInput{\@basefile}%
%    \end{macrocode}
%
% That's it.  I'm done.
%
%    \begin{macrocode}
  \end{document}
}
%    \end{macrocode}
%
% \end{macro}
%
%
% \subsection{And finally\dots}
%
% Right at the end I'll put a hook for the configuration file.
%
%    \begin{macrocode}
\ifx\mdwhook\@@undefined\else\expandafter\mdwhook\fi
%    \end{macrocode}
%
% That's all the code done now.  I'll change back to `user' mode, where
% all the magic control sequences aren't allowed any more.
%
%    \begin{macrocode}
\makeatother
%</mdwtools>
%    \end{macrocode}
%
% Oh, wait!  What if I want to typeset this documentation?  Aha.  I'll cope
% with that by comparing |\jobname| with my filename |mdwtools|.  However,
% there's some fun here, because |\jobname| contains category-12 letters,
% while my letters are category-11.  Time to play with |\string| in a messy
% way.
%
%    \begin{macrocode}
%<*driver>
\makeatletter
\edef\@tempa{\expandafter\@gobble\string\mdwtools}
\edef\@tempb{\jobname}
\ifx\@tempa\@tempb
  \describesfile*{mdwtools.tex}
  \docfile{mdwtools.tex}
  \makeatother
  \expandafter\mdwdoc
\fi
\makeatother
%</driver>
%    \end{macrocode}
%
% That's it.  Done!
%
% \hfill Mark Wooding, \today
%
% \Finale
%
\endinput
|
% \<declarations>
% |\mdwdoc|
% \end{listinglist}
% The initial |\input| reads in this file and sets up the various commands
% which may be needed.  The final |\mdwdoc| actually starts the document,
% inserting a title (which is automatically generated), a table of
% contents etc., and reads the documentation file in (using the |\DocInput|
% command from the \package{doc} package.
%
% \subsubsection{Describing packages}
%
% \DescribeMacro{\describespackage}
% \DescribeMacro{\describesclass}
% \DescribeMacro{\describesfile}
% \DescribeMacro{\describesfile*}
% The most important declarations are those which declare what the
% documentation describes.  Saying \syntax{"\\describespackage{<package>}"}
% loads the \<package> (if necessary) and adds it to the auto-generated
% title, along with a footnote containing version information.  Similarly,
% |\describesclass| adds a document class name to the title (without loading
% it -- the document itself must do this, with the |\documentclass| command).
% For files which aren't packages or classes, use the |\describesfile| or
% |\describesfile*| command (the $*$-version won't |\input| the file, which
% is handy for files like |mdwtools.tex|, which are already input).
%
% \DescribeMacro{\author}
% \DescribeMacro{\date}
% \DescribeMacro{\title}
% The |\author|, |\date| and |\title| declarations work slightly differently
% to normal -- they ensure that only the \emph{first} declaration has an
% effect.  (Don't you play with |\author|, please, unless you're using this
% program to document your own packages.)  Using |\title| suppresses the
% automatic title generation.
%
% \DescribeMacro{\docdate}
% The default date is worked out from the version string of the package or
% document class whose name is the same as that of the documentation file.
% You can choose a different `main' file by saying
% \syntax{"\\docdate{"<file>"}"}.
%
% \subsubsection{Contents handling}
%
% \DescribeMacro{\addcontents}
% A documentation file always has a table of contents.  Other
% contents-like lists can be added by saying
% \syntax{"\\addcontents{"<extension>"}{"<command>"}"}.  The \<extension>
% is the file extension of the contents file (e.g., \lit{lot} for the
% list of tables); the \<command> is the command to actually typeset the
% contents file (e.g., |\listoftables|).
%
% \subsubsection{Other declarations}
%
% \DescribeMacro{\implementation}
% The \package{doc} package wants you to say
% \syntax{"\\StopEventually{"<stuff>"}"}' before describing the package
% implementation.  Using |mdwtools.tex|, you just say |\implementation|, and
% everything works.  It will automatically read in the licence text (from
% |gpl.tex|, and wraps some other things up.
%
% 
% \subsection{Other commands}
%
% The |mdwtools.tex| file includes the \package{syntax} and \package{sverb}
% packages so that they can be used in documentation files.  It also defines
% some trivial commands of its own.
%
% \DescribeMacro{\<}
% Saying \syntax{"\\<"<text>">" is the same as "\\synt{"<text>"}"}; this
% is a simple abbreviation.
%
% \DescribeMacro{\smallf}
% Saying \syntax{"\\smallf" <number>"/"<number>} typesets a little fraction,
% like this: \smallf 3/4.  It's useful when you want to say that the default
% value of a length is 2 \smallf 1/2\,pt, or something like that.
%
%
% \subsection{Customisation}
%
% You can customise the way that the package documentation looks by writing
% a file called |mdwtools.cfg|.  You can redefine various commands (before
% they're defined here, even; |mdwtools.tex| checks most of the commands that
% it defines to make sure they haven't been defined already.
%
% \DescribeMacro{\indexing}
% If you don't want the prompt about whether to generate index files, you
% can define the |\indexing| command to either \lit{y} or \lit{n}.  I'd
% recommend that you use |\providecommand| for this, to allow further
% customisation from the command line.
%
% \DescribeMacro{\mdwdateformat}
% If you don't like my date format (maybe you're American or something),
% you can redefine the |\mdwdateformat| command.  It takes three arguments:
% the year, month and date, as numbers; it should expand to something which
% typesets the date nicely.  The default format gives something like
% `10 May 1996'.  You can produce something rather more exotic, like
% `10\textsuperscript{th} May \textsc{\romannumeral 1996}' by saying
%\begin{listing}
%\newcommand{\mdwdateformat}[3]{%
%  \number#3\textsuperscript{\numsuffix{#3}}\ %
%  \monthname{#2}\ %
%  \textsc{\romannumeral #1}%
%}
%\end{listing}
% \DescribeMacro{\monthname}
% \DescribeMacro{\numsuffix}
% Saying \syntax{"\\monthname{"<number>"}"} expands to the name of the
% numbered month (which can be useful when doing date formats).  Saying
% \syntax{"\\numsuffix{"<number>"}"} will expand to the appropriate suffix
% (`th' or `rd' or whatever) for the \<number>.  You'll have to superscript
% it yourself, if this is what you want to do.  Putting the year number
% in roman numerals is just pretentious |;-)|.
%
% \DescribeMacro{\mdwhook}
% After all the declarations in |mdwtools.tex|, the command |\mdwhook| is
% executed, if it exists.  This can be set up by the configuration file
% to do whatever you want.
%
% There are lots of other things you can play with; you should look at the
% implementation section to see what's possible.
%
% \implementation
%
% \section{Implementation}
%
%    \begin{macrocode}
%<*mdwtools>
%    \end{macrocode}
%
% The first thing is that I'm not a \LaTeX\ package or anything official
% like that, so I must enable `|@|' as a letter by hand.
%
%    \begin{macrocode}
\makeatletter
%    \end{macrocode}
%
% Now input the user's configuration file, if it exists.  This is fairly
% simple stuff.
%
%    \begin{macrocode}
\@input{mdwtools.cfg}
%    \end{macrocode}
%
% Well, that's the easy bit done.
%
%
% \subsection{Initialisation}
%
% Obviously the first thing to do is to obtain a document class.  Obviously,
% it would be silly to do this if a document class has already been loaded,
% either by the package documentation or by the configuration file.
%
% The only way I can think of for finding out if a document class is already
% loaded is by seeing if the |\documentclass| command has been redefined
% to raise an error.  This isn't too hard, really.
%
%    \begin{macrocode}
\ifx\documentclass\@twoclasseserror\else
  \documentclass[a4paper]{ltxdoc}
  \ifx\doneclasses\mdw@undefined\else\doneclasses\fi
\fi
%    \end{macrocode}
%
% As part of my standard environment, I'll load some of my more useful
% packages.  If they're already loaded (possibly with different options),
% I'll not try to load them again.
%
%    \begin{macrocode}
\@ifpackageloaded{doc}{}{\usepackage{doc}}
\@ifpackageloaded{syntax}{}{\usepackage[rounded]{syntax}}
\@ifpackageloaded{sverb}{}{\usepackage{sverb}}
%    \end{macrocode}
%
%
% \subsection{Some macros for interaction}
%
% I like the \LaTeX\ star-boxes, although it's a pain having to cope with
% \TeX's space-handling rules.  I'll define a new typing-out macro which
% makes spaces more significant, and has a $*$-version which doesn't put
% a newline on the end, and interacts prettily with |\read|.
%
% First of all, I need to make spaces active, so I can define things about
% active spaces.
%
%    \begin{macrocode}
\begingroup\obeyspaces
%    \end{macrocode}
%
% Now to define the main macro.  This is easy stuff.  Spaces must be
% carefully rationed here, though.
%
% I'll start a group, make spaces active, and make spaces expand to ordinary
% space-like spaces.  Then I'll look for a star, and pass either |\message|
% (which doesn't start a newline, and interacts with |\read| well) or
% |\immediate\write 16| which does a normal write well.
%
%    \begin{macrocode}
\gdef\mdwtype{%
\begingroup\catcode`\ \active\let \space%
\@ifstar{\mdwtype@i{\message}}{\mdwtype@i{\immediate\write\sixt@@n}}%
}
\endgroup
%    \end{macrocode}
%
% Now for the easy bit.  I have the thing to do, and the thing to do it to,
% so do that and end the group.
%
%    \begin{macrocode}
\def\mdwtype@i#1#2{#1{#2}\endgroup}
%    \end{macrocode}
%
%
% \subsection{Decide on indexing}
%
% A configuration file can decide on indexing by defining the |\indexing|
% macro to either \lit{y} or \lit{n}.  If it's not set, then I'll prompt
% the user.
%
% First of all, I want a switch to say whether I'm indexing.
%
%    \begin{macrocode}
\newif\ifcreateindex
%    \end{macrocode}
%
% Right: now I need to decide how to make progress.  If the macro's not set,
% then I want to set it, and start a row of stars.
%
%    \begin{macrocode}
\ifx\indexing\@@undefined
  \mdwtype{*****************************}
  \def\indexing{?}
\fi
%    \end{macrocode}
%
% Now enter a loop, asking the user whether to do indexing, until I get
% a sensible answer.
%
%    \begin{macrocode}
\loop
  \@tempswafalse
  \if y\indexing\@tempswatrue\createindextrue\fi
  \if Y\indexing\@tempswatrue\createindextrue\fi
  \if n\indexing\@tempswatrue\createindexfalse\fi
  \if N\indexing\@tempswatrue\createindexfalse\fi
  \if@tempswa\else
  \mdwtype*{* Create index files? (y/n) *}
  \read\sixt@@n to\indexing%
\repeat
%    \end{macrocode}
%
% Now, based on the results of that, display a message about the indexing.
%
%    \begin{macrocode}
\mdwtype{*****************************}
\ifcreateindex
  \mdwtype{* Creating index files      *}
  \mdwtype{* This may take some time   *}
\else
  \mdwtype{* Not creating index files  *}
\fi
\mdwtype{*****************************}
%    \end{macrocode}
%
% Now I can play with the indexing commands of the \package{doc} package
% to do whatever it is that the user wants.
%
%    \begin{macrocode}
\ifcreateindex
  \CodelineIndex
  \EnableCrossrefs
\else
  \CodelineNumbered
  \DisableCrossrefs
\fi
%    \end{macrocode}
%
% And register lots of plain \TeX\ things which shouldn't be indexed.
% This contains lots of |\if|\dots\ things which don't fit nicely in
% conditionals, which is a shame.  Still, it doesn't matter that much,
% really.
%
%    \begin{macrocode}
\DoNotIndex{\def,\long,\edef,\xdef,\gdef,\let,\global}
\DoNotIndex{\if,\ifnum,\ifdim,\ifcat,\ifmmode,\ifvmode,\ifhmode,%
            \iftrue,\iffalse,\ifvoid,\ifx,\ifeof,\ifcase,\else,\or,\fi}
\DoNotIndex{\box,\copy,\setbox,\unvbox,\unhbox,\hbox,%
            \vbox,\vtop,\vcenter}
\DoNotIndex{\@empty,\immediate,\write}
\DoNotIndex{\egroup,\bgroup,\expandafter,\begingroup,\endgroup}
\DoNotIndex{\divide,\advance,\multiply,\count,\dimen}
\DoNotIndex{\relax,\space,\string}
\DoNotIndex{\csname,\endcsname,\@spaces,\openin,\openout,%
            \closein,\closeout}
\DoNotIndex{\catcode,\endinput}
\DoNotIndex{\jobname,\message,\read,\the,\m@ne,\noexpand}
\DoNotIndex{\hsize,\vsize,\hskip,\vskip,\kern,\hfil,\hfill,\hss}
\DoNotIndex{\m@ne,\z@,\z@skip,\@ne,\tw@,\p@}
\DoNotIndex{\dp,\wd,\ht,\vss,\unskip}
%    \end{macrocode}
%
% Last bit of indexing stuff, for now: I'll typeset the index in two columns
% (the default is three, which makes them too narrow for my tastes).
%
%    \begin{macrocode}
\setcounter{IndexColumns}{2}
%    \end{macrocode}
%
%
% \subsection{Selectively defining things}
%
% I don't want to tread on anyone's toes if they redefine any of these
% commands and things in a configuration file.  The following definitions
% are fairly evil, but should do the job OK.
%
% \begin{macro}{\@gobbledef}
%
% This macro eats the following |\def|inition, leaving not a trace behind.
%
%    \begin{macrocode}
\def\@gobbledef#1#{\@gobble}
%    \end{macrocode}
%
% \end{macro}
%
% \begin{macro}{\tdef}
% \begin{macro}{\tlet}
%
% The |\tdef| command is a sort of `tentative' definition -- it's like
% |\def| if the control sequence named doesn't already have a definition.
% |\tlet| does the same thing with |\let|.
%
%    \begin{macrocode}
\def\tdef#1{
  \ifx#1\@@undefined%
    \expandafter\def\expandafter#1%
  \else%
    \expandafter\@gobbledef%
  \fi%
}
\def\tlet#1#2{\ifx#1\@@undefined\let#1=#2\fi}
%    \end{macrocode}
%
% \end{macro}
% \end{macro}
%
%
% \subsection{General markup things}
%
% Now for some really simple things.  I'll define how to typeset package
% names and environment names (both in the sans serif font, for now).
%
%    \begin{macrocode}
\tlet\package\textsf
\tlet\env\textsf
%    \end{macrocode}
%
% I'll define the |\<|\dots|>| shortcut for syntax items suggested in the
% \package{syntax} package.
%
%    \begin{macrocode}
\tdef\<#1>{\synt{#1}}
%    \end{macrocode}
%
% And because it's used in a few places (mainly for typesetting lengths),
% here's a command for typesetting fractions in text.
%
%    \begin{macrocode}
\tdef\smallf#1/#2{\ensuremath{^{#1}\!/\!_{#2}}}
%    \end{macrocode}
%
%
% \subsection{A table environment}
%
% \begin{environment}{tab}
%
% Most of the packages don't use the (obviously perfect) \package{mdwtab}
% package, because it's big, and takes a while to load.  Here's an
% environment for typesetting centred tables.  The first (optional) argument
% is some declarations to perform.  The mandatory argument is the table
% preamble (obviously).
%
%    \begin{macrocode}
\@ifundefined{tab}{%
  \newenvironment{tab}[2][\relax]{%
    \par\vskip2ex%
    \centering%
    #1%
    \begin{tabular}{#2}%
  }{%
    \end{tabular}%
    \par\vskip2ex%
  }
}{}
%    \end{macrocode}
%
% \end{environment}
%
%
% \subsection{Commenting out of stuff}
%
% \begin{environment}{meta-comment}
%
% Using |\iffalse|\dots|\fi| isn't much fun.  I'll define a gobbling
% environment using the \package{sverb} stuff.
%
%    \begin{macrocode}
\ignoreenv{meta-comment}
%    \end{macrocode}
%
% \end{environment}
%
%
% \subsection{Float handling}
%
% This gubbins will try to avoid float pages as much as possible, and (with
% any luck) encourage floats to be put on the same pages as text.
%
%    \begin{macrocode}
\def\textfraction{0.1}
\def\topfraction{0.9}
\def\bottomfraction{0.9}
\def\floatpagefraction{0.7}
%    \end{macrocode}
%
% Now redefine the default float-placement parameters to allow `here' floats.
%
%    \begin{macrocode}
\def\fps@figure{htbp}
\def\fps@table{htbp}
%    \end{macrocode}
%
%
% \subsection{Other bits of parameter tweaking}
%
% Make \env{grammar} environments look pretty, by indenting the left hand
% sides by a large amount.
%
%    \begin{macrocode}
\grammarindent1in
%    \end{macrocode}
%
% I don't like being told by \TeX\ that my paragraphs are hard to linebreak:
% I know this already.  This lot should shut \TeX\ up about most problems.
%
%    \begin{macrocode}
\sloppy
\hbadness\@M
\hfuzz10\p@
%    \end{macrocode}
%
% Also make \TeX\ shut up in the index.  The \package{multicol} package
% irritatingly plays with |\hbadness|.  This is the best hook I could find
% for playing with this setting.
%
%    \begin{macrocode}
\expandafter\def\expandafter\IndexParms\expandafter{%
  \IndexParms%
  \hbadness\@M%
}
%    \end{macrocode}
%
% The other thing I really don't like is `Marginpar moved' warnings.  This
% will get rid of them, and lots of other \LaTeX\ warnings at the same time.
%
%    \begin{macrocode}
\let\@latex@warning@no@line\@gobble
%    \end{macrocode}
%
% Put some extra space between table rows, please.
%
%    \begin{macrocode}
\def\arraystretch{1.2}
%    \end{macrocode}
%
% Most of the code is at guard level one, so typeset that in upright text.
%
%    \begin{macrocode}
\setcounter{StandardModuleDepth}{1}
%    \end{macrocode}
%
%
% \subsection{Contents handling}
%
% I use at least one contents file (the main table of contents) although
% I may want more.  I'll keep a list of contents files which I need to
% handle.
%
% There are two things I need to do to contents files here:
% \begin{itemize}
% \item I must typeset the table of contents at the beginning of the
%       document; and
% \item I want to typeset tables of contents in two columns (using the
%       \package{multicol} package).
% \end{itemize}
%
% The list consists of items of the form
% \syntax{"\\do{"<extension>"}{"<command>"}"}, where \<extension> is the
% file extension of the contents file, and \<command> is the command to
% typeset it.
%
% \begin{macro}{\docontents}
%
% This is where I keep the list of contents files.  I'll initialise it to
% just do the standard contents table.
%
%    \begin{macrocode}
\def\docontents{\do{toc}{\tableofcontents}}
%    \end{macrocode}
%
% \end{macro}
%
% \begin{macro}{\addcontents}
%
% By saying \syntax{"\\addcontents{"<extension>"}{"<command>"}"}, a document
% can register a new table of contents which gets given the two-column
% treatment properly.  This is really easy to implement.
%
%    \begin{macrocode}
\def\addcontents#1#2{%
  \toks@\expandafter{\docontents\do{#1}{#2}}%
  \edef\docontents{\the\toks@}%
}
%    \end{macrocode}
%
% \end{macro}
%
%
% \subsection{Finishing it all off}
%
% \begin{macro}{\finalstuff}
%
% The |\finalstuff| macro is a hook for doing things at the end of the
% document.  Currently, it inputs the licence agreement as an appendix.
%
%    \begin{macrocode}
\tdef\finalstuff{\appendix\part*{Appendix}% \iffalse <meta-comment>
%
% $Id: gpl.tex,v 1.1 1996/11/19 20:51:14 mdw Exp $
%
% The GNU General Public Licence as a LaTeX section
%
% (c) 1989, 1991 Free Software Foundation, Inc.
%   LaTeX markup and minor formatting changes by Mark Wooding
%

%----- Revision history -----------------------------------------------------
%
% $Log: gpl.tex,v $
% Revision 1.1  1996/11/19 20:51:14  mdw
% Initial revision
%

% --- Chapter heading ---
%
% We don't know whether this ought to be a section or a chapter.  Easy.
% We'll see if chapters are possible.
%
% \fi

\begingroup
\makeatletter

\edef\next#1#2#3{\relax
  \ifx\chapter\@@undefined
    \ifx\documentclass\@notprerr#2\else#3\fi
  \else#1\fi
}

\expandafter\endgroup\next
{
  \let\gpltoplevel\chapter
  \let\gplsec\section
  \let\gplend\endinput
}{
  \let\gpltoplevel\section
  \let\gplsec\subsection
  \let\gplend\endinput
}{
  \documentclass[a4paper]{article}
  \def\gpltoplevel#1{%
    \vspace*{1in}%
    \hbox to\hsize{\hfil\LARGE\bfseries#1\hfil}%
    \vspace{1in}%
  }
  \let\gplsec\section
  \def\gplend{\end{document}}
  \advance\textwidth1in
  \advance\oddsidemargin-.5in
  \sloppy
  \begin{document}
}

%^^A-------------------------------------------------------------------------
\gpltoplevel{The GNU General Public Licence}


The following is the text of the GNU General Public Licence, under the terms
of which this software is distrubuted.

\vspace{12pt}

\begin{center}
\textbf{GNU GENERAL PUBLIC LICENSE} \\
Version 2, June 1991
\end{center}

\begin{center}
Copyright (C) 1989, 1991 Free Software Foundation, Inc. \\
675 Mass Ave, Cambridge, MA 02139, USA

Everyone is permitted to copy and distribute verbatim copies \\
of this license document, but changing it is not allowed.
\end{center}


\gplsec{Preamble}

The licenses for most software are designed to take away your freedom to
share and change it.  By contrast, the GNU General Public License is intended
to guarantee your freedom to share and change free software---to make sure
the software is free for all its users.  This General Public License applies
to most of the Free Software Foundation's software and to any other program
whose authors commit to using it.  (Some other Free Software Foundation
software is covered by the GNU Library General Public License instead.)  You
can apply it to your programs, too.

When we speak of free software, we are referring to freedom, not price.  Our
General Public Licenses are designed to make sure that you have the freedom
to distribute copies of free software (and charge for this service if you
wish), that you receive source code or can get it if you want it, that you
can change the software or use pieces of it in new free programs; and that
you know you can do these things.

To protect your rights, we need to make restrictions that forbid anyone to
deny you these rights or to ask you to surrender the rights.  These
restrictions translate to certain responsibilities for you if you distribute
copies of the software, or if you modify it.

For example, if you distribute copies of such a program, whether gratis or
for a fee, you must give the recipients all the rights that you have.  You
must make sure that they, too, receive or can get the source code.  And you
must show them these terms so they know their rights.

We protect your rights with two steps: (1) copyright the software, and (2)
offer you this license which gives you legal permission to copy, distribute
and/or modify the software.

Also, for each author's protection and ours, we want to make certain that
everyone understands that there is no warranty for this free software.  If
the software is modified by someone else and passed on, we want its
recipients to know that what they have is not the original, so that any
problems introduced by others will not reflect on the original authors'
reputations.

Finally, any free program is threatened constantly by software patents.  We
wish to avoid the danger that redistributors of a free program will
individually obtain patent licenses, in effect making the program
proprietary.  To prevent this, we have made it clear that any patent must be
licensed for everyone's free use or not licensed at all.

The precise terms and conditions for copying, distribution and modification
follow.


\gplsec{Terms and conditions for copying, distribution and modification}

\begin{enumerate}

\makeatletter \setcounter{\@listctr}{-1} \makeatother

\item [0.] This License applies to any program or other work which contains a
      notice placed by the copyright holder saying it may be distributed
      under the terms of this General Public License.  The ``Program'',
      below, refers to any such program or work, and a ``work based on the
      Program'' means either the Program or any derivative work under
      copyright law: that is to say, a work containing the Program or a
      portion of it, either verbatim or with modifications and/or translated
      into another language.  (Hereinafter, translation is included without
      limitation in the term ``modification''.)  Each licensee is addressed
      as ``you''.

      Activities other than copying, distribution and modification are not
      covered by this License; they are outside its scope.  The act of
      running the Program is not restricted, and the output from the Program
      is covered only if its contents constitute a work based on the Program
      (independent of having been made by running the Program).  Whether that
      is true depends on what the Program does.

\item [1.] You may copy and distribute verbatim copies of the Program's
      source code as you receive it, in any medium, provided that you
      conspicuously and appropriately publish on each copy an appropriate
      copyright notice and disclaimer of warranty; keep intact all the
      notices that refer to this License and to the absence of any warranty;
      and give any other recipients of the Program a copy of this License
      along with the Program.

      You may charge a fee for the physical act of transferring a copy, and
      you may at your option offer warranty protection in exchange for a fee.

\item [2.] You may modify your copy or copies of the Program or any portion
      of it, thus forming a work based on the Program, and copy and
      distribute such modifications or work under the terms of Section 1
      above, provided that you also meet all of these conditions:

      \begin{enumerate}

      \item [(a)] You must cause the modified files to carry prominent
            notices stating that you changed the files and the date of any
            change.

      \item [(b)] You must cause any work that you distribute or publish,
            that in whole or in part contains or is derived from the Program
            or any part thereof, to be licensed as a whole at no charge to
            all third parties under the terms of this License.

      \item [(c)] If the modified program normally reads commands
            interactively when run, you must cause it, when started running
            for such interactive use in the most ordinary way, to print or
            display an announcement including an appropriate copyright notice
            and a notice that there is no warranty (or else, saying that you
            provide a warranty) and that users may redistribute the program
            under these conditions, and telling the user how to view a copy
            of this License.  (Exception: if the Program itself is
            interactive but does not normally print such an announcement,
            your work based on the Program is not required to print an
            announcement.)

      \end{enumerate}

      These requirements apply to the modified work as a whole.  If
      identifiable sections of that work are not derived from the Program,
      and can be reasonably considered independent and separate works in
      themselves, then this License, and its terms, do not apply to those
      sections when you distribute them as separate works.  But when you
      distribute the same sections as part of a whole which is a work based
      on the Program, the distribution of the whole must be on the terms of
      this License, whose permissions for other licensees extend to the
      entire whole, and thus to each and every part regardless of who wrote
      it.

      Thus, it is not the intent of this section to claim rights or contest
      your rights to work written entirely by you; rather, the intent is to
      exercise the right to control the distribution of derivative or
      collective works based on the Program.

      In addition, mere aggregation of another work not based on the Program
      with the Program (or with a work based on the Program) on a volume of a
      storage or distribution medium does not bring the other work under the
      scope of this License.

\item [3.] You may copy and distribute the Program (or a work based on it,
      under Section 2) in object code or executable form under the terms of
      Sections 1 and 2 above provided that you also do one of the following:

      \begin{enumerate}

      \item [(a)] Accompany it with the complete corresponding
            machine-readable source code, which must be distributed under the
            terms of Sections 1 and 2 above on a medium customarily used for
            software interchange; or,

      \item [(b)] Accompany it with a written offer, valid for at least three
            years, to give any third party, for a charge no more than your
            cost of physically performing source distribution, a complete
            machine-readable copy of the corresponding source code, to be
            distributed under the terms of Sections 1 and 2 above on a medium
            customarily used for software interchange; or,

      \item [(c)] Accompany it with the information you received as to the
            offer to distribute corresponding source code.  (This alternative
            is allowed only for noncommercial distribution and only if you
            received the program in object code or executable form with such
            an offer, in accord with Subsection b above.)

      \end{enumerate}

      The source code for a work means the preferred form of the work for
      making modifications to it.  For an executable work, complete source
      code means all the source code for all modules it contains, plus any
      associated interface definition files, plus the scripts used to control
      compilation and installation of the executable.  However, as a special
      exception, the source code distributed need not include anything that
      is normally distributed (in either source or binary form) with the
      major components (compiler, kernel, and so on) of the operating system
      on which the executable runs, unless that component itself accompanies
      the executable.

      If distribution of executable or object code is made by offering access
      to copy from a designated place, then offering equivalent access to
      copy the source code from the same place counts as distribution of the
      source code, even though third parties are not compelled to copy the
      source along with the object code.

\item [4.] You may not copy, modify, sublicense, or distribute the Program
      except as expressly provided under this License.  Any attempt otherwise
      to copy, modify, sublicense or distribute the Program is void, and will
      automatically terminate your rights under this License.  However,
      parties who have received copies, or rights, from you under this
      License will not have their licenses terminated so long as such parties
      remain in full compliance.

\item [5.] You are not required to accept this License, since you have not
      signed it.  However, nothing else grants you permission to modify or
      distribute the Program or its derivative works.  These actions are
      prohibited by law if you do not accept this License.  Therefore, by
      modifying or distributing the Program (or any work based on the
      Program), you indicate your acceptance of this License to do so, and
      all its terms and conditions for copying, distributing or modifying the
      Program or works based on it.

\item [6.] Each time you redistribute the Program (or any work based on the
      Program), the recipient automatically receives a license from the
      original licensor to copy, distribute or modify the Program subject to
      these terms and conditions.  You may not impose any further
      restrictions on the recipients' exercise of the rights granted herein.
      You are not responsible for enforcing compliance by third parties to
      this License.

\item [7.] If, as a consequence of a court judgment or allegation of patent
      infringement or for any other reason (not limited to patent issues),
      conditions are imposed on you (whether by court order, agreement or
      otherwise) that contradict the conditions of this License, they do not
      excuse you from the conditions of this License.  If you cannot
      distribute so as to satisfy simultaneously your obligations under this
      License and any other pertinent obligations, then as a consequence you
      may not distribute the Program at all.  For example, if a patent
      license would not permit royalty-free redistribution of the Program by
      all those who receive copies directly or indirectly through you, then
      the only way you could satisfy both it and this License would be to
      refrain entirely from distribution of the Program.

      If any portion of this section is held invalid or unenforceable under
      any particular circumstance, the balance of the section is intended to
      apply and the section as a whole is intended to apply in other
      circumstances.

      It is not the purpose of this section to induce you to infringe any
      patents or other property right claims or to contest validity of any
      such claims; this section has the sole purpose of protecting the
      integrity of the free software distribution system, which is
      implemented by public license practices.  Many people have made
      generous contributions to the wide range of software distributed
      through that system in reliance on consistent application of that
      system; it is up to the author/donor to decide if he or she is willing
      to distribute software through any other system and a licensee cannot
      impose that choice.

      This section is intended to make thoroughly clear what is believed to
      be a consequence of the rest of this License.

\item [8.] If the distribution and/or use of the Program is restricted in
      certain countries either by patents or by copyrighted interfaces, the
      original copyright holder who places the Program under this License may
      add an explicit geographical distribution limitation excluding those
      countries, so that distribution is permitted only in or among countries
      not thus excluded.  In such case, this License incorporates the
      limitation as if written in the body of this License.

\item [9.] The Free Software Foundation may publish revised and/or new
      versions of the General Public License from time to time.  Such new
      versions will be similar in spirit to the present version, but may
      differ in detail to address new problems or concerns.

      Each version is given a distinguishing version number.  If the Program
      specifies a version number of this License which applies to it and
      ``any later version'', you have the option of following the terms and
      conditions either of that version or of any later version published by
      the Free Software Foundation.  If the Program does not specify a
      version number of this License, you may choose any version ever
      published by the Free Software Foundation.

\item [10.] If you wish to incorporate parts of the Program into other free
      programs whose distribution conditions are different, write to the
      author to ask for permission.  For software which is copyrighted by the
      Free Software Foundation, write to the Free Software Foundation; we
      sometimes make exceptions for this.  Our decision will be guided by the
      two goals of preserving the free status of all derivatives of our free
      software and of promoting the sharing and reuse of software generally.

\begin{center}
NO WARRANTY
\end{center}

\bfseries

\item [11.] Because the Program is licensed free of charge, there is no
      warranty for the Program, to the extent permitted by applicable law.
      except when otherwise stated in writing the copyright holders and/or
      other parties provide the program ``as is'' without warranty of any
      kind, either expressed or implied, including, but not limited to, the
      implied warranties of merchantability and fitness for a particular
      purpose.  The entire risk as to the quality and performance of the
      Program is with you.  Should the Program prove defective, you assume
      the cost of all necessary servicing, repair or correction.

\item [12.] In no event unless required by applicable law or agreed to in
      writing will any copyright holder, or any other party who may modify
      and/or redistribute the program as permitted above, be liable to you
      for damages, including any general, special, incidental or
      consequential damages arising out of the use or inability to use the
      program (including but not limited to loss of data or data being
      rendered inaccurate or losses sustained by you or third parties or a
      failure of the Program to operate with any other programs), even if
      such holder or other party has been advised of the possibility of such
      damages.

\end{enumerate}

\begin{center}
\textbf{END OF TERMS AND CONDITIONS}
\end{center}


\gplsec{Appendix: How to Apply These Terms to Your New Programs}

If you develop a new program, and you want it to be of the greatest possible
use to the public, the best way to achieve this is to make it free software
which everyone can redistribute and change under these terms.

To do so, attach the following notices to the program.  It is safest to
attach them to the start of each source file to most effectively convey the
exclusion of warranty; and each file should have at least the ``copyright''
line and a pointer to where the full notice is found.

\begin{verbatim}
<one line to give the program's name and a brief idea of what it does.>
Copyright (C) 19yy  <name of author>

This program is free software; you can redistribute it and/or modify
it under the terms of the GNU General Public License as published by
the Free Software Foundation; either version 2 of the License, or
(at your option) any later version.

This program is distributed in the hope that it will be useful,
but WITHOUT ANY WARRANTY; without even the implied warranty of
MERCHANTABILITY or FITNESS FOR A PARTICULAR PURPOSE.  See the
GNU General Public License for more details.

You should have received a copy of the GNU General Public License
along with this program; if not, write to the Free Software
Foundation, Inc., 675 Mass Ave, Cambridge, MA 02139, USA.
\end{verbatim}

Also add information on how to contact you by electronic and paper mail.

If the program is interactive, make it output a short notice like this when
it starts in an interactive mode:

\begin{verbatim}
Gnomovision version 69, Copyright (C) 19yy name of author
Gnomovision comes with ABSOLUTELY NO WARRANTY; for details type `show w'.
This is free software, and you are welcome to redistribute it
under certain conditions; type `show c' for details.
\end{verbatim}

The hypothetical commands `show w' and `show c' should show the appropriate
parts of the General Public License.  Of course, the commands you use may be
called something other than `show w' and `show c'; they could even be
mouse-clicks or menu items--whatever suits your program.

You should also get your employer (if you work as a programmer) or your
school, if any, to sign a ``copyright disclaimer'' for the program, if
necessary.  Here is a sample; alter the names:

\begin{verbatim}
Yoyodyne, Inc., hereby disclaims all copyright interest in the program
`Gnomovision' (which makes passes at compilers) written by James Hacker.

<signature of Ty Coon>, 1 April 1989
Ty Coon, President of Vice
\end{verbatim}

This General Public License does not permit incorporating your program into
proprietary programs.  If your program is a subroutine library, you may
consider it more useful to permit linking proprietary applications with the
library.  If this is what you want to do, use the GNU Library General Public
License instead of this License.

\gplend
}
%    \end{macrocode}
%
% \end{macro}
%
% \begin{macro}{\implementation}
%
% The |\implementation| macro starts typesetting the implementation of
% the package(s).  If we're not doing the implementation, it just does
% this lot and ends the input file.
%
% I define a macro with arguments inside the |\StopEventually|, which causes
% problems, since the code gets put through an extra level of |\def|fing
% depending on whether the implementation stuff gets typeset or not.  I'll
% store the code I want to do in a separate macro.
%
%    \begin{macrocode}
\def\implementation{\StopEventually{\attheend}}
%    \end{macrocode}
%
% Now for the actual activity.  First, I'll do the |\finalstuff|.  Then, if
% \package{doc}'s managed to find the \package{multicol} package, I'll add
% the end of the environment to the end of each contents file in the list.
% Finally, I'll read the index in from its formatted |.ind| file.
%
%    \begin{macrocode}
\tdef\attheend{%
  \finalstuff%
  \ifhave@multicol%
    \def\do##1##2{\addtocontents{##1}{\protect\end{multicols}}}%
    \docontents%
  \fi%
  \PrintIndex%
}
%    \end{macrocode}
%
% \end{macro}
%
%
% \subsection{File version information}
%
% \begin{macro}{\mdwpkginfo}
%
% For setting up the automatic titles, I'll need to be able to work out
% file versions and things.  This macro will, given a file name, extract
% from \LaTeX\ the version information and format it into a sensible string.
%
% First of all, I'll put the original string (direct from the
% |\Provides|\dots\ command).  Then I'll pass it to another macro which can
% parse up the string into its various bits, along with the original
% filename.
%
%    \begin{macrocode}
\def\mdwpkginfo#1{%
  \edef\@tempa{\csname ver@#1\endcsname}%
  \expandafter\mdwpkginfo@i\@tempa\@@#1\@@%
}
%    \end{macrocode}
%
% Now for the real business.  I'll store the string I build in macros called
% \syntax{"\\"<filename>"date", "\\"<filename>"version" and
% "\\"<filename>"info"}, which store the file's date, version and
% `information string' respectively.  (Note that the file extension isn't
% included in the name.)
%
% This is mainly just tedious playing with |\expandafter|.  The date format
% is defined by a separate macro, which can be modified from the
% configuration file.
%
%    \begin{macrocode}
\def\mdwpkginfo@i#1/#2/#3 #4 #5\@@#6.#7\@@{%
  \expandafter\def\csname #6date\endcsname%
    {\protect\mdwdateformat{#1}{#2}{#3}}%
  \expandafter\def\csname #6version\endcsname{#4}%
  \expandafter\def\csname #6info\endcsname{#5}%
}
%    \end{macrocode}
%
% \end{macro}
%
% \begin{macro}{\mdwdateformat}
%
% Given three arguments, a year, a month and a date (all numeric), build a
% pretty date string.  This is fairly simple really.
%
%    \begin{macrocode}
\tdef\mdwdateformat#1#2#3{\number#3\ \monthname{#2}\ \number#1}
\def\monthname#1{%
  \ifcase#1\or%
     January\or February\or March\or April\or May\or June\or%
     July\or August\or September\or October\or November\or December%
  \fi%
}
\def\numsuffix#1{%
  \ifnum#1=1 st\else%
  \ifnum#1=2 nd\else%
  \ifnum#1=3 rd\else%
  \ifnum#1=21 st\else%
  \ifnum#1=22 nd\else%
  \ifnum#1=23 rd\else%
  \ifnum#1=31 st\else%
  th%
  \fi\fi\fi\fi\fi\fi\fi%
}
%    \end{macrocode}
%
% \end{macro}
%
% \begin{macro}{\mdwfileinfo}
%
% Saying \syntax{"\\mdwfileinfo{"<file-name>"}{"<info>"}"} extracts the
% wanted item of \<info> from the version information for file \<file-name>.
%
%    \begin{macrocode}
\def\mdwfileinfo#1#2{\mdwfileinfo@i{#2}#1.\@@}
\def\mdwfileinfo@i#1#2.#3\@@{\csname#2#1\endcsname}
%    \end{macrocode}
%
% \end{macro}
%
%
% \subsection{List handling}
%
% There are several other lists I need to build.  These macros will do
% the necessary stuff.
%
% \begin{macro}{\mdw@ifitem}
%
% The macro \syntax{"\\mdw@ifitem"<item>"\\in"<list>"{"<true-text>"}"^^A
% "{"<false-text>"}"} does \<true-text> if the \<item> matches any item in
% the \<list>; otherwise it does \<false-text>.
%
%    \begin{macrocode}
\def\mdw@ifitem#1\in#2{%
  \@tempswafalse%
  \def\@tempa{#1}%
  \def\do##1{\def\@tempb{##1}\ifx\@tempa\@tempb\@tempswatrue\fi}%
  #2%
  \if@tempswa\expandafter\@firstoftwo\else\expandafter\@secondoftwo\fi%
}
%    \end{macrocode}
%
% \end{macro}
%
% \begin{macro}{\mdw@append}
%
% Saying \syntax{"\\mdw@append"<item>"\\to"<list>} adds the given \<item>
% to the end of the given \<list>.
%
%    \begin{macrocode}
\def\mdw@append#1\to#2{%
  \toks@{\do{#1}}%
  \toks\tw@\expandafter{#2}%
  \edef#2{\the\toks\tw@\the\toks@}%
}
%    \end{macrocode}
%
% \end{macro}
%
% \begin{macro}{\mdw@prepend}
%
% Saying \syntax{"\\mdw@prepend"<item>"\\to"<list>} adds the \<item> to the
% beginning of the \<list>.
%
%    \begin{macrocode}
\def\mdw@prepend#1\to#2{%
  \toks@{\do{#1}}%
  \toks\tw@\expandafter{#2}%
  \edef#2{\the\toks@\the\toks\tw@}%
}
%    \end{macrocode}
%
% \end{macro}
%
% \begin{macro}{\mdw@add}
%
% Finally, saying \syntax{"\\mdw@add"<item>"\\to"<list>} adds the \<item>
% to the list only if it isn't there already.
%
%    \begin{macrocode}
\def\mdw@add#1\to#2{\mdw@ifitem#1\in#2{}{\mdw@append#1\to#2}}
%    \end{macrocode}
%
% \end{macro}
%
%
% \subsection{Described file handling}
%
% I'l maintain lists of packages, document classes, and other files
% described by the current documentation file.
%
% First of all, I'll declare the various list macros.
%
%    \begin{macrocode}
\def\dopackages{}
\def\doclasses{}
\def\dootherfiles{}
%    \end{macrocode}
%
% \begin{macro}{\describespackage}
%
% A document file can declare that it describes a package by saying
% \syntax{"\\describespackage{"<package-name>"}"}.  I add the package to
% my list, read the package into memory (so that the documentation can
% offer demonstrations of it) and read the version information.
%
%    \begin{macrocode}
\def\describespackage#1{%
  \mdw@ifitem#1\in\dopackages{}{%
    \mdw@append#1\to\dopackages%
    \usepackage{#1}%
    \mdwpkginfo{#1.sty}%
  }%
}
%    \end{macrocode}
%
% \end{macro}
%
% \begin{macro}{\describesclass}
%
% By saying \syntax{"\\describesclass{"<class-name>"}"}, a document file
% can declare that it describes a document class.  I'll assume that the
% document class is already loaded, because it's much too late to load
% it now.
%
%    \begin{macrocode}
\def\describesclass#1{\mdw@add#1\to\doclasses\mdwpkginfo{#1.cls}}
%    \end{macrocode}
%
% \end{macro}
%
% \begin{macro}{\describesfile}
%
% Finally, other `random' files, which don't have the status of real \LaTeX\
% packages or document classes, can be described by saying \syntax{^^A
% "\\describesfile{"<file-name>"}" or "\\describesfile*{"<file-name>"}"}.
% The difference is that the starred version will not |\input| the file.
%
%    \begin{macrocode}
\def\describesfile{%
  \@ifstar{\describesfile@i\@gobble}{\describesfile@i\input}%
}
\def\describesfile@i#1#2{%
  \mdw@ifitem#2\in\dootherfiles{}{%
    \mdw@add#2\to\dootherfiles%
    #1{#2}%
    \mdwpkginfo{#2}%
  }%
}
%    \end{macrocode}
%
% \end{macro}
%
%
% \subsection{Author and title handling}
%
% I'll redefine the |\author| and |\title| commands so that I get told
% whether I need to do it myself.
%
% \begin{macro}{\author}
%
% This is easy: I'll save the old meaning, and then redefine |\author| to
% do the old thing and redefine itself to then do nothing.
%
%    \begin{macrocode}
\let\mdw@author\author
\def\author{\let\author\@gobble\mdw@author}
%    \end{macrocode}
%
% \end{macro}
%
% \begin{macro}{\title}
%
% And oddly enough, I'll do exactly the same thing for the title, except
% that I'll also disable the |\mdw@buildtitle| command, which constructs
% the title automatically.
%
%    \begin{macrocode}
\let\mdw@title\title
\def\title{\let\title\@gobble\let\mdw@buildtitle\relax\mdw@title}
%    \end{macrocode}
%
% \end{macro}
%
% \begin{macro}{\date}
%
% This works in a very similar sort of way.
%
%    \begin{macrocode}
\def\date#1{\let\date\@gobble\def\today{#1}}
%    \end{macrocode}
%
% \end{macro}
%
% \begin{macro}{\datefrom}
%
% Saying \syntax{"\\datefrom{"<file-name>"}"} sets the document date from
% the given filename.
%
%    \begin{macrocode}
\def\datefrom#1{%
  \protected@edef\@tempa{\noexpand\date{\csname #1date\endcsname}}%
  \@tempa%
}
%    \end{macrocode}
%
% \end{macro}
%
% \begin{macro}{\docfile}
%
% Saying \syntax{"\\docfile{"<file-name>"}"} sets up the file name from which
% documentation will be read.
%
%    \begin{macrocode}
\def\docfile#1{%
  \def\@tempa##1.##2\@@{\def\@basefile{##1.##2}\def\@basename{##1}}%
  \edef\@tempb{\noexpand\@tempa#1\noexpand\@@}%
  \@tempb%
}
%    \end{macrocode}
%
% I'll set up a default value as well.
%
%    \begin{macrocode}
\docfile{\jobname.dtx}
%    \end{macrocode}
%
% \end{macro}
%
%
% \subsection{Building title strings}
%
% This is rather tricky.  For each list, I need to build a legible looking
% string.
%
% \begin{macro}{\mdw@addtotitle}
%
% By saying
%\syntax{"\\mdw@addtotitle{"<list>"}{"<command>"}{"<singular>"}{"<plural>"}"}
% I can add the contents of a list to the current title string in the
% |\mdw@title| macro.
%
%    \begin{macrocode}
\tdef\mdw@addtotitle#1#2#3#4{%
%    \end{macrocode}
%
% Now to get to work.  I need to keep one `lookahead' list item, and a count
% of the number of items read so far.  I'll keep the lookahead item in
% |\@nextitem| and the counter in |\count@|.
%
%    \begin{macrocode}
  \count@\z@%
%    \end{macrocode}
%
% Now I'll define what to do for each list item.  The |\protect| command is
% already set up appropriately for playing with |\edef| commands.
%
%    \begin{macrocode}
  \def\do##1{%
%    \end{macrocode}
%
% The first job is to add the previous item to the title string.  If this
% is the first item, though, I'll just add the appropriate \lit{The } or
% \lit{ and the } string to the title (this is stored in the |\@prefix|
% macro).
%
%    \begin{macrocode}
    \edef\mdw@title{%
      \mdw@title%
      \ifcase\count@\@prefix%
      \or\@nextitem%
      \else, \@nextitem%
      \fi%
    }%
%    \end{macrocode}
%
% That was rather easy.  Now I'll set up the |\@nextitem| macro for the
% next time around the loop.
%
%    \begin{macrocode}
    \edef\@nextitem{%
      \protect#2{##1}%
      \protect\footnote{%
        The \protect#2{##1} #3 is currently at version %
        \mdwfileinfo{##1}{version}, dated \mdwfileinfo{##1}{date}.%
      }\space%
    }%
%    \end{macrocode}
%
% Finally, I need to increment the counter.
%
%    \begin{macrocode}
    \advance\count@\@ne%
  }%
%    \end{macrocode}
%
% Now execute the list.
%
%    \begin{macrocode}
  #1%
%    \end{macrocode}
%
% I still have one item left over, unless the list was empty.  I'll add
% that now.
%
%    \begin{macrocode}
  \edef\mdw@title{%
    \mdw@title%
    \ifcase\count@%
    \or\@nextitem\space#3%
    \or\ and \@nextitem\space#4%
    \fi%
  }%
%    \end{macrocode}
%
% Finally, if $|\count@| \ne 0$, I must set |\@prefix| to \lit{ and the }.
%
%    \begin{macrocode}
  \ifnum\count@>\z@\def\@prefix{ and the }\fi%
}
%    \end{macrocode}
%
% \end{macro}
%
% \begin{macro}{\mdw@buildtitle}
%
% This macro will actually do the job of building the title string.
%
%    \begin{macrocode}
\tdef\mdw@buildtitle{%
%    \end{macrocode}
%
% First of all, I'll open a group to avoid polluting the namespace with
% my gubbins (although the code is now much tidier than it has been in
% earlier releases).
%
%    \begin{macrocode}
  \begingroup%
%    \end{macrocode}
%
% The title building stuff makes extensive use of |\edef|.  I'll set
% |\protect| appropriately.  (For those not in the know,
% |\@unexpandable@protect| expands to `|\noexpand\protect\noexpand|',
% which prevents expansion of the following macro, and inserts a |\protect|
% in front of it ready for the next |\edef|.)
%
%    \begin{macrocode}
  \let\@@protect\protect\let\protect\@unexpandable@protect%
%    \end{macrocode}
%
% Set up some simple macros ready for the main code.
%
%    \begin{macrocode}
  \def\mdw@title{}%
  \def\@prefix{The }%
%    \end{macrocode}
%
% Now build the title.  This is fun.
%
%    \begin{macrocode}
  \mdw@addtotitle\dopackages\package{package}{packages}%
  \mdw@addtotitle\doclasses\package{document class}{document classes}%
  \mdw@addtotitle\dootherfiles\texttt{file}{files}%
%    \end{macrocode}
%
% Now I want to end the group and set the title from my string.  The
% following hacking will do this.
%
%    \begin{macrocode}
  \edef\next{\endgroup\noexpand\title{\mdw@title}}%
  \next%
}
%    \end{macrocode}
%
% \end{macro}
%
%
% \subsection{Starting the main document}
%
% \begin{macro}{\mdwdoc}
%
% Once the document preamble has done all of its stuff, it calls the
% |\mdwdoc| command, which takes over and really starts the documentation
% going.
%
%    \begin{macrocode}
\def\mdwdoc{%
%    \end{macrocode}
%
% First, I'll construct the title string.
%
%    \begin{macrocode}
  \mdw@buildtitle%
  \author{Mark Wooding}%
%    \end{macrocode}
%
% Set up the date string based on the date of the package which shares
% the same name as the current file.
%
%    \begin{macrocode}
  \datefrom\@basename%
%    \end{macrocode}
%
% Set up verbatim characters after all the packages have started.
%
%    \begin{macrocode}
  \shortverb\|%
  \shortverb\"%
%    \end{macrocode}
%
% Start the document, and put the title in.
%
%    \begin{macrocode}
  \begin{document}
  \maketitle%
%    \end{macrocode}
%
% This is nasty.  It makes maths displays work properly in demo environments.
% \emph{The \LaTeX\ Companion} exhibits the bug which this hack fixes.  So
% ner.
%
%    \begin{macrocode}
  \abovedisplayskip\z@%
%    \end{macrocode}
%
% Now start the contents tables.  After starting each one, I'll make it
% be multicolumnar.
%
%    \begin{macrocode}
  \def\do##1##2{%
    ##2%
    \ifhave@multicol\addtocontents{##1}{%
      \protect\begin{multicols}{2}%
      \hbadness\@M%
    }\fi%
  }%
  \docontents%
%    \end{macrocode}
%
% Input the main file now.
%
%    \begin{macrocode}
  \DocInput{\@basefile}%
%    \end{macrocode}
%
% That's it.  I'm done.
%
%    \begin{macrocode}
  \end{document}
}
%    \end{macrocode}
%
% \end{macro}
%
%
% \subsection{And finally\dots}
%
% Right at the end I'll put a hook for the configuration file.
%
%    \begin{macrocode}
\ifx\mdwhook\@@undefined\else\expandafter\mdwhook\fi
%    \end{macrocode}
%
% That's all the code done now.  I'll change back to `user' mode, where
% all the magic control sequences aren't allowed any more.
%
%    \begin{macrocode}
\makeatother
%</mdwtools>
%    \end{macrocode}
%
% Oh, wait!  What if I want to typeset this documentation?  Aha.  I'll cope
% with that by comparing |\jobname| with my filename |mdwtools|.  However,
% there's some fun here, because |\jobname| contains category-12 letters,
% while my letters are category-11.  Time to play with |\string| in a messy
% way.
%
%    \begin{macrocode}
%<*driver>
\makeatletter
\edef\@tempa{\expandafter\@gobble\string\mdwtools}
\edef\@tempb{\jobname}
\ifx\@tempa\@tempb
  \describesfile*{mdwtools.tex}
  \docfile{mdwtools.tex}
  \makeatother
  \expandafter\mdwdoc
\fi
\makeatother
%</driver>
%    \end{macrocode}
%
% That's it.  Done!
%
% \hfill Mark Wooding, \today
%
% \Finale
%
\endinput
|
% \<declarations>
% |\mdwdoc|
% \end{listinglist}
% The initial |\input| reads in this file and sets up the various commands
% which may be needed.  The final |\mdwdoc| actually starts the document,
% inserting a title (which is automatically generated), a table of
% contents etc., and reads the documentation file in (using the |\DocInput|
% command from the \package{doc} package.
%
% \subsubsection{Describing packages}
%
% \DescribeMacro{\describespackage}
% \DescribeMacro{\describesclass}
% \DescribeMacro{\describesfile}
% \DescribeMacro{\describesfile*}
% The most important declarations are those which declare what the
% documentation describes.  Saying \syntax{"\\describespackage{<package>}"}
% loads the \<package> (if necessary) and adds it to the auto-generated
% title, along with a footnote containing version information.  Similarly,
% |\describesclass| adds a document class name to the title (without loading
% it -- the document itself must do this, with the |\documentclass| command).
% For files which aren't packages or classes, use the |\describesfile| or
% |\describesfile*| command (the $*$-version won't |\input| the file, which
% is handy for files like |mdwtools.tex|, which are already input).
%
% \DescribeMacro{\author}
% \DescribeMacro{\date}
% \DescribeMacro{\title}
% The |\author|, |\date| and |\title| declarations work slightly differently
% to normal -- they ensure that only the \emph{first} declaration has an
% effect.  (Don't you play with |\author|, please, unless you're using this
% program to document your own packages.)  Using |\title| suppresses the
% automatic title generation.
%
% \DescribeMacro{\docdate}
% The default date is worked out from the version string of the package or
% document class whose name is the same as that of the documentation file.
% You can choose a different `main' file by saying
% \syntax{"\\docdate{"<file>"}"}.
%
% \subsubsection{Contents handling}
%
% \DescribeMacro{\addcontents}
% A documentation file always has a table of contents.  Other
% contents-like lists can be added by saying
% \syntax{"\\addcontents{"<extension>"}{"<command>"}"}.  The \<extension>
% is the file extension of the contents file (e.g., \lit{lot} for the
% list of tables); the \<command> is the command to actually typeset the
% contents file (e.g., |\listoftables|).
%
% \subsubsection{Other declarations}
%
% \DescribeMacro{\implementation}
% The \package{doc} package wants you to say
% \syntax{"\\StopEventually{"<stuff>"}"}' before describing the package
% implementation.  Using |mdwtools.tex|, you just say |\implementation|, and
% everything works.  It will automatically read in the licence text (from
% |gpl.tex|, and wraps some other things up.
%
% 
% \subsection{Other commands}
%
% The |mdwtools.tex| file includes the \package{syntax} and \package{sverb}
% packages so that they can be used in documentation files.  It also defines
% some trivial commands of its own.
%
% \DescribeMacro{\<}
% Saying \syntax{"\\<"<text>">" is the same as "\\synt{"<text>"}"}; this
% is a simple abbreviation.
%
% \DescribeMacro{\smallf}
% Saying \syntax{"\\smallf" <number>"/"<number>} typesets a little fraction,
% like this: \smallf 3/4.  It's useful when you want to say that the default
% value of a length is 2 \smallf 1/2\,pt, or something like that.
%
%
% \subsection{Customisation}
%
% You can customise the way that the package documentation looks by writing
% a file called |mdwtools.cfg|.  You can redefine various commands (before
% they're defined here, even; |mdwtools.tex| checks most of the commands that
% it defines to make sure they haven't been defined already.
%
% \DescribeMacro{\indexing}
% If you don't want the prompt about whether to generate index files, you
% can define the |\indexing| command to either \lit{y} or \lit{n}.  I'd
% recommend that you use |\providecommand| for this, to allow further
% customisation from the command line.
%
% \DescribeMacro{\mdwdateformat}
% If you don't like my date format (maybe you're American or something),
% you can redefine the |\mdwdateformat| command.  It takes three arguments:
% the year, month and date, as numbers; it should expand to something which
% typesets the date nicely.  The default format gives something like
% `10 May 1996'.  You can produce something rather more exotic, like
% `10\textsuperscript{th} May \textsc{\romannumeral 1996}' by saying
%\begin{listing}
%\newcommand{\mdwdateformat}[3]{%
%  \number#3\textsuperscript{\numsuffix{#3}}\ %
%  \monthname{#2}\ %
%  \textsc{\romannumeral #1}%
%}
%\end{listing}
% \DescribeMacro{\monthname}
% \DescribeMacro{\numsuffix}
% Saying \syntax{"\\monthname{"<number>"}"} expands to the name of the
% numbered month (which can be useful when doing date formats).  Saying
% \syntax{"\\numsuffix{"<number>"}"} will expand to the appropriate suffix
% (`th' or `rd' or whatever) for the \<number>.  You'll have to superscript
% it yourself, if this is what you want to do.  Putting the year number
% in roman numerals is just pretentious |;-)|.
%
% \DescribeMacro{\mdwhook}
% After all the declarations in |mdwtools.tex|, the command |\mdwhook| is
% executed, if it exists.  This can be set up by the configuration file
% to do whatever you want.
%
% There are lots of other things you can play with; you should look at the
% implementation section to see what's possible.
%
% \implementation
%
% \section{Implementation}
%
%    \begin{macrocode}
%<*mdwtools>
%    \end{macrocode}
%
% The first thing is that I'm not a \LaTeX\ package or anything official
% like that, so I must enable `|@|' as a letter by hand.
%
%    \begin{macrocode}
\makeatletter
%    \end{macrocode}
%
% Now input the user's configuration file, if it exists.  This is fairly
% simple stuff.
%
%    \begin{macrocode}
\@input{mdwtools.cfg}
%    \end{macrocode}
%
% Well, that's the easy bit done.
%
%
% \subsection{Initialisation}
%
% Obviously the first thing to do is to obtain a document class.  Obviously,
% it would be silly to do this if a document class has already been loaded,
% either by the package documentation or by the configuration file.
%
% The only way I can think of for finding out if a document class is already
% loaded is by seeing if the |\documentclass| command has been redefined
% to raise an error.  This isn't too hard, really.
%
%    \begin{macrocode}
\ifx\documentclass\@twoclasseserror\else
  \documentclass[a4paper]{ltxdoc}
  \ifx\doneclasses\mdw@undefined\else\doneclasses\fi
\fi
%    \end{macrocode}
%
% As part of my standard environment, I'll load some of my more useful
% packages.  If they're already loaded (possibly with different options),
% I'll not try to load them again.
%
%    \begin{macrocode}
\@ifpackageloaded{doc}{}{\usepackage{doc}}
\@ifpackageloaded{syntax}{}{\usepackage[rounded]{syntax}}
\@ifpackageloaded{sverb}{}{\usepackage{sverb}}
%    \end{macrocode}
%
%
% \subsection{Some macros for interaction}
%
% I like the \LaTeX\ star-boxes, although it's a pain having to cope with
% \TeX's space-handling rules.  I'll define a new typing-out macro which
% makes spaces more significant, and has a $*$-version which doesn't put
% a newline on the end, and interacts prettily with |\read|.
%
% First of all, I need to make spaces active, so I can define things about
% active spaces.
%
%    \begin{macrocode}
\begingroup\obeyspaces
%    \end{macrocode}
%
% Now to define the main macro.  This is easy stuff.  Spaces must be
% carefully rationed here, though.
%
% I'll start a group, make spaces active, and make spaces expand to ordinary
% space-like spaces.  Then I'll look for a star, and pass either |\message|
% (which doesn't start a newline, and interacts with |\read| well) or
% |\immediate\write 16| which does a normal write well.
%
%    \begin{macrocode}
\gdef\mdwtype{%
\begingroup\catcode`\ \active\let \space%
\@ifstar{\mdwtype@i{\message}}{\mdwtype@i{\immediate\write\sixt@@n}}%
}
\endgroup
%    \end{macrocode}
%
% Now for the easy bit.  I have the thing to do, and the thing to do it to,
% so do that and end the group.
%
%    \begin{macrocode}
\def\mdwtype@i#1#2{#1{#2}\endgroup}
%    \end{macrocode}
%
%
% \subsection{Decide on indexing}
%
% A configuration file can decide on indexing by defining the |\indexing|
% macro to either \lit{y} or \lit{n}.  If it's not set, then I'll prompt
% the user.
%
% First of all, I want a switch to say whether I'm indexing.
%
%    \begin{macrocode}
\newif\ifcreateindex
%    \end{macrocode}
%
% Right: now I need to decide how to make progress.  If the macro's not set,
% then I want to set it, and start a row of stars.
%
%    \begin{macrocode}
\ifx\indexing\@@undefined
  \mdwtype{*****************************}
  \def\indexing{?}
\fi
%    \end{macrocode}
%
% Now enter a loop, asking the user whether to do indexing, until I get
% a sensible answer.
%
%    \begin{macrocode}
\loop
  \@tempswafalse
  \if y\indexing\@tempswatrue\createindextrue\fi
  \if Y\indexing\@tempswatrue\createindextrue\fi
  \if n\indexing\@tempswatrue\createindexfalse\fi
  \if N\indexing\@tempswatrue\createindexfalse\fi
  \if@tempswa\else
  \mdwtype*{* Create index files? (y/n) *}
  \read\sixt@@n to\indexing%
\repeat
%    \end{macrocode}
%
% Now, based on the results of that, display a message about the indexing.
%
%    \begin{macrocode}
\mdwtype{*****************************}
\ifcreateindex
  \mdwtype{* Creating index files      *}
  \mdwtype{* This may take some time   *}
\else
  \mdwtype{* Not creating index files  *}
\fi
\mdwtype{*****************************}
%    \end{macrocode}
%
% Now I can play with the indexing commands of the \package{doc} package
% to do whatever it is that the user wants.
%
%    \begin{macrocode}
\ifcreateindex
  \CodelineIndex
  \EnableCrossrefs
\else
  \CodelineNumbered
  \DisableCrossrefs
\fi
%    \end{macrocode}
%
% And register lots of plain \TeX\ things which shouldn't be indexed.
% This contains lots of |\if|\dots\ things which don't fit nicely in
% conditionals, which is a shame.  Still, it doesn't matter that much,
% really.
%
%    \begin{macrocode}
\DoNotIndex{\def,\long,\edef,\xdef,\gdef,\let,\global}
\DoNotIndex{\if,\ifnum,\ifdim,\ifcat,\ifmmode,\ifvmode,\ifhmode,%
            \iftrue,\iffalse,\ifvoid,\ifx,\ifeof,\ifcase,\else,\or,\fi}
\DoNotIndex{\box,\copy,\setbox,\unvbox,\unhbox,\hbox,%
            \vbox,\vtop,\vcenter}
\DoNotIndex{\@empty,\immediate,\write}
\DoNotIndex{\egroup,\bgroup,\expandafter,\begingroup,\endgroup}
\DoNotIndex{\divide,\advance,\multiply,\count,\dimen}
\DoNotIndex{\relax,\space,\string}
\DoNotIndex{\csname,\endcsname,\@spaces,\openin,\openout,%
            \closein,\closeout}
\DoNotIndex{\catcode,\endinput}
\DoNotIndex{\jobname,\message,\read,\the,\m@ne,\noexpand}
\DoNotIndex{\hsize,\vsize,\hskip,\vskip,\kern,\hfil,\hfill,\hss}
\DoNotIndex{\m@ne,\z@,\z@skip,\@ne,\tw@,\p@}
\DoNotIndex{\dp,\wd,\ht,\vss,\unskip}
%    \end{macrocode}
%
% Last bit of indexing stuff, for now: I'll typeset the index in two columns
% (the default is three, which makes them too narrow for my tastes).
%
%    \begin{macrocode}
\setcounter{IndexColumns}{2}
%    \end{macrocode}
%
%
% \subsection{Selectively defining things}
%
% I don't want to tread on anyone's toes if they redefine any of these
% commands and things in a configuration file.  The following definitions
% are fairly evil, but should do the job OK.
%
% \begin{macro}{\@gobbledef}
%
% This macro eats the following |\def|inition, leaving not a trace behind.
%
%    \begin{macrocode}
\def\@gobbledef#1#{\@gobble}
%    \end{macrocode}
%
% \end{macro}
%
% \begin{macro}{\tdef}
% \begin{macro}{\tlet}
%
% The |\tdef| command is a sort of `tentative' definition -- it's like
% |\def| if the control sequence named doesn't already have a definition.
% |\tlet| does the same thing with |\let|.
%
%    \begin{macrocode}
\def\tdef#1{
  \ifx#1\@@undefined%
    \expandafter\def\expandafter#1%
  \else%
    \expandafter\@gobbledef%
  \fi%
}
\def\tlet#1#2{\ifx#1\@@undefined\let#1=#2\fi}
%    \end{macrocode}
%
% \end{macro}
% \end{macro}
%
%
% \subsection{General markup things}
%
% Now for some really simple things.  I'll define how to typeset package
% names and environment names (both in the sans serif font, for now).
%
%    \begin{macrocode}
\tlet\package\textsf
\tlet\env\textsf
%    \end{macrocode}
%
% I'll define the |\<|\dots|>| shortcut for syntax items suggested in the
% \package{syntax} package.
%
%    \begin{macrocode}
\tdef\<#1>{\synt{#1}}
%    \end{macrocode}
%
% And because it's used in a few places (mainly for typesetting lengths),
% here's a command for typesetting fractions in text.
%
%    \begin{macrocode}
\tdef\smallf#1/#2{\ensuremath{^{#1}\!/\!_{#2}}}
%    \end{macrocode}
%
%
% \subsection{A table environment}
%
% \begin{environment}{tab}
%
% Most of the packages don't use the (obviously perfect) \package{mdwtab}
% package, because it's big, and takes a while to load.  Here's an
% environment for typesetting centred tables.  The first (optional) argument
% is some declarations to perform.  The mandatory argument is the table
% preamble (obviously).
%
%    \begin{macrocode}
\@ifundefined{tab}{%
  \newenvironment{tab}[2][\relax]{%
    \par\vskip2ex%
    \centering%
    #1%
    \begin{tabular}{#2}%
  }{%
    \end{tabular}%
    \par\vskip2ex%
  }
}{}
%    \end{macrocode}
%
% \end{environment}
%
%
% \subsection{Commenting out of stuff}
%
% \begin{environment}{meta-comment}
%
% Using |\iffalse|\dots|\fi| isn't much fun.  I'll define a gobbling
% environment using the \package{sverb} stuff.
%
%    \begin{macrocode}
\ignoreenv{meta-comment}
%    \end{macrocode}
%
% \end{environment}
%
%
% \subsection{Float handling}
%
% This gubbins will try to avoid float pages as much as possible, and (with
% any luck) encourage floats to be put on the same pages as text.
%
%    \begin{macrocode}
\def\textfraction{0.1}
\def\topfraction{0.9}
\def\bottomfraction{0.9}
\def\floatpagefraction{0.7}
%    \end{macrocode}
%
% Now redefine the default float-placement parameters to allow `here' floats.
%
%    \begin{macrocode}
\def\fps@figure{htbp}
\def\fps@table{htbp}
%    \end{macrocode}
%
%
% \subsection{Other bits of parameter tweaking}
%
% Make \env{grammar} environments look pretty, by indenting the left hand
% sides by a large amount.
%
%    \begin{macrocode}
\grammarindent1in
%    \end{macrocode}
%
% I don't like being told by \TeX\ that my paragraphs are hard to linebreak:
% I know this already.  This lot should shut \TeX\ up about most problems.
%
%    \begin{macrocode}
\sloppy
\hbadness\@M
\hfuzz10\p@
%    \end{macrocode}
%
% Also make \TeX\ shut up in the index.  The \package{multicol} package
% irritatingly plays with |\hbadness|.  This is the best hook I could find
% for playing with this setting.
%
%    \begin{macrocode}
\expandafter\def\expandafter\IndexParms\expandafter{%
  \IndexParms%
  \hbadness\@M%
}
%    \end{macrocode}
%
% The other thing I really don't like is `Marginpar moved' warnings.  This
% will get rid of them, and lots of other \LaTeX\ warnings at the same time.
%
%    \begin{macrocode}
\let\@latex@warning@no@line\@gobble
%    \end{macrocode}
%
% Put some extra space between table rows, please.
%
%    \begin{macrocode}
\def\arraystretch{1.2}
%    \end{macrocode}
%
% Most of the code is at guard level one, so typeset that in upright text.
%
%    \begin{macrocode}
\setcounter{StandardModuleDepth}{1}
%    \end{macrocode}
%
%
% \subsection{Contents handling}
%
% I use at least one contents file (the main table of contents) although
% I may want more.  I'll keep a list of contents files which I need to
% handle.
%
% There are two things I need to do to contents files here:
% \begin{itemize}
% \item I must typeset the table of contents at the beginning of the
%       document; and
% \item I want to typeset tables of contents in two columns (using the
%       \package{multicol} package).
% \end{itemize}
%
% The list consists of items of the form
% \syntax{"\\do{"<extension>"}{"<command>"}"}, where \<extension> is the
% file extension of the contents file, and \<command> is the command to
% typeset it.
%
% \begin{macro}{\docontents}
%
% This is where I keep the list of contents files.  I'll initialise it to
% just do the standard contents table.
%
%    \begin{macrocode}
\def\docontents{\do{toc}{\tableofcontents}}
%    \end{macrocode}
%
% \end{macro}
%
% \begin{macro}{\addcontents}
%
% By saying \syntax{"\\addcontents{"<extension>"}{"<command>"}"}, a document
% can register a new table of contents which gets given the two-column
% treatment properly.  This is really easy to implement.
%
%    \begin{macrocode}
\def\addcontents#1#2{%
  \toks@\expandafter{\docontents\do{#1}{#2}}%
  \edef\docontents{\the\toks@}%
}
%    \end{macrocode}
%
% \end{macro}
%
%
% \subsection{Finishing it all off}
%
% \begin{macro}{\finalstuff}
%
% The |\finalstuff| macro is a hook for doing things at the end of the
% document.  Currently, it inputs the licence agreement as an appendix.
%
%    \begin{macrocode}
\tdef\finalstuff{\appendix\part*{Appendix}% \iffalse <meta-comment>
%
% $Id: gpl.tex,v 1.1 1996/11/19 20:51:14 mdw Exp $
%
% The GNU General Public Licence as a LaTeX section
%
% (c) 1989, 1991 Free Software Foundation, Inc.
%   LaTeX markup and minor formatting changes by Mark Wooding
%

%----- Revision history -----------------------------------------------------
%
% $Log: gpl.tex,v $
% Revision 1.1  1996/11/19 20:51:14  mdw
% Initial revision
%

% --- Chapter heading ---
%
% We don't know whether this ought to be a section or a chapter.  Easy.
% We'll see if chapters are possible.
%
% \fi

\begingroup
\makeatletter

\edef\next#1#2#3{\relax
  \ifx\chapter\@@undefined
    \ifx\documentclass\@notprerr#2\else#3\fi
  \else#1\fi
}

\expandafter\endgroup\next
{
  \let\gpltoplevel\chapter
  \let\gplsec\section
  \let\gplend\endinput
}{
  \let\gpltoplevel\section
  \let\gplsec\subsection
  \let\gplend\endinput
}{
  \documentclass[a4paper]{article}
  \def\gpltoplevel#1{%
    \vspace*{1in}%
    \hbox to\hsize{\hfil\LARGE\bfseries#1\hfil}%
    \vspace{1in}%
  }
  \let\gplsec\section
  \def\gplend{\end{document}}
  \advance\textwidth1in
  \advance\oddsidemargin-.5in
  \sloppy
  \begin{document}
}

%^^A-------------------------------------------------------------------------
\gpltoplevel{The GNU General Public Licence}


The following is the text of the GNU General Public Licence, under the terms
of which this software is distrubuted.

\vspace{12pt}

\begin{center}
\textbf{GNU GENERAL PUBLIC LICENSE} \\
Version 2, June 1991
\end{center}

\begin{center}
Copyright (C) 1989, 1991 Free Software Foundation, Inc. \\
675 Mass Ave, Cambridge, MA 02139, USA

Everyone is permitted to copy and distribute verbatim copies \\
of this license document, but changing it is not allowed.
\end{center}


\gplsec{Preamble}

The licenses for most software are designed to take away your freedom to
share and change it.  By contrast, the GNU General Public License is intended
to guarantee your freedom to share and change free software---to make sure
the software is free for all its users.  This General Public License applies
to most of the Free Software Foundation's software and to any other program
whose authors commit to using it.  (Some other Free Software Foundation
software is covered by the GNU Library General Public License instead.)  You
can apply it to your programs, too.

When we speak of free software, we are referring to freedom, not price.  Our
General Public Licenses are designed to make sure that you have the freedom
to distribute copies of free software (and charge for this service if you
wish), that you receive source code or can get it if you want it, that you
can change the software or use pieces of it in new free programs; and that
you know you can do these things.

To protect your rights, we need to make restrictions that forbid anyone to
deny you these rights or to ask you to surrender the rights.  These
restrictions translate to certain responsibilities for you if you distribute
copies of the software, or if you modify it.

For example, if you distribute copies of such a program, whether gratis or
for a fee, you must give the recipients all the rights that you have.  You
must make sure that they, too, receive or can get the source code.  And you
must show them these terms so they know their rights.

We protect your rights with two steps: (1) copyright the software, and (2)
offer you this license which gives you legal permission to copy, distribute
and/or modify the software.

Also, for each author's protection and ours, we want to make certain that
everyone understands that there is no warranty for this free software.  If
the software is modified by someone else and passed on, we want its
recipients to know that what they have is not the original, so that any
problems introduced by others will not reflect on the original authors'
reputations.

Finally, any free program is threatened constantly by software patents.  We
wish to avoid the danger that redistributors of a free program will
individually obtain patent licenses, in effect making the program
proprietary.  To prevent this, we have made it clear that any patent must be
licensed for everyone's free use or not licensed at all.

The precise terms and conditions for copying, distribution and modification
follow.


\gplsec{Terms and conditions for copying, distribution and modification}

\begin{enumerate}

\makeatletter \setcounter{\@listctr}{-1} \makeatother

\item [0.] This License applies to any program or other work which contains a
      notice placed by the copyright holder saying it may be distributed
      under the terms of this General Public License.  The ``Program'',
      below, refers to any such program or work, and a ``work based on the
      Program'' means either the Program or any derivative work under
      copyright law: that is to say, a work containing the Program or a
      portion of it, either verbatim or with modifications and/or translated
      into another language.  (Hereinafter, translation is included without
      limitation in the term ``modification''.)  Each licensee is addressed
      as ``you''.

      Activities other than copying, distribution and modification are not
      covered by this License; they are outside its scope.  The act of
      running the Program is not restricted, and the output from the Program
      is covered only if its contents constitute a work based on the Program
      (independent of having been made by running the Program).  Whether that
      is true depends on what the Program does.

\item [1.] You may copy and distribute verbatim copies of the Program's
      source code as you receive it, in any medium, provided that you
      conspicuously and appropriately publish on each copy an appropriate
      copyright notice and disclaimer of warranty; keep intact all the
      notices that refer to this License and to the absence of any warranty;
      and give any other recipients of the Program a copy of this License
      along with the Program.

      You may charge a fee for the physical act of transferring a copy, and
      you may at your option offer warranty protection in exchange for a fee.

\item [2.] You may modify your copy or copies of the Program or any portion
      of it, thus forming a work based on the Program, and copy and
      distribute such modifications or work under the terms of Section 1
      above, provided that you also meet all of these conditions:

      \begin{enumerate}

      \item [(a)] You must cause the modified files to carry prominent
            notices stating that you changed the files and the date of any
            change.

      \item [(b)] You must cause any work that you distribute or publish,
            that in whole or in part contains or is derived from the Program
            or any part thereof, to be licensed as a whole at no charge to
            all third parties under the terms of this License.

      \item [(c)] If the modified program normally reads commands
            interactively when run, you must cause it, when started running
            for such interactive use in the most ordinary way, to print or
            display an announcement including an appropriate copyright notice
            and a notice that there is no warranty (or else, saying that you
            provide a warranty) and that users may redistribute the program
            under these conditions, and telling the user how to view a copy
            of this License.  (Exception: if the Program itself is
            interactive but does not normally print such an announcement,
            your work based on the Program is not required to print an
            announcement.)

      \end{enumerate}

      These requirements apply to the modified work as a whole.  If
      identifiable sections of that work are not derived from the Program,
      and can be reasonably considered independent and separate works in
      themselves, then this License, and its terms, do not apply to those
      sections when you distribute them as separate works.  But when you
      distribute the same sections as part of a whole which is a work based
      on the Program, the distribution of the whole must be on the terms of
      this License, whose permissions for other licensees extend to the
      entire whole, and thus to each and every part regardless of who wrote
      it.

      Thus, it is not the intent of this section to claim rights or contest
      your rights to work written entirely by you; rather, the intent is to
      exercise the right to control the distribution of derivative or
      collective works based on the Program.

      In addition, mere aggregation of another work not based on the Program
      with the Program (or with a work based on the Program) on a volume of a
      storage or distribution medium does not bring the other work under the
      scope of this License.

\item [3.] You may copy and distribute the Program (or a work based on it,
      under Section 2) in object code or executable form under the terms of
      Sections 1 and 2 above provided that you also do one of the following:

      \begin{enumerate}

      \item [(a)] Accompany it with the complete corresponding
            machine-readable source code, which must be distributed under the
            terms of Sections 1 and 2 above on a medium customarily used for
            software interchange; or,

      \item [(b)] Accompany it with a written offer, valid for at least three
            years, to give any third party, for a charge no more than your
            cost of physically performing source distribution, a complete
            machine-readable copy of the corresponding source code, to be
            distributed under the terms of Sections 1 and 2 above on a medium
            customarily used for software interchange; or,

      \item [(c)] Accompany it with the information you received as to the
            offer to distribute corresponding source code.  (This alternative
            is allowed only for noncommercial distribution and only if you
            received the program in object code or executable form with such
            an offer, in accord with Subsection b above.)

      \end{enumerate}

      The source code for a work means the preferred form of the work for
      making modifications to it.  For an executable work, complete source
      code means all the source code for all modules it contains, plus any
      associated interface definition files, plus the scripts used to control
      compilation and installation of the executable.  However, as a special
      exception, the source code distributed need not include anything that
      is normally distributed (in either source or binary form) with the
      major components (compiler, kernel, and so on) of the operating system
      on which the executable runs, unless that component itself accompanies
      the executable.

      If distribution of executable or object code is made by offering access
      to copy from a designated place, then offering equivalent access to
      copy the source code from the same place counts as distribution of the
      source code, even though third parties are not compelled to copy the
      source along with the object code.

\item [4.] You may not copy, modify, sublicense, or distribute the Program
      except as expressly provided under this License.  Any attempt otherwise
      to copy, modify, sublicense or distribute the Program is void, and will
      automatically terminate your rights under this License.  However,
      parties who have received copies, or rights, from you under this
      License will not have their licenses terminated so long as such parties
      remain in full compliance.

\item [5.] You are not required to accept this License, since you have not
      signed it.  However, nothing else grants you permission to modify or
      distribute the Program or its derivative works.  These actions are
      prohibited by law if you do not accept this License.  Therefore, by
      modifying or distributing the Program (or any work based on the
      Program), you indicate your acceptance of this License to do so, and
      all its terms and conditions for copying, distributing or modifying the
      Program or works based on it.

\item [6.] Each time you redistribute the Program (or any work based on the
      Program), the recipient automatically receives a license from the
      original licensor to copy, distribute or modify the Program subject to
      these terms and conditions.  You may not impose any further
      restrictions on the recipients' exercise of the rights granted herein.
      You are not responsible for enforcing compliance by third parties to
      this License.

\item [7.] If, as a consequence of a court judgment or allegation of patent
      infringement or for any other reason (not limited to patent issues),
      conditions are imposed on you (whether by court order, agreement or
      otherwise) that contradict the conditions of this License, they do not
      excuse you from the conditions of this License.  If you cannot
      distribute so as to satisfy simultaneously your obligations under this
      License and any other pertinent obligations, then as a consequence you
      may not distribute the Program at all.  For example, if a patent
      license would not permit royalty-free redistribution of the Program by
      all those who receive copies directly or indirectly through you, then
      the only way you could satisfy both it and this License would be to
      refrain entirely from distribution of the Program.

      If any portion of this section is held invalid or unenforceable under
      any particular circumstance, the balance of the section is intended to
      apply and the section as a whole is intended to apply in other
      circumstances.

      It is not the purpose of this section to induce you to infringe any
      patents or other property right claims or to contest validity of any
      such claims; this section has the sole purpose of protecting the
      integrity of the free software distribution system, which is
      implemented by public license practices.  Many people have made
      generous contributions to the wide range of software distributed
      through that system in reliance on consistent application of that
      system; it is up to the author/donor to decide if he or she is willing
      to distribute software through any other system and a licensee cannot
      impose that choice.

      This section is intended to make thoroughly clear what is believed to
      be a consequence of the rest of this License.

\item [8.] If the distribution and/or use of the Program is restricted in
      certain countries either by patents or by copyrighted interfaces, the
      original copyright holder who places the Program under this License may
      add an explicit geographical distribution limitation excluding those
      countries, so that distribution is permitted only in or among countries
      not thus excluded.  In such case, this License incorporates the
      limitation as if written in the body of this License.

\item [9.] The Free Software Foundation may publish revised and/or new
      versions of the General Public License from time to time.  Such new
      versions will be similar in spirit to the present version, but may
      differ in detail to address new problems or concerns.

      Each version is given a distinguishing version number.  If the Program
      specifies a version number of this License which applies to it and
      ``any later version'', you have the option of following the terms and
      conditions either of that version or of any later version published by
      the Free Software Foundation.  If the Program does not specify a
      version number of this License, you may choose any version ever
      published by the Free Software Foundation.

\item [10.] If you wish to incorporate parts of the Program into other free
      programs whose distribution conditions are different, write to the
      author to ask for permission.  For software which is copyrighted by the
      Free Software Foundation, write to the Free Software Foundation; we
      sometimes make exceptions for this.  Our decision will be guided by the
      two goals of preserving the free status of all derivatives of our free
      software and of promoting the sharing and reuse of software generally.

\begin{center}
NO WARRANTY
\end{center}

\bfseries

\item [11.] Because the Program is licensed free of charge, there is no
      warranty for the Program, to the extent permitted by applicable law.
      except when otherwise stated in writing the copyright holders and/or
      other parties provide the program ``as is'' without warranty of any
      kind, either expressed or implied, including, but not limited to, the
      implied warranties of merchantability and fitness for a particular
      purpose.  The entire risk as to the quality and performance of the
      Program is with you.  Should the Program prove defective, you assume
      the cost of all necessary servicing, repair or correction.

\item [12.] In no event unless required by applicable law or agreed to in
      writing will any copyright holder, or any other party who may modify
      and/or redistribute the program as permitted above, be liable to you
      for damages, including any general, special, incidental or
      consequential damages arising out of the use or inability to use the
      program (including but not limited to loss of data or data being
      rendered inaccurate or losses sustained by you or third parties or a
      failure of the Program to operate with any other programs), even if
      such holder or other party has been advised of the possibility of such
      damages.

\end{enumerate}

\begin{center}
\textbf{END OF TERMS AND CONDITIONS}
\end{center}


\gplsec{Appendix: How to Apply These Terms to Your New Programs}

If you develop a new program, and you want it to be of the greatest possible
use to the public, the best way to achieve this is to make it free software
which everyone can redistribute and change under these terms.

To do so, attach the following notices to the program.  It is safest to
attach them to the start of each source file to most effectively convey the
exclusion of warranty; and each file should have at least the ``copyright''
line and a pointer to where the full notice is found.

\begin{verbatim}
<one line to give the program's name and a brief idea of what it does.>
Copyright (C) 19yy  <name of author>

This program is free software; you can redistribute it and/or modify
it under the terms of the GNU General Public License as published by
the Free Software Foundation; either version 2 of the License, or
(at your option) any later version.

This program is distributed in the hope that it will be useful,
but WITHOUT ANY WARRANTY; without even the implied warranty of
MERCHANTABILITY or FITNESS FOR A PARTICULAR PURPOSE.  See the
GNU General Public License for more details.

You should have received a copy of the GNU General Public License
along with this program; if not, write to the Free Software
Foundation, Inc., 675 Mass Ave, Cambridge, MA 02139, USA.
\end{verbatim}

Also add information on how to contact you by electronic and paper mail.

If the program is interactive, make it output a short notice like this when
it starts in an interactive mode:

\begin{verbatim}
Gnomovision version 69, Copyright (C) 19yy name of author
Gnomovision comes with ABSOLUTELY NO WARRANTY; for details type `show w'.
This is free software, and you are welcome to redistribute it
under certain conditions; type `show c' for details.
\end{verbatim}

The hypothetical commands `show w' and `show c' should show the appropriate
parts of the General Public License.  Of course, the commands you use may be
called something other than `show w' and `show c'; they could even be
mouse-clicks or menu items--whatever suits your program.

You should also get your employer (if you work as a programmer) or your
school, if any, to sign a ``copyright disclaimer'' for the program, if
necessary.  Here is a sample; alter the names:

\begin{verbatim}
Yoyodyne, Inc., hereby disclaims all copyright interest in the program
`Gnomovision' (which makes passes at compilers) written by James Hacker.

<signature of Ty Coon>, 1 April 1989
Ty Coon, President of Vice
\end{verbatim}

This General Public License does not permit incorporating your program into
proprietary programs.  If your program is a subroutine library, you may
consider it more useful to permit linking proprietary applications with the
library.  If this is what you want to do, use the GNU Library General Public
License instead of this License.

\gplend
}
%    \end{macrocode}
%
% \end{macro}
%
% \begin{macro}{\implementation}
%
% The |\implementation| macro starts typesetting the implementation of
% the package(s).  If we're not doing the implementation, it just does
% this lot and ends the input file.
%
% I define a macro with arguments inside the |\StopEventually|, which causes
% problems, since the code gets put through an extra level of |\def|fing
% depending on whether the implementation stuff gets typeset or not.  I'll
% store the code I want to do in a separate macro.
%
%    \begin{macrocode}
\def\implementation{\StopEventually{\attheend}}
%    \end{macrocode}
%
% Now for the actual activity.  First, I'll do the |\finalstuff|.  Then, if
% \package{doc}'s managed to find the \package{multicol} package, I'll add
% the end of the environment to the end of each contents file in the list.
% Finally, I'll read the index in from its formatted |.ind| file.
%
%    \begin{macrocode}
\tdef\attheend{%
  \finalstuff%
  \ifhave@multicol%
    \def\do##1##2{\addtocontents{##1}{\protect\end{multicols}}}%
    \docontents%
  \fi%
  \PrintIndex%
}
%    \end{macrocode}
%
% \end{macro}
%
%
% \subsection{File version information}
%
% \begin{macro}{\mdwpkginfo}
%
% For setting up the automatic titles, I'll need to be able to work out
% file versions and things.  This macro will, given a file name, extract
% from \LaTeX\ the version information and format it into a sensible string.
%
% First of all, I'll put the original string (direct from the
% |\Provides|\dots\ command).  Then I'll pass it to another macro which can
% parse up the string into its various bits, along with the original
% filename.
%
%    \begin{macrocode}
\def\mdwpkginfo#1{%
  \edef\@tempa{\csname ver@#1\endcsname}%
  \expandafter\mdwpkginfo@i\@tempa\@@#1\@@%
}
%    \end{macrocode}
%
% Now for the real business.  I'll store the string I build in macros called
% \syntax{"\\"<filename>"date", "\\"<filename>"version" and
% "\\"<filename>"info"}, which store the file's date, version and
% `information string' respectively.  (Note that the file extension isn't
% included in the name.)
%
% This is mainly just tedious playing with |\expandafter|.  The date format
% is defined by a separate macro, which can be modified from the
% configuration file.
%
%    \begin{macrocode}
\def\mdwpkginfo@i#1/#2/#3 #4 #5\@@#6.#7\@@{%
  \expandafter\def\csname #6date\endcsname%
    {\protect\mdwdateformat{#1}{#2}{#3}}%
  \expandafter\def\csname #6version\endcsname{#4}%
  \expandafter\def\csname #6info\endcsname{#5}%
}
%    \end{macrocode}
%
% \end{macro}
%
% \begin{macro}{\mdwdateformat}
%
% Given three arguments, a year, a month and a date (all numeric), build a
% pretty date string.  This is fairly simple really.
%
%    \begin{macrocode}
\tdef\mdwdateformat#1#2#3{\number#3\ \monthname{#2}\ \number#1}
\def\monthname#1{%
  \ifcase#1\or%
     January\or February\or March\or April\or May\or June\or%
     July\or August\or September\or October\or November\or December%
  \fi%
}
\def\numsuffix#1{%
  \ifnum#1=1 st\else%
  \ifnum#1=2 nd\else%
  \ifnum#1=3 rd\else%
  \ifnum#1=21 st\else%
  \ifnum#1=22 nd\else%
  \ifnum#1=23 rd\else%
  \ifnum#1=31 st\else%
  th%
  \fi\fi\fi\fi\fi\fi\fi%
}
%    \end{macrocode}
%
% \end{macro}
%
% \begin{macro}{\mdwfileinfo}
%
% Saying \syntax{"\\mdwfileinfo{"<file-name>"}{"<info>"}"} extracts the
% wanted item of \<info> from the version information for file \<file-name>.
%
%    \begin{macrocode}
\def\mdwfileinfo#1#2{\mdwfileinfo@i{#2}#1.\@@}
\def\mdwfileinfo@i#1#2.#3\@@{\csname#2#1\endcsname}
%    \end{macrocode}
%
% \end{macro}
%
%
% \subsection{List handling}
%
% There are several other lists I need to build.  These macros will do
% the necessary stuff.
%
% \begin{macro}{\mdw@ifitem}
%
% The macro \syntax{"\\mdw@ifitem"<item>"\\in"<list>"{"<true-text>"}"^^A
% "{"<false-text>"}"} does \<true-text> if the \<item> matches any item in
% the \<list>; otherwise it does \<false-text>.
%
%    \begin{macrocode}
\def\mdw@ifitem#1\in#2{%
  \@tempswafalse%
  \def\@tempa{#1}%
  \def\do##1{\def\@tempb{##1}\ifx\@tempa\@tempb\@tempswatrue\fi}%
  #2%
  \if@tempswa\expandafter\@firstoftwo\else\expandafter\@secondoftwo\fi%
}
%    \end{macrocode}
%
% \end{macro}
%
% \begin{macro}{\mdw@append}
%
% Saying \syntax{"\\mdw@append"<item>"\\to"<list>} adds the given \<item>
% to the end of the given \<list>.
%
%    \begin{macrocode}
\def\mdw@append#1\to#2{%
  \toks@{\do{#1}}%
  \toks\tw@\expandafter{#2}%
  \edef#2{\the\toks\tw@\the\toks@}%
}
%    \end{macrocode}
%
% \end{macro}
%
% \begin{macro}{\mdw@prepend}
%
% Saying \syntax{"\\mdw@prepend"<item>"\\to"<list>} adds the \<item> to the
% beginning of the \<list>.
%
%    \begin{macrocode}
\def\mdw@prepend#1\to#2{%
  \toks@{\do{#1}}%
  \toks\tw@\expandafter{#2}%
  \edef#2{\the\toks@\the\toks\tw@}%
}
%    \end{macrocode}
%
% \end{macro}
%
% \begin{macro}{\mdw@add}
%
% Finally, saying \syntax{"\\mdw@add"<item>"\\to"<list>} adds the \<item>
% to the list only if it isn't there already.
%
%    \begin{macrocode}
\def\mdw@add#1\to#2{\mdw@ifitem#1\in#2{}{\mdw@append#1\to#2}}
%    \end{macrocode}
%
% \end{macro}
%
%
% \subsection{Described file handling}
%
% I'l maintain lists of packages, document classes, and other files
% described by the current documentation file.
%
% First of all, I'll declare the various list macros.
%
%    \begin{macrocode}
\def\dopackages{}
\def\doclasses{}
\def\dootherfiles{}
%    \end{macrocode}
%
% \begin{macro}{\describespackage}
%
% A document file can declare that it describes a package by saying
% \syntax{"\\describespackage{"<package-name>"}"}.  I add the package to
% my list, read the package into memory (so that the documentation can
% offer demonstrations of it) and read the version information.
%
%    \begin{macrocode}
\def\describespackage#1{%
  \mdw@ifitem#1\in\dopackages{}{%
    \mdw@append#1\to\dopackages%
    \usepackage{#1}%
    \mdwpkginfo{#1.sty}%
  }%
}
%    \end{macrocode}
%
% \end{macro}
%
% \begin{macro}{\describesclass}
%
% By saying \syntax{"\\describesclass{"<class-name>"}"}, a document file
% can declare that it describes a document class.  I'll assume that the
% document class is already loaded, because it's much too late to load
% it now.
%
%    \begin{macrocode}
\def\describesclass#1{\mdw@add#1\to\doclasses\mdwpkginfo{#1.cls}}
%    \end{macrocode}
%
% \end{macro}
%
% \begin{macro}{\describesfile}
%
% Finally, other `random' files, which don't have the status of real \LaTeX\
% packages or document classes, can be described by saying \syntax{^^A
% "\\describesfile{"<file-name>"}" or "\\describesfile*{"<file-name>"}"}.
% The difference is that the starred version will not |\input| the file.
%
%    \begin{macrocode}
\def\describesfile{%
  \@ifstar{\describesfile@i\@gobble}{\describesfile@i\input}%
}
\def\describesfile@i#1#2{%
  \mdw@ifitem#2\in\dootherfiles{}{%
    \mdw@add#2\to\dootherfiles%
    #1{#2}%
    \mdwpkginfo{#2}%
  }%
}
%    \end{macrocode}
%
% \end{macro}
%
%
% \subsection{Author and title handling}
%
% I'll redefine the |\author| and |\title| commands so that I get told
% whether I need to do it myself.
%
% \begin{macro}{\author}
%
% This is easy: I'll save the old meaning, and then redefine |\author| to
% do the old thing and redefine itself to then do nothing.
%
%    \begin{macrocode}
\let\mdw@author\author
\def\author{\let\author\@gobble\mdw@author}
%    \end{macrocode}
%
% \end{macro}
%
% \begin{macro}{\title}
%
% And oddly enough, I'll do exactly the same thing for the title, except
% that I'll also disable the |\mdw@buildtitle| command, which constructs
% the title automatically.
%
%    \begin{macrocode}
\let\mdw@title\title
\def\title{\let\title\@gobble\let\mdw@buildtitle\relax\mdw@title}
%    \end{macrocode}
%
% \end{macro}
%
% \begin{macro}{\date}
%
% This works in a very similar sort of way.
%
%    \begin{macrocode}
\def\date#1{\let\date\@gobble\def\today{#1}}
%    \end{macrocode}
%
% \end{macro}
%
% \begin{macro}{\datefrom}
%
% Saying \syntax{"\\datefrom{"<file-name>"}"} sets the document date from
% the given filename.
%
%    \begin{macrocode}
\def\datefrom#1{%
  \protected@edef\@tempa{\noexpand\date{\csname #1date\endcsname}}%
  \@tempa%
}
%    \end{macrocode}
%
% \end{macro}
%
% \begin{macro}{\docfile}
%
% Saying \syntax{"\\docfile{"<file-name>"}"} sets up the file name from which
% documentation will be read.
%
%    \begin{macrocode}
\def\docfile#1{%
  \def\@tempa##1.##2\@@{\def\@basefile{##1.##2}\def\@basename{##1}}%
  \edef\@tempb{\noexpand\@tempa#1\noexpand\@@}%
  \@tempb%
}
%    \end{macrocode}
%
% I'll set up a default value as well.
%
%    \begin{macrocode}
\docfile{\jobname.dtx}
%    \end{macrocode}
%
% \end{macro}
%
%
% \subsection{Building title strings}
%
% This is rather tricky.  For each list, I need to build a legible looking
% string.
%
% \begin{macro}{\mdw@addtotitle}
%
% By saying
%\syntax{"\\mdw@addtotitle{"<list>"}{"<command>"}{"<singular>"}{"<plural>"}"}
% I can add the contents of a list to the current title string in the
% |\mdw@title| macro.
%
%    \begin{macrocode}
\tdef\mdw@addtotitle#1#2#3#4{%
%    \end{macrocode}
%
% Now to get to work.  I need to keep one `lookahead' list item, and a count
% of the number of items read so far.  I'll keep the lookahead item in
% |\@nextitem| and the counter in |\count@|.
%
%    \begin{macrocode}
  \count@\z@%
%    \end{macrocode}
%
% Now I'll define what to do for each list item.  The |\protect| command is
% already set up appropriately for playing with |\edef| commands.
%
%    \begin{macrocode}
  \def\do##1{%
%    \end{macrocode}
%
% The first job is to add the previous item to the title string.  If this
% is the first item, though, I'll just add the appropriate \lit{The } or
% \lit{ and the } string to the title (this is stored in the |\@prefix|
% macro).
%
%    \begin{macrocode}
    \edef\mdw@title{%
      \mdw@title%
      \ifcase\count@\@prefix%
      \or\@nextitem%
      \else, \@nextitem%
      \fi%
    }%
%    \end{macrocode}
%
% That was rather easy.  Now I'll set up the |\@nextitem| macro for the
% next time around the loop.
%
%    \begin{macrocode}
    \edef\@nextitem{%
      \protect#2{##1}%
      \protect\footnote{%
        The \protect#2{##1} #3 is currently at version %
        \mdwfileinfo{##1}{version}, dated \mdwfileinfo{##1}{date}.%
      }\space%
    }%
%    \end{macrocode}
%
% Finally, I need to increment the counter.
%
%    \begin{macrocode}
    \advance\count@\@ne%
  }%
%    \end{macrocode}
%
% Now execute the list.
%
%    \begin{macrocode}
  #1%
%    \end{macrocode}
%
% I still have one item left over, unless the list was empty.  I'll add
% that now.
%
%    \begin{macrocode}
  \edef\mdw@title{%
    \mdw@title%
    \ifcase\count@%
    \or\@nextitem\space#3%
    \or\ and \@nextitem\space#4%
    \fi%
  }%
%    \end{macrocode}
%
% Finally, if $|\count@| \ne 0$, I must set |\@prefix| to \lit{ and the }.
%
%    \begin{macrocode}
  \ifnum\count@>\z@\def\@prefix{ and the }\fi%
}
%    \end{macrocode}
%
% \end{macro}
%
% \begin{macro}{\mdw@buildtitle}
%
% This macro will actually do the job of building the title string.
%
%    \begin{macrocode}
\tdef\mdw@buildtitle{%
%    \end{macrocode}
%
% First of all, I'll open a group to avoid polluting the namespace with
% my gubbins (although the code is now much tidier than it has been in
% earlier releases).
%
%    \begin{macrocode}
  \begingroup%
%    \end{macrocode}
%
% The title building stuff makes extensive use of |\edef|.  I'll set
% |\protect| appropriately.  (For those not in the know,
% |\@unexpandable@protect| expands to `|\noexpand\protect\noexpand|',
% which prevents expansion of the following macro, and inserts a |\protect|
% in front of it ready for the next |\edef|.)
%
%    \begin{macrocode}
  \let\@@protect\protect\let\protect\@unexpandable@protect%
%    \end{macrocode}
%
% Set up some simple macros ready for the main code.
%
%    \begin{macrocode}
  \def\mdw@title{}%
  \def\@prefix{The }%
%    \end{macrocode}
%
% Now build the title.  This is fun.
%
%    \begin{macrocode}
  \mdw@addtotitle\dopackages\package{package}{packages}%
  \mdw@addtotitle\doclasses\package{document class}{document classes}%
  \mdw@addtotitle\dootherfiles\texttt{file}{files}%
%    \end{macrocode}
%
% Now I want to end the group and set the title from my string.  The
% following hacking will do this.
%
%    \begin{macrocode}
  \edef\next{\endgroup\noexpand\title{\mdw@title}}%
  \next%
}
%    \end{macrocode}
%
% \end{macro}
%
%
% \subsection{Starting the main document}
%
% \begin{macro}{\mdwdoc}
%
% Once the document preamble has done all of its stuff, it calls the
% |\mdwdoc| command, which takes over and really starts the documentation
% going.
%
%    \begin{macrocode}
\def\mdwdoc{%
%    \end{macrocode}
%
% First, I'll construct the title string.
%
%    \begin{macrocode}
  \mdw@buildtitle%
  \author{Mark Wooding}%
%    \end{macrocode}
%
% Set up the date string based on the date of the package which shares
% the same name as the current file.
%
%    \begin{macrocode}
  \datefrom\@basename%
%    \end{macrocode}
%
% Set up verbatim characters after all the packages have started.
%
%    \begin{macrocode}
  \shortverb\|%
  \shortverb\"%
%    \end{macrocode}
%
% Start the document, and put the title in.
%
%    \begin{macrocode}
  \begin{document}
  \maketitle%
%    \end{macrocode}
%
% This is nasty.  It makes maths displays work properly in demo environments.
% \emph{The \LaTeX\ Companion} exhibits the bug which this hack fixes.  So
% ner.
%
%    \begin{macrocode}
  \abovedisplayskip\z@%
%    \end{macrocode}
%
% Now start the contents tables.  After starting each one, I'll make it
% be multicolumnar.
%
%    \begin{macrocode}
  \def\do##1##2{%
    ##2%
    \ifhave@multicol\addtocontents{##1}{%
      \protect\begin{multicols}{2}%
      \hbadness\@M%
    }\fi%
  }%
  \docontents%
%    \end{macrocode}
%
% Input the main file now.
%
%    \begin{macrocode}
  \DocInput{\@basefile}%
%    \end{macrocode}
%
% That's it.  I'm done.
%
%    \begin{macrocode}
  \end{document}
}
%    \end{macrocode}
%
% \end{macro}
%
%
% \subsection{And finally\dots}
%
% Right at the end I'll put a hook for the configuration file.
%
%    \begin{macrocode}
\ifx\mdwhook\@@undefined\else\expandafter\mdwhook\fi
%    \end{macrocode}
%
% That's all the code done now.  I'll change back to `user' mode, where
% all the magic control sequences aren't allowed any more.
%
%    \begin{macrocode}
\makeatother
%</mdwtools>
%    \end{macrocode}
%
% Oh, wait!  What if I want to typeset this documentation?  Aha.  I'll cope
% with that by comparing |\jobname| with my filename |mdwtools|.  However,
% there's some fun here, because |\jobname| contains category-12 letters,
% while my letters are category-11.  Time to play with |\string| in a messy
% way.
%
%    \begin{macrocode}
%<*driver>
\makeatletter
\edef\@tempa{\expandafter\@gobble\string\mdwtools}
\edef\@tempb{\jobname}
\ifx\@tempa\@tempb
  \describesfile*{mdwtools.tex}
  \docfile{mdwtools.tex}
  \makeatother
  \expandafter\mdwdoc
\fi
\makeatother
%</driver>
%    \end{macrocode}
%
% That's it.  Done!
%
% \hfill Mark Wooding, \today
%
% \Finale
%
\endinput

\describespackage{footnote}
\mdwdoc
%</driver>
%
% \end{meta-comment}
%
% \section{User guide}
%
% This package provides some commands for handling footnotes slightly
% better than \LaTeX\ usually does; there are several commands and
% environments (notably |\parbox|, \env{minipage} and \env{tabular}
% \begin{footnote}
%   The \package{mdwtab} package, provided in this distribution, handles
%   footnotes correctly anyway; it uses an internal version of this package
%   to do so.
% \end{footnote})
% which `trap' footnotes so that they can't escape and appear at the bottom
% of the page.
%
% \DescribeEnv{savenotes}
% The \env{savenotes} environment saves up any footnotes encountered within
% it, and performs them all at the end.
%
% \DescribeMacro{\savenotes}
% \DescribeMacro{\spewnotes}
% If you're defining a command or environment, you can use the |\savenotes|
% command to start saving up footnotes, and the |\spewnotes| command to
% execute them all at the end.  Note that |\savenotes| and |\spewnotes|
% enclose a group, so watch out.  You can safely nest the commands and
% environments -- they work out if they're already working and behave
% appropriately.
%
% \DescribeEnv{minipage*}
% To help things along a bit, the package provides a $*$-version of the
% \env{minipage} environment, which doesn't trap footnotes for itself (and
% in fact sends any footnotes it contains to the bottom of the page, where
% they belong).
%
% \DescribeMacro{\makesavenoteenv}
% The new \env{minipage$*$} environment was created with a magic command
% called |\makesavenoteenv|.  It has a fairly simple syntax:
%
% \begin{grammar}
% <make-save-note-env-cmd> ::= \[[
%   "\\makesavenoteenv"
%   \begin{stack} \\ "[" <new-env-name> "]" \end{stack}
%   "{" <env-name> "}"
% \]]
% \end{grammar}
%
% Without the optional argument, it redefines the named environment so that
% it handles footnotes correctly.  With the optional argument, it makes
% the new environment named by \<new-env-name> into a footnote-friendly
% version of the \<env-name> environment.
%
% \DescribeMacro{\parbox}
% The package also redefines the |\parbox| command so that it works properly
% with footnotes.
%
% \DescribeEnv{footnote}
% The other problem which people tend to experience with footnotes is that
% you can't put verbatim text (with the |\verb| comamnd or the \env{verbatim}
% environment) into the |\footnote| command's argument.  This package
% provides a \env{footnote} \emph{environment}, which \emph{does} allow
% verbatim things.  You use the environment just like you do the command.
% It's really easy.  It even has an optional argument, which works the same
% way.
%
% \DescribeEnv{footnotetext}
% To go with the \env{footnote} environment, there's a \env{footnotetext}
% environment, which just puts the text in the bottom of the page, like
% |\footnotetext| does.
%
% There's a snag with these environments, though.  Some other nonstandard
% environments, like \env{tabularx}, try to handle footnotes their own
% way, because they won't work otherwise.  The way they do this is not
% compatible with the way that the \env{footnote} and \env{footnotetext}
% environments work, and you will get strange results if you try (there'll
% be odd vertical spacing, and the footnote text may well be incorrect).
% \begin{footnote}
%   The solution to this problem is to send mail to David Carlisle persuading
%   him to use this package to handle footnotes, rather than doing it his
%   way.
% \end{footnote}
%
% \implementation
%
% \section{Implementation}
%
% Most implementations of footnote-saving (in particular, that used in
% the \package{tabularx} and \package{longtable} packages) use a token
% list register to store the footnote text, and then expand it when whatever
% was preventing footnotes (usually a vbox) stops.  This is no good at all
% if the footnotes contain things which might not be there by the time the
% expansion occurs.  For example, references to things in temporary boxes
% won't work.
%
% This implementation therefore stores the footnotes up in a box register.
% This must be just as valid as using tokens, because all I'm going to do
% at the end is unbox the box).
%
%    \begin{macrocode}
%<*macro|package>
\ifx\fn@notes\@@undefined%
  \newbox\fn@notes%
\fi
%    \end{macrocode}
%
% I'll need a length to tell me how wide the footnotes should be at the
% moment.
%
%    \begin{macrocode}
\newdimen\fn@width
%    \end{macrocode}
%
% Of course, I can't set this up until I actually start saving footnotes.
% Until then I'll use |\columnwidth| (which works in \package{multicol}
% even though it doesn't have any right to).
%
%    \begin{macrocode}
\let\fn@colwidth\columnwidth
%    \end{macrocode}
%
% And now a switch to remember if we're already handling footnotes,
%
%    \begin{macrocode}
\newif\if@savingnotes
%    \end{macrocode}
%
%
% \subsection{Building footnote text}
%
% I need to emulate \LaTeX's footnote handling when I'm putting the notes
% into my box; this is also useful in the verbatim-in-footnotes stuff.
%
% \begin{macro}{\fn@startnote}
%
% Here's how a footnote gets started.  Most of the code here is stolen
% from |\@footnotetext|.
%
%    \begin{macrocode}
\def\fn@startnote{%
  \hsize\fn@colwidth%
  \interlinepenalty\interfootnotelinepenalty%
  \reset@font\footnotesize%
  \floatingpenalty\@MM% Is this right???
  \@parboxrestore%
  \protected@edef\@currentlabel{\csname p@\@mpfn\endcsname\@thefnmark}%
  \color@begingroup%
}
%    \end{macrocode}
%
% \end{macro}
%
% \begin{macro}{\fn@endnote}
%
% Footnotes are finished off by this macro.  This is the easy bit.
%
%    \begin{macrocode}
\let\fn@endnote\color@endgroup
%    \end{macrocode}
%
% \end{macro}
%
%
% \subsection{Footnote saving}
%
% \begin{macro}{\fn@fntext}
%
% Now to define how to actually do footnotes.  I'll just add the notes to
% the bottom of the footnote box I'm building.
%
% There's some hacking added here to handle the case that a footnote is
% in an |\intertext| command within a broken \package{amsmath} alignment
% environment -- otherwise the footnotes get duplicated due to the way that
% that package measures equations.
% \begin{footnote}
%   The correct solution of course is to
%   implement aligning environments in a sensible way, by building the table
%   and leaving penalties describing the intended format, and then pick that
%   apart in a postprocessing phase.  If I get the time, I'll start working
%   on this again.  I have a design worked out and the beginnings of an
%   implementation, but it's going to be a long time coming.
% \end{footnote}
%
%    \begin{macrocode}
\def\fn@fntext#1{%
  \ifx\ifmeasuring@\@@undefined%
    \expandafter\@secondoftwo\else\expandafter\@iden%
  \fi%
  {\ifmeasuring@\expandafter\@gobble\else\expandafter\@iden\fi}%
  {%
    \global\setbox\fn@notes\vbox{%
      \unvbox\fn@notes%
      \fn@startnote%
      \@makefntext{%
        \rule\z@\footnotesep%
        \ignorespaces%
        #1%
        \@finalstrut\strutbox%
      }%
      \fn@endnote%
    }%
  }%
}
%    \end{macrocode}
%
% \end{macro}
%
% \begin{macro}{\savenotes}
%
% The |\savenotes| declaration starts saving footnotes, to be spewed at a
% later date.  We'll also remember which counter we're meant to use, and
% redefine the footnotes used by minipages.
%
% The idea here is that we'll gather up footnotes within the environment,
% and output them in whatever format they were being typeset outside the
% environment.
%
% I'll take this a bit at a time.  The start is easy: we need a group in
% which to keep our local definitions.
%
%    \begin{macrocode}
\def\savenotes{%
  \begingroup%
%    \end{macrocode}
%
% Now, if I'm already saving footnotes away, I won't bother doing anything
% here.  Otherwise I need to start hacking, and set the switch.
%
%    \begin{macrocode}
  \if@savingnotes\else%
    \@savingnotestrue%
%    \end{macrocode}
%
% I redefine the |\@footnotetext| command, which is responsible for adding
% a footnote to the appropriate insert.  I'll redefine both the current
% version, and \env{minipage}'s specific version, in case there's a nested
% minipage.
%
%    \begin{macrocode}
    \let\@footnotetext\fn@fntext%
    \let\@mpfootnotetext\fn@fntext%
%    \end{macrocode}
%
% I'd better make sure my box is empty before I start, and I must set up
% the column width so that later changes (e.g., in \env{minipage}) don't
% upset things too much.
%
%    \begin{macrocode}
    \fn@width\columnwidth%
    \let\fn@colwidth\fn@width%
    \global\setbox\fn@notes\box\voidb@x%
%    \end{macrocode}
%
% Now for some yuckiness.  I want to ensure that \env{minipage} doesn't
% change how footnotes are handled once I've taken charge.  I'll store the
% current values of |\thempfn| (which typesets a footnote marker) and
% |\@mpfn| (which contains the name of the current footnote counter).
%
%    \begin{macrocode}
    \let\fn@thempfn\thempfn%
    \let\fn@mpfn\@mpfn%
%    \end{macrocode}
%
% The \env{minipage} environment provides a hook, called |\@minipagerestore|.
% Initially it's set to |\relax|, which is unfortunately unexpandable, so if
% I want to add code to it, I must check this possibility.  I'll make it
% |\@empty| (which expands to nothing) if it's still |\relax|.  Then I'll
% add my code to the hook, to override |\thempfn| and |\@mpfn| set up by
% \env{minipage}.
%
% Note that I can't just force the |mpfootnote| counter to be equal to
% the |footnote| one, because \env{minipage} clears |\c@mpfootnote| to zero
% when it starts.  This method will ensure that even so, the current counter
% works OK.
%
%    \begin{macrocode}
    \ifx\@minipagerestore\relax\let\@minipagerestore\@empty\fi%
    \expandafter\def\expandafter\@minipagerestore\expandafter{%
      \@minipagerestore%
      \let\thempfn\fn@thempfn%
      \let\@mpfn\fn@mpfn%
    }%
  \fi%
}
%    \end{macrocode}
%
% \end{macro}
%
% \begin{macro}{\spewnotes}
%
% Now I can spew out the notes we saved.  This is a bit messy, actually.
% Since the standard |\@footnotetext| implementation tries to insert funny
% struts and things, I must be a bit careful.  I'll disable all this bits
% which start paragraphs prematurely.
%
%    \begin{macrocode}
\def\spewnotes{%
  \endgroup%
  \if@savingnotes\else\ifvoid\fn@notes\else\begingroup%
    \let\@makefntext\@empty%
    \let\@finalstrut\@gobble%
    \let\rule\@gobbletwo%
    \@footnotetext{\unvbox\fn@notes}%
  \endgroup\fi\fi%
}
%    \end{macrocode}
%
% \end{macro}
%
% Now make an environment, for users.
%
%    \begin{macrocode}
\let\endsavenotes\spewnotes
%    \end{macrocode}
%
% That's all that needs to be in the shared code section.
%
%    \begin{macrocode}
%</macro|package>
%<*package>
%    \end{macrocode}
%
%
% \subsection{The \env{footnote} environment}
%
% Since |\footnote| is a command with an argument, things like \env{verbatim}
% are unwelcome in it.  Every so often someone on |comp.text.tex| moans
% about it and I post a nasty hack to make it work.  However, as a more
% permanent and `official' solution, here's an environment which does the
% job rather better.  Lots of this is based on code from my latest attempt
% on the newsgroup.
%
% I'll work on this in a funny order, although I think it's easier to
% understand.  First, I'll do some macros for reading the optional argument
% of footnote-related commands.
%
% \begin{macro}{\fn@getmark}
%
% Saying \syntax{"\\fn@getmark{"<default-code>"}{"<cont-code>"}"} will read
% an optional argument giving a value for the footnote counter; if the
% argument isn't there, the \<default-code> is executed, and it's expected
% to set up the appropriate counter to the current value.  The footnote
% marker text is stored in the macro |\@thefnmark|, as is conventional for
% \LaTeX's footnote handling macros.  Once this is done properly, the
% \<cont-code> is called to continue handling things.
%
% Since the handling of the optional argument plays with the footnote
% counter locally, I'll start a group right now to save some code.  Then I'll
% decide what to do based on the presence of the argument.
%
%    \begin{macrocode}
\def\fn@getmark#1#2{%
  \begingroup%
  \@ifnextchar[%
    {\fn@getmark@i{#1}}%
    {#1\fn@getmark@ii{#2}}%
}
%    \end{macrocode}
%
% There's an optional argument, so I need to read it and assign it to the
% footnote counter.
%
%    \begin{macrocode}
\def\fn@getmark@i#1[#2]{%
  \csname c@\@mpfn\endcsname#2%
  \fn@getmark@ii%
}
%    \end{macrocode}
%
% Finally, set up the macro properly, and end the group.
%
%    \begin{macrocode}
\def\fn@getmark@ii#1{%
  \unrestored@protected@xdef\@thefnmark{\thempfn}%
  \endgroup%
  #1%
}
%    \end{macrocode}
%
% \end{macro}
%
% From argument reading, I'll move on to footnote typesetting.
%
% \begin{macro}{\fn@startfntext}
%
% The |\fn@startfntext| macro sets everything up for building the footnote
% in a box register, ready for unboxing into the footnotes insert.  The
% |\fn@prefntext| macro is a style hook I'll set up later.
%
%    \begin{macrocode}
\def\fn@startfntext{%
  \setbox\z@\vbox\bgroup%
    \fn@startnote%
    \fn@prefntext%
    \rule\z@\footnotesep%
    \ignorespaces%
}
%    \end{macrocode}
%
% \end{macro}
%
% \begin{macro}{\fn@endfntext}
%
% Now I'll end the vbox, and add it to the footnote insertion.  Again, I
% must be careful to prevent |\@footnotetext| from adding horizontal mode
% things in bad places.
%
%    \begin{macrocode}
\def\fn@endfntext{%
    \@finalstrut\strutbox%
    \fn@postfntext%
  \egroup%
  \begingroup%
    \let\@makefntext\@empty%
    \let\@finalstrut\@gobble%
    \let\rule\@gobbletwo%
    \@footnotetext{\unvbox\z@}%
  \endgroup%
}
%    \end{macrocode}
%
% \end{macro}
%
% \begin{environment}{footnote}
%
% I can now start on the environment proper.  First I'll look for an
% optional argument.
%
%    \begin{listing}
%\def\footnote{%
%    \end{listing}
%
% Oh.  I've already come up against the first problem: that name's already
% used.  I'd better save the original version.
%
%    \begin{macrocode}
\let\fn@latex@@footnote\footnote
%    \end{macrocode}
%
% The best way I can think of for seeing if I'm in an environment is to
% look at |\@currenvir|.  I'll need something to compare with, then.
%
%    \begin{macrocode}
\def\fn@footnote{footnote}
%    \end{macrocode}
%
% Now to start properly.  |;-)|
%
%    \begin{macrocode}
\def\footnote{%
  \ifx\@currenvir\fn@footnote%
    \expandafter\@firstoftwo%
  \else%
    \expandafter\@secondoftwo%
  \fi%
  {\fn@getmark{\stepcounter\@mpfn}%
              {\leavevmode\unskip\@footnotemark\fn@startfntext}}%
  {\fn@latex@@footnote}%
}
%    \end{macrocode}
%
% Ending the environment is simple.
%
%    \begin{macrocode}
\let\endfootnote\fn@endfntext
%    \end{macrocode}
%
% \end{environment}
%
% \begin{environment}{footnotetext}
%
% I'll do the same magic as before for |\footnotetext|.
%
%    \begin{macrocode}
\def\fn@footnotetext{footnotetext}
\let\fn@latex@@footnotetext\footnotetext
\def\footnotetext{%
  \ifx\@currenvir\fn@footnotetext%
    \expandafter\@firstoftwo%
  \else%
    \expandafter\@secondoftwo%
  \fi%
  {\fn@getmark{}\fn@startfntext}%
  {\fn@latex@@footnotetext}%
}
\let\endfootnotetext\endfootnote
%    \end{macrocode}
%
% \end{environment}
%
% \begin{macro}{\fn@prefntext}
% \begin{macro}{\fn@postfntext}
%
% Now for one final problem.  The style hook for footnotes is the command
% |\@makefntext|, which takes the footnote text as its argument.  Clearly
% this is utterly unsuitable, so I need to split it into two bits, where
% the argument is.  This is very tricky, and doesn't deserve to work,
% although it appears to be a good deal more effective than it has any right
% to be.
%
%    \begin{macrocode}
\long\def\@tempa#1\@@#2\@@@{\def\fn@prefntext{#1}\def\fn@postfntext{#2}}
\expandafter\@tempa\@makefntext\@@\@@@
%    \end{macrocode}
%
% \end{macro}
% \end{macro}
%
%
% \subsection{Hacking existing environments}
%
% Some existing \LaTeX\ environments ought to have footnote handling but
% don't.  Now's our chance.
%
% \begin{macro}{\makesavenoteenv}
%
% The |\makesavenoteenv| command makes an environment save footnotes around
% itself.
%
% It would also be nice to make |\parbox| work with footnotes.  I'll do this
% later.
%
%    \begin{macrocode}
\def\makesavenoteenv{\@ifnextchar[\fn@msne@ii\fn@msne@i}
%    \end{macrocode}
%
% We're meant to redefine the environment.  We'll copy it (using |\let|) to
% a magic name, and then pass it on to stage~2.
%
%    \begin{macrocode}
\def\fn@msne@i#1{%
  \expandafter\let\csname msne$#1\expandafter\endcsname%
                  \csname #1\endcsname%
  \expandafter\let\csname endmsne$#1\expandafter\endcsname%
                  \csname end#1\endcsname%
  \fn@msne@ii[#1]{msne$#1}%
}
%    \end{macrocode}
%
% Now we'll define the new environment.  The start is really easy, since we
% just need to insert a |\savenotes|.  The end is more complex, since we
% need to preserve the |\if@endpe| flag so that |\end| can pick it up.  I
% reckon that proper hooks should be added to |\begin| and |\end| so that
% environments can define things to be done outside the main group as
% well as within it; still, we can't all have what we want, can we?
%
%    \begin{macrocode}
\def\fn@msne@ii[#1]#2{%
  \expandafter\edef\csname#1\endcsname{%
    \noexpand\savenotes%
    \expandafter\noexpand\csname#2\endcsname%
  }%
  \expandafter\edef\csname end#1\endcsname{%
    \expandafter\noexpand\csname end#2\endcsname%
    \noexpand\expandafter%
    \noexpand\spewnotes%
    \noexpand\if@endpe\noexpand\@endpetrue\noexpand\fi%
  }%
}
%    \end{macrocode}
%
% \end{macro}
%
% \begin{environment}{minipage*}
%
% Let's define a \env{minipage$*$} environment which handles footnotes
% nicely.  Really easy:
%
%    \begin{macrocode}
\makesavenoteenv[minipage*]{minipage}
%    \end{macrocode}
%
% \end{environment}
%
% \begin{macro}{\parbox}
%
% Now to alter |\parbox| slightly, so that it handles footnotes properly.
% I'm going to do this fairly inefficiently, because I'm going to try and
% change it as little as possible.
%
% First, I'll save the old |\parbox| command.  If I don't find a \lit{*},
% I'll just call this command.
%
%    \begin{macrocode}
\let\fn@parbox\parbox
%    \end{macrocode}
%
% This is the clever bit: I don't know how many optional arguments
% Mr~Mittelbach and his chums will add to |\parbox|, so I'll handle any
% number.  I'll store them all up in my first argument and call myself
% every time I find a new one.  If I run out of optional arguments,
% I'll call the original |\parbox| command, surrounding it with |\savenotes|
% and |\spewnotes|.
%
%    \begin{macrocode}
\def\parbox{\@ifnextchar[{\fn@parbox@i{}}{\fn@parbox@ii{}}}
\def\fn@parbox@i#1[#2]{%
  \@ifnextchar[{\fn@parbox@i{#1[#2]}}{\fn@parbox@ii{#1[#2]}}%
}
\long\def\fn@parbox@ii#1#2#3{\savenotes\fn@parbox#1{#2}{#3}\spewnotes}
%    \end{macrocode}
%
% \end{macro}
%
% Done!
%
%    \begin{macrocode}
%</package>
%    \end{macrocode}
%
% \hfill Mark Wooding, \today
%
% \Finale
%
\endinput
