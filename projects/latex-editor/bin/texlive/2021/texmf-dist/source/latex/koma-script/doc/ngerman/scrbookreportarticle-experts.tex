% ======================================================================
% scrbookreportarticle-experts.tex
% Copyright (c) Markus Kohm, 2001-2021
%
% This file is part of the LaTeX2e KOMA-Script bundle.
%
% This work may be distributed and/or modified under the conditions of
% the LaTeX Project Public License, version 1.3c of the license.
% The latest version of this license is in
%   http://www.latex-project.org/lppl.txt
% and version 1.3c or later is part of all distributions of LaTeX 
% version 2005/12/01 or later and of this work.
%
% This work has the LPPL maintenance status "author-maintained".
%
% The Current Maintainer and author of this work is Markus Kohm.
%
% This work consists of all files listed in manifest.txt.
% ----------------------------------------------------------------------
% scrbookreportarticle-experts.tex
% Copyright (c) Markus Kohm, 2001-2021
%
% Dieses Werk darf nach den Bedingungen der LaTeX Project Public Lizenz,
% Version 1.3c, verteilt und/oder veraendert werden.
% Die neuste Version dieser Lizenz ist
%   http://www.latex-project.org/lppl.txt
% und Version 1.3c ist Teil aller Verteilungen von LaTeX
% Version 2005/12/01 oder spaeter und dieses Werks.
%
% Dieses Werk hat den LPPL-Verwaltungs-Status "author-maintained"
% (allein durch den Autor verwaltet).
%
% Der Aktuelle Verwalter und Autor dieses Werkes ist Markus Kohm.
% 
% Dieses Werk besteht aus den in manifest.txt aufgefuehrten Dateien.
% ======================================================================
%
% Chapter about scrbook, scrreprt, and scrartcl of the KOMA-Script guide
% expert part
% Maintained by Markus Kohm
%
% ----------------------------------------------------------------------
%
% Kapitel über scrbook, scrreprt und scrartcl im Experten-Teil der
% KOMA-Script-Anleitung
% Verwaltet von Markus Kohm
%
% ============================================================================

\KOMAProvidesFile{scrbookreportarticle-experts.tex}%
                 [$Date: 2021-06-25 10:07:17 +0200 (Fri, 25 Jun 2021) $
                  KOMA-Script guide (chapter: scrbook, scrreprt, scrartcl for
                  experts)]

\chapter[{Zusätzliche Informationen zu den Hauptklassen und 
  \Package{scrextend}}]{Zusätzliche
  Informationen zu den Hauptklassen 
  \Class{scrbook}, \Class{scrreprt} und
  \Class{scrartcl} sowie dem 
  Paket \Package{scrextend}}
\labelbase{maincls-experts}

\BeginIndexGroup%
\BeginIndex{Class}{scrartcl}%
\BeginIndex{Class}{scrbook}%
\BeginIndex{Class}{scrreprt}%
\BeginIndex{Package}{scrextend}%
In diesem Kapitel finden Sie zusätzliche Informationen zu den
\KOMAScript-Klassen \Class{scrbook}, \Class{scrreprt} und \Class{scrartcl} und
einigen Anweisungen, die auch in \Package{scrextend} vorhanden sind.
\iffree{Einige Teile des Kapitels sind dabei dem \KOMAScript-Buch
  \cite{book:komascript} vorbehalten. Dies sollte kein Problem sein, denn
  der}{Der} normale Anwender, der die Klassen einfach nur verwenden will, wird
diese Informationen eher selten benötigen. Ein Teil der Informationen richtet
sich an Anwender, die ausgefallene Aufgaben lösen oder eigene Klassen
schreiben wollen, die auf einer \KOMAScript-Klasse basieren. Da sich die
entsprechenden Erklärungen ausdrücklich nicht an \LaTeX-Anfänger richten, sind
sie teilweise deutlich kürzer gefasst und setzen ein vertieftes Wissen über
\LaTeX{} voraus. Anderes existiert nur aus Gründen der Kompatibilität zu den
Standardklassen\iffree{ oder früheren Versionen von \KOMAScript}{}.\iffree{}{
  Teile, die nur aus Gründen der Kompatibilität zu früheren Versionen von
  \KOMAScript{} existieren, sind in serifenloser Schrift gesetzt und sollten
  nicht mehr verwendet werden.}

\LoadNonFree{scrbookreportarticle-experts}{0}

\section{Ergänzungen zu Benutzeranweisungen}
\seclabel{addInfos}

\LoadNonFree{scrbookreportarticle-experts}{1}


\section{Zusammenspiel von \KOMAScript{} und anderen Paketen}
\seclabel{coexistence}

\LoadNonFree{scrbookreportarticle-experts}{2}


\section{Erkennung von \KOMAScript-Klassen}
\seclabel{identify}

Für Paketautoren besteht manchmal die Notwendigkeit, eine \KOMAScript-Klasse
zu erkennen. Für Anwender besteht diese Notwendigkeit dagegen eher
nicht. Bezüglich der verwendeten \KOMAScript-Version sei auf
\DescRef{scrbase.cmd.KOMAScriptVersion} in \autoref{sec:scrbase.identify},
\DescPageRef{scrbase.cmd.KOMAScriptVersion} verwiesen.

\begin{Declaration}
  \Macro{KOMAClassName}
  \Macro{ClassName}
\end{Declaration}
In \Macro{KOMAClassName} ist der Name der aktuell verwendeten
\KOMAScript-Klasse abgelegt. Will man also wissen, ob eine \KOMAScript-Klasse
verwendet wird, so kann man einfach mit \Macro{@ifundefined} auf diese
Anweisung testen. Tests auf eine konkrete \KOMAScript-Klasse, beispielsweise
mit \DescRef{scrbase.cmd.Ifstr}, sind ebenfalls möglich. Demgegenüber gibt
\Macro{ClassName} Auskunft, welche Standardklasse durch diese
\KOMAScript-Klasse ersetzt wird.%
\iffalse% Umbruchkorrektur
\par
Es\textnote{Achtung!} sei in diesem Zusammenhang darauf hingewiesen, dass
dagegen die Existenz von
\DescRef{scrbase.cmd.KOMAScript}\IndexCmd{KOMAScript} nicht als Indiz für die
Verwendung einer \KOMAScript-Klasse dienen kann. Zum einen definieren alle
\KOMAScript-Pakete diese Anweisung, zum anderen können auch andere Pakete es
für sinnvoll erachten, das \KOMAScript-Piktogramm unter diesem Namen zu
definieren.%
\fi%
\EndIndexGroup


\section{Einträge ins Inhaltsverzeichnis}
\seclabel{toc}

\KOMAScript-Klassen bieten erweiterte Möglichkeiten zur Erstellung und
Manipulation von Einträgen in das Inhaltsverzeichnis. Einige davon basieren
auf der Verwendung von \Package{tocbasic} (siehe
\autoref{sec:tocbasic.tocstyle} ab
\autopageref{sec:tocbasic.tocstyle}). Andere sind direkt in den Klassen
implementiert.


\begin{Declaration}
  \Macro{raggedchapterentry}
\end{Declaration}
Bei früheren\ChangedAt{v3.21}{\Class{scrbook}\and \Class{scrreprt}} Version
von \KOMAScript{} gab es die Möglichkeit, das Makro \Macro{raggedchapterentry}
als \Macro{raggedright} zu definieren, um den Text der Kapiteleinträge ins
Inhaltsverzeichnis im linksbündigen Flattersatz zu setzen. Offiziell existiert
diese Möglichkeit seit \KOMAScript-Version~3.21 nicht mehr.

Tatsächlich ist aber die Eigenschaft \PValue{raggedentrytext} für den
Eintrags-Stil \PValue{tocline} im Paket
\hyperref[cha:tocbasic]{\Package{tocbasic}}\IndexPackage{tocbasic} so
implementiert, dass sie das Makro \Macro{ragged\PName{Eintragsebene}entry}
entweder auf \Macro{relax} oder auf \Macro{raggedright} setzt. Bei der
Auswertung der Eigenschaft wird dann getestet, ob das entsprechende Makro
entweder \Macro{raggedright} ist oder als \Macro{raggedright} definiert
ist. In beiden Fällen wird Flattersatz verwendet. In allen anderen Fällen wird
kein Flattersatz verwendet.

Da schon früher dokumentiert war, dass \Macro{raggedchapterentry} nicht
als etwas anderes als \Macro{raggedright} definiert werden sollte, ist damit
Kompatibilität zum dokumentierten Verhalten früherer Versionen erreicht. Wie
in früheren Versionen gewarnt, führen andere Definitionen von
\Macro{raggedchapterentry} -- nun aber auch von \Macro{raggedsectionentry} und
entsprechend für die anderen Eintragsebenen -- möglicherweise zu unerwarteten
Ergebnissen.

Empfohlen wird, eventuell gewünschten Flattersatz für Verzeichniseinträge
stattdessen über die genannte Eigenschaft des Verzeichniseintragsstils
\PValue{tocline} zu wählen.%
\EndIndexGroup


\begin{Declaration}
  \Macro{addtocentrydefault}\Parameter{Ebene}\Parameter{Nummer}%
                            \Parameter{Überschrift}
\end{Declaration}
Die\ChangedAt{v3.08}{\Class{scrbook}\and \Class{scrreprt}\and
  \Class{scrartcl}} \KOMAScript-Klassen verwenden
\Macro{addcontentsline}\IndexCmd{addcontentsline}%
\important{\Macro{addcontentsline}} nicht direkt, um Einträge ins
Inhaltsverzeichnis vorzunehmen. Stattdessen wird \Macro{addtocentrydefault}
mit ganz ähnlichen Argumenten aufgerufen. Die Anweisung kann sowohl für
nummerierte als auch für nicht nummerierte Einträge verwendet werden. Dabei
gibt \PName{Ebene} die Gliederungsebene in Textform an, also \PValue{part},
\PValue{chapter}, \PValue{section}, \PValue{subsection},
\PValue{subsubsection}, \PValue{paragraph} oder \PValue{subparagraph}. Die
formatierte Gliederungsnummer wird über das zweite Argument \PName{Nummer}
übergeben. Dieses Argument darf auch leer sein. Der Text des Eintrags wird mit
\PName{Überschrift} angegeben. Zerbrechliche Befehle in
diesem Argument sind mit
\Macro{protect}\IndexCmd{protect}\important{\Macro{protect}} zu schützen.

Für das Argument \PName{Nummer} gilt noch eine Besonderheit. Ist das Argument
leer, so signalisiert dies, dass ein nicht nummerierter Eintrag erzeugt werden
soll. In der Voreinstellung wird dies mit
\begin{quote}\small\raggedright
  \Macro{addcontentsline}\PParameter{toc}\Parameter{Ebene}%
  \PParameter{\%\\
    \quad\Macro{protect}\DescRef{tocbasic.cmd.nonumberline} 
    \PName{Überschrift}\%\\
  }
\end{quote}
erreicht. Ist das Argument jedoch nicht leer, so soll ein nummerierter Eintrag
erzeugt werden und \PName{Nummer} ist die vorformatierte Gliederungsnummer. In
der Voreinstellung verwendet \KOMAScript{} dann:
\begin{quote}\small\raggedright
  \Macro{addcontentsline}\PParameter{toc}\Parameter{Ebene}%
  \PParameter{\%\\
    \quad\Macro{protect}\DescRef{tocbasic.cmd.numberline}\Parameter{Nummer}%
    \PName{Überschrift}\%\\
  }
\end{quote}
\iffalse zur Erzeugung dieses Eintrags.\fi

Paketautoren und Autoren von Wrapper-Klassen können diese Anweisung
umdefinieren, um Einfluss auf die Einträge zu nehmen. So\textnote{Beispiel}
wäre beispielsweise denkbar, mit
\begin{lstcode}
  \renewcommand{\addtocentrydefault}[3]{%
    \IfArgIsEmpty{#3}{%
    }{%
      \IfArgIsEmpty{#2}{%
        \addcontentsline{toc}{#1}{\protect\nonumberline#3}%
      }{%
        \addcontentsline{toc}{#1}{\protect\numberline{#2}#3}%
      }%
    }%
  }%
\end{lstcode}
dafür\IndexCmd{IfArgIsEmpty} zu sorgen, dass Einträge mit leerer
\PName{Überschrift} erst gar nicht vorgenommen werden. Eine solche Änderung
ist in der Praxis jedoch nicht notwendig, da die Unterdrückung leerer Einträge
bereits auf andere Weise in die \KOMAScript-Klassen eingebaut ist. Siehe
hierzu auch die Erklärung zu den Gliederungsbefehlen in
\autoref{sec:maincls.structure} ab \DescPageRef{maincls.cmd.part}.%
%
\EndIndexGroup


\begin{Declaration}
  \Macro{addparttocentry}\Parameter{Nummer}\Parameter{Überschrift}
  \Macro{addchaptertocentry}\Parameter{Nummer}\Parameter{Überschrift}
  \Macro{addsectiontocentry}\Parameter{Nummer}\Parameter{Überschrift}
  \Macro{addsubsectiontocentry}\Parameter{Nummer}\Parameter{Überschrift}
  \Macro{addsubsubsectiontocentry}\Parameter{Nummer}\Parameter{Überschrift}
  \Macro{addparagraphtocentry}\Parameter{Nummer}\Parameter{Überschrift}
  \Macro{addsubparagraphtocentry}\Parameter{Nummer}\Parameter{Überschrift}
\end{Declaration}%
Auch \ChangedAt{v3.08}{\Class{scrbook}\and \Class{scrreprt}\and
  \Class{scrartcl}} die oben dokumentierte Anweisung
\DescRef{\LabelBase.cmd.addtocentrydefault}\IndexCmd{addtocentrydefault}%
\important{\DescRef{\LabelBase.cmd.addtocentrydefault}} wird von den
\KOMAScript-Klassen nur dann direkt aufgerufen, wenn für die angegebene
\PName{Ebene} keine direkte Anweisung definiert oder diese
\Macro{relax}\IndexCmd{relax}\important{\Macro{relax}} ist. In der
Voreinstellung sind die angegebenen Anweisungen alle so definiert, dass sie
ihre \PName{Ebene} und die Argumente direkt an
\DescRef{\LabelBase.cmd.addtocentrydefault} weitergeben.%
%
\EndIndexGroup


\section{Schrifteinstellungen}
\seclabel{fonts}

\KOMAScript-Klassen verfügen nicht nur über erweiterte Möglichkeiten zur
Auswahl der Grundschriftgröße. Sie erlauben auch die Definition von Elementen
mit eigenen Schrifteinstellungen, sowie deren Manipulation und dedizierter
Anwendung.

\LoadNonFree{scrbookreportarticle-experts}{5}

\begin{Declaration}
  \Macro{newkomafont}\OParameter{Warnung}\Parameter{Element}
                     \Parameter{Voreinstellung}
  \Macro{aliaskomafont}\Parameter{Aliasname}\Parameter{Element}
\end{Declaration}
Experten können mit \Macro{newkomafont} eine \PName{Voreinstellung} für die
Schrift eines \PName{Element}s definieren. Anschließend kann diese
Voreinstellung mit den Anweisungen \DescRef{maincls.cmd.setkomafont} und
\DescRef{maincls.cmd.addtokomafont} (siehe \autoref{sec:maincls.textmarkup},
\DescPageRef{maincls.cmd.setkomafont}) verändert werden. Natürlich wird diese
Schrift damit noch lange nicht verwendet. Der Experte muss selbst Sorge dafür
tragen, dass er an den entsprechenden Stellen die Anweisung
\DescRef{maincls.cmd.usekomafont}%
\important{\DescRef{maincls.cmd.usekomafont}}\IndexCmd{usekomafont} (siehe
\DescPageRef{maincls.cmd.usekomafont}) für dieses Element in seine
Definitionen einbaut. Der Aufruf von \Macro{newkomafont} für ein bereits
existierendes Element führt zu Fehlermeldungen.

Das optionale Argument \PName{Warnung} definiert eine Warnmeldung. Diese wird
bei den \KOMAScript-Klassen per \Macro{ClassWarning} oder beim Paket
\Package{scrextend} per \Macro{PackageWarning} immer dann ausgegeben, wenn die
Voreinstellung für das Element verändert wird. Als Urheber der
Warnung wird das Paket
\Package{scrkbase}\IndexPackage{scrkbase}\important{\Package{scrkbase}}
angegeben.

Mit \Macro{aliaskomafont} kann für ein bereits existierendes \PName{Element}
ein \PName{Aliasname} definiert werden. \KOMAScript{} informiert den Benutzer
in der \File{log}-Datei über den Namen des tatsächlichen Elements, wenn dieser
den \PName{Aliasname} verwendet. \PName{Aliasname}n\textnote{Tipp!} können
beispielsweise dann eingesetzt werden, wenn der Entwickler sich später einen
besseren Namen für ein Element überlegt und der alte Name aus
Kompatibilitätsgründen weiter verwendbar bleiben soll. Außerdem kann damit die
Benutzerfreundlichkeit erhöht werden, indem einem Element all die Namen als
Alias zugeordnet werden, die unterschiedliche Benutzer intuitiv wählen
würden. \KOMAScript{} selbst macht von dieser Möglichkeit regen Gebrauch.
%
\EndIndexGroup


\begin{Declaration}
  \Macro{addtokomafontrelaxlist}\Parameter{Makro}
  \Macro{addtokomafontonearglist}\Parameter{Makro}
  \Macro{addtokomafontgobblelist}\Parameter{Makro}
\end{Declaration}
Wie bereits in \autoref{part:forAuthors} der Anleitung erklärt, dürfen in den
Schrifteinstellungen der Elemente nur Befehle zur Wahl der Größe, Familie,
Codierung, Strichstärke, Form und Farbe enthalten sein. Dabei erfolgt schon
die Änderung der Farbe bei \LaTeX{} nicht transparent und kann damit
unerwünschte Effekte hervorrufen, wenn man \DescRef{maincls.cmd.usekomafont}
an ungünstiger Stelle verwendet.

Nun neigen Anwender dazu, in die Schrifteinstellungen auch ganz andere,
teilweise sehr kritische Dinge zu packen, beispielsweise ein
\Macro{MakeUppercase} ganz am Ende der Einstellung. Bei der internen
Verwendung der Schrifteinstellungen wurde daher möglichst so vorgegangen, dass
viele dieser eigentlich verbotenen Einstellungen trotzdem keinen Schaden
anrichten und es meist sogar funktioniert, wenn der letzte Befehl in der
Schrifteinstellung ein Argument erwartet, also beispielsweise \Macro{textbf}
anstelle von \Macro{bfseries} verwendet wird. Eine Garantie gibt es dafür
jedoch nicht. 

In Einzelfällen war es innerhalb von \KOMAScript{} notwendig, die Umschaltung
wirklich auf Schrifteinstellungen zu beschränken. Dies erfolgt dann
beispielsweise mit
\DescRef{maincls.cmd.usefontofkomafont}\IndexCmd{usefontofkomafont}%
\IndexCmd{usesizeofkomafont}\IndexCmd{useencodingofkomafont}%
\IndexCmd{usefamilyofkomafont}\IndexCmd{useseriesofkomafont}%
\IndexCmd{useshapeofkomafont} statt \DescRef{maincls.cmd.usekomafont} (siehe
\autoref{sec:maincls.textmarkup},
\DescPageRef{maincls.cmd.usefontofkomafont}).

Die Anweisung \DescRef{maincls.cmd.usefontofkomafont} und ihre Geschwister
haben allerdings ihre Grenzen. Deshalb darf die vermeintliche Schrifteinstellung
eines Elements keinesfalls ein voll expandierbares Argument erwarten. Genau
das ist aber beispielsweise bei \Macro{MakeUppercase} der
Fall. Daher\ChangedAt{v3.17}{\Class{scrbook}\and \Class{scrreprt}\and
  \Class{scrartcl}} verwaltet \KOMAScript{} eine interne Liste von Makros, die
innerhalb von \DescRef{maincls.cmd.usefontofkomafont} und ihren Geschwistern
zu \Macro{relax} werden sollen. In der Voreinstellung ist das seit
\KOMAScript~3.24\ChangedAt{v3.24}{\Class{scrbook}\and \Class{scrreprt}\and
  \Class{scrartcl}} nur noch \Macro{normalcolor}.

Es ist zu beachten, dass das angegebene \PName{Makro} wirklich stur auf
\Macro{relax} gesetzt wird. Irgendwelche Argumente innerhalb der
Schrifteinstellung werden also gegebenenfalls lokal ausgeführt. Daher dürfen
Anweisungen wie \Macro{setlength} keinesfalls zu dieser Liste hinzugefügt
werden. Für alle Fehler, die durch die Verwendung von
\Macro{addtokomafontrelaxlist} entstehen, ist der Anwender selbst
verantwortlich. Außerdem sollte diese Möglichkeit nicht als Legitimation dafür
missverstanden werden, den Schrifteinstellungen alle möglichen Anweisungen
hinzuzufügen!

Für\ChangedAt{v3.24}{\Class{scrbook}\and \Class{scrreprt}\and
  \Class{scrartcl}} Befehle, deren erstes Argument noch ohne zusätzliche
Gruppe ausgeführt werden soll, gibt es
\Macro{addtokomafontonearglist}. Das angegebene \PName{Makro} wird dabei auf
\Macro{@firstofone} gesetzt. In der Voreinstellung wird dies für
\Macro{MakeUppercase} und \Macro{MakeLowercase} verwendet.

Soll\ChangedAt{v3.19}{\Class{scrbook}\and \Class{scrreprt}\and
  \Class{scrartcl}} hingegen ein \PName{Makro} innerhalb von
\DescRef{maincls.cmd.usefontofkomafont} und ihren Geschwistern zusammen mit
seinem ersten Argument ignoriert werden, so ist stattdessen
\Macro{addtokomafontgobblelist} zu verwenden. Ein Beispiel dafür ist die
Anweisung \Macro{color}, die einschließlich des Namens der Farbe ignoriert
werden muss und deshalb bereits in der Voreinstellung Teil dieser Liste ist.

Es ist zu beachten, dass sich die hier genannten Voreinstellungen in
zukünftigen Versionen ändern können. Wenn Sie bestimmte Befehle in einer der
Listen zwingend benötigen, sollten Sie diese also selbst explizit hinzufügen.%
\EndIndexGroup


\begin{Declaration}
  \Macro{IfExistskomafont}\Parameter{Element}\Parameter{Dann-Code}%
                          \Parameter{Sonst-Code}
  \Macro{IfIsAliaskomafont}\Parameter{Element}\Parameter{Dann-Code}%
                           \Parameter{Sonst-Code}
\end{Declaration}
Da\ChangedAt{v3.15}{\Class{scrbook}\and \Class{scrreprt}\and
  \Class{scrartcl}\and \Class{scrlttr2}} die Schrift mancher Elemente erst ab
bestimmten Versionen von \KOMAScript{} geändert werden kann, ist es manchmal
sinnvoll, vorher testen zu können, ob ein \PName{Element} mit dieser
Möglichkeit überhaupt existiert. Die Anweisung \Macro{IfExistskomafont} führt
den \PName{Dann-Code} genau dann aus, wenn das \PName{Element} über
\DescRef{\LabelBase.cmd.newkomafont} oder
\DescRef{\LabelBase.cmd.aliaskomafont} definiert wurde und daher auch mit
\DescRef{maincls.cmd.setkomafont} oder \DescRef{maincls.cmd.addtokomafont}
geändert und mit den \Macro{use\dots komafont}-Anweisungen abgefragt werden
kann. Anderenfalls wird der \PName{Sonst-Code} ausgeführt.

Im Unterschied dazu führt
\Macro{IfIsAliaskomafont}\ChangedAt{v3.25}{\Class{scrbook}\and
  \Class{scrreprt}\and \Class{scrartcl}\and \Class{scrlttr2}} den
\PName{Dann-Code} nur aus, wenn \PName{Element} über
\DescRef{\LabelBase.cmd.aliaskomafont} als Alias für ein anderes Element
definiert wurde. Sowohl für nicht definierte Elemente als auch für mit
\DescRef{\LabelBase.cmd.newkomafont} definierte Elemente wird hingegen der
\PName{Sonst-Code} ausgeführt.%
\EndIndexGroup


\section{Absatzmarkierung}
\seclabel{parskip}

Nicht nur, aber insbesondere aufgrund der erweiterten Möglichkeiten der
Absatzmarkierung bei den \KOMAScript-Klassen sollte auf die direkte Änderung
der Standardlängen \Length{parskip}\IndexLength{parskip},
\Length{parindent}\IndexLength{parskip} und
\Length{parfillskip}\IndexLength{parskip} weitgehend verzichtet werden.

\begin{Declaration}
  \Macro{setparsizes}\Parameter{Einzug}\Parameter{Abstand}
                     \Parameter{Endzeilenleerraum}
\end{Declaration}
\KOMAScript{} bietet mit dieser Anweisung die Möglichkeit, sowohl den
Absatzeinzug als auch den Absatzabstand und den Freiraum am Ende der letzten
Zeile des Absatzes einzustellen. Diese Anweisung ist immer dann zu verwenden,
wenn die Änderungen auch bei Einstellung
\OptionValueRef{\LabelBase}{parskip}{relative} beachtet werden
sollen. \KOMAScript{}\textnote{Beispiel} selbst verwendet sie beispielsweise
in der Form
\begin{lstcode}
  \setparsizes{0pt}{0pt}{0pt plus 1fil}
\end{lstcode}
um sowohl den Einzug als auch den Abstand abzuschalten und am Ende des
Absatzes beliebigen Freiraum zu erlauben. Eine solche Maßnahme ist sinnvoll,
wenn ein Absatz nur aus einer Box besteht, die ohne Abstand nach oben oder
unten gesetzt werden soll und die gesamte Spaltenbreite einnimmt. Soll
demgegenüber die Box nur die gesamte Breite einnehmen, jedoch mit der
aktuellen Einstellung bezüglich des Absatzabstandes gesetzt werden, so ist
\begin{lstcode}
  \setlength{\parfillskip}{0pt plus 1fil}
\end{lstcode}
vorzuziehen.

Eine\ChangedAt{v3.17}{\Class{scrbook}\and \Class{scrreprt}\and
  \Class{scrartcl}} Neuberechnung\IndexCmd{activateareas} oder Reaktivierung
der Einstellungen für den Satzspiegel und die Ränder (siehe
\autoref{cha:typearea}) führt seit \KOMAScript~3.17 übrigens immer auch zu
einer Neueinstellung der via \Macro{setparsizes} gesetzten Werte, falls die
Werte nicht zwischenzeitlich geändert wurden. Dies sollte ein Grund mehr sein,
nicht an \KOMAScript{} vorbei die Einstellungen zu ändern. Die Neuberechnung
wird bei einer Kompatibilitätseinstellung zu einer früheren Version (siehe
\autoref{sec:maincls.compatibilityOptions},
\DescPageRef{maincls.option.version}, Option
\DescRef{maincls.option.version}%
\IndexOption{version}\important{\OptionValueRef{maincls}{version}{3.17}})
deaktiviert.%
%
\EndIndexGroup

\section{Zähler}
\seclabel{counter}

Zähler sind bei \LaTeX{} eine durchaus komplexere Angelegenheit, als dies auf
den ersten Blick erscheinen mag. Das gilt insbesondere, wenn die
Referenzierung von Zählern über \Macro{refstepcounter}, \Macro{label} und
\Macro{ref} mit betrachtet wird.

\LoadNonFree{scrbookreportarticle-experts}{3}

\section{Gliederung}
\seclabel{sections}

Die \KOMAScript-Klassen bieten weitreichende Möglichkeiten, um Einfluss auf
die Gliederungsebenen und die zugehörigen Überschriften zu nehmen. Selbst die
Definition neuer Ebenen ist damit möglich.

\begin{Declaration}
  \Macro{DeclareSectionCommand}\OParameter{Einstellungen}\Parameter{Name}
  \Macro{DeclareNewSectionCommand}\OParameter{Einstellungen}\Parameter{Name}
  \Macro{RedeclareSectionCommand}\OParameter{Einstellungen}\Parameter{Name}
  \Macro{ProvideSectionCommand}\OParameter{Einstellungen}\Parameter{Name}
\end{Declaration}
Diese\ChangedAt{v3.15}{\Class{scrbook}\and \Class{scrreprt}\and
  \Class{scrartcl}} Anweisungen dienen dazu, einen neuen Gliederungsbefehl
\Macro{\PName{Name}} zu definieren beziehungsweise einen vorhandenen
Gliederungsbefehl \Macro{\PName{Name}} zu verändern. Dazu werden über das
optionale Argument \PName{Einstellungen} vorgenommen. Die
\PName{Einstellungen} sind dabei eine durch Komma separierte Liste von
\PName{Schlüssel}=\PName{Wert}-Zuweisungen. Neben den vom Stil der Überschrift
unabhängigen Eigenschaften, die
\autoref{tab:maincls-experts.declaresection.keys},
\autopageref{tab:maincls-experts.declaresection.keys} zu entnehmen sind, gibt
es auch Eigenschaften, die vom jeweiligen Stil abhängig sind. Derzeit stehen
die folgenden Stile zur Verfügung:
\begin{description}\setkomafont{descriptionlabel}{}
\item[\PValue{part}] \ChangedAt{v3.18}{\Class{scrbook}\and
    \Class{scrreprt}\and \Class{scrartcl}}ist der
  Stil\textnote{Gliederungsstil
    \PValue{part}}\Index{Gliederung>Stil>~\PValue{part}} für
  Teileüberschriften. Dieser Stil wird in der Voreinstellung für
  \DescRef{maincls.cmd.part}\IndexCmd{part} und indirekt für
  \DescRef{maincls.cmd.addpart}\IndexCmd{addpart} verwendet. Neue
  Überschriften in diesem Stil können definiert werden, haben dann aber nicht
  automatisch auch eine \Macro{add\dots}-Variante. Für die Konfiguration der
  vorhandenen oder neuer Überschriften stehen zusätzlich die Eigenschaften aus
  \autoref{tab:maincls-experts.declarepartstyle.keys},
  \autopageref{tab:maincls-experts.declarepartstyle.keys} zur
  Verfügung. Die\textnote{Achtung!} Anweisung \DescRef{maincls.cmd.addpart}
  wird ebenso wie die Sternformen automatisch zusammen mit
  \DescRef{maincls.cmd.part} umkonfiguriert und kann nicht unabhängig davon
  verändert werden.
\item[\PValue{chapter}] \ChangedAt{v3.18}{\Class{scrbook}\and
    \Class{scrreprt}}ist der Stil\textnote{Gliederungsstil
    \PValue{chapter}}\Index{Gliederung>Stil>~\PValue{chapter}} für
  Kapitelüberschriften. Dieser Stil wird in der Voreinstellung für
  \DescRef{maincls.cmd.chapter}\IndexCmd{chapter} und indirekt für
  \DescRef{maincls.cmd.addchap}\IndexCmd{addchap} verwendet. Neue
  Überschriften in diesem Stil können definiert werden, haben dann aber nicht
  automatisch auch eine \Macro{add\dots}-Variante. Für die Konfiguration der
  vorhandenen oder neuer Überschriften stehen zusätzlich die Eigenschaften aus
  \autoref{tab:maincls-experts.declarechapterstyle.keys},
  \autopageref{tab:maincls-experts.declarechapterstyle.keys} zur
  Verfügung. Die\textnote{Achtung!} Anweisung \DescRef{maincls.cmd.addchap}
  wird ebenso wie die Sternformen automatisch zusammen mit
  \DescRef{maincls.cmd.chapter} umkonfiguriert und kann nicht unabhängig davon
  verändert werden. Es ist zu beachten, dass dieser Stil von
  \Class{scrartcl}\OnlyAt{\Class{scrbook}\and \Class{scrreprt}} nicht
  bereitgestellt wird.
\item[\PValue{section}] ist der Stil\textnote{Gliederungsstil
    \PValue{section}}\Index{Gliederung>Stil>~\PValue{section}} für
  Abschnittsüberschriften und tieferer Ebenen. Dieser Stil wird derzeit sowohl
  für \DescRef{maincls.cmd.section}\IndexCmd{section},
  \DescRef{maincls.cmd.subsection}\IndexCmd{subsection},
  \DescRef{maincls.cmd.subsubsection}\IndexCmd{subsubsection} als auch
  \DescRef{maincls.cmd.paragraph}\IndexCmd{paragraph} und
  \DescRef{maincls.cmd.subparagraph}\IndexCmd{subparagraph} verwendet. Neue
  Überschriften in diesem Stil können definiert werden. Für die Konfiguration
  der vorhandenen oder neuer Überschriften stehen zusätzlich die Eigenschaften
  aus \autoref{tab:maincls-experts.declaresectionstyle.keys},
  \autopageref{tab:maincls-experts.declaresectionstyle.keys} zur
  Verfügung. Bei\ChangedAt{v3.24}{\Class{scrbook}\and \Class{scrreprt}\and
    \Class{scrartcl}} der Neudefinition sind die \PName{Schlüssel}
  \PValue{style}, \PValue{afterskip}, \PValue{beforeskip} und \PValue{level}
  zwingend.  Die \PName{Schlüssel}
  \PValue{afterindent}\ChangedAt{v3.26}{\Class{scrbook}\and
    \Class{scrreprt}\and \Class{scrartcl}}, \PValue{font}, \PValue{indent} und
  \PValue{runin} sind empfohlen.  \PValue{tocindent} und \PValue{tocnumwidth}
  können abhängig vom Namen der Gliederungsebene ebenfalls zwingend sein. Das
  gilt auch, falls ein Befehl, der bisher kein Gliederungsbefehl war, mit
  \Macro{RedeclareSectionCommand} zu einem Gliederungsbefehl umdefiniert
  wird. Die\textnote{Achtung!}  Anweisung \Macro{addsec} wird ebenso wie die
  Sternformen zusammen mit \DescRef{maincls.cmd.section} umkonfiguriert und
  kann nicht unabhängig davon verändert werden.
\end{description}

\BeginIndexGroup%
\BeginIndex{FontElement}{\PName{Gliederungsebene}}%
\BeginIndex{FontElement}{\PName{Gliederungsebene}prefix}%
Bei der Definition eines Gliederungsbefehls wird ein gleichnamiges
Element\textnote{Elemente}%
\Index{Element=\UseIndex {gen}\protect\GuideFontElement} % Getrickst!
angelegt, falls es noch nicht existiert. Bei \PValue{chapter} und
\PValue{part} werden ebenso Elemente für die Präfixzeile erzeugt. Die
Schrifteinstellung der Elemente kann mit \DescRef{maincls.cmd.setkomafont} und
\DescRef{maincls.cmd.addtokomafont} (siehe \autoref{sec:maincls.textmarkup},
\DescPageRef{maincls.cmd.setkomafont}) geändert werden.%
\EndIndexGroup

\begin{table}
  \caption[{Stil unabhängige Eigenschaften bei der Konfiguration von
    Gliederungsbefehlen}]%
  {Mögliche vom Stil der Überschrift unabhängige
    \PName{Schlüssel} und \PName{Werte} für die \PName{Eigenschaften} bei der
    Konfiguration von Gliederungsbefehlen}%
    \label{tab:maincls-experts.declaresection.keys}%
  \begin{tabularx}{\linewidth}{llX}
    \toprule
    \PName{Schlüssel}
    & \PName{Wert}
    & Bedeutung \\
    \midrule 
    \PValue{counterwithin}
    & \PName{Zählername}
    & Der zur Gliederungsebene gehörende Zähler soll vom als Wert angegebenen
      Zähler abhängig sein. Wird \PName{Zählername} über \Macro{stepcounter}
      oder \Macro{refstepcounter} erhöht, so wird der zur Gliederungsebene
      gehörende Zähler auf 0 zurückgesetzt. Darüber hinaus wird
      \Macro{the\PName{Zählername}}, gefolgt von einem Punkt in der Ausgabe
      des zur Gliederungsebene gehörenden Zählers vorangestellt.\\
    \PValue{counterwithout}%
    \ChangedAt{v3.19}{\Class{scrbook}\and \Class{scrreprt}\and
      \Class{scrartcl}}%
    & \PName{Zählername}
    & Hebt eine früher vorgenommene \PValue{counterwithin}-Einstellung auf und
      ist daher nur bei Änderung vorhandener Gliederungsbefehle sinnvoll.\\
    \PValue{expandtopt}
    & \PName{Schalter}
    & Ist dieser Schalter aktiv, so werden alle in den \PName{Einstellungen}
      nachfolgend angegebenen Werte für Längen vollständig expandiert,
      ausgewertet und in \texttt{pt} umgerechnet gespeichert.  Ist der
      Schalter nicht aktiv, so werden alle nachfolgend angegebenen
      Werte für Längen nur testweise expandiert und ausgewertet, aber
      lediglich expandiert gespeichert. Es werden die Werte für einfache
      Schalter aus \autoref{tab:truefalseswitch},
      \autopageref{tab:truefalseswitch} verstanden.\\
    \PValue{level}
    & \PName{Ganzzahl}
    & Numerischer Wert der Gliederungsebene (siehe Zähler
      \DescRef{maincls.counter.secnumdepth}, \autoref{sec:maincls.structure},
      \DescPageRef{maincls.counter.secnumdepth}); der Wert sollte eindeutig
      sein und ist für neue Ebenen zwingend.\\
    \PValue{style}
    & \PName{Name}
    & Legt den Stil der Überschrift fest und ist für neue Ebenen zwingend.\\
    \PValue{tocstyle}%
    \ChangedAt{v3.20}{\Class{scrbook}\and \Class{scrreprt}\and
      \Class{scrartcl}}%
    & \PName{Name}
    & Legt den Stil des zur Überschrift gehörenden Verzeichniseintrags fest.
      Es können alle bisher definierten Verzeichniseintragsstile (siehe
      \autoref{sec:tocbasic.tocstyle}) verwendet werden. Ein leerer
      \PName{Name} verhindert die Umdefinierung des Befehls
      \Macro{l@\dots} für die Verzeichniseinträge.\\
    \PValue{toc\PName{Option}}%
    \ChangedAt{v3.20}{\Class{scrbook}\and \Class{scrreprt}\and
      \Class{scrartcl}}%
    & \PName{Wert}
    & Weitere Optionen in Abhängigkeit vom via \Option{tocstyle} gewählten
      Verzeichniseintragsstil. Siehe dazu \autoref{sec:tocbasic.tocstyle} ab 
      \autopageref{sec:tocbasic.tocstyle}. Für die von
      \hyperref[cha:tocbasic]{\Package{tocbasic}} vordefinierten
      Verzeichniseintragsstile finden sich die als \PName{Option} verwendbaren
      Attribute in \autoref{tab:tocbasic.tocstyle.attributes}, ab 
      \autopageref{tab:tocbasic.tocstyle.attributes}.\\
  % \PValue{tocindent}
  % & \PName{Länge}
  % & Der horizontale Einzug des zum Gliederungsbefehls gehörenden
  %   Eintrags ins Inhaltsverzeichnis, falls ein solcher abhängig von
  %   \DescRef{maincls.counter.tocdepth} (siehe \autoref{sec:maincls.toc},
  %   \DescPageRef{maincls.counter.tocdepth}) erzeugt wird.\\ 
  % \PValue{toclevel}
  % & \PName{Ganzzahl}
  % & Die Ebene des zum Gliederungsbefehl gehörenden Eintrags ins
  %   Inhaltsverzeichnis, falls diese von \PValue{level} abweichen soll
  %   (siehe auch \DescRef{maincls.counter.tocdepth} in
  %   \autoref{sec:maincls.toc}, \DescPageRef{maincls.counter.tocdepth}). \\
  % PValue{tocnumwidth}
  % & \PName{Länge}
  % & Die Breite, die für die Gliederungsnummer im zum Gliederungsbefehl
  %   gehörenden Eintrag ins Inhaltsverzeichnis reserviert wird.\\
    \bottomrule
  \end{tabularx}
\end{table}

\begin{table}
  \caption[{Eigenschaften des Stils \PValue{part} bei der Konfiguration von
    Gliederungsbefehlen}]{Zusätzliche \PName{Schlüssel} und \PName{Werte} für
    die \PName{Eigenschaften} bei der Konfiguration von Gliederungsbefehlen des
    Stils \PValue{part}\Index{Gliederung>Stil>~\PValue{part}}}%
  \label{tab:maincls-experts.declarepartstyle.keys}%
  \begin{tabularx}{\linewidth}{llX}
    \toprule
    \PName{Schlüssel}
    & \PName{Wert}
    & Bedeutung \\
    \midrule
    \PValue{afterindent}%
    \ChangedAt{v3.26}{\Class{scrbook}\and \Class{scrreprt}\and
      \Class{scrartcl}}%
    & \PName{Schalter}
    & Es wird bestimmt, ob auf die erste Zeile nach der
      Überschrift der aktuelle Absatzeinzug angewendet
      wird. Bei der Einstellung \PValue{bysign} bestimmt das Vorzeichen von
      \PValue{beforeskip} das Verhalten. Ein negativer Wert für
      \PValue{beforeskip} bewirkt dann, dass der Absatzeinzug entfällt. Mit
      den Werten für einfache Schalter (siehe \autoref{tab:truefalseswitch},
      \autopageref{tab:truefalseswitch}) kann die Anwendung des aktuellen
      Absatzeinzugs explizit aktiviert oder deaktiviert werden. Aus
      Kompatibilitätsgründen ist die Voreinstellung
      für \Class{scrartcl} \PValue{false}, für \Class{scrbook} und
      \Class{scrreprt} \PValue{true}.\\
    \PValue{afterskip}
    & \PName{Länge}
    & Der Betrag gibt den vertikalen Abstand nach der Überschrift an.\\
    \PValue{beforeskip}
    & \PName{Länge}
    & Gibt den vertikalen Abstand vor der Überschrift an. Ist
      \OptionValue{afterindent}{bysign}, so wird für den Abstand der Betrag
      von \PName{Länge} verwendet. Es wird also trotzdem ein positiver
      Abstand eingefügt. Negative Werte bedeuten in diesem Fall, dass ein
      Absatzeinzug nach der Überschrift entfällt.\\
    \PValue{font}
    & \PName{Befehle}
    & Die Schrifteinstellungen, die zusätzlich zum Element
      \DescRef{maincls.fontelement.disposition} bei der Ausgabe des
      Textes der Überschrift verwendet werden sollen. Hier sind alle
      \PName{Befehle} erlaubt, die auch über \DescRef{maincls.cmd.setkomafont}
      und \DescRef{maincls.cmd.addtokomafont} für das Element des
      Gliederungsbefehls erlaubt sind.\\
    \PValue{innerskip}
    & \PName{Länge}
    & %\OnlyAt{\Class{scrbook}\and \Class{scrreprt}}%
      Der vertikale Abstand zwischen Präfixzeile und Text der Überschrift bei
      \Class{scrbook} und \Class{scrreprt}.\\
    \PValue{pagestyle}
    & \PName{Seitenstil}
    & %\OnlyAt{\Class{scrbook}\and \Class{scrreprt}}%
      Der Name des Seitenstils, der für die Seite mit der Überschrift
      verwendet werden soll. Es findet keine Überprüfung statt, ob der
      angegebene \PName{Seitenstil} gültig ist. Fehlerhafte Angaben führen
      daher zu Fehlermeldungen bei Verwendung des Gliederungsbefehls. Diese
      Möglichkeit existiert nur bei \Class{scrbook} und \Class{scrreprt}. \\
    \PValue{prefixfont}
    & \PName{Befehle}
    & Die Schrifteinstellungen, die zusätzlich zum Element
      \DescRef{maincls.fontelement.disposition} ab der Ausgabe einer
      Präfixzeile oder Nummer in der Überschrift verwendet werden sollen. Hier
      sind alle \PName{Befehle} erlaubt, die 
      auch über \DescRef{maincls.cmd.setkomafont} und 
      \DescRef{maincls.cmd.addtokomafont} für das Element der Präfixzeile des
      Gliederungsbefehls erlaubt sind.\\
    \bottomrule
  \end{tabularx}
\end{table}

\begin{table}
  \caption[{Eigenschaften des Stils \PValue{chapter} bei der Konfiguration von
    Gliederungsbefehlen}]{Zusätzliche \PName{Schlüssel} und \PName{Werte} für
    die \PName{Eigenschaften} bei der Konfiguration von Gliederungsbefehlen des
    Stils \PValue{chapter}\Index{Gliederung>Stil>~\PValue{chapter}}}%
  \label{tab:maincls-experts.declarechapterstyle.keys}%
  \begin{tabularx}{\linewidth}{llX}
    \toprule
    \PName{Schlüssel} & \PName{Wert} & Bedeutung \\
    \midrule
    \PValue{afterindent}%
    \ChangedAt{v3.26}{\Class{scrbook}\and \Class{scrreprt}}%
    & \PName{Schalter}
    & Es wird bestimmt, ob auf die erste Zeile nach der Überschrift der
      aktuelle Absatzeinzug angewendet wird. Bei der Voreinstellung
      \PValue{bysign} bestimmt das Vorzeichen von \PValue{beforeskip} das
      Verhalten. Ein negativer Wert für \PValue{beforeskip} bewirkt dann, dass
      der Absatzeinzug entfällt. Mit den Werten für einfache Schalter (siehe
      \autoref{tab:truefalseswitch}, \autopageref{tab:truefalseswitch}) kann
      die Anwendung des aktuellen Absatzeinzugs explizit aktiviert oder
      deaktiviert werden.\\
    \PValue{afterskip}%
    \ChangedAt{v3.26}{\Class{scrbook}\and \Class{scrreprt}}%
    & \PName{Länge}
    & Gibt den vertikalen Abstand nach der Überschrift an.\\
    \PValue{beforeskip}
    & \PName{Länge}
    & Gibt den vertikalen Abstand vor der Überschrift an. Ist
      \OptionValue{afterindent}{bysign}, so wird für den Abstand der Betrag
      von \PName{Länge} verwendet. Es wird also trotzdem ein positiver
      Abstand eingefügt. Negative Werte bedeuten in diesem Fall, dass ein
      Absatzeinzug nach der Überschrift entfällt.\\
    \PValue{font}
    & \PName{Befehle}
    & Die Schrifteinstellungen, die zusätzlich zum Element
      \DescRef{maincls.fontelement.disposition} bei der Ausgabe der
      Überschrift verwendet werden sollen. Hier sind alle \PName{Befehle}
      erlaubt, die auch über \DescRef{maincls.cmd.setkomafont} und 
      \DescRef{maincls.cmd.addtokomafont} für das Element des
      Gliederungsbefehls erlaubt sind.\\
    \PValue{innerskip}
    & \PName{Länge}
    & Der vertikale Abstand zwischen Präfixzeile und Text der
      Überschrift, falls eine Präfixzeile verwendet wird.\\
    \PValue{pagestyle}
    & \PName{Seitenstil}
    & Der Name des Seitenstils, der für die Seite mit der Überschrift
      verwendet werden soll. Es findet keine Überprüfung statt, ob der
      angegebene \PName{Seitenstil} gültig ist. Fehlerhafte Angaben führen
      daher zu Fehlermeldungen bei Verwendung des Gliederungsbefehls.\\
    \PValue{prefixfont}
    & \PName{Befehle}
    & Die Schrifteinstellungen, die zusätzlich zum Element
      \DescRef{maincls.fontelement.disposition} und dem Element des
      Gliederungsbefehls bei der Ausgabe einer Präfixzeile in der Überschrift
      verwendet werden sollen. Hier sind alle \PName{Befehle} erlaubt, die
      auch über \DescRef{maincls.cmd.setkomafont} und 
      \DescRef{maincls.cmd.addtokomafont}für das Element der Präfixzeile des
      Gliederungsbefehls erlaubt sind.\\    
    \bottomrule
  \end{tabularx}
\end{table}

\begin{table}
  \caption[{Eigenschaften des Stils \PValue{section} bei der Konfiguration von
    Gliederungsbefehlen}]{Zusätzliche \PName{Schlüssel} und \PName{Werte} für
    die \PName{Eigenschaften} bei der Konfiguration von Gliederungsbefehlen des
    Stils \PValue{section}\Index{Gliederung>Stil>~\PValue{section}}}%
  \label{tab:maincls-experts.declaresectionstyle.keys}%
  \begin{tabularx}{\linewidth}{llX}
    \toprule
    \PName{Schlüssel} & \PName{Wert} & Bedeutung \\
    \midrule
    \PValue{afterindent}%
    \ChangedAt{v3.26}{\Class{scrbook}\and \Class{scrreprt}\and
      \Class{scrartcl}}%
    & \PName{Schalter}
    & Es wird bestimmt, ob auf die erste Zeile nach einer frei stehenden
      Überschrift (siehe \PValue{runin}) der aktuelle Absatzeinzug angewendet
      wird. Bei der Voreinstellung \PValue{bysign} bestimmt das Vorzeichen von
      \PValue{beforeskip} das Verhalten. Ein negativer Wert für
      \PValue{beforeskip} bewirkt dann, dass der Absatzeinzug entfällt. Mit
      den Werten für einfache Schalter (siehe \autoref{tab:truefalseswitch},
      \autopageref{tab:truefalseswitch}) kann die Anwendung des aktuellen
      Absatzeinzugs explizit aktiviert oder deaktiviert werden.\\
    \PValue{afterskip}
    & \PName{Länge}
    & Im Fall einer Spitzmarke (siehe \PValue{runin}) ist der Betrag der
      \PName{Länge} der horizontale Abstand nach der Überschrift. Es
      wird in diesem Fall also immer ein positiver Abstand eingefügt. Im Fall
      einer frei stehenden Überschrift ist \PName{Länge} der vertikale Abstand
      nach der Überschrift. Ist \OptionValue{runin}{bysign}, so
      führt ein positiver Wert zu einer frei stehenden Überschrift, während
      ein negativer Wert oder Null zu einer Spitzmarke führt.\\
    \PValue{beforeskip}
    & \PName{Länge}
    & Gibt den vertikalen Abstand vor der Überschrift an. Ist
      \OptionValue{afterindent}{bysign}, so wird für den Abstand der Betrag
      von \PName{Länge} verwendet. Es wird also trotzdem ein positiver
      Abstand eingefügt. Negative Werte bedeuten in diesem Fall, dass ein
      Absatzeinzug nach der Überschrift entfällt.\\
    \PValue{font}
    & \PName{Befehle}
    & Die Schrifteinstellungen, die zusätzlich zum Element
      \DescRef{maincls.fontelement.disposition} bei der Ausgabe der
      Überschrift verwendet werden sollen. Hier sind alle \PName{Befehle}
      erlaubt, die auch über \DescRef{maincls.cmd.setkomafont} und
      \DescRef{maincls.cmd.addtokomafont} für das Element des
      Gliederungsbefehls erlaubt sind.\\
    \PValue{indent}
    & \PName{Länge}
    & Einzug vom linken Rand vor der Ausgabe der Nummer und des Textes
      der Überschrift.\\
    \PValue{runin}%
    \ChangedAt{v3.26}{\Class{scrbook}\and \Class{scrreprt}\and
      \Class{scrartcl}}%
    & \PName{Schalter}
    & Es wird bestimmt, ob die Überschrift als Spitzmarke\iffree{ (am
        Zeilenanfang)}{} oder frei stehend gesetzt wird. \iffree{}{Bei einer
        Spitzmarke schließt sich der nachfolgende Text direkt an die
        Überschrift an, während bei einer frei stehenden Überschrift erst noch
        ein Absatz und ein vertikaler Abstand (siehe \PValue{afterskip})
        folgen. }%
      Bei der Voreinstellung \PValue{bysign} bestimmt das Vorzeichen von
      \PValue{afterskip} das Verhalten. Ein positiver Wert für
      \PValue{afterskip} bewirkt dann eine frei stehende Überschrift.
      Darüber hinaus kann mit den Werten für einfache Schalter (siehe
      \autoref{tab:truefalseswitch}, \autopageref{tab:truefalseswitch})
      eine Spitzmarke explizit aktiviert oder deaktiviert werden.\\
    \bottomrule
  \end{tabularx}
\end{table}

\Macro{DeclareNewSectionCommand}\textnote{Neudefinition} dient der Definition
eines neuen Gliederungsbefehls. Ist derselbe \PName{Name} von \TeX{} bereits
anderweitig belegt, so wird ein Fehler ausgegeben und es findet keine
Umdefinierung statt.

\Macro{ProvideSectionCommand}\textnote{bedingte Definition} verhält sich
ähnlich, gibt aber keine Fehlermeldung aus, wenn der Gliederungsbefehl bereits
existiert. Es findet dann nur keine Umdefinierung statt.

\Macro{RedeclareSectionCommand}\textnote{Änderung} kann hingegen nur verwendet
werden, um eine existierende Anweisung zu einem Gliederungsbefehl mit den
angegebenen \PName{Eigenschaften} zu ändern. Dabei wird nicht überprüft, ob
\Macro{\PName{Name}} bereits zuvor ein Gliederungsbefehl war. Es muss nur ein
von \TeX{} bereits belegter \PName{Name} sein.

Bei \Macro{DeclareSectionCommand}\textnote{unbedingte Definition} findet
keinerlei Überprüfung statt, ob \PName{Name} von \TeX{} bereits anderweitig
belegt ist. Stattdessen wird der Gliederungsbefehl \Macro{\PName{Name}}
unbedingt entsprechend der angegebenen \PName{Eigenschaften} definiert.

Zu jeder Gliederungsanweisung gehört außerdem ein Zähler: \PName{Name}, der
bei Bedarf von allen vier Befehlen mit \Macro{newcounter} neu angelegt
wird. Dasselbe gilt für die Ausgabe des Zählers: \Macro{the\PName{Name}}, die
Formatierung des Zählers: \Macro{\PName{Name}format}, die Anweisung zur
Erstellung eines Kolumnentitels: \Macro{\PName{Name}mark}, die dabei
verwendete Formatierung des Zählers: \Macro{\PName{Name}markformat}, die oben
erwähnten Elemente: \FontElement{\PName{Name}} und gegebenenfalls
\FontElement{\PName{Name}prefix}, die numerische Gliederungsebene:
\Macro{\PName{Name}numdepth}. Die Anweisung \Macro{\PName{Name}mark} wird
gegebenenfalls so vordefiniert, dass kein Kolumnentitel erzeugt wird. Die
Ausgabe des Zählers, \Macro{the\PName{Name}}, wird als arabische Zahl
vordefiniert. Wird über den \PName{Schlüssel} \PValue{counterwithin} der
Zähler als von einem anderen Zähler abhängig definiert, so wird in der Ausgabe
dieser andere Zähler mit einem Punkt getrennt vorangestellt.

Neben\ChangedAt[2016/03]{v3.20}{\Class{scrbook}\and \Class{scrreprt}\and
  \Class{scrartcl}} dem Gliederungsbefehl selbst wird auch ein Befehl für
einen Eintrag ins Inhaltsverzeichnis definiert. Dafür wird auf das Paket
\hyperref[cha:tocbasic]{\Package{tocbasic}}%
\important{\hyperref[cha:tocbasic]{\Package{tocbasic}}}\IndexPackage{tocbasic}
zurückgegriffen. Der Stil des Verzeichniseintrags wird über die Eigenschaft
\PValue{tocstyle} festgelegt. Wird hier mit \OptionValue{tocstyle}{} oder
\OptionValue{tocstyle}{\{\}} ein leerer \PName{Name} angegeben, so erfolgt
keine Umdefinierung des Befehls für den Verzeichniseintrag. Das ist
beispielsweise dann wichtig, wenn Sie ein zusätzliches Paket zur Modifikation
des Inhaltsverzeichnisses verwenden. Fehlt die Eigenschaft \PValue{tocstyle},
so wird bei der Umdefinierung der bisherige Stil erneut verwendet.

Unterschiedliche\ChangedAt[2016/03]{v3.20}{\Class{scrbook}\and
  \Class{scrreprt}\and \Class{scrartcl}} Stile für Verzeichniseinträge haben
unterschiedliche zusätzliche Eigenschaften. Diese können, mit dem Präfix
\PValue{toc} versehen, direkt mit angegeben werden. So kann beispielsweise die
Ebene des Verzeichniseintrags, die bei allen von den \KOMAScript-Klassen und
\hyperref[cha:tocbasic]{\Package{tocbasic}} definierten Stilen als Eigenschaft
\PValue{level} bekannt ist, mit \PValue{toclevel} gesetzt werden, der Einzug
des Eintrags, \PValue{indent}, über \PValue{tocindent} und die für die Nummer
reservierte Breite, \PValue{numwidth}, mit \PValue{tocnumwidth}. Für weitere
Eigenschaften der Verzeichniseinträge siehe \autoref{sec:tocbasic.tocstyle} ab
Seite \autopageref{sec:tocbasic.tocstyle}.

\begin{Example}
  Aus unerfindlichen Gründen sollen die Überschriften von
  \DescRef{maincls.cmd.paragraph} nicht mehr als Spitzmarken, sondern als
  Überschriften ähnlich \DescRef{maincls.cmd.subsubsection} umdefiniert
  werden. Dabei soll über der Überschrift ein kleiner Abstand von 10\,pt und
  unter der Überschrift kein zusätzlicher Abstand eingefügt werden. Das wäre
  bereits mit
\begin{lstcode}
  \RedeclareSectionCommand[%
    beforeskip=-10pt,%
    afterskip=1sp%
  ]{paragraph}
\end{lstcode}
  möglich. Durch den negativen Wert bei \PValue{beforeskip} wird der vertikale
  Abstand über der Überschrift erzeugt und gleichzeitig der Einzug des ersten
  Abschnitts nach der Überschrift abgeschaltet. Obwohl eigentlich nach der
  Überschrift kein vertikaler Abstand gewünscht wird, wurde als Wert hier
  1\,sp angegeben. Der Grund ist einfach: Einen Wert von 0\,pt betrachtet
  \TeX{} nicht als positiven Wert und \Macro{RedeclareSectionCommand} erzeugt
  damit unter Berücksichtigung der Voreinstellung \OptionValue{runin}{bysign}
  eine Überschrift in Form einer Spitzmarke. Der kleinste positive Wert
  ist 1\,sp. Natürlich könnte man stattdessen auch explizit die Verwendung von
  Spitzmarken aktivieren und dann wirklich einen Wert von 0\,pt einstellen:
\begin{lstcode}
  \RedeclareSectionCommand[%
    beforeskip=-10pt,%
    runin=true,%
    afterskip=0pt%
  ]{paragraph}
\end{lstcode}

  In der Regel ist es für den vertikalen Ausgleich (siehe
  \DescRef{maincls.cmd.flushbottom}, \autoref{sec:maincls.typearea},
  \DescPageRef{maincls.cmd.flushbottom}) besser, wenn man die Abstände
  mit etwas Spielraum, dem sogenannten Leim, versieht:
\begin{lstcode}
  \RedeclareSectionCommand[%
    beforeskip=-10pt plus -2pt minus -1pt,%
    runin=true,%
    afterskip=0pt%
  ]{paragraph}
\end{lstcode}
  Dabei ist zu beachten, dass natürlich auch der Leim bei der Anwendung als
  vertikaler Abstand das Vorzeichen wechselt, also bei \PValue{beforeskip} im
  Beispiel negativ angegeben wird. Auch diese Überlegung kann man sich jedoch
  sparen, indem man den Erstzeileneinzug nach der Überschrift nicht über das
  Vorzeichen von \Option{beforeskip} steuert, sondern über
  \Option{afterindent}:
\begin{lstcode}
  \RedeclareSectionCommand[%
    afterindent=false,%
    beforeskip=10pt plus 2pt minus 1pt,%
    runin=true,%
    afterskip=0pt%
  ]{paragraph}
\end{lstcode}
\end{Example}

Dass im Beispiel nur die Schlüssel verwendet werden mussten, deren Einstellung
geändert werden soll, liegt daran, dass seit \KOMAScript{} 3.15
\DescRef{maincls.cmd.paragraph} intern bereits mit
\Macro{DeclareSectionCommand} definiert wird. Alle anderen Einstellungen
aus der Originaldefinition bleiben so enthalten. Bei
\Class{scrartcl}\OnlyAt{scrartcl} entspräche dies der Definition:
\begin{lstcode}
  \DeclareSectionCommand[%
    level=4,
    indent=0pt,
    afterindent=bysign,
    runin=bysign,
    beforeskip=3.25ex plus 1ex minus .2ex,
    afterskip=-1em,
    font={},
    tocindent=7em,
    tocnumwidth=4.1em,
    counterwithin=subsubsection
  ]{paragraph}
\end{lstcode}

% Umbruchkorrektur: Tabelle verschoben
\begin{table}
  %\centering
  \KOMAoptions{captions=topbeside}%
  \setcapindent{0pt}%
  % \caption
  \begin{captionbeside}
    [{Voreinstellungen für die Kapitelüberschriften von \Class{scrbook}
      und \Class{scrreprt} in Abhängigkeit von Option
      \DescRef{maincls.option.headings}}]
    {\hskip 0pt plus .5em
      Voreinstellungen für die Kapitelüberschriften von \Class{scrbook}
      und \Class{scrreprt} in Abhängigkeit von Option
      \DescRef{maincls.option.headings}%
      \label{tab:maincls.chapter.skips}}
    [l]
  \begin{tabular}[t]{ll}
    \multicolumn{2}{@{}l}{\bfseries Mit \OptionValueRef{maincls}{headings}{big}:}\\
    \toprule
    Einstellung & voreingestellter Wert \\
    \midrule
    \PValue{afterskip}
      & \PValue{1.725\Length{baselineskip} plus .115\Length{baselineskip}} \\
      & \PValue{\phantom{1.725\Length{baselineskip}}
         minus .192\Length{baselineskip}} \\
    \PValue{beforeskip} 
      & \PValue{-3.3\Length{baselineskip}-\Length{parskip}} \\
    \PValue{font} & \Macro{huge} \\
    \bottomrule\\
    \multicolumn{2}{@{}l}{\bfseries Mit \OptionValueRef{maincls}{headings}{normal}:}\\
    \toprule
    Einstellung & voreingestellter Wert \\
    \midrule
    \PValue{afterskip}
      & \PValue{1.5\Length{baselineskip} plus .1\Length{baselineskip}} \\
      & \PValue{\phantom{1.5\Length{baselineskip}}
        minus .167\Length{baselineskip}} \\
    \PValue{beforeskip} 
      & \PValue{-3\Length{baselineskip}-\Length{parskip}} \\
    \PValue{font} & \Macro{LARGE} \\
    \bottomrule\\
    \multicolumn{2}{@{}l}{\bfseries Mit \OptionValueRef{maincls}{headings}{small}:}\\
    \toprule
    Einstellung & voreingestellter Wert \\
    \midrule
    \PValue{afterskip}
      & \PValue{1.35\Length{baselineskip} plus .09\Length{baselineskip}} \\
      & \PValue{\phantom{1.35\Length{baselineskip}}
        minus .15\Length{baselineskip}} \\
    \PValue{beforeskip} 
      & \PValue{-2.8\Length{baselineskip}-\Length{parskip}} \\
    \PValue{font} & \Macro{Large} \\
    \bottomrule
  \end{tabular}
  \end{captionbeside}
\end{table}

Die Werte für \Class{scrreprt} und \Class{scrbook}\OnlyAt{\Class{scrbook}\and
  \Class{scrreprt}} weichen teilweise ab.  Für \DescRef{maincls.cmd.chapter}
sind einige Einstellungen für die Überschriften von Option
\DescRef{maincls.option.headings} (siehe \autoref{sec:maincls.structure},
\DescPageRef{maincls.option.headings}) abhängig. Diese abhängigen
Einstellungen sind in \autoref{tab:maincls.chapter.skips} zu finden. Eine
Übersicht über alle Voreinstellungen bietet
\autoref{tab:maincls.section.defaults}. Es ist zu beachten, dass dabei
\PValue{1ex} und \Length{baselineskip} von der voreingestellten Größe der
Überschrift beziehungsweise des Inhaltsverzeichniseintrags abhängig
sind. Weitere Voreinstellungen\ChangedAt[2016/03]{v3.20}{\Class{scrbook}\and
  \Class{scrreprt}\and \Class{scrartcl}} für die Stile der Verzeichniseinträge
sind \autoref{sec:tocbasic.tocstyle}%
\important{\hyperref[cha:tocbasic]{\Package{tocbasic}}} ab
\autopageref{sec:tocbasic.tocstyle} zu entnehmen.

Das in den Einstellungen für \DescRef{maincls.cmd.subparagraph} verwendete
interne Makro \Macro{scr@parindent} ist übrigens, der per Option
\DescRef{maincls.option.parskip} oder Befehl
\DescRef{maincls-experts.cmd.setparsizes} eingestellte Absatzeinzug.%
% Umbruchkorrektur: longtable verschoben
% begin{table}
% \centering
  \begin{longtable}{@{}p{\columnwidth}@{}}
    \caption{Voreinstellungen für die Formatierung der Überschriften von
      \Class{scrbook} und \Class{scrreprt}}%
    \label{tab:maincls.section.defaults}\\
    \endfirsthead
    \caption[]{Voreinstellungen für die Formatierung der Überschriften von
      \Class{scrbook} und
      \Class{scrreprt} \emph{(Fortsetzung)}}\\
    \addlinespace[-\normalbaselineskip] \endhead
    \raggedleft\dots\\
    \endfoot \endlastfoot
    \DescRef{maincls.cmd.part}: \\*
    \begin{tabularx}{\linewidth}{ll}
    \toprule
    Einstellung & voreingestellter Wert \\
    \midrule
    \PValue{afterskip}   & \PValue{0pt plus 1fil} \\
    \PValue{beforeskip}  & \PValue{0pt plus 1fil + \Length{baselineskip}} \\
    \PValue{font}        & siehe  \DescRef{maincls.fontelement.part}, 
                           \autoref{tab:maincls.structureElementsFont}, 
                           \autopageref{tab:maincls.structureElementsFont} \\
    \PValue{innerskip}   & \PValue{20pt} \\
    \PValue{level}       & -1 \\
    \PValue{prefixfont}  & siehe 
                           \DescRef{maincls.fontelement.partnumber},
                           \autoref{tab:maincls.structureElementsFont}, 
                           \autopageref{tab:maincls.structureElementsFont} \\
    \PValue{tocindent}   & \PValue{0pt} \\
    \PValue{toclevel}    & -1 \\
    \PValue{tocnumwidth} & \PValue{2em} \\
    \PValue{tocstyle}    & \PValue{part} \\
    \bottomrule
    \end{tabularx} \\
    \addlinespace[\normalbaselineskip]
    \DescRef{maincls.cmd.chapter}: \\*
    \begin{tabularx}{\linewidth}{ll}
    \toprule
    Einstellung & voreingestellter Wert \\
    \midrule
    \PValue{afterskip}   & siehe \autoref{tab:maincls.chapter.skips} \\
    \PValue{beforeskip}  & siehe \autoref{tab:maincls.chapter.skips} \\
    \PValue{font}        & siehe  \DescRef{maincls.fontelement.chapter},
                           \autoref{tab:maincls.structureElementsFont}, 
                           \autopageref{tab:maincls.structureElementsFont} \\
    \PValue{innerskip}   & \PValue{0.5\Length{baselineskip}} \\
    \PValue{level}       & 0 \\
    \PValue{prefixfont}  & siehe 
                           \DescRef{maincls.fontelement.chapterprefix},
                           \autoref{tab:maincls.structureElementsFont}, 
                           \autopageref{tab:maincls.structureElementsFont} \\
    \PValue{tocindent}   & \PValue{0pt} \\
    \PValue{toclevel}    & 0 \\
    \PValue{tocnumwidth} & \PValue{1.5em} \\
    \PValue{tocstyle}    & \PValue{chapter} \\
    \bottomrule
    \end{tabularx} \\
    \addlinespace[\normalbaselineskip]
    \DescRef{maincls.cmd.section}: \\*
    \begin{tabularx}{\linewidth}{ll}
    \toprule
    Einstellung & voreingestellter Wert \\
    \midrule
    \PValue{afterskip}   & \PValue{2.3ex plus .2ex} \\
    \PValue{beforeskip}  & \PValue{-3.5ex plus -1ex minus -.2ex} \\
    \PValue{font}        & siehe  \DescRef{maincls.fontelement.section},
                           \autoref{tab:maincls.structureElementsFont}, 
                           \autopageref{tab:maincls.structureElementsFont} \\
    \PValue{indent}      & \PValue{0pt} \\
    \PValue{level}       & 1 \\
    \PValue{tocindent}   & \PValue{1.5em}\\
    \PValue{toclevel}    & 1 \\
    \PValue{tocnumwidth} & \PValue{2.3em}\\
    \PValue{tocstyle}    & \PValue{section} \\
    \bottomrule
    \end{tabularx} \\
    \addlinespace[\normalbaselineskip]
    \DescRef{maincls.cmd.subsection}: \\*
    \begin{tabularx}{\linewidth}{ll}
    \toprule
    Einstellung & voreingestellter Wert \\
    \midrule\nopagebreak
    \PValue{afterskip}   & \PValue{1.5ex plus .2ex} \\
    \PValue{beforeskip}  & \PValue{-3.25ex plus -1ex minus -.2ex} \\
    \PValue{font}        & siehe 
                           \DescRef{maincls.fontelement.subsection},
                           \autoref{tab:maincls.structureElementsFont}, 
                           \autopageref{tab:maincls.structureElementsFont} \\
    \PValue{indent}      & \PValue{0pt} \\
    \PValue{level}       & 2 \\
    \PValue{tocindent}   & \PValue{3.8em}\\
    \PValue{toclevel}    & 2 \\
    \PValue{tocnumwidth} & \PValue{3.2em}\\
    \PValue{tocstyle}    & \PValue{section} \\
    \bottomrule
    \end{tabularx} \\
    \addlinespace[\normalbaselineskip]
    \DescRef{maincls.cmd.subsubsection}: \\*
    \begin{tabularx}{\linewidth}{ll}
    \toprule
    Einstellung & voreingestellter Wert \\
    \midrule\nopagebreak
    \PValue{afterskip}   & \PValue{1.5ex plus .2ex} \\
    \PValue{beforeskip}  & \PValue{-3.25ex plus -1ex minus -.2ex} \\
    \PValue{font}        & siehe 
                           \DescRef{maincls.fontelement.subsubsection},
                           \autoref{tab:maincls.structureElementsFont}, 
                           \autopageref{tab:maincls.structureElementsFont} \\
    \PValue{indent}      & \PValue{0pt} \\
    \PValue{level}       & 3 \\
    \PValue{tocindent}   & \PValue{7.0em}\\
    \PValue{toclevel}    & 3 \\
    \PValue{tocnumwidth} & \PValue{4.1em}\\
    \PValue{tocstyle}    & \PValue{section} \\
    \bottomrule
    \end{tabularx} \\
    \addlinespace[\normalbaselineskip]
    \DescRef{maincls.cmd.paragraph}: \\*
    \begin{tabularx}{\linewidth}{ll}
    \toprule
    Einstellung & voreingestellter Wert \\
    \midrule\nopagebreak
    \PValue{afterskip}   & \PValue{-1em} \\
    \PValue{beforeskip}  & \PValue{3.25ex plus 1ex minus .2ex} \\
    \PValue{font}        & siehe 
                           \DescRef{maincls.fontelement.paragraph},
                           \autoref{tab:maincls.structureElementsFont}, 
                           \autopageref{tab:maincls.structureElementsFont} \\
    \PValue{indent}      & \PValue{0pt} \\
    \PValue{level}       & 4 \\
    \PValue{tocindent}   & \PValue{10em}\\
    \PValue{toclevel}    & 4 \\
    \PValue{tocnumwidth} & \PValue{5em}\\
    \PValue{tocstyle}    & \PValue{section} \\
    \bottomrule
    \end{tabularx} \\
    \addlinespace[\normalbaselineskip]
    \DescRef{maincls.cmd.subparagraph}:  \\*
    \begin{tabularx}{\linewidth}{ll}
    \toprule
    Einstellung & voreingestellter Wert \\
    \midrule\nopagebreak
    \PValue{afterskip}   & \PValue{-1em} \\
    \PValue{beforeskip}  & \PValue{3.25ex plus 1ex minus .2ex} \\
    \PValue{font}        & siehe 
                           \DescRef{maincls.fontelement.subparagraph},
                           \autoref{tab:maincls.structureElementsFont}, 
                           \autopageref{tab:maincls.structureElementsFont} \\
    \PValue{indent}      & \Macro{scr@parindent} \\
    \PValue{level}       & 5 \\
    \PValue{tocindent}   & \PValue{12em}\\
    \PValue{toclevel}    & 5 \\
    \PValue{tocnumwidth} & \PValue{6em}\\
    \PValue{tocstyle}    & \PValue{section} \\
    \bottomrule
    \end{tabularx}
  \end{longtable}
%end{table}
\EndIndexGroup


\begin{Declaration}
  \Macro{DeclareSectionCommands}\OParameter{Einstellungen}%
                                \Parameter{Namensliste}
  \Macro{DeclareNewSectionCommands}\OParameter{Einstellungen}%
                                   \Parameter{Namensliste}
  \Macro{RedeclareSectionCommands}\OParameter{Einstellungen}%
                                  \Parameter{Namensliste}
  \Macro{ProvideSectionCommands}\OParameter{Einstellungen}%
                                \Parameter{Namensliste}
\end{Declaration}
Diese\ChangedAt{v3.15}{\Class{scrbook}\and \Class{scrreprt}\and
  \Class{scrartcl}} Anweisungen können gleich eine ganze Reihe von
Gliederungsbefehlen definieren oder ändern. Dabei ist \PName{Namensliste} eine
durch Komma separierte Liste von Namen der Gliederungsbefehle.

Die Befehle unterscheiden sich in zwei weiteren Punkten von den zuvor
erklärten Anweisungen zur Definition oder Änderung eines einzelnen
Gliederungsbefehls. Zum einen wird im Fehlerfall\textnote{Fehlerbehandlung},
also wenn eine Anweisung bei \Macro{DeclareNewSectionCommands} bereits zuvor
existierte oder bei \Macro{RedeclareSectionCommands} noch nicht existierte,
die Definition dennoch vorgenommen. Ein entsprechender Fehler wird natürlich
trotzdem gemeldet.  Zum anderen gibt es eine weitere Einstellung:
\important{\PValue{increaselevel}}\OptionVName{increaselevel}{Ganzzahl}.
Damit ändert sich die Bedeutung von \PValue{level} und \PValue{toclevel}
(siehe \autoref{tab:maincls-experts.declaresection.keys},
\autopageref{tab:maincls-experts.declaresection.keys}) dahingehend, dass deren
Werte lediglich als \PName{Einstellungen} des ersten Gliederungsbefehls aus
der \PName{Namensliste} dienen. Für alle weiteren Gliederungsbefehle werden
die Werte von \PValue{level} und \PValue{toclevel} um den Wert von
\PValue{increaselevel} erhöht. Wurde die Einstellung \PValue{increaselevel}
ohne Wertzuweisung verwendet, so wird der Wert 1 angenommen.%
\EndIndexGroup


\begin{Declaration}[0]
  \Macro{IfSectionCommandStyleIs}\Parameter{Name}\Parameter{Stil}
  \Parameter{Dann-Code}\Parameter{Sonst-Code}
\end{Declaration}
In\ChangedAt{v3.27}{\Class{scrbook}\and \Class{scrreprt}\and \Class{scrartcl}}
seltenen Fällen ist es nützlich, testen zu können, ob ein Gliederungsbefehl
einem bestimmten \PName{Stil} angehört. Ist der mittels \PName{Name} bestimmte
Gliederungsbefehl aus \KOMAScript-Sicht derzeit mit dem angebenen \PName{Stil}
definiert, so wird der \PName{Dann-Code} ausgeführt, anderenfalls der
\PName{Sonst-Code}. Ist \Macro{\PName{Name}} nicht definiert oder kein mit
\KOMAScript-Mitteln definierter Gliederungsbefehl, so wird dies als Fehler
gemeldet.%
\EndIndexGroup


\begin{Declaration}
  \Macro{chapterheadstartvskip}
  \Macro{chapterheadmidvskip}
  \Macro{chapterheadendvskip}
  \Macro{partheadstartvskip}
  \Macro{partheadmidvskip}
  \Macro{partheadendvskip}
  \Macro{partheademptypage}
\end{Declaration}
Diese\important[i]{\DescRef{maincls.cmd.chapter}\\
  \DescRef{maincls.cmd.part}\\
  \DescRef{maincls.cmd.addchap}\\
  \DescRef{maincls.cmd.addpart}\\
  \DescRef{maincls.cmd.chapter*}\\
  \DescRef{maincls.cmd.part*}\\
  \DescRef{maincls.cmd.addchap*}\\
  \DescRef{maincls.cmd.addpart*}} Anweisungen werden innerhalb von
Überschriften der zuvor erklärten Stile \PValue{chapter} und \PValue{part} und
damit für die Definition der Überschriften
\DescRef{maincls.cmd.chapter}\IndexCmd{chapter},
\DescRef{maincls.cmd.part}\IndexCmd{part},
\DescRef{maincls.cmd.addchap}\IndexCmd{addchap},
\DescRef{maincls.cmd.addpart}\IndexCmd{addpart} und deren Sternvarianten
\DescRef{maincls.cmd.chapter*}\IndexCmd{chapter*},
\DescRef{maincls.cmd.part*}\IndexCmd{part*},
\DescRef{maincls.cmd.addchap*}\IndexCmd{addchap*},
\DescRef{maincls.cmd.addpart*}\IndexCmd{addpart*} verwendet. Dabei ist
\Macro{chapterheadstartvskip} eine Anweisung, die dafür vorgesehen ist, vor
der Kapitelüberschrift einen vertikalen Abstand einzufügen. Entsprechend ist
\Macro{chapterheadendvskip} eine Anweisung, die dafür vorgesehen ist, nach der
Kapitelüberschrift einen vertikalen Abstand
einzufügen. Bei\ChangedAt{v3.15}{\Class{scrbook}\and \Class{scrreprt}}
Kapitelüberschriften mit eigener Nummernzeile (siehe Option
\DescRef{maincls.option.chapterprefix} in \autoref{sec:maincls.structure},
\DescPageRef{maincls.option.chapterprefix}) wird zwischen der Nummernzeile und
der eigentlichen Überschrift außerdem \Macro{chapterheadmidvskip} ausgeführt.

Für das Einfügen der vertikalen Abstände über und unter Teile-Überschriften
sind die Anweisungen \Macro{partheadstartvskip} und \Macro{partheadendvskip}
vorgesehen. Dabei wird ein Seitenumbruch als Teil des vertikalen Abstandes
interpretiert. Ein solcher Seitenumbruch ist in der Voreinstellung sowohl bei
\Class{scrbook}\OnlyAt{\Class{scrbook}\and \Class{scrreprt}} als auch
\Class{scrreprt} in der Definition von \Macro{partheadendvskip} enthalten. Die
Anweisung \Macro{partheadmidvskip} ist für den Abstand zwischen der
Teile-Nummer und dem Text der Teile-Überschrift vorgesehen. Die Anweisung
\Macro{partheademptypage}\ChangedAt{v3.02}{\Class{scrbook}\and
  \Class{scrreprt}} wird bei \Class{scrbook} und \Class{scrreprt}
gegebenenfalls für die leere Seite nach der Überschrift verwendet.

Die Voreinstellungen der sieben Anweisungen sind seit
\KOMAScript~3.15\ChangedAt{v3.15}{\Class{scrbook}\and \Class{scrreprt}\and
  \Class{scrartcl}} von der Einstellung von Option
\DescRef{maincls.option.headings} (siehe \autoref{sec:maincls.structure},
\DescPageRef{maincls.option.headings}) unabhängig. Die Original-Definitionen
für die Kapitelüberschriften ab
\KOMAScript~3.17\ChangedAt{v3.17}{\Class{scrbook}\and \Class{scrreprt}\and
  \Class{scrartcl}} entsprechen:\IndexLength{@tempskipa}
\begin{lstcode}
  \newcommand*{\chapterheadstartvskip}{\vspace{\@tempskipa}}
  \newcommand*{\chapterheadmidvskip}{\par\nobreak
                                     \vskip\@tempskipa}
  \newcommand*{\chapterheadendvskip}{\vskip\@tempskipa}
\end{lstcode}
Diese werden auch bei jeder Verwendung von Option
\OptionValueRef{maincls}{headings}{big}\IndexOption{headings},
\OptionValueRef{maincls}{headings}{normal} oder
\OptionValueRef{maincls}{headings}{small} reaktiviert. Als Seiteneffekt haben
diese Optionen also gegebenenfalls nicht nur Auswirkungen auf
Kapitelüberschriften, sondern auf alle Überschriften im Stil \PValue{chapter}.

Der Stil \PValue{chapter} setzt die interne Länge
\Length{@tempskipa}\IndexLength{@tempskipa} vor Aufruf von
\Macro{chapterheadstartvskip} automatisch auf den Wert, der sich aus der
\DescRef{\LabelBase.cmd.DeclareSectionCommand}-Einstellung%
\IndexCmd{DeclareSectionCommand} \PValue{beforeskip} ergibt. Vor dem Aufruf
von \Macro{chapterheadendvskip} geschieht entsprechendes mit dem Wert, der
sich aus der Einstellung
\PValue{afterskip} ergibt, und vor dem Aufruf von \Macro{chapterheadmidvskip}
mit dem Wert, der sich aus der Einstellung \PValue{innerskip}
ergibt.

Da die Voreinstellungen für die Abstände von \DescRef{maincls.cmd.part} nicht
von Option \DescRef{maincls.option.headings} abhängen, werden auch die
zugehörigen Anweisungen von dieser Option nicht neu
definiert. Ihre\ChangedAt{v3.17}{\Class{scrbook}\and \Class{scrreprt}\and
  \Class{scrartcl}} Original-Definitionen entsprechen bei \Class{scrbook} und
\Class{scrreprt}:
\begin{lstcode}
  \newcommand*{\partheadstartvskip}{%
    \null\vskip-\baselineskip\vskip\@tempskipa
  }
  \newcommand*{\partheadmidvskip}{%
    \par\nobreak\vskip\@tempskipa
  }
  \newcommand*{\partheadendvskip}{%
    \vskip\@tempskipa\newpage
  }
\end{lstcode}
und bei \Class{scrartcl}:
\begin{lstcode}
  \newcommand*{\partheadstartvskip}{%
    \addvspace{\@tempskipa}%
  }
  \newcommand*{\partheadmidvskip}{%
    \par\nobreak
  }
  \newcommand*{\partheadendvskip}{%
    \vskip\@tempskipa
  }
\end{lstcode}
Auch hier wird vom Stil \PValue{part} die interne Länge
\Length{@tempskipa}\IndexLength{@tempskipa}\IndexLength{@tempskipa} vor der
Verwendung der Befehle entsprechend der Einstellungen von
\DescRef{\LabelBase.cmd.DeclareSectionCommand}\IndexCmd{DeclareSectionCommand}
gesetzt.

Wird eine der Anweisungen, die im Original \Length{@tempskipa} für den
vertikalen Abstand verwendet, umdefiniert und sollen die Abstände weiterhin
beispielsweise mit \DescRef{\LabelBase.cmd.RedeclareSectionCommand}
konfigurierbar sein, so sollte man in der neuen Definition ebenfalls
\Length{@tempskipa} verwenden. Da\ChangedAt{v3.17}{\Class{scrbook}\and
  \Class{scrreprt}\and \Class{scrartcl}} die Abstände über, innerhalb und
unter den Überschriften einfacher mit
\DescRef{\LabelBase.cmd.RedeclareSectionCommand} eingestellt werden können,
wird generell nicht empfohlen, zu diesem Zweck stattdessen die hier
beschriebenen Anweisungen umzudefinieren. Sie sollten für tiefgreifendere
Änderungen reserviert bleiben, die nicht über
\DescRef{\LabelBase.cmd.RedeclareSectionCommand} zu erreichen sind.
Auf\textnote{Beispiel!} \cite{homepage} findet sich dazu ein Beispiel, bei
dem durch Umdefinierung von \Macro{chapterheadstartvskip} und
\Macro{chapterheadendvskip} Linien über und unter der Kapitelüberschrift
gesetzt werden.%
%
\EndIndexGroup


\begin{Declaration}
  \Macro{partlineswithprefixformat}%
    \Parameter{Ebene}\Parameter{Nummer}\Parameter{Text}
\end{Declaration}
Diese\ChangedAt{v3.25}{\Class{scrbook}\and \Class{scrreprt}\and
  \Class{scrartcl}} Anweisung wird von Überschriften des Stils
\PValue{part}\textnote{Gliederungsstil
  \PValue{part}}\Index{Gliederung>Stil>~\PValue{part}} verwendet. Die beiden
Argumente \PName{Nummer} und \PName{Text} sind dabei einschließlich
Einstellung der Fonts für sich bereits fertig formatiert. Letztlich regelt die
Anweisung also die Anordnung der beiden Teile der Überschrift. Bei nicht
nummerierten Überschriften ist \PName{Nummer} ein komplett leeres Argument,
enthält also auch keine Formatierungsanweisungen.

Vordefiniert ist die Anweisung mit:
\begin{lstcode}
  \newcommand{\partlineswithprefixformat}[3]{#2#3}
\end{lstcode}
\iffalse bisher eher spartanisch.\fi% Umbruchkorrektur

\begin{Example}
  Sie wollen die Teile-Überschriften in eine hellblaue Box mit blauer
  Umrandung stellen, die nur etwa drei Viertel der Breite des Textbereichs
  einnimmt. Hierzu verwenden Sie
\begin{lstcode}[moretexcs={fcolorbox}]
  \documentclass{scrbook}
  \usepackage{xcolor}
  \renewcommand*{\partlineswithprefixformat}[3]{%
    \fcolorbox{blue}{blue!25}{%
      \parbox{.75\linewidth}{#2#3}%
    }%
  }
  \begin{document}
  \part{Umrahmte Teile}
  \end{document}  
\end{lstcode}
  Allerdings fällt Ihnen auf, dass die Überschrift dabei nicht wie sonst
  für Teile üblich zentriert wird -- weder die Box selbst noch der Text
  innerhalb der Box.

  Die Ursache für die fehlende Zentrierung der Box liegt
  darin, dass aufgrund der Änderung das in Argument 3 versteckte Absatzende
  nur noch den Absatz innerhalb der Box beendet, aber nicht mehr den Absatz mit
  der \Macro{parbox}. Also ergänzen Sie ein \Macro{par} am Ende der Definition.

  Die Ursache der fehlenden Zentrierung innerhalb der Box ist, dass in der
  \Macro{parbox}-Anweisung die Ausrichtung von
  \DescRef{maincls.cmd.raggedpart} nicht automatisch gültig ist. Daher
  ergänzen Sie diese Anweisung innerhalb der Box.

  Mit
\begin{lstcode}[moretexcs={fcolorbox}]
  \documentclass{scrbook}
  \usepackage{xcolor}
  \renewcommand*{\partlineswithprefixformat}[3]{%
    \fcolorbox{blue}{blue!25}{%
      \parbox{.75\linewidth}{\raggedpart #2#3}%
    }%
    \par
  }
  \begin{document}
  \part{Umrahmte Teile}
  \end{document}  
\end{lstcode}
  erhalten Sie die gewünschte Formatierung.
\end{Example}

Wie im Beispiel gezeigt, ist der Anwender bei der Umdefinierung der Anweisung
für einige Dinge selbst verantwortlich. Dazu gehört neben dem Erhalt der
voreingestellten Ausrichtung auch, dass innerhalb der Überschrift kein
Seitenumbruch, beispielsweise an zusätzlich eingefügten Absätzen oder
Abständen, erfolgen kann. Das gezeigte Beispiel ist diesbezüglich
unproblematisch. Nicht nur, dass die beiden Boxen ohnehin keinen Seitenumbruch
erlauben. \KOMAScript{} verändert außerdem \Macro{interlinepenalty} als Teil
von \PName{Text} so, dass darin kein Seitenumbruch stattfinden
darf. Gleichzeitig endet \PName{Text} immer mit einem internen Absatz,
\Macro{@@par}.

Das Argument \PName{Ebene} wird von
\Macro{partlineswithprefixformat} in der Voreinstellung nicht verwendet und
ist auch im Beispiel nicht erforderlich. Erst wenn der Anwender mehrere
Anweisungen im Stil \PValue{part} definiert und man innerhalb der Definition
nach den Anweisungen unterscheiden will, kann dies über \PName{Ebene}
erfolgen. Dabei ist \PName{Ebene} der vom Namen der Anweisung abgeleitete Name
der Gliederungsebene. Bei \DescRef{maincls.cmd.part},
\DescRef{maincls.cmd.part*}, \DescRef{maincls.cmd.addpart} und
\DescRef{maincls.cmd.addpart*} ist \PName{Ebene} daher einheitlich
\PValue{part}.%
\EndIndexGroup

\begin{Declaration}
  \Macro{chapterlineswithprefixformat}%
    \Parameter{Ebene}\Parameter{Nummer}\Parameter{Text}
  \Macro{chapterlinesformat}%
    \Parameter{Ebene}\Parameter{Nummer}\Parameter{Text}
\end{Declaration}
Diese\ChangedAt{v3.19}{\Class{scrbook}\and \Class{scrreprt}} Anweisungen
werden von Überschriften des Stils \PValue{chapter}\textnote{Gliederungsstil
  \PValue{chapter}}\Index{Gliederung>Stil>~\PValue{chapter}} in Abhängigkeit
von Option \DescRef{maincls.option.chapterprefix}\IndexOption{chapterprefix}%
\textnote{\DescRef{maincls.option.chapterprefix}} (siehe
\autoref{sec:maincls.structure}, \DescPageRef{maincls.option.chapterprefix})
verwendet. Dabei gibt \Macro{chapterlineswithprefixformat} die Ausgabe bei
aktivierter Option vor. Bei deaktivierter Option
bestimmt dagegen \Macro{chapterlinesformat} die Ausgabe.

Die beiden Argumente \PName{Nummer} und \PName{Text} sind dabei
einschließlich Einstellung der Fonts für sich bereits fertig
formatiert. Letztlich regeln die Anweisungen also die Anordnung der beiden
Teile der Überschrift. Bei nicht nummerierten Überschriften ist \PName{Nummer}
ein komplett leeres Argument, enthält also auch keine
Formatierungsanweisungen.

Vordefiniert\textnote{Voreinstellung} sind die beiden Anweisungen mit:
\begin{lstcode}
  \newcommand{\chapterlinesformat}[3]{%
    \@hangfrom{#2}{#3}%
  }
  \newcommand{\chapterlineswithprefixformat}[3]{#2#3}
\end{lstcode}

\begin{Example}
  Sie wollen Überschriften im Kapitelstil gelb hinterlegen. Für
  Überschriften ohne Präfixzeile definieren Sie daher in der Präambel des
  Dokument:
\begin{lstcode}[moretexcs={colorbox}]
  \makeatletter
  \renewcommand{\chapterlinesformat}[3]{%
    \colorbox{yellow}{%
      \parbox{\dimexpr\linewidth
                      -2\fboxrule-2\fboxsep}{%
        \@hangfrom{#2}#3%
      }%
    }%
  }
  \makeatother
\end{lstcode}
  und für Überschriften mit Präfixzeile:
\begin{lstcode}[moretexcs={colorbox}]
  \renewcommand{\chapterlineswithprefixformat}[3]{%
    \colorbox{yellow}{%
      \parbox{\dimexpr\linewidth
                      -2\fboxrule-2\fboxsep}{%
        #2#3%
      }%
    }%
  }
\end{lstcode}
  Allerdings fällt Ihnen nach einiger Zeit auf, dass mit dieser Umdefinierung 
  die Überschriften wieder im Blocksatz gesetzt werden. Das liegt daran, dass
  \Macro{parbox} sein Argument so setzt. Um dies zu korrigieren, fügen Sie die
  Anweisung \DescRef{maincls.cmd.raggedchapter} (siehe
  \autoref{sec:maincls.structure}, \DescPageRef{maincls.cmd.raggedchapter}),
  die automatisch bereits vor \Macro{chapterlinesformat} und
  \Macro{chapterlineswithprefixformat} aufgerufen wird, in die Definitionen
  ein:
\begin{lstcode}[moretexcs={colorbox}]
  \makeatletter
  \renewcommand{\chapterlinesformat}[3]{%
    \colorbox{yellow}{%
      \parbox{\dimexpr\linewidth
                      -2\fboxrule-2\fboxsep}{%
        \raggedchapter
        \@hangfrom{#2}#3%
      }%
    }%
  }
  \makeatother
  \renewcommand{\chapterlineswithprefixformat}[3]{%
    \colorbox{yellow}{%
      \parbox{\dimexpr\linewidth
                      -2\fboxrule-2\fboxsep}{%
        \raggedchapter
        #2#3%
      }%
    }%
  }
\end{lstcode}
  Es sei daran erinnert, dass die Klammerung mit den Anweisungen
  \Macro{makeatletter} und \Macro{makeatother} nur in der Dokumentpräambel zu
  verwenden ist. In einer eigenen Wrapper-Klasse oder einem Paket haben sie zu
  entfallen. Sie werden auch nur wegen \Macro{@hangfrom} in der Definition von
  \Macro{chapterlinesformat} benötigt.%
\end{Example}

Wie im Beispiel gezeigt, ist der Anwender bei der Umdefinierung der
Anweisungen für einige Dinge selbst verantwortlich. Dazu gehört neben dem
Erhalt der voreingestellten Ausrichtung auch, dass innerhalb der Überschrift
kein Seitenumbruch, beispielsweise an zusätzlich eingefügten Absätzen oder
Abständen, erfolgen kann. Das gezeigte Beispiel ist diesbezüglich
unproblematisch. Nicht nur, dass die beiden Boxen ohnehin keinen Seitenumbruch
erlauben. \KOMAScript{} verändert außerdem \Macro{interlinepenalty} als Teil
von \PName{Text} so, dass darin kein Seitenumbruch stattfinden
darf. Gleichzeitig endet \PName{Text} immer mit einem internen Absatz,
\Macro{@@par}.

Die Anweisung \DescRef{maincls.cmd.raggedchapter} ist übrigens nicht
Bestandteil von \PName{Text}, weil anderenfalls beispielsweise die Verwendung
von \Macro{MakeUppercase} innerhalb der Umdefinierung der beiden Anweisungen
erheblich erschwert wäre. Dennoch sei darauf hingewiesen, dass typografischen
Regeln zufolge Versalsatz mit gängigen Fonts immer der Sperrung und des
Ausgleichs bedarf. Die \LaTeX-Anweisung \Macro{MakeUppercase} leistet dies
jedoch nicht.

Das erste Argument, \PName{Ebene}, wird von den Anweisungen in der
Voreinstellung nicht verwendet und ist auch im Beispiel nicht
erforderlich. Erst wenn der Anwender mehrere Anweisungen im Stil
\PValue{chapter} definiert und man innerhalb der Definition nach den
Anweisungen unterscheiden will, kann dies über \PName{Ebene} erfolgen. Dabei
ist \PName{Ebene} der vom Namen der Anweisung abgeleitete Name der
Gliederungsebene. Bei \DescRef{maincls.cmd.chapter},
\DescRef{maincls.cmd.chapter*}, \DescRef{maincls.cmd.addchap} und
\DescRef{maincls.cmd.addchap*} ist \PName{Ebene} daher einheitlich
\PValue{chapter}.%
\EndIndexGroup


\begin{Declaration}
  \Macro{sectionlinesformat}%
    \Parameter{Ebene}\Parameter{Einzug}\Parameter{Nummer}\Parameter{Text}
  \Macro{sectioncatchphraseformat}%
    \Parameter{Ebene}\Parameter{Einzug}\Parameter{Nummer}\Parameter{Text}
\end{Declaration}
Diese\ChangedAt{v3.19}{\Class{scrbook}\and \Class{scrreprt}\and
  \Class{scrartcl}} Anweisungen werden von Überschriften des Stils
\PValue{section}\textnote{Gliederungsstil
  \PValue{section}}\Index{Gliederung>Stil>~\PValue{section}} in Abhängigkeit
davon verwendet, ob die jeweilige Überschrift eine Spitzmarke erzeugt oder
eine frei stehende Überschrift. Frei stehende Überschriften werden dabei per
\Macro{sectionlinesformat} ausgegeben, während
\Macro{sectioncatchphraseformat} für Spitzmarken zuständig ist.

In beiden Fällen gibt \PName{Einzug} den Wert eines horizontalen Einzugs
der Überschrift gegenüber dem Textbereich an. Durch Angabe eines negativen
Wertes soll es auch möglich sein, die Überschrift in den linken Rand zu
rücken.

Die beiden Argumente \PName{Nummer} und \PName{Text} sind einschließlich
Einstellung der Fonts für sich bereits fertig formatiert. Letztlich regeln die
Anweisungen also die Anordnung der beiden Teile der Überschrift. Bei nicht
nummerierten Überschriften ist \PName{Nummer} ein komplett leeres Argument,
enthält also auch keine Formatierungsanweisungen.

Vordefiniert sind die beiden Anweisungen als:
\begin{lstcode}
\newcommand{\sectionlinesformat}[4]{%
  \@hangfrom{\hskip #2#3}{#4}%
}
\newcommand{\sectioncatchphraseformat}[4]{\hskip #2#3#4}
\end{lstcode}

Bei Umdefinierung einer der beiden Anweisungen ist wiederum der Anwender
selbst dafür verantwortlich, Seitenumbrüche innerhalb der Ausgabe zu
verhindern. \KOMAScript{} hilft hier lediglich mit entsprechend gesetztem
\Macro{interlinepenalty}.

\begin{Example}
  Wie schon im Beispiel der Kapitelüberschriften sollen nun die frei stehenden
  Überschriften der Ebene \PValue{section} mit einer Farbe hinterlegt
  werden. Die frei stehenden Überschriften tieferer Ebenen sollen nicht
  verändert werden:
\begin{lstcode}[moretexcs={colorbox}]
  \makeatletter
  \renewcommand{\sectionlinesformat}[4]{%
    \Ifstr{#1}{section}{%
      \hspace*{#2}%
      \colorbox{yellow}{%
        \parbox{\dimexpr\linewidth
                        -2\fboxrule-2\fboxsep-#2}{%
          \raggedsection
          \@hangfrom{#3}{#4}%
        }%
      }%
    }{%
      \@hangfrom{\hskip #2#3}{#4}%
    }%
  }
  \makeatother
\end{lstcode}
  Mit dem gezeigten Code wird im Falle eines Einzugs der Überschrift der
  Bereich des Einzugs nicht mit gefärbt. Wird hingegen die Überschrift in den
  linken Rand gestellt, so wird dieser Bereich des Randes ebenfalls farbig
  hinterlegt. Durch Verschiebung der \Macro{hspace*}-Anweisung in die
  \Macro{colorbox} kann dieses Verhalten verändert werden.%
  \iffalse Umbruchkorrekturtext
  
  Erneut sei daran erinnert, dass \Macro{makeatletter} und \Macro{makeatother}
  nur in der Dokumentpräambel zu verwenden sind. In einer eigenen
  Wrapper-Klasse oder einem Paket haben sie zu entfallen. Sie werden auch nur
  wegen \Macro{@hangfrom} in der Definition von \Macro{sectionlinesformat}
  benötigt.%
  \else%

  \Macro{makeatletter} und \Macro{makeatother} werden in der Dokumentpräambel
  erneut wegen \Macro{@hangfrom} benötigt.%
  \fi%
\end{Example}

Das erste Argument, \PName{Ebene}, wird von den Anweisungen in der
Voreinstellung nicht verwendet. Wie das Beispiel zeigt, kann es aber sehr gut
dazu verwendet werden, nach unterschiedlichen Gliederungsebenen im gemeinsamen
Stil \PValue{section} zu unterscheiden.%
\EndIndexGroup


\begin{Declaration}
\DoHook{heading/preinit/\PName{Name}}%
\DoHook{heading/postinit/\PName{Name}}%
\DoHook{heading/branch/star/\PName{Name}}%
\DoHook{heading/branch/nostar/\PName{Name}}%
\DoHook{heading/begingroup/\PName{Name}}%
\DoHook{heading/endgroup/\PName{Name}}%
\end{Declaration}
\BeginIndex{}{Haken}%
Abgesehen\ChangedAt[2019/01]{v3.27}{\Class{scrbook}\and \Class{scrreprt}\and
  \Class{scrartcl}} von den übrigen Einstellmöglichkeiten bieten sämtliche mit
\DescRef{\LabelBase.cmd.DeclareSectionCommand},
\DescRef{\LabelBase.cmd.DeclareNewSectionCommand} und
\DescRef{\LabelBase.cmd.ProvideSectionCommand} definierten und ebenso auch die
mit \DescRef{\LabelBase.cmd.RedeclareSectionCommand} umdefinierten
Gliederungsanweisungen eine ganze Reihe an Haken, die per
\DescRef{scrbase.cmd.AddtoDoHook} manipuliert werden können. Zur
Funktionsweise dieser \emph{do-hook} genannten Haken sei auf
\autoref{sec:scrbase.hooks} ab \autopageref{sec:scrbase.hooks} verwiesen. Der
letzte Teil des Spezifikators ist dabei der \PName{Name} der
Gliederungsanweisung, wie er auch bei den oben genannten Befehlen als letztes
Argument anzugeben ist.

Wichtig\important{Achtung!} ist, dass keiner dieser Haken für Anweisungen
verwendet werden sollte, die Auswirkungen auf den Seitenumbruch oder die
Positionierung der Überschriften haben. Daher sollten nur wirklich erfahrene
Anwender Gebrauch von diesen Haken machen. Im Zweifelsfall ist außerdem der
Quellcode der Klasse bezüglich des genauen Ausführungszeitpunktes des
jeweiligen Hakens zu konsultieren. Diese Haken dienen als letzter Notanker vor
einer tatsächlichen Umdefinierung von Gliederungsbefehlen außerhalb der von
\KOMAScript{} gebotenen Mittel.

Der\textnote{\DoHook{\dots/preinit/\dots}} Haken
\DoHook{heading/preinit/\PName{Name}} wird unmittelbar vor der
Ini"-tialisierung der Anweisungen ausgeführt. Zu diesem Zeitpunkt wurden noch
keine Einstellungen vorgenommen. Selbst der Absatz, der einer Überschrift
vorausgeht, ist noch nicht zwingend beendet.

Der\textnote{\DoHook{\dots/postinit/\dots}} Haken
\DoHook{heading/postinit/\PName{Name}} wird etwas später ausgeführt. Zu
diesem Zeitpunkt wurden einige Einstellungen bereits vorgenommen und auch ein
vorausgehender Absatz wurde bereits beendet.

Der Haken \textnote{\DoHook{\dots/branch/\dots}}
\DoHook{heading/branch/nostar/\PName{Name}} wird ausgeführt, sobald feststeht,
dass keine Sternform der Gliederungsanweisung vorliegt. Für die Sternform wird
hingegen \DoHook{heading/branch/star/\PName{Name}} ausgeführt. Zu diesem
Zeitpunkt sind gegebenenfalls auch bereits vertikale Abstände über der
jeweiligen Überschrift eingefügt.

Der\textnote{\DoHook{\dots/begingroup/\dots}} Haken
\DoHook{heading/begingroup/\PName{Name}} wird am Anfang der Gruppe
ausgeführt, in der schließlich die Überschrift gesetzt wird. Das ist also die
letzte Eingriffsmöglichkeit vor der Ausgabe der Überschrift.

Entsprechend\textnote{\DoHook{\dots/endgroup/\dots}} wird
\DoHook{heading/endgroup/\PName{Name}} vor dem Ende der Gruppe ausgeführt, in
der die Überschrift gesetzt wurde. Derzeit ist das auch der letzte Haken
innerhalb einer Gliederungsanweisung.

Zu beachten\textnote{\DescRef{maincls.cmd.minisec}} ist, dass
\DescRef{maincls.cmd.minisec} keine echte Gliederungsanweisung ist und die
genannten Haken daher auf diese Anweisung keine Anwendung finden.%
\EndIndexGroup


\begin{Declaration}
  \Macro{IfUseNumber}\Parameter{Dann-Code}\Parameter{Sonst-Code}
\end{Declaration}
Streng\ChangedAt{v3.27}{\Class{scrbook}\and \Class{scrreprt}\and
  \Class{scrartcl}} genommen handelt es sich hier um eine interne
Anweisung. Sie ist nur innerhalb von Überschriften zwischen den beiden Haken
\DoHook{\dots/begingroup/\dots} und \DoHook{\dots/endgroup/\dots}
spezifiziert. In diesem Fall wird der \PName{Dann-Code} ausgeführt, wenn die
aktuelle Überschrift gemäß Einstellung von
\DescRef{maincls.counter.secnumdepth} und aufgrund der Nichtverwendung einer
Sternform nummeriert werden soll. Soll die Überschrift aufgrund der Verwendung
einer Sternform oder der Einstellung von \DescRef{maincls.counter.secnumdepth}
nicht nummeriert werden, so wird der \PName{Sonst-Code} ausgeführt. Bei
Überschriften im Stil \PValue{chapter} wird für die Entscheidung außerdem
berücksichtigt, ob die Überschrift im Hauptteil gesetzt wird.

Wird die Anweisung außerhalb von
Überschriften verwendet, so ist ihr Ergebnis nicht spezifiziert. In der Regel
gibt sie dann eine Fehlermeldung aus und führt weder den \PName{Dann-Code}
noch den \PName{Sonst-Code} aus.%
\EndIndexGroup


\begin{Declaration}
  \Macro{SecDef}\Parameter{Sternanweisung}\Parameter{Normalanweisung}
  \Macro{scr@startsection}\Parameter{Name}\Parameter{Ebene}%
  \Parameter{Einzug}\Parameter{Abstand~davor}\Parameter{Abstand~danach}
                    \Parameter{Stilanweisungen}%
                    \OParameter{Kurzform}\Parameter{Überschrift}
  \labelsuffix[*]%
  \Macro{scr@startsection}\Parameter{Name}\Parameter{Ebene}%
                          \Parameter{Einzug}\Parameter{Abstand~davor}
                          \Parameter{Abstand~danach}
                          \Parameter{Stilanweisungen}%
                          \nobreak\texttt{*}\Parameter{Überschrift}
  \labelsuffix
\end{Declaration}
Wie\ChangedAt{v3.15}{\Class{scrbook}\and \Class{scrreprt}\and
  \Class{scrartcl}} bereits in \autoref{sec:maincls.structure} bei der
Beschreibung zu den Gliederungsbefehlen ab
\DescPageRef{maincls.cmd.chapter} erklärt, verfügt \KOMAScript{}
bezüglich des optionalen Arguments der Gliederungsbefehle über erweiterte
Möglichkeiten. Um dies zu erreichen, war es notwendig, einige Anweisungen
des \LaTeX-Kerns zu ersetzen:
\begin{itemize}
\item Statt
  \Macro{@startsection}\IndexCmd{@startsection}\important{\Macro{@startsection}}
  wird \Macro{scr@startsection} verwendet. Die Definition
  von \Macro{@startsection} wird jedoch geprüft. Entspricht diese beim Laden
  der Klasse nicht den Erwartungen, so wird eine Warnung
  ausgegeben, diverse
  Möglichkeiten von \KOMAScript{} werden deaktiviert und
  \Macro{scr@startsection} stützt sich auf eine Kopie von
  \Macro{@startsection}, während \Macro{@startsection} selbst
  \Macro{scr@startsection} aufruft.
\item Statt \Macro{@dblarg}\IndexCmd{@dblarg}\important{\Macro{@dblarg}} wird
  von den Gliederungsbefehlen eine eigene, interne Anweisung
  verwendet.
\item Statt \Macro{secdef}\IndexCmd{secdef}\important{\Macro{secdef}} wird
  \Macro{SecDef} verwendet, um die erwähnte Änderung
  bezüglich \Macro{@dblarg} zu erreichen. Sollte die Definition von
  \Macro{secdef} nicht den Erwartungen entsprechen, wird eine Warnung
  ausgegeben.
\item \Macro{@sect}\IndexCmd{@sect}\important{\Macro{@sect}} wird umdefiniert,
  um diverse Erweiterungen durch \KOMAScript{} zu realisieren.
\item \Macro{@ssect}\IndexCmd{@ssect}\important{\Macro{@ssect}} wird
  umdefiniert, um diverse Erweiterungen von \KOMAScript{} zu realisieren.
\item \Macro{@xsect}\IndexCmd{@xsect}\important{\Macro{@xsect}} wird
  umdefiniert, um diverse Erweiterungen von \KOMAScript{} zu realisieren.
\end{itemize}
Es gibt Überlegungen, in künftigen Versionen von \KOMAScript{} die erwähnten
Anweisungen aus dem \LaTeX-Kern nicht mehr anzutasten, sondern intern komplett
durch Eigenentwicklungen zu ersetzen. Bei Verwendung eigentlich inkompatibler
Pakete würden dadurch automatisch Erweiterungen von \KOMAScript{} deaktiviert
und die Gliederungsbefehle in die Hände jener Pakete gelegt. Gleichzeitig
müssten aber zum Erhalt der Kompatibilität mit anderen Paketen zusätzliche
Maßnahmen ergriffen werden.

Die erwähnten Ersatzanweisungen können von Paketautoren genau wie die
\LaTeX-Kern-Anweisungen verwendet werden, bieten dann aber automatisch die
erweiterte Funktionalität von \KOMAScript{}. Jedoch sollten sie nicht
umdefiniert werden, da sie sich jederzeit ändern können und dann die
Funktionalität von \KOMAScript{} durch diese Umdefinierung erneut
beeinträchtigt werden könnte. Die Bedeutung der Parameter ist der
Anleitung zum \LaTeX-Kern \cite{latex:source2e} zu entnehmen.
Als\ChangedAt[2019/01]{v3.27}{\Class{scrbook}\and \Class{scrreprt}\and
  \Class{scrartcl}} Ersatz für die Umdefinierung von Anweisungen bietet
\KOMAScript{} die zuvor dokumentierten Haken.%
\EndIndexGroup

\begin{Explain}
  \begin{Declaration}
    \Macro{At@startsection}\Parameter{Code}
    \Macro{Before@ssect}\Parameter{Code}
    \Macro{Before@sect}\Parameter{Code}
  \end{Declaration}
  Bis\ChangedAt[2019/01]{v3.27}{\Class{scrbook}\and \Class{scrreprt}\and
    \Class{scrartcl}} \KOMAScript{}~3.26b dienten diese Anweisungen ebenfalls
  als Ersatz für die Umdefinierung von
  \DescRef{\LabelBase.cmd.scr@startsection} und
  \DescRef{\LabelBase.cmd.SecDef}. Seit\textnote{veraltete
    \KOMAScript-Anweisungen} \KOMAScript~3.27 gelten sie jedoch als veraltet.

  Intern wird \Macro{At@startsection} seit \KOMAScript~3.27 über den Haken
  \DescRef{\LabelBase.dohook.heading/postinit}
  realisiert. \Macro{Before@ssect} ist mit Hilfe von
  \DescRef{\LabelBase.dohook.heading/branch/star} implementiert und
  \Macro{Before@sect} über \DescRef{\LabelBase.dohook.heading/branch/nostar}.
  \PName{Code} wird den Haken per \DescRef{scrbase.cmd.AddtoDoHook}
  hinzugefügt. Es ist nicht vorgesehen, einmal eingefügten \PName{Code} wieder
  zu entfernen.%
  \EndIndexGroup
\end{Explain}

% \section{Anhang}
% \seclabel{appendix}

\begin{Declaration}
  \Macro{appendixmore}
\end{Declaration}%
Bei den {\KOMAScript}-Klassen gibt es innerhalb der Anweisung
\DescRef{maincls.cmd.appendix}%
\IndexCmd{appendix}\important{\DescRef{maincls.cmd.appendix}} eine
Besonderheit. Ist \Macro{appendixmore} definiert, so wird diese Anweisung von
\DescRef{maincls.cmd.appendix} ebenfalls ausgeführt. Intern wird das von den
{\KOMAScript}-Klassen \Class{scrbook}\OnlyAt{\Class{scrbook}\and
  \Class{scrreprt}} und \Class{scrreprt} für die Realisierung der Layoutoption
\DescRef{maincls.option.appendixprefix} genutzt (siehe
\autoref{sec:maincls.structure},
\DescPageRef{maincls.option.appendixprefix}). Dies\textnote{Achtung!}  sollten
Sie unbedingt beachten, falls Sie selbst das Makro \Macro{appendixmore}
definieren oder umdefinieren wollen. Wurde diese Option bereits verwendet, so
erhalten Sie bei \verb|\newcommand{\appendixmore}{|\dots\verb|}| eine
Fehlermeldung.  Dadurch wird verhindert, dass Sie die Option außer
Kraft setzen, ohne es zu merken.

\begin{Example}
  Sie wollen nicht, dass bei Verwendung der Klasse \Class{scrbook} oder
  \Class{scrreprt} im Hauptteil die Kapitel mit einer Präfixzeile versehen
  werden (siehe Layoutoption \DescRef{maincls.option.chapterprefix}
  in \autoref{sec:maincls.structure},
  \DescPageRef{maincls.option.chapterprefix}).  Damit die Konsistenz
  gewahrt bleibt, wollen Sie auch nicht, dass eine solche Zeile im Anhang
  verwendet wird. Stattdessen soll in den Anhängen direkt vor dem
  Kapitelbuchstaben das Wort »Anhang« in der jeweiligen Sprache stehen. Dies
  soll auch für die Kolumnentitel gelten. Also verwenden Sie nicht die
  Layoutoption \DescRef{maincls.option.appendixprefix}, sondern
  definieren in der Dokumentpräambel:
\begin{lstcode}
  \newcommand*{\appendixmore}{%
    \renewcommand*{\chapterformat}{%
      \appendixname~\thechapter\autodot\enskip}%
    \renewcommand*{\chaptermarkformat}{%
      \appendixname~\thechapter\autodot\enskip}}
\end{lstcode}
  Sollten Sie doch noch entscheiden, dass Sie die Option
  \DescRef{maincls.option.appendixprefix} bei
  \DescRef{maincls.cmd.documentclass} setzen wollen, so erhalten Sie aufgrund
  der dann bereits definierten Anweisung \Macro{appendixmore} eine
  Fehlermeldung. Damit wird verhindert, dass obige Definition unbemerkt die
  Einstellungen überschreibt, die Sie per Option getroffen haben.

  Wenn Sie ein vergleichbares Verhalten des Anhangs für die Klasse
  \Class{scrartcl} erreichen wollen, so ist dies ebenfalls möglich. Schreiben
  Sie dazu beispielsweise Folgendes in die  Präambel Ihres Dokuments:
\begin{lstcode}
  \newcommand*{\appendixmore}{%
    \renewcommand*{\sectionformat}{%
      \appendixname~\thesection\autodot\enskip}%
    \renewcommand*{\sectionmarkformat}{%
      \appendixname~\thesection\autodot\enskip}}
\end{lstcode}

  Die Erklärungen zu den in diesem Beispiel umdefinierten Anweisungen finden
  Sie in \autoref{sec:maincls.structure},
  \DescPageRef{maincls.cmd.chapterformat} und
  \DescPageRef{maincls.cmd.chaptermarkformat}.
\end{Example}%
%
\EndIndexGroup
\ExampleEndFix


\section{Literaturverzeichnis}
\seclabel{bibliography}

Die Erklärungen in diesem Abschnitt verlieren mit der Verwendung von Paketen
wie \Package{biblatex} zunehmend an Bedeutung. Die weitreichenden
Möglichkeiten solcher Pakete ersetzen dann die hier beschriebenen
Erweiterungen der \KOMAScript-Klassen.

\begin{Declaration}
  \Macro{newbibstyle}\OParameter{Elternstil}\Parameter{Name}%
                     \Parameter{Anweisungen}
  \Macro{newblock}
  \Macro{@openbib@code}
  \Macro{bib@beginhook}
  \Macro{bib@endhook}
\end{Declaration}
Schon die Standardklassen\textnote{Standardklassen} kennen zur Unterteilung
der Einträge in das Literaturverzeichnis die Anweisung \Macro{newblock}. Was
diese Anweisung genau macht, hängt dabei von den Klassenoptionen ab. Wird die
Option \Option{openbib}\important{\Option{openbib}} verwendet, so werden am
Ende der Standardklasse die Anweisungen \Macro{@openbib@code} und
\Macro{newblock} selbst umdefiniert. Von den Standardklassen wird die
Anweisung \Macro{@openbib@code} beim Start der Liste für das
Literaturverzeichnis --~genauer: bei der Festlegung der Parameter für diese
Liste~-- ausgeführt. Es darf davon ausgegangen werden, dass auch viele Pakete,
die das Literaturverzeichnis umdefinieren, diese Anweisung entsprechend
abarbeiten.

Bei den \KOMAScript-Klassen\textnote{\KOMAScript} geschieht etwas
ähnliches. Allerdings wird \Macro{@openbib@code} nicht am Ende der Klasse
umdefiniert. Stattdessen wird mit \Macro{newbibstyle} der Stil
\PValue{openstyle} für das Literaturverzeichnis definiert. Die
\PName{Anweisungen}, die dabei in der Implementierung angegeben wurden,
beinhalten die gewünschten Umdefinierung von \Macro{@openbib@code} und von
\Macro{newblock}. Wird nun mit Hilfe der Option
\OptionValueRef{maincls}{bibliography}{openstyle}%
\IndexOption{bibliography~=\textKValue{openstyle}}%
\important{\OptionValueRef{maincls}{bibliography}{openstyle}} dieser
Literaturverzeichnisstil gewählt, so werden die \PName{Anweisungen}
unmittelbar ausgeführt, also \Macro{@openbib@code} und \Macro{newblock}
umdefiniert.

Neben \Macro{@openbib@code} und \Macro{newblock} können in \PName{Anweisungen}
auch noch \Macro{bib@beginhook} und \Macro{bib@endhook} umdefiniert
werden. Die Anweisung \Macro{bib@beginhook} wird unmittelbar nach der
Überschrift und der Präambel des Literaturverzeichnisses, aber noch vor der
Liste mit den Literatureinträgen ausgeführt. Die Anweisung \Macro{bib@endhook}
wird direkt nach dieser Liste am Ende des Literaturverzeichnisses
ausgeführt. Im Falle eines mit \DescRef{maincls.cmd.BreakBibliography} (siehe
\autoref{sec:maincls.bibliography},
\DescPageRef{maincls.cmd.BreakBibliography}) unterbrochenen
Literaturverzeichnisses werden diese Anweisungen außerdem am Anfang und Ende
jedes Teils, also unmittelbar vor und nach
\DescRef{maincls.cmd.BreakBibliography} ausgeführt.

Die Anweisungen \Macro{newblock}, \Macro{@openbib@code}, \Macro{bib@beginhook}
und \Macro{bib@endhook} werden bei der Verwendung eines neuen
Literaturverzeichnisstils zunächst als leer definiert. Danach werden die
\PName{Anweisungen} des bei der Definition des Stils optional angegebenen
Elternstils ausgeführt und dann erst die \PName{Anweisungen}, die bei
Definition des neuen Stils angegeben wurden. Daraus ergibt sich auch, %
\iffalse % Umbruchkorrektur
dass jede der vier Anweisungen innerhalb von \PName{Anweisung} bei Bedarf
keinesfalls mit \Macro{newcommand}, sondern mit
\Macro{renewcommand}\IndexCmd{renewcommand}\important{\Macro{renewcommand}}
definiert werden sollte.%
\else%
\enlargethispage{3pt}% Umbruchkorrektur (im Paar)
dass die vier Anweisungen innerhalb von \PName{Anweisung} bei Bedarf mit
\Macro{renewcommand}\IndexCmd{renewcommand}\important{\Macro{renewcommand}}
umzudefinieren sind.%
\par\enlargethispage{3pt}% Umbruchkorrektur (im Paar)
\fi%

Setzt der Anwender mit den Anweisungen
\DescRef{maincls.cmd.AtEndBibliography}%\important[i]{\DescRef{maincls.cmd.AtEndBibliography}\\
%  \DescRef{maincls.cmd.AfterBibliographyPreamble}}% Passt blöder Weise nicht!
\IndexCmd{AtEndBibliography} und
\DescRef{maincls.cmd.AfterBibliographyPreamble}%
\IndexCmd{AfterBibliographyPreamble} weitere \PName{Anweisungen} für die
Ausführung nach der Präambel und am Ende des Literaturverzeichnisses, so
werden die mit \DescRef{maincls.cmd.AfterBibliographyPreamble} festgelegten
\PName{Anweisungen} einmalig am Anfang des Literaturverzeichnisses nach
\Macro{bib@beginhook} und die mit \DescRef{maincls.cmd.AtEndBibliography}
festgelegten \PName{Anweisungen} einmalig am Ende des Literaturverzeichnisses
vor \Macro{bib@endhook} ausgeführt.

Mit\textnote{Beispiel!} Hilfe des Pakets
\Package{multicol}\IndexPackage{multicol}\important{\Package{multicol}} (siehe
\cite{package:multicol}) könnte man beispielsweise einen Literaturstil für ein
zweispaltiges Literaturverzeichnis definieren:
% Umbruchkorrektur: listings korrigiert
\begin{lstcode}
  \newbibstyle{twocolumstyle}{%
    \renewcommand*{\bib@beginhook}{\begin{multicols}{2}}%
    \renewcommand*{\bib@endhook}{\end{multicols}}}%
\end{lstcode}
Soll es außerdem eine \emph{open}-Variante davon geben, kann man hier die
Möglichkeiten der Vererbung verwenden und bei der Definition einen Elternstil
mit angeben:
% Umbruchkorrektur: listings korrigiert
\begin{lstcode}
  \newbibstyle[openstyle]{twocolumopenstyle}{%
    \renewcommand*{\bib@beginhook}{\begin{multicols}{2}}%
    \renewcommand*{\bib@endhook}{\end{multicols}}}%
\end{lstcode}
Die Auswahl eines dieser neuen Stile erfolgt dann einfach wieder über die
Option \DescRef{maincls.option.bibliography}.

Wie\textnote{Achtung!} schon \DescRef{maincls.cmd.BreakBibliography} verlieren
auch diese Anweisungen ganz oder teilweise ihre Wirkung, wenn
\Environment{thebibliography} beispielsweise durch Verwendung von
\Package{biblatex}\IndexPackage{biblatex}\important{\Package{biblatex}}
umdefiniert wird.%
\EndIndexGroup


\section{Mehr oder weniger obsolete Optionen und Anweisungen}
\seclabel{obsolete}

\LoadNonFree{scrbookreportarticle-experts}{4}
%
\EndIndexGroup

\endinput

%%% Local Variables:
%%% mode: latex
%%% mode: flyspell
%%% coding: utf-8
%%% ispell-local-dictionary: "de_DE"
%%% TeX-master: "guide.tex"
%%% End:

% LocalWords:  Sternvarianten Kapitelüberschrift Standardklassen Paketautoren
% LocalWords:  Klassenoptionen Gliederungsbefehlen Namensliste umzudefinieren
% LocalWords:  Inhaltsverzeichniseintrags Abstandsanweisungen Sternformen
% LocalWords:  Teilüberschriften Benutzerfreundlichkeit Absatzmarkierung
% LocalWords:  umkonfiguriert Teileüberschrift Abschnittsüberschrift
% LocalWords:  Absolutwert Gliederungsbefehls Präfixzeile Schrifteinstellung
% LocalWords:  Zählername Spitzmarke Kolumnentitels Schrifteinstellungen
% LocalWords:  Strichstärke expandierbar expandierbaren expandierbares
% LocalWords:  spartanisch Kapitelstil Spitzmarken Formatierungsanweisungen
% LocalWords:  Kapitelüberschriften Gliederungsebenen Verzeichniseinträge
% LocalWords:  Verzeichniseintragsstile Gliederungsbefehl Verzeichniseintrag
% LocalWords:  Verzeichniseintrags Verzeichniseintragsstilen Gliederungsstil
% LocalWords:  Gliederungsebene Eintragsebenen Absatzeinzug Dokumentpräambel
