% \iffalse meta-comment
%
% File: settobox.dtx
% Version: 2016/05/16 v1.5
% Info: Assign box dimensions to length registers
%
% Copyright (C)
%    2000, 2006-2008 Heiko Oberdiek
%    2016-2019 Oberdiek Package Support Group
%    https://github.com/ho-tex/oberdiek/issues
%
% This work may be distributed and/or modified under the
% conditions of the LaTeX Project Public License, either
% version 1.3c of this license or (at your option) any later
% version. This version of this license is in
%    https://www.latex-project.org/lppl/lppl-1-3c.txt
% and the latest version of this license is in
%    https://www.latex-project.org/lppl.txt
% and version 1.3 or later is part of all distributions of
% LaTeX version 2005/12/01 or later.
%
% This work has the LPPL maintenance status "maintained".
%
% The Current Maintainers of this work are
% Heiko Oberdiek and the Oberdiek Package Support Group
% https://github.com/ho-tex/oberdiek/issues
%
% This work consists of the main source file settobox.dtx
% and the derived files
%    settobox.sty, settobox.pdf, settobox.ins, settobox.drv,
%    settobox-example.tex.
%
% Distribution:
%    CTAN:macros/latex/contrib/oberdiek/settobox.dtx
%    CTAN:macros/latex/contrib/oberdiek/settobox.pdf
%
% Unpacking:
%    (a) If settobox.ins is present:
%           tex settobox.ins
%    (b) Without settobox.ins:
%           tex settobox.dtx
%    (c) If you insist on using LaTeX
%           latex \let\install=y% \iffalse meta-comment
%
% File: settobox.dtx
% Version: 2016/05/16 v1.5
% Info: Assign box dimensions to length registers
%
% Copyright (C)
%    2000, 2006-2008 Heiko Oberdiek
%    2016-2019 Oberdiek Package Support Group
%    https://github.com/ho-tex/oberdiek/issues
%
% This work may be distributed and/or modified under the
% conditions of the LaTeX Project Public License, either
% version 1.3c of this license or (at your option) any later
% version. This version of this license is in
%    https://www.latex-project.org/lppl/lppl-1-3c.txt
% and the latest version of this license is in
%    https://www.latex-project.org/lppl.txt
% and version 1.3 or later is part of all distributions of
% LaTeX version 2005/12/01 or later.
%
% This work has the LPPL maintenance status "maintained".
%
% The Current Maintainers of this work are
% Heiko Oberdiek and the Oberdiek Package Support Group
% https://github.com/ho-tex/oberdiek/issues
%
% This work consists of the main source file settobox.dtx
% and the derived files
%    settobox.sty, settobox.pdf, settobox.ins, settobox.drv,
%    settobox-example.tex.
%
% Distribution:
%    CTAN:macros/latex/contrib/oberdiek/settobox.dtx
%    CTAN:macros/latex/contrib/oberdiek/settobox.pdf
%
% Unpacking:
%    (a) If settobox.ins is present:
%           tex settobox.ins
%    (b) Without settobox.ins:
%           tex settobox.dtx
%    (c) If you insist on using LaTeX
%           latex \let\install=y% \iffalse meta-comment
%
% File: settobox.dtx
% Version: 2016/05/16 v1.5
% Info: Assign box dimensions to length registers
%
% Copyright (C)
%    2000, 2006-2008 Heiko Oberdiek
%    2016-2019 Oberdiek Package Support Group
%    https://github.com/ho-tex/oberdiek/issues
%
% This work may be distributed and/or modified under the
% conditions of the LaTeX Project Public License, either
% version 1.3c of this license or (at your option) any later
% version. This version of this license is in
%    https://www.latex-project.org/lppl/lppl-1-3c.txt
% and the latest version of this license is in
%    https://www.latex-project.org/lppl.txt
% and version 1.3 or later is part of all distributions of
% LaTeX version 2005/12/01 or later.
%
% This work has the LPPL maintenance status "maintained".
%
% The Current Maintainers of this work are
% Heiko Oberdiek and the Oberdiek Package Support Group
% https://github.com/ho-tex/oberdiek/issues
%
% This work consists of the main source file settobox.dtx
% and the derived files
%    settobox.sty, settobox.pdf, settobox.ins, settobox.drv,
%    settobox-example.tex.
%
% Distribution:
%    CTAN:macros/latex/contrib/oberdiek/settobox.dtx
%    CTAN:macros/latex/contrib/oberdiek/settobox.pdf
%
% Unpacking:
%    (a) If settobox.ins is present:
%           tex settobox.ins
%    (b) Without settobox.ins:
%           tex settobox.dtx
%    (c) If you insist on using LaTeX
%           latex \let\install=y% \iffalse meta-comment
%
% File: settobox.dtx
% Version: 2016/05/16 v1.5
% Info: Assign box dimensions to length registers
%
% Copyright (C)
%    2000, 2006-2008 Heiko Oberdiek
%    2016-2019 Oberdiek Package Support Group
%    https://github.com/ho-tex/oberdiek/issues
%
% This work may be distributed and/or modified under the
% conditions of the LaTeX Project Public License, either
% version 1.3c of this license or (at your option) any later
% version. This version of this license is in
%    https://www.latex-project.org/lppl/lppl-1-3c.txt
% and the latest version of this license is in
%    https://www.latex-project.org/lppl.txt
% and version 1.3 or later is part of all distributions of
% LaTeX version 2005/12/01 or later.
%
% This work has the LPPL maintenance status "maintained".
%
% The Current Maintainers of this work are
% Heiko Oberdiek and the Oberdiek Package Support Group
% https://github.com/ho-tex/oberdiek/issues
%
% This work consists of the main source file settobox.dtx
% and the derived files
%    settobox.sty, settobox.pdf, settobox.ins, settobox.drv,
%    settobox-example.tex.
%
% Distribution:
%    CTAN:macros/latex/contrib/oberdiek/settobox.dtx
%    CTAN:macros/latex/contrib/oberdiek/settobox.pdf
%
% Unpacking:
%    (a) If settobox.ins is present:
%           tex settobox.ins
%    (b) Without settobox.ins:
%           tex settobox.dtx
%    (c) If you insist on using LaTeX
%           latex \let\install=y\input{settobox.dtx}
%        (quote the arguments according to the demands of your shell)
%
% Documentation:
%    (a) If settobox.drv is present:
%           latex settobox.drv
%    (b) Without settobox.drv:
%           latex settobox.dtx; ...
%    The class ltxdoc loads the configuration file ltxdoc.cfg
%    if available. Here you can specify further options, e.g.
%    use A4 as paper format:
%       \PassOptionsToClass{a4paper}{article}
%
%    Programm calls to get the documentation (example):
%       pdflatex settobox.dtx
%       makeindex -s gind.ist settobox.idx
%       pdflatex settobox.dtx
%       makeindex -s gind.ist settobox.idx
%       pdflatex settobox.dtx
%
% Installation:
%    TDS:tex/latex/oberdiek/settobox.sty
%    TDS:doc/latex/oberdiek/settobox.pdf
%    TDS:doc/latex/oberdiek/settobox-example.tex
%    TDS:source/latex/oberdiek/settobox.dtx
%
%<*ignore>
\begingroup
  \catcode123=1 %
  \catcode125=2 %
  \def\x{LaTeX2e}%
\expandafter\endgroup
\ifcase 0\ifx\install y1\fi\expandafter
         \ifx\csname processbatchFile\endcsname\relax\else1\fi
         \ifx\fmtname\x\else 1\fi\relax
\else\csname fi\endcsname
%</ignore>
%<*install>
\input docstrip.tex
\Msg{************************************************************************}
\Msg{* Installation}
\Msg{* Package: settobox 2016/05/16 v1.5 Assign box dimensions to length registers (HO)}
\Msg{************************************************************************}

\keepsilent
\askforoverwritefalse

\let\MetaPrefix\relax
\preamble

This is a generated file.

Project: settobox
Version: 2016/05/16 v1.5

Copyright (C)
   2000, 2006-2008 Heiko Oberdiek
   2016-2019 Oberdiek Package Support Group

This work may be distributed and/or modified under the
conditions of the LaTeX Project Public License, either
version 1.3c of this license or (at your option) any later
version. This version of this license is in
   https://www.latex-project.org/lppl/lppl-1-3c.txt
and the latest version of this license is in
   https://www.latex-project.org/lppl.txt
and version 1.3 or later is part of all distributions of
LaTeX version 2005/12/01 or later.

This work has the LPPL maintenance status "maintained".

The Current Maintainers of this work are
Heiko Oberdiek and the Oberdiek Package Support Group
https://github.com/ho-tex/oberdiek/issues


This work consists of the main source file settobox.dtx
and the derived files
   settobox.sty, settobox.pdf, settobox.ins, settobox.drv,
   settobox-example.tex.

\endpreamble
\let\MetaPrefix\DoubleperCent

\generate{%
  \file{settobox.ins}{\from{settobox.dtx}{install}}%
  \file{settobox.drv}{\from{settobox.dtx}{driver}}%
  \usedir{tex/latex/oberdiek}%
  \file{settobox.sty}{\from{settobox.dtx}{package}}%
  \usedir{doc/latex/oberdiek}%
  \file{settobox-example.tex}{\from{settobox.dtx}{example}}%
}

\catcode32=13\relax% active space
\let =\space%
\Msg{************************************************************************}
\Msg{*}
\Msg{* To finish the installation you have to move the following}
\Msg{* file into a directory searched by TeX:}
\Msg{*}
\Msg{*     settobox.sty}
\Msg{*}
\Msg{* To produce the documentation run the file `settobox.drv'}
\Msg{* through LaTeX.}
\Msg{*}
\Msg{* Happy TeXing!}
\Msg{*}
\Msg{************************************************************************}

\endbatchfile
%</install>
%<*ignore>
\fi
%</ignore>
%<*driver>
\NeedsTeXFormat{LaTeX2e}
\ProvidesFile{settobox.drv}%
  [2016/05/16 v1.5 Assign box dimensions to length registers (HO)]%
\documentclass{ltxdoc}
\usepackage{holtxdoc}[2011/11/22]
\usepackage{calc}
\usepackage{settobox}
\begin{document}
  \DocInput{settobox.dtx}%
\end{document}
%</driver>
% \fi
%
%
%
% \GetFileInfo{settobox.drv}
%
% \title{The \xpackage{settobox} package}
% \date{2016/05/16 v1.5}
% \author{Heiko Oberdiek\thanks
% {Please report any issues at \url{https://github.com/ho-tex/oberdiek/issues}}}
%
% \maketitle
%
% \begin{abstract}
% Commands are defined for getting box sizes similar
% to \LaTeX's \cs{settowidth} commands.
% \end{abstract}
%
% \tableofcontents
%
% \section{Usage}
%
% \subsection{Get box dimensions}
%
% \begin{declcs}^^A
%   {settoboxwidth}\,\M{\LaTeX\ length}\,\M{\LaTeX\ box}\\
%   \SpecialUsageIndex{\settoboxheight}^^A
%   \cs{settoboxheight}\,\M{\LaTeX\ length}\,\M{\LaTeX\ box}\\
%   \SpecialUsageIndex{\settoboxdepth}^^A
%   \cs{settoboxdepth}\,\M{\LaTeX\ length}\,\M{\LaTeX\ box}\\
%   \SpecialUsageIndex{\settoboxtotalheight}^^A
%   \cs{settoboxtotalheight}\,\M{\LaTeX\ length}\,\M{\LaTeX\ box}
% \end{declcs}
% A \meta{\LaTeX\ box} is allocated by \cs{newsavebox}.
% It can be filled by \cs{sbox} or the environment \texttt{lrbox}.
% The commands above extract then the desired lengths.
%
% \subsection{Set box dimensions}
%
% \begin{declcs}^^A
%   {setboxwidth}\,\M{\LaTeX\ box}\,\M{\LaTeX\ length expression}\\
%   \SpecialUsageIndex{\setboxheight}^^A
%   \cs{setboxheight}\,\M{\LaTeX\ box}\,\M{\LaTeX\ length expression}\\
%   \SpecialUsageIndex{\setboxdepth}^^A
%   \cs{setboxdepth}\,\M{\LaTeX\ box}\,\M{\LaTeX\ length expression}
% \end{declcs}
% These commands allow the manipulation of the box. Package \xpackage{calc}
% is supported in the \meta{\LaTeX\ length expression}.
% Also the following length are available in this expression:
% \begin{quote}
% \begin{tabular}{@{}ll@{}}
%   \cs{width}& width of the box\\
%   \cs{height}& height of the box\\
%   \cs{depth}& depth of the box\\
%   \cs{totalheight}& totalheight of the box\\
% \end{tabular}
% \end{quote}
% Note, the base point (point at the left margin of the baseline)
% always remain constant.
%
% \subsection{Move box}
%
% \begin{declcs}^^A
%   {setboxmoveleft}\,\M{\LaTeX\ box}\,\M{\LaTeX\ length expression}\\
%   \SpecialUsageIndex{\setboxmoveright}^^A
%   \cs{setboxmoveright}\,\M{\LaTeX\ box}\,\M{\LaTeX\ length expression}\\
%   \SpecialUsageIndex{\setboxlower}^^A
%   \cs{setboxlower}\,\M{\LaTeX\ box}\,\M{\LaTeX\ length expression}\\
%   \SpecialUsageIndex{\setboxright}^^A
%   \cs{setboxright}\,\M{\LaTeX\ box}\,\M{\LaTeX\ length expression}
% \end{declcs}
% Note, the box is shifted relative to the base point. The base point
% is always inside the box, however the width and height of the
% box change along with the movement.
%
% \subsection{Example}
%
% \subsubsection{Short example}
%
% \begin{quote}
%\begin{verbatim}
%\newsavebox{\mybox}
%\newlength{\mylength}
%\sbox{\mybox}{Hello World}
%\settoboxwidth{\mylength}{\mybox}
%\end{verbatim}
% \end{quote}
%
% \subsubsection{Test file that shows box manipulations}
%
%    \begin{macrocode}
%<*example>
%<<END
\documentclass{article}

\usepackage{settobox}
\usepackage{calc}

\newsavebox{\mybox}

\setlength{\fboxsep}{0pt}
\setlength{\parindent}{20pt}
\setlength{\parskip}{10pt}
\pagestyle{empty}

% \test{#1}
% The macro is called with commands in #1 that manipulates
% the box \mybox. These commands along with the result of
% the manipulation is shown. Thus the essence of the
% macro is:
%
%   a) \sbox{\mybox}{The cracy fox.}
%   b) #1 % manipulates \mybox
%   c) Print #1 commands.
%   d) Print box with frame
%
% The implemenation looks more weird:
\makeatletter
\newcommand*{\test}[1]{%
  \par
  \begingroup
    \raggedright
    \edef\x{\detokenize{#1}}%
    \let\do\@makeother
    \dospecials
    \catcode`\~\active
    \catcode`\ =10\relax
    \def~{\\}%
    \noindent
    \texttt{\scantokens\expandafter{\x}}%
    \par
  \endgroup
  \begingroup
    \let~\relax
    \sbox{\mybox}{The cracy fox.}%
     #1%
     A---\fbox{\usebox\mybox}---B%
  \endgroup
  \par
}
\makeatother

\begin{document}

\test{\setboxwidth{\mybox}{1.25\width}}
\test{\setboxheight{\mybox}{0pt}}
\test{\setboxheight{\mybox}{2\height}}
\test{\setboxdepth{\mybox}{\height}}
\test{\setboxmoveleft{\mybox}{5pt}}
\test{%
  \setboxmoveleft{\mybox}{5pt}~%
  \setboxwidth{\mybox}{\width + 5pt}%
}
\test{\setboxmoveright{\mybox}{0.5\width}}
\test{\setboxlower{\mybox}{\height}}
\test{\setboxraise{\mybox}{\depth}}
\test{%
  \setboxmoveright{\mybox}{5pt}~%
  \setboxwidth{\mybox}{\width + 5pt}~%
  \setboxheight{\mybox}{\height + 5pt}~%
  \setboxdepth{\mybox}{\depth + 5pt}%
}

\end{document}
%END
%</example>
%    \end{macrocode}
%
% \noindent
%    The result:
%
% \vspace{1ex}
% \hrule
%
% \begingroup
% \newsavebox{\mybox}
%
% \setlength{\fboxsep}{0pt}
% \setlength{\parindent}{20pt}
% \setlength{\parskip}{10pt}
%
% \makeatletter
% \newcommand*{\test}[1]{^^A
%   \par
%   \begingroup
%     \raggedright
%     \edef\x{\detokenize{#1}}
%     \let\do\@makeother
%     \dospecials
%     \catcode`\~\active
%     \catcode`\ =10\relax
%     \def~{\\}^^A
%     \noindent
%     \texttt{\scantokens\expandafter{\x}}
%     \par
%   \endgroup
%   \begingroup
%     \let~\relax
%     \sbox{\mybox}{The cracy fox.}
%      #1^^A
%      A---\fbox{\usebox\mybox}---B
%   \endgroup
%   \par
% }
% \makeatother
%
% \test{\setboxwidth{\mybox}{1.25\width}}
% \test{\setboxheight{\mybox}{0pt}}
% \test{\setboxheight{\mybox}{2\height}}
% \test{\setboxdepth{\mybox}{\height}}
% \test{\setboxmoveleft{\mybox}{5pt}}
% \test{^^A
%   \setboxmoveleft{\mybox}{5pt}~^^A
%   \setboxwidth{\mybox}{\width + 5pt}^^A
% }
% \test{\setboxmoveright{\mybox}{0.5\width}}
% \test{\setboxlower{\mybox}{\height}}
% \test{\setboxraise{\mybox}{\depth}}
% \test{^^A
%   \setboxmoveright{\mybox}{5pt}~^^A
%   \setboxwidth{\mybox}{\width + 5pt}~^^A
%   \setboxheight{\mybox}{\height + 5pt}~^^A
%   \setboxdepth{\mybox}{\depth + 5pt}^^A
% }
%
% \endgroup
% \vspace{1ex}
% \hrule
% \vspace{4ex}
%
% \StopEventually{
% }
%
% \section{Implementation}
%
%    \begin{macrocode}
%<*package>
%    \end{macrocode}
%    Package identification.
%    \begin{macrocode}
\NeedsTeXFormat{LaTeX2e}
\ProvidesPackage{settobox}%
  [2016/05/16 v1.5 Assign box dimensions to length registers (HO)]
%    \end{macrocode}
%
%    \begin{macrocode}
\newcommand*{\settoboxwidth}[2]{\setlength{#1}{\wd#2}}
\newcommand*{\settoboxheight}[2]{\setlength{#1}{\ht#2}}
\newcommand*{\settoboxdepth}[2]{\setlength{#1}{\dp#2}}
\newcommand*{\settoboxtotalheight}[2]{%
  \setlength{#1}{\ht#2}%
  \addtolength{#1}{\dp#2}%
}
%    \end{macrocode}
%
%    \begin{macro}{\setboxwidth}
%    \begin{macrocode}
\newcommand*{\setboxwidth}[2]{%
  \settobox@length\wd{#1}{#2}%
}
%    \end{macrocode}
%    \end{macro}
%    \begin{macro}{\setboxheight}
%    \begin{macrocode}
\newcommand*{\setboxheight}[2]{%
  \settobox@length\ht{#1}{#2}%
}
%    \end{macrocode}
%    \end{macro}
%    \begin{macro}{\setboxheight}
%    \begin{macrocode}
\newcommand*{\setboxdepth}[2]{%
  \settobox@length\dp{#1}{#2}%
}
%    \end{macrocode}
%    \end{macro}
%    \begin{macro}{\setboxmoveleft}
%    \begin{macrocode}
\newcommand*{\setboxmoveleft}[2]{%
  \settobox@horiz{-}{#1}{#2}%
}
%    \end{macrocode}
%    \end{macro}
%    \begin{macro}{\setboxmoveright}
%    \begin{macrocode}
\newcommand*{\setboxmoveright}[2]{%
  \settobox@horiz{}{#1}{#2}%
}
%    \end{macrocode}
%    \end{macro}
%    \begin{macro}{\setboxlower}
%    \begin{macrocode}
\newcommand*{\setboxlower}[2]{%
  \settobox@vert\lower{#1}{#2}%
}
%    \end{macrocode}
%    \end{macro}
%    \begin{macro}{\setboxraise}
%    \begin{macrocode}
\newcommand*{\setboxraise}[2]{%
  \settobox@vert\raise{#1}{#2}%
}
%    \end{macrocode}
%    \end{macro}
%    \begin{macro}{\settobox@length}
%    The work for the \cs{setbox...} commands is done by
%    \cs{settobox@length}. Inside the length expression
%    \cs{width}, \cs{height}, \cs{depth}, \cs{totalheight}
%    are set to the dimensions of the box.\\
%    \begin{tabular}{@{}ll@{}}
%    |#1|:& the property of the box that is to be changed
%           (\cs{wd}, \cs{ht}, \cs{dp})\\
%    |#2|:& the box\\
%    |#3|:& length expression
%    \end{tabular}
%    \begin{macrocode}
\def\settobox@length#1#2#3{%
  \settobox@calc{#2}{#3}{#1#2=##1sp\relax}%
}
%    \end{macrocode}
%    \end{macro}
%
%    \begin{macro}{\settobox@horiz}
%    \begin{macrocode}
\def\settobox@horiz#1#2#3{%
  \settobox@calc{#2}{#3}{\setbox#2=\hbox{\kern#1##1sp\copy#2}}%
}
%    \end{macrocode}
%    \end{macro}
%    \begin{macro}{\settobox@vert}
%    \begin{macrocode}
\def\settobox@vert#1#2#3{%
  \settobox@calc{#2}{#3}{\setbox#2=\hbox{#1##1sp\copy#2}}%
}
%    \end{macrocode}
%    \end{macro}
%
%    \begin{macro}{\settobox@calc}
%    \begin{macrocode}
\def\settobox@calc#1#2#3{%
  \begingroup
    \def\width{\wd#1}%
    \def\height{\ht#1}%
    \def\depth{\dp#1}%
    \dimen@\ht#1\relax
    \advance\dimen@\dp#1\relax
    \def\totalheight{\dimen@}%
    \setlength{\dimen@}{#2}%
    \count@\dimen@
    \def\x##1{\endgroup
      #3%
    }%
  \expandafter\x\expandafter{\the\count@}%
}
%    \end{macrocode}
%    \end{macro}
%
%    \begin{macrocode}
%</package>
%    \end{macrocode}
%
% \section{Installation}
%
% \subsection{Download}
%
% \paragraph{Package.} This package is available on
% CTAN\footnote{\CTANpkg{settobox}}:
% \begin{description}
% \item[\CTAN{macros/latex/contrib/oberdiek/settobox.dtx}] The source file.
% \item[\CTAN{macros/latex/contrib/oberdiek/settobox.pdf}] Documentation.
% \end{description}
%
%
% \paragraph{Bundle.} All the packages of the bundle `oberdiek'
% are also available in a TDS compliant ZIP archive. There
% the packages are already unpacked and the documentation files
% are generated. The files and directories obey the TDS standard.
% \begin{description}
% \item[\CTANinstall{install/macros/latex/contrib/oberdiek.tds.zip}]
% \end{description}
% \emph{TDS} refers to the standard ``A Directory Structure
% for \TeX\ Files'' (\CTANpkg{tds}). Directories
% with \xfile{texmf} in their name are usually organized this way.
%
% \subsection{Bundle installation}
%
% \paragraph{Unpacking.} Unpack the \xfile{oberdiek.tds.zip} in the
% TDS tree (also known as \xfile{texmf} tree) of your choice.
% Example (linux):
% \begin{quote}
%   |unzip oberdiek.tds.zip -d ~/texmf|
% \end{quote}
%
% \subsection{Package installation}
%
% \paragraph{Unpacking.} The \xfile{.dtx} file is a self-extracting
% \docstrip\ archive. The files are extracted by running the
% \xfile{.dtx} through \plainTeX:
% \begin{quote}
%   \verb|tex settobox.dtx|
% \end{quote}
%
% \paragraph{TDS.} Now the different files must be moved into
% the different directories in your installation TDS tree
% (also known as \xfile{texmf} tree):
% \begin{quote}
% \def\t{^^A
% \begin{tabular}{@{}>{\ttfamily}l@{ $\rightarrow$ }>{\ttfamily}l@{}}
%   settobox.sty & tex/latex/oberdiek/settobox.sty\\
%   settobox.pdf & doc/latex/oberdiek/settobox.pdf\\
%   settobox-example.tex & doc/latex/oberdiek/settobox-example.tex\\
%   settobox.dtx & source/latex/oberdiek/settobox.dtx\\
% \end{tabular}^^A
% }^^A
% \sbox0{\t}^^A
% \ifdim\wd0>\linewidth
%   \begingroup
%     \advance\linewidth by\leftmargin
%     \advance\linewidth by\rightmargin
%   \edef\x{\endgroup
%     \def\noexpand\lw{\the\linewidth}^^A
%   }\x
%   \def\lwbox{^^A
%     \leavevmode
%     \hbox to \linewidth{^^A
%       \kern-\leftmargin\relax
%       \hss
%       \usebox0
%       \hss
%       \kern-\rightmargin\relax
%     }^^A
%   }^^A
%   \ifdim\wd0>\lw
%     \sbox0{\small\t}^^A
%     \ifdim\wd0>\linewidth
%       \ifdim\wd0>\lw
%         \sbox0{\footnotesize\t}^^A
%         \ifdim\wd0>\linewidth
%           \ifdim\wd0>\lw
%             \sbox0{\scriptsize\t}^^A
%             \ifdim\wd0>\linewidth
%               \ifdim\wd0>\lw
%                 \sbox0{\tiny\t}^^A
%                 \ifdim\wd0>\linewidth
%                   \lwbox
%                 \else
%                   \usebox0
%                 \fi
%               \else
%                 \lwbox
%               \fi
%             \else
%               \usebox0
%             \fi
%           \else
%             \lwbox
%           \fi
%         \else
%           \usebox0
%         \fi
%       \else
%         \lwbox
%       \fi
%     \else
%       \usebox0
%     \fi
%   \else
%     \lwbox
%   \fi
% \else
%   \usebox0
% \fi
% \end{quote}
% If you have a \xfile{docstrip.cfg} that configures and enables \docstrip's
% TDS installing feature, then some files can already be in the right
% place, see the documentation of \docstrip.
%
% \subsection{Refresh file name databases}
%
% If your \TeX~distribution
% (\TeX\,Live, \mikTeX, \dots) relies on file name databases, you must refresh
% these. For example, \TeX\,Live\ users run \verb|texhash| or
% \verb|mktexlsr|.
%
% \subsection{Some details for the interested}
%
% \paragraph{Unpacking with \LaTeX.}
% The \xfile{.dtx} chooses its action depending on the format:
% \begin{description}
% \item[\plainTeX:] Run \docstrip\ and extract the files.
% \item[\LaTeX:] Generate the documentation.
% \end{description}
% If you insist on using \LaTeX\ for \docstrip\ (really,
% \docstrip\ does not need \LaTeX), then inform the autodetect routine
% about your intention:
% \begin{quote}
%   \verb|latex \let\install=y\input{settobox.dtx}|
% \end{quote}
% Do not forget to quote the argument according to the demands
% of your shell.
%
% \paragraph{Generating the documentation.}
% You can use both the \xfile{.dtx} or the \xfile{.drv} to generate
% the documentation. The process can be configured by the
% configuration file \xfile{ltxdoc.cfg}. For instance, put this
% line into this file, if you want to have A4 as paper format:
% \begin{quote}
%   \verb|\PassOptionsToClass{a4paper}{article}|
% \end{quote}
% An example follows how to generate the
% documentation with pdf\LaTeX:
% \begin{quote}
%\begin{verbatim}
%pdflatex settobox.dtx
%makeindex -s gind.ist settobox.idx
%pdflatex settobox.dtx
%makeindex -s gind.ist settobox.idx
%pdflatex settobox.dtx
%\end{verbatim}
% \end{quote}
%
% \begin{History}
%   \begin{Version}{2000/02/11 v1.0}
%   \item
%     First public release, written as answer in the
%     newsgroup \xnewsgroup{de.comp.text.tex}:
%     \URL{``\link{Die Hoehe von Minipages und Bild}''}^^A
%     {https://groups.google.com/group/de.comp.text.tex/msg/c3f6446f54f66c02}
%   \end{Version}
%   \begin{Version}{2000/09/07 v1.1}
%   \item
%     Documentation added.
%   \item
%     CTAN release.
%   \end{Version}
%   \begin{Version}{2006/02/20 v1.2}
%   \item
%     \cs{setboxwidth}, \cs{setboxheight}, \cs{setboxdepth} added.
%   \item
%     Box move commands added.
%   \item
%     DTX framework.
%   \item
%     LPPL 1.3
%   \end{Version}
%   \begin{Version}{2007/04/11 v1.3}
%   \item
%     Line ends sanitized.
%   \end{Version}
%   \begin{Version}{2008/08/11 v1.4}
%   \item
%     Code is not changed.
%   \item
%     URLs updated.
%   \end{Version}
%   \begin{Version}{2016/05/16 v1.5}
%   \item
%     Documentation updates.
%   \end{Version}
% \end{History}
%
% \PrintIndex
%
% \Finale
\endinput

%        (quote the arguments according to the demands of your shell)
%
% Documentation:
%    (a) If settobox.drv is present:
%           latex settobox.drv
%    (b) Without settobox.drv:
%           latex settobox.dtx; ...
%    The class ltxdoc loads the configuration file ltxdoc.cfg
%    if available. Here you can specify further options, e.g.
%    use A4 as paper format:
%       \PassOptionsToClass{a4paper}{article}
%
%    Programm calls to get the documentation (example):
%       pdflatex settobox.dtx
%       makeindex -s gind.ist settobox.idx
%       pdflatex settobox.dtx
%       makeindex -s gind.ist settobox.idx
%       pdflatex settobox.dtx
%
% Installation:
%    TDS:tex/latex/oberdiek/settobox.sty
%    TDS:doc/latex/oberdiek/settobox.pdf
%    TDS:doc/latex/oberdiek/settobox-example.tex
%    TDS:source/latex/oberdiek/settobox.dtx
%
%<*ignore>
\begingroup
  \catcode123=1 %
  \catcode125=2 %
  \def\x{LaTeX2e}%
\expandafter\endgroup
\ifcase 0\ifx\install y1\fi\expandafter
         \ifx\csname processbatchFile\endcsname\relax\else1\fi
         \ifx\fmtname\x\else 1\fi\relax
\else\csname fi\endcsname
%</ignore>
%<*install>
\input docstrip.tex
\Msg{************************************************************************}
\Msg{* Installation}
\Msg{* Package: settobox 2016/05/16 v1.5 Assign box dimensions to length registers (HO)}
\Msg{************************************************************************}

\keepsilent
\askforoverwritefalse

\let\MetaPrefix\relax
\preamble

This is a generated file.

Project: settobox
Version: 2016/05/16 v1.5

Copyright (C)
   2000, 2006-2008 Heiko Oberdiek
   2016-2019 Oberdiek Package Support Group

This work may be distributed and/or modified under the
conditions of the LaTeX Project Public License, either
version 1.3c of this license or (at your option) any later
version. This version of this license is in
   https://www.latex-project.org/lppl/lppl-1-3c.txt
and the latest version of this license is in
   https://www.latex-project.org/lppl.txt
and version 1.3 or later is part of all distributions of
LaTeX version 2005/12/01 or later.

This work has the LPPL maintenance status "maintained".

The Current Maintainers of this work are
Heiko Oberdiek and the Oberdiek Package Support Group
https://github.com/ho-tex/oberdiek/issues


This work consists of the main source file settobox.dtx
and the derived files
   settobox.sty, settobox.pdf, settobox.ins, settobox.drv,
   settobox-example.tex.

\endpreamble
\let\MetaPrefix\DoubleperCent

\generate{%
  \file{settobox.ins}{\from{settobox.dtx}{install}}%
  \file{settobox.drv}{\from{settobox.dtx}{driver}}%
  \usedir{tex/latex/oberdiek}%
  \file{settobox.sty}{\from{settobox.dtx}{package}}%
  \usedir{doc/latex/oberdiek}%
  \file{settobox-example.tex}{\from{settobox.dtx}{example}}%
}

\catcode32=13\relax% active space
\let =\space%
\Msg{************************************************************************}
\Msg{*}
\Msg{* To finish the installation you have to move the following}
\Msg{* file into a directory searched by TeX:}
\Msg{*}
\Msg{*     settobox.sty}
\Msg{*}
\Msg{* To produce the documentation run the file `settobox.drv'}
\Msg{* through LaTeX.}
\Msg{*}
\Msg{* Happy TeXing!}
\Msg{*}
\Msg{************************************************************************}

\endbatchfile
%</install>
%<*ignore>
\fi
%</ignore>
%<*driver>
\NeedsTeXFormat{LaTeX2e}
\ProvidesFile{settobox.drv}%
  [2016/05/16 v1.5 Assign box dimensions to length registers (HO)]%
\documentclass{ltxdoc}
\usepackage{holtxdoc}[2011/11/22]
\usepackage{calc}
\usepackage{settobox}
\begin{document}
  \DocInput{settobox.dtx}%
\end{document}
%</driver>
% \fi
%
%
%
% \GetFileInfo{settobox.drv}
%
% \title{The \xpackage{settobox} package}
% \date{2016/05/16 v1.5}
% \author{Heiko Oberdiek\thanks
% {Please report any issues at \url{https://github.com/ho-tex/oberdiek/issues}}}
%
% \maketitle
%
% \begin{abstract}
% Commands are defined for getting box sizes similar
% to \LaTeX's \cs{settowidth} commands.
% \end{abstract}
%
% \tableofcontents
%
% \section{Usage}
%
% \subsection{Get box dimensions}
%
% \begin{declcs}^^A
%   {settoboxwidth}\,\M{\LaTeX\ length}\,\M{\LaTeX\ box}\\
%   \SpecialUsageIndex{\settoboxheight}^^A
%   \cs{settoboxheight}\,\M{\LaTeX\ length}\,\M{\LaTeX\ box}\\
%   \SpecialUsageIndex{\settoboxdepth}^^A
%   \cs{settoboxdepth}\,\M{\LaTeX\ length}\,\M{\LaTeX\ box}\\
%   \SpecialUsageIndex{\settoboxtotalheight}^^A
%   \cs{settoboxtotalheight}\,\M{\LaTeX\ length}\,\M{\LaTeX\ box}
% \end{declcs}
% A \meta{\LaTeX\ box} is allocated by \cs{newsavebox}.
% It can be filled by \cs{sbox} or the environment \texttt{lrbox}.
% The commands above extract then the desired lengths.
%
% \subsection{Set box dimensions}
%
% \begin{declcs}^^A
%   {setboxwidth}\,\M{\LaTeX\ box}\,\M{\LaTeX\ length expression}\\
%   \SpecialUsageIndex{\setboxheight}^^A
%   \cs{setboxheight}\,\M{\LaTeX\ box}\,\M{\LaTeX\ length expression}\\
%   \SpecialUsageIndex{\setboxdepth}^^A
%   \cs{setboxdepth}\,\M{\LaTeX\ box}\,\M{\LaTeX\ length expression}
% \end{declcs}
% These commands allow the manipulation of the box. Package \xpackage{calc}
% is supported in the \meta{\LaTeX\ length expression}.
% Also the following length are available in this expression:
% \begin{quote}
% \begin{tabular}{@{}ll@{}}
%   \cs{width}& width of the box\\
%   \cs{height}& height of the box\\
%   \cs{depth}& depth of the box\\
%   \cs{totalheight}& totalheight of the box\\
% \end{tabular}
% \end{quote}
% Note, the base point (point at the left margin of the baseline)
% always remain constant.
%
% \subsection{Move box}
%
% \begin{declcs}^^A
%   {setboxmoveleft}\,\M{\LaTeX\ box}\,\M{\LaTeX\ length expression}\\
%   \SpecialUsageIndex{\setboxmoveright}^^A
%   \cs{setboxmoveright}\,\M{\LaTeX\ box}\,\M{\LaTeX\ length expression}\\
%   \SpecialUsageIndex{\setboxlower}^^A
%   \cs{setboxlower}\,\M{\LaTeX\ box}\,\M{\LaTeX\ length expression}\\
%   \SpecialUsageIndex{\setboxright}^^A
%   \cs{setboxright}\,\M{\LaTeX\ box}\,\M{\LaTeX\ length expression}
% \end{declcs}
% Note, the box is shifted relative to the base point. The base point
% is always inside the box, however the width and height of the
% box change along with the movement.
%
% \subsection{Example}
%
% \subsubsection{Short example}
%
% \begin{quote}
%\begin{verbatim}
%\newsavebox{\mybox}
%\newlength{\mylength}
%\sbox{\mybox}{Hello World}
%\settoboxwidth{\mylength}{\mybox}
%\end{verbatim}
% \end{quote}
%
% \subsubsection{Test file that shows box manipulations}
%
%    \begin{macrocode}
%<*example>
%<<END
\documentclass{article}

\usepackage{settobox}
\usepackage{calc}

\newsavebox{\mybox}

\setlength{\fboxsep}{0pt}
\setlength{\parindent}{20pt}
\setlength{\parskip}{10pt}
\pagestyle{empty}

% \test{#1}
% The macro is called with commands in #1 that manipulates
% the box \mybox. These commands along with the result of
% the manipulation is shown. Thus the essence of the
% macro is:
%
%   a) \sbox{\mybox}{The cracy fox.}
%   b) #1 % manipulates \mybox
%   c) Print #1 commands.
%   d) Print box with frame
%
% The implemenation looks more weird:
\makeatletter
\newcommand*{\test}[1]{%
  \par
  \begingroup
    \raggedright
    \edef\x{\detokenize{#1}}%
    \let\do\@makeother
    \dospecials
    \catcode`\~\active
    \catcode`\ =10\relax
    \def~{\\}%
    \noindent
    \texttt{\scantokens\expandafter{\x}}%
    \par
  \endgroup
  \begingroup
    \let~\relax
    \sbox{\mybox}{The cracy fox.}%
     #1%
     A---\fbox{\usebox\mybox}---B%
  \endgroup
  \par
}
\makeatother

\begin{document}

\test{\setboxwidth{\mybox}{1.25\width}}
\test{\setboxheight{\mybox}{0pt}}
\test{\setboxheight{\mybox}{2\height}}
\test{\setboxdepth{\mybox}{\height}}
\test{\setboxmoveleft{\mybox}{5pt}}
\test{%
  \setboxmoveleft{\mybox}{5pt}~%
  \setboxwidth{\mybox}{\width + 5pt}%
}
\test{\setboxmoveright{\mybox}{0.5\width}}
\test{\setboxlower{\mybox}{\height}}
\test{\setboxraise{\mybox}{\depth}}
\test{%
  \setboxmoveright{\mybox}{5pt}~%
  \setboxwidth{\mybox}{\width + 5pt}~%
  \setboxheight{\mybox}{\height + 5pt}~%
  \setboxdepth{\mybox}{\depth + 5pt}%
}

\end{document}
%END
%</example>
%    \end{macrocode}
%
% \noindent
%    The result:
%
% \vspace{1ex}
% \hrule
%
% \begingroup
% \newsavebox{\mybox}
%
% \setlength{\fboxsep}{0pt}
% \setlength{\parindent}{20pt}
% \setlength{\parskip}{10pt}
%
% \makeatletter
% \newcommand*{\test}[1]{^^A
%   \par
%   \begingroup
%     \raggedright
%     \edef\x{\detokenize{#1}}
%     \let\do\@makeother
%     \dospecials
%     \catcode`\~\active
%     \catcode`\ =10\relax
%     \def~{\\}^^A
%     \noindent
%     \texttt{\scantokens\expandafter{\x}}
%     \par
%   \endgroup
%   \begingroup
%     \let~\relax
%     \sbox{\mybox}{The cracy fox.}
%      #1^^A
%      A---\fbox{\usebox\mybox}---B
%   \endgroup
%   \par
% }
% \makeatother
%
% \test{\setboxwidth{\mybox}{1.25\width}}
% \test{\setboxheight{\mybox}{0pt}}
% \test{\setboxheight{\mybox}{2\height}}
% \test{\setboxdepth{\mybox}{\height}}
% \test{\setboxmoveleft{\mybox}{5pt}}
% \test{^^A
%   \setboxmoveleft{\mybox}{5pt}~^^A
%   \setboxwidth{\mybox}{\width + 5pt}^^A
% }
% \test{\setboxmoveright{\mybox}{0.5\width}}
% \test{\setboxlower{\mybox}{\height}}
% \test{\setboxraise{\mybox}{\depth}}
% \test{^^A
%   \setboxmoveright{\mybox}{5pt}~^^A
%   \setboxwidth{\mybox}{\width + 5pt}~^^A
%   \setboxheight{\mybox}{\height + 5pt}~^^A
%   \setboxdepth{\mybox}{\depth + 5pt}^^A
% }
%
% \endgroup
% \vspace{1ex}
% \hrule
% \vspace{4ex}
%
% \StopEventually{
% }
%
% \section{Implementation}
%
%    \begin{macrocode}
%<*package>
%    \end{macrocode}
%    Package identification.
%    \begin{macrocode}
\NeedsTeXFormat{LaTeX2e}
\ProvidesPackage{settobox}%
  [2016/05/16 v1.5 Assign box dimensions to length registers (HO)]
%    \end{macrocode}
%
%    \begin{macrocode}
\newcommand*{\settoboxwidth}[2]{\setlength{#1}{\wd#2}}
\newcommand*{\settoboxheight}[2]{\setlength{#1}{\ht#2}}
\newcommand*{\settoboxdepth}[2]{\setlength{#1}{\dp#2}}
\newcommand*{\settoboxtotalheight}[2]{%
  \setlength{#1}{\ht#2}%
  \addtolength{#1}{\dp#2}%
}
%    \end{macrocode}
%
%    \begin{macro}{\setboxwidth}
%    \begin{macrocode}
\newcommand*{\setboxwidth}[2]{%
  \settobox@length\wd{#1}{#2}%
}
%    \end{macrocode}
%    \end{macro}
%    \begin{macro}{\setboxheight}
%    \begin{macrocode}
\newcommand*{\setboxheight}[2]{%
  \settobox@length\ht{#1}{#2}%
}
%    \end{macrocode}
%    \end{macro}
%    \begin{macro}{\setboxheight}
%    \begin{macrocode}
\newcommand*{\setboxdepth}[2]{%
  \settobox@length\dp{#1}{#2}%
}
%    \end{macrocode}
%    \end{macro}
%    \begin{macro}{\setboxmoveleft}
%    \begin{macrocode}
\newcommand*{\setboxmoveleft}[2]{%
  \settobox@horiz{-}{#1}{#2}%
}
%    \end{macrocode}
%    \end{macro}
%    \begin{macro}{\setboxmoveright}
%    \begin{macrocode}
\newcommand*{\setboxmoveright}[2]{%
  \settobox@horiz{}{#1}{#2}%
}
%    \end{macrocode}
%    \end{macro}
%    \begin{macro}{\setboxlower}
%    \begin{macrocode}
\newcommand*{\setboxlower}[2]{%
  \settobox@vert\lower{#1}{#2}%
}
%    \end{macrocode}
%    \end{macro}
%    \begin{macro}{\setboxraise}
%    \begin{macrocode}
\newcommand*{\setboxraise}[2]{%
  \settobox@vert\raise{#1}{#2}%
}
%    \end{macrocode}
%    \end{macro}
%    \begin{macro}{\settobox@length}
%    The work for the \cs{setbox...} commands is done by
%    \cs{settobox@length}. Inside the length expression
%    \cs{width}, \cs{height}, \cs{depth}, \cs{totalheight}
%    are set to the dimensions of the box.\\
%    \begin{tabular}{@{}ll@{}}
%    |#1|:& the property of the box that is to be changed
%           (\cs{wd}, \cs{ht}, \cs{dp})\\
%    |#2|:& the box\\
%    |#3|:& length expression
%    \end{tabular}
%    \begin{macrocode}
\def\settobox@length#1#2#3{%
  \settobox@calc{#2}{#3}{#1#2=##1sp\relax}%
}
%    \end{macrocode}
%    \end{macro}
%
%    \begin{macro}{\settobox@horiz}
%    \begin{macrocode}
\def\settobox@horiz#1#2#3{%
  \settobox@calc{#2}{#3}{\setbox#2=\hbox{\kern#1##1sp\copy#2}}%
}
%    \end{macrocode}
%    \end{macro}
%    \begin{macro}{\settobox@vert}
%    \begin{macrocode}
\def\settobox@vert#1#2#3{%
  \settobox@calc{#2}{#3}{\setbox#2=\hbox{#1##1sp\copy#2}}%
}
%    \end{macrocode}
%    \end{macro}
%
%    \begin{macro}{\settobox@calc}
%    \begin{macrocode}
\def\settobox@calc#1#2#3{%
  \begingroup
    \def\width{\wd#1}%
    \def\height{\ht#1}%
    \def\depth{\dp#1}%
    \dimen@\ht#1\relax
    \advance\dimen@\dp#1\relax
    \def\totalheight{\dimen@}%
    \setlength{\dimen@}{#2}%
    \count@\dimen@
    \def\x##1{\endgroup
      #3%
    }%
  \expandafter\x\expandafter{\the\count@}%
}
%    \end{macrocode}
%    \end{macro}
%
%    \begin{macrocode}
%</package>
%    \end{macrocode}
%
% \section{Installation}
%
% \subsection{Download}
%
% \paragraph{Package.} This package is available on
% CTAN\footnote{\CTANpkg{settobox}}:
% \begin{description}
% \item[\CTAN{macros/latex/contrib/oberdiek/settobox.dtx}] The source file.
% \item[\CTAN{macros/latex/contrib/oberdiek/settobox.pdf}] Documentation.
% \end{description}
%
%
% \paragraph{Bundle.} All the packages of the bundle `oberdiek'
% are also available in a TDS compliant ZIP archive. There
% the packages are already unpacked and the documentation files
% are generated. The files and directories obey the TDS standard.
% \begin{description}
% \item[\CTANinstall{install/macros/latex/contrib/oberdiek.tds.zip}]
% \end{description}
% \emph{TDS} refers to the standard ``A Directory Structure
% for \TeX\ Files'' (\CTANpkg{tds}). Directories
% with \xfile{texmf} in their name are usually organized this way.
%
% \subsection{Bundle installation}
%
% \paragraph{Unpacking.} Unpack the \xfile{oberdiek.tds.zip} in the
% TDS tree (also known as \xfile{texmf} tree) of your choice.
% Example (linux):
% \begin{quote}
%   |unzip oberdiek.tds.zip -d ~/texmf|
% \end{quote}
%
% \subsection{Package installation}
%
% \paragraph{Unpacking.} The \xfile{.dtx} file is a self-extracting
% \docstrip\ archive. The files are extracted by running the
% \xfile{.dtx} through \plainTeX:
% \begin{quote}
%   \verb|tex settobox.dtx|
% \end{quote}
%
% \paragraph{TDS.} Now the different files must be moved into
% the different directories in your installation TDS tree
% (also known as \xfile{texmf} tree):
% \begin{quote}
% \def\t{^^A
% \begin{tabular}{@{}>{\ttfamily}l@{ $\rightarrow$ }>{\ttfamily}l@{}}
%   settobox.sty & tex/latex/oberdiek/settobox.sty\\
%   settobox.pdf & doc/latex/oberdiek/settobox.pdf\\
%   settobox-example.tex & doc/latex/oberdiek/settobox-example.tex\\
%   settobox.dtx & source/latex/oberdiek/settobox.dtx\\
% \end{tabular}^^A
% }^^A
% \sbox0{\t}^^A
% \ifdim\wd0>\linewidth
%   \begingroup
%     \advance\linewidth by\leftmargin
%     \advance\linewidth by\rightmargin
%   \edef\x{\endgroup
%     \def\noexpand\lw{\the\linewidth}^^A
%   }\x
%   \def\lwbox{^^A
%     \leavevmode
%     \hbox to \linewidth{^^A
%       \kern-\leftmargin\relax
%       \hss
%       \usebox0
%       \hss
%       \kern-\rightmargin\relax
%     }^^A
%   }^^A
%   \ifdim\wd0>\lw
%     \sbox0{\small\t}^^A
%     \ifdim\wd0>\linewidth
%       \ifdim\wd0>\lw
%         \sbox0{\footnotesize\t}^^A
%         \ifdim\wd0>\linewidth
%           \ifdim\wd0>\lw
%             \sbox0{\scriptsize\t}^^A
%             \ifdim\wd0>\linewidth
%               \ifdim\wd0>\lw
%                 \sbox0{\tiny\t}^^A
%                 \ifdim\wd0>\linewidth
%                   \lwbox
%                 \else
%                   \usebox0
%                 \fi
%               \else
%                 \lwbox
%               \fi
%             \else
%               \usebox0
%             \fi
%           \else
%             \lwbox
%           \fi
%         \else
%           \usebox0
%         \fi
%       \else
%         \lwbox
%       \fi
%     \else
%       \usebox0
%     \fi
%   \else
%     \lwbox
%   \fi
% \else
%   \usebox0
% \fi
% \end{quote}
% If you have a \xfile{docstrip.cfg} that configures and enables \docstrip's
% TDS installing feature, then some files can already be in the right
% place, see the documentation of \docstrip.
%
% \subsection{Refresh file name databases}
%
% If your \TeX~distribution
% (\TeX\,Live, \mikTeX, \dots) relies on file name databases, you must refresh
% these. For example, \TeX\,Live\ users run \verb|texhash| or
% \verb|mktexlsr|.
%
% \subsection{Some details for the interested}
%
% \paragraph{Unpacking with \LaTeX.}
% The \xfile{.dtx} chooses its action depending on the format:
% \begin{description}
% \item[\plainTeX:] Run \docstrip\ and extract the files.
% \item[\LaTeX:] Generate the documentation.
% \end{description}
% If you insist on using \LaTeX\ for \docstrip\ (really,
% \docstrip\ does not need \LaTeX), then inform the autodetect routine
% about your intention:
% \begin{quote}
%   \verb|latex \let\install=y% \iffalse meta-comment
%
% File: settobox.dtx
% Version: 2016/05/16 v1.5
% Info: Assign box dimensions to length registers
%
% Copyright (C)
%    2000, 2006-2008 Heiko Oberdiek
%    2016-2019 Oberdiek Package Support Group
%    https://github.com/ho-tex/oberdiek/issues
%
% This work may be distributed and/or modified under the
% conditions of the LaTeX Project Public License, either
% version 1.3c of this license or (at your option) any later
% version. This version of this license is in
%    https://www.latex-project.org/lppl/lppl-1-3c.txt
% and the latest version of this license is in
%    https://www.latex-project.org/lppl.txt
% and version 1.3 or later is part of all distributions of
% LaTeX version 2005/12/01 or later.
%
% This work has the LPPL maintenance status "maintained".
%
% The Current Maintainers of this work are
% Heiko Oberdiek and the Oberdiek Package Support Group
% https://github.com/ho-tex/oberdiek/issues
%
% This work consists of the main source file settobox.dtx
% and the derived files
%    settobox.sty, settobox.pdf, settobox.ins, settobox.drv,
%    settobox-example.tex.
%
% Distribution:
%    CTAN:macros/latex/contrib/oberdiek/settobox.dtx
%    CTAN:macros/latex/contrib/oberdiek/settobox.pdf
%
% Unpacking:
%    (a) If settobox.ins is present:
%           tex settobox.ins
%    (b) Without settobox.ins:
%           tex settobox.dtx
%    (c) If you insist on using LaTeX
%           latex \let\install=y\input{settobox.dtx}
%        (quote the arguments according to the demands of your shell)
%
% Documentation:
%    (a) If settobox.drv is present:
%           latex settobox.drv
%    (b) Without settobox.drv:
%           latex settobox.dtx; ...
%    The class ltxdoc loads the configuration file ltxdoc.cfg
%    if available. Here you can specify further options, e.g.
%    use A4 as paper format:
%       \PassOptionsToClass{a4paper}{article}
%
%    Programm calls to get the documentation (example):
%       pdflatex settobox.dtx
%       makeindex -s gind.ist settobox.idx
%       pdflatex settobox.dtx
%       makeindex -s gind.ist settobox.idx
%       pdflatex settobox.dtx
%
% Installation:
%    TDS:tex/latex/oberdiek/settobox.sty
%    TDS:doc/latex/oberdiek/settobox.pdf
%    TDS:doc/latex/oberdiek/settobox-example.tex
%    TDS:source/latex/oberdiek/settobox.dtx
%
%<*ignore>
\begingroup
  \catcode123=1 %
  \catcode125=2 %
  \def\x{LaTeX2e}%
\expandafter\endgroup
\ifcase 0\ifx\install y1\fi\expandafter
         \ifx\csname processbatchFile\endcsname\relax\else1\fi
         \ifx\fmtname\x\else 1\fi\relax
\else\csname fi\endcsname
%</ignore>
%<*install>
\input docstrip.tex
\Msg{************************************************************************}
\Msg{* Installation}
\Msg{* Package: settobox 2016/05/16 v1.5 Assign box dimensions to length registers (HO)}
\Msg{************************************************************************}

\keepsilent
\askforoverwritefalse

\let\MetaPrefix\relax
\preamble

This is a generated file.

Project: settobox
Version: 2016/05/16 v1.5

Copyright (C)
   2000, 2006-2008 Heiko Oberdiek
   2016-2019 Oberdiek Package Support Group

This work may be distributed and/or modified under the
conditions of the LaTeX Project Public License, either
version 1.3c of this license or (at your option) any later
version. This version of this license is in
   https://www.latex-project.org/lppl/lppl-1-3c.txt
and the latest version of this license is in
   https://www.latex-project.org/lppl.txt
and version 1.3 or later is part of all distributions of
LaTeX version 2005/12/01 or later.

This work has the LPPL maintenance status "maintained".

The Current Maintainers of this work are
Heiko Oberdiek and the Oberdiek Package Support Group
https://github.com/ho-tex/oberdiek/issues


This work consists of the main source file settobox.dtx
and the derived files
   settobox.sty, settobox.pdf, settobox.ins, settobox.drv,
   settobox-example.tex.

\endpreamble
\let\MetaPrefix\DoubleperCent

\generate{%
  \file{settobox.ins}{\from{settobox.dtx}{install}}%
  \file{settobox.drv}{\from{settobox.dtx}{driver}}%
  \usedir{tex/latex/oberdiek}%
  \file{settobox.sty}{\from{settobox.dtx}{package}}%
  \usedir{doc/latex/oberdiek}%
  \file{settobox-example.tex}{\from{settobox.dtx}{example}}%
}

\catcode32=13\relax% active space
\let =\space%
\Msg{************************************************************************}
\Msg{*}
\Msg{* To finish the installation you have to move the following}
\Msg{* file into a directory searched by TeX:}
\Msg{*}
\Msg{*     settobox.sty}
\Msg{*}
\Msg{* To produce the documentation run the file `settobox.drv'}
\Msg{* through LaTeX.}
\Msg{*}
\Msg{* Happy TeXing!}
\Msg{*}
\Msg{************************************************************************}

\endbatchfile
%</install>
%<*ignore>
\fi
%</ignore>
%<*driver>
\NeedsTeXFormat{LaTeX2e}
\ProvidesFile{settobox.drv}%
  [2016/05/16 v1.5 Assign box dimensions to length registers (HO)]%
\documentclass{ltxdoc}
\usepackage{holtxdoc}[2011/11/22]
\usepackage{calc}
\usepackage{settobox}
\begin{document}
  \DocInput{settobox.dtx}%
\end{document}
%</driver>
% \fi
%
%
%
% \GetFileInfo{settobox.drv}
%
% \title{The \xpackage{settobox} package}
% \date{2016/05/16 v1.5}
% \author{Heiko Oberdiek\thanks
% {Please report any issues at \url{https://github.com/ho-tex/oberdiek/issues}}}
%
% \maketitle
%
% \begin{abstract}
% Commands are defined for getting box sizes similar
% to \LaTeX's \cs{settowidth} commands.
% \end{abstract}
%
% \tableofcontents
%
% \section{Usage}
%
% \subsection{Get box dimensions}
%
% \begin{declcs}^^A
%   {settoboxwidth}\,\M{\LaTeX\ length}\,\M{\LaTeX\ box}\\
%   \SpecialUsageIndex{\settoboxheight}^^A
%   \cs{settoboxheight}\,\M{\LaTeX\ length}\,\M{\LaTeX\ box}\\
%   \SpecialUsageIndex{\settoboxdepth}^^A
%   \cs{settoboxdepth}\,\M{\LaTeX\ length}\,\M{\LaTeX\ box}\\
%   \SpecialUsageIndex{\settoboxtotalheight}^^A
%   \cs{settoboxtotalheight}\,\M{\LaTeX\ length}\,\M{\LaTeX\ box}
% \end{declcs}
% A \meta{\LaTeX\ box} is allocated by \cs{newsavebox}.
% It can be filled by \cs{sbox} or the environment \texttt{lrbox}.
% The commands above extract then the desired lengths.
%
% \subsection{Set box dimensions}
%
% \begin{declcs}^^A
%   {setboxwidth}\,\M{\LaTeX\ box}\,\M{\LaTeX\ length expression}\\
%   \SpecialUsageIndex{\setboxheight}^^A
%   \cs{setboxheight}\,\M{\LaTeX\ box}\,\M{\LaTeX\ length expression}\\
%   \SpecialUsageIndex{\setboxdepth}^^A
%   \cs{setboxdepth}\,\M{\LaTeX\ box}\,\M{\LaTeX\ length expression}
% \end{declcs}
% These commands allow the manipulation of the box. Package \xpackage{calc}
% is supported in the \meta{\LaTeX\ length expression}.
% Also the following length are available in this expression:
% \begin{quote}
% \begin{tabular}{@{}ll@{}}
%   \cs{width}& width of the box\\
%   \cs{height}& height of the box\\
%   \cs{depth}& depth of the box\\
%   \cs{totalheight}& totalheight of the box\\
% \end{tabular}
% \end{quote}
% Note, the base point (point at the left margin of the baseline)
% always remain constant.
%
% \subsection{Move box}
%
% \begin{declcs}^^A
%   {setboxmoveleft}\,\M{\LaTeX\ box}\,\M{\LaTeX\ length expression}\\
%   \SpecialUsageIndex{\setboxmoveright}^^A
%   \cs{setboxmoveright}\,\M{\LaTeX\ box}\,\M{\LaTeX\ length expression}\\
%   \SpecialUsageIndex{\setboxlower}^^A
%   \cs{setboxlower}\,\M{\LaTeX\ box}\,\M{\LaTeX\ length expression}\\
%   \SpecialUsageIndex{\setboxright}^^A
%   \cs{setboxright}\,\M{\LaTeX\ box}\,\M{\LaTeX\ length expression}
% \end{declcs}
% Note, the box is shifted relative to the base point. The base point
% is always inside the box, however the width and height of the
% box change along with the movement.
%
% \subsection{Example}
%
% \subsubsection{Short example}
%
% \begin{quote}
%\begin{verbatim}
%\newsavebox{\mybox}
%\newlength{\mylength}
%\sbox{\mybox}{Hello World}
%\settoboxwidth{\mylength}{\mybox}
%\end{verbatim}
% \end{quote}
%
% \subsubsection{Test file that shows box manipulations}
%
%    \begin{macrocode}
%<*example>
%<<END
\documentclass{article}

\usepackage{settobox}
\usepackage{calc}

\newsavebox{\mybox}

\setlength{\fboxsep}{0pt}
\setlength{\parindent}{20pt}
\setlength{\parskip}{10pt}
\pagestyle{empty}

% \test{#1}
% The macro is called with commands in #1 that manipulates
% the box \mybox. These commands along with the result of
% the manipulation is shown. Thus the essence of the
% macro is:
%
%   a) \sbox{\mybox}{The cracy fox.}
%   b) #1 % manipulates \mybox
%   c) Print #1 commands.
%   d) Print box with frame
%
% The implemenation looks more weird:
\makeatletter
\newcommand*{\test}[1]{%
  \par
  \begingroup
    \raggedright
    \edef\x{\detokenize{#1}}%
    \let\do\@makeother
    \dospecials
    \catcode`\~\active
    \catcode`\ =10\relax
    \def~{\\}%
    \noindent
    \texttt{\scantokens\expandafter{\x}}%
    \par
  \endgroup
  \begingroup
    \let~\relax
    \sbox{\mybox}{The cracy fox.}%
     #1%
     A---\fbox{\usebox\mybox}---B%
  \endgroup
  \par
}
\makeatother

\begin{document}

\test{\setboxwidth{\mybox}{1.25\width}}
\test{\setboxheight{\mybox}{0pt}}
\test{\setboxheight{\mybox}{2\height}}
\test{\setboxdepth{\mybox}{\height}}
\test{\setboxmoveleft{\mybox}{5pt}}
\test{%
  \setboxmoveleft{\mybox}{5pt}~%
  \setboxwidth{\mybox}{\width + 5pt}%
}
\test{\setboxmoveright{\mybox}{0.5\width}}
\test{\setboxlower{\mybox}{\height}}
\test{\setboxraise{\mybox}{\depth}}
\test{%
  \setboxmoveright{\mybox}{5pt}~%
  \setboxwidth{\mybox}{\width + 5pt}~%
  \setboxheight{\mybox}{\height + 5pt}~%
  \setboxdepth{\mybox}{\depth + 5pt}%
}

\end{document}
%END
%</example>
%    \end{macrocode}
%
% \noindent
%    The result:
%
% \vspace{1ex}
% \hrule
%
% \begingroup
% \newsavebox{\mybox}
%
% \setlength{\fboxsep}{0pt}
% \setlength{\parindent}{20pt}
% \setlength{\parskip}{10pt}
%
% \makeatletter
% \newcommand*{\test}[1]{^^A
%   \par
%   \begingroup
%     \raggedright
%     \edef\x{\detokenize{#1}}
%     \let\do\@makeother
%     \dospecials
%     \catcode`\~\active
%     \catcode`\ =10\relax
%     \def~{\\}^^A
%     \noindent
%     \texttt{\scantokens\expandafter{\x}}
%     \par
%   \endgroup
%   \begingroup
%     \let~\relax
%     \sbox{\mybox}{The cracy fox.}
%      #1^^A
%      A---\fbox{\usebox\mybox}---B
%   \endgroup
%   \par
% }
% \makeatother
%
% \test{\setboxwidth{\mybox}{1.25\width}}
% \test{\setboxheight{\mybox}{0pt}}
% \test{\setboxheight{\mybox}{2\height}}
% \test{\setboxdepth{\mybox}{\height}}
% \test{\setboxmoveleft{\mybox}{5pt}}
% \test{^^A
%   \setboxmoveleft{\mybox}{5pt}~^^A
%   \setboxwidth{\mybox}{\width + 5pt}^^A
% }
% \test{\setboxmoveright{\mybox}{0.5\width}}
% \test{\setboxlower{\mybox}{\height}}
% \test{\setboxraise{\mybox}{\depth}}
% \test{^^A
%   \setboxmoveright{\mybox}{5pt}~^^A
%   \setboxwidth{\mybox}{\width + 5pt}~^^A
%   \setboxheight{\mybox}{\height + 5pt}~^^A
%   \setboxdepth{\mybox}{\depth + 5pt}^^A
% }
%
% \endgroup
% \vspace{1ex}
% \hrule
% \vspace{4ex}
%
% \StopEventually{
% }
%
% \section{Implementation}
%
%    \begin{macrocode}
%<*package>
%    \end{macrocode}
%    Package identification.
%    \begin{macrocode}
\NeedsTeXFormat{LaTeX2e}
\ProvidesPackage{settobox}%
  [2016/05/16 v1.5 Assign box dimensions to length registers (HO)]
%    \end{macrocode}
%
%    \begin{macrocode}
\newcommand*{\settoboxwidth}[2]{\setlength{#1}{\wd#2}}
\newcommand*{\settoboxheight}[2]{\setlength{#1}{\ht#2}}
\newcommand*{\settoboxdepth}[2]{\setlength{#1}{\dp#2}}
\newcommand*{\settoboxtotalheight}[2]{%
  \setlength{#1}{\ht#2}%
  \addtolength{#1}{\dp#2}%
}
%    \end{macrocode}
%
%    \begin{macro}{\setboxwidth}
%    \begin{macrocode}
\newcommand*{\setboxwidth}[2]{%
  \settobox@length\wd{#1}{#2}%
}
%    \end{macrocode}
%    \end{macro}
%    \begin{macro}{\setboxheight}
%    \begin{macrocode}
\newcommand*{\setboxheight}[2]{%
  \settobox@length\ht{#1}{#2}%
}
%    \end{macrocode}
%    \end{macro}
%    \begin{macro}{\setboxheight}
%    \begin{macrocode}
\newcommand*{\setboxdepth}[2]{%
  \settobox@length\dp{#1}{#2}%
}
%    \end{macrocode}
%    \end{macro}
%    \begin{macro}{\setboxmoveleft}
%    \begin{macrocode}
\newcommand*{\setboxmoveleft}[2]{%
  \settobox@horiz{-}{#1}{#2}%
}
%    \end{macrocode}
%    \end{macro}
%    \begin{macro}{\setboxmoveright}
%    \begin{macrocode}
\newcommand*{\setboxmoveright}[2]{%
  \settobox@horiz{}{#1}{#2}%
}
%    \end{macrocode}
%    \end{macro}
%    \begin{macro}{\setboxlower}
%    \begin{macrocode}
\newcommand*{\setboxlower}[2]{%
  \settobox@vert\lower{#1}{#2}%
}
%    \end{macrocode}
%    \end{macro}
%    \begin{macro}{\setboxraise}
%    \begin{macrocode}
\newcommand*{\setboxraise}[2]{%
  \settobox@vert\raise{#1}{#2}%
}
%    \end{macrocode}
%    \end{macro}
%    \begin{macro}{\settobox@length}
%    The work for the \cs{setbox...} commands is done by
%    \cs{settobox@length}. Inside the length expression
%    \cs{width}, \cs{height}, \cs{depth}, \cs{totalheight}
%    are set to the dimensions of the box.\\
%    \begin{tabular}{@{}ll@{}}
%    |#1|:& the property of the box that is to be changed
%           (\cs{wd}, \cs{ht}, \cs{dp})\\
%    |#2|:& the box\\
%    |#3|:& length expression
%    \end{tabular}
%    \begin{macrocode}
\def\settobox@length#1#2#3{%
  \settobox@calc{#2}{#3}{#1#2=##1sp\relax}%
}
%    \end{macrocode}
%    \end{macro}
%
%    \begin{macro}{\settobox@horiz}
%    \begin{macrocode}
\def\settobox@horiz#1#2#3{%
  \settobox@calc{#2}{#3}{\setbox#2=\hbox{\kern#1##1sp\copy#2}}%
}
%    \end{macrocode}
%    \end{macro}
%    \begin{macro}{\settobox@vert}
%    \begin{macrocode}
\def\settobox@vert#1#2#3{%
  \settobox@calc{#2}{#3}{\setbox#2=\hbox{#1##1sp\copy#2}}%
}
%    \end{macrocode}
%    \end{macro}
%
%    \begin{macro}{\settobox@calc}
%    \begin{macrocode}
\def\settobox@calc#1#2#3{%
  \begingroup
    \def\width{\wd#1}%
    \def\height{\ht#1}%
    \def\depth{\dp#1}%
    \dimen@\ht#1\relax
    \advance\dimen@\dp#1\relax
    \def\totalheight{\dimen@}%
    \setlength{\dimen@}{#2}%
    \count@\dimen@
    \def\x##1{\endgroup
      #3%
    }%
  \expandafter\x\expandafter{\the\count@}%
}
%    \end{macrocode}
%    \end{macro}
%
%    \begin{macrocode}
%</package>
%    \end{macrocode}
%
% \section{Installation}
%
% \subsection{Download}
%
% \paragraph{Package.} This package is available on
% CTAN\footnote{\CTANpkg{settobox}}:
% \begin{description}
% \item[\CTAN{macros/latex/contrib/oberdiek/settobox.dtx}] The source file.
% \item[\CTAN{macros/latex/contrib/oberdiek/settobox.pdf}] Documentation.
% \end{description}
%
%
% \paragraph{Bundle.} All the packages of the bundle `oberdiek'
% are also available in a TDS compliant ZIP archive. There
% the packages are already unpacked and the documentation files
% are generated. The files and directories obey the TDS standard.
% \begin{description}
% \item[\CTANinstall{install/macros/latex/contrib/oberdiek.tds.zip}]
% \end{description}
% \emph{TDS} refers to the standard ``A Directory Structure
% for \TeX\ Files'' (\CTANpkg{tds}). Directories
% with \xfile{texmf} in their name are usually organized this way.
%
% \subsection{Bundle installation}
%
% \paragraph{Unpacking.} Unpack the \xfile{oberdiek.tds.zip} in the
% TDS tree (also known as \xfile{texmf} tree) of your choice.
% Example (linux):
% \begin{quote}
%   |unzip oberdiek.tds.zip -d ~/texmf|
% \end{quote}
%
% \subsection{Package installation}
%
% \paragraph{Unpacking.} The \xfile{.dtx} file is a self-extracting
% \docstrip\ archive. The files are extracted by running the
% \xfile{.dtx} through \plainTeX:
% \begin{quote}
%   \verb|tex settobox.dtx|
% \end{quote}
%
% \paragraph{TDS.} Now the different files must be moved into
% the different directories in your installation TDS tree
% (also known as \xfile{texmf} tree):
% \begin{quote}
% \def\t{^^A
% \begin{tabular}{@{}>{\ttfamily}l@{ $\rightarrow$ }>{\ttfamily}l@{}}
%   settobox.sty & tex/latex/oberdiek/settobox.sty\\
%   settobox.pdf & doc/latex/oberdiek/settobox.pdf\\
%   settobox-example.tex & doc/latex/oberdiek/settobox-example.tex\\
%   settobox.dtx & source/latex/oberdiek/settobox.dtx\\
% \end{tabular}^^A
% }^^A
% \sbox0{\t}^^A
% \ifdim\wd0>\linewidth
%   \begingroup
%     \advance\linewidth by\leftmargin
%     \advance\linewidth by\rightmargin
%   \edef\x{\endgroup
%     \def\noexpand\lw{\the\linewidth}^^A
%   }\x
%   \def\lwbox{^^A
%     \leavevmode
%     \hbox to \linewidth{^^A
%       \kern-\leftmargin\relax
%       \hss
%       \usebox0
%       \hss
%       \kern-\rightmargin\relax
%     }^^A
%   }^^A
%   \ifdim\wd0>\lw
%     \sbox0{\small\t}^^A
%     \ifdim\wd0>\linewidth
%       \ifdim\wd0>\lw
%         \sbox0{\footnotesize\t}^^A
%         \ifdim\wd0>\linewidth
%           \ifdim\wd0>\lw
%             \sbox0{\scriptsize\t}^^A
%             \ifdim\wd0>\linewidth
%               \ifdim\wd0>\lw
%                 \sbox0{\tiny\t}^^A
%                 \ifdim\wd0>\linewidth
%                   \lwbox
%                 \else
%                   \usebox0
%                 \fi
%               \else
%                 \lwbox
%               \fi
%             \else
%               \usebox0
%             \fi
%           \else
%             \lwbox
%           \fi
%         \else
%           \usebox0
%         \fi
%       \else
%         \lwbox
%       \fi
%     \else
%       \usebox0
%     \fi
%   \else
%     \lwbox
%   \fi
% \else
%   \usebox0
% \fi
% \end{quote}
% If you have a \xfile{docstrip.cfg} that configures and enables \docstrip's
% TDS installing feature, then some files can already be in the right
% place, see the documentation of \docstrip.
%
% \subsection{Refresh file name databases}
%
% If your \TeX~distribution
% (\TeX\,Live, \mikTeX, \dots) relies on file name databases, you must refresh
% these. For example, \TeX\,Live\ users run \verb|texhash| or
% \verb|mktexlsr|.
%
% \subsection{Some details for the interested}
%
% \paragraph{Unpacking with \LaTeX.}
% The \xfile{.dtx} chooses its action depending on the format:
% \begin{description}
% \item[\plainTeX:] Run \docstrip\ and extract the files.
% \item[\LaTeX:] Generate the documentation.
% \end{description}
% If you insist on using \LaTeX\ for \docstrip\ (really,
% \docstrip\ does not need \LaTeX), then inform the autodetect routine
% about your intention:
% \begin{quote}
%   \verb|latex \let\install=y\input{settobox.dtx}|
% \end{quote}
% Do not forget to quote the argument according to the demands
% of your shell.
%
% \paragraph{Generating the documentation.}
% You can use both the \xfile{.dtx} or the \xfile{.drv} to generate
% the documentation. The process can be configured by the
% configuration file \xfile{ltxdoc.cfg}. For instance, put this
% line into this file, if you want to have A4 as paper format:
% \begin{quote}
%   \verb|\PassOptionsToClass{a4paper}{article}|
% \end{quote}
% An example follows how to generate the
% documentation with pdf\LaTeX:
% \begin{quote}
%\begin{verbatim}
%pdflatex settobox.dtx
%makeindex -s gind.ist settobox.idx
%pdflatex settobox.dtx
%makeindex -s gind.ist settobox.idx
%pdflatex settobox.dtx
%\end{verbatim}
% \end{quote}
%
% \begin{History}
%   \begin{Version}{2000/02/11 v1.0}
%   \item
%     First public release, written as answer in the
%     newsgroup \xnewsgroup{de.comp.text.tex}:
%     \URL{``\link{Die Hoehe von Minipages und Bild}''}^^A
%     {https://groups.google.com/group/de.comp.text.tex/msg/c3f6446f54f66c02}
%   \end{Version}
%   \begin{Version}{2000/09/07 v1.1}
%   \item
%     Documentation added.
%   \item
%     CTAN release.
%   \end{Version}
%   \begin{Version}{2006/02/20 v1.2}
%   \item
%     \cs{setboxwidth}, \cs{setboxheight}, \cs{setboxdepth} added.
%   \item
%     Box move commands added.
%   \item
%     DTX framework.
%   \item
%     LPPL 1.3
%   \end{Version}
%   \begin{Version}{2007/04/11 v1.3}
%   \item
%     Line ends sanitized.
%   \end{Version}
%   \begin{Version}{2008/08/11 v1.4}
%   \item
%     Code is not changed.
%   \item
%     URLs updated.
%   \end{Version}
%   \begin{Version}{2016/05/16 v1.5}
%   \item
%     Documentation updates.
%   \end{Version}
% \end{History}
%
% \PrintIndex
%
% \Finale
\endinput
|
% \end{quote}
% Do not forget to quote the argument according to the demands
% of your shell.
%
% \paragraph{Generating the documentation.}
% You can use both the \xfile{.dtx} or the \xfile{.drv} to generate
% the documentation. The process can be configured by the
% configuration file \xfile{ltxdoc.cfg}. For instance, put this
% line into this file, if you want to have A4 as paper format:
% \begin{quote}
%   \verb|\PassOptionsToClass{a4paper}{article}|
% \end{quote}
% An example follows how to generate the
% documentation with pdf\LaTeX:
% \begin{quote}
%\begin{verbatim}
%pdflatex settobox.dtx
%makeindex -s gind.ist settobox.idx
%pdflatex settobox.dtx
%makeindex -s gind.ist settobox.idx
%pdflatex settobox.dtx
%\end{verbatim}
% \end{quote}
%
% \begin{History}
%   \begin{Version}{2000/02/11 v1.0}
%   \item
%     First public release, written as answer in the
%     newsgroup \xnewsgroup{de.comp.text.tex}:
%     \URL{``\link{Die Hoehe von Minipages und Bild}''}^^A
%     {https://groups.google.com/group/de.comp.text.tex/msg/c3f6446f54f66c02}
%   \end{Version}
%   \begin{Version}{2000/09/07 v1.1}
%   \item
%     Documentation added.
%   \item
%     CTAN release.
%   \end{Version}
%   \begin{Version}{2006/02/20 v1.2}
%   \item
%     \cs{setboxwidth}, \cs{setboxheight}, \cs{setboxdepth} added.
%   \item
%     Box move commands added.
%   \item
%     DTX framework.
%   \item
%     LPPL 1.3
%   \end{Version}
%   \begin{Version}{2007/04/11 v1.3}
%   \item
%     Line ends sanitized.
%   \end{Version}
%   \begin{Version}{2008/08/11 v1.4}
%   \item
%     Code is not changed.
%   \item
%     URLs updated.
%   \end{Version}
%   \begin{Version}{2016/05/16 v1.5}
%   \item
%     Documentation updates.
%   \end{Version}
% \end{History}
%
% \PrintIndex
%
% \Finale
\endinput

%        (quote the arguments according to the demands of your shell)
%
% Documentation:
%    (a) If settobox.drv is present:
%           latex settobox.drv
%    (b) Without settobox.drv:
%           latex settobox.dtx; ...
%    The class ltxdoc loads the configuration file ltxdoc.cfg
%    if available. Here you can specify further options, e.g.
%    use A4 as paper format:
%       \PassOptionsToClass{a4paper}{article}
%
%    Programm calls to get the documentation (example):
%       pdflatex settobox.dtx
%       makeindex -s gind.ist settobox.idx
%       pdflatex settobox.dtx
%       makeindex -s gind.ist settobox.idx
%       pdflatex settobox.dtx
%
% Installation:
%    TDS:tex/latex/oberdiek/settobox.sty
%    TDS:doc/latex/oberdiek/settobox.pdf
%    TDS:doc/latex/oberdiek/settobox-example.tex
%    TDS:source/latex/oberdiek/settobox.dtx
%
%<*ignore>
\begingroup
  \catcode123=1 %
  \catcode125=2 %
  \def\x{LaTeX2e}%
\expandafter\endgroup
\ifcase 0\ifx\install y1\fi\expandafter
         \ifx\csname processbatchFile\endcsname\relax\else1\fi
         \ifx\fmtname\x\else 1\fi\relax
\else\csname fi\endcsname
%</ignore>
%<*install>
\input docstrip.tex
\Msg{************************************************************************}
\Msg{* Installation}
\Msg{* Package: settobox 2016/05/16 v1.5 Assign box dimensions to length registers (HO)}
\Msg{************************************************************************}

\keepsilent
\askforoverwritefalse

\let\MetaPrefix\relax
\preamble

This is a generated file.

Project: settobox
Version: 2016/05/16 v1.5

Copyright (C)
   2000, 2006-2008 Heiko Oberdiek
   2016-2019 Oberdiek Package Support Group

This work may be distributed and/or modified under the
conditions of the LaTeX Project Public License, either
version 1.3c of this license or (at your option) any later
version. This version of this license is in
   https://www.latex-project.org/lppl/lppl-1-3c.txt
and the latest version of this license is in
   https://www.latex-project.org/lppl.txt
and version 1.3 or later is part of all distributions of
LaTeX version 2005/12/01 or later.

This work has the LPPL maintenance status "maintained".

The Current Maintainers of this work are
Heiko Oberdiek and the Oberdiek Package Support Group
https://github.com/ho-tex/oberdiek/issues


This work consists of the main source file settobox.dtx
and the derived files
   settobox.sty, settobox.pdf, settobox.ins, settobox.drv,
   settobox-example.tex.

\endpreamble
\let\MetaPrefix\DoubleperCent

\generate{%
  \file{settobox.ins}{\from{settobox.dtx}{install}}%
  \file{settobox.drv}{\from{settobox.dtx}{driver}}%
  \usedir{tex/latex/oberdiek}%
  \file{settobox.sty}{\from{settobox.dtx}{package}}%
  \usedir{doc/latex/oberdiek}%
  \file{settobox-example.tex}{\from{settobox.dtx}{example}}%
}

\catcode32=13\relax% active space
\let =\space%
\Msg{************************************************************************}
\Msg{*}
\Msg{* To finish the installation you have to move the following}
\Msg{* file into a directory searched by TeX:}
\Msg{*}
\Msg{*     settobox.sty}
\Msg{*}
\Msg{* To produce the documentation run the file `settobox.drv'}
\Msg{* through LaTeX.}
\Msg{*}
\Msg{* Happy TeXing!}
\Msg{*}
\Msg{************************************************************************}

\endbatchfile
%</install>
%<*ignore>
\fi
%</ignore>
%<*driver>
\NeedsTeXFormat{LaTeX2e}
\ProvidesFile{settobox.drv}%
  [2016/05/16 v1.5 Assign box dimensions to length registers (HO)]%
\documentclass{ltxdoc}
\usepackage{holtxdoc}[2011/11/22]
\usepackage{calc}
\usepackage{settobox}
\begin{document}
  \DocInput{settobox.dtx}%
\end{document}
%</driver>
% \fi
%
%
%
% \GetFileInfo{settobox.drv}
%
% \title{The \xpackage{settobox} package}
% \date{2016/05/16 v1.5}
% \author{Heiko Oberdiek\thanks
% {Please report any issues at \url{https://github.com/ho-tex/oberdiek/issues}}}
%
% \maketitle
%
% \begin{abstract}
% Commands are defined for getting box sizes similar
% to \LaTeX's \cs{settowidth} commands.
% \end{abstract}
%
% \tableofcontents
%
% \section{Usage}
%
% \subsection{Get box dimensions}
%
% \begin{declcs}^^A
%   {settoboxwidth}\,\M{\LaTeX\ length}\,\M{\LaTeX\ box}\\
%   \SpecialUsageIndex{\settoboxheight}^^A
%   \cs{settoboxheight}\,\M{\LaTeX\ length}\,\M{\LaTeX\ box}\\
%   \SpecialUsageIndex{\settoboxdepth}^^A
%   \cs{settoboxdepth}\,\M{\LaTeX\ length}\,\M{\LaTeX\ box}\\
%   \SpecialUsageIndex{\settoboxtotalheight}^^A
%   \cs{settoboxtotalheight}\,\M{\LaTeX\ length}\,\M{\LaTeX\ box}
% \end{declcs}
% A \meta{\LaTeX\ box} is allocated by \cs{newsavebox}.
% It can be filled by \cs{sbox} or the environment \texttt{lrbox}.
% The commands above extract then the desired lengths.
%
% \subsection{Set box dimensions}
%
% \begin{declcs}^^A
%   {setboxwidth}\,\M{\LaTeX\ box}\,\M{\LaTeX\ length expression}\\
%   \SpecialUsageIndex{\setboxheight}^^A
%   \cs{setboxheight}\,\M{\LaTeX\ box}\,\M{\LaTeX\ length expression}\\
%   \SpecialUsageIndex{\setboxdepth}^^A
%   \cs{setboxdepth}\,\M{\LaTeX\ box}\,\M{\LaTeX\ length expression}
% \end{declcs}
% These commands allow the manipulation of the box. Package \xpackage{calc}
% is supported in the \meta{\LaTeX\ length expression}.
% Also the following length are available in this expression:
% \begin{quote}
% \begin{tabular}{@{}ll@{}}
%   \cs{width}& width of the box\\
%   \cs{height}& height of the box\\
%   \cs{depth}& depth of the box\\
%   \cs{totalheight}& totalheight of the box\\
% \end{tabular}
% \end{quote}
% Note, the base point (point at the left margin of the baseline)
% always remain constant.
%
% \subsection{Move box}
%
% \begin{declcs}^^A
%   {setboxmoveleft}\,\M{\LaTeX\ box}\,\M{\LaTeX\ length expression}\\
%   \SpecialUsageIndex{\setboxmoveright}^^A
%   \cs{setboxmoveright}\,\M{\LaTeX\ box}\,\M{\LaTeX\ length expression}\\
%   \SpecialUsageIndex{\setboxlower}^^A
%   \cs{setboxlower}\,\M{\LaTeX\ box}\,\M{\LaTeX\ length expression}\\
%   \SpecialUsageIndex{\setboxright}^^A
%   \cs{setboxright}\,\M{\LaTeX\ box}\,\M{\LaTeX\ length expression}
% \end{declcs}
% Note, the box is shifted relative to the base point. The base point
% is always inside the box, however the width and height of the
% box change along with the movement.
%
% \subsection{Example}
%
% \subsubsection{Short example}
%
% \begin{quote}
%\begin{verbatim}
%\newsavebox{\mybox}
%\newlength{\mylength}
%\sbox{\mybox}{Hello World}
%\settoboxwidth{\mylength}{\mybox}
%\end{verbatim}
% \end{quote}
%
% \subsubsection{Test file that shows box manipulations}
%
%    \begin{macrocode}
%<*example>
%<<END
\documentclass{article}

\usepackage{settobox}
\usepackage{calc}

\newsavebox{\mybox}

\setlength{\fboxsep}{0pt}
\setlength{\parindent}{20pt}
\setlength{\parskip}{10pt}
\pagestyle{empty}

% \test{#1}
% The macro is called with commands in #1 that manipulates
% the box \mybox. These commands along with the result of
% the manipulation is shown. Thus the essence of the
% macro is:
%
%   a) \sbox{\mybox}{The cracy fox.}
%   b) #1 % manipulates \mybox
%   c) Print #1 commands.
%   d) Print box with frame
%
% The implemenation looks more weird:
\makeatletter
\newcommand*{\test}[1]{%
  \par
  \begingroup
    \raggedright
    \edef\x{\detokenize{#1}}%
    \let\do\@makeother
    \dospecials
    \catcode`\~\active
    \catcode`\ =10\relax
    \def~{\\}%
    \noindent
    \texttt{\scantokens\expandafter{\x}}%
    \par
  \endgroup
  \begingroup
    \let~\relax
    \sbox{\mybox}{The cracy fox.}%
     #1%
     A---\fbox{\usebox\mybox}---B%
  \endgroup
  \par
}
\makeatother

\begin{document}

\test{\setboxwidth{\mybox}{1.25\width}}
\test{\setboxheight{\mybox}{0pt}}
\test{\setboxheight{\mybox}{2\height}}
\test{\setboxdepth{\mybox}{\height}}
\test{\setboxmoveleft{\mybox}{5pt}}
\test{%
  \setboxmoveleft{\mybox}{5pt}~%
  \setboxwidth{\mybox}{\width + 5pt}%
}
\test{\setboxmoveright{\mybox}{0.5\width}}
\test{\setboxlower{\mybox}{\height}}
\test{\setboxraise{\mybox}{\depth}}
\test{%
  \setboxmoveright{\mybox}{5pt}~%
  \setboxwidth{\mybox}{\width + 5pt}~%
  \setboxheight{\mybox}{\height + 5pt}~%
  \setboxdepth{\mybox}{\depth + 5pt}%
}

\end{document}
%END
%</example>
%    \end{macrocode}
%
% \noindent
%    The result:
%
% \vspace{1ex}
% \hrule
%
% \begingroup
% \newsavebox{\mybox}
%
% \setlength{\fboxsep}{0pt}
% \setlength{\parindent}{20pt}
% \setlength{\parskip}{10pt}
%
% \makeatletter
% \newcommand*{\test}[1]{^^A
%   \par
%   \begingroup
%     \raggedright
%     \edef\x{\detokenize{#1}}
%     \let\do\@makeother
%     \dospecials
%     \catcode`\~\active
%     \catcode`\ =10\relax
%     \def~{\\}^^A
%     \noindent
%     \texttt{\scantokens\expandafter{\x}}
%     \par
%   \endgroup
%   \begingroup
%     \let~\relax
%     \sbox{\mybox}{The cracy fox.}
%      #1^^A
%      A---\fbox{\usebox\mybox}---B
%   \endgroup
%   \par
% }
% \makeatother
%
% \test{\setboxwidth{\mybox}{1.25\width}}
% \test{\setboxheight{\mybox}{0pt}}
% \test{\setboxheight{\mybox}{2\height}}
% \test{\setboxdepth{\mybox}{\height}}
% \test{\setboxmoveleft{\mybox}{5pt}}
% \test{^^A
%   \setboxmoveleft{\mybox}{5pt}~^^A
%   \setboxwidth{\mybox}{\width + 5pt}^^A
% }
% \test{\setboxmoveright{\mybox}{0.5\width}}
% \test{\setboxlower{\mybox}{\height}}
% \test{\setboxraise{\mybox}{\depth}}
% \test{^^A
%   \setboxmoveright{\mybox}{5pt}~^^A
%   \setboxwidth{\mybox}{\width + 5pt}~^^A
%   \setboxheight{\mybox}{\height + 5pt}~^^A
%   \setboxdepth{\mybox}{\depth + 5pt}^^A
% }
%
% \endgroup
% \vspace{1ex}
% \hrule
% \vspace{4ex}
%
% \StopEventually{
% }
%
% \section{Implementation}
%
%    \begin{macrocode}
%<*package>
%    \end{macrocode}
%    Package identification.
%    \begin{macrocode}
\NeedsTeXFormat{LaTeX2e}
\ProvidesPackage{settobox}%
  [2016/05/16 v1.5 Assign box dimensions to length registers (HO)]
%    \end{macrocode}
%
%    \begin{macrocode}
\newcommand*{\settoboxwidth}[2]{\setlength{#1}{\wd#2}}
\newcommand*{\settoboxheight}[2]{\setlength{#1}{\ht#2}}
\newcommand*{\settoboxdepth}[2]{\setlength{#1}{\dp#2}}
\newcommand*{\settoboxtotalheight}[2]{%
  \setlength{#1}{\ht#2}%
  \addtolength{#1}{\dp#2}%
}
%    \end{macrocode}
%
%    \begin{macro}{\setboxwidth}
%    \begin{macrocode}
\newcommand*{\setboxwidth}[2]{%
  \settobox@length\wd{#1}{#2}%
}
%    \end{macrocode}
%    \end{macro}
%    \begin{macro}{\setboxheight}
%    \begin{macrocode}
\newcommand*{\setboxheight}[2]{%
  \settobox@length\ht{#1}{#2}%
}
%    \end{macrocode}
%    \end{macro}
%    \begin{macro}{\setboxheight}
%    \begin{macrocode}
\newcommand*{\setboxdepth}[2]{%
  \settobox@length\dp{#1}{#2}%
}
%    \end{macrocode}
%    \end{macro}
%    \begin{macro}{\setboxmoveleft}
%    \begin{macrocode}
\newcommand*{\setboxmoveleft}[2]{%
  \settobox@horiz{-}{#1}{#2}%
}
%    \end{macrocode}
%    \end{macro}
%    \begin{macro}{\setboxmoveright}
%    \begin{macrocode}
\newcommand*{\setboxmoveright}[2]{%
  \settobox@horiz{}{#1}{#2}%
}
%    \end{macrocode}
%    \end{macro}
%    \begin{macro}{\setboxlower}
%    \begin{macrocode}
\newcommand*{\setboxlower}[2]{%
  \settobox@vert\lower{#1}{#2}%
}
%    \end{macrocode}
%    \end{macro}
%    \begin{macro}{\setboxraise}
%    \begin{macrocode}
\newcommand*{\setboxraise}[2]{%
  \settobox@vert\raise{#1}{#2}%
}
%    \end{macrocode}
%    \end{macro}
%    \begin{macro}{\settobox@length}
%    The work for the \cs{setbox...} commands is done by
%    \cs{settobox@length}. Inside the length expression
%    \cs{width}, \cs{height}, \cs{depth}, \cs{totalheight}
%    are set to the dimensions of the box.\\
%    \begin{tabular}{@{}ll@{}}
%    |#1|:& the property of the box that is to be changed
%           (\cs{wd}, \cs{ht}, \cs{dp})\\
%    |#2|:& the box\\
%    |#3|:& length expression
%    \end{tabular}
%    \begin{macrocode}
\def\settobox@length#1#2#3{%
  \settobox@calc{#2}{#3}{#1#2=##1sp\relax}%
}
%    \end{macrocode}
%    \end{macro}
%
%    \begin{macro}{\settobox@horiz}
%    \begin{macrocode}
\def\settobox@horiz#1#2#3{%
  \settobox@calc{#2}{#3}{\setbox#2=\hbox{\kern#1##1sp\copy#2}}%
}
%    \end{macrocode}
%    \end{macro}
%    \begin{macro}{\settobox@vert}
%    \begin{macrocode}
\def\settobox@vert#1#2#3{%
  \settobox@calc{#2}{#3}{\setbox#2=\hbox{#1##1sp\copy#2}}%
}
%    \end{macrocode}
%    \end{macro}
%
%    \begin{macro}{\settobox@calc}
%    \begin{macrocode}
\def\settobox@calc#1#2#3{%
  \begingroup
    \def\width{\wd#1}%
    \def\height{\ht#1}%
    \def\depth{\dp#1}%
    \dimen@\ht#1\relax
    \advance\dimen@\dp#1\relax
    \def\totalheight{\dimen@}%
    \setlength{\dimen@}{#2}%
    \count@\dimen@
    \def\x##1{\endgroup
      #3%
    }%
  \expandafter\x\expandafter{\the\count@}%
}
%    \end{macrocode}
%    \end{macro}
%
%    \begin{macrocode}
%</package>
%    \end{macrocode}
%
% \section{Installation}
%
% \subsection{Download}
%
% \paragraph{Package.} This package is available on
% CTAN\footnote{\CTANpkg{settobox}}:
% \begin{description}
% \item[\CTAN{macros/latex/contrib/oberdiek/settobox.dtx}] The source file.
% \item[\CTAN{macros/latex/contrib/oberdiek/settobox.pdf}] Documentation.
% \end{description}
%
%
% \paragraph{Bundle.} All the packages of the bundle `oberdiek'
% are also available in a TDS compliant ZIP archive. There
% the packages are already unpacked and the documentation files
% are generated. The files and directories obey the TDS standard.
% \begin{description}
% \item[\CTANinstall{install/macros/latex/contrib/oberdiek.tds.zip}]
% \end{description}
% \emph{TDS} refers to the standard ``A Directory Structure
% for \TeX\ Files'' (\CTANpkg{tds}). Directories
% with \xfile{texmf} in their name are usually organized this way.
%
% \subsection{Bundle installation}
%
% \paragraph{Unpacking.} Unpack the \xfile{oberdiek.tds.zip} in the
% TDS tree (also known as \xfile{texmf} tree) of your choice.
% Example (linux):
% \begin{quote}
%   |unzip oberdiek.tds.zip -d ~/texmf|
% \end{quote}
%
% \subsection{Package installation}
%
% \paragraph{Unpacking.} The \xfile{.dtx} file is a self-extracting
% \docstrip\ archive. The files are extracted by running the
% \xfile{.dtx} through \plainTeX:
% \begin{quote}
%   \verb|tex settobox.dtx|
% \end{quote}
%
% \paragraph{TDS.} Now the different files must be moved into
% the different directories in your installation TDS tree
% (also known as \xfile{texmf} tree):
% \begin{quote}
% \def\t{^^A
% \begin{tabular}{@{}>{\ttfamily}l@{ $\rightarrow$ }>{\ttfamily}l@{}}
%   settobox.sty & tex/latex/oberdiek/settobox.sty\\
%   settobox.pdf & doc/latex/oberdiek/settobox.pdf\\
%   settobox-example.tex & doc/latex/oberdiek/settobox-example.tex\\
%   settobox.dtx & source/latex/oberdiek/settobox.dtx\\
% \end{tabular}^^A
% }^^A
% \sbox0{\t}^^A
% \ifdim\wd0>\linewidth
%   \begingroup
%     \advance\linewidth by\leftmargin
%     \advance\linewidth by\rightmargin
%   \edef\x{\endgroup
%     \def\noexpand\lw{\the\linewidth}^^A
%   }\x
%   \def\lwbox{^^A
%     \leavevmode
%     \hbox to \linewidth{^^A
%       \kern-\leftmargin\relax
%       \hss
%       \usebox0
%       \hss
%       \kern-\rightmargin\relax
%     }^^A
%   }^^A
%   \ifdim\wd0>\lw
%     \sbox0{\small\t}^^A
%     \ifdim\wd0>\linewidth
%       \ifdim\wd0>\lw
%         \sbox0{\footnotesize\t}^^A
%         \ifdim\wd0>\linewidth
%           \ifdim\wd0>\lw
%             \sbox0{\scriptsize\t}^^A
%             \ifdim\wd0>\linewidth
%               \ifdim\wd0>\lw
%                 \sbox0{\tiny\t}^^A
%                 \ifdim\wd0>\linewidth
%                   \lwbox
%                 \else
%                   \usebox0
%                 \fi
%               \else
%                 \lwbox
%               \fi
%             \else
%               \usebox0
%             \fi
%           \else
%             \lwbox
%           \fi
%         \else
%           \usebox0
%         \fi
%       \else
%         \lwbox
%       \fi
%     \else
%       \usebox0
%     \fi
%   \else
%     \lwbox
%   \fi
% \else
%   \usebox0
% \fi
% \end{quote}
% If you have a \xfile{docstrip.cfg} that configures and enables \docstrip's
% TDS installing feature, then some files can already be in the right
% place, see the documentation of \docstrip.
%
% \subsection{Refresh file name databases}
%
% If your \TeX~distribution
% (\TeX\,Live, \mikTeX, \dots) relies on file name databases, you must refresh
% these. For example, \TeX\,Live\ users run \verb|texhash| or
% \verb|mktexlsr|.
%
% \subsection{Some details for the interested}
%
% \paragraph{Unpacking with \LaTeX.}
% The \xfile{.dtx} chooses its action depending on the format:
% \begin{description}
% \item[\plainTeX:] Run \docstrip\ and extract the files.
% \item[\LaTeX:] Generate the documentation.
% \end{description}
% If you insist on using \LaTeX\ for \docstrip\ (really,
% \docstrip\ does not need \LaTeX), then inform the autodetect routine
% about your intention:
% \begin{quote}
%   \verb|latex \let\install=y% \iffalse meta-comment
%
% File: settobox.dtx
% Version: 2016/05/16 v1.5
% Info: Assign box dimensions to length registers
%
% Copyright (C)
%    2000, 2006-2008 Heiko Oberdiek
%    2016-2019 Oberdiek Package Support Group
%    https://github.com/ho-tex/oberdiek/issues
%
% This work may be distributed and/or modified under the
% conditions of the LaTeX Project Public License, either
% version 1.3c of this license or (at your option) any later
% version. This version of this license is in
%    https://www.latex-project.org/lppl/lppl-1-3c.txt
% and the latest version of this license is in
%    https://www.latex-project.org/lppl.txt
% and version 1.3 or later is part of all distributions of
% LaTeX version 2005/12/01 or later.
%
% This work has the LPPL maintenance status "maintained".
%
% The Current Maintainers of this work are
% Heiko Oberdiek and the Oberdiek Package Support Group
% https://github.com/ho-tex/oberdiek/issues
%
% This work consists of the main source file settobox.dtx
% and the derived files
%    settobox.sty, settobox.pdf, settobox.ins, settobox.drv,
%    settobox-example.tex.
%
% Distribution:
%    CTAN:macros/latex/contrib/oberdiek/settobox.dtx
%    CTAN:macros/latex/contrib/oberdiek/settobox.pdf
%
% Unpacking:
%    (a) If settobox.ins is present:
%           tex settobox.ins
%    (b) Without settobox.ins:
%           tex settobox.dtx
%    (c) If you insist on using LaTeX
%           latex \let\install=y% \iffalse meta-comment
%
% File: settobox.dtx
% Version: 2016/05/16 v1.5
% Info: Assign box dimensions to length registers
%
% Copyright (C)
%    2000, 2006-2008 Heiko Oberdiek
%    2016-2019 Oberdiek Package Support Group
%    https://github.com/ho-tex/oberdiek/issues
%
% This work may be distributed and/or modified under the
% conditions of the LaTeX Project Public License, either
% version 1.3c of this license or (at your option) any later
% version. This version of this license is in
%    https://www.latex-project.org/lppl/lppl-1-3c.txt
% and the latest version of this license is in
%    https://www.latex-project.org/lppl.txt
% and version 1.3 or later is part of all distributions of
% LaTeX version 2005/12/01 or later.
%
% This work has the LPPL maintenance status "maintained".
%
% The Current Maintainers of this work are
% Heiko Oberdiek and the Oberdiek Package Support Group
% https://github.com/ho-tex/oberdiek/issues
%
% This work consists of the main source file settobox.dtx
% and the derived files
%    settobox.sty, settobox.pdf, settobox.ins, settobox.drv,
%    settobox-example.tex.
%
% Distribution:
%    CTAN:macros/latex/contrib/oberdiek/settobox.dtx
%    CTAN:macros/latex/contrib/oberdiek/settobox.pdf
%
% Unpacking:
%    (a) If settobox.ins is present:
%           tex settobox.ins
%    (b) Without settobox.ins:
%           tex settobox.dtx
%    (c) If you insist on using LaTeX
%           latex \let\install=y\input{settobox.dtx}
%        (quote the arguments according to the demands of your shell)
%
% Documentation:
%    (a) If settobox.drv is present:
%           latex settobox.drv
%    (b) Without settobox.drv:
%           latex settobox.dtx; ...
%    The class ltxdoc loads the configuration file ltxdoc.cfg
%    if available. Here you can specify further options, e.g.
%    use A4 as paper format:
%       \PassOptionsToClass{a4paper}{article}
%
%    Programm calls to get the documentation (example):
%       pdflatex settobox.dtx
%       makeindex -s gind.ist settobox.idx
%       pdflatex settobox.dtx
%       makeindex -s gind.ist settobox.idx
%       pdflatex settobox.dtx
%
% Installation:
%    TDS:tex/latex/oberdiek/settobox.sty
%    TDS:doc/latex/oberdiek/settobox.pdf
%    TDS:doc/latex/oberdiek/settobox-example.tex
%    TDS:source/latex/oberdiek/settobox.dtx
%
%<*ignore>
\begingroup
  \catcode123=1 %
  \catcode125=2 %
  \def\x{LaTeX2e}%
\expandafter\endgroup
\ifcase 0\ifx\install y1\fi\expandafter
         \ifx\csname processbatchFile\endcsname\relax\else1\fi
         \ifx\fmtname\x\else 1\fi\relax
\else\csname fi\endcsname
%</ignore>
%<*install>
\input docstrip.tex
\Msg{************************************************************************}
\Msg{* Installation}
\Msg{* Package: settobox 2016/05/16 v1.5 Assign box dimensions to length registers (HO)}
\Msg{************************************************************************}

\keepsilent
\askforoverwritefalse

\let\MetaPrefix\relax
\preamble

This is a generated file.

Project: settobox
Version: 2016/05/16 v1.5

Copyright (C)
   2000, 2006-2008 Heiko Oberdiek
   2016-2019 Oberdiek Package Support Group

This work may be distributed and/or modified under the
conditions of the LaTeX Project Public License, either
version 1.3c of this license or (at your option) any later
version. This version of this license is in
   https://www.latex-project.org/lppl/lppl-1-3c.txt
and the latest version of this license is in
   https://www.latex-project.org/lppl.txt
and version 1.3 or later is part of all distributions of
LaTeX version 2005/12/01 or later.

This work has the LPPL maintenance status "maintained".

The Current Maintainers of this work are
Heiko Oberdiek and the Oberdiek Package Support Group
https://github.com/ho-tex/oberdiek/issues


This work consists of the main source file settobox.dtx
and the derived files
   settobox.sty, settobox.pdf, settobox.ins, settobox.drv,
   settobox-example.tex.

\endpreamble
\let\MetaPrefix\DoubleperCent

\generate{%
  \file{settobox.ins}{\from{settobox.dtx}{install}}%
  \file{settobox.drv}{\from{settobox.dtx}{driver}}%
  \usedir{tex/latex/oberdiek}%
  \file{settobox.sty}{\from{settobox.dtx}{package}}%
  \usedir{doc/latex/oberdiek}%
  \file{settobox-example.tex}{\from{settobox.dtx}{example}}%
}

\catcode32=13\relax% active space
\let =\space%
\Msg{************************************************************************}
\Msg{*}
\Msg{* To finish the installation you have to move the following}
\Msg{* file into a directory searched by TeX:}
\Msg{*}
\Msg{*     settobox.sty}
\Msg{*}
\Msg{* To produce the documentation run the file `settobox.drv'}
\Msg{* through LaTeX.}
\Msg{*}
\Msg{* Happy TeXing!}
\Msg{*}
\Msg{************************************************************************}

\endbatchfile
%</install>
%<*ignore>
\fi
%</ignore>
%<*driver>
\NeedsTeXFormat{LaTeX2e}
\ProvidesFile{settobox.drv}%
  [2016/05/16 v1.5 Assign box dimensions to length registers (HO)]%
\documentclass{ltxdoc}
\usepackage{holtxdoc}[2011/11/22]
\usepackage{calc}
\usepackage{settobox}
\begin{document}
  \DocInput{settobox.dtx}%
\end{document}
%</driver>
% \fi
%
%
%
% \GetFileInfo{settobox.drv}
%
% \title{The \xpackage{settobox} package}
% \date{2016/05/16 v1.5}
% \author{Heiko Oberdiek\thanks
% {Please report any issues at \url{https://github.com/ho-tex/oberdiek/issues}}}
%
% \maketitle
%
% \begin{abstract}
% Commands are defined for getting box sizes similar
% to \LaTeX's \cs{settowidth} commands.
% \end{abstract}
%
% \tableofcontents
%
% \section{Usage}
%
% \subsection{Get box dimensions}
%
% \begin{declcs}^^A
%   {settoboxwidth}\,\M{\LaTeX\ length}\,\M{\LaTeX\ box}\\
%   \SpecialUsageIndex{\settoboxheight}^^A
%   \cs{settoboxheight}\,\M{\LaTeX\ length}\,\M{\LaTeX\ box}\\
%   \SpecialUsageIndex{\settoboxdepth}^^A
%   \cs{settoboxdepth}\,\M{\LaTeX\ length}\,\M{\LaTeX\ box}\\
%   \SpecialUsageIndex{\settoboxtotalheight}^^A
%   \cs{settoboxtotalheight}\,\M{\LaTeX\ length}\,\M{\LaTeX\ box}
% \end{declcs}
% A \meta{\LaTeX\ box} is allocated by \cs{newsavebox}.
% It can be filled by \cs{sbox} or the environment \texttt{lrbox}.
% The commands above extract then the desired lengths.
%
% \subsection{Set box dimensions}
%
% \begin{declcs}^^A
%   {setboxwidth}\,\M{\LaTeX\ box}\,\M{\LaTeX\ length expression}\\
%   \SpecialUsageIndex{\setboxheight}^^A
%   \cs{setboxheight}\,\M{\LaTeX\ box}\,\M{\LaTeX\ length expression}\\
%   \SpecialUsageIndex{\setboxdepth}^^A
%   \cs{setboxdepth}\,\M{\LaTeX\ box}\,\M{\LaTeX\ length expression}
% \end{declcs}
% These commands allow the manipulation of the box. Package \xpackage{calc}
% is supported in the \meta{\LaTeX\ length expression}.
% Also the following length are available in this expression:
% \begin{quote}
% \begin{tabular}{@{}ll@{}}
%   \cs{width}& width of the box\\
%   \cs{height}& height of the box\\
%   \cs{depth}& depth of the box\\
%   \cs{totalheight}& totalheight of the box\\
% \end{tabular}
% \end{quote}
% Note, the base point (point at the left margin of the baseline)
% always remain constant.
%
% \subsection{Move box}
%
% \begin{declcs}^^A
%   {setboxmoveleft}\,\M{\LaTeX\ box}\,\M{\LaTeX\ length expression}\\
%   \SpecialUsageIndex{\setboxmoveright}^^A
%   \cs{setboxmoveright}\,\M{\LaTeX\ box}\,\M{\LaTeX\ length expression}\\
%   \SpecialUsageIndex{\setboxlower}^^A
%   \cs{setboxlower}\,\M{\LaTeX\ box}\,\M{\LaTeX\ length expression}\\
%   \SpecialUsageIndex{\setboxright}^^A
%   \cs{setboxright}\,\M{\LaTeX\ box}\,\M{\LaTeX\ length expression}
% \end{declcs}
% Note, the box is shifted relative to the base point. The base point
% is always inside the box, however the width and height of the
% box change along with the movement.
%
% \subsection{Example}
%
% \subsubsection{Short example}
%
% \begin{quote}
%\begin{verbatim}
%\newsavebox{\mybox}
%\newlength{\mylength}
%\sbox{\mybox}{Hello World}
%\settoboxwidth{\mylength}{\mybox}
%\end{verbatim}
% \end{quote}
%
% \subsubsection{Test file that shows box manipulations}
%
%    \begin{macrocode}
%<*example>
%<<END
\documentclass{article}

\usepackage{settobox}
\usepackage{calc}

\newsavebox{\mybox}

\setlength{\fboxsep}{0pt}
\setlength{\parindent}{20pt}
\setlength{\parskip}{10pt}
\pagestyle{empty}

% \test{#1}
% The macro is called with commands in #1 that manipulates
% the box \mybox. These commands along with the result of
% the manipulation is shown. Thus the essence of the
% macro is:
%
%   a) \sbox{\mybox}{The cracy fox.}
%   b) #1 % manipulates \mybox
%   c) Print #1 commands.
%   d) Print box with frame
%
% The implemenation looks more weird:
\makeatletter
\newcommand*{\test}[1]{%
  \par
  \begingroup
    \raggedright
    \edef\x{\detokenize{#1}}%
    \let\do\@makeother
    \dospecials
    \catcode`\~\active
    \catcode`\ =10\relax
    \def~{\\}%
    \noindent
    \texttt{\scantokens\expandafter{\x}}%
    \par
  \endgroup
  \begingroup
    \let~\relax
    \sbox{\mybox}{The cracy fox.}%
     #1%
     A---\fbox{\usebox\mybox}---B%
  \endgroup
  \par
}
\makeatother

\begin{document}

\test{\setboxwidth{\mybox}{1.25\width}}
\test{\setboxheight{\mybox}{0pt}}
\test{\setboxheight{\mybox}{2\height}}
\test{\setboxdepth{\mybox}{\height}}
\test{\setboxmoveleft{\mybox}{5pt}}
\test{%
  \setboxmoveleft{\mybox}{5pt}~%
  \setboxwidth{\mybox}{\width + 5pt}%
}
\test{\setboxmoveright{\mybox}{0.5\width}}
\test{\setboxlower{\mybox}{\height}}
\test{\setboxraise{\mybox}{\depth}}
\test{%
  \setboxmoveright{\mybox}{5pt}~%
  \setboxwidth{\mybox}{\width + 5pt}~%
  \setboxheight{\mybox}{\height + 5pt}~%
  \setboxdepth{\mybox}{\depth + 5pt}%
}

\end{document}
%END
%</example>
%    \end{macrocode}
%
% \noindent
%    The result:
%
% \vspace{1ex}
% \hrule
%
% \begingroup
% \newsavebox{\mybox}
%
% \setlength{\fboxsep}{0pt}
% \setlength{\parindent}{20pt}
% \setlength{\parskip}{10pt}
%
% \makeatletter
% \newcommand*{\test}[1]{^^A
%   \par
%   \begingroup
%     \raggedright
%     \edef\x{\detokenize{#1}}
%     \let\do\@makeother
%     \dospecials
%     \catcode`\~\active
%     \catcode`\ =10\relax
%     \def~{\\}^^A
%     \noindent
%     \texttt{\scantokens\expandafter{\x}}
%     \par
%   \endgroup
%   \begingroup
%     \let~\relax
%     \sbox{\mybox}{The cracy fox.}
%      #1^^A
%      A---\fbox{\usebox\mybox}---B
%   \endgroup
%   \par
% }
% \makeatother
%
% \test{\setboxwidth{\mybox}{1.25\width}}
% \test{\setboxheight{\mybox}{0pt}}
% \test{\setboxheight{\mybox}{2\height}}
% \test{\setboxdepth{\mybox}{\height}}
% \test{\setboxmoveleft{\mybox}{5pt}}
% \test{^^A
%   \setboxmoveleft{\mybox}{5pt}~^^A
%   \setboxwidth{\mybox}{\width + 5pt}^^A
% }
% \test{\setboxmoveright{\mybox}{0.5\width}}
% \test{\setboxlower{\mybox}{\height}}
% \test{\setboxraise{\mybox}{\depth}}
% \test{^^A
%   \setboxmoveright{\mybox}{5pt}~^^A
%   \setboxwidth{\mybox}{\width + 5pt}~^^A
%   \setboxheight{\mybox}{\height + 5pt}~^^A
%   \setboxdepth{\mybox}{\depth + 5pt}^^A
% }
%
% \endgroup
% \vspace{1ex}
% \hrule
% \vspace{4ex}
%
% \StopEventually{
% }
%
% \section{Implementation}
%
%    \begin{macrocode}
%<*package>
%    \end{macrocode}
%    Package identification.
%    \begin{macrocode}
\NeedsTeXFormat{LaTeX2e}
\ProvidesPackage{settobox}%
  [2016/05/16 v1.5 Assign box dimensions to length registers (HO)]
%    \end{macrocode}
%
%    \begin{macrocode}
\newcommand*{\settoboxwidth}[2]{\setlength{#1}{\wd#2}}
\newcommand*{\settoboxheight}[2]{\setlength{#1}{\ht#2}}
\newcommand*{\settoboxdepth}[2]{\setlength{#1}{\dp#2}}
\newcommand*{\settoboxtotalheight}[2]{%
  \setlength{#1}{\ht#2}%
  \addtolength{#1}{\dp#2}%
}
%    \end{macrocode}
%
%    \begin{macro}{\setboxwidth}
%    \begin{macrocode}
\newcommand*{\setboxwidth}[2]{%
  \settobox@length\wd{#1}{#2}%
}
%    \end{macrocode}
%    \end{macro}
%    \begin{macro}{\setboxheight}
%    \begin{macrocode}
\newcommand*{\setboxheight}[2]{%
  \settobox@length\ht{#1}{#2}%
}
%    \end{macrocode}
%    \end{macro}
%    \begin{macro}{\setboxheight}
%    \begin{macrocode}
\newcommand*{\setboxdepth}[2]{%
  \settobox@length\dp{#1}{#2}%
}
%    \end{macrocode}
%    \end{macro}
%    \begin{macro}{\setboxmoveleft}
%    \begin{macrocode}
\newcommand*{\setboxmoveleft}[2]{%
  \settobox@horiz{-}{#1}{#2}%
}
%    \end{macrocode}
%    \end{macro}
%    \begin{macro}{\setboxmoveright}
%    \begin{macrocode}
\newcommand*{\setboxmoveright}[2]{%
  \settobox@horiz{}{#1}{#2}%
}
%    \end{macrocode}
%    \end{macro}
%    \begin{macro}{\setboxlower}
%    \begin{macrocode}
\newcommand*{\setboxlower}[2]{%
  \settobox@vert\lower{#1}{#2}%
}
%    \end{macrocode}
%    \end{macro}
%    \begin{macro}{\setboxraise}
%    \begin{macrocode}
\newcommand*{\setboxraise}[2]{%
  \settobox@vert\raise{#1}{#2}%
}
%    \end{macrocode}
%    \end{macro}
%    \begin{macro}{\settobox@length}
%    The work for the \cs{setbox...} commands is done by
%    \cs{settobox@length}. Inside the length expression
%    \cs{width}, \cs{height}, \cs{depth}, \cs{totalheight}
%    are set to the dimensions of the box.\\
%    \begin{tabular}{@{}ll@{}}
%    |#1|:& the property of the box that is to be changed
%           (\cs{wd}, \cs{ht}, \cs{dp})\\
%    |#2|:& the box\\
%    |#3|:& length expression
%    \end{tabular}
%    \begin{macrocode}
\def\settobox@length#1#2#3{%
  \settobox@calc{#2}{#3}{#1#2=##1sp\relax}%
}
%    \end{macrocode}
%    \end{macro}
%
%    \begin{macro}{\settobox@horiz}
%    \begin{macrocode}
\def\settobox@horiz#1#2#3{%
  \settobox@calc{#2}{#3}{\setbox#2=\hbox{\kern#1##1sp\copy#2}}%
}
%    \end{macrocode}
%    \end{macro}
%    \begin{macro}{\settobox@vert}
%    \begin{macrocode}
\def\settobox@vert#1#2#3{%
  \settobox@calc{#2}{#3}{\setbox#2=\hbox{#1##1sp\copy#2}}%
}
%    \end{macrocode}
%    \end{macro}
%
%    \begin{macro}{\settobox@calc}
%    \begin{macrocode}
\def\settobox@calc#1#2#3{%
  \begingroup
    \def\width{\wd#1}%
    \def\height{\ht#1}%
    \def\depth{\dp#1}%
    \dimen@\ht#1\relax
    \advance\dimen@\dp#1\relax
    \def\totalheight{\dimen@}%
    \setlength{\dimen@}{#2}%
    \count@\dimen@
    \def\x##1{\endgroup
      #3%
    }%
  \expandafter\x\expandafter{\the\count@}%
}
%    \end{macrocode}
%    \end{macro}
%
%    \begin{macrocode}
%</package>
%    \end{macrocode}
%
% \section{Installation}
%
% \subsection{Download}
%
% \paragraph{Package.} This package is available on
% CTAN\footnote{\CTANpkg{settobox}}:
% \begin{description}
% \item[\CTAN{macros/latex/contrib/oberdiek/settobox.dtx}] The source file.
% \item[\CTAN{macros/latex/contrib/oberdiek/settobox.pdf}] Documentation.
% \end{description}
%
%
% \paragraph{Bundle.} All the packages of the bundle `oberdiek'
% are also available in a TDS compliant ZIP archive. There
% the packages are already unpacked and the documentation files
% are generated. The files and directories obey the TDS standard.
% \begin{description}
% \item[\CTANinstall{install/macros/latex/contrib/oberdiek.tds.zip}]
% \end{description}
% \emph{TDS} refers to the standard ``A Directory Structure
% for \TeX\ Files'' (\CTANpkg{tds}). Directories
% with \xfile{texmf} in their name are usually organized this way.
%
% \subsection{Bundle installation}
%
% \paragraph{Unpacking.} Unpack the \xfile{oberdiek.tds.zip} in the
% TDS tree (also known as \xfile{texmf} tree) of your choice.
% Example (linux):
% \begin{quote}
%   |unzip oberdiek.tds.zip -d ~/texmf|
% \end{quote}
%
% \subsection{Package installation}
%
% \paragraph{Unpacking.} The \xfile{.dtx} file is a self-extracting
% \docstrip\ archive. The files are extracted by running the
% \xfile{.dtx} through \plainTeX:
% \begin{quote}
%   \verb|tex settobox.dtx|
% \end{quote}
%
% \paragraph{TDS.} Now the different files must be moved into
% the different directories in your installation TDS tree
% (also known as \xfile{texmf} tree):
% \begin{quote}
% \def\t{^^A
% \begin{tabular}{@{}>{\ttfamily}l@{ $\rightarrow$ }>{\ttfamily}l@{}}
%   settobox.sty & tex/latex/oberdiek/settobox.sty\\
%   settobox.pdf & doc/latex/oberdiek/settobox.pdf\\
%   settobox-example.tex & doc/latex/oberdiek/settobox-example.tex\\
%   settobox.dtx & source/latex/oberdiek/settobox.dtx\\
% \end{tabular}^^A
% }^^A
% \sbox0{\t}^^A
% \ifdim\wd0>\linewidth
%   \begingroup
%     \advance\linewidth by\leftmargin
%     \advance\linewidth by\rightmargin
%   \edef\x{\endgroup
%     \def\noexpand\lw{\the\linewidth}^^A
%   }\x
%   \def\lwbox{^^A
%     \leavevmode
%     \hbox to \linewidth{^^A
%       \kern-\leftmargin\relax
%       \hss
%       \usebox0
%       \hss
%       \kern-\rightmargin\relax
%     }^^A
%   }^^A
%   \ifdim\wd0>\lw
%     \sbox0{\small\t}^^A
%     \ifdim\wd0>\linewidth
%       \ifdim\wd0>\lw
%         \sbox0{\footnotesize\t}^^A
%         \ifdim\wd0>\linewidth
%           \ifdim\wd0>\lw
%             \sbox0{\scriptsize\t}^^A
%             \ifdim\wd0>\linewidth
%               \ifdim\wd0>\lw
%                 \sbox0{\tiny\t}^^A
%                 \ifdim\wd0>\linewidth
%                   \lwbox
%                 \else
%                   \usebox0
%                 \fi
%               \else
%                 \lwbox
%               \fi
%             \else
%               \usebox0
%             \fi
%           \else
%             \lwbox
%           \fi
%         \else
%           \usebox0
%         \fi
%       \else
%         \lwbox
%       \fi
%     \else
%       \usebox0
%     \fi
%   \else
%     \lwbox
%   \fi
% \else
%   \usebox0
% \fi
% \end{quote}
% If you have a \xfile{docstrip.cfg} that configures and enables \docstrip's
% TDS installing feature, then some files can already be in the right
% place, see the documentation of \docstrip.
%
% \subsection{Refresh file name databases}
%
% If your \TeX~distribution
% (\TeX\,Live, \mikTeX, \dots) relies on file name databases, you must refresh
% these. For example, \TeX\,Live\ users run \verb|texhash| or
% \verb|mktexlsr|.
%
% \subsection{Some details for the interested}
%
% \paragraph{Unpacking with \LaTeX.}
% The \xfile{.dtx} chooses its action depending on the format:
% \begin{description}
% \item[\plainTeX:] Run \docstrip\ and extract the files.
% \item[\LaTeX:] Generate the documentation.
% \end{description}
% If you insist on using \LaTeX\ for \docstrip\ (really,
% \docstrip\ does not need \LaTeX), then inform the autodetect routine
% about your intention:
% \begin{quote}
%   \verb|latex \let\install=y\input{settobox.dtx}|
% \end{quote}
% Do not forget to quote the argument according to the demands
% of your shell.
%
% \paragraph{Generating the documentation.}
% You can use both the \xfile{.dtx} or the \xfile{.drv} to generate
% the documentation. The process can be configured by the
% configuration file \xfile{ltxdoc.cfg}. For instance, put this
% line into this file, if you want to have A4 as paper format:
% \begin{quote}
%   \verb|\PassOptionsToClass{a4paper}{article}|
% \end{quote}
% An example follows how to generate the
% documentation with pdf\LaTeX:
% \begin{quote}
%\begin{verbatim}
%pdflatex settobox.dtx
%makeindex -s gind.ist settobox.idx
%pdflatex settobox.dtx
%makeindex -s gind.ist settobox.idx
%pdflatex settobox.dtx
%\end{verbatim}
% \end{quote}
%
% \begin{History}
%   \begin{Version}{2000/02/11 v1.0}
%   \item
%     First public release, written as answer in the
%     newsgroup \xnewsgroup{de.comp.text.tex}:
%     \URL{``\link{Die Hoehe von Minipages und Bild}''}^^A
%     {https://groups.google.com/group/de.comp.text.tex/msg/c3f6446f54f66c02}
%   \end{Version}
%   \begin{Version}{2000/09/07 v1.1}
%   \item
%     Documentation added.
%   \item
%     CTAN release.
%   \end{Version}
%   \begin{Version}{2006/02/20 v1.2}
%   \item
%     \cs{setboxwidth}, \cs{setboxheight}, \cs{setboxdepth} added.
%   \item
%     Box move commands added.
%   \item
%     DTX framework.
%   \item
%     LPPL 1.3
%   \end{Version}
%   \begin{Version}{2007/04/11 v1.3}
%   \item
%     Line ends sanitized.
%   \end{Version}
%   \begin{Version}{2008/08/11 v1.4}
%   \item
%     Code is not changed.
%   \item
%     URLs updated.
%   \end{Version}
%   \begin{Version}{2016/05/16 v1.5}
%   \item
%     Documentation updates.
%   \end{Version}
% \end{History}
%
% \PrintIndex
%
% \Finale
\endinput

%        (quote the arguments according to the demands of your shell)
%
% Documentation:
%    (a) If settobox.drv is present:
%           latex settobox.drv
%    (b) Without settobox.drv:
%           latex settobox.dtx; ...
%    The class ltxdoc loads the configuration file ltxdoc.cfg
%    if available. Here you can specify further options, e.g.
%    use A4 as paper format:
%       \PassOptionsToClass{a4paper}{article}
%
%    Programm calls to get the documentation (example):
%       pdflatex settobox.dtx
%       makeindex -s gind.ist settobox.idx
%       pdflatex settobox.dtx
%       makeindex -s gind.ist settobox.idx
%       pdflatex settobox.dtx
%
% Installation:
%    TDS:tex/latex/oberdiek/settobox.sty
%    TDS:doc/latex/oberdiek/settobox.pdf
%    TDS:doc/latex/oberdiek/settobox-example.tex
%    TDS:source/latex/oberdiek/settobox.dtx
%
%<*ignore>
\begingroup
  \catcode123=1 %
  \catcode125=2 %
  \def\x{LaTeX2e}%
\expandafter\endgroup
\ifcase 0\ifx\install y1\fi\expandafter
         \ifx\csname processbatchFile\endcsname\relax\else1\fi
         \ifx\fmtname\x\else 1\fi\relax
\else\csname fi\endcsname
%</ignore>
%<*install>
\input docstrip.tex
\Msg{************************************************************************}
\Msg{* Installation}
\Msg{* Package: settobox 2016/05/16 v1.5 Assign box dimensions to length registers (HO)}
\Msg{************************************************************************}

\keepsilent
\askforoverwritefalse

\let\MetaPrefix\relax
\preamble

This is a generated file.

Project: settobox
Version: 2016/05/16 v1.5

Copyright (C)
   2000, 2006-2008 Heiko Oberdiek
   2016-2019 Oberdiek Package Support Group

This work may be distributed and/or modified under the
conditions of the LaTeX Project Public License, either
version 1.3c of this license or (at your option) any later
version. This version of this license is in
   https://www.latex-project.org/lppl/lppl-1-3c.txt
and the latest version of this license is in
   https://www.latex-project.org/lppl.txt
and version 1.3 or later is part of all distributions of
LaTeX version 2005/12/01 or later.

This work has the LPPL maintenance status "maintained".

The Current Maintainers of this work are
Heiko Oberdiek and the Oberdiek Package Support Group
https://github.com/ho-tex/oberdiek/issues


This work consists of the main source file settobox.dtx
and the derived files
   settobox.sty, settobox.pdf, settobox.ins, settobox.drv,
   settobox-example.tex.

\endpreamble
\let\MetaPrefix\DoubleperCent

\generate{%
  \file{settobox.ins}{\from{settobox.dtx}{install}}%
  \file{settobox.drv}{\from{settobox.dtx}{driver}}%
  \usedir{tex/latex/oberdiek}%
  \file{settobox.sty}{\from{settobox.dtx}{package}}%
  \usedir{doc/latex/oberdiek}%
  \file{settobox-example.tex}{\from{settobox.dtx}{example}}%
}

\catcode32=13\relax% active space
\let =\space%
\Msg{************************************************************************}
\Msg{*}
\Msg{* To finish the installation you have to move the following}
\Msg{* file into a directory searched by TeX:}
\Msg{*}
\Msg{*     settobox.sty}
\Msg{*}
\Msg{* To produce the documentation run the file `settobox.drv'}
\Msg{* through LaTeX.}
\Msg{*}
\Msg{* Happy TeXing!}
\Msg{*}
\Msg{************************************************************************}

\endbatchfile
%</install>
%<*ignore>
\fi
%</ignore>
%<*driver>
\NeedsTeXFormat{LaTeX2e}
\ProvidesFile{settobox.drv}%
  [2016/05/16 v1.5 Assign box dimensions to length registers (HO)]%
\documentclass{ltxdoc}
\usepackage{holtxdoc}[2011/11/22]
\usepackage{calc}
\usepackage{settobox}
\begin{document}
  \DocInput{settobox.dtx}%
\end{document}
%</driver>
% \fi
%
%
%
% \GetFileInfo{settobox.drv}
%
% \title{The \xpackage{settobox} package}
% \date{2016/05/16 v1.5}
% \author{Heiko Oberdiek\thanks
% {Please report any issues at \url{https://github.com/ho-tex/oberdiek/issues}}}
%
% \maketitle
%
% \begin{abstract}
% Commands are defined for getting box sizes similar
% to \LaTeX's \cs{settowidth} commands.
% \end{abstract}
%
% \tableofcontents
%
% \section{Usage}
%
% \subsection{Get box dimensions}
%
% \begin{declcs}^^A
%   {settoboxwidth}\,\M{\LaTeX\ length}\,\M{\LaTeX\ box}\\
%   \SpecialUsageIndex{\settoboxheight}^^A
%   \cs{settoboxheight}\,\M{\LaTeX\ length}\,\M{\LaTeX\ box}\\
%   \SpecialUsageIndex{\settoboxdepth}^^A
%   \cs{settoboxdepth}\,\M{\LaTeX\ length}\,\M{\LaTeX\ box}\\
%   \SpecialUsageIndex{\settoboxtotalheight}^^A
%   \cs{settoboxtotalheight}\,\M{\LaTeX\ length}\,\M{\LaTeX\ box}
% \end{declcs}
% A \meta{\LaTeX\ box} is allocated by \cs{newsavebox}.
% It can be filled by \cs{sbox} or the environment \texttt{lrbox}.
% The commands above extract then the desired lengths.
%
% \subsection{Set box dimensions}
%
% \begin{declcs}^^A
%   {setboxwidth}\,\M{\LaTeX\ box}\,\M{\LaTeX\ length expression}\\
%   \SpecialUsageIndex{\setboxheight}^^A
%   \cs{setboxheight}\,\M{\LaTeX\ box}\,\M{\LaTeX\ length expression}\\
%   \SpecialUsageIndex{\setboxdepth}^^A
%   \cs{setboxdepth}\,\M{\LaTeX\ box}\,\M{\LaTeX\ length expression}
% \end{declcs}
% These commands allow the manipulation of the box. Package \xpackage{calc}
% is supported in the \meta{\LaTeX\ length expression}.
% Also the following length are available in this expression:
% \begin{quote}
% \begin{tabular}{@{}ll@{}}
%   \cs{width}& width of the box\\
%   \cs{height}& height of the box\\
%   \cs{depth}& depth of the box\\
%   \cs{totalheight}& totalheight of the box\\
% \end{tabular}
% \end{quote}
% Note, the base point (point at the left margin of the baseline)
% always remain constant.
%
% \subsection{Move box}
%
% \begin{declcs}^^A
%   {setboxmoveleft}\,\M{\LaTeX\ box}\,\M{\LaTeX\ length expression}\\
%   \SpecialUsageIndex{\setboxmoveright}^^A
%   \cs{setboxmoveright}\,\M{\LaTeX\ box}\,\M{\LaTeX\ length expression}\\
%   \SpecialUsageIndex{\setboxlower}^^A
%   \cs{setboxlower}\,\M{\LaTeX\ box}\,\M{\LaTeX\ length expression}\\
%   \SpecialUsageIndex{\setboxright}^^A
%   \cs{setboxright}\,\M{\LaTeX\ box}\,\M{\LaTeX\ length expression}
% \end{declcs}
% Note, the box is shifted relative to the base point. The base point
% is always inside the box, however the width and height of the
% box change along with the movement.
%
% \subsection{Example}
%
% \subsubsection{Short example}
%
% \begin{quote}
%\begin{verbatim}
%\newsavebox{\mybox}
%\newlength{\mylength}
%\sbox{\mybox}{Hello World}
%\settoboxwidth{\mylength}{\mybox}
%\end{verbatim}
% \end{quote}
%
% \subsubsection{Test file that shows box manipulations}
%
%    \begin{macrocode}
%<*example>
%<<END
\documentclass{article}

\usepackage{settobox}
\usepackage{calc}

\newsavebox{\mybox}

\setlength{\fboxsep}{0pt}
\setlength{\parindent}{20pt}
\setlength{\parskip}{10pt}
\pagestyle{empty}

% \test{#1}
% The macro is called with commands in #1 that manipulates
% the box \mybox. These commands along with the result of
% the manipulation is shown. Thus the essence of the
% macro is:
%
%   a) \sbox{\mybox}{The cracy fox.}
%   b) #1 % manipulates \mybox
%   c) Print #1 commands.
%   d) Print box with frame
%
% The implemenation looks more weird:
\makeatletter
\newcommand*{\test}[1]{%
  \par
  \begingroup
    \raggedright
    \edef\x{\detokenize{#1}}%
    \let\do\@makeother
    \dospecials
    \catcode`\~\active
    \catcode`\ =10\relax
    \def~{\\}%
    \noindent
    \texttt{\scantokens\expandafter{\x}}%
    \par
  \endgroup
  \begingroup
    \let~\relax
    \sbox{\mybox}{The cracy fox.}%
     #1%
     A---\fbox{\usebox\mybox}---B%
  \endgroup
  \par
}
\makeatother

\begin{document}

\test{\setboxwidth{\mybox}{1.25\width}}
\test{\setboxheight{\mybox}{0pt}}
\test{\setboxheight{\mybox}{2\height}}
\test{\setboxdepth{\mybox}{\height}}
\test{\setboxmoveleft{\mybox}{5pt}}
\test{%
  \setboxmoveleft{\mybox}{5pt}~%
  \setboxwidth{\mybox}{\width + 5pt}%
}
\test{\setboxmoveright{\mybox}{0.5\width}}
\test{\setboxlower{\mybox}{\height}}
\test{\setboxraise{\mybox}{\depth}}
\test{%
  \setboxmoveright{\mybox}{5pt}~%
  \setboxwidth{\mybox}{\width + 5pt}~%
  \setboxheight{\mybox}{\height + 5pt}~%
  \setboxdepth{\mybox}{\depth + 5pt}%
}

\end{document}
%END
%</example>
%    \end{macrocode}
%
% \noindent
%    The result:
%
% \vspace{1ex}
% \hrule
%
% \begingroup
% \newsavebox{\mybox}
%
% \setlength{\fboxsep}{0pt}
% \setlength{\parindent}{20pt}
% \setlength{\parskip}{10pt}
%
% \makeatletter
% \newcommand*{\test}[1]{^^A
%   \par
%   \begingroup
%     \raggedright
%     \edef\x{\detokenize{#1}}
%     \let\do\@makeother
%     \dospecials
%     \catcode`\~\active
%     \catcode`\ =10\relax
%     \def~{\\}^^A
%     \noindent
%     \texttt{\scantokens\expandafter{\x}}
%     \par
%   \endgroup
%   \begingroup
%     \let~\relax
%     \sbox{\mybox}{The cracy fox.}
%      #1^^A
%      A---\fbox{\usebox\mybox}---B
%   \endgroup
%   \par
% }
% \makeatother
%
% \test{\setboxwidth{\mybox}{1.25\width}}
% \test{\setboxheight{\mybox}{0pt}}
% \test{\setboxheight{\mybox}{2\height}}
% \test{\setboxdepth{\mybox}{\height}}
% \test{\setboxmoveleft{\mybox}{5pt}}
% \test{^^A
%   \setboxmoveleft{\mybox}{5pt}~^^A
%   \setboxwidth{\mybox}{\width + 5pt}^^A
% }
% \test{\setboxmoveright{\mybox}{0.5\width}}
% \test{\setboxlower{\mybox}{\height}}
% \test{\setboxraise{\mybox}{\depth}}
% \test{^^A
%   \setboxmoveright{\mybox}{5pt}~^^A
%   \setboxwidth{\mybox}{\width + 5pt}~^^A
%   \setboxheight{\mybox}{\height + 5pt}~^^A
%   \setboxdepth{\mybox}{\depth + 5pt}^^A
% }
%
% \endgroup
% \vspace{1ex}
% \hrule
% \vspace{4ex}
%
% \StopEventually{
% }
%
% \section{Implementation}
%
%    \begin{macrocode}
%<*package>
%    \end{macrocode}
%    Package identification.
%    \begin{macrocode}
\NeedsTeXFormat{LaTeX2e}
\ProvidesPackage{settobox}%
  [2016/05/16 v1.5 Assign box dimensions to length registers (HO)]
%    \end{macrocode}
%
%    \begin{macrocode}
\newcommand*{\settoboxwidth}[2]{\setlength{#1}{\wd#2}}
\newcommand*{\settoboxheight}[2]{\setlength{#1}{\ht#2}}
\newcommand*{\settoboxdepth}[2]{\setlength{#1}{\dp#2}}
\newcommand*{\settoboxtotalheight}[2]{%
  \setlength{#1}{\ht#2}%
  \addtolength{#1}{\dp#2}%
}
%    \end{macrocode}
%
%    \begin{macro}{\setboxwidth}
%    \begin{macrocode}
\newcommand*{\setboxwidth}[2]{%
  \settobox@length\wd{#1}{#2}%
}
%    \end{macrocode}
%    \end{macro}
%    \begin{macro}{\setboxheight}
%    \begin{macrocode}
\newcommand*{\setboxheight}[2]{%
  \settobox@length\ht{#1}{#2}%
}
%    \end{macrocode}
%    \end{macro}
%    \begin{macro}{\setboxheight}
%    \begin{macrocode}
\newcommand*{\setboxdepth}[2]{%
  \settobox@length\dp{#1}{#2}%
}
%    \end{macrocode}
%    \end{macro}
%    \begin{macro}{\setboxmoveleft}
%    \begin{macrocode}
\newcommand*{\setboxmoveleft}[2]{%
  \settobox@horiz{-}{#1}{#2}%
}
%    \end{macrocode}
%    \end{macro}
%    \begin{macro}{\setboxmoveright}
%    \begin{macrocode}
\newcommand*{\setboxmoveright}[2]{%
  \settobox@horiz{}{#1}{#2}%
}
%    \end{macrocode}
%    \end{macro}
%    \begin{macro}{\setboxlower}
%    \begin{macrocode}
\newcommand*{\setboxlower}[2]{%
  \settobox@vert\lower{#1}{#2}%
}
%    \end{macrocode}
%    \end{macro}
%    \begin{macro}{\setboxraise}
%    \begin{macrocode}
\newcommand*{\setboxraise}[2]{%
  \settobox@vert\raise{#1}{#2}%
}
%    \end{macrocode}
%    \end{macro}
%    \begin{macro}{\settobox@length}
%    The work for the \cs{setbox...} commands is done by
%    \cs{settobox@length}. Inside the length expression
%    \cs{width}, \cs{height}, \cs{depth}, \cs{totalheight}
%    are set to the dimensions of the box.\\
%    \begin{tabular}{@{}ll@{}}
%    |#1|:& the property of the box that is to be changed
%           (\cs{wd}, \cs{ht}, \cs{dp})\\
%    |#2|:& the box\\
%    |#3|:& length expression
%    \end{tabular}
%    \begin{macrocode}
\def\settobox@length#1#2#3{%
  \settobox@calc{#2}{#3}{#1#2=##1sp\relax}%
}
%    \end{macrocode}
%    \end{macro}
%
%    \begin{macro}{\settobox@horiz}
%    \begin{macrocode}
\def\settobox@horiz#1#2#3{%
  \settobox@calc{#2}{#3}{\setbox#2=\hbox{\kern#1##1sp\copy#2}}%
}
%    \end{macrocode}
%    \end{macro}
%    \begin{macro}{\settobox@vert}
%    \begin{macrocode}
\def\settobox@vert#1#2#3{%
  \settobox@calc{#2}{#3}{\setbox#2=\hbox{#1##1sp\copy#2}}%
}
%    \end{macrocode}
%    \end{macro}
%
%    \begin{macro}{\settobox@calc}
%    \begin{macrocode}
\def\settobox@calc#1#2#3{%
  \begingroup
    \def\width{\wd#1}%
    \def\height{\ht#1}%
    \def\depth{\dp#1}%
    \dimen@\ht#1\relax
    \advance\dimen@\dp#1\relax
    \def\totalheight{\dimen@}%
    \setlength{\dimen@}{#2}%
    \count@\dimen@
    \def\x##1{\endgroup
      #3%
    }%
  \expandafter\x\expandafter{\the\count@}%
}
%    \end{macrocode}
%    \end{macro}
%
%    \begin{macrocode}
%</package>
%    \end{macrocode}
%
% \section{Installation}
%
% \subsection{Download}
%
% \paragraph{Package.} This package is available on
% CTAN\footnote{\CTANpkg{settobox}}:
% \begin{description}
% \item[\CTAN{macros/latex/contrib/oberdiek/settobox.dtx}] The source file.
% \item[\CTAN{macros/latex/contrib/oberdiek/settobox.pdf}] Documentation.
% \end{description}
%
%
% \paragraph{Bundle.} All the packages of the bundle `oberdiek'
% are also available in a TDS compliant ZIP archive. There
% the packages are already unpacked and the documentation files
% are generated. The files and directories obey the TDS standard.
% \begin{description}
% \item[\CTANinstall{install/macros/latex/contrib/oberdiek.tds.zip}]
% \end{description}
% \emph{TDS} refers to the standard ``A Directory Structure
% for \TeX\ Files'' (\CTANpkg{tds}). Directories
% with \xfile{texmf} in their name are usually organized this way.
%
% \subsection{Bundle installation}
%
% \paragraph{Unpacking.} Unpack the \xfile{oberdiek.tds.zip} in the
% TDS tree (also known as \xfile{texmf} tree) of your choice.
% Example (linux):
% \begin{quote}
%   |unzip oberdiek.tds.zip -d ~/texmf|
% \end{quote}
%
% \subsection{Package installation}
%
% \paragraph{Unpacking.} The \xfile{.dtx} file is a self-extracting
% \docstrip\ archive. The files are extracted by running the
% \xfile{.dtx} through \plainTeX:
% \begin{quote}
%   \verb|tex settobox.dtx|
% \end{quote}
%
% \paragraph{TDS.} Now the different files must be moved into
% the different directories in your installation TDS tree
% (also known as \xfile{texmf} tree):
% \begin{quote}
% \def\t{^^A
% \begin{tabular}{@{}>{\ttfamily}l@{ $\rightarrow$ }>{\ttfamily}l@{}}
%   settobox.sty & tex/latex/oberdiek/settobox.sty\\
%   settobox.pdf & doc/latex/oberdiek/settobox.pdf\\
%   settobox-example.tex & doc/latex/oberdiek/settobox-example.tex\\
%   settobox.dtx & source/latex/oberdiek/settobox.dtx\\
% \end{tabular}^^A
% }^^A
% \sbox0{\t}^^A
% \ifdim\wd0>\linewidth
%   \begingroup
%     \advance\linewidth by\leftmargin
%     \advance\linewidth by\rightmargin
%   \edef\x{\endgroup
%     \def\noexpand\lw{\the\linewidth}^^A
%   }\x
%   \def\lwbox{^^A
%     \leavevmode
%     \hbox to \linewidth{^^A
%       \kern-\leftmargin\relax
%       \hss
%       \usebox0
%       \hss
%       \kern-\rightmargin\relax
%     }^^A
%   }^^A
%   \ifdim\wd0>\lw
%     \sbox0{\small\t}^^A
%     \ifdim\wd0>\linewidth
%       \ifdim\wd0>\lw
%         \sbox0{\footnotesize\t}^^A
%         \ifdim\wd0>\linewidth
%           \ifdim\wd0>\lw
%             \sbox0{\scriptsize\t}^^A
%             \ifdim\wd0>\linewidth
%               \ifdim\wd0>\lw
%                 \sbox0{\tiny\t}^^A
%                 \ifdim\wd0>\linewidth
%                   \lwbox
%                 \else
%                   \usebox0
%                 \fi
%               \else
%                 \lwbox
%               \fi
%             \else
%               \usebox0
%             \fi
%           \else
%             \lwbox
%           \fi
%         \else
%           \usebox0
%         \fi
%       \else
%         \lwbox
%       \fi
%     \else
%       \usebox0
%     \fi
%   \else
%     \lwbox
%   \fi
% \else
%   \usebox0
% \fi
% \end{quote}
% If you have a \xfile{docstrip.cfg} that configures and enables \docstrip's
% TDS installing feature, then some files can already be in the right
% place, see the documentation of \docstrip.
%
% \subsection{Refresh file name databases}
%
% If your \TeX~distribution
% (\TeX\,Live, \mikTeX, \dots) relies on file name databases, you must refresh
% these. For example, \TeX\,Live\ users run \verb|texhash| or
% \verb|mktexlsr|.
%
% \subsection{Some details for the interested}
%
% \paragraph{Unpacking with \LaTeX.}
% The \xfile{.dtx} chooses its action depending on the format:
% \begin{description}
% \item[\plainTeX:] Run \docstrip\ and extract the files.
% \item[\LaTeX:] Generate the documentation.
% \end{description}
% If you insist on using \LaTeX\ for \docstrip\ (really,
% \docstrip\ does not need \LaTeX), then inform the autodetect routine
% about your intention:
% \begin{quote}
%   \verb|latex \let\install=y% \iffalse meta-comment
%
% File: settobox.dtx
% Version: 2016/05/16 v1.5
% Info: Assign box dimensions to length registers
%
% Copyright (C)
%    2000, 2006-2008 Heiko Oberdiek
%    2016-2019 Oberdiek Package Support Group
%    https://github.com/ho-tex/oberdiek/issues
%
% This work may be distributed and/or modified under the
% conditions of the LaTeX Project Public License, either
% version 1.3c of this license or (at your option) any later
% version. This version of this license is in
%    https://www.latex-project.org/lppl/lppl-1-3c.txt
% and the latest version of this license is in
%    https://www.latex-project.org/lppl.txt
% and version 1.3 or later is part of all distributions of
% LaTeX version 2005/12/01 or later.
%
% This work has the LPPL maintenance status "maintained".
%
% The Current Maintainers of this work are
% Heiko Oberdiek and the Oberdiek Package Support Group
% https://github.com/ho-tex/oberdiek/issues
%
% This work consists of the main source file settobox.dtx
% and the derived files
%    settobox.sty, settobox.pdf, settobox.ins, settobox.drv,
%    settobox-example.tex.
%
% Distribution:
%    CTAN:macros/latex/contrib/oberdiek/settobox.dtx
%    CTAN:macros/latex/contrib/oberdiek/settobox.pdf
%
% Unpacking:
%    (a) If settobox.ins is present:
%           tex settobox.ins
%    (b) Without settobox.ins:
%           tex settobox.dtx
%    (c) If you insist on using LaTeX
%           latex \let\install=y\input{settobox.dtx}
%        (quote the arguments according to the demands of your shell)
%
% Documentation:
%    (a) If settobox.drv is present:
%           latex settobox.drv
%    (b) Without settobox.drv:
%           latex settobox.dtx; ...
%    The class ltxdoc loads the configuration file ltxdoc.cfg
%    if available. Here you can specify further options, e.g.
%    use A4 as paper format:
%       \PassOptionsToClass{a4paper}{article}
%
%    Programm calls to get the documentation (example):
%       pdflatex settobox.dtx
%       makeindex -s gind.ist settobox.idx
%       pdflatex settobox.dtx
%       makeindex -s gind.ist settobox.idx
%       pdflatex settobox.dtx
%
% Installation:
%    TDS:tex/latex/oberdiek/settobox.sty
%    TDS:doc/latex/oberdiek/settobox.pdf
%    TDS:doc/latex/oberdiek/settobox-example.tex
%    TDS:source/latex/oberdiek/settobox.dtx
%
%<*ignore>
\begingroup
  \catcode123=1 %
  \catcode125=2 %
  \def\x{LaTeX2e}%
\expandafter\endgroup
\ifcase 0\ifx\install y1\fi\expandafter
         \ifx\csname processbatchFile\endcsname\relax\else1\fi
         \ifx\fmtname\x\else 1\fi\relax
\else\csname fi\endcsname
%</ignore>
%<*install>
\input docstrip.tex
\Msg{************************************************************************}
\Msg{* Installation}
\Msg{* Package: settobox 2016/05/16 v1.5 Assign box dimensions to length registers (HO)}
\Msg{************************************************************************}

\keepsilent
\askforoverwritefalse

\let\MetaPrefix\relax
\preamble

This is a generated file.

Project: settobox
Version: 2016/05/16 v1.5

Copyright (C)
   2000, 2006-2008 Heiko Oberdiek
   2016-2019 Oberdiek Package Support Group

This work may be distributed and/or modified under the
conditions of the LaTeX Project Public License, either
version 1.3c of this license or (at your option) any later
version. This version of this license is in
   https://www.latex-project.org/lppl/lppl-1-3c.txt
and the latest version of this license is in
   https://www.latex-project.org/lppl.txt
and version 1.3 or later is part of all distributions of
LaTeX version 2005/12/01 or later.

This work has the LPPL maintenance status "maintained".

The Current Maintainers of this work are
Heiko Oberdiek and the Oberdiek Package Support Group
https://github.com/ho-tex/oberdiek/issues


This work consists of the main source file settobox.dtx
and the derived files
   settobox.sty, settobox.pdf, settobox.ins, settobox.drv,
   settobox-example.tex.

\endpreamble
\let\MetaPrefix\DoubleperCent

\generate{%
  \file{settobox.ins}{\from{settobox.dtx}{install}}%
  \file{settobox.drv}{\from{settobox.dtx}{driver}}%
  \usedir{tex/latex/oberdiek}%
  \file{settobox.sty}{\from{settobox.dtx}{package}}%
  \usedir{doc/latex/oberdiek}%
  \file{settobox-example.tex}{\from{settobox.dtx}{example}}%
}

\catcode32=13\relax% active space
\let =\space%
\Msg{************************************************************************}
\Msg{*}
\Msg{* To finish the installation you have to move the following}
\Msg{* file into a directory searched by TeX:}
\Msg{*}
\Msg{*     settobox.sty}
\Msg{*}
\Msg{* To produce the documentation run the file `settobox.drv'}
\Msg{* through LaTeX.}
\Msg{*}
\Msg{* Happy TeXing!}
\Msg{*}
\Msg{************************************************************************}

\endbatchfile
%</install>
%<*ignore>
\fi
%</ignore>
%<*driver>
\NeedsTeXFormat{LaTeX2e}
\ProvidesFile{settobox.drv}%
  [2016/05/16 v1.5 Assign box dimensions to length registers (HO)]%
\documentclass{ltxdoc}
\usepackage{holtxdoc}[2011/11/22]
\usepackage{calc}
\usepackage{settobox}
\begin{document}
  \DocInput{settobox.dtx}%
\end{document}
%</driver>
% \fi
%
%
%
% \GetFileInfo{settobox.drv}
%
% \title{The \xpackage{settobox} package}
% \date{2016/05/16 v1.5}
% \author{Heiko Oberdiek\thanks
% {Please report any issues at \url{https://github.com/ho-tex/oberdiek/issues}}}
%
% \maketitle
%
% \begin{abstract}
% Commands are defined for getting box sizes similar
% to \LaTeX's \cs{settowidth} commands.
% \end{abstract}
%
% \tableofcontents
%
% \section{Usage}
%
% \subsection{Get box dimensions}
%
% \begin{declcs}^^A
%   {settoboxwidth}\,\M{\LaTeX\ length}\,\M{\LaTeX\ box}\\
%   \SpecialUsageIndex{\settoboxheight}^^A
%   \cs{settoboxheight}\,\M{\LaTeX\ length}\,\M{\LaTeX\ box}\\
%   \SpecialUsageIndex{\settoboxdepth}^^A
%   \cs{settoboxdepth}\,\M{\LaTeX\ length}\,\M{\LaTeX\ box}\\
%   \SpecialUsageIndex{\settoboxtotalheight}^^A
%   \cs{settoboxtotalheight}\,\M{\LaTeX\ length}\,\M{\LaTeX\ box}
% \end{declcs}
% A \meta{\LaTeX\ box} is allocated by \cs{newsavebox}.
% It can be filled by \cs{sbox} or the environment \texttt{lrbox}.
% The commands above extract then the desired lengths.
%
% \subsection{Set box dimensions}
%
% \begin{declcs}^^A
%   {setboxwidth}\,\M{\LaTeX\ box}\,\M{\LaTeX\ length expression}\\
%   \SpecialUsageIndex{\setboxheight}^^A
%   \cs{setboxheight}\,\M{\LaTeX\ box}\,\M{\LaTeX\ length expression}\\
%   \SpecialUsageIndex{\setboxdepth}^^A
%   \cs{setboxdepth}\,\M{\LaTeX\ box}\,\M{\LaTeX\ length expression}
% \end{declcs}
% These commands allow the manipulation of the box. Package \xpackage{calc}
% is supported in the \meta{\LaTeX\ length expression}.
% Also the following length are available in this expression:
% \begin{quote}
% \begin{tabular}{@{}ll@{}}
%   \cs{width}& width of the box\\
%   \cs{height}& height of the box\\
%   \cs{depth}& depth of the box\\
%   \cs{totalheight}& totalheight of the box\\
% \end{tabular}
% \end{quote}
% Note, the base point (point at the left margin of the baseline)
% always remain constant.
%
% \subsection{Move box}
%
% \begin{declcs}^^A
%   {setboxmoveleft}\,\M{\LaTeX\ box}\,\M{\LaTeX\ length expression}\\
%   \SpecialUsageIndex{\setboxmoveright}^^A
%   \cs{setboxmoveright}\,\M{\LaTeX\ box}\,\M{\LaTeX\ length expression}\\
%   \SpecialUsageIndex{\setboxlower}^^A
%   \cs{setboxlower}\,\M{\LaTeX\ box}\,\M{\LaTeX\ length expression}\\
%   \SpecialUsageIndex{\setboxright}^^A
%   \cs{setboxright}\,\M{\LaTeX\ box}\,\M{\LaTeX\ length expression}
% \end{declcs}
% Note, the box is shifted relative to the base point. The base point
% is always inside the box, however the width and height of the
% box change along with the movement.
%
% \subsection{Example}
%
% \subsubsection{Short example}
%
% \begin{quote}
%\begin{verbatim}
%\newsavebox{\mybox}
%\newlength{\mylength}
%\sbox{\mybox}{Hello World}
%\settoboxwidth{\mylength}{\mybox}
%\end{verbatim}
% \end{quote}
%
% \subsubsection{Test file that shows box manipulations}
%
%    \begin{macrocode}
%<*example>
%<<END
\documentclass{article}

\usepackage{settobox}
\usepackage{calc}

\newsavebox{\mybox}

\setlength{\fboxsep}{0pt}
\setlength{\parindent}{20pt}
\setlength{\parskip}{10pt}
\pagestyle{empty}

% \test{#1}
% The macro is called with commands in #1 that manipulates
% the box \mybox. These commands along with the result of
% the manipulation is shown. Thus the essence of the
% macro is:
%
%   a) \sbox{\mybox}{The cracy fox.}
%   b) #1 % manipulates \mybox
%   c) Print #1 commands.
%   d) Print box with frame
%
% The implemenation looks more weird:
\makeatletter
\newcommand*{\test}[1]{%
  \par
  \begingroup
    \raggedright
    \edef\x{\detokenize{#1}}%
    \let\do\@makeother
    \dospecials
    \catcode`\~\active
    \catcode`\ =10\relax
    \def~{\\}%
    \noindent
    \texttt{\scantokens\expandafter{\x}}%
    \par
  \endgroup
  \begingroup
    \let~\relax
    \sbox{\mybox}{The cracy fox.}%
     #1%
     A---\fbox{\usebox\mybox}---B%
  \endgroup
  \par
}
\makeatother

\begin{document}

\test{\setboxwidth{\mybox}{1.25\width}}
\test{\setboxheight{\mybox}{0pt}}
\test{\setboxheight{\mybox}{2\height}}
\test{\setboxdepth{\mybox}{\height}}
\test{\setboxmoveleft{\mybox}{5pt}}
\test{%
  \setboxmoveleft{\mybox}{5pt}~%
  \setboxwidth{\mybox}{\width + 5pt}%
}
\test{\setboxmoveright{\mybox}{0.5\width}}
\test{\setboxlower{\mybox}{\height}}
\test{\setboxraise{\mybox}{\depth}}
\test{%
  \setboxmoveright{\mybox}{5pt}~%
  \setboxwidth{\mybox}{\width + 5pt}~%
  \setboxheight{\mybox}{\height + 5pt}~%
  \setboxdepth{\mybox}{\depth + 5pt}%
}

\end{document}
%END
%</example>
%    \end{macrocode}
%
% \noindent
%    The result:
%
% \vspace{1ex}
% \hrule
%
% \begingroup
% \newsavebox{\mybox}
%
% \setlength{\fboxsep}{0pt}
% \setlength{\parindent}{20pt}
% \setlength{\parskip}{10pt}
%
% \makeatletter
% \newcommand*{\test}[1]{^^A
%   \par
%   \begingroup
%     \raggedright
%     \edef\x{\detokenize{#1}}
%     \let\do\@makeother
%     \dospecials
%     \catcode`\~\active
%     \catcode`\ =10\relax
%     \def~{\\}^^A
%     \noindent
%     \texttt{\scantokens\expandafter{\x}}
%     \par
%   \endgroup
%   \begingroup
%     \let~\relax
%     \sbox{\mybox}{The cracy fox.}
%      #1^^A
%      A---\fbox{\usebox\mybox}---B
%   \endgroup
%   \par
% }
% \makeatother
%
% \test{\setboxwidth{\mybox}{1.25\width}}
% \test{\setboxheight{\mybox}{0pt}}
% \test{\setboxheight{\mybox}{2\height}}
% \test{\setboxdepth{\mybox}{\height}}
% \test{\setboxmoveleft{\mybox}{5pt}}
% \test{^^A
%   \setboxmoveleft{\mybox}{5pt}~^^A
%   \setboxwidth{\mybox}{\width + 5pt}^^A
% }
% \test{\setboxmoveright{\mybox}{0.5\width}}
% \test{\setboxlower{\mybox}{\height}}
% \test{\setboxraise{\mybox}{\depth}}
% \test{^^A
%   \setboxmoveright{\mybox}{5pt}~^^A
%   \setboxwidth{\mybox}{\width + 5pt}~^^A
%   \setboxheight{\mybox}{\height + 5pt}~^^A
%   \setboxdepth{\mybox}{\depth + 5pt}^^A
% }
%
% \endgroup
% \vspace{1ex}
% \hrule
% \vspace{4ex}
%
% \StopEventually{
% }
%
% \section{Implementation}
%
%    \begin{macrocode}
%<*package>
%    \end{macrocode}
%    Package identification.
%    \begin{macrocode}
\NeedsTeXFormat{LaTeX2e}
\ProvidesPackage{settobox}%
  [2016/05/16 v1.5 Assign box dimensions to length registers (HO)]
%    \end{macrocode}
%
%    \begin{macrocode}
\newcommand*{\settoboxwidth}[2]{\setlength{#1}{\wd#2}}
\newcommand*{\settoboxheight}[2]{\setlength{#1}{\ht#2}}
\newcommand*{\settoboxdepth}[2]{\setlength{#1}{\dp#2}}
\newcommand*{\settoboxtotalheight}[2]{%
  \setlength{#1}{\ht#2}%
  \addtolength{#1}{\dp#2}%
}
%    \end{macrocode}
%
%    \begin{macro}{\setboxwidth}
%    \begin{macrocode}
\newcommand*{\setboxwidth}[2]{%
  \settobox@length\wd{#1}{#2}%
}
%    \end{macrocode}
%    \end{macro}
%    \begin{macro}{\setboxheight}
%    \begin{macrocode}
\newcommand*{\setboxheight}[2]{%
  \settobox@length\ht{#1}{#2}%
}
%    \end{macrocode}
%    \end{macro}
%    \begin{macro}{\setboxheight}
%    \begin{macrocode}
\newcommand*{\setboxdepth}[2]{%
  \settobox@length\dp{#1}{#2}%
}
%    \end{macrocode}
%    \end{macro}
%    \begin{macro}{\setboxmoveleft}
%    \begin{macrocode}
\newcommand*{\setboxmoveleft}[2]{%
  \settobox@horiz{-}{#1}{#2}%
}
%    \end{macrocode}
%    \end{macro}
%    \begin{macro}{\setboxmoveright}
%    \begin{macrocode}
\newcommand*{\setboxmoveright}[2]{%
  \settobox@horiz{}{#1}{#2}%
}
%    \end{macrocode}
%    \end{macro}
%    \begin{macro}{\setboxlower}
%    \begin{macrocode}
\newcommand*{\setboxlower}[2]{%
  \settobox@vert\lower{#1}{#2}%
}
%    \end{macrocode}
%    \end{macro}
%    \begin{macro}{\setboxraise}
%    \begin{macrocode}
\newcommand*{\setboxraise}[2]{%
  \settobox@vert\raise{#1}{#2}%
}
%    \end{macrocode}
%    \end{macro}
%    \begin{macro}{\settobox@length}
%    The work for the \cs{setbox...} commands is done by
%    \cs{settobox@length}. Inside the length expression
%    \cs{width}, \cs{height}, \cs{depth}, \cs{totalheight}
%    are set to the dimensions of the box.\\
%    \begin{tabular}{@{}ll@{}}
%    |#1|:& the property of the box that is to be changed
%           (\cs{wd}, \cs{ht}, \cs{dp})\\
%    |#2|:& the box\\
%    |#3|:& length expression
%    \end{tabular}
%    \begin{macrocode}
\def\settobox@length#1#2#3{%
  \settobox@calc{#2}{#3}{#1#2=##1sp\relax}%
}
%    \end{macrocode}
%    \end{macro}
%
%    \begin{macro}{\settobox@horiz}
%    \begin{macrocode}
\def\settobox@horiz#1#2#3{%
  \settobox@calc{#2}{#3}{\setbox#2=\hbox{\kern#1##1sp\copy#2}}%
}
%    \end{macrocode}
%    \end{macro}
%    \begin{macro}{\settobox@vert}
%    \begin{macrocode}
\def\settobox@vert#1#2#3{%
  \settobox@calc{#2}{#3}{\setbox#2=\hbox{#1##1sp\copy#2}}%
}
%    \end{macrocode}
%    \end{macro}
%
%    \begin{macro}{\settobox@calc}
%    \begin{macrocode}
\def\settobox@calc#1#2#3{%
  \begingroup
    \def\width{\wd#1}%
    \def\height{\ht#1}%
    \def\depth{\dp#1}%
    \dimen@\ht#1\relax
    \advance\dimen@\dp#1\relax
    \def\totalheight{\dimen@}%
    \setlength{\dimen@}{#2}%
    \count@\dimen@
    \def\x##1{\endgroup
      #3%
    }%
  \expandafter\x\expandafter{\the\count@}%
}
%    \end{macrocode}
%    \end{macro}
%
%    \begin{macrocode}
%</package>
%    \end{macrocode}
%
% \section{Installation}
%
% \subsection{Download}
%
% \paragraph{Package.} This package is available on
% CTAN\footnote{\CTANpkg{settobox}}:
% \begin{description}
% \item[\CTAN{macros/latex/contrib/oberdiek/settobox.dtx}] The source file.
% \item[\CTAN{macros/latex/contrib/oberdiek/settobox.pdf}] Documentation.
% \end{description}
%
%
% \paragraph{Bundle.} All the packages of the bundle `oberdiek'
% are also available in a TDS compliant ZIP archive. There
% the packages are already unpacked and the documentation files
% are generated. The files and directories obey the TDS standard.
% \begin{description}
% \item[\CTANinstall{install/macros/latex/contrib/oberdiek.tds.zip}]
% \end{description}
% \emph{TDS} refers to the standard ``A Directory Structure
% for \TeX\ Files'' (\CTANpkg{tds}). Directories
% with \xfile{texmf} in their name are usually organized this way.
%
% \subsection{Bundle installation}
%
% \paragraph{Unpacking.} Unpack the \xfile{oberdiek.tds.zip} in the
% TDS tree (also known as \xfile{texmf} tree) of your choice.
% Example (linux):
% \begin{quote}
%   |unzip oberdiek.tds.zip -d ~/texmf|
% \end{quote}
%
% \subsection{Package installation}
%
% \paragraph{Unpacking.} The \xfile{.dtx} file is a self-extracting
% \docstrip\ archive. The files are extracted by running the
% \xfile{.dtx} through \plainTeX:
% \begin{quote}
%   \verb|tex settobox.dtx|
% \end{quote}
%
% \paragraph{TDS.} Now the different files must be moved into
% the different directories in your installation TDS tree
% (also known as \xfile{texmf} tree):
% \begin{quote}
% \def\t{^^A
% \begin{tabular}{@{}>{\ttfamily}l@{ $\rightarrow$ }>{\ttfamily}l@{}}
%   settobox.sty & tex/latex/oberdiek/settobox.sty\\
%   settobox.pdf & doc/latex/oberdiek/settobox.pdf\\
%   settobox-example.tex & doc/latex/oberdiek/settobox-example.tex\\
%   settobox.dtx & source/latex/oberdiek/settobox.dtx\\
% \end{tabular}^^A
% }^^A
% \sbox0{\t}^^A
% \ifdim\wd0>\linewidth
%   \begingroup
%     \advance\linewidth by\leftmargin
%     \advance\linewidth by\rightmargin
%   \edef\x{\endgroup
%     \def\noexpand\lw{\the\linewidth}^^A
%   }\x
%   \def\lwbox{^^A
%     \leavevmode
%     \hbox to \linewidth{^^A
%       \kern-\leftmargin\relax
%       \hss
%       \usebox0
%       \hss
%       \kern-\rightmargin\relax
%     }^^A
%   }^^A
%   \ifdim\wd0>\lw
%     \sbox0{\small\t}^^A
%     \ifdim\wd0>\linewidth
%       \ifdim\wd0>\lw
%         \sbox0{\footnotesize\t}^^A
%         \ifdim\wd0>\linewidth
%           \ifdim\wd0>\lw
%             \sbox0{\scriptsize\t}^^A
%             \ifdim\wd0>\linewidth
%               \ifdim\wd0>\lw
%                 \sbox0{\tiny\t}^^A
%                 \ifdim\wd0>\linewidth
%                   \lwbox
%                 \else
%                   \usebox0
%                 \fi
%               \else
%                 \lwbox
%               \fi
%             \else
%               \usebox0
%             \fi
%           \else
%             \lwbox
%           \fi
%         \else
%           \usebox0
%         \fi
%       \else
%         \lwbox
%       \fi
%     \else
%       \usebox0
%     \fi
%   \else
%     \lwbox
%   \fi
% \else
%   \usebox0
% \fi
% \end{quote}
% If you have a \xfile{docstrip.cfg} that configures and enables \docstrip's
% TDS installing feature, then some files can already be in the right
% place, see the documentation of \docstrip.
%
% \subsection{Refresh file name databases}
%
% If your \TeX~distribution
% (\TeX\,Live, \mikTeX, \dots) relies on file name databases, you must refresh
% these. For example, \TeX\,Live\ users run \verb|texhash| or
% \verb|mktexlsr|.
%
% \subsection{Some details for the interested}
%
% \paragraph{Unpacking with \LaTeX.}
% The \xfile{.dtx} chooses its action depending on the format:
% \begin{description}
% \item[\plainTeX:] Run \docstrip\ and extract the files.
% \item[\LaTeX:] Generate the documentation.
% \end{description}
% If you insist on using \LaTeX\ for \docstrip\ (really,
% \docstrip\ does not need \LaTeX), then inform the autodetect routine
% about your intention:
% \begin{quote}
%   \verb|latex \let\install=y\input{settobox.dtx}|
% \end{quote}
% Do not forget to quote the argument according to the demands
% of your shell.
%
% \paragraph{Generating the documentation.}
% You can use both the \xfile{.dtx} or the \xfile{.drv} to generate
% the documentation. The process can be configured by the
% configuration file \xfile{ltxdoc.cfg}. For instance, put this
% line into this file, if you want to have A4 as paper format:
% \begin{quote}
%   \verb|\PassOptionsToClass{a4paper}{article}|
% \end{quote}
% An example follows how to generate the
% documentation with pdf\LaTeX:
% \begin{quote}
%\begin{verbatim}
%pdflatex settobox.dtx
%makeindex -s gind.ist settobox.idx
%pdflatex settobox.dtx
%makeindex -s gind.ist settobox.idx
%pdflatex settobox.dtx
%\end{verbatim}
% \end{quote}
%
% \begin{History}
%   \begin{Version}{2000/02/11 v1.0}
%   \item
%     First public release, written as answer in the
%     newsgroup \xnewsgroup{de.comp.text.tex}:
%     \URL{``\link{Die Hoehe von Minipages und Bild}''}^^A
%     {https://groups.google.com/group/de.comp.text.tex/msg/c3f6446f54f66c02}
%   \end{Version}
%   \begin{Version}{2000/09/07 v1.1}
%   \item
%     Documentation added.
%   \item
%     CTAN release.
%   \end{Version}
%   \begin{Version}{2006/02/20 v1.2}
%   \item
%     \cs{setboxwidth}, \cs{setboxheight}, \cs{setboxdepth} added.
%   \item
%     Box move commands added.
%   \item
%     DTX framework.
%   \item
%     LPPL 1.3
%   \end{Version}
%   \begin{Version}{2007/04/11 v1.3}
%   \item
%     Line ends sanitized.
%   \end{Version}
%   \begin{Version}{2008/08/11 v1.4}
%   \item
%     Code is not changed.
%   \item
%     URLs updated.
%   \end{Version}
%   \begin{Version}{2016/05/16 v1.5}
%   \item
%     Documentation updates.
%   \end{Version}
% \end{History}
%
% \PrintIndex
%
% \Finale
\endinput
|
% \end{quote}
% Do not forget to quote the argument according to the demands
% of your shell.
%
% \paragraph{Generating the documentation.}
% You can use both the \xfile{.dtx} or the \xfile{.drv} to generate
% the documentation. The process can be configured by the
% configuration file \xfile{ltxdoc.cfg}. For instance, put this
% line into this file, if you want to have A4 as paper format:
% \begin{quote}
%   \verb|\PassOptionsToClass{a4paper}{article}|
% \end{quote}
% An example follows how to generate the
% documentation with pdf\LaTeX:
% \begin{quote}
%\begin{verbatim}
%pdflatex settobox.dtx
%makeindex -s gind.ist settobox.idx
%pdflatex settobox.dtx
%makeindex -s gind.ist settobox.idx
%pdflatex settobox.dtx
%\end{verbatim}
% \end{quote}
%
% \begin{History}
%   \begin{Version}{2000/02/11 v1.0}
%   \item
%     First public release, written as answer in the
%     newsgroup \xnewsgroup{de.comp.text.tex}:
%     \URL{``\link{Die Hoehe von Minipages und Bild}''}^^A
%     {https://groups.google.com/group/de.comp.text.tex/msg/c3f6446f54f66c02}
%   \end{Version}
%   \begin{Version}{2000/09/07 v1.1}
%   \item
%     Documentation added.
%   \item
%     CTAN release.
%   \end{Version}
%   \begin{Version}{2006/02/20 v1.2}
%   \item
%     \cs{setboxwidth}, \cs{setboxheight}, \cs{setboxdepth} added.
%   \item
%     Box move commands added.
%   \item
%     DTX framework.
%   \item
%     LPPL 1.3
%   \end{Version}
%   \begin{Version}{2007/04/11 v1.3}
%   \item
%     Line ends sanitized.
%   \end{Version}
%   \begin{Version}{2008/08/11 v1.4}
%   \item
%     Code is not changed.
%   \item
%     URLs updated.
%   \end{Version}
%   \begin{Version}{2016/05/16 v1.5}
%   \item
%     Documentation updates.
%   \end{Version}
% \end{History}
%
% \PrintIndex
%
% \Finale
\endinput
|
% \end{quote}
% Do not forget to quote the argument according to the demands
% of your shell.
%
% \paragraph{Generating the documentation.}
% You can use both the \xfile{.dtx} or the \xfile{.drv} to generate
% the documentation. The process can be configured by the
% configuration file \xfile{ltxdoc.cfg}. For instance, put this
% line into this file, if you want to have A4 as paper format:
% \begin{quote}
%   \verb|\PassOptionsToClass{a4paper}{article}|
% \end{quote}
% An example follows how to generate the
% documentation with pdf\LaTeX:
% \begin{quote}
%\begin{verbatim}
%pdflatex settobox.dtx
%makeindex -s gind.ist settobox.idx
%pdflatex settobox.dtx
%makeindex -s gind.ist settobox.idx
%pdflatex settobox.dtx
%\end{verbatim}
% \end{quote}
%
% \begin{History}
%   \begin{Version}{2000/02/11 v1.0}
%   \item
%     First public release, written as answer in the
%     newsgroup \xnewsgroup{de.comp.text.tex}:
%     \URL{``\link{Die Hoehe von Minipages und Bild}''}^^A
%     {https://groups.google.com/group/de.comp.text.tex/msg/c3f6446f54f66c02}
%   \end{Version}
%   \begin{Version}{2000/09/07 v1.1}
%   \item
%     Documentation added.
%   \item
%     CTAN release.
%   \end{Version}
%   \begin{Version}{2006/02/20 v1.2}
%   \item
%     \cs{setboxwidth}, \cs{setboxheight}, \cs{setboxdepth} added.
%   \item
%     Box move commands added.
%   \item
%     DTX framework.
%   \item
%     LPPL 1.3
%   \end{Version}
%   \begin{Version}{2007/04/11 v1.3}
%   \item
%     Line ends sanitized.
%   \end{Version}
%   \begin{Version}{2008/08/11 v1.4}
%   \item
%     Code is not changed.
%   \item
%     URLs updated.
%   \end{Version}
%   \begin{Version}{2016/05/16 v1.5}
%   \item
%     Documentation updates.
%   \end{Version}
% \end{History}
%
% \PrintIndex
%
% \Finale
\endinput

%        (quote the arguments according to the demands of your shell)
%
% Documentation:
%    (a) If settobox.drv is present:
%           latex settobox.drv
%    (b) Without settobox.drv:
%           latex settobox.dtx; ...
%    The class ltxdoc loads the configuration file ltxdoc.cfg
%    if available. Here you can specify further options, e.g.
%    use A4 as paper format:
%       \PassOptionsToClass{a4paper}{article}
%
%    Programm calls to get the documentation (example):
%       pdflatex settobox.dtx
%       makeindex -s gind.ist settobox.idx
%       pdflatex settobox.dtx
%       makeindex -s gind.ist settobox.idx
%       pdflatex settobox.dtx
%
% Installation:
%    TDS:tex/latex/oberdiek/settobox.sty
%    TDS:doc/latex/oberdiek/settobox.pdf
%    TDS:doc/latex/oberdiek/settobox-example.tex
%    TDS:source/latex/oberdiek/settobox.dtx
%
%<*ignore>
\begingroup
  \catcode123=1 %
  \catcode125=2 %
  \def\x{LaTeX2e}%
\expandafter\endgroup
\ifcase 0\ifx\install y1\fi\expandafter
         \ifx\csname processbatchFile\endcsname\relax\else1\fi
         \ifx\fmtname\x\else 1\fi\relax
\else\csname fi\endcsname
%</ignore>
%<*install>
\input docstrip.tex
\Msg{************************************************************************}
\Msg{* Installation}
\Msg{* Package: settobox 2016/05/16 v1.5 Assign box dimensions to length registers (HO)}
\Msg{************************************************************************}

\keepsilent
\askforoverwritefalse

\let\MetaPrefix\relax
\preamble

This is a generated file.

Project: settobox
Version: 2016/05/16 v1.5

Copyright (C)
   2000, 2006-2008 Heiko Oberdiek
   2016-2019 Oberdiek Package Support Group

This work may be distributed and/or modified under the
conditions of the LaTeX Project Public License, either
version 1.3c of this license or (at your option) any later
version. This version of this license is in
   https://www.latex-project.org/lppl/lppl-1-3c.txt
and the latest version of this license is in
   https://www.latex-project.org/lppl.txt
and version 1.3 or later is part of all distributions of
LaTeX version 2005/12/01 or later.

This work has the LPPL maintenance status "maintained".

The Current Maintainers of this work are
Heiko Oberdiek and the Oberdiek Package Support Group
https://github.com/ho-tex/oberdiek/issues


This work consists of the main source file settobox.dtx
and the derived files
   settobox.sty, settobox.pdf, settobox.ins, settobox.drv,
   settobox-example.tex.

\endpreamble
\let\MetaPrefix\DoubleperCent

\generate{%
  \file{settobox.ins}{\from{settobox.dtx}{install}}%
  \file{settobox.drv}{\from{settobox.dtx}{driver}}%
  \usedir{tex/latex/oberdiek}%
  \file{settobox.sty}{\from{settobox.dtx}{package}}%
  \usedir{doc/latex/oberdiek}%
  \file{settobox-example.tex}{\from{settobox.dtx}{example}}%
}

\catcode32=13\relax% active space
\let =\space%
\Msg{************************************************************************}
\Msg{*}
\Msg{* To finish the installation you have to move the following}
\Msg{* file into a directory searched by TeX:}
\Msg{*}
\Msg{*     settobox.sty}
\Msg{*}
\Msg{* To produce the documentation run the file `settobox.drv'}
\Msg{* through LaTeX.}
\Msg{*}
\Msg{* Happy TeXing!}
\Msg{*}
\Msg{************************************************************************}

\endbatchfile
%</install>
%<*ignore>
\fi
%</ignore>
%<*driver>
\NeedsTeXFormat{LaTeX2e}
\ProvidesFile{settobox.drv}%
  [2016/05/16 v1.5 Assign box dimensions to length registers (HO)]%
\documentclass{ltxdoc}
\usepackage{holtxdoc}[2011/11/22]
\usepackage{calc}
\usepackage{settobox}
\begin{document}
  \DocInput{settobox.dtx}%
\end{document}
%</driver>
% \fi
%
%
%
% \GetFileInfo{settobox.drv}
%
% \title{The \xpackage{settobox} package}
% \date{2016/05/16 v1.5}
% \author{Heiko Oberdiek\thanks
% {Please report any issues at \url{https://github.com/ho-tex/oberdiek/issues}}}
%
% \maketitle
%
% \begin{abstract}
% Commands are defined for getting box sizes similar
% to \LaTeX's \cs{settowidth} commands.
% \end{abstract}
%
% \tableofcontents
%
% \section{Usage}
%
% \subsection{Get box dimensions}
%
% \begin{declcs}^^A
%   {settoboxwidth}\,\M{\LaTeX\ length}\,\M{\LaTeX\ box}\\
%   \SpecialUsageIndex{\settoboxheight}^^A
%   \cs{settoboxheight}\,\M{\LaTeX\ length}\,\M{\LaTeX\ box}\\
%   \SpecialUsageIndex{\settoboxdepth}^^A
%   \cs{settoboxdepth}\,\M{\LaTeX\ length}\,\M{\LaTeX\ box}\\
%   \SpecialUsageIndex{\settoboxtotalheight}^^A
%   \cs{settoboxtotalheight}\,\M{\LaTeX\ length}\,\M{\LaTeX\ box}
% \end{declcs}
% A \meta{\LaTeX\ box} is allocated by \cs{newsavebox}.
% It can be filled by \cs{sbox} or the environment \texttt{lrbox}.
% The commands above extract then the desired lengths.
%
% \subsection{Set box dimensions}
%
% \begin{declcs}^^A
%   {setboxwidth}\,\M{\LaTeX\ box}\,\M{\LaTeX\ length expression}\\
%   \SpecialUsageIndex{\setboxheight}^^A
%   \cs{setboxheight}\,\M{\LaTeX\ box}\,\M{\LaTeX\ length expression}\\
%   \SpecialUsageIndex{\setboxdepth}^^A
%   \cs{setboxdepth}\,\M{\LaTeX\ box}\,\M{\LaTeX\ length expression}
% \end{declcs}
% These commands allow the manipulation of the box. Package \xpackage{calc}
% is supported in the \meta{\LaTeX\ length expression}.
% Also the following length are available in this expression:
% \begin{quote}
% \begin{tabular}{@{}ll@{}}
%   \cs{width}& width of the box\\
%   \cs{height}& height of the box\\
%   \cs{depth}& depth of the box\\
%   \cs{totalheight}& totalheight of the box\\
% \end{tabular}
% \end{quote}
% Note, the base point (point at the left margin of the baseline)
% always remain constant.
%
% \subsection{Move box}
%
% \begin{declcs}^^A
%   {setboxmoveleft}\,\M{\LaTeX\ box}\,\M{\LaTeX\ length expression}\\
%   \SpecialUsageIndex{\setboxmoveright}^^A
%   \cs{setboxmoveright}\,\M{\LaTeX\ box}\,\M{\LaTeX\ length expression}\\
%   \SpecialUsageIndex{\setboxlower}^^A
%   \cs{setboxlower}\,\M{\LaTeX\ box}\,\M{\LaTeX\ length expression}\\
%   \SpecialUsageIndex{\setboxright}^^A
%   \cs{setboxright}\,\M{\LaTeX\ box}\,\M{\LaTeX\ length expression}
% \end{declcs}
% Note, the box is shifted relative to the base point. The base point
% is always inside the box, however the width and height of the
% box change along with the movement.
%
% \subsection{Example}
%
% \subsubsection{Short example}
%
% \begin{quote}
%\begin{verbatim}
%\newsavebox{\mybox}
%\newlength{\mylength}
%\sbox{\mybox}{Hello World}
%\settoboxwidth{\mylength}{\mybox}
%\end{verbatim}
% \end{quote}
%
% \subsubsection{Test file that shows box manipulations}
%
%    \begin{macrocode}
%<*example>
%<<END
\documentclass{article}

\usepackage{settobox}
\usepackage{calc}

\newsavebox{\mybox}

\setlength{\fboxsep}{0pt}
\setlength{\parindent}{20pt}
\setlength{\parskip}{10pt}
\pagestyle{empty}

% \test{#1}
% The macro is called with commands in #1 that manipulates
% the box \mybox. These commands along with the result of
% the manipulation is shown. Thus the essence of the
% macro is:
%
%   a) \sbox{\mybox}{The cracy fox.}
%   b) #1 % manipulates \mybox
%   c) Print #1 commands.
%   d) Print box with frame
%
% The implemenation looks more weird:
\makeatletter
\newcommand*{\test}[1]{%
  \par
  \begingroup
    \raggedright
    \edef\x{\detokenize{#1}}%
    \let\do\@makeother
    \dospecials
    \catcode`\~\active
    \catcode`\ =10\relax
    \def~{\\}%
    \noindent
    \texttt{\scantokens\expandafter{\x}}%
    \par
  \endgroup
  \begingroup
    \let~\relax
    \sbox{\mybox}{The cracy fox.}%
     #1%
     A---\fbox{\usebox\mybox}---B%
  \endgroup
  \par
}
\makeatother

\begin{document}

\test{\setboxwidth{\mybox}{1.25\width}}
\test{\setboxheight{\mybox}{0pt}}
\test{\setboxheight{\mybox}{2\height}}
\test{\setboxdepth{\mybox}{\height}}
\test{\setboxmoveleft{\mybox}{5pt}}
\test{%
  \setboxmoveleft{\mybox}{5pt}~%
  \setboxwidth{\mybox}{\width + 5pt}%
}
\test{\setboxmoveright{\mybox}{0.5\width}}
\test{\setboxlower{\mybox}{\height}}
\test{\setboxraise{\mybox}{\depth}}
\test{%
  \setboxmoveright{\mybox}{5pt}~%
  \setboxwidth{\mybox}{\width + 5pt}~%
  \setboxheight{\mybox}{\height + 5pt}~%
  \setboxdepth{\mybox}{\depth + 5pt}%
}

\end{document}
%END
%</example>
%    \end{macrocode}
%
% \noindent
%    The result:
%
% \vspace{1ex}
% \hrule
%
% \begingroup
% \newsavebox{\mybox}
%
% \setlength{\fboxsep}{0pt}
% \setlength{\parindent}{20pt}
% \setlength{\parskip}{10pt}
%
% \makeatletter
% \newcommand*{\test}[1]{^^A
%   \par
%   \begingroup
%     \raggedright
%     \edef\x{\detokenize{#1}}
%     \let\do\@makeother
%     \dospecials
%     \catcode`\~\active
%     \catcode`\ =10\relax
%     \def~{\\}^^A
%     \noindent
%     \texttt{\scantokens\expandafter{\x}}
%     \par
%   \endgroup
%   \begingroup
%     \let~\relax
%     \sbox{\mybox}{The cracy fox.}
%      #1^^A
%      A---\fbox{\usebox\mybox}---B
%   \endgroup
%   \par
% }
% \makeatother
%
% \test{\setboxwidth{\mybox}{1.25\width}}
% \test{\setboxheight{\mybox}{0pt}}
% \test{\setboxheight{\mybox}{2\height}}
% \test{\setboxdepth{\mybox}{\height}}
% \test{\setboxmoveleft{\mybox}{5pt}}
% \test{^^A
%   \setboxmoveleft{\mybox}{5pt}~^^A
%   \setboxwidth{\mybox}{\width + 5pt}^^A
% }
% \test{\setboxmoveright{\mybox}{0.5\width}}
% \test{\setboxlower{\mybox}{\height}}
% \test{\setboxraise{\mybox}{\depth}}
% \test{^^A
%   \setboxmoveright{\mybox}{5pt}~^^A
%   \setboxwidth{\mybox}{\width + 5pt}~^^A
%   \setboxheight{\mybox}{\height + 5pt}~^^A
%   \setboxdepth{\mybox}{\depth + 5pt}^^A
% }
%
% \endgroup
% \vspace{1ex}
% \hrule
% \vspace{4ex}
%
% \StopEventually{
% }
%
% \section{Implementation}
%
%    \begin{macrocode}
%<*package>
%    \end{macrocode}
%    Package identification.
%    \begin{macrocode}
\NeedsTeXFormat{LaTeX2e}
\ProvidesPackage{settobox}%
  [2016/05/16 v1.5 Assign box dimensions to length registers (HO)]
%    \end{macrocode}
%
%    \begin{macrocode}
\newcommand*{\settoboxwidth}[2]{\setlength{#1}{\wd#2}}
\newcommand*{\settoboxheight}[2]{\setlength{#1}{\ht#2}}
\newcommand*{\settoboxdepth}[2]{\setlength{#1}{\dp#2}}
\newcommand*{\settoboxtotalheight}[2]{%
  \setlength{#1}{\ht#2}%
  \addtolength{#1}{\dp#2}%
}
%    \end{macrocode}
%
%    \begin{macro}{\setboxwidth}
%    \begin{macrocode}
\newcommand*{\setboxwidth}[2]{%
  \settobox@length\wd{#1}{#2}%
}
%    \end{macrocode}
%    \end{macro}
%    \begin{macro}{\setboxheight}
%    \begin{macrocode}
\newcommand*{\setboxheight}[2]{%
  \settobox@length\ht{#1}{#2}%
}
%    \end{macrocode}
%    \end{macro}
%    \begin{macro}{\setboxheight}
%    \begin{macrocode}
\newcommand*{\setboxdepth}[2]{%
  \settobox@length\dp{#1}{#2}%
}
%    \end{macrocode}
%    \end{macro}
%    \begin{macro}{\setboxmoveleft}
%    \begin{macrocode}
\newcommand*{\setboxmoveleft}[2]{%
  \settobox@horiz{-}{#1}{#2}%
}
%    \end{macrocode}
%    \end{macro}
%    \begin{macro}{\setboxmoveright}
%    \begin{macrocode}
\newcommand*{\setboxmoveright}[2]{%
  \settobox@horiz{}{#1}{#2}%
}
%    \end{macrocode}
%    \end{macro}
%    \begin{macro}{\setboxlower}
%    \begin{macrocode}
\newcommand*{\setboxlower}[2]{%
  \settobox@vert\lower{#1}{#2}%
}
%    \end{macrocode}
%    \end{macro}
%    \begin{macro}{\setboxraise}
%    \begin{macrocode}
\newcommand*{\setboxraise}[2]{%
  \settobox@vert\raise{#1}{#2}%
}
%    \end{macrocode}
%    \end{macro}
%    \begin{macro}{\settobox@length}
%    The work for the \cs{setbox...} commands is done by
%    \cs{settobox@length}. Inside the length expression
%    \cs{width}, \cs{height}, \cs{depth}, \cs{totalheight}
%    are set to the dimensions of the box.\\
%    \begin{tabular}{@{}ll@{}}
%    |#1|:& the property of the box that is to be changed
%           (\cs{wd}, \cs{ht}, \cs{dp})\\
%    |#2|:& the box\\
%    |#3|:& length expression
%    \end{tabular}
%    \begin{macrocode}
\def\settobox@length#1#2#3{%
  \settobox@calc{#2}{#3}{#1#2=##1sp\relax}%
}
%    \end{macrocode}
%    \end{macro}
%
%    \begin{macro}{\settobox@horiz}
%    \begin{macrocode}
\def\settobox@horiz#1#2#3{%
  \settobox@calc{#2}{#3}{\setbox#2=\hbox{\kern#1##1sp\copy#2}}%
}
%    \end{macrocode}
%    \end{macro}
%    \begin{macro}{\settobox@vert}
%    \begin{macrocode}
\def\settobox@vert#1#2#3{%
  \settobox@calc{#2}{#3}{\setbox#2=\hbox{#1##1sp\copy#2}}%
}
%    \end{macrocode}
%    \end{macro}
%
%    \begin{macro}{\settobox@calc}
%    \begin{macrocode}
\def\settobox@calc#1#2#3{%
  \begingroup
    \def\width{\wd#1}%
    \def\height{\ht#1}%
    \def\depth{\dp#1}%
    \dimen@\ht#1\relax
    \advance\dimen@\dp#1\relax
    \def\totalheight{\dimen@}%
    \setlength{\dimen@}{#2}%
    \count@\dimen@
    \def\x##1{\endgroup
      #3%
    }%
  \expandafter\x\expandafter{\the\count@}%
}
%    \end{macrocode}
%    \end{macro}
%
%    \begin{macrocode}
%</package>
%    \end{macrocode}
%
% \section{Installation}
%
% \subsection{Download}
%
% \paragraph{Package.} This package is available on
% CTAN\footnote{\CTANpkg{settobox}}:
% \begin{description}
% \item[\CTAN{macros/latex/contrib/oberdiek/settobox.dtx}] The source file.
% \item[\CTAN{macros/latex/contrib/oberdiek/settobox.pdf}] Documentation.
% \end{description}
%
%
% \paragraph{Bundle.} All the packages of the bundle `oberdiek'
% are also available in a TDS compliant ZIP archive. There
% the packages are already unpacked and the documentation files
% are generated. The files and directories obey the TDS standard.
% \begin{description}
% \item[\CTANinstall{install/macros/latex/contrib/oberdiek.tds.zip}]
% \end{description}
% \emph{TDS} refers to the standard ``A Directory Structure
% for \TeX\ Files'' (\CTANpkg{tds}). Directories
% with \xfile{texmf} in their name are usually organized this way.
%
% \subsection{Bundle installation}
%
% \paragraph{Unpacking.} Unpack the \xfile{oberdiek.tds.zip} in the
% TDS tree (also known as \xfile{texmf} tree) of your choice.
% Example (linux):
% \begin{quote}
%   |unzip oberdiek.tds.zip -d ~/texmf|
% \end{quote}
%
% \subsection{Package installation}
%
% \paragraph{Unpacking.} The \xfile{.dtx} file is a self-extracting
% \docstrip\ archive. The files are extracted by running the
% \xfile{.dtx} through \plainTeX:
% \begin{quote}
%   \verb|tex settobox.dtx|
% \end{quote}
%
% \paragraph{TDS.} Now the different files must be moved into
% the different directories in your installation TDS tree
% (also known as \xfile{texmf} tree):
% \begin{quote}
% \def\t{^^A
% \begin{tabular}{@{}>{\ttfamily}l@{ $\rightarrow$ }>{\ttfamily}l@{}}
%   settobox.sty & tex/latex/oberdiek/settobox.sty\\
%   settobox.pdf & doc/latex/oberdiek/settobox.pdf\\
%   settobox-example.tex & doc/latex/oberdiek/settobox-example.tex\\
%   settobox.dtx & source/latex/oberdiek/settobox.dtx\\
% \end{tabular}^^A
% }^^A
% \sbox0{\t}^^A
% \ifdim\wd0>\linewidth
%   \begingroup
%     \advance\linewidth by\leftmargin
%     \advance\linewidth by\rightmargin
%   \edef\x{\endgroup
%     \def\noexpand\lw{\the\linewidth}^^A
%   }\x
%   \def\lwbox{^^A
%     \leavevmode
%     \hbox to \linewidth{^^A
%       \kern-\leftmargin\relax
%       \hss
%       \usebox0
%       \hss
%       \kern-\rightmargin\relax
%     }^^A
%   }^^A
%   \ifdim\wd0>\lw
%     \sbox0{\small\t}^^A
%     \ifdim\wd0>\linewidth
%       \ifdim\wd0>\lw
%         \sbox0{\footnotesize\t}^^A
%         \ifdim\wd0>\linewidth
%           \ifdim\wd0>\lw
%             \sbox0{\scriptsize\t}^^A
%             \ifdim\wd0>\linewidth
%               \ifdim\wd0>\lw
%                 \sbox0{\tiny\t}^^A
%                 \ifdim\wd0>\linewidth
%                   \lwbox
%                 \else
%                   \usebox0
%                 \fi
%               \else
%                 \lwbox
%               \fi
%             \else
%               \usebox0
%             \fi
%           \else
%             \lwbox
%           \fi
%         \else
%           \usebox0
%         \fi
%       \else
%         \lwbox
%       \fi
%     \else
%       \usebox0
%     \fi
%   \else
%     \lwbox
%   \fi
% \else
%   \usebox0
% \fi
% \end{quote}
% If you have a \xfile{docstrip.cfg} that configures and enables \docstrip's
% TDS installing feature, then some files can already be in the right
% place, see the documentation of \docstrip.
%
% \subsection{Refresh file name databases}
%
% If your \TeX~distribution
% (\TeX\,Live, \mikTeX, \dots) relies on file name databases, you must refresh
% these. For example, \TeX\,Live\ users run \verb|texhash| or
% \verb|mktexlsr|.
%
% \subsection{Some details for the interested}
%
% \paragraph{Unpacking with \LaTeX.}
% The \xfile{.dtx} chooses its action depending on the format:
% \begin{description}
% \item[\plainTeX:] Run \docstrip\ and extract the files.
% \item[\LaTeX:] Generate the documentation.
% \end{description}
% If you insist on using \LaTeX\ for \docstrip\ (really,
% \docstrip\ does not need \LaTeX), then inform the autodetect routine
% about your intention:
% \begin{quote}
%   \verb|latex \let\install=y% \iffalse meta-comment
%
% File: settobox.dtx
% Version: 2016/05/16 v1.5
% Info: Assign box dimensions to length registers
%
% Copyright (C)
%    2000, 2006-2008 Heiko Oberdiek
%    2016-2019 Oberdiek Package Support Group
%    https://github.com/ho-tex/oberdiek/issues
%
% This work may be distributed and/or modified under the
% conditions of the LaTeX Project Public License, either
% version 1.3c of this license or (at your option) any later
% version. This version of this license is in
%    https://www.latex-project.org/lppl/lppl-1-3c.txt
% and the latest version of this license is in
%    https://www.latex-project.org/lppl.txt
% and version 1.3 or later is part of all distributions of
% LaTeX version 2005/12/01 or later.
%
% This work has the LPPL maintenance status "maintained".
%
% The Current Maintainers of this work are
% Heiko Oberdiek and the Oberdiek Package Support Group
% https://github.com/ho-tex/oberdiek/issues
%
% This work consists of the main source file settobox.dtx
% and the derived files
%    settobox.sty, settobox.pdf, settobox.ins, settobox.drv,
%    settobox-example.tex.
%
% Distribution:
%    CTAN:macros/latex/contrib/oberdiek/settobox.dtx
%    CTAN:macros/latex/contrib/oberdiek/settobox.pdf
%
% Unpacking:
%    (a) If settobox.ins is present:
%           tex settobox.ins
%    (b) Without settobox.ins:
%           tex settobox.dtx
%    (c) If you insist on using LaTeX
%           latex \let\install=y% \iffalse meta-comment
%
% File: settobox.dtx
% Version: 2016/05/16 v1.5
% Info: Assign box dimensions to length registers
%
% Copyright (C)
%    2000, 2006-2008 Heiko Oberdiek
%    2016-2019 Oberdiek Package Support Group
%    https://github.com/ho-tex/oberdiek/issues
%
% This work may be distributed and/or modified under the
% conditions of the LaTeX Project Public License, either
% version 1.3c of this license or (at your option) any later
% version. This version of this license is in
%    https://www.latex-project.org/lppl/lppl-1-3c.txt
% and the latest version of this license is in
%    https://www.latex-project.org/lppl.txt
% and version 1.3 or later is part of all distributions of
% LaTeX version 2005/12/01 or later.
%
% This work has the LPPL maintenance status "maintained".
%
% The Current Maintainers of this work are
% Heiko Oberdiek and the Oberdiek Package Support Group
% https://github.com/ho-tex/oberdiek/issues
%
% This work consists of the main source file settobox.dtx
% and the derived files
%    settobox.sty, settobox.pdf, settobox.ins, settobox.drv,
%    settobox-example.tex.
%
% Distribution:
%    CTAN:macros/latex/contrib/oberdiek/settobox.dtx
%    CTAN:macros/latex/contrib/oberdiek/settobox.pdf
%
% Unpacking:
%    (a) If settobox.ins is present:
%           tex settobox.ins
%    (b) Without settobox.ins:
%           tex settobox.dtx
%    (c) If you insist on using LaTeX
%           latex \let\install=y% \iffalse meta-comment
%
% File: settobox.dtx
% Version: 2016/05/16 v1.5
% Info: Assign box dimensions to length registers
%
% Copyright (C)
%    2000, 2006-2008 Heiko Oberdiek
%    2016-2019 Oberdiek Package Support Group
%    https://github.com/ho-tex/oberdiek/issues
%
% This work may be distributed and/or modified under the
% conditions of the LaTeX Project Public License, either
% version 1.3c of this license or (at your option) any later
% version. This version of this license is in
%    https://www.latex-project.org/lppl/lppl-1-3c.txt
% and the latest version of this license is in
%    https://www.latex-project.org/lppl.txt
% and version 1.3 or later is part of all distributions of
% LaTeX version 2005/12/01 or later.
%
% This work has the LPPL maintenance status "maintained".
%
% The Current Maintainers of this work are
% Heiko Oberdiek and the Oberdiek Package Support Group
% https://github.com/ho-tex/oberdiek/issues
%
% This work consists of the main source file settobox.dtx
% and the derived files
%    settobox.sty, settobox.pdf, settobox.ins, settobox.drv,
%    settobox-example.tex.
%
% Distribution:
%    CTAN:macros/latex/contrib/oberdiek/settobox.dtx
%    CTAN:macros/latex/contrib/oberdiek/settobox.pdf
%
% Unpacking:
%    (a) If settobox.ins is present:
%           tex settobox.ins
%    (b) Without settobox.ins:
%           tex settobox.dtx
%    (c) If you insist on using LaTeX
%           latex \let\install=y\input{settobox.dtx}
%        (quote the arguments according to the demands of your shell)
%
% Documentation:
%    (a) If settobox.drv is present:
%           latex settobox.drv
%    (b) Without settobox.drv:
%           latex settobox.dtx; ...
%    The class ltxdoc loads the configuration file ltxdoc.cfg
%    if available. Here you can specify further options, e.g.
%    use A4 as paper format:
%       \PassOptionsToClass{a4paper}{article}
%
%    Programm calls to get the documentation (example):
%       pdflatex settobox.dtx
%       makeindex -s gind.ist settobox.idx
%       pdflatex settobox.dtx
%       makeindex -s gind.ist settobox.idx
%       pdflatex settobox.dtx
%
% Installation:
%    TDS:tex/latex/oberdiek/settobox.sty
%    TDS:doc/latex/oberdiek/settobox.pdf
%    TDS:doc/latex/oberdiek/settobox-example.tex
%    TDS:source/latex/oberdiek/settobox.dtx
%
%<*ignore>
\begingroup
  \catcode123=1 %
  \catcode125=2 %
  \def\x{LaTeX2e}%
\expandafter\endgroup
\ifcase 0\ifx\install y1\fi\expandafter
         \ifx\csname processbatchFile\endcsname\relax\else1\fi
         \ifx\fmtname\x\else 1\fi\relax
\else\csname fi\endcsname
%</ignore>
%<*install>
\input docstrip.tex
\Msg{************************************************************************}
\Msg{* Installation}
\Msg{* Package: settobox 2016/05/16 v1.5 Assign box dimensions to length registers (HO)}
\Msg{************************************************************************}

\keepsilent
\askforoverwritefalse

\let\MetaPrefix\relax
\preamble

This is a generated file.

Project: settobox
Version: 2016/05/16 v1.5

Copyright (C)
   2000, 2006-2008 Heiko Oberdiek
   2016-2019 Oberdiek Package Support Group

This work may be distributed and/or modified under the
conditions of the LaTeX Project Public License, either
version 1.3c of this license or (at your option) any later
version. This version of this license is in
   https://www.latex-project.org/lppl/lppl-1-3c.txt
and the latest version of this license is in
   https://www.latex-project.org/lppl.txt
and version 1.3 or later is part of all distributions of
LaTeX version 2005/12/01 or later.

This work has the LPPL maintenance status "maintained".

The Current Maintainers of this work are
Heiko Oberdiek and the Oberdiek Package Support Group
https://github.com/ho-tex/oberdiek/issues


This work consists of the main source file settobox.dtx
and the derived files
   settobox.sty, settobox.pdf, settobox.ins, settobox.drv,
   settobox-example.tex.

\endpreamble
\let\MetaPrefix\DoubleperCent

\generate{%
  \file{settobox.ins}{\from{settobox.dtx}{install}}%
  \file{settobox.drv}{\from{settobox.dtx}{driver}}%
  \usedir{tex/latex/oberdiek}%
  \file{settobox.sty}{\from{settobox.dtx}{package}}%
  \usedir{doc/latex/oberdiek}%
  \file{settobox-example.tex}{\from{settobox.dtx}{example}}%
}

\catcode32=13\relax% active space
\let =\space%
\Msg{************************************************************************}
\Msg{*}
\Msg{* To finish the installation you have to move the following}
\Msg{* file into a directory searched by TeX:}
\Msg{*}
\Msg{*     settobox.sty}
\Msg{*}
\Msg{* To produce the documentation run the file `settobox.drv'}
\Msg{* through LaTeX.}
\Msg{*}
\Msg{* Happy TeXing!}
\Msg{*}
\Msg{************************************************************************}

\endbatchfile
%</install>
%<*ignore>
\fi
%</ignore>
%<*driver>
\NeedsTeXFormat{LaTeX2e}
\ProvidesFile{settobox.drv}%
  [2016/05/16 v1.5 Assign box dimensions to length registers (HO)]%
\documentclass{ltxdoc}
\usepackage{holtxdoc}[2011/11/22]
\usepackage{calc}
\usepackage{settobox}
\begin{document}
  \DocInput{settobox.dtx}%
\end{document}
%</driver>
% \fi
%
%
%
% \GetFileInfo{settobox.drv}
%
% \title{The \xpackage{settobox} package}
% \date{2016/05/16 v1.5}
% \author{Heiko Oberdiek\thanks
% {Please report any issues at \url{https://github.com/ho-tex/oberdiek/issues}}}
%
% \maketitle
%
% \begin{abstract}
% Commands are defined for getting box sizes similar
% to \LaTeX's \cs{settowidth} commands.
% \end{abstract}
%
% \tableofcontents
%
% \section{Usage}
%
% \subsection{Get box dimensions}
%
% \begin{declcs}^^A
%   {settoboxwidth}\,\M{\LaTeX\ length}\,\M{\LaTeX\ box}\\
%   \SpecialUsageIndex{\settoboxheight}^^A
%   \cs{settoboxheight}\,\M{\LaTeX\ length}\,\M{\LaTeX\ box}\\
%   \SpecialUsageIndex{\settoboxdepth}^^A
%   \cs{settoboxdepth}\,\M{\LaTeX\ length}\,\M{\LaTeX\ box}\\
%   \SpecialUsageIndex{\settoboxtotalheight}^^A
%   \cs{settoboxtotalheight}\,\M{\LaTeX\ length}\,\M{\LaTeX\ box}
% \end{declcs}
% A \meta{\LaTeX\ box} is allocated by \cs{newsavebox}.
% It can be filled by \cs{sbox} or the environment \texttt{lrbox}.
% The commands above extract then the desired lengths.
%
% \subsection{Set box dimensions}
%
% \begin{declcs}^^A
%   {setboxwidth}\,\M{\LaTeX\ box}\,\M{\LaTeX\ length expression}\\
%   \SpecialUsageIndex{\setboxheight}^^A
%   \cs{setboxheight}\,\M{\LaTeX\ box}\,\M{\LaTeX\ length expression}\\
%   \SpecialUsageIndex{\setboxdepth}^^A
%   \cs{setboxdepth}\,\M{\LaTeX\ box}\,\M{\LaTeX\ length expression}
% \end{declcs}
% These commands allow the manipulation of the box. Package \xpackage{calc}
% is supported in the \meta{\LaTeX\ length expression}.
% Also the following length are available in this expression:
% \begin{quote}
% \begin{tabular}{@{}ll@{}}
%   \cs{width}& width of the box\\
%   \cs{height}& height of the box\\
%   \cs{depth}& depth of the box\\
%   \cs{totalheight}& totalheight of the box\\
% \end{tabular}
% \end{quote}
% Note, the base point (point at the left margin of the baseline)
% always remain constant.
%
% \subsection{Move box}
%
% \begin{declcs}^^A
%   {setboxmoveleft}\,\M{\LaTeX\ box}\,\M{\LaTeX\ length expression}\\
%   \SpecialUsageIndex{\setboxmoveright}^^A
%   \cs{setboxmoveright}\,\M{\LaTeX\ box}\,\M{\LaTeX\ length expression}\\
%   \SpecialUsageIndex{\setboxlower}^^A
%   \cs{setboxlower}\,\M{\LaTeX\ box}\,\M{\LaTeX\ length expression}\\
%   \SpecialUsageIndex{\setboxright}^^A
%   \cs{setboxright}\,\M{\LaTeX\ box}\,\M{\LaTeX\ length expression}
% \end{declcs}
% Note, the box is shifted relative to the base point. The base point
% is always inside the box, however the width and height of the
% box change along with the movement.
%
% \subsection{Example}
%
% \subsubsection{Short example}
%
% \begin{quote}
%\begin{verbatim}
%\newsavebox{\mybox}
%\newlength{\mylength}
%\sbox{\mybox}{Hello World}
%\settoboxwidth{\mylength}{\mybox}
%\end{verbatim}
% \end{quote}
%
% \subsubsection{Test file that shows box manipulations}
%
%    \begin{macrocode}
%<*example>
%<<END
\documentclass{article}

\usepackage{settobox}
\usepackage{calc}

\newsavebox{\mybox}

\setlength{\fboxsep}{0pt}
\setlength{\parindent}{20pt}
\setlength{\parskip}{10pt}
\pagestyle{empty}

% \test{#1}
% The macro is called with commands in #1 that manipulates
% the box \mybox. These commands along with the result of
% the manipulation is shown. Thus the essence of the
% macro is:
%
%   a) \sbox{\mybox}{The cracy fox.}
%   b) #1 % manipulates \mybox
%   c) Print #1 commands.
%   d) Print box with frame
%
% The implemenation looks more weird:
\makeatletter
\newcommand*{\test}[1]{%
  \par
  \begingroup
    \raggedright
    \edef\x{\detokenize{#1}}%
    \let\do\@makeother
    \dospecials
    \catcode`\~\active
    \catcode`\ =10\relax
    \def~{\\}%
    \noindent
    \texttt{\scantokens\expandafter{\x}}%
    \par
  \endgroup
  \begingroup
    \let~\relax
    \sbox{\mybox}{The cracy fox.}%
     #1%
     A---\fbox{\usebox\mybox}---B%
  \endgroup
  \par
}
\makeatother

\begin{document}

\test{\setboxwidth{\mybox}{1.25\width}}
\test{\setboxheight{\mybox}{0pt}}
\test{\setboxheight{\mybox}{2\height}}
\test{\setboxdepth{\mybox}{\height}}
\test{\setboxmoveleft{\mybox}{5pt}}
\test{%
  \setboxmoveleft{\mybox}{5pt}~%
  \setboxwidth{\mybox}{\width + 5pt}%
}
\test{\setboxmoveright{\mybox}{0.5\width}}
\test{\setboxlower{\mybox}{\height}}
\test{\setboxraise{\mybox}{\depth}}
\test{%
  \setboxmoveright{\mybox}{5pt}~%
  \setboxwidth{\mybox}{\width + 5pt}~%
  \setboxheight{\mybox}{\height + 5pt}~%
  \setboxdepth{\mybox}{\depth + 5pt}%
}

\end{document}
%END
%</example>
%    \end{macrocode}
%
% \noindent
%    The result:
%
% \vspace{1ex}
% \hrule
%
% \begingroup
% \newsavebox{\mybox}
%
% \setlength{\fboxsep}{0pt}
% \setlength{\parindent}{20pt}
% \setlength{\parskip}{10pt}
%
% \makeatletter
% \newcommand*{\test}[1]{^^A
%   \par
%   \begingroup
%     \raggedright
%     \edef\x{\detokenize{#1}}
%     \let\do\@makeother
%     \dospecials
%     \catcode`\~\active
%     \catcode`\ =10\relax
%     \def~{\\}^^A
%     \noindent
%     \texttt{\scantokens\expandafter{\x}}
%     \par
%   \endgroup
%   \begingroup
%     \let~\relax
%     \sbox{\mybox}{The cracy fox.}
%      #1^^A
%      A---\fbox{\usebox\mybox}---B
%   \endgroup
%   \par
% }
% \makeatother
%
% \test{\setboxwidth{\mybox}{1.25\width}}
% \test{\setboxheight{\mybox}{0pt}}
% \test{\setboxheight{\mybox}{2\height}}
% \test{\setboxdepth{\mybox}{\height}}
% \test{\setboxmoveleft{\mybox}{5pt}}
% \test{^^A
%   \setboxmoveleft{\mybox}{5pt}~^^A
%   \setboxwidth{\mybox}{\width + 5pt}^^A
% }
% \test{\setboxmoveright{\mybox}{0.5\width}}
% \test{\setboxlower{\mybox}{\height}}
% \test{\setboxraise{\mybox}{\depth}}
% \test{^^A
%   \setboxmoveright{\mybox}{5pt}~^^A
%   \setboxwidth{\mybox}{\width + 5pt}~^^A
%   \setboxheight{\mybox}{\height + 5pt}~^^A
%   \setboxdepth{\mybox}{\depth + 5pt}^^A
% }
%
% \endgroup
% \vspace{1ex}
% \hrule
% \vspace{4ex}
%
% \StopEventually{
% }
%
% \section{Implementation}
%
%    \begin{macrocode}
%<*package>
%    \end{macrocode}
%    Package identification.
%    \begin{macrocode}
\NeedsTeXFormat{LaTeX2e}
\ProvidesPackage{settobox}%
  [2016/05/16 v1.5 Assign box dimensions to length registers (HO)]
%    \end{macrocode}
%
%    \begin{macrocode}
\newcommand*{\settoboxwidth}[2]{\setlength{#1}{\wd#2}}
\newcommand*{\settoboxheight}[2]{\setlength{#1}{\ht#2}}
\newcommand*{\settoboxdepth}[2]{\setlength{#1}{\dp#2}}
\newcommand*{\settoboxtotalheight}[2]{%
  \setlength{#1}{\ht#2}%
  \addtolength{#1}{\dp#2}%
}
%    \end{macrocode}
%
%    \begin{macro}{\setboxwidth}
%    \begin{macrocode}
\newcommand*{\setboxwidth}[2]{%
  \settobox@length\wd{#1}{#2}%
}
%    \end{macrocode}
%    \end{macro}
%    \begin{macro}{\setboxheight}
%    \begin{macrocode}
\newcommand*{\setboxheight}[2]{%
  \settobox@length\ht{#1}{#2}%
}
%    \end{macrocode}
%    \end{macro}
%    \begin{macro}{\setboxheight}
%    \begin{macrocode}
\newcommand*{\setboxdepth}[2]{%
  \settobox@length\dp{#1}{#2}%
}
%    \end{macrocode}
%    \end{macro}
%    \begin{macro}{\setboxmoveleft}
%    \begin{macrocode}
\newcommand*{\setboxmoveleft}[2]{%
  \settobox@horiz{-}{#1}{#2}%
}
%    \end{macrocode}
%    \end{macro}
%    \begin{macro}{\setboxmoveright}
%    \begin{macrocode}
\newcommand*{\setboxmoveright}[2]{%
  \settobox@horiz{}{#1}{#2}%
}
%    \end{macrocode}
%    \end{macro}
%    \begin{macro}{\setboxlower}
%    \begin{macrocode}
\newcommand*{\setboxlower}[2]{%
  \settobox@vert\lower{#1}{#2}%
}
%    \end{macrocode}
%    \end{macro}
%    \begin{macro}{\setboxraise}
%    \begin{macrocode}
\newcommand*{\setboxraise}[2]{%
  \settobox@vert\raise{#1}{#2}%
}
%    \end{macrocode}
%    \end{macro}
%    \begin{macro}{\settobox@length}
%    The work for the \cs{setbox...} commands is done by
%    \cs{settobox@length}. Inside the length expression
%    \cs{width}, \cs{height}, \cs{depth}, \cs{totalheight}
%    are set to the dimensions of the box.\\
%    \begin{tabular}{@{}ll@{}}
%    |#1|:& the property of the box that is to be changed
%           (\cs{wd}, \cs{ht}, \cs{dp})\\
%    |#2|:& the box\\
%    |#3|:& length expression
%    \end{tabular}
%    \begin{macrocode}
\def\settobox@length#1#2#3{%
  \settobox@calc{#2}{#3}{#1#2=##1sp\relax}%
}
%    \end{macrocode}
%    \end{macro}
%
%    \begin{macro}{\settobox@horiz}
%    \begin{macrocode}
\def\settobox@horiz#1#2#3{%
  \settobox@calc{#2}{#3}{\setbox#2=\hbox{\kern#1##1sp\copy#2}}%
}
%    \end{macrocode}
%    \end{macro}
%    \begin{macro}{\settobox@vert}
%    \begin{macrocode}
\def\settobox@vert#1#2#3{%
  \settobox@calc{#2}{#3}{\setbox#2=\hbox{#1##1sp\copy#2}}%
}
%    \end{macrocode}
%    \end{macro}
%
%    \begin{macro}{\settobox@calc}
%    \begin{macrocode}
\def\settobox@calc#1#2#3{%
  \begingroup
    \def\width{\wd#1}%
    \def\height{\ht#1}%
    \def\depth{\dp#1}%
    \dimen@\ht#1\relax
    \advance\dimen@\dp#1\relax
    \def\totalheight{\dimen@}%
    \setlength{\dimen@}{#2}%
    \count@\dimen@
    \def\x##1{\endgroup
      #3%
    }%
  \expandafter\x\expandafter{\the\count@}%
}
%    \end{macrocode}
%    \end{macro}
%
%    \begin{macrocode}
%</package>
%    \end{macrocode}
%
% \section{Installation}
%
% \subsection{Download}
%
% \paragraph{Package.} This package is available on
% CTAN\footnote{\CTANpkg{settobox}}:
% \begin{description}
% \item[\CTAN{macros/latex/contrib/oberdiek/settobox.dtx}] The source file.
% \item[\CTAN{macros/latex/contrib/oberdiek/settobox.pdf}] Documentation.
% \end{description}
%
%
% \paragraph{Bundle.} All the packages of the bundle `oberdiek'
% are also available in a TDS compliant ZIP archive. There
% the packages are already unpacked and the documentation files
% are generated. The files and directories obey the TDS standard.
% \begin{description}
% \item[\CTANinstall{install/macros/latex/contrib/oberdiek.tds.zip}]
% \end{description}
% \emph{TDS} refers to the standard ``A Directory Structure
% for \TeX\ Files'' (\CTANpkg{tds}). Directories
% with \xfile{texmf} in their name are usually organized this way.
%
% \subsection{Bundle installation}
%
% \paragraph{Unpacking.} Unpack the \xfile{oberdiek.tds.zip} in the
% TDS tree (also known as \xfile{texmf} tree) of your choice.
% Example (linux):
% \begin{quote}
%   |unzip oberdiek.tds.zip -d ~/texmf|
% \end{quote}
%
% \subsection{Package installation}
%
% \paragraph{Unpacking.} The \xfile{.dtx} file is a self-extracting
% \docstrip\ archive. The files are extracted by running the
% \xfile{.dtx} through \plainTeX:
% \begin{quote}
%   \verb|tex settobox.dtx|
% \end{quote}
%
% \paragraph{TDS.} Now the different files must be moved into
% the different directories in your installation TDS tree
% (also known as \xfile{texmf} tree):
% \begin{quote}
% \def\t{^^A
% \begin{tabular}{@{}>{\ttfamily}l@{ $\rightarrow$ }>{\ttfamily}l@{}}
%   settobox.sty & tex/latex/oberdiek/settobox.sty\\
%   settobox.pdf & doc/latex/oberdiek/settobox.pdf\\
%   settobox-example.tex & doc/latex/oberdiek/settobox-example.tex\\
%   settobox.dtx & source/latex/oberdiek/settobox.dtx\\
% \end{tabular}^^A
% }^^A
% \sbox0{\t}^^A
% \ifdim\wd0>\linewidth
%   \begingroup
%     \advance\linewidth by\leftmargin
%     \advance\linewidth by\rightmargin
%   \edef\x{\endgroup
%     \def\noexpand\lw{\the\linewidth}^^A
%   }\x
%   \def\lwbox{^^A
%     \leavevmode
%     \hbox to \linewidth{^^A
%       \kern-\leftmargin\relax
%       \hss
%       \usebox0
%       \hss
%       \kern-\rightmargin\relax
%     }^^A
%   }^^A
%   \ifdim\wd0>\lw
%     \sbox0{\small\t}^^A
%     \ifdim\wd0>\linewidth
%       \ifdim\wd0>\lw
%         \sbox0{\footnotesize\t}^^A
%         \ifdim\wd0>\linewidth
%           \ifdim\wd0>\lw
%             \sbox0{\scriptsize\t}^^A
%             \ifdim\wd0>\linewidth
%               \ifdim\wd0>\lw
%                 \sbox0{\tiny\t}^^A
%                 \ifdim\wd0>\linewidth
%                   \lwbox
%                 \else
%                   \usebox0
%                 \fi
%               \else
%                 \lwbox
%               \fi
%             \else
%               \usebox0
%             \fi
%           \else
%             \lwbox
%           \fi
%         \else
%           \usebox0
%         \fi
%       \else
%         \lwbox
%       \fi
%     \else
%       \usebox0
%     \fi
%   \else
%     \lwbox
%   \fi
% \else
%   \usebox0
% \fi
% \end{quote}
% If you have a \xfile{docstrip.cfg} that configures and enables \docstrip's
% TDS installing feature, then some files can already be in the right
% place, see the documentation of \docstrip.
%
% \subsection{Refresh file name databases}
%
% If your \TeX~distribution
% (\TeX\,Live, \mikTeX, \dots) relies on file name databases, you must refresh
% these. For example, \TeX\,Live\ users run \verb|texhash| or
% \verb|mktexlsr|.
%
% \subsection{Some details for the interested}
%
% \paragraph{Unpacking with \LaTeX.}
% The \xfile{.dtx} chooses its action depending on the format:
% \begin{description}
% \item[\plainTeX:] Run \docstrip\ and extract the files.
% \item[\LaTeX:] Generate the documentation.
% \end{description}
% If you insist on using \LaTeX\ for \docstrip\ (really,
% \docstrip\ does not need \LaTeX), then inform the autodetect routine
% about your intention:
% \begin{quote}
%   \verb|latex \let\install=y\input{settobox.dtx}|
% \end{quote}
% Do not forget to quote the argument according to the demands
% of your shell.
%
% \paragraph{Generating the documentation.}
% You can use both the \xfile{.dtx} or the \xfile{.drv} to generate
% the documentation. The process can be configured by the
% configuration file \xfile{ltxdoc.cfg}. For instance, put this
% line into this file, if you want to have A4 as paper format:
% \begin{quote}
%   \verb|\PassOptionsToClass{a4paper}{article}|
% \end{quote}
% An example follows how to generate the
% documentation with pdf\LaTeX:
% \begin{quote}
%\begin{verbatim}
%pdflatex settobox.dtx
%makeindex -s gind.ist settobox.idx
%pdflatex settobox.dtx
%makeindex -s gind.ist settobox.idx
%pdflatex settobox.dtx
%\end{verbatim}
% \end{quote}
%
% \begin{History}
%   \begin{Version}{2000/02/11 v1.0}
%   \item
%     First public release, written as answer in the
%     newsgroup \xnewsgroup{de.comp.text.tex}:
%     \URL{``\link{Die Hoehe von Minipages und Bild}''}^^A
%     {https://groups.google.com/group/de.comp.text.tex/msg/c3f6446f54f66c02}
%   \end{Version}
%   \begin{Version}{2000/09/07 v1.1}
%   \item
%     Documentation added.
%   \item
%     CTAN release.
%   \end{Version}
%   \begin{Version}{2006/02/20 v1.2}
%   \item
%     \cs{setboxwidth}, \cs{setboxheight}, \cs{setboxdepth} added.
%   \item
%     Box move commands added.
%   \item
%     DTX framework.
%   \item
%     LPPL 1.3
%   \end{Version}
%   \begin{Version}{2007/04/11 v1.3}
%   \item
%     Line ends sanitized.
%   \end{Version}
%   \begin{Version}{2008/08/11 v1.4}
%   \item
%     Code is not changed.
%   \item
%     URLs updated.
%   \end{Version}
%   \begin{Version}{2016/05/16 v1.5}
%   \item
%     Documentation updates.
%   \end{Version}
% \end{History}
%
% \PrintIndex
%
% \Finale
\endinput

%        (quote the arguments according to the demands of your shell)
%
% Documentation:
%    (a) If settobox.drv is present:
%           latex settobox.drv
%    (b) Without settobox.drv:
%           latex settobox.dtx; ...
%    The class ltxdoc loads the configuration file ltxdoc.cfg
%    if available. Here you can specify further options, e.g.
%    use A4 as paper format:
%       \PassOptionsToClass{a4paper}{article}
%
%    Programm calls to get the documentation (example):
%       pdflatex settobox.dtx
%       makeindex -s gind.ist settobox.idx
%       pdflatex settobox.dtx
%       makeindex -s gind.ist settobox.idx
%       pdflatex settobox.dtx
%
% Installation:
%    TDS:tex/latex/oberdiek/settobox.sty
%    TDS:doc/latex/oberdiek/settobox.pdf
%    TDS:doc/latex/oberdiek/settobox-example.tex
%    TDS:source/latex/oberdiek/settobox.dtx
%
%<*ignore>
\begingroup
  \catcode123=1 %
  \catcode125=2 %
  \def\x{LaTeX2e}%
\expandafter\endgroup
\ifcase 0\ifx\install y1\fi\expandafter
         \ifx\csname processbatchFile\endcsname\relax\else1\fi
         \ifx\fmtname\x\else 1\fi\relax
\else\csname fi\endcsname
%</ignore>
%<*install>
\input docstrip.tex
\Msg{************************************************************************}
\Msg{* Installation}
\Msg{* Package: settobox 2016/05/16 v1.5 Assign box dimensions to length registers (HO)}
\Msg{************************************************************************}

\keepsilent
\askforoverwritefalse

\let\MetaPrefix\relax
\preamble

This is a generated file.

Project: settobox
Version: 2016/05/16 v1.5

Copyright (C)
   2000, 2006-2008 Heiko Oberdiek
   2016-2019 Oberdiek Package Support Group

This work may be distributed and/or modified under the
conditions of the LaTeX Project Public License, either
version 1.3c of this license or (at your option) any later
version. This version of this license is in
   https://www.latex-project.org/lppl/lppl-1-3c.txt
and the latest version of this license is in
   https://www.latex-project.org/lppl.txt
and version 1.3 or later is part of all distributions of
LaTeX version 2005/12/01 or later.

This work has the LPPL maintenance status "maintained".

The Current Maintainers of this work are
Heiko Oberdiek and the Oberdiek Package Support Group
https://github.com/ho-tex/oberdiek/issues


This work consists of the main source file settobox.dtx
and the derived files
   settobox.sty, settobox.pdf, settobox.ins, settobox.drv,
   settobox-example.tex.

\endpreamble
\let\MetaPrefix\DoubleperCent

\generate{%
  \file{settobox.ins}{\from{settobox.dtx}{install}}%
  \file{settobox.drv}{\from{settobox.dtx}{driver}}%
  \usedir{tex/latex/oberdiek}%
  \file{settobox.sty}{\from{settobox.dtx}{package}}%
  \usedir{doc/latex/oberdiek}%
  \file{settobox-example.tex}{\from{settobox.dtx}{example}}%
}

\catcode32=13\relax% active space
\let =\space%
\Msg{************************************************************************}
\Msg{*}
\Msg{* To finish the installation you have to move the following}
\Msg{* file into a directory searched by TeX:}
\Msg{*}
\Msg{*     settobox.sty}
\Msg{*}
\Msg{* To produce the documentation run the file `settobox.drv'}
\Msg{* through LaTeX.}
\Msg{*}
\Msg{* Happy TeXing!}
\Msg{*}
\Msg{************************************************************************}

\endbatchfile
%</install>
%<*ignore>
\fi
%</ignore>
%<*driver>
\NeedsTeXFormat{LaTeX2e}
\ProvidesFile{settobox.drv}%
  [2016/05/16 v1.5 Assign box dimensions to length registers (HO)]%
\documentclass{ltxdoc}
\usepackage{holtxdoc}[2011/11/22]
\usepackage{calc}
\usepackage{settobox}
\begin{document}
  \DocInput{settobox.dtx}%
\end{document}
%</driver>
% \fi
%
%
%
% \GetFileInfo{settobox.drv}
%
% \title{The \xpackage{settobox} package}
% \date{2016/05/16 v1.5}
% \author{Heiko Oberdiek\thanks
% {Please report any issues at \url{https://github.com/ho-tex/oberdiek/issues}}}
%
% \maketitle
%
% \begin{abstract}
% Commands are defined for getting box sizes similar
% to \LaTeX's \cs{settowidth} commands.
% \end{abstract}
%
% \tableofcontents
%
% \section{Usage}
%
% \subsection{Get box dimensions}
%
% \begin{declcs}^^A
%   {settoboxwidth}\,\M{\LaTeX\ length}\,\M{\LaTeX\ box}\\
%   \SpecialUsageIndex{\settoboxheight}^^A
%   \cs{settoboxheight}\,\M{\LaTeX\ length}\,\M{\LaTeX\ box}\\
%   \SpecialUsageIndex{\settoboxdepth}^^A
%   \cs{settoboxdepth}\,\M{\LaTeX\ length}\,\M{\LaTeX\ box}\\
%   \SpecialUsageIndex{\settoboxtotalheight}^^A
%   \cs{settoboxtotalheight}\,\M{\LaTeX\ length}\,\M{\LaTeX\ box}
% \end{declcs}
% A \meta{\LaTeX\ box} is allocated by \cs{newsavebox}.
% It can be filled by \cs{sbox} or the environment \texttt{lrbox}.
% The commands above extract then the desired lengths.
%
% \subsection{Set box dimensions}
%
% \begin{declcs}^^A
%   {setboxwidth}\,\M{\LaTeX\ box}\,\M{\LaTeX\ length expression}\\
%   \SpecialUsageIndex{\setboxheight}^^A
%   \cs{setboxheight}\,\M{\LaTeX\ box}\,\M{\LaTeX\ length expression}\\
%   \SpecialUsageIndex{\setboxdepth}^^A
%   \cs{setboxdepth}\,\M{\LaTeX\ box}\,\M{\LaTeX\ length expression}
% \end{declcs}
% These commands allow the manipulation of the box. Package \xpackage{calc}
% is supported in the \meta{\LaTeX\ length expression}.
% Also the following length are available in this expression:
% \begin{quote}
% \begin{tabular}{@{}ll@{}}
%   \cs{width}& width of the box\\
%   \cs{height}& height of the box\\
%   \cs{depth}& depth of the box\\
%   \cs{totalheight}& totalheight of the box\\
% \end{tabular}
% \end{quote}
% Note, the base point (point at the left margin of the baseline)
% always remain constant.
%
% \subsection{Move box}
%
% \begin{declcs}^^A
%   {setboxmoveleft}\,\M{\LaTeX\ box}\,\M{\LaTeX\ length expression}\\
%   \SpecialUsageIndex{\setboxmoveright}^^A
%   \cs{setboxmoveright}\,\M{\LaTeX\ box}\,\M{\LaTeX\ length expression}\\
%   \SpecialUsageIndex{\setboxlower}^^A
%   \cs{setboxlower}\,\M{\LaTeX\ box}\,\M{\LaTeX\ length expression}\\
%   \SpecialUsageIndex{\setboxright}^^A
%   \cs{setboxright}\,\M{\LaTeX\ box}\,\M{\LaTeX\ length expression}
% \end{declcs}
% Note, the box is shifted relative to the base point. The base point
% is always inside the box, however the width and height of the
% box change along with the movement.
%
% \subsection{Example}
%
% \subsubsection{Short example}
%
% \begin{quote}
%\begin{verbatim}
%\newsavebox{\mybox}
%\newlength{\mylength}
%\sbox{\mybox}{Hello World}
%\settoboxwidth{\mylength}{\mybox}
%\end{verbatim}
% \end{quote}
%
% \subsubsection{Test file that shows box manipulations}
%
%    \begin{macrocode}
%<*example>
%<<END
\documentclass{article}

\usepackage{settobox}
\usepackage{calc}

\newsavebox{\mybox}

\setlength{\fboxsep}{0pt}
\setlength{\parindent}{20pt}
\setlength{\parskip}{10pt}
\pagestyle{empty}

% \test{#1}
% The macro is called with commands in #1 that manipulates
% the box \mybox. These commands along with the result of
% the manipulation is shown. Thus the essence of the
% macro is:
%
%   a) \sbox{\mybox}{The cracy fox.}
%   b) #1 % manipulates \mybox
%   c) Print #1 commands.
%   d) Print box with frame
%
% The implemenation looks more weird:
\makeatletter
\newcommand*{\test}[1]{%
  \par
  \begingroup
    \raggedright
    \edef\x{\detokenize{#1}}%
    \let\do\@makeother
    \dospecials
    \catcode`\~\active
    \catcode`\ =10\relax
    \def~{\\}%
    \noindent
    \texttt{\scantokens\expandafter{\x}}%
    \par
  \endgroup
  \begingroup
    \let~\relax
    \sbox{\mybox}{The cracy fox.}%
     #1%
     A---\fbox{\usebox\mybox}---B%
  \endgroup
  \par
}
\makeatother

\begin{document}

\test{\setboxwidth{\mybox}{1.25\width}}
\test{\setboxheight{\mybox}{0pt}}
\test{\setboxheight{\mybox}{2\height}}
\test{\setboxdepth{\mybox}{\height}}
\test{\setboxmoveleft{\mybox}{5pt}}
\test{%
  \setboxmoveleft{\mybox}{5pt}~%
  \setboxwidth{\mybox}{\width + 5pt}%
}
\test{\setboxmoveright{\mybox}{0.5\width}}
\test{\setboxlower{\mybox}{\height}}
\test{\setboxraise{\mybox}{\depth}}
\test{%
  \setboxmoveright{\mybox}{5pt}~%
  \setboxwidth{\mybox}{\width + 5pt}~%
  \setboxheight{\mybox}{\height + 5pt}~%
  \setboxdepth{\mybox}{\depth + 5pt}%
}

\end{document}
%END
%</example>
%    \end{macrocode}
%
% \noindent
%    The result:
%
% \vspace{1ex}
% \hrule
%
% \begingroup
% \newsavebox{\mybox}
%
% \setlength{\fboxsep}{0pt}
% \setlength{\parindent}{20pt}
% \setlength{\parskip}{10pt}
%
% \makeatletter
% \newcommand*{\test}[1]{^^A
%   \par
%   \begingroup
%     \raggedright
%     \edef\x{\detokenize{#1}}
%     \let\do\@makeother
%     \dospecials
%     \catcode`\~\active
%     \catcode`\ =10\relax
%     \def~{\\}^^A
%     \noindent
%     \texttt{\scantokens\expandafter{\x}}
%     \par
%   \endgroup
%   \begingroup
%     \let~\relax
%     \sbox{\mybox}{The cracy fox.}
%      #1^^A
%      A---\fbox{\usebox\mybox}---B
%   \endgroup
%   \par
% }
% \makeatother
%
% \test{\setboxwidth{\mybox}{1.25\width}}
% \test{\setboxheight{\mybox}{0pt}}
% \test{\setboxheight{\mybox}{2\height}}
% \test{\setboxdepth{\mybox}{\height}}
% \test{\setboxmoveleft{\mybox}{5pt}}
% \test{^^A
%   \setboxmoveleft{\mybox}{5pt}~^^A
%   \setboxwidth{\mybox}{\width + 5pt}^^A
% }
% \test{\setboxmoveright{\mybox}{0.5\width}}
% \test{\setboxlower{\mybox}{\height}}
% \test{\setboxraise{\mybox}{\depth}}
% \test{^^A
%   \setboxmoveright{\mybox}{5pt}~^^A
%   \setboxwidth{\mybox}{\width + 5pt}~^^A
%   \setboxheight{\mybox}{\height + 5pt}~^^A
%   \setboxdepth{\mybox}{\depth + 5pt}^^A
% }
%
% \endgroup
% \vspace{1ex}
% \hrule
% \vspace{4ex}
%
% \StopEventually{
% }
%
% \section{Implementation}
%
%    \begin{macrocode}
%<*package>
%    \end{macrocode}
%    Package identification.
%    \begin{macrocode}
\NeedsTeXFormat{LaTeX2e}
\ProvidesPackage{settobox}%
  [2016/05/16 v1.5 Assign box dimensions to length registers (HO)]
%    \end{macrocode}
%
%    \begin{macrocode}
\newcommand*{\settoboxwidth}[2]{\setlength{#1}{\wd#2}}
\newcommand*{\settoboxheight}[2]{\setlength{#1}{\ht#2}}
\newcommand*{\settoboxdepth}[2]{\setlength{#1}{\dp#2}}
\newcommand*{\settoboxtotalheight}[2]{%
  \setlength{#1}{\ht#2}%
  \addtolength{#1}{\dp#2}%
}
%    \end{macrocode}
%
%    \begin{macro}{\setboxwidth}
%    \begin{macrocode}
\newcommand*{\setboxwidth}[2]{%
  \settobox@length\wd{#1}{#2}%
}
%    \end{macrocode}
%    \end{macro}
%    \begin{macro}{\setboxheight}
%    \begin{macrocode}
\newcommand*{\setboxheight}[2]{%
  \settobox@length\ht{#1}{#2}%
}
%    \end{macrocode}
%    \end{macro}
%    \begin{macro}{\setboxheight}
%    \begin{macrocode}
\newcommand*{\setboxdepth}[2]{%
  \settobox@length\dp{#1}{#2}%
}
%    \end{macrocode}
%    \end{macro}
%    \begin{macro}{\setboxmoveleft}
%    \begin{macrocode}
\newcommand*{\setboxmoveleft}[2]{%
  \settobox@horiz{-}{#1}{#2}%
}
%    \end{macrocode}
%    \end{macro}
%    \begin{macro}{\setboxmoveright}
%    \begin{macrocode}
\newcommand*{\setboxmoveright}[2]{%
  \settobox@horiz{}{#1}{#2}%
}
%    \end{macrocode}
%    \end{macro}
%    \begin{macro}{\setboxlower}
%    \begin{macrocode}
\newcommand*{\setboxlower}[2]{%
  \settobox@vert\lower{#1}{#2}%
}
%    \end{macrocode}
%    \end{macro}
%    \begin{macro}{\setboxraise}
%    \begin{macrocode}
\newcommand*{\setboxraise}[2]{%
  \settobox@vert\raise{#1}{#2}%
}
%    \end{macrocode}
%    \end{macro}
%    \begin{macro}{\settobox@length}
%    The work for the \cs{setbox...} commands is done by
%    \cs{settobox@length}. Inside the length expression
%    \cs{width}, \cs{height}, \cs{depth}, \cs{totalheight}
%    are set to the dimensions of the box.\\
%    \begin{tabular}{@{}ll@{}}
%    |#1|:& the property of the box that is to be changed
%           (\cs{wd}, \cs{ht}, \cs{dp})\\
%    |#2|:& the box\\
%    |#3|:& length expression
%    \end{tabular}
%    \begin{macrocode}
\def\settobox@length#1#2#3{%
  \settobox@calc{#2}{#3}{#1#2=##1sp\relax}%
}
%    \end{macrocode}
%    \end{macro}
%
%    \begin{macro}{\settobox@horiz}
%    \begin{macrocode}
\def\settobox@horiz#1#2#3{%
  \settobox@calc{#2}{#3}{\setbox#2=\hbox{\kern#1##1sp\copy#2}}%
}
%    \end{macrocode}
%    \end{macro}
%    \begin{macro}{\settobox@vert}
%    \begin{macrocode}
\def\settobox@vert#1#2#3{%
  \settobox@calc{#2}{#3}{\setbox#2=\hbox{#1##1sp\copy#2}}%
}
%    \end{macrocode}
%    \end{macro}
%
%    \begin{macro}{\settobox@calc}
%    \begin{macrocode}
\def\settobox@calc#1#2#3{%
  \begingroup
    \def\width{\wd#1}%
    \def\height{\ht#1}%
    \def\depth{\dp#1}%
    \dimen@\ht#1\relax
    \advance\dimen@\dp#1\relax
    \def\totalheight{\dimen@}%
    \setlength{\dimen@}{#2}%
    \count@\dimen@
    \def\x##1{\endgroup
      #3%
    }%
  \expandafter\x\expandafter{\the\count@}%
}
%    \end{macrocode}
%    \end{macro}
%
%    \begin{macrocode}
%</package>
%    \end{macrocode}
%
% \section{Installation}
%
% \subsection{Download}
%
% \paragraph{Package.} This package is available on
% CTAN\footnote{\CTANpkg{settobox}}:
% \begin{description}
% \item[\CTAN{macros/latex/contrib/oberdiek/settobox.dtx}] The source file.
% \item[\CTAN{macros/latex/contrib/oberdiek/settobox.pdf}] Documentation.
% \end{description}
%
%
% \paragraph{Bundle.} All the packages of the bundle `oberdiek'
% are also available in a TDS compliant ZIP archive. There
% the packages are already unpacked and the documentation files
% are generated. The files and directories obey the TDS standard.
% \begin{description}
% \item[\CTANinstall{install/macros/latex/contrib/oberdiek.tds.zip}]
% \end{description}
% \emph{TDS} refers to the standard ``A Directory Structure
% for \TeX\ Files'' (\CTANpkg{tds}). Directories
% with \xfile{texmf} in their name are usually organized this way.
%
% \subsection{Bundle installation}
%
% \paragraph{Unpacking.} Unpack the \xfile{oberdiek.tds.zip} in the
% TDS tree (also known as \xfile{texmf} tree) of your choice.
% Example (linux):
% \begin{quote}
%   |unzip oberdiek.tds.zip -d ~/texmf|
% \end{quote}
%
% \subsection{Package installation}
%
% \paragraph{Unpacking.} The \xfile{.dtx} file is a self-extracting
% \docstrip\ archive. The files are extracted by running the
% \xfile{.dtx} through \plainTeX:
% \begin{quote}
%   \verb|tex settobox.dtx|
% \end{quote}
%
% \paragraph{TDS.} Now the different files must be moved into
% the different directories in your installation TDS tree
% (also known as \xfile{texmf} tree):
% \begin{quote}
% \def\t{^^A
% \begin{tabular}{@{}>{\ttfamily}l@{ $\rightarrow$ }>{\ttfamily}l@{}}
%   settobox.sty & tex/latex/oberdiek/settobox.sty\\
%   settobox.pdf & doc/latex/oberdiek/settobox.pdf\\
%   settobox-example.tex & doc/latex/oberdiek/settobox-example.tex\\
%   settobox.dtx & source/latex/oberdiek/settobox.dtx\\
% \end{tabular}^^A
% }^^A
% \sbox0{\t}^^A
% \ifdim\wd0>\linewidth
%   \begingroup
%     \advance\linewidth by\leftmargin
%     \advance\linewidth by\rightmargin
%   \edef\x{\endgroup
%     \def\noexpand\lw{\the\linewidth}^^A
%   }\x
%   \def\lwbox{^^A
%     \leavevmode
%     \hbox to \linewidth{^^A
%       \kern-\leftmargin\relax
%       \hss
%       \usebox0
%       \hss
%       \kern-\rightmargin\relax
%     }^^A
%   }^^A
%   \ifdim\wd0>\lw
%     \sbox0{\small\t}^^A
%     \ifdim\wd0>\linewidth
%       \ifdim\wd0>\lw
%         \sbox0{\footnotesize\t}^^A
%         \ifdim\wd0>\linewidth
%           \ifdim\wd0>\lw
%             \sbox0{\scriptsize\t}^^A
%             \ifdim\wd0>\linewidth
%               \ifdim\wd0>\lw
%                 \sbox0{\tiny\t}^^A
%                 \ifdim\wd0>\linewidth
%                   \lwbox
%                 \else
%                   \usebox0
%                 \fi
%               \else
%                 \lwbox
%               \fi
%             \else
%               \usebox0
%             \fi
%           \else
%             \lwbox
%           \fi
%         \else
%           \usebox0
%         \fi
%       \else
%         \lwbox
%       \fi
%     \else
%       \usebox0
%     \fi
%   \else
%     \lwbox
%   \fi
% \else
%   \usebox0
% \fi
% \end{quote}
% If you have a \xfile{docstrip.cfg} that configures and enables \docstrip's
% TDS installing feature, then some files can already be in the right
% place, see the documentation of \docstrip.
%
% \subsection{Refresh file name databases}
%
% If your \TeX~distribution
% (\TeX\,Live, \mikTeX, \dots) relies on file name databases, you must refresh
% these. For example, \TeX\,Live\ users run \verb|texhash| or
% \verb|mktexlsr|.
%
% \subsection{Some details for the interested}
%
% \paragraph{Unpacking with \LaTeX.}
% The \xfile{.dtx} chooses its action depending on the format:
% \begin{description}
% \item[\plainTeX:] Run \docstrip\ and extract the files.
% \item[\LaTeX:] Generate the documentation.
% \end{description}
% If you insist on using \LaTeX\ for \docstrip\ (really,
% \docstrip\ does not need \LaTeX), then inform the autodetect routine
% about your intention:
% \begin{quote}
%   \verb|latex \let\install=y% \iffalse meta-comment
%
% File: settobox.dtx
% Version: 2016/05/16 v1.5
% Info: Assign box dimensions to length registers
%
% Copyright (C)
%    2000, 2006-2008 Heiko Oberdiek
%    2016-2019 Oberdiek Package Support Group
%    https://github.com/ho-tex/oberdiek/issues
%
% This work may be distributed and/or modified under the
% conditions of the LaTeX Project Public License, either
% version 1.3c of this license or (at your option) any later
% version. This version of this license is in
%    https://www.latex-project.org/lppl/lppl-1-3c.txt
% and the latest version of this license is in
%    https://www.latex-project.org/lppl.txt
% and version 1.3 or later is part of all distributions of
% LaTeX version 2005/12/01 or later.
%
% This work has the LPPL maintenance status "maintained".
%
% The Current Maintainers of this work are
% Heiko Oberdiek and the Oberdiek Package Support Group
% https://github.com/ho-tex/oberdiek/issues
%
% This work consists of the main source file settobox.dtx
% and the derived files
%    settobox.sty, settobox.pdf, settobox.ins, settobox.drv,
%    settobox-example.tex.
%
% Distribution:
%    CTAN:macros/latex/contrib/oberdiek/settobox.dtx
%    CTAN:macros/latex/contrib/oberdiek/settobox.pdf
%
% Unpacking:
%    (a) If settobox.ins is present:
%           tex settobox.ins
%    (b) Without settobox.ins:
%           tex settobox.dtx
%    (c) If you insist on using LaTeX
%           latex \let\install=y\input{settobox.dtx}
%        (quote the arguments according to the demands of your shell)
%
% Documentation:
%    (a) If settobox.drv is present:
%           latex settobox.drv
%    (b) Without settobox.drv:
%           latex settobox.dtx; ...
%    The class ltxdoc loads the configuration file ltxdoc.cfg
%    if available. Here you can specify further options, e.g.
%    use A4 as paper format:
%       \PassOptionsToClass{a4paper}{article}
%
%    Programm calls to get the documentation (example):
%       pdflatex settobox.dtx
%       makeindex -s gind.ist settobox.idx
%       pdflatex settobox.dtx
%       makeindex -s gind.ist settobox.idx
%       pdflatex settobox.dtx
%
% Installation:
%    TDS:tex/latex/oberdiek/settobox.sty
%    TDS:doc/latex/oberdiek/settobox.pdf
%    TDS:doc/latex/oberdiek/settobox-example.tex
%    TDS:source/latex/oberdiek/settobox.dtx
%
%<*ignore>
\begingroup
  \catcode123=1 %
  \catcode125=2 %
  \def\x{LaTeX2e}%
\expandafter\endgroup
\ifcase 0\ifx\install y1\fi\expandafter
         \ifx\csname processbatchFile\endcsname\relax\else1\fi
         \ifx\fmtname\x\else 1\fi\relax
\else\csname fi\endcsname
%</ignore>
%<*install>
\input docstrip.tex
\Msg{************************************************************************}
\Msg{* Installation}
\Msg{* Package: settobox 2016/05/16 v1.5 Assign box dimensions to length registers (HO)}
\Msg{************************************************************************}

\keepsilent
\askforoverwritefalse

\let\MetaPrefix\relax
\preamble

This is a generated file.

Project: settobox
Version: 2016/05/16 v1.5

Copyright (C)
   2000, 2006-2008 Heiko Oberdiek
   2016-2019 Oberdiek Package Support Group

This work may be distributed and/or modified under the
conditions of the LaTeX Project Public License, either
version 1.3c of this license or (at your option) any later
version. This version of this license is in
   https://www.latex-project.org/lppl/lppl-1-3c.txt
and the latest version of this license is in
   https://www.latex-project.org/lppl.txt
and version 1.3 or later is part of all distributions of
LaTeX version 2005/12/01 or later.

This work has the LPPL maintenance status "maintained".

The Current Maintainers of this work are
Heiko Oberdiek and the Oberdiek Package Support Group
https://github.com/ho-tex/oberdiek/issues


This work consists of the main source file settobox.dtx
and the derived files
   settobox.sty, settobox.pdf, settobox.ins, settobox.drv,
   settobox-example.tex.

\endpreamble
\let\MetaPrefix\DoubleperCent

\generate{%
  \file{settobox.ins}{\from{settobox.dtx}{install}}%
  \file{settobox.drv}{\from{settobox.dtx}{driver}}%
  \usedir{tex/latex/oberdiek}%
  \file{settobox.sty}{\from{settobox.dtx}{package}}%
  \usedir{doc/latex/oberdiek}%
  \file{settobox-example.tex}{\from{settobox.dtx}{example}}%
}

\catcode32=13\relax% active space
\let =\space%
\Msg{************************************************************************}
\Msg{*}
\Msg{* To finish the installation you have to move the following}
\Msg{* file into a directory searched by TeX:}
\Msg{*}
\Msg{*     settobox.sty}
\Msg{*}
\Msg{* To produce the documentation run the file `settobox.drv'}
\Msg{* through LaTeX.}
\Msg{*}
\Msg{* Happy TeXing!}
\Msg{*}
\Msg{************************************************************************}

\endbatchfile
%</install>
%<*ignore>
\fi
%</ignore>
%<*driver>
\NeedsTeXFormat{LaTeX2e}
\ProvidesFile{settobox.drv}%
  [2016/05/16 v1.5 Assign box dimensions to length registers (HO)]%
\documentclass{ltxdoc}
\usepackage{holtxdoc}[2011/11/22]
\usepackage{calc}
\usepackage{settobox}
\begin{document}
  \DocInput{settobox.dtx}%
\end{document}
%</driver>
% \fi
%
%
%
% \GetFileInfo{settobox.drv}
%
% \title{The \xpackage{settobox} package}
% \date{2016/05/16 v1.5}
% \author{Heiko Oberdiek\thanks
% {Please report any issues at \url{https://github.com/ho-tex/oberdiek/issues}}}
%
% \maketitle
%
% \begin{abstract}
% Commands are defined for getting box sizes similar
% to \LaTeX's \cs{settowidth} commands.
% \end{abstract}
%
% \tableofcontents
%
% \section{Usage}
%
% \subsection{Get box dimensions}
%
% \begin{declcs}^^A
%   {settoboxwidth}\,\M{\LaTeX\ length}\,\M{\LaTeX\ box}\\
%   \SpecialUsageIndex{\settoboxheight}^^A
%   \cs{settoboxheight}\,\M{\LaTeX\ length}\,\M{\LaTeX\ box}\\
%   \SpecialUsageIndex{\settoboxdepth}^^A
%   \cs{settoboxdepth}\,\M{\LaTeX\ length}\,\M{\LaTeX\ box}\\
%   \SpecialUsageIndex{\settoboxtotalheight}^^A
%   \cs{settoboxtotalheight}\,\M{\LaTeX\ length}\,\M{\LaTeX\ box}
% \end{declcs}
% A \meta{\LaTeX\ box} is allocated by \cs{newsavebox}.
% It can be filled by \cs{sbox} or the environment \texttt{lrbox}.
% The commands above extract then the desired lengths.
%
% \subsection{Set box dimensions}
%
% \begin{declcs}^^A
%   {setboxwidth}\,\M{\LaTeX\ box}\,\M{\LaTeX\ length expression}\\
%   \SpecialUsageIndex{\setboxheight}^^A
%   \cs{setboxheight}\,\M{\LaTeX\ box}\,\M{\LaTeX\ length expression}\\
%   \SpecialUsageIndex{\setboxdepth}^^A
%   \cs{setboxdepth}\,\M{\LaTeX\ box}\,\M{\LaTeX\ length expression}
% \end{declcs}
% These commands allow the manipulation of the box. Package \xpackage{calc}
% is supported in the \meta{\LaTeX\ length expression}.
% Also the following length are available in this expression:
% \begin{quote}
% \begin{tabular}{@{}ll@{}}
%   \cs{width}& width of the box\\
%   \cs{height}& height of the box\\
%   \cs{depth}& depth of the box\\
%   \cs{totalheight}& totalheight of the box\\
% \end{tabular}
% \end{quote}
% Note, the base point (point at the left margin of the baseline)
% always remain constant.
%
% \subsection{Move box}
%
% \begin{declcs}^^A
%   {setboxmoveleft}\,\M{\LaTeX\ box}\,\M{\LaTeX\ length expression}\\
%   \SpecialUsageIndex{\setboxmoveright}^^A
%   \cs{setboxmoveright}\,\M{\LaTeX\ box}\,\M{\LaTeX\ length expression}\\
%   \SpecialUsageIndex{\setboxlower}^^A
%   \cs{setboxlower}\,\M{\LaTeX\ box}\,\M{\LaTeX\ length expression}\\
%   \SpecialUsageIndex{\setboxright}^^A
%   \cs{setboxright}\,\M{\LaTeX\ box}\,\M{\LaTeX\ length expression}
% \end{declcs}
% Note, the box is shifted relative to the base point. The base point
% is always inside the box, however the width and height of the
% box change along with the movement.
%
% \subsection{Example}
%
% \subsubsection{Short example}
%
% \begin{quote}
%\begin{verbatim}
%\newsavebox{\mybox}
%\newlength{\mylength}
%\sbox{\mybox}{Hello World}
%\settoboxwidth{\mylength}{\mybox}
%\end{verbatim}
% \end{quote}
%
% \subsubsection{Test file that shows box manipulations}
%
%    \begin{macrocode}
%<*example>
%<<END
\documentclass{article}

\usepackage{settobox}
\usepackage{calc}

\newsavebox{\mybox}

\setlength{\fboxsep}{0pt}
\setlength{\parindent}{20pt}
\setlength{\parskip}{10pt}
\pagestyle{empty}

% \test{#1}
% The macro is called with commands in #1 that manipulates
% the box \mybox. These commands along with the result of
% the manipulation is shown. Thus the essence of the
% macro is:
%
%   a) \sbox{\mybox}{The cracy fox.}
%   b) #1 % manipulates \mybox
%   c) Print #1 commands.
%   d) Print box with frame
%
% The implemenation looks more weird:
\makeatletter
\newcommand*{\test}[1]{%
  \par
  \begingroup
    \raggedright
    \edef\x{\detokenize{#1}}%
    \let\do\@makeother
    \dospecials
    \catcode`\~\active
    \catcode`\ =10\relax
    \def~{\\}%
    \noindent
    \texttt{\scantokens\expandafter{\x}}%
    \par
  \endgroup
  \begingroup
    \let~\relax
    \sbox{\mybox}{The cracy fox.}%
     #1%
     A---\fbox{\usebox\mybox}---B%
  \endgroup
  \par
}
\makeatother

\begin{document}

\test{\setboxwidth{\mybox}{1.25\width}}
\test{\setboxheight{\mybox}{0pt}}
\test{\setboxheight{\mybox}{2\height}}
\test{\setboxdepth{\mybox}{\height}}
\test{\setboxmoveleft{\mybox}{5pt}}
\test{%
  \setboxmoveleft{\mybox}{5pt}~%
  \setboxwidth{\mybox}{\width + 5pt}%
}
\test{\setboxmoveright{\mybox}{0.5\width}}
\test{\setboxlower{\mybox}{\height}}
\test{\setboxraise{\mybox}{\depth}}
\test{%
  \setboxmoveright{\mybox}{5pt}~%
  \setboxwidth{\mybox}{\width + 5pt}~%
  \setboxheight{\mybox}{\height + 5pt}~%
  \setboxdepth{\mybox}{\depth + 5pt}%
}

\end{document}
%END
%</example>
%    \end{macrocode}
%
% \noindent
%    The result:
%
% \vspace{1ex}
% \hrule
%
% \begingroup
% \newsavebox{\mybox}
%
% \setlength{\fboxsep}{0pt}
% \setlength{\parindent}{20pt}
% \setlength{\parskip}{10pt}
%
% \makeatletter
% \newcommand*{\test}[1]{^^A
%   \par
%   \begingroup
%     \raggedright
%     \edef\x{\detokenize{#1}}
%     \let\do\@makeother
%     \dospecials
%     \catcode`\~\active
%     \catcode`\ =10\relax
%     \def~{\\}^^A
%     \noindent
%     \texttt{\scantokens\expandafter{\x}}
%     \par
%   \endgroup
%   \begingroup
%     \let~\relax
%     \sbox{\mybox}{The cracy fox.}
%      #1^^A
%      A---\fbox{\usebox\mybox}---B
%   \endgroup
%   \par
% }
% \makeatother
%
% \test{\setboxwidth{\mybox}{1.25\width}}
% \test{\setboxheight{\mybox}{0pt}}
% \test{\setboxheight{\mybox}{2\height}}
% \test{\setboxdepth{\mybox}{\height}}
% \test{\setboxmoveleft{\mybox}{5pt}}
% \test{^^A
%   \setboxmoveleft{\mybox}{5pt}~^^A
%   \setboxwidth{\mybox}{\width + 5pt}^^A
% }
% \test{\setboxmoveright{\mybox}{0.5\width}}
% \test{\setboxlower{\mybox}{\height}}
% \test{\setboxraise{\mybox}{\depth}}
% \test{^^A
%   \setboxmoveright{\mybox}{5pt}~^^A
%   \setboxwidth{\mybox}{\width + 5pt}~^^A
%   \setboxheight{\mybox}{\height + 5pt}~^^A
%   \setboxdepth{\mybox}{\depth + 5pt}^^A
% }
%
% \endgroup
% \vspace{1ex}
% \hrule
% \vspace{4ex}
%
% \StopEventually{
% }
%
% \section{Implementation}
%
%    \begin{macrocode}
%<*package>
%    \end{macrocode}
%    Package identification.
%    \begin{macrocode}
\NeedsTeXFormat{LaTeX2e}
\ProvidesPackage{settobox}%
  [2016/05/16 v1.5 Assign box dimensions to length registers (HO)]
%    \end{macrocode}
%
%    \begin{macrocode}
\newcommand*{\settoboxwidth}[2]{\setlength{#1}{\wd#2}}
\newcommand*{\settoboxheight}[2]{\setlength{#1}{\ht#2}}
\newcommand*{\settoboxdepth}[2]{\setlength{#1}{\dp#2}}
\newcommand*{\settoboxtotalheight}[2]{%
  \setlength{#1}{\ht#2}%
  \addtolength{#1}{\dp#2}%
}
%    \end{macrocode}
%
%    \begin{macro}{\setboxwidth}
%    \begin{macrocode}
\newcommand*{\setboxwidth}[2]{%
  \settobox@length\wd{#1}{#2}%
}
%    \end{macrocode}
%    \end{macro}
%    \begin{macro}{\setboxheight}
%    \begin{macrocode}
\newcommand*{\setboxheight}[2]{%
  \settobox@length\ht{#1}{#2}%
}
%    \end{macrocode}
%    \end{macro}
%    \begin{macro}{\setboxheight}
%    \begin{macrocode}
\newcommand*{\setboxdepth}[2]{%
  \settobox@length\dp{#1}{#2}%
}
%    \end{macrocode}
%    \end{macro}
%    \begin{macro}{\setboxmoveleft}
%    \begin{macrocode}
\newcommand*{\setboxmoveleft}[2]{%
  \settobox@horiz{-}{#1}{#2}%
}
%    \end{macrocode}
%    \end{macro}
%    \begin{macro}{\setboxmoveright}
%    \begin{macrocode}
\newcommand*{\setboxmoveright}[2]{%
  \settobox@horiz{}{#1}{#2}%
}
%    \end{macrocode}
%    \end{macro}
%    \begin{macro}{\setboxlower}
%    \begin{macrocode}
\newcommand*{\setboxlower}[2]{%
  \settobox@vert\lower{#1}{#2}%
}
%    \end{macrocode}
%    \end{macro}
%    \begin{macro}{\setboxraise}
%    \begin{macrocode}
\newcommand*{\setboxraise}[2]{%
  \settobox@vert\raise{#1}{#2}%
}
%    \end{macrocode}
%    \end{macro}
%    \begin{macro}{\settobox@length}
%    The work for the \cs{setbox...} commands is done by
%    \cs{settobox@length}. Inside the length expression
%    \cs{width}, \cs{height}, \cs{depth}, \cs{totalheight}
%    are set to the dimensions of the box.\\
%    \begin{tabular}{@{}ll@{}}
%    |#1|:& the property of the box that is to be changed
%           (\cs{wd}, \cs{ht}, \cs{dp})\\
%    |#2|:& the box\\
%    |#3|:& length expression
%    \end{tabular}
%    \begin{macrocode}
\def\settobox@length#1#2#3{%
  \settobox@calc{#2}{#3}{#1#2=##1sp\relax}%
}
%    \end{macrocode}
%    \end{macro}
%
%    \begin{macro}{\settobox@horiz}
%    \begin{macrocode}
\def\settobox@horiz#1#2#3{%
  \settobox@calc{#2}{#3}{\setbox#2=\hbox{\kern#1##1sp\copy#2}}%
}
%    \end{macrocode}
%    \end{macro}
%    \begin{macro}{\settobox@vert}
%    \begin{macrocode}
\def\settobox@vert#1#2#3{%
  \settobox@calc{#2}{#3}{\setbox#2=\hbox{#1##1sp\copy#2}}%
}
%    \end{macrocode}
%    \end{macro}
%
%    \begin{macro}{\settobox@calc}
%    \begin{macrocode}
\def\settobox@calc#1#2#3{%
  \begingroup
    \def\width{\wd#1}%
    \def\height{\ht#1}%
    \def\depth{\dp#1}%
    \dimen@\ht#1\relax
    \advance\dimen@\dp#1\relax
    \def\totalheight{\dimen@}%
    \setlength{\dimen@}{#2}%
    \count@\dimen@
    \def\x##1{\endgroup
      #3%
    }%
  \expandafter\x\expandafter{\the\count@}%
}
%    \end{macrocode}
%    \end{macro}
%
%    \begin{macrocode}
%</package>
%    \end{macrocode}
%
% \section{Installation}
%
% \subsection{Download}
%
% \paragraph{Package.} This package is available on
% CTAN\footnote{\CTANpkg{settobox}}:
% \begin{description}
% \item[\CTAN{macros/latex/contrib/oberdiek/settobox.dtx}] The source file.
% \item[\CTAN{macros/latex/contrib/oberdiek/settobox.pdf}] Documentation.
% \end{description}
%
%
% \paragraph{Bundle.} All the packages of the bundle `oberdiek'
% are also available in a TDS compliant ZIP archive. There
% the packages are already unpacked and the documentation files
% are generated. The files and directories obey the TDS standard.
% \begin{description}
% \item[\CTANinstall{install/macros/latex/contrib/oberdiek.tds.zip}]
% \end{description}
% \emph{TDS} refers to the standard ``A Directory Structure
% for \TeX\ Files'' (\CTANpkg{tds}). Directories
% with \xfile{texmf} in their name are usually organized this way.
%
% \subsection{Bundle installation}
%
% \paragraph{Unpacking.} Unpack the \xfile{oberdiek.tds.zip} in the
% TDS tree (also known as \xfile{texmf} tree) of your choice.
% Example (linux):
% \begin{quote}
%   |unzip oberdiek.tds.zip -d ~/texmf|
% \end{quote}
%
% \subsection{Package installation}
%
% \paragraph{Unpacking.} The \xfile{.dtx} file is a self-extracting
% \docstrip\ archive. The files are extracted by running the
% \xfile{.dtx} through \plainTeX:
% \begin{quote}
%   \verb|tex settobox.dtx|
% \end{quote}
%
% \paragraph{TDS.} Now the different files must be moved into
% the different directories in your installation TDS tree
% (also known as \xfile{texmf} tree):
% \begin{quote}
% \def\t{^^A
% \begin{tabular}{@{}>{\ttfamily}l@{ $\rightarrow$ }>{\ttfamily}l@{}}
%   settobox.sty & tex/latex/oberdiek/settobox.sty\\
%   settobox.pdf & doc/latex/oberdiek/settobox.pdf\\
%   settobox-example.tex & doc/latex/oberdiek/settobox-example.tex\\
%   settobox.dtx & source/latex/oberdiek/settobox.dtx\\
% \end{tabular}^^A
% }^^A
% \sbox0{\t}^^A
% \ifdim\wd0>\linewidth
%   \begingroup
%     \advance\linewidth by\leftmargin
%     \advance\linewidth by\rightmargin
%   \edef\x{\endgroup
%     \def\noexpand\lw{\the\linewidth}^^A
%   }\x
%   \def\lwbox{^^A
%     \leavevmode
%     \hbox to \linewidth{^^A
%       \kern-\leftmargin\relax
%       \hss
%       \usebox0
%       \hss
%       \kern-\rightmargin\relax
%     }^^A
%   }^^A
%   \ifdim\wd0>\lw
%     \sbox0{\small\t}^^A
%     \ifdim\wd0>\linewidth
%       \ifdim\wd0>\lw
%         \sbox0{\footnotesize\t}^^A
%         \ifdim\wd0>\linewidth
%           \ifdim\wd0>\lw
%             \sbox0{\scriptsize\t}^^A
%             \ifdim\wd0>\linewidth
%               \ifdim\wd0>\lw
%                 \sbox0{\tiny\t}^^A
%                 \ifdim\wd0>\linewidth
%                   \lwbox
%                 \else
%                   \usebox0
%                 \fi
%               \else
%                 \lwbox
%               \fi
%             \else
%               \usebox0
%             \fi
%           \else
%             \lwbox
%           \fi
%         \else
%           \usebox0
%         \fi
%       \else
%         \lwbox
%       \fi
%     \else
%       \usebox0
%     \fi
%   \else
%     \lwbox
%   \fi
% \else
%   \usebox0
% \fi
% \end{quote}
% If you have a \xfile{docstrip.cfg} that configures and enables \docstrip's
% TDS installing feature, then some files can already be in the right
% place, see the documentation of \docstrip.
%
% \subsection{Refresh file name databases}
%
% If your \TeX~distribution
% (\TeX\,Live, \mikTeX, \dots) relies on file name databases, you must refresh
% these. For example, \TeX\,Live\ users run \verb|texhash| or
% \verb|mktexlsr|.
%
% \subsection{Some details for the interested}
%
% \paragraph{Unpacking with \LaTeX.}
% The \xfile{.dtx} chooses its action depending on the format:
% \begin{description}
% \item[\plainTeX:] Run \docstrip\ and extract the files.
% \item[\LaTeX:] Generate the documentation.
% \end{description}
% If you insist on using \LaTeX\ for \docstrip\ (really,
% \docstrip\ does not need \LaTeX), then inform the autodetect routine
% about your intention:
% \begin{quote}
%   \verb|latex \let\install=y\input{settobox.dtx}|
% \end{quote}
% Do not forget to quote the argument according to the demands
% of your shell.
%
% \paragraph{Generating the documentation.}
% You can use both the \xfile{.dtx} or the \xfile{.drv} to generate
% the documentation. The process can be configured by the
% configuration file \xfile{ltxdoc.cfg}. For instance, put this
% line into this file, if you want to have A4 as paper format:
% \begin{quote}
%   \verb|\PassOptionsToClass{a4paper}{article}|
% \end{quote}
% An example follows how to generate the
% documentation with pdf\LaTeX:
% \begin{quote}
%\begin{verbatim}
%pdflatex settobox.dtx
%makeindex -s gind.ist settobox.idx
%pdflatex settobox.dtx
%makeindex -s gind.ist settobox.idx
%pdflatex settobox.dtx
%\end{verbatim}
% \end{quote}
%
% \begin{History}
%   \begin{Version}{2000/02/11 v1.0}
%   \item
%     First public release, written as answer in the
%     newsgroup \xnewsgroup{de.comp.text.tex}:
%     \URL{``\link{Die Hoehe von Minipages und Bild}''}^^A
%     {https://groups.google.com/group/de.comp.text.tex/msg/c3f6446f54f66c02}
%   \end{Version}
%   \begin{Version}{2000/09/07 v1.1}
%   \item
%     Documentation added.
%   \item
%     CTAN release.
%   \end{Version}
%   \begin{Version}{2006/02/20 v1.2}
%   \item
%     \cs{setboxwidth}, \cs{setboxheight}, \cs{setboxdepth} added.
%   \item
%     Box move commands added.
%   \item
%     DTX framework.
%   \item
%     LPPL 1.3
%   \end{Version}
%   \begin{Version}{2007/04/11 v1.3}
%   \item
%     Line ends sanitized.
%   \end{Version}
%   \begin{Version}{2008/08/11 v1.4}
%   \item
%     Code is not changed.
%   \item
%     URLs updated.
%   \end{Version}
%   \begin{Version}{2016/05/16 v1.5}
%   \item
%     Documentation updates.
%   \end{Version}
% \end{History}
%
% \PrintIndex
%
% \Finale
\endinput
|
% \end{quote}
% Do not forget to quote the argument according to the demands
% of your shell.
%
% \paragraph{Generating the documentation.}
% You can use both the \xfile{.dtx} or the \xfile{.drv} to generate
% the documentation. The process can be configured by the
% configuration file \xfile{ltxdoc.cfg}. For instance, put this
% line into this file, if you want to have A4 as paper format:
% \begin{quote}
%   \verb|\PassOptionsToClass{a4paper}{article}|
% \end{quote}
% An example follows how to generate the
% documentation with pdf\LaTeX:
% \begin{quote}
%\begin{verbatim}
%pdflatex settobox.dtx
%makeindex -s gind.ist settobox.idx
%pdflatex settobox.dtx
%makeindex -s gind.ist settobox.idx
%pdflatex settobox.dtx
%\end{verbatim}
% \end{quote}
%
% \begin{History}
%   \begin{Version}{2000/02/11 v1.0}
%   \item
%     First public release, written as answer in the
%     newsgroup \xnewsgroup{de.comp.text.tex}:
%     \URL{``\link{Die Hoehe von Minipages und Bild}''}^^A
%     {https://groups.google.com/group/de.comp.text.tex/msg/c3f6446f54f66c02}
%   \end{Version}
%   \begin{Version}{2000/09/07 v1.1}
%   \item
%     Documentation added.
%   \item
%     CTAN release.
%   \end{Version}
%   \begin{Version}{2006/02/20 v1.2}
%   \item
%     \cs{setboxwidth}, \cs{setboxheight}, \cs{setboxdepth} added.
%   \item
%     Box move commands added.
%   \item
%     DTX framework.
%   \item
%     LPPL 1.3
%   \end{Version}
%   \begin{Version}{2007/04/11 v1.3}
%   \item
%     Line ends sanitized.
%   \end{Version}
%   \begin{Version}{2008/08/11 v1.4}
%   \item
%     Code is not changed.
%   \item
%     URLs updated.
%   \end{Version}
%   \begin{Version}{2016/05/16 v1.5}
%   \item
%     Documentation updates.
%   \end{Version}
% \end{History}
%
% \PrintIndex
%
% \Finale
\endinput

%        (quote the arguments according to the demands of your shell)
%
% Documentation:
%    (a) If settobox.drv is present:
%           latex settobox.drv
%    (b) Without settobox.drv:
%           latex settobox.dtx; ...
%    The class ltxdoc loads the configuration file ltxdoc.cfg
%    if available. Here you can specify further options, e.g.
%    use A4 as paper format:
%       \PassOptionsToClass{a4paper}{article}
%
%    Programm calls to get the documentation (example):
%       pdflatex settobox.dtx
%       makeindex -s gind.ist settobox.idx
%       pdflatex settobox.dtx
%       makeindex -s gind.ist settobox.idx
%       pdflatex settobox.dtx
%
% Installation:
%    TDS:tex/latex/oberdiek/settobox.sty
%    TDS:doc/latex/oberdiek/settobox.pdf
%    TDS:doc/latex/oberdiek/settobox-example.tex
%    TDS:source/latex/oberdiek/settobox.dtx
%
%<*ignore>
\begingroup
  \catcode123=1 %
  \catcode125=2 %
  \def\x{LaTeX2e}%
\expandafter\endgroup
\ifcase 0\ifx\install y1\fi\expandafter
         \ifx\csname processbatchFile\endcsname\relax\else1\fi
         \ifx\fmtname\x\else 1\fi\relax
\else\csname fi\endcsname
%</ignore>
%<*install>
\input docstrip.tex
\Msg{************************************************************************}
\Msg{* Installation}
\Msg{* Package: settobox 2016/05/16 v1.5 Assign box dimensions to length registers (HO)}
\Msg{************************************************************************}

\keepsilent
\askforoverwritefalse

\let\MetaPrefix\relax
\preamble

This is a generated file.

Project: settobox
Version: 2016/05/16 v1.5

Copyright (C)
   2000, 2006-2008 Heiko Oberdiek
   2016-2019 Oberdiek Package Support Group

This work may be distributed and/or modified under the
conditions of the LaTeX Project Public License, either
version 1.3c of this license or (at your option) any later
version. This version of this license is in
   https://www.latex-project.org/lppl/lppl-1-3c.txt
and the latest version of this license is in
   https://www.latex-project.org/lppl.txt
and version 1.3 or later is part of all distributions of
LaTeX version 2005/12/01 or later.

This work has the LPPL maintenance status "maintained".

The Current Maintainers of this work are
Heiko Oberdiek and the Oberdiek Package Support Group
https://github.com/ho-tex/oberdiek/issues


This work consists of the main source file settobox.dtx
and the derived files
   settobox.sty, settobox.pdf, settobox.ins, settobox.drv,
   settobox-example.tex.

\endpreamble
\let\MetaPrefix\DoubleperCent

\generate{%
  \file{settobox.ins}{\from{settobox.dtx}{install}}%
  \file{settobox.drv}{\from{settobox.dtx}{driver}}%
  \usedir{tex/latex/oberdiek}%
  \file{settobox.sty}{\from{settobox.dtx}{package}}%
  \usedir{doc/latex/oberdiek}%
  \file{settobox-example.tex}{\from{settobox.dtx}{example}}%
}

\catcode32=13\relax% active space
\let =\space%
\Msg{************************************************************************}
\Msg{*}
\Msg{* To finish the installation you have to move the following}
\Msg{* file into a directory searched by TeX:}
\Msg{*}
\Msg{*     settobox.sty}
\Msg{*}
\Msg{* To produce the documentation run the file `settobox.drv'}
\Msg{* through LaTeX.}
\Msg{*}
\Msg{* Happy TeXing!}
\Msg{*}
\Msg{************************************************************************}

\endbatchfile
%</install>
%<*ignore>
\fi
%</ignore>
%<*driver>
\NeedsTeXFormat{LaTeX2e}
\ProvidesFile{settobox.drv}%
  [2016/05/16 v1.5 Assign box dimensions to length registers (HO)]%
\documentclass{ltxdoc}
\usepackage{holtxdoc}[2011/11/22]
\usepackage{calc}
\usepackage{settobox}
\begin{document}
  \DocInput{settobox.dtx}%
\end{document}
%</driver>
% \fi
%
%
%
% \GetFileInfo{settobox.drv}
%
% \title{The \xpackage{settobox} package}
% \date{2016/05/16 v1.5}
% \author{Heiko Oberdiek\thanks
% {Please report any issues at \url{https://github.com/ho-tex/oberdiek/issues}}}
%
% \maketitle
%
% \begin{abstract}
% Commands are defined for getting box sizes similar
% to \LaTeX's \cs{settowidth} commands.
% \end{abstract}
%
% \tableofcontents
%
% \section{Usage}
%
% \subsection{Get box dimensions}
%
% \begin{declcs}^^A
%   {settoboxwidth}\,\M{\LaTeX\ length}\,\M{\LaTeX\ box}\\
%   \SpecialUsageIndex{\settoboxheight}^^A
%   \cs{settoboxheight}\,\M{\LaTeX\ length}\,\M{\LaTeX\ box}\\
%   \SpecialUsageIndex{\settoboxdepth}^^A
%   \cs{settoboxdepth}\,\M{\LaTeX\ length}\,\M{\LaTeX\ box}\\
%   \SpecialUsageIndex{\settoboxtotalheight}^^A
%   \cs{settoboxtotalheight}\,\M{\LaTeX\ length}\,\M{\LaTeX\ box}
% \end{declcs}
% A \meta{\LaTeX\ box} is allocated by \cs{newsavebox}.
% It can be filled by \cs{sbox} or the environment \texttt{lrbox}.
% The commands above extract then the desired lengths.
%
% \subsection{Set box dimensions}
%
% \begin{declcs}^^A
%   {setboxwidth}\,\M{\LaTeX\ box}\,\M{\LaTeX\ length expression}\\
%   \SpecialUsageIndex{\setboxheight}^^A
%   \cs{setboxheight}\,\M{\LaTeX\ box}\,\M{\LaTeX\ length expression}\\
%   \SpecialUsageIndex{\setboxdepth}^^A
%   \cs{setboxdepth}\,\M{\LaTeX\ box}\,\M{\LaTeX\ length expression}
% \end{declcs}
% These commands allow the manipulation of the box. Package \xpackage{calc}
% is supported in the \meta{\LaTeX\ length expression}.
% Also the following length are available in this expression:
% \begin{quote}
% \begin{tabular}{@{}ll@{}}
%   \cs{width}& width of the box\\
%   \cs{height}& height of the box\\
%   \cs{depth}& depth of the box\\
%   \cs{totalheight}& totalheight of the box\\
% \end{tabular}
% \end{quote}
% Note, the base point (point at the left margin of the baseline)
% always remain constant.
%
% \subsection{Move box}
%
% \begin{declcs}^^A
%   {setboxmoveleft}\,\M{\LaTeX\ box}\,\M{\LaTeX\ length expression}\\
%   \SpecialUsageIndex{\setboxmoveright}^^A
%   \cs{setboxmoveright}\,\M{\LaTeX\ box}\,\M{\LaTeX\ length expression}\\
%   \SpecialUsageIndex{\setboxlower}^^A
%   \cs{setboxlower}\,\M{\LaTeX\ box}\,\M{\LaTeX\ length expression}\\
%   \SpecialUsageIndex{\setboxright}^^A
%   \cs{setboxright}\,\M{\LaTeX\ box}\,\M{\LaTeX\ length expression}
% \end{declcs}
% Note, the box is shifted relative to the base point. The base point
% is always inside the box, however the width and height of the
% box change along with the movement.
%
% \subsection{Example}
%
% \subsubsection{Short example}
%
% \begin{quote}
%\begin{verbatim}
%\newsavebox{\mybox}
%\newlength{\mylength}
%\sbox{\mybox}{Hello World}
%\settoboxwidth{\mylength}{\mybox}
%\end{verbatim}
% \end{quote}
%
% \subsubsection{Test file that shows box manipulations}
%
%    \begin{macrocode}
%<*example>
%<<END
\documentclass{article}

\usepackage{settobox}
\usepackage{calc}

\newsavebox{\mybox}

\setlength{\fboxsep}{0pt}
\setlength{\parindent}{20pt}
\setlength{\parskip}{10pt}
\pagestyle{empty}

% \test{#1}
% The macro is called with commands in #1 that manipulates
% the box \mybox. These commands along with the result of
% the manipulation is shown. Thus the essence of the
% macro is:
%
%   a) \sbox{\mybox}{The cracy fox.}
%   b) #1 % manipulates \mybox
%   c) Print #1 commands.
%   d) Print box with frame
%
% The implemenation looks more weird:
\makeatletter
\newcommand*{\test}[1]{%
  \par
  \begingroup
    \raggedright
    \edef\x{\detokenize{#1}}%
    \let\do\@makeother
    \dospecials
    \catcode`\~\active
    \catcode`\ =10\relax
    \def~{\\}%
    \noindent
    \texttt{\scantokens\expandafter{\x}}%
    \par
  \endgroup
  \begingroup
    \let~\relax
    \sbox{\mybox}{The cracy fox.}%
     #1%
     A---\fbox{\usebox\mybox}---B%
  \endgroup
  \par
}
\makeatother

\begin{document}

\test{\setboxwidth{\mybox}{1.25\width}}
\test{\setboxheight{\mybox}{0pt}}
\test{\setboxheight{\mybox}{2\height}}
\test{\setboxdepth{\mybox}{\height}}
\test{\setboxmoveleft{\mybox}{5pt}}
\test{%
  \setboxmoveleft{\mybox}{5pt}~%
  \setboxwidth{\mybox}{\width + 5pt}%
}
\test{\setboxmoveright{\mybox}{0.5\width}}
\test{\setboxlower{\mybox}{\height}}
\test{\setboxraise{\mybox}{\depth}}
\test{%
  \setboxmoveright{\mybox}{5pt}~%
  \setboxwidth{\mybox}{\width + 5pt}~%
  \setboxheight{\mybox}{\height + 5pt}~%
  \setboxdepth{\mybox}{\depth + 5pt}%
}

\end{document}
%END
%</example>
%    \end{macrocode}
%
% \noindent
%    The result:
%
% \vspace{1ex}
% \hrule
%
% \begingroup
% \newsavebox{\mybox}
%
% \setlength{\fboxsep}{0pt}
% \setlength{\parindent}{20pt}
% \setlength{\parskip}{10pt}
%
% \makeatletter
% \newcommand*{\test}[1]{^^A
%   \par
%   \begingroup
%     \raggedright
%     \edef\x{\detokenize{#1}}
%     \let\do\@makeother
%     \dospecials
%     \catcode`\~\active
%     \catcode`\ =10\relax
%     \def~{\\}^^A
%     \noindent
%     \texttt{\scantokens\expandafter{\x}}
%     \par
%   \endgroup
%   \begingroup
%     \let~\relax
%     \sbox{\mybox}{The cracy fox.}
%      #1^^A
%      A---\fbox{\usebox\mybox}---B
%   \endgroup
%   \par
% }
% \makeatother
%
% \test{\setboxwidth{\mybox}{1.25\width}}
% \test{\setboxheight{\mybox}{0pt}}
% \test{\setboxheight{\mybox}{2\height}}
% \test{\setboxdepth{\mybox}{\height}}
% \test{\setboxmoveleft{\mybox}{5pt}}
% \test{^^A
%   \setboxmoveleft{\mybox}{5pt}~^^A
%   \setboxwidth{\mybox}{\width + 5pt}^^A
% }
% \test{\setboxmoveright{\mybox}{0.5\width}}
% \test{\setboxlower{\mybox}{\height}}
% \test{\setboxraise{\mybox}{\depth}}
% \test{^^A
%   \setboxmoveright{\mybox}{5pt}~^^A
%   \setboxwidth{\mybox}{\width + 5pt}~^^A
%   \setboxheight{\mybox}{\height + 5pt}~^^A
%   \setboxdepth{\mybox}{\depth + 5pt}^^A
% }
%
% \endgroup
% \vspace{1ex}
% \hrule
% \vspace{4ex}
%
% \StopEventually{
% }
%
% \section{Implementation}
%
%    \begin{macrocode}
%<*package>
%    \end{macrocode}
%    Package identification.
%    \begin{macrocode}
\NeedsTeXFormat{LaTeX2e}
\ProvidesPackage{settobox}%
  [2016/05/16 v1.5 Assign box dimensions to length registers (HO)]
%    \end{macrocode}
%
%    \begin{macrocode}
\newcommand*{\settoboxwidth}[2]{\setlength{#1}{\wd#2}}
\newcommand*{\settoboxheight}[2]{\setlength{#1}{\ht#2}}
\newcommand*{\settoboxdepth}[2]{\setlength{#1}{\dp#2}}
\newcommand*{\settoboxtotalheight}[2]{%
  \setlength{#1}{\ht#2}%
  \addtolength{#1}{\dp#2}%
}
%    \end{macrocode}
%
%    \begin{macro}{\setboxwidth}
%    \begin{macrocode}
\newcommand*{\setboxwidth}[2]{%
  \settobox@length\wd{#1}{#2}%
}
%    \end{macrocode}
%    \end{macro}
%    \begin{macro}{\setboxheight}
%    \begin{macrocode}
\newcommand*{\setboxheight}[2]{%
  \settobox@length\ht{#1}{#2}%
}
%    \end{macrocode}
%    \end{macro}
%    \begin{macro}{\setboxheight}
%    \begin{macrocode}
\newcommand*{\setboxdepth}[2]{%
  \settobox@length\dp{#1}{#2}%
}
%    \end{macrocode}
%    \end{macro}
%    \begin{macro}{\setboxmoveleft}
%    \begin{macrocode}
\newcommand*{\setboxmoveleft}[2]{%
  \settobox@horiz{-}{#1}{#2}%
}
%    \end{macrocode}
%    \end{macro}
%    \begin{macro}{\setboxmoveright}
%    \begin{macrocode}
\newcommand*{\setboxmoveright}[2]{%
  \settobox@horiz{}{#1}{#2}%
}
%    \end{macrocode}
%    \end{macro}
%    \begin{macro}{\setboxlower}
%    \begin{macrocode}
\newcommand*{\setboxlower}[2]{%
  \settobox@vert\lower{#1}{#2}%
}
%    \end{macrocode}
%    \end{macro}
%    \begin{macro}{\setboxraise}
%    \begin{macrocode}
\newcommand*{\setboxraise}[2]{%
  \settobox@vert\raise{#1}{#2}%
}
%    \end{macrocode}
%    \end{macro}
%    \begin{macro}{\settobox@length}
%    The work for the \cs{setbox...} commands is done by
%    \cs{settobox@length}. Inside the length expression
%    \cs{width}, \cs{height}, \cs{depth}, \cs{totalheight}
%    are set to the dimensions of the box.\\
%    \begin{tabular}{@{}ll@{}}
%    |#1|:& the property of the box that is to be changed
%           (\cs{wd}, \cs{ht}, \cs{dp})\\
%    |#2|:& the box\\
%    |#3|:& length expression
%    \end{tabular}
%    \begin{macrocode}
\def\settobox@length#1#2#3{%
  \settobox@calc{#2}{#3}{#1#2=##1sp\relax}%
}
%    \end{macrocode}
%    \end{macro}
%
%    \begin{macro}{\settobox@horiz}
%    \begin{macrocode}
\def\settobox@horiz#1#2#3{%
  \settobox@calc{#2}{#3}{\setbox#2=\hbox{\kern#1##1sp\copy#2}}%
}
%    \end{macrocode}
%    \end{macro}
%    \begin{macro}{\settobox@vert}
%    \begin{macrocode}
\def\settobox@vert#1#2#3{%
  \settobox@calc{#2}{#3}{\setbox#2=\hbox{#1##1sp\copy#2}}%
}
%    \end{macrocode}
%    \end{macro}
%
%    \begin{macro}{\settobox@calc}
%    \begin{macrocode}
\def\settobox@calc#1#2#3{%
  \begingroup
    \def\width{\wd#1}%
    \def\height{\ht#1}%
    \def\depth{\dp#1}%
    \dimen@\ht#1\relax
    \advance\dimen@\dp#1\relax
    \def\totalheight{\dimen@}%
    \setlength{\dimen@}{#2}%
    \count@\dimen@
    \def\x##1{\endgroup
      #3%
    }%
  \expandafter\x\expandafter{\the\count@}%
}
%    \end{macrocode}
%    \end{macro}
%
%    \begin{macrocode}
%</package>
%    \end{macrocode}
%
% \section{Installation}
%
% \subsection{Download}
%
% \paragraph{Package.} This package is available on
% CTAN\footnote{\CTANpkg{settobox}}:
% \begin{description}
% \item[\CTAN{macros/latex/contrib/oberdiek/settobox.dtx}] The source file.
% \item[\CTAN{macros/latex/contrib/oberdiek/settobox.pdf}] Documentation.
% \end{description}
%
%
% \paragraph{Bundle.} All the packages of the bundle `oberdiek'
% are also available in a TDS compliant ZIP archive. There
% the packages are already unpacked and the documentation files
% are generated. The files and directories obey the TDS standard.
% \begin{description}
% \item[\CTANinstall{install/macros/latex/contrib/oberdiek.tds.zip}]
% \end{description}
% \emph{TDS} refers to the standard ``A Directory Structure
% for \TeX\ Files'' (\CTANpkg{tds}). Directories
% with \xfile{texmf} in their name are usually organized this way.
%
% \subsection{Bundle installation}
%
% \paragraph{Unpacking.} Unpack the \xfile{oberdiek.tds.zip} in the
% TDS tree (also known as \xfile{texmf} tree) of your choice.
% Example (linux):
% \begin{quote}
%   |unzip oberdiek.tds.zip -d ~/texmf|
% \end{quote}
%
% \subsection{Package installation}
%
% \paragraph{Unpacking.} The \xfile{.dtx} file is a self-extracting
% \docstrip\ archive. The files are extracted by running the
% \xfile{.dtx} through \plainTeX:
% \begin{quote}
%   \verb|tex settobox.dtx|
% \end{quote}
%
% \paragraph{TDS.} Now the different files must be moved into
% the different directories in your installation TDS tree
% (also known as \xfile{texmf} tree):
% \begin{quote}
% \def\t{^^A
% \begin{tabular}{@{}>{\ttfamily}l@{ $\rightarrow$ }>{\ttfamily}l@{}}
%   settobox.sty & tex/latex/oberdiek/settobox.sty\\
%   settobox.pdf & doc/latex/oberdiek/settobox.pdf\\
%   settobox-example.tex & doc/latex/oberdiek/settobox-example.tex\\
%   settobox.dtx & source/latex/oberdiek/settobox.dtx\\
% \end{tabular}^^A
% }^^A
% \sbox0{\t}^^A
% \ifdim\wd0>\linewidth
%   \begingroup
%     \advance\linewidth by\leftmargin
%     \advance\linewidth by\rightmargin
%   \edef\x{\endgroup
%     \def\noexpand\lw{\the\linewidth}^^A
%   }\x
%   \def\lwbox{^^A
%     \leavevmode
%     \hbox to \linewidth{^^A
%       \kern-\leftmargin\relax
%       \hss
%       \usebox0
%       \hss
%       \kern-\rightmargin\relax
%     }^^A
%   }^^A
%   \ifdim\wd0>\lw
%     \sbox0{\small\t}^^A
%     \ifdim\wd0>\linewidth
%       \ifdim\wd0>\lw
%         \sbox0{\footnotesize\t}^^A
%         \ifdim\wd0>\linewidth
%           \ifdim\wd0>\lw
%             \sbox0{\scriptsize\t}^^A
%             \ifdim\wd0>\linewidth
%               \ifdim\wd0>\lw
%                 \sbox0{\tiny\t}^^A
%                 \ifdim\wd0>\linewidth
%                   \lwbox
%                 \else
%                   \usebox0
%                 \fi
%               \else
%                 \lwbox
%               \fi
%             \else
%               \usebox0
%             \fi
%           \else
%             \lwbox
%           \fi
%         \else
%           \usebox0
%         \fi
%       \else
%         \lwbox
%       \fi
%     \else
%       \usebox0
%     \fi
%   \else
%     \lwbox
%   \fi
% \else
%   \usebox0
% \fi
% \end{quote}
% If you have a \xfile{docstrip.cfg} that configures and enables \docstrip's
% TDS installing feature, then some files can already be in the right
% place, see the documentation of \docstrip.
%
% \subsection{Refresh file name databases}
%
% If your \TeX~distribution
% (\TeX\,Live, \mikTeX, \dots) relies on file name databases, you must refresh
% these. For example, \TeX\,Live\ users run \verb|texhash| or
% \verb|mktexlsr|.
%
% \subsection{Some details for the interested}
%
% \paragraph{Unpacking with \LaTeX.}
% The \xfile{.dtx} chooses its action depending on the format:
% \begin{description}
% \item[\plainTeX:] Run \docstrip\ and extract the files.
% \item[\LaTeX:] Generate the documentation.
% \end{description}
% If you insist on using \LaTeX\ for \docstrip\ (really,
% \docstrip\ does not need \LaTeX), then inform the autodetect routine
% about your intention:
% \begin{quote}
%   \verb|latex \let\install=y% \iffalse meta-comment
%
% File: settobox.dtx
% Version: 2016/05/16 v1.5
% Info: Assign box dimensions to length registers
%
% Copyright (C)
%    2000, 2006-2008 Heiko Oberdiek
%    2016-2019 Oberdiek Package Support Group
%    https://github.com/ho-tex/oberdiek/issues
%
% This work may be distributed and/or modified under the
% conditions of the LaTeX Project Public License, either
% version 1.3c of this license or (at your option) any later
% version. This version of this license is in
%    https://www.latex-project.org/lppl/lppl-1-3c.txt
% and the latest version of this license is in
%    https://www.latex-project.org/lppl.txt
% and version 1.3 or later is part of all distributions of
% LaTeX version 2005/12/01 or later.
%
% This work has the LPPL maintenance status "maintained".
%
% The Current Maintainers of this work are
% Heiko Oberdiek and the Oberdiek Package Support Group
% https://github.com/ho-tex/oberdiek/issues
%
% This work consists of the main source file settobox.dtx
% and the derived files
%    settobox.sty, settobox.pdf, settobox.ins, settobox.drv,
%    settobox-example.tex.
%
% Distribution:
%    CTAN:macros/latex/contrib/oberdiek/settobox.dtx
%    CTAN:macros/latex/contrib/oberdiek/settobox.pdf
%
% Unpacking:
%    (a) If settobox.ins is present:
%           tex settobox.ins
%    (b) Without settobox.ins:
%           tex settobox.dtx
%    (c) If you insist on using LaTeX
%           latex \let\install=y% \iffalse meta-comment
%
% File: settobox.dtx
% Version: 2016/05/16 v1.5
% Info: Assign box dimensions to length registers
%
% Copyright (C)
%    2000, 2006-2008 Heiko Oberdiek
%    2016-2019 Oberdiek Package Support Group
%    https://github.com/ho-tex/oberdiek/issues
%
% This work may be distributed and/or modified under the
% conditions of the LaTeX Project Public License, either
% version 1.3c of this license or (at your option) any later
% version. This version of this license is in
%    https://www.latex-project.org/lppl/lppl-1-3c.txt
% and the latest version of this license is in
%    https://www.latex-project.org/lppl.txt
% and version 1.3 or later is part of all distributions of
% LaTeX version 2005/12/01 or later.
%
% This work has the LPPL maintenance status "maintained".
%
% The Current Maintainers of this work are
% Heiko Oberdiek and the Oberdiek Package Support Group
% https://github.com/ho-tex/oberdiek/issues
%
% This work consists of the main source file settobox.dtx
% and the derived files
%    settobox.sty, settobox.pdf, settobox.ins, settobox.drv,
%    settobox-example.tex.
%
% Distribution:
%    CTAN:macros/latex/contrib/oberdiek/settobox.dtx
%    CTAN:macros/latex/contrib/oberdiek/settobox.pdf
%
% Unpacking:
%    (a) If settobox.ins is present:
%           tex settobox.ins
%    (b) Without settobox.ins:
%           tex settobox.dtx
%    (c) If you insist on using LaTeX
%           latex \let\install=y\input{settobox.dtx}
%        (quote the arguments according to the demands of your shell)
%
% Documentation:
%    (a) If settobox.drv is present:
%           latex settobox.drv
%    (b) Without settobox.drv:
%           latex settobox.dtx; ...
%    The class ltxdoc loads the configuration file ltxdoc.cfg
%    if available. Here you can specify further options, e.g.
%    use A4 as paper format:
%       \PassOptionsToClass{a4paper}{article}
%
%    Programm calls to get the documentation (example):
%       pdflatex settobox.dtx
%       makeindex -s gind.ist settobox.idx
%       pdflatex settobox.dtx
%       makeindex -s gind.ist settobox.idx
%       pdflatex settobox.dtx
%
% Installation:
%    TDS:tex/latex/oberdiek/settobox.sty
%    TDS:doc/latex/oberdiek/settobox.pdf
%    TDS:doc/latex/oberdiek/settobox-example.tex
%    TDS:source/latex/oberdiek/settobox.dtx
%
%<*ignore>
\begingroup
  \catcode123=1 %
  \catcode125=2 %
  \def\x{LaTeX2e}%
\expandafter\endgroup
\ifcase 0\ifx\install y1\fi\expandafter
         \ifx\csname processbatchFile\endcsname\relax\else1\fi
         \ifx\fmtname\x\else 1\fi\relax
\else\csname fi\endcsname
%</ignore>
%<*install>
\input docstrip.tex
\Msg{************************************************************************}
\Msg{* Installation}
\Msg{* Package: settobox 2016/05/16 v1.5 Assign box dimensions to length registers (HO)}
\Msg{************************************************************************}

\keepsilent
\askforoverwritefalse

\let\MetaPrefix\relax
\preamble

This is a generated file.

Project: settobox
Version: 2016/05/16 v1.5

Copyright (C)
   2000, 2006-2008 Heiko Oberdiek
   2016-2019 Oberdiek Package Support Group

This work may be distributed and/or modified under the
conditions of the LaTeX Project Public License, either
version 1.3c of this license or (at your option) any later
version. This version of this license is in
   https://www.latex-project.org/lppl/lppl-1-3c.txt
and the latest version of this license is in
   https://www.latex-project.org/lppl.txt
and version 1.3 or later is part of all distributions of
LaTeX version 2005/12/01 or later.

This work has the LPPL maintenance status "maintained".

The Current Maintainers of this work are
Heiko Oberdiek and the Oberdiek Package Support Group
https://github.com/ho-tex/oberdiek/issues


This work consists of the main source file settobox.dtx
and the derived files
   settobox.sty, settobox.pdf, settobox.ins, settobox.drv,
   settobox-example.tex.

\endpreamble
\let\MetaPrefix\DoubleperCent

\generate{%
  \file{settobox.ins}{\from{settobox.dtx}{install}}%
  \file{settobox.drv}{\from{settobox.dtx}{driver}}%
  \usedir{tex/latex/oberdiek}%
  \file{settobox.sty}{\from{settobox.dtx}{package}}%
  \usedir{doc/latex/oberdiek}%
  \file{settobox-example.tex}{\from{settobox.dtx}{example}}%
}

\catcode32=13\relax% active space
\let =\space%
\Msg{************************************************************************}
\Msg{*}
\Msg{* To finish the installation you have to move the following}
\Msg{* file into a directory searched by TeX:}
\Msg{*}
\Msg{*     settobox.sty}
\Msg{*}
\Msg{* To produce the documentation run the file `settobox.drv'}
\Msg{* through LaTeX.}
\Msg{*}
\Msg{* Happy TeXing!}
\Msg{*}
\Msg{************************************************************************}

\endbatchfile
%</install>
%<*ignore>
\fi
%</ignore>
%<*driver>
\NeedsTeXFormat{LaTeX2e}
\ProvidesFile{settobox.drv}%
  [2016/05/16 v1.5 Assign box dimensions to length registers (HO)]%
\documentclass{ltxdoc}
\usepackage{holtxdoc}[2011/11/22]
\usepackage{calc}
\usepackage{settobox}
\begin{document}
  \DocInput{settobox.dtx}%
\end{document}
%</driver>
% \fi
%
%
%
% \GetFileInfo{settobox.drv}
%
% \title{The \xpackage{settobox} package}
% \date{2016/05/16 v1.5}
% \author{Heiko Oberdiek\thanks
% {Please report any issues at \url{https://github.com/ho-tex/oberdiek/issues}}}
%
% \maketitle
%
% \begin{abstract}
% Commands are defined for getting box sizes similar
% to \LaTeX's \cs{settowidth} commands.
% \end{abstract}
%
% \tableofcontents
%
% \section{Usage}
%
% \subsection{Get box dimensions}
%
% \begin{declcs}^^A
%   {settoboxwidth}\,\M{\LaTeX\ length}\,\M{\LaTeX\ box}\\
%   \SpecialUsageIndex{\settoboxheight}^^A
%   \cs{settoboxheight}\,\M{\LaTeX\ length}\,\M{\LaTeX\ box}\\
%   \SpecialUsageIndex{\settoboxdepth}^^A
%   \cs{settoboxdepth}\,\M{\LaTeX\ length}\,\M{\LaTeX\ box}\\
%   \SpecialUsageIndex{\settoboxtotalheight}^^A
%   \cs{settoboxtotalheight}\,\M{\LaTeX\ length}\,\M{\LaTeX\ box}
% \end{declcs}
% A \meta{\LaTeX\ box} is allocated by \cs{newsavebox}.
% It can be filled by \cs{sbox} or the environment \texttt{lrbox}.
% The commands above extract then the desired lengths.
%
% \subsection{Set box dimensions}
%
% \begin{declcs}^^A
%   {setboxwidth}\,\M{\LaTeX\ box}\,\M{\LaTeX\ length expression}\\
%   \SpecialUsageIndex{\setboxheight}^^A
%   \cs{setboxheight}\,\M{\LaTeX\ box}\,\M{\LaTeX\ length expression}\\
%   \SpecialUsageIndex{\setboxdepth}^^A
%   \cs{setboxdepth}\,\M{\LaTeX\ box}\,\M{\LaTeX\ length expression}
% \end{declcs}
% These commands allow the manipulation of the box. Package \xpackage{calc}
% is supported in the \meta{\LaTeX\ length expression}.
% Also the following length are available in this expression:
% \begin{quote}
% \begin{tabular}{@{}ll@{}}
%   \cs{width}& width of the box\\
%   \cs{height}& height of the box\\
%   \cs{depth}& depth of the box\\
%   \cs{totalheight}& totalheight of the box\\
% \end{tabular}
% \end{quote}
% Note, the base point (point at the left margin of the baseline)
% always remain constant.
%
% \subsection{Move box}
%
% \begin{declcs}^^A
%   {setboxmoveleft}\,\M{\LaTeX\ box}\,\M{\LaTeX\ length expression}\\
%   \SpecialUsageIndex{\setboxmoveright}^^A
%   \cs{setboxmoveright}\,\M{\LaTeX\ box}\,\M{\LaTeX\ length expression}\\
%   \SpecialUsageIndex{\setboxlower}^^A
%   \cs{setboxlower}\,\M{\LaTeX\ box}\,\M{\LaTeX\ length expression}\\
%   \SpecialUsageIndex{\setboxright}^^A
%   \cs{setboxright}\,\M{\LaTeX\ box}\,\M{\LaTeX\ length expression}
% \end{declcs}
% Note, the box is shifted relative to the base point. The base point
% is always inside the box, however the width and height of the
% box change along with the movement.
%
% \subsection{Example}
%
% \subsubsection{Short example}
%
% \begin{quote}
%\begin{verbatim}
%\newsavebox{\mybox}
%\newlength{\mylength}
%\sbox{\mybox}{Hello World}
%\settoboxwidth{\mylength}{\mybox}
%\end{verbatim}
% \end{quote}
%
% \subsubsection{Test file that shows box manipulations}
%
%    \begin{macrocode}
%<*example>
%<<END
\documentclass{article}

\usepackage{settobox}
\usepackage{calc}

\newsavebox{\mybox}

\setlength{\fboxsep}{0pt}
\setlength{\parindent}{20pt}
\setlength{\parskip}{10pt}
\pagestyle{empty}

% \test{#1}
% The macro is called with commands in #1 that manipulates
% the box \mybox. These commands along with the result of
% the manipulation is shown. Thus the essence of the
% macro is:
%
%   a) \sbox{\mybox}{The cracy fox.}
%   b) #1 % manipulates \mybox
%   c) Print #1 commands.
%   d) Print box with frame
%
% The implemenation looks more weird:
\makeatletter
\newcommand*{\test}[1]{%
  \par
  \begingroup
    \raggedright
    \edef\x{\detokenize{#1}}%
    \let\do\@makeother
    \dospecials
    \catcode`\~\active
    \catcode`\ =10\relax
    \def~{\\}%
    \noindent
    \texttt{\scantokens\expandafter{\x}}%
    \par
  \endgroup
  \begingroup
    \let~\relax
    \sbox{\mybox}{The cracy fox.}%
     #1%
     A---\fbox{\usebox\mybox}---B%
  \endgroup
  \par
}
\makeatother

\begin{document}

\test{\setboxwidth{\mybox}{1.25\width}}
\test{\setboxheight{\mybox}{0pt}}
\test{\setboxheight{\mybox}{2\height}}
\test{\setboxdepth{\mybox}{\height}}
\test{\setboxmoveleft{\mybox}{5pt}}
\test{%
  \setboxmoveleft{\mybox}{5pt}~%
  \setboxwidth{\mybox}{\width + 5pt}%
}
\test{\setboxmoveright{\mybox}{0.5\width}}
\test{\setboxlower{\mybox}{\height}}
\test{\setboxraise{\mybox}{\depth}}
\test{%
  \setboxmoveright{\mybox}{5pt}~%
  \setboxwidth{\mybox}{\width + 5pt}~%
  \setboxheight{\mybox}{\height + 5pt}~%
  \setboxdepth{\mybox}{\depth + 5pt}%
}

\end{document}
%END
%</example>
%    \end{macrocode}
%
% \noindent
%    The result:
%
% \vspace{1ex}
% \hrule
%
% \begingroup
% \newsavebox{\mybox}
%
% \setlength{\fboxsep}{0pt}
% \setlength{\parindent}{20pt}
% \setlength{\parskip}{10pt}
%
% \makeatletter
% \newcommand*{\test}[1]{^^A
%   \par
%   \begingroup
%     \raggedright
%     \edef\x{\detokenize{#1}}
%     \let\do\@makeother
%     \dospecials
%     \catcode`\~\active
%     \catcode`\ =10\relax
%     \def~{\\}^^A
%     \noindent
%     \texttt{\scantokens\expandafter{\x}}
%     \par
%   \endgroup
%   \begingroup
%     \let~\relax
%     \sbox{\mybox}{The cracy fox.}
%      #1^^A
%      A---\fbox{\usebox\mybox}---B
%   \endgroup
%   \par
% }
% \makeatother
%
% \test{\setboxwidth{\mybox}{1.25\width}}
% \test{\setboxheight{\mybox}{0pt}}
% \test{\setboxheight{\mybox}{2\height}}
% \test{\setboxdepth{\mybox}{\height}}
% \test{\setboxmoveleft{\mybox}{5pt}}
% \test{^^A
%   \setboxmoveleft{\mybox}{5pt}~^^A
%   \setboxwidth{\mybox}{\width + 5pt}^^A
% }
% \test{\setboxmoveright{\mybox}{0.5\width}}
% \test{\setboxlower{\mybox}{\height}}
% \test{\setboxraise{\mybox}{\depth}}
% \test{^^A
%   \setboxmoveright{\mybox}{5pt}~^^A
%   \setboxwidth{\mybox}{\width + 5pt}~^^A
%   \setboxheight{\mybox}{\height + 5pt}~^^A
%   \setboxdepth{\mybox}{\depth + 5pt}^^A
% }
%
% \endgroup
% \vspace{1ex}
% \hrule
% \vspace{4ex}
%
% \StopEventually{
% }
%
% \section{Implementation}
%
%    \begin{macrocode}
%<*package>
%    \end{macrocode}
%    Package identification.
%    \begin{macrocode}
\NeedsTeXFormat{LaTeX2e}
\ProvidesPackage{settobox}%
  [2016/05/16 v1.5 Assign box dimensions to length registers (HO)]
%    \end{macrocode}
%
%    \begin{macrocode}
\newcommand*{\settoboxwidth}[2]{\setlength{#1}{\wd#2}}
\newcommand*{\settoboxheight}[2]{\setlength{#1}{\ht#2}}
\newcommand*{\settoboxdepth}[2]{\setlength{#1}{\dp#2}}
\newcommand*{\settoboxtotalheight}[2]{%
  \setlength{#1}{\ht#2}%
  \addtolength{#1}{\dp#2}%
}
%    \end{macrocode}
%
%    \begin{macro}{\setboxwidth}
%    \begin{macrocode}
\newcommand*{\setboxwidth}[2]{%
  \settobox@length\wd{#1}{#2}%
}
%    \end{macrocode}
%    \end{macro}
%    \begin{macro}{\setboxheight}
%    \begin{macrocode}
\newcommand*{\setboxheight}[2]{%
  \settobox@length\ht{#1}{#2}%
}
%    \end{macrocode}
%    \end{macro}
%    \begin{macro}{\setboxheight}
%    \begin{macrocode}
\newcommand*{\setboxdepth}[2]{%
  \settobox@length\dp{#1}{#2}%
}
%    \end{macrocode}
%    \end{macro}
%    \begin{macro}{\setboxmoveleft}
%    \begin{macrocode}
\newcommand*{\setboxmoveleft}[2]{%
  \settobox@horiz{-}{#1}{#2}%
}
%    \end{macrocode}
%    \end{macro}
%    \begin{macro}{\setboxmoveright}
%    \begin{macrocode}
\newcommand*{\setboxmoveright}[2]{%
  \settobox@horiz{}{#1}{#2}%
}
%    \end{macrocode}
%    \end{macro}
%    \begin{macro}{\setboxlower}
%    \begin{macrocode}
\newcommand*{\setboxlower}[2]{%
  \settobox@vert\lower{#1}{#2}%
}
%    \end{macrocode}
%    \end{macro}
%    \begin{macro}{\setboxraise}
%    \begin{macrocode}
\newcommand*{\setboxraise}[2]{%
  \settobox@vert\raise{#1}{#2}%
}
%    \end{macrocode}
%    \end{macro}
%    \begin{macro}{\settobox@length}
%    The work for the \cs{setbox...} commands is done by
%    \cs{settobox@length}. Inside the length expression
%    \cs{width}, \cs{height}, \cs{depth}, \cs{totalheight}
%    are set to the dimensions of the box.\\
%    \begin{tabular}{@{}ll@{}}
%    |#1|:& the property of the box that is to be changed
%           (\cs{wd}, \cs{ht}, \cs{dp})\\
%    |#2|:& the box\\
%    |#3|:& length expression
%    \end{tabular}
%    \begin{macrocode}
\def\settobox@length#1#2#3{%
  \settobox@calc{#2}{#3}{#1#2=##1sp\relax}%
}
%    \end{macrocode}
%    \end{macro}
%
%    \begin{macro}{\settobox@horiz}
%    \begin{macrocode}
\def\settobox@horiz#1#2#3{%
  \settobox@calc{#2}{#3}{\setbox#2=\hbox{\kern#1##1sp\copy#2}}%
}
%    \end{macrocode}
%    \end{macro}
%    \begin{macro}{\settobox@vert}
%    \begin{macrocode}
\def\settobox@vert#1#2#3{%
  \settobox@calc{#2}{#3}{\setbox#2=\hbox{#1##1sp\copy#2}}%
}
%    \end{macrocode}
%    \end{macro}
%
%    \begin{macro}{\settobox@calc}
%    \begin{macrocode}
\def\settobox@calc#1#2#3{%
  \begingroup
    \def\width{\wd#1}%
    \def\height{\ht#1}%
    \def\depth{\dp#1}%
    \dimen@\ht#1\relax
    \advance\dimen@\dp#1\relax
    \def\totalheight{\dimen@}%
    \setlength{\dimen@}{#2}%
    \count@\dimen@
    \def\x##1{\endgroup
      #3%
    }%
  \expandafter\x\expandafter{\the\count@}%
}
%    \end{macrocode}
%    \end{macro}
%
%    \begin{macrocode}
%</package>
%    \end{macrocode}
%
% \section{Installation}
%
% \subsection{Download}
%
% \paragraph{Package.} This package is available on
% CTAN\footnote{\CTANpkg{settobox}}:
% \begin{description}
% \item[\CTAN{macros/latex/contrib/oberdiek/settobox.dtx}] The source file.
% \item[\CTAN{macros/latex/contrib/oberdiek/settobox.pdf}] Documentation.
% \end{description}
%
%
% \paragraph{Bundle.} All the packages of the bundle `oberdiek'
% are also available in a TDS compliant ZIP archive. There
% the packages are already unpacked and the documentation files
% are generated. The files and directories obey the TDS standard.
% \begin{description}
% \item[\CTANinstall{install/macros/latex/contrib/oberdiek.tds.zip}]
% \end{description}
% \emph{TDS} refers to the standard ``A Directory Structure
% for \TeX\ Files'' (\CTANpkg{tds}). Directories
% with \xfile{texmf} in their name are usually organized this way.
%
% \subsection{Bundle installation}
%
% \paragraph{Unpacking.} Unpack the \xfile{oberdiek.tds.zip} in the
% TDS tree (also known as \xfile{texmf} tree) of your choice.
% Example (linux):
% \begin{quote}
%   |unzip oberdiek.tds.zip -d ~/texmf|
% \end{quote}
%
% \subsection{Package installation}
%
% \paragraph{Unpacking.} The \xfile{.dtx} file is a self-extracting
% \docstrip\ archive. The files are extracted by running the
% \xfile{.dtx} through \plainTeX:
% \begin{quote}
%   \verb|tex settobox.dtx|
% \end{quote}
%
% \paragraph{TDS.} Now the different files must be moved into
% the different directories in your installation TDS tree
% (also known as \xfile{texmf} tree):
% \begin{quote}
% \def\t{^^A
% \begin{tabular}{@{}>{\ttfamily}l@{ $\rightarrow$ }>{\ttfamily}l@{}}
%   settobox.sty & tex/latex/oberdiek/settobox.sty\\
%   settobox.pdf & doc/latex/oberdiek/settobox.pdf\\
%   settobox-example.tex & doc/latex/oberdiek/settobox-example.tex\\
%   settobox.dtx & source/latex/oberdiek/settobox.dtx\\
% \end{tabular}^^A
% }^^A
% \sbox0{\t}^^A
% \ifdim\wd0>\linewidth
%   \begingroup
%     \advance\linewidth by\leftmargin
%     \advance\linewidth by\rightmargin
%   \edef\x{\endgroup
%     \def\noexpand\lw{\the\linewidth}^^A
%   }\x
%   \def\lwbox{^^A
%     \leavevmode
%     \hbox to \linewidth{^^A
%       \kern-\leftmargin\relax
%       \hss
%       \usebox0
%       \hss
%       \kern-\rightmargin\relax
%     }^^A
%   }^^A
%   \ifdim\wd0>\lw
%     \sbox0{\small\t}^^A
%     \ifdim\wd0>\linewidth
%       \ifdim\wd0>\lw
%         \sbox0{\footnotesize\t}^^A
%         \ifdim\wd0>\linewidth
%           \ifdim\wd0>\lw
%             \sbox0{\scriptsize\t}^^A
%             \ifdim\wd0>\linewidth
%               \ifdim\wd0>\lw
%                 \sbox0{\tiny\t}^^A
%                 \ifdim\wd0>\linewidth
%                   \lwbox
%                 \else
%                   \usebox0
%                 \fi
%               \else
%                 \lwbox
%               \fi
%             \else
%               \usebox0
%             \fi
%           \else
%             \lwbox
%           \fi
%         \else
%           \usebox0
%         \fi
%       \else
%         \lwbox
%       \fi
%     \else
%       \usebox0
%     \fi
%   \else
%     \lwbox
%   \fi
% \else
%   \usebox0
% \fi
% \end{quote}
% If you have a \xfile{docstrip.cfg} that configures and enables \docstrip's
% TDS installing feature, then some files can already be in the right
% place, see the documentation of \docstrip.
%
% \subsection{Refresh file name databases}
%
% If your \TeX~distribution
% (\TeX\,Live, \mikTeX, \dots) relies on file name databases, you must refresh
% these. For example, \TeX\,Live\ users run \verb|texhash| or
% \verb|mktexlsr|.
%
% \subsection{Some details for the interested}
%
% \paragraph{Unpacking with \LaTeX.}
% The \xfile{.dtx} chooses its action depending on the format:
% \begin{description}
% \item[\plainTeX:] Run \docstrip\ and extract the files.
% \item[\LaTeX:] Generate the documentation.
% \end{description}
% If you insist on using \LaTeX\ for \docstrip\ (really,
% \docstrip\ does not need \LaTeX), then inform the autodetect routine
% about your intention:
% \begin{quote}
%   \verb|latex \let\install=y\input{settobox.dtx}|
% \end{quote}
% Do not forget to quote the argument according to the demands
% of your shell.
%
% \paragraph{Generating the documentation.}
% You can use both the \xfile{.dtx} or the \xfile{.drv} to generate
% the documentation. The process can be configured by the
% configuration file \xfile{ltxdoc.cfg}. For instance, put this
% line into this file, if you want to have A4 as paper format:
% \begin{quote}
%   \verb|\PassOptionsToClass{a4paper}{article}|
% \end{quote}
% An example follows how to generate the
% documentation with pdf\LaTeX:
% \begin{quote}
%\begin{verbatim}
%pdflatex settobox.dtx
%makeindex -s gind.ist settobox.idx
%pdflatex settobox.dtx
%makeindex -s gind.ist settobox.idx
%pdflatex settobox.dtx
%\end{verbatim}
% \end{quote}
%
% \begin{History}
%   \begin{Version}{2000/02/11 v1.0}
%   \item
%     First public release, written as answer in the
%     newsgroup \xnewsgroup{de.comp.text.tex}:
%     \URL{``\link{Die Hoehe von Minipages und Bild}''}^^A
%     {https://groups.google.com/group/de.comp.text.tex/msg/c3f6446f54f66c02}
%   \end{Version}
%   \begin{Version}{2000/09/07 v1.1}
%   \item
%     Documentation added.
%   \item
%     CTAN release.
%   \end{Version}
%   \begin{Version}{2006/02/20 v1.2}
%   \item
%     \cs{setboxwidth}, \cs{setboxheight}, \cs{setboxdepth} added.
%   \item
%     Box move commands added.
%   \item
%     DTX framework.
%   \item
%     LPPL 1.3
%   \end{Version}
%   \begin{Version}{2007/04/11 v1.3}
%   \item
%     Line ends sanitized.
%   \end{Version}
%   \begin{Version}{2008/08/11 v1.4}
%   \item
%     Code is not changed.
%   \item
%     URLs updated.
%   \end{Version}
%   \begin{Version}{2016/05/16 v1.5}
%   \item
%     Documentation updates.
%   \end{Version}
% \end{History}
%
% \PrintIndex
%
% \Finale
\endinput

%        (quote the arguments according to the demands of your shell)
%
% Documentation:
%    (a) If settobox.drv is present:
%           latex settobox.drv
%    (b) Without settobox.drv:
%           latex settobox.dtx; ...
%    The class ltxdoc loads the configuration file ltxdoc.cfg
%    if available. Here you can specify further options, e.g.
%    use A4 as paper format:
%       \PassOptionsToClass{a4paper}{article}
%
%    Programm calls to get the documentation (example):
%       pdflatex settobox.dtx
%       makeindex -s gind.ist settobox.idx
%       pdflatex settobox.dtx
%       makeindex -s gind.ist settobox.idx
%       pdflatex settobox.dtx
%
% Installation:
%    TDS:tex/latex/oberdiek/settobox.sty
%    TDS:doc/latex/oberdiek/settobox.pdf
%    TDS:doc/latex/oberdiek/settobox-example.tex
%    TDS:source/latex/oberdiek/settobox.dtx
%
%<*ignore>
\begingroup
  \catcode123=1 %
  \catcode125=2 %
  \def\x{LaTeX2e}%
\expandafter\endgroup
\ifcase 0\ifx\install y1\fi\expandafter
         \ifx\csname processbatchFile\endcsname\relax\else1\fi
         \ifx\fmtname\x\else 1\fi\relax
\else\csname fi\endcsname
%</ignore>
%<*install>
\input docstrip.tex
\Msg{************************************************************************}
\Msg{* Installation}
\Msg{* Package: settobox 2016/05/16 v1.5 Assign box dimensions to length registers (HO)}
\Msg{************************************************************************}

\keepsilent
\askforoverwritefalse

\let\MetaPrefix\relax
\preamble

This is a generated file.

Project: settobox
Version: 2016/05/16 v1.5

Copyright (C)
   2000, 2006-2008 Heiko Oberdiek
   2016-2019 Oberdiek Package Support Group

This work may be distributed and/or modified under the
conditions of the LaTeX Project Public License, either
version 1.3c of this license or (at your option) any later
version. This version of this license is in
   https://www.latex-project.org/lppl/lppl-1-3c.txt
and the latest version of this license is in
   https://www.latex-project.org/lppl.txt
and version 1.3 or later is part of all distributions of
LaTeX version 2005/12/01 or later.

This work has the LPPL maintenance status "maintained".

The Current Maintainers of this work are
Heiko Oberdiek and the Oberdiek Package Support Group
https://github.com/ho-tex/oberdiek/issues


This work consists of the main source file settobox.dtx
and the derived files
   settobox.sty, settobox.pdf, settobox.ins, settobox.drv,
   settobox-example.tex.

\endpreamble
\let\MetaPrefix\DoubleperCent

\generate{%
  \file{settobox.ins}{\from{settobox.dtx}{install}}%
  \file{settobox.drv}{\from{settobox.dtx}{driver}}%
  \usedir{tex/latex/oberdiek}%
  \file{settobox.sty}{\from{settobox.dtx}{package}}%
  \usedir{doc/latex/oberdiek}%
  \file{settobox-example.tex}{\from{settobox.dtx}{example}}%
}

\catcode32=13\relax% active space
\let =\space%
\Msg{************************************************************************}
\Msg{*}
\Msg{* To finish the installation you have to move the following}
\Msg{* file into a directory searched by TeX:}
\Msg{*}
\Msg{*     settobox.sty}
\Msg{*}
\Msg{* To produce the documentation run the file `settobox.drv'}
\Msg{* through LaTeX.}
\Msg{*}
\Msg{* Happy TeXing!}
\Msg{*}
\Msg{************************************************************************}

\endbatchfile
%</install>
%<*ignore>
\fi
%</ignore>
%<*driver>
\NeedsTeXFormat{LaTeX2e}
\ProvidesFile{settobox.drv}%
  [2016/05/16 v1.5 Assign box dimensions to length registers (HO)]%
\documentclass{ltxdoc}
\usepackage{holtxdoc}[2011/11/22]
\usepackage{calc}
\usepackage{settobox}
\begin{document}
  \DocInput{settobox.dtx}%
\end{document}
%</driver>
% \fi
%
%
%
% \GetFileInfo{settobox.drv}
%
% \title{The \xpackage{settobox} package}
% \date{2016/05/16 v1.5}
% \author{Heiko Oberdiek\thanks
% {Please report any issues at \url{https://github.com/ho-tex/oberdiek/issues}}}
%
% \maketitle
%
% \begin{abstract}
% Commands are defined for getting box sizes similar
% to \LaTeX's \cs{settowidth} commands.
% \end{abstract}
%
% \tableofcontents
%
% \section{Usage}
%
% \subsection{Get box dimensions}
%
% \begin{declcs}^^A
%   {settoboxwidth}\,\M{\LaTeX\ length}\,\M{\LaTeX\ box}\\
%   \SpecialUsageIndex{\settoboxheight}^^A
%   \cs{settoboxheight}\,\M{\LaTeX\ length}\,\M{\LaTeX\ box}\\
%   \SpecialUsageIndex{\settoboxdepth}^^A
%   \cs{settoboxdepth}\,\M{\LaTeX\ length}\,\M{\LaTeX\ box}\\
%   \SpecialUsageIndex{\settoboxtotalheight}^^A
%   \cs{settoboxtotalheight}\,\M{\LaTeX\ length}\,\M{\LaTeX\ box}
% \end{declcs}
% A \meta{\LaTeX\ box} is allocated by \cs{newsavebox}.
% It can be filled by \cs{sbox} or the environment \texttt{lrbox}.
% The commands above extract then the desired lengths.
%
% \subsection{Set box dimensions}
%
% \begin{declcs}^^A
%   {setboxwidth}\,\M{\LaTeX\ box}\,\M{\LaTeX\ length expression}\\
%   \SpecialUsageIndex{\setboxheight}^^A
%   \cs{setboxheight}\,\M{\LaTeX\ box}\,\M{\LaTeX\ length expression}\\
%   \SpecialUsageIndex{\setboxdepth}^^A
%   \cs{setboxdepth}\,\M{\LaTeX\ box}\,\M{\LaTeX\ length expression}
% \end{declcs}
% These commands allow the manipulation of the box. Package \xpackage{calc}
% is supported in the \meta{\LaTeX\ length expression}.
% Also the following length are available in this expression:
% \begin{quote}
% \begin{tabular}{@{}ll@{}}
%   \cs{width}& width of the box\\
%   \cs{height}& height of the box\\
%   \cs{depth}& depth of the box\\
%   \cs{totalheight}& totalheight of the box\\
% \end{tabular}
% \end{quote}
% Note, the base point (point at the left margin of the baseline)
% always remain constant.
%
% \subsection{Move box}
%
% \begin{declcs}^^A
%   {setboxmoveleft}\,\M{\LaTeX\ box}\,\M{\LaTeX\ length expression}\\
%   \SpecialUsageIndex{\setboxmoveright}^^A
%   \cs{setboxmoveright}\,\M{\LaTeX\ box}\,\M{\LaTeX\ length expression}\\
%   \SpecialUsageIndex{\setboxlower}^^A
%   \cs{setboxlower}\,\M{\LaTeX\ box}\,\M{\LaTeX\ length expression}\\
%   \SpecialUsageIndex{\setboxright}^^A
%   \cs{setboxright}\,\M{\LaTeX\ box}\,\M{\LaTeX\ length expression}
% \end{declcs}
% Note, the box is shifted relative to the base point. The base point
% is always inside the box, however the width and height of the
% box change along with the movement.
%
% \subsection{Example}
%
% \subsubsection{Short example}
%
% \begin{quote}
%\begin{verbatim}
%\newsavebox{\mybox}
%\newlength{\mylength}
%\sbox{\mybox}{Hello World}
%\settoboxwidth{\mylength}{\mybox}
%\end{verbatim}
% \end{quote}
%
% \subsubsection{Test file that shows box manipulations}
%
%    \begin{macrocode}
%<*example>
%<<END
\documentclass{article}

\usepackage{settobox}
\usepackage{calc}

\newsavebox{\mybox}

\setlength{\fboxsep}{0pt}
\setlength{\parindent}{20pt}
\setlength{\parskip}{10pt}
\pagestyle{empty}

% \test{#1}
% The macro is called with commands in #1 that manipulates
% the box \mybox. These commands along with the result of
% the manipulation is shown. Thus the essence of the
% macro is:
%
%   a) \sbox{\mybox}{The cracy fox.}
%   b) #1 % manipulates \mybox
%   c) Print #1 commands.
%   d) Print box with frame
%
% The implemenation looks more weird:
\makeatletter
\newcommand*{\test}[1]{%
  \par
  \begingroup
    \raggedright
    \edef\x{\detokenize{#1}}%
    \let\do\@makeother
    \dospecials
    \catcode`\~\active
    \catcode`\ =10\relax
    \def~{\\}%
    \noindent
    \texttt{\scantokens\expandafter{\x}}%
    \par
  \endgroup
  \begingroup
    \let~\relax
    \sbox{\mybox}{The cracy fox.}%
     #1%
     A---\fbox{\usebox\mybox}---B%
  \endgroup
  \par
}
\makeatother

\begin{document}

\test{\setboxwidth{\mybox}{1.25\width}}
\test{\setboxheight{\mybox}{0pt}}
\test{\setboxheight{\mybox}{2\height}}
\test{\setboxdepth{\mybox}{\height}}
\test{\setboxmoveleft{\mybox}{5pt}}
\test{%
  \setboxmoveleft{\mybox}{5pt}~%
  \setboxwidth{\mybox}{\width + 5pt}%
}
\test{\setboxmoveright{\mybox}{0.5\width}}
\test{\setboxlower{\mybox}{\height}}
\test{\setboxraise{\mybox}{\depth}}
\test{%
  \setboxmoveright{\mybox}{5pt}~%
  \setboxwidth{\mybox}{\width + 5pt}~%
  \setboxheight{\mybox}{\height + 5pt}~%
  \setboxdepth{\mybox}{\depth + 5pt}%
}

\end{document}
%END
%</example>
%    \end{macrocode}
%
% \noindent
%    The result:
%
% \vspace{1ex}
% \hrule
%
% \begingroup
% \newsavebox{\mybox}
%
% \setlength{\fboxsep}{0pt}
% \setlength{\parindent}{20pt}
% \setlength{\parskip}{10pt}
%
% \makeatletter
% \newcommand*{\test}[1]{^^A
%   \par
%   \begingroup
%     \raggedright
%     \edef\x{\detokenize{#1}}
%     \let\do\@makeother
%     \dospecials
%     \catcode`\~\active
%     \catcode`\ =10\relax
%     \def~{\\}^^A
%     \noindent
%     \texttt{\scantokens\expandafter{\x}}
%     \par
%   \endgroup
%   \begingroup
%     \let~\relax
%     \sbox{\mybox}{The cracy fox.}
%      #1^^A
%      A---\fbox{\usebox\mybox}---B
%   \endgroup
%   \par
% }
% \makeatother
%
% \test{\setboxwidth{\mybox}{1.25\width}}
% \test{\setboxheight{\mybox}{0pt}}
% \test{\setboxheight{\mybox}{2\height}}
% \test{\setboxdepth{\mybox}{\height}}
% \test{\setboxmoveleft{\mybox}{5pt}}
% \test{^^A
%   \setboxmoveleft{\mybox}{5pt}~^^A
%   \setboxwidth{\mybox}{\width + 5pt}^^A
% }
% \test{\setboxmoveright{\mybox}{0.5\width}}
% \test{\setboxlower{\mybox}{\height}}
% \test{\setboxraise{\mybox}{\depth}}
% \test{^^A
%   \setboxmoveright{\mybox}{5pt}~^^A
%   \setboxwidth{\mybox}{\width + 5pt}~^^A
%   \setboxheight{\mybox}{\height + 5pt}~^^A
%   \setboxdepth{\mybox}{\depth + 5pt}^^A
% }
%
% \endgroup
% \vspace{1ex}
% \hrule
% \vspace{4ex}
%
% \StopEventually{
% }
%
% \section{Implementation}
%
%    \begin{macrocode}
%<*package>
%    \end{macrocode}
%    Package identification.
%    \begin{macrocode}
\NeedsTeXFormat{LaTeX2e}
\ProvidesPackage{settobox}%
  [2016/05/16 v1.5 Assign box dimensions to length registers (HO)]
%    \end{macrocode}
%
%    \begin{macrocode}
\newcommand*{\settoboxwidth}[2]{\setlength{#1}{\wd#2}}
\newcommand*{\settoboxheight}[2]{\setlength{#1}{\ht#2}}
\newcommand*{\settoboxdepth}[2]{\setlength{#1}{\dp#2}}
\newcommand*{\settoboxtotalheight}[2]{%
  \setlength{#1}{\ht#2}%
  \addtolength{#1}{\dp#2}%
}
%    \end{macrocode}
%
%    \begin{macro}{\setboxwidth}
%    \begin{macrocode}
\newcommand*{\setboxwidth}[2]{%
  \settobox@length\wd{#1}{#2}%
}
%    \end{macrocode}
%    \end{macro}
%    \begin{macro}{\setboxheight}
%    \begin{macrocode}
\newcommand*{\setboxheight}[2]{%
  \settobox@length\ht{#1}{#2}%
}
%    \end{macrocode}
%    \end{macro}
%    \begin{macro}{\setboxheight}
%    \begin{macrocode}
\newcommand*{\setboxdepth}[2]{%
  \settobox@length\dp{#1}{#2}%
}
%    \end{macrocode}
%    \end{macro}
%    \begin{macro}{\setboxmoveleft}
%    \begin{macrocode}
\newcommand*{\setboxmoveleft}[2]{%
  \settobox@horiz{-}{#1}{#2}%
}
%    \end{macrocode}
%    \end{macro}
%    \begin{macro}{\setboxmoveright}
%    \begin{macrocode}
\newcommand*{\setboxmoveright}[2]{%
  \settobox@horiz{}{#1}{#2}%
}
%    \end{macrocode}
%    \end{macro}
%    \begin{macro}{\setboxlower}
%    \begin{macrocode}
\newcommand*{\setboxlower}[2]{%
  \settobox@vert\lower{#1}{#2}%
}
%    \end{macrocode}
%    \end{macro}
%    \begin{macro}{\setboxraise}
%    \begin{macrocode}
\newcommand*{\setboxraise}[2]{%
  \settobox@vert\raise{#1}{#2}%
}
%    \end{macrocode}
%    \end{macro}
%    \begin{macro}{\settobox@length}
%    The work for the \cs{setbox...} commands is done by
%    \cs{settobox@length}. Inside the length expression
%    \cs{width}, \cs{height}, \cs{depth}, \cs{totalheight}
%    are set to the dimensions of the box.\\
%    \begin{tabular}{@{}ll@{}}
%    |#1|:& the property of the box that is to be changed
%           (\cs{wd}, \cs{ht}, \cs{dp})\\
%    |#2|:& the box\\
%    |#3|:& length expression
%    \end{tabular}
%    \begin{macrocode}
\def\settobox@length#1#2#3{%
  \settobox@calc{#2}{#3}{#1#2=##1sp\relax}%
}
%    \end{macrocode}
%    \end{macro}
%
%    \begin{macro}{\settobox@horiz}
%    \begin{macrocode}
\def\settobox@horiz#1#2#3{%
  \settobox@calc{#2}{#3}{\setbox#2=\hbox{\kern#1##1sp\copy#2}}%
}
%    \end{macrocode}
%    \end{macro}
%    \begin{macro}{\settobox@vert}
%    \begin{macrocode}
\def\settobox@vert#1#2#3{%
  \settobox@calc{#2}{#3}{\setbox#2=\hbox{#1##1sp\copy#2}}%
}
%    \end{macrocode}
%    \end{macro}
%
%    \begin{macro}{\settobox@calc}
%    \begin{macrocode}
\def\settobox@calc#1#2#3{%
  \begingroup
    \def\width{\wd#1}%
    \def\height{\ht#1}%
    \def\depth{\dp#1}%
    \dimen@\ht#1\relax
    \advance\dimen@\dp#1\relax
    \def\totalheight{\dimen@}%
    \setlength{\dimen@}{#2}%
    \count@\dimen@
    \def\x##1{\endgroup
      #3%
    }%
  \expandafter\x\expandafter{\the\count@}%
}
%    \end{macrocode}
%    \end{macro}
%
%    \begin{macrocode}
%</package>
%    \end{macrocode}
%
% \section{Installation}
%
% \subsection{Download}
%
% \paragraph{Package.} This package is available on
% CTAN\footnote{\CTANpkg{settobox}}:
% \begin{description}
% \item[\CTAN{macros/latex/contrib/oberdiek/settobox.dtx}] The source file.
% \item[\CTAN{macros/latex/contrib/oberdiek/settobox.pdf}] Documentation.
% \end{description}
%
%
% \paragraph{Bundle.} All the packages of the bundle `oberdiek'
% are also available in a TDS compliant ZIP archive. There
% the packages are already unpacked and the documentation files
% are generated. The files and directories obey the TDS standard.
% \begin{description}
% \item[\CTANinstall{install/macros/latex/contrib/oberdiek.tds.zip}]
% \end{description}
% \emph{TDS} refers to the standard ``A Directory Structure
% for \TeX\ Files'' (\CTANpkg{tds}). Directories
% with \xfile{texmf} in their name are usually organized this way.
%
% \subsection{Bundle installation}
%
% \paragraph{Unpacking.} Unpack the \xfile{oberdiek.tds.zip} in the
% TDS tree (also known as \xfile{texmf} tree) of your choice.
% Example (linux):
% \begin{quote}
%   |unzip oberdiek.tds.zip -d ~/texmf|
% \end{quote}
%
% \subsection{Package installation}
%
% \paragraph{Unpacking.} The \xfile{.dtx} file is a self-extracting
% \docstrip\ archive. The files are extracted by running the
% \xfile{.dtx} through \plainTeX:
% \begin{quote}
%   \verb|tex settobox.dtx|
% \end{quote}
%
% \paragraph{TDS.} Now the different files must be moved into
% the different directories in your installation TDS tree
% (also known as \xfile{texmf} tree):
% \begin{quote}
% \def\t{^^A
% \begin{tabular}{@{}>{\ttfamily}l@{ $\rightarrow$ }>{\ttfamily}l@{}}
%   settobox.sty & tex/latex/oberdiek/settobox.sty\\
%   settobox.pdf & doc/latex/oberdiek/settobox.pdf\\
%   settobox-example.tex & doc/latex/oberdiek/settobox-example.tex\\
%   settobox.dtx & source/latex/oberdiek/settobox.dtx\\
% \end{tabular}^^A
% }^^A
% \sbox0{\t}^^A
% \ifdim\wd0>\linewidth
%   \begingroup
%     \advance\linewidth by\leftmargin
%     \advance\linewidth by\rightmargin
%   \edef\x{\endgroup
%     \def\noexpand\lw{\the\linewidth}^^A
%   }\x
%   \def\lwbox{^^A
%     \leavevmode
%     \hbox to \linewidth{^^A
%       \kern-\leftmargin\relax
%       \hss
%       \usebox0
%       \hss
%       \kern-\rightmargin\relax
%     }^^A
%   }^^A
%   \ifdim\wd0>\lw
%     \sbox0{\small\t}^^A
%     \ifdim\wd0>\linewidth
%       \ifdim\wd0>\lw
%         \sbox0{\footnotesize\t}^^A
%         \ifdim\wd0>\linewidth
%           \ifdim\wd0>\lw
%             \sbox0{\scriptsize\t}^^A
%             \ifdim\wd0>\linewidth
%               \ifdim\wd0>\lw
%                 \sbox0{\tiny\t}^^A
%                 \ifdim\wd0>\linewidth
%                   \lwbox
%                 \else
%                   \usebox0
%                 \fi
%               \else
%                 \lwbox
%               \fi
%             \else
%               \usebox0
%             \fi
%           \else
%             \lwbox
%           \fi
%         \else
%           \usebox0
%         \fi
%       \else
%         \lwbox
%       \fi
%     \else
%       \usebox0
%     \fi
%   \else
%     \lwbox
%   \fi
% \else
%   \usebox0
% \fi
% \end{quote}
% If you have a \xfile{docstrip.cfg} that configures and enables \docstrip's
% TDS installing feature, then some files can already be in the right
% place, see the documentation of \docstrip.
%
% \subsection{Refresh file name databases}
%
% If your \TeX~distribution
% (\TeX\,Live, \mikTeX, \dots) relies on file name databases, you must refresh
% these. For example, \TeX\,Live\ users run \verb|texhash| or
% \verb|mktexlsr|.
%
% \subsection{Some details for the interested}
%
% \paragraph{Unpacking with \LaTeX.}
% The \xfile{.dtx} chooses its action depending on the format:
% \begin{description}
% \item[\plainTeX:] Run \docstrip\ and extract the files.
% \item[\LaTeX:] Generate the documentation.
% \end{description}
% If you insist on using \LaTeX\ for \docstrip\ (really,
% \docstrip\ does not need \LaTeX), then inform the autodetect routine
% about your intention:
% \begin{quote}
%   \verb|latex \let\install=y% \iffalse meta-comment
%
% File: settobox.dtx
% Version: 2016/05/16 v1.5
% Info: Assign box dimensions to length registers
%
% Copyright (C)
%    2000, 2006-2008 Heiko Oberdiek
%    2016-2019 Oberdiek Package Support Group
%    https://github.com/ho-tex/oberdiek/issues
%
% This work may be distributed and/or modified under the
% conditions of the LaTeX Project Public License, either
% version 1.3c of this license or (at your option) any later
% version. This version of this license is in
%    https://www.latex-project.org/lppl/lppl-1-3c.txt
% and the latest version of this license is in
%    https://www.latex-project.org/lppl.txt
% and version 1.3 or later is part of all distributions of
% LaTeX version 2005/12/01 or later.
%
% This work has the LPPL maintenance status "maintained".
%
% The Current Maintainers of this work are
% Heiko Oberdiek and the Oberdiek Package Support Group
% https://github.com/ho-tex/oberdiek/issues
%
% This work consists of the main source file settobox.dtx
% and the derived files
%    settobox.sty, settobox.pdf, settobox.ins, settobox.drv,
%    settobox-example.tex.
%
% Distribution:
%    CTAN:macros/latex/contrib/oberdiek/settobox.dtx
%    CTAN:macros/latex/contrib/oberdiek/settobox.pdf
%
% Unpacking:
%    (a) If settobox.ins is present:
%           tex settobox.ins
%    (b) Without settobox.ins:
%           tex settobox.dtx
%    (c) If you insist on using LaTeX
%           latex \let\install=y\input{settobox.dtx}
%        (quote the arguments according to the demands of your shell)
%
% Documentation:
%    (a) If settobox.drv is present:
%           latex settobox.drv
%    (b) Without settobox.drv:
%           latex settobox.dtx; ...
%    The class ltxdoc loads the configuration file ltxdoc.cfg
%    if available. Here you can specify further options, e.g.
%    use A4 as paper format:
%       \PassOptionsToClass{a4paper}{article}
%
%    Programm calls to get the documentation (example):
%       pdflatex settobox.dtx
%       makeindex -s gind.ist settobox.idx
%       pdflatex settobox.dtx
%       makeindex -s gind.ist settobox.idx
%       pdflatex settobox.dtx
%
% Installation:
%    TDS:tex/latex/oberdiek/settobox.sty
%    TDS:doc/latex/oberdiek/settobox.pdf
%    TDS:doc/latex/oberdiek/settobox-example.tex
%    TDS:source/latex/oberdiek/settobox.dtx
%
%<*ignore>
\begingroup
  \catcode123=1 %
  \catcode125=2 %
  \def\x{LaTeX2e}%
\expandafter\endgroup
\ifcase 0\ifx\install y1\fi\expandafter
         \ifx\csname processbatchFile\endcsname\relax\else1\fi
         \ifx\fmtname\x\else 1\fi\relax
\else\csname fi\endcsname
%</ignore>
%<*install>
\input docstrip.tex
\Msg{************************************************************************}
\Msg{* Installation}
\Msg{* Package: settobox 2016/05/16 v1.5 Assign box dimensions to length registers (HO)}
\Msg{************************************************************************}

\keepsilent
\askforoverwritefalse

\let\MetaPrefix\relax
\preamble

This is a generated file.

Project: settobox
Version: 2016/05/16 v1.5

Copyright (C)
   2000, 2006-2008 Heiko Oberdiek
   2016-2019 Oberdiek Package Support Group

This work may be distributed and/or modified under the
conditions of the LaTeX Project Public License, either
version 1.3c of this license or (at your option) any later
version. This version of this license is in
   https://www.latex-project.org/lppl/lppl-1-3c.txt
and the latest version of this license is in
   https://www.latex-project.org/lppl.txt
and version 1.3 or later is part of all distributions of
LaTeX version 2005/12/01 or later.

This work has the LPPL maintenance status "maintained".

The Current Maintainers of this work are
Heiko Oberdiek and the Oberdiek Package Support Group
https://github.com/ho-tex/oberdiek/issues


This work consists of the main source file settobox.dtx
and the derived files
   settobox.sty, settobox.pdf, settobox.ins, settobox.drv,
   settobox-example.tex.

\endpreamble
\let\MetaPrefix\DoubleperCent

\generate{%
  \file{settobox.ins}{\from{settobox.dtx}{install}}%
  \file{settobox.drv}{\from{settobox.dtx}{driver}}%
  \usedir{tex/latex/oberdiek}%
  \file{settobox.sty}{\from{settobox.dtx}{package}}%
  \usedir{doc/latex/oberdiek}%
  \file{settobox-example.tex}{\from{settobox.dtx}{example}}%
}

\catcode32=13\relax% active space
\let =\space%
\Msg{************************************************************************}
\Msg{*}
\Msg{* To finish the installation you have to move the following}
\Msg{* file into a directory searched by TeX:}
\Msg{*}
\Msg{*     settobox.sty}
\Msg{*}
\Msg{* To produce the documentation run the file `settobox.drv'}
\Msg{* through LaTeX.}
\Msg{*}
\Msg{* Happy TeXing!}
\Msg{*}
\Msg{************************************************************************}

\endbatchfile
%</install>
%<*ignore>
\fi
%</ignore>
%<*driver>
\NeedsTeXFormat{LaTeX2e}
\ProvidesFile{settobox.drv}%
  [2016/05/16 v1.5 Assign box dimensions to length registers (HO)]%
\documentclass{ltxdoc}
\usepackage{holtxdoc}[2011/11/22]
\usepackage{calc}
\usepackage{settobox}
\begin{document}
  \DocInput{settobox.dtx}%
\end{document}
%</driver>
% \fi
%
%
%
% \GetFileInfo{settobox.drv}
%
% \title{The \xpackage{settobox} package}
% \date{2016/05/16 v1.5}
% \author{Heiko Oberdiek\thanks
% {Please report any issues at \url{https://github.com/ho-tex/oberdiek/issues}}}
%
% \maketitle
%
% \begin{abstract}
% Commands are defined for getting box sizes similar
% to \LaTeX's \cs{settowidth} commands.
% \end{abstract}
%
% \tableofcontents
%
% \section{Usage}
%
% \subsection{Get box dimensions}
%
% \begin{declcs}^^A
%   {settoboxwidth}\,\M{\LaTeX\ length}\,\M{\LaTeX\ box}\\
%   \SpecialUsageIndex{\settoboxheight}^^A
%   \cs{settoboxheight}\,\M{\LaTeX\ length}\,\M{\LaTeX\ box}\\
%   \SpecialUsageIndex{\settoboxdepth}^^A
%   \cs{settoboxdepth}\,\M{\LaTeX\ length}\,\M{\LaTeX\ box}\\
%   \SpecialUsageIndex{\settoboxtotalheight}^^A
%   \cs{settoboxtotalheight}\,\M{\LaTeX\ length}\,\M{\LaTeX\ box}
% \end{declcs}
% A \meta{\LaTeX\ box} is allocated by \cs{newsavebox}.
% It can be filled by \cs{sbox} or the environment \texttt{lrbox}.
% The commands above extract then the desired lengths.
%
% \subsection{Set box dimensions}
%
% \begin{declcs}^^A
%   {setboxwidth}\,\M{\LaTeX\ box}\,\M{\LaTeX\ length expression}\\
%   \SpecialUsageIndex{\setboxheight}^^A
%   \cs{setboxheight}\,\M{\LaTeX\ box}\,\M{\LaTeX\ length expression}\\
%   \SpecialUsageIndex{\setboxdepth}^^A
%   \cs{setboxdepth}\,\M{\LaTeX\ box}\,\M{\LaTeX\ length expression}
% \end{declcs}
% These commands allow the manipulation of the box. Package \xpackage{calc}
% is supported in the \meta{\LaTeX\ length expression}.
% Also the following length are available in this expression:
% \begin{quote}
% \begin{tabular}{@{}ll@{}}
%   \cs{width}& width of the box\\
%   \cs{height}& height of the box\\
%   \cs{depth}& depth of the box\\
%   \cs{totalheight}& totalheight of the box\\
% \end{tabular}
% \end{quote}
% Note, the base point (point at the left margin of the baseline)
% always remain constant.
%
% \subsection{Move box}
%
% \begin{declcs}^^A
%   {setboxmoveleft}\,\M{\LaTeX\ box}\,\M{\LaTeX\ length expression}\\
%   \SpecialUsageIndex{\setboxmoveright}^^A
%   \cs{setboxmoveright}\,\M{\LaTeX\ box}\,\M{\LaTeX\ length expression}\\
%   \SpecialUsageIndex{\setboxlower}^^A
%   \cs{setboxlower}\,\M{\LaTeX\ box}\,\M{\LaTeX\ length expression}\\
%   \SpecialUsageIndex{\setboxright}^^A
%   \cs{setboxright}\,\M{\LaTeX\ box}\,\M{\LaTeX\ length expression}
% \end{declcs}
% Note, the box is shifted relative to the base point. The base point
% is always inside the box, however the width and height of the
% box change along with the movement.
%
% \subsection{Example}
%
% \subsubsection{Short example}
%
% \begin{quote}
%\begin{verbatim}
%\newsavebox{\mybox}
%\newlength{\mylength}
%\sbox{\mybox}{Hello World}
%\settoboxwidth{\mylength}{\mybox}
%\end{verbatim}
% \end{quote}
%
% \subsubsection{Test file that shows box manipulations}
%
%    \begin{macrocode}
%<*example>
%<<END
\documentclass{article}

\usepackage{settobox}
\usepackage{calc}

\newsavebox{\mybox}

\setlength{\fboxsep}{0pt}
\setlength{\parindent}{20pt}
\setlength{\parskip}{10pt}
\pagestyle{empty}

% \test{#1}
% The macro is called with commands in #1 that manipulates
% the box \mybox. These commands along with the result of
% the manipulation is shown. Thus the essence of the
% macro is:
%
%   a) \sbox{\mybox}{The cracy fox.}
%   b) #1 % manipulates \mybox
%   c) Print #1 commands.
%   d) Print box with frame
%
% The implemenation looks more weird:
\makeatletter
\newcommand*{\test}[1]{%
  \par
  \begingroup
    \raggedright
    \edef\x{\detokenize{#1}}%
    \let\do\@makeother
    \dospecials
    \catcode`\~\active
    \catcode`\ =10\relax
    \def~{\\}%
    \noindent
    \texttt{\scantokens\expandafter{\x}}%
    \par
  \endgroup
  \begingroup
    \let~\relax
    \sbox{\mybox}{The cracy fox.}%
     #1%
     A---\fbox{\usebox\mybox}---B%
  \endgroup
  \par
}
\makeatother

\begin{document}

\test{\setboxwidth{\mybox}{1.25\width}}
\test{\setboxheight{\mybox}{0pt}}
\test{\setboxheight{\mybox}{2\height}}
\test{\setboxdepth{\mybox}{\height}}
\test{\setboxmoveleft{\mybox}{5pt}}
\test{%
  \setboxmoveleft{\mybox}{5pt}~%
  \setboxwidth{\mybox}{\width + 5pt}%
}
\test{\setboxmoveright{\mybox}{0.5\width}}
\test{\setboxlower{\mybox}{\height}}
\test{\setboxraise{\mybox}{\depth}}
\test{%
  \setboxmoveright{\mybox}{5pt}~%
  \setboxwidth{\mybox}{\width + 5pt}~%
  \setboxheight{\mybox}{\height + 5pt}~%
  \setboxdepth{\mybox}{\depth + 5pt}%
}

\end{document}
%END
%</example>
%    \end{macrocode}
%
% \noindent
%    The result:
%
% \vspace{1ex}
% \hrule
%
% \begingroup
% \newsavebox{\mybox}
%
% \setlength{\fboxsep}{0pt}
% \setlength{\parindent}{20pt}
% \setlength{\parskip}{10pt}
%
% \makeatletter
% \newcommand*{\test}[1]{^^A
%   \par
%   \begingroup
%     \raggedright
%     \edef\x{\detokenize{#1}}
%     \let\do\@makeother
%     \dospecials
%     \catcode`\~\active
%     \catcode`\ =10\relax
%     \def~{\\}^^A
%     \noindent
%     \texttt{\scantokens\expandafter{\x}}
%     \par
%   \endgroup
%   \begingroup
%     \let~\relax
%     \sbox{\mybox}{The cracy fox.}
%      #1^^A
%      A---\fbox{\usebox\mybox}---B
%   \endgroup
%   \par
% }
% \makeatother
%
% \test{\setboxwidth{\mybox}{1.25\width}}
% \test{\setboxheight{\mybox}{0pt}}
% \test{\setboxheight{\mybox}{2\height}}
% \test{\setboxdepth{\mybox}{\height}}
% \test{\setboxmoveleft{\mybox}{5pt}}
% \test{^^A
%   \setboxmoveleft{\mybox}{5pt}~^^A
%   \setboxwidth{\mybox}{\width + 5pt}^^A
% }
% \test{\setboxmoveright{\mybox}{0.5\width}}
% \test{\setboxlower{\mybox}{\height}}
% \test{\setboxraise{\mybox}{\depth}}
% \test{^^A
%   \setboxmoveright{\mybox}{5pt}~^^A
%   \setboxwidth{\mybox}{\width + 5pt}~^^A
%   \setboxheight{\mybox}{\height + 5pt}~^^A
%   \setboxdepth{\mybox}{\depth + 5pt}^^A
% }
%
% \endgroup
% \vspace{1ex}
% \hrule
% \vspace{4ex}
%
% \StopEventually{
% }
%
% \section{Implementation}
%
%    \begin{macrocode}
%<*package>
%    \end{macrocode}
%    Package identification.
%    \begin{macrocode}
\NeedsTeXFormat{LaTeX2e}
\ProvidesPackage{settobox}%
  [2016/05/16 v1.5 Assign box dimensions to length registers (HO)]
%    \end{macrocode}
%
%    \begin{macrocode}
\newcommand*{\settoboxwidth}[2]{\setlength{#1}{\wd#2}}
\newcommand*{\settoboxheight}[2]{\setlength{#1}{\ht#2}}
\newcommand*{\settoboxdepth}[2]{\setlength{#1}{\dp#2}}
\newcommand*{\settoboxtotalheight}[2]{%
  \setlength{#1}{\ht#2}%
  \addtolength{#1}{\dp#2}%
}
%    \end{macrocode}
%
%    \begin{macro}{\setboxwidth}
%    \begin{macrocode}
\newcommand*{\setboxwidth}[2]{%
  \settobox@length\wd{#1}{#2}%
}
%    \end{macrocode}
%    \end{macro}
%    \begin{macro}{\setboxheight}
%    \begin{macrocode}
\newcommand*{\setboxheight}[2]{%
  \settobox@length\ht{#1}{#2}%
}
%    \end{macrocode}
%    \end{macro}
%    \begin{macro}{\setboxheight}
%    \begin{macrocode}
\newcommand*{\setboxdepth}[2]{%
  \settobox@length\dp{#1}{#2}%
}
%    \end{macrocode}
%    \end{macro}
%    \begin{macro}{\setboxmoveleft}
%    \begin{macrocode}
\newcommand*{\setboxmoveleft}[2]{%
  \settobox@horiz{-}{#1}{#2}%
}
%    \end{macrocode}
%    \end{macro}
%    \begin{macro}{\setboxmoveright}
%    \begin{macrocode}
\newcommand*{\setboxmoveright}[2]{%
  \settobox@horiz{}{#1}{#2}%
}
%    \end{macrocode}
%    \end{macro}
%    \begin{macro}{\setboxlower}
%    \begin{macrocode}
\newcommand*{\setboxlower}[2]{%
  \settobox@vert\lower{#1}{#2}%
}
%    \end{macrocode}
%    \end{macro}
%    \begin{macro}{\setboxraise}
%    \begin{macrocode}
\newcommand*{\setboxraise}[2]{%
  \settobox@vert\raise{#1}{#2}%
}
%    \end{macrocode}
%    \end{macro}
%    \begin{macro}{\settobox@length}
%    The work for the \cs{setbox...} commands is done by
%    \cs{settobox@length}. Inside the length expression
%    \cs{width}, \cs{height}, \cs{depth}, \cs{totalheight}
%    are set to the dimensions of the box.\\
%    \begin{tabular}{@{}ll@{}}
%    |#1|:& the property of the box that is to be changed
%           (\cs{wd}, \cs{ht}, \cs{dp})\\
%    |#2|:& the box\\
%    |#3|:& length expression
%    \end{tabular}
%    \begin{macrocode}
\def\settobox@length#1#2#3{%
  \settobox@calc{#2}{#3}{#1#2=##1sp\relax}%
}
%    \end{macrocode}
%    \end{macro}
%
%    \begin{macro}{\settobox@horiz}
%    \begin{macrocode}
\def\settobox@horiz#1#2#3{%
  \settobox@calc{#2}{#3}{\setbox#2=\hbox{\kern#1##1sp\copy#2}}%
}
%    \end{macrocode}
%    \end{macro}
%    \begin{macro}{\settobox@vert}
%    \begin{macrocode}
\def\settobox@vert#1#2#3{%
  \settobox@calc{#2}{#3}{\setbox#2=\hbox{#1##1sp\copy#2}}%
}
%    \end{macrocode}
%    \end{macro}
%
%    \begin{macro}{\settobox@calc}
%    \begin{macrocode}
\def\settobox@calc#1#2#3{%
  \begingroup
    \def\width{\wd#1}%
    \def\height{\ht#1}%
    \def\depth{\dp#1}%
    \dimen@\ht#1\relax
    \advance\dimen@\dp#1\relax
    \def\totalheight{\dimen@}%
    \setlength{\dimen@}{#2}%
    \count@\dimen@
    \def\x##1{\endgroup
      #3%
    }%
  \expandafter\x\expandafter{\the\count@}%
}
%    \end{macrocode}
%    \end{macro}
%
%    \begin{macrocode}
%</package>
%    \end{macrocode}
%
% \section{Installation}
%
% \subsection{Download}
%
% \paragraph{Package.} This package is available on
% CTAN\footnote{\CTANpkg{settobox}}:
% \begin{description}
% \item[\CTAN{macros/latex/contrib/oberdiek/settobox.dtx}] The source file.
% \item[\CTAN{macros/latex/contrib/oberdiek/settobox.pdf}] Documentation.
% \end{description}
%
%
% \paragraph{Bundle.} All the packages of the bundle `oberdiek'
% are also available in a TDS compliant ZIP archive. There
% the packages are already unpacked and the documentation files
% are generated. The files and directories obey the TDS standard.
% \begin{description}
% \item[\CTANinstall{install/macros/latex/contrib/oberdiek.tds.zip}]
% \end{description}
% \emph{TDS} refers to the standard ``A Directory Structure
% for \TeX\ Files'' (\CTANpkg{tds}). Directories
% with \xfile{texmf} in their name are usually organized this way.
%
% \subsection{Bundle installation}
%
% \paragraph{Unpacking.} Unpack the \xfile{oberdiek.tds.zip} in the
% TDS tree (also known as \xfile{texmf} tree) of your choice.
% Example (linux):
% \begin{quote}
%   |unzip oberdiek.tds.zip -d ~/texmf|
% \end{quote}
%
% \subsection{Package installation}
%
% \paragraph{Unpacking.} The \xfile{.dtx} file is a self-extracting
% \docstrip\ archive. The files are extracted by running the
% \xfile{.dtx} through \plainTeX:
% \begin{quote}
%   \verb|tex settobox.dtx|
% \end{quote}
%
% \paragraph{TDS.} Now the different files must be moved into
% the different directories in your installation TDS tree
% (also known as \xfile{texmf} tree):
% \begin{quote}
% \def\t{^^A
% \begin{tabular}{@{}>{\ttfamily}l@{ $\rightarrow$ }>{\ttfamily}l@{}}
%   settobox.sty & tex/latex/oberdiek/settobox.sty\\
%   settobox.pdf & doc/latex/oberdiek/settobox.pdf\\
%   settobox-example.tex & doc/latex/oberdiek/settobox-example.tex\\
%   settobox.dtx & source/latex/oberdiek/settobox.dtx\\
% \end{tabular}^^A
% }^^A
% \sbox0{\t}^^A
% \ifdim\wd0>\linewidth
%   \begingroup
%     \advance\linewidth by\leftmargin
%     \advance\linewidth by\rightmargin
%   \edef\x{\endgroup
%     \def\noexpand\lw{\the\linewidth}^^A
%   }\x
%   \def\lwbox{^^A
%     \leavevmode
%     \hbox to \linewidth{^^A
%       \kern-\leftmargin\relax
%       \hss
%       \usebox0
%       \hss
%       \kern-\rightmargin\relax
%     }^^A
%   }^^A
%   \ifdim\wd0>\lw
%     \sbox0{\small\t}^^A
%     \ifdim\wd0>\linewidth
%       \ifdim\wd0>\lw
%         \sbox0{\footnotesize\t}^^A
%         \ifdim\wd0>\linewidth
%           \ifdim\wd0>\lw
%             \sbox0{\scriptsize\t}^^A
%             \ifdim\wd0>\linewidth
%               \ifdim\wd0>\lw
%                 \sbox0{\tiny\t}^^A
%                 \ifdim\wd0>\linewidth
%                   \lwbox
%                 \else
%                   \usebox0
%                 \fi
%               \else
%                 \lwbox
%               \fi
%             \else
%               \usebox0
%             \fi
%           \else
%             \lwbox
%           \fi
%         \else
%           \usebox0
%         \fi
%       \else
%         \lwbox
%       \fi
%     \else
%       \usebox0
%     \fi
%   \else
%     \lwbox
%   \fi
% \else
%   \usebox0
% \fi
% \end{quote}
% If you have a \xfile{docstrip.cfg} that configures and enables \docstrip's
% TDS installing feature, then some files can already be in the right
% place, see the documentation of \docstrip.
%
% \subsection{Refresh file name databases}
%
% If your \TeX~distribution
% (\TeX\,Live, \mikTeX, \dots) relies on file name databases, you must refresh
% these. For example, \TeX\,Live\ users run \verb|texhash| or
% \verb|mktexlsr|.
%
% \subsection{Some details for the interested}
%
% \paragraph{Unpacking with \LaTeX.}
% The \xfile{.dtx} chooses its action depending on the format:
% \begin{description}
% \item[\plainTeX:] Run \docstrip\ and extract the files.
% \item[\LaTeX:] Generate the documentation.
% \end{description}
% If you insist on using \LaTeX\ for \docstrip\ (really,
% \docstrip\ does not need \LaTeX), then inform the autodetect routine
% about your intention:
% \begin{quote}
%   \verb|latex \let\install=y\input{settobox.dtx}|
% \end{quote}
% Do not forget to quote the argument according to the demands
% of your shell.
%
% \paragraph{Generating the documentation.}
% You can use both the \xfile{.dtx} or the \xfile{.drv} to generate
% the documentation. The process can be configured by the
% configuration file \xfile{ltxdoc.cfg}. For instance, put this
% line into this file, if you want to have A4 as paper format:
% \begin{quote}
%   \verb|\PassOptionsToClass{a4paper}{article}|
% \end{quote}
% An example follows how to generate the
% documentation with pdf\LaTeX:
% \begin{quote}
%\begin{verbatim}
%pdflatex settobox.dtx
%makeindex -s gind.ist settobox.idx
%pdflatex settobox.dtx
%makeindex -s gind.ist settobox.idx
%pdflatex settobox.dtx
%\end{verbatim}
% \end{quote}
%
% \begin{History}
%   \begin{Version}{2000/02/11 v1.0}
%   \item
%     First public release, written as answer in the
%     newsgroup \xnewsgroup{de.comp.text.tex}:
%     \URL{``\link{Die Hoehe von Minipages und Bild}''}^^A
%     {https://groups.google.com/group/de.comp.text.tex/msg/c3f6446f54f66c02}
%   \end{Version}
%   \begin{Version}{2000/09/07 v1.1}
%   \item
%     Documentation added.
%   \item
%     CTAN release.
%   \end{Version}
%   \begin{Version}{2006/02/20 v1.2}
%   \item
%     \cs{setboxwidth}, \cs{setboxheight}, \cs{setboxdepth} added.
%   \item
%     Box move commands added.
%   \item
%     DTX framework.
%   \item
%     LPPL 1.3
%   \end{Version}
%   \begin{Version}{2007/04/11 v1.3}
%   \item
%     Line ends sanitized.
%   \end{Version}
%   \begin{Version}{2008/08/11 v1.4}
%   \item
%     Code is not changed.
%   \item
%     URLs updated.
%   \end{Version}
%   \begin{Version}{2016/05/16 v1.5}
%   \item
%     Documentation updates.
%   \end{Version}
% \end{History}
%
% \PrintIndex
%
% \Finale
\endinput
|
% \end{quote}
% Do not forget to quote the argument according to the demands
% of your shell.
%
% \paragraph{Generating the documentation.}
% You can use both the \xfile{.dtx} or the \xfile{.drv} to generate
% the documentation. The process can be configured by the
% configuration file \xfile{ltxdoc.cfg}. For instance, put this
% line into this file, if you want to have A4 as paper format:
% \begin{quote}
%   \verb|\PassOptionsToClass{a4paper}{article}|
% \end{quote}
% An example follows how to generate the
% documentation with pdf\LaTeX:
% \begin{quote}
%\begin{verbatim}
%pdflatex settobox.dtx
%makeindex -s gind.ist settobox.idx
%pdflatex settobox.dtx
%makeindex -s gind.ist settobox.idx
%pdflatex settobox.dtx
%\end{verbatim}
% \end{quote}
%
% \begin{History}
%   \begin{Version}{2000/02/11 v1.0}
%   \item
%     First public release, written as answer in the
%     newsgroup \xnewsgroup{de.comp.text.tex}:
%     \URL{``\link{Die Hoehe von Minipages und Bild}''}^^A
%     {https://groups.google.com/group/de.comp.text.tex/msg/c3f6446f54f66c02}
%   \end{Version}
%   \begin{Version}{2000/09/07 v1.1}
%   \item
%     Documentation added.
%   \item
%     CTAN release.
%   \end{Version}
%   \begin{Version}{2006/02/20 v1.2}
%   \item
%     \cs{setboxwidth}, \cs{setboxheight}, \cs{setboxdepth} added.
%   \item
%     Box move commands added.
%   \item
%     DTX framework.
%   \item
%     LPPL 1.3
%   \end{Version}
%   \begin{Version}{2007/04/11 v1.3}
%   \item
%     Line ends sanitized.
%   \end{Version}
%   \begin{Version}{2008/08/11 v1.4}
%   \item
%     Code is not changed.
%   \item
%     URLs updated.
%   \end{Version}
%   \begin{Version}{2016/05/16 v1.5}
%   \item
%     Documentation updates.
%   \end{Version}
% \end{History}
%
% \PrintIndex
%
% \Finale
\endinput
|
% \end{quote}
% Do not forget to quote the argument according to the demands
% of your shell.
%
% \paragraph{Generating the documentation.}
% You can use both the \xfile{.dtx} or the \xfile{.drv} to generate
% the documentation. The process can be configured by the
% configuration file \xfile{ltxdoc.cfg}. For instance, put this
% line into this file, if you want to have A4 as paper format:
% \begin{quote}
%   \verb|\PassOptionsToClass{a4paper}{article}|
% \end{quote}
% An example follows how to generate the
% documentation with pdf\LaTeX:
% \begin{quote}
%\begin{verbatim}
%pdflatex settobox.dtx
%makeindex -s gind.ist settobox.idx
%pdflatex settobox.dtx
%makeindex -s gind.ist settobox.idx
%pdflatex settobox.dtx
%\end{verbatim}
% \end{quote}
%
% \begin{History}
%   \begin{Version}{2000/02/11 v1.0}
%   \item
%     First public release, written as answer in the
%     newsgroup \xnewsgroup{de.comp.text.tex}:
%     \URL{``\link{Die Hoehe von Minipages und Bild}''}^^A
%     {https://groups.google.com/group/de.comp.text.tex/msg/c3f6446f54f66c02}
%   \end{Version}
%   \begin{Version}{2000/09/07 v1.1}
%   \item
%     Documentation added.
%   \item
%     CTAN release.
%   \end{Version}
%   \begin{Version}{2006/02/20 v1.2}
%   \item
%     \cs{setboxwidth}, \cs{setboxheight}, \cs{setboxdepth} added.
%   \item
%     Box move commands added.
%   \item
%     DTX framework.
%   \item
%     LPPL 1.3
%   \end{Version}
%   \begin{Version}{2007/04/11 v1.3}
%   \item
%     Line ends sanitized.
%   \end{Version}
%   \begin{Version}{2008/08/11 v1.4}
%   \item
%     Code is not changed.
%   \item
%     URLs updated.
%   \end{Version}
%   \begin{Version}{2016/05/16 v1.5}
%   \item
%     Documentation updates.
%   \end{Version}
% \end{History}
%
% \PrintIndex
%
% \Finale
\endinput
|
% \end{quote}
% Do not forget to quote the argument according to the demands
% of your shell.
%
% \paragraph{Generating the documentation.}
% You can use both the \xfile{.dtx} or the \xfile{.drv} to generate
% the documentation. The process can be configured by the
% configuration file \xfile{ltxdoc.cfg}. For instance, put this
% line into this file, if you want to have A4 as paper format:
% \begin{quote}
%   \verb|\PassOptionsToClass{a4paper}{article}|
% \end{quote}
% An example follows how to generate the
% documentation with pdf\LaTeX:
% \begin{quote}
%\begin{verbatim}
%pdflatex settobox.dtx
%makeindex -s gind.ist settobox.idx
%pdflatex settobox.dtx
%makeindex -s gind.ist settobox.idx
%pdflatex settobox.dtx
%\end{verbatim}
% \end{quote}
%
% \begin{History}
%   \begin{Version}{2000/02/11 v1.0}
%   \item
%     First public release, written as answer in the
%     newsgroup \xnewsgroup{de.comp.text.tex}:
%     \URL{``\link{Die Hoehe von Minipages und Bild}''}^^A
%     {https://groups.google.com/group/de.comp.text.tex/msg/c3f6446f54f66c02}
%   \end{Version}
%   \begin{Version}{2000/09/07 v1.1}
%   \item
%     Documentation added.
%   \item
%     CTAN release.
%   \end{Version}
%   \begin{Version}{2006/02/20 v1.2}
%   \item
%     \cs{setboxwidth}, \cs{setboxheight}, \cs{setboxdepth} added.
%   \item
%     Box move commands added.
%   \item
%     DTX framework.
%   \item
%     LPPL 1.3
%   \end{Version}
%   \begin{Version}{2007/04/11 v1.3}
%   \item
%     Line ends sanitized.
%   \end{Version}
%   \begin{Version}{2008/08/11 v1.4}
%   \item
%     Code is not changed.
%   \item
%     URLs updated.
%   \end{Version}
%   \begin{Version}{2016/05/16 v1.5}
%   \item
%     Documentation updates.
%   \end{Version}
% \end{History}
%
% \PrintIndex
%
% \Finale
\endinput
