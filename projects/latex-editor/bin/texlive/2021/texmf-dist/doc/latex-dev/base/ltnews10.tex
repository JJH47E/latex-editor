% \iffalse meta-comment
%
% Copyright (C) 1993-2022
% The LaTeX Project and any individual authors listed elsewhere
% in this file.
%
% This file is part of the LaTeX base system.
% -------------------------------------------
%
% It may be distributed and/or modified under the
% conditions of the LaTeX Project Public License, either version 1.3c
% of this license or (at your option) any later version.
% The latest version of this license is in
%    http://www.latex-project.org/lppl.txt
% and version 1.3c or later is part of all distributions of LaTeX
% version 2008 or later.
%
% This file has the LPPL maintenance status "maintained".
%
% The list of all files belonging to the LaTeX base distribution is
% given in the file `manifest.txt'. See also `legal.txt' for additional
% information.
%
% The list of derived (unpacked) files belonging to the distribution
% and covered by LPPL is defined by the unpacking scripts (with
% extension .ins) which are part of the distribution.
%
% \fi
% Filename: ltnews10.tex 12/01/1998

% This is issue 10 of LaTeX News.

\documentclass
%    [lw35fonts]
   {ltnews}[1999/02/23]

% \usepackage[T1]{fontenc}

\publicationmonth{December}
\publicationyear{1998}
\publicationissue{10}

\providecommand\pkg[1]{\texttt{#1}}
\providecommand\cls[1]{\texttt{#1}}
\providecommand\option[1]{\texttt{#1}}
\providecommand\env[1]{\texttt{#1}}
\providecommand\file[1]{\texttt{#1}}

\begin{document}

\maketitle


\section{Five years of \LaTeXe}

Since this is the 10th edition of \LaTeX{} News, the (no longer) New
Standard \LaTeX{} must have hit the streets almost this long ago.  In
fact it was only the beta-version that some people got just in time for
Christmas~1993, and since then there has been a lot of tidying-up and
smoothing of rough edges (not to mention a few bug fixes!).

Maybe it is time for something more radically different to emerge and
be hungrily adopted by the world; but don't panic, we shall be
maintaining what you have now for a long time yet.  Amongst the more
polite things that have been written about our efforts, we found that this
quote (somewhat censored to protect the guilty) well reflects some of
our feelings about working on \LaTeX{} over the years: \textit{the mere
existence of \LaTeXe{} is a great miracle}.


  \section{Restructuring the \LaTeX{} distribution}

  Since the (once) `new' standard \LaTeX{} has reached such a venerable
  age, we are reviewing the way in which the system is presented to the
  world.

  An early intention is to define, given the wide variety of good
  packages now available, what now constitutes a useful installation of
  \LaTeX{}. We also hope that such a definition will help document
  portability if it leads to a future in which a \LaTeX{} class
  designer can reasonably assume that a known list of
  facilities will be there for all users (so that each class
  need not supply them).

  As a first small step towards this definition, we shall replace the
  \file{latex/packages} subdirectory on \ctan{}.
  This directory was a curious mixture of the important, such as the
  \LaTeX{} \textsf{tools}, that any self-respecting \LaTeX{}
  installation ought to have, and the esoteric or experimental.

  The esoterica from \textsf{packages} will be moved to
  new locations, as follows:
  \begin{quote}
    \textsf{expl3}  to \file{latex/exptl/project}\\
    \textsf{mfnfss} to \file{latex/contrib/supported/mfnfss}
  \end{quote}

  The subdirectory that replaces \textsf{packages} will be called
  \file{latex/required}; all the other sub-directories of
  \file{packages} will be moved there.

\vspace{17pt}
\pagebreak

 \section{\LaTeX\ Project on the Internet}
A new \texttt{latex-project.org} domain has been registered.
The web site is not yet fully functional but the old \LaTeX\ pages from
\ctan\ are available at \texttt{http://www.latex-project.org/} and the
\LaTeX\ bug reporting address has been changed to
\texttt{latex-bugs@latex-project.org}.

  \section{Restructuring the \LaTeX{} package licenses}

Several people have requested an easy mechanism for the distribution
of \LaTeX\ packages and other software ``under the same conditions as
\LaTeX''.  The old \file{legal.txt} file was unsuitable as a general
licence as it referred to specific \LaTeX\ authors, and to specific
files.

Therefore, in this release \file{legal.txt} contains just the
copyright notice and a reference to the new \emph{\LaTeX\ Project
Public License} (LPPL) for the distribution and modification
conditions. The \textsf{tools}, \textsf{graphics}, and \textsf{mfnfss}
packages also now refer to this license in their distribution notices.

  \section{Support for Cyrillic encodings}

Basic Cyrillic support, as announced in \LaTeX{} News~9, is now
finally an official part of \LaTeX{}.  It includes support for the
following standard Cyrillic font encodings (this list may
grow):~\mbox{\texttt{T2A T2B T2C X2}}.

It also includes various Cyrillic input encodings (20~in total,
including commonly used variants and Mongolian Cyrillic
encodings). This provides platform independent and
sophisticated basic support for high-quality typesetting in various
Cyrillic-based languages.

For further information see the file \file{cyrguide.tex}.

\section{Tools distribution}

The \pkg{varioref} package has been extended to support textual
page references to a range of objects: e.g.,~if \texttt{eq-first}
and \texttt{eq-last} are the label names for the first and last
equation in a sequence, then you can now write
\begin{verbatim}
  see~\vrefrange{eq-first}{eq-last}
\end{verbatim}
This results in different text depending on whether both
labels fall on the same page.

Some additional user commands, as well as building-blocks for writing
private extensions, are described in the accompanying documentation.



\end{document}
