% Это пример входного файла LaTeX (Версия от 11 апреля 1994 года.)
%
% Знак '%' говорит TeX'у, что весь следующий за ним на данной строке текст
% будет проигнорирован, и поэтому этот знак используется для комментариев
% (как здесь).

\documentclass{article}      % Определяем класс (тип) документа
\usepackage[russian]{babel}
\usepackage[colorlinks=true]{hyperref}

                             % Здесь начинается преамбула

\usepackage{fontspec}
\setmainfont{CMU Serif}

\nonfrenchspacing            % Разрешаем увеличивать пробел
                             % после конца предложения

\title{Пример документа}     % Определяем заголовок
\author{Лесли Лэмпорт}       % Определяем имя автора
\date{21 января 1994 г.}     % Если убрать эту команду, будет напечатана
                             % текущая дата

\newcommand{\ip}[2]{(#1, #2)}% Определяем новую команду \ip{arg1}{arg2},
                             % которая раскрывается в (arg1, arg2).

%\newcommand{\ip}[2]{\langle #1 | #2\rangle}
                             % Это другое определение \ip (закомментировано).

\begin{document}             % Конец преамбулы и начало текста

\maketitle                   % Выводит заголовок

Это пример входного файла.  Сравнивая его с выходным (PDF)
файлом, Вы можете понять, как
сделать простой документ самостоятельно.

\tableofcontents

\section{Простой текст}      % Производит заголовок раздела. Разделы нижних
                             % уровней начинаются с похожих команд:
                             % \subsection и \subsubsection.

Окончание каждого слова и предложения определяется
  пробелами.    Не имеет значения, сколько пробелов Вы
напечатаете;           один пробел~--- то же, что 100.  Конец
строки считается пробелом.

Одна    или   несколько пустых строк определяют  конец
абзаца.

Поскольку любое количество последовательных пробелов отвечает одному
пробелу, форматирование исходного файла не имеет значения
для
      \LaTeX,                % Команда \LaTeX выводит логотип LaTeX'а.
но может иметь значение для Вас.  Когда Вы используете \LaTeX, делайте
свой входной файл, насколько это возможно, легким для чтения~---
это поможет Вам переписать Ваш документ, когда Вы захотите его изменить.
Этот файл показывает, как можно добавлять комментарии ко входному файлу.

Печать и набор~--- разные вещи, поэтому есть множество случаев,
когда то, что Вы напечатали во входном файле, отличается от того, что
Вы хотите получить в документе. Кавычки типа
        <<таких>>
нужно получать специально, также как и кавычки внутри кавычек:
        <<\,``эти'',        % \, отделяет внешние и внутренние кавычки
        которые я сейчас
        напечатал,
        не  "`такие"'\,>>.

Тире бывают трех размеров:
        внутри-слова
(дефис), среднее тире для соединения цифр типа
        1--2,
и тире как знак препинания
        --- как
это.

Пробел в конце предложения должен быть немного больше, чем
обычный пробел.  Иногда необходимо вместе со знаками пунктуации
напечатать специальные команды, чтобы получить правильные
пробелы, как в следующем предложении.
        Комары, козлы и        % `\ ' производит `нормальный' пробел
        пр.\ начинаются
        с К\@.                 % \@ определяет, где конец предложения.
Нужно проверять размер пробелов после знаков пунктуации, когда
Вы просматриваете выходной файл, чтобы быть уверенным в том, что Вы
не пропустили ни один специальный случай.
Многоточие производится командой
       \ldots\               % `\ ' нужно поставить после `\ldots', т.к. TeX
                             % пропускает все пробелы после команд, состоящих
                             % из \ и букв
                             %
                             % Обратите внимание на то, что знак `%' приказывает TeX'у
                             % проигнорировать конец входной строки, и поэтому такие
                             % строки не производят новый абзац.
                             %
(правильный пробел после точек требует введения специальной
команды).

\LaTeX\ считает несколько обычных символов командами,
поэтому, чтобы напечатать некоторые из них,     нужны
специальные команды. Вот эти символы:
       \$ \& \% \# \{ и \}.
%$

Для печати текст лучше высего выделять \emph{курсивом}.

\begin{em}
   Длинный отрывок текста также может быть выделен
   таким способом. Текст внутри отрывка может
   иметь \emph{добавочное} выделение.
\end{em}

Иногда необходимо предохранить какую-либо фразу от разрыва
между строками в неподходящем месте. Таким местом может
быть пробел, например, между <<Г-н>> и <<Джонс>> в
        <<Г-н Джонс>>,    % ~ дает неразрывный междусловный пробел.
или внутри слова~--- особенно если это некоторый термин типа
        \mbox{\emph{вхождение}},
так что не имеет особого смыла разбивать его между строками.

Сноски\footnote{Это пример сноски.} не создают никаких проблем.

\LaTeX\ хорош для того, чтобы набирать математические формулы
типа
       \( x-3y + z = 7 \)
или
       \( a_{1} > x^{2n} + y^{2n} > x' \)
или
       \( \ip{A}{B} = \sum_{i} a_{i} b_{i} \).
Пробелы, которые Вы печатаете внутри формулы, будут
проигнорированы. Запомните, что буквы типа
       $x$                   % $ ... $  и  \( ... \)  эквивалетнты
являются формулой, поскольку они изображают математический символ,
и должны набираться как формула.

\section{Выделенный текст}

Текст выделяется при помощи отступа от левого края. Так обычно
выделяются цитаты. Вот короткая цитата:
\begin{quote}
   Это~--- короткая цитата (quote). Она содержит одит абзац
   текста. Посмотрите, как она выглядит.
\end{quote}
и более длинная:
\begin{quotation}
   Это~--- длинная цитата (quotation). Она содержит
   два абзаца текста, ни один из которых не представляет
   для Вас никакого интереса.

   Это~--- второй абзац длинной цитаты. Он такой же дурацкий,
   как и первый.
\end{quotation}
Другой тип выделенного текста~--- это список. Следующие строки
представляют пример \emph{маркированного} списка:
\begin{itemize}
   \item Это~--- первый пункт маркированного списка.
         Каждый пункт списка маркирован кружком. Вы не
         должны заботиться о том, какой значок выбрать.

   \item Это~--- второй пункт списка. Он содержит другой,
         вложенный список. Вложенный список~--- пример
         \emph{нумерованного} списка.
         \begin{enumerate}
            \item Это~--- первый пункт нумерованного спика, который
                  вложен в маркированный список.

            \item Это~--- второй пункт вложенного списка.
                  \LaTeX\ разрешает вкладывать разные списки друг
                  в друга, пока это занятие Вам не надоест.
         \end{enumerate}
         Это~--- конец второго пункта внешнего списка. Он не
         интереснее любой другой его части.
   \item Это~--- третий пункт списка.
\end{itemize}
Можно даже набирать стихи:
\begin{verse}
   Нет хуже сего окружения,\\ % \\ разделяет строки в строфе
   Когда хочешь сочинить
   стихотворение.

                             % Одна или больше пустых строчек
                             % разделяет строфы.

   Ты концы всех строчек расставляешь,\\
   Вдохновенье при этом теряешь.

   \emph{Сколько хочешь} слов на строчку здесь влезает~--- это
   лаконичности весьма мешает.
\end{verse}

Математические формулы также могут быть выделенными.
Выделенная формула должна быть длиной в одну строку;
многострочные формулы требуют специальных инструкций
для форматирования.
   \[  \ip{\Gamma}{\psi'} = x'' + y^{2} + z_{i}^{n}\]
Не начинайте абзац с выделенной формулы, и не делайте из
нее отдельный абзац.

\end{document}               % Конец документа

Дальше можно печатать что угодно~--- оно все равно не будет
транслироваться:

"sample2e.tex" перевел на русский (с некоторыми изменениями)
А. Шипунов (plantago@herba.msu.ru)
