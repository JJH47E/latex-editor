\documentclass[a4paper,12pt]{article}
\usepackage[russian]{babel}
\usepackage{a4wide,onepagem,qqru}

\font\manual=logo10
\def\PiCTeX{{\rm P\kern-.12em\lower.5ex\hbox{I}\kern-.075emC\kern-.11em\TeX}}
\def\MF{{\manual META}\-{\manual FONT}}
\def\MP{{\scshape Meta\-post}}
\def\Xy{\leavevmode\hbox{\kern-.1em X\kern-.3em\lower.4ex\hbox{Y\kern-.15em}}}

% russification of 'enumerate'
\makeatletter
\def\labelenumi{\theenumi)}
\def\theenumii{\asbuk{enumii}}
\def\labelenumii{\theenumii)}
\def\p@enumii{\theenumi}
\def\labelenumiii{{\bf--}}
\let\theenumiii\relax
\def\p@enumiii{\theenumi\theenumii}
\makeatother

\usepackage{fontspec}
\setmainfont{CMU Serif}

\author{А. Шипунов}
\title{Графика в \TeX/\LaTeX}
\date{19.10.1999}
\begin{document}
\maketitle

\begin{enumerate}

\item Работа с внешними объектами

        \begin{enumerate}

        \item Графический код (\<векторный\>), используемый для
        препроцессинга или получаемый в результате компиляции
        \TeX-файла, а затем встраиваемый в \texttt{dvi}-файл при
        помощи команды \verb|\special|.

                \begin{itemize}

                \item \MF\ (и его вариант \MP)~--- пакеты \MP\ и
                \texttt{mfpic}. Перед встраиванием в \texttt{dvi}-файл
                требуется компиляция кода.

                \item PostScript~--- пакеты из набора
                \<\texttt{graphics}\>, PSTricks, \TeX{}\-Draw,
                \Xy-pic, \texttt{pspicture}, \texttt{psfrag} и~т.д.

                \item TPIC~--- \Xy-pic, \texttt{eepic}.

                \item Em\TeX~--- \texttt{curvesls}, \texttt{stree}, код,
                производимый пакетом \TeX{}cad и~т.д.

                \item Различные языки для препроцессоров,
                переводящих внешний код во внутренний: mathsPIC
                ($\rightarrow$ \PiCTeX), Flow ($\rightarrow$
                \LaTeX) и~т.д.

                \end{itemize}

        \item Готовые рисунки: растровые (\texttt{BMP}, \texttt{PCX} и~т.д.)
        и векторные (\texttt{WMF}, \texttt{FIG}, \texttt{EPS}).

                \begin{itemize}

                \item Встраивание за счет команды
                \verb|\special| в \texttt{dvi}-файл~--- пакеты из
                набора \<\texttt{graphics}\> и~т.д.

                \item Препроцессинг для преобразования в линейки
                (\texttt{bm2latex}), в шрифты (\texttt{bm2font}) или в
                другие внешние графические форматы
                (\texttt{transfig}).

                \end{itemize}

        \item \<Чужие\> шрифты: \texttt{tfm} + \texttt{vf} + Type1 или
        TrueType

        \end{enumerate}

\item Работа с внутренними объектами

        \begin{enumerate}

        \item Линейки (rules)~--- \<\LaTeX\ picture
        environment\> и его расширения (пакеты \texttt{bar},
        \texttt{bez123}, \texttt{epic}, \texttt{pmgraph} и~т.д.), \PiCTeX,
        \Xy-pic, Dra\TeX.

        \item Шрифты (\texttt{tfm} + \texttt{pk})~--- \LaTeX\ picture с
        расширениями (см. выше), \Xy-pic и~т.д.

        \end{enumerate}

\end{enumerate}

\end{document}
