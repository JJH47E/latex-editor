\documentclass{article}
\usepackage{polyglossia}
\usepackage{xecyr}

\setmainlanguage{russian}

\setmainfont[Mapping=tex-text]{CMU Serif}
\setmonofont{CMU Typewriter Text}

\title{Распространенные русские кодировки}
\author{Из русской <<Википедии>>}
\date{}

\begin{document}
\maketitle

\tableofcontents

% Обратите внимание на названия кодировок! Они берутся отсюда: http://www.iana.org/assignments/character-sets

\XeTeXdefaultencoding "IBM866"
\input 866

\XeTeXdefaultencoding "KOI8-R"
\input koi8-r

\XeTeXdefaultencoding "windows-1251"
\input 1251

\XeTeXdefaultencoding "ISO-8859-5"
\input iso

\XeTeXdefaultencoding "UTF-8"

\end{document}
