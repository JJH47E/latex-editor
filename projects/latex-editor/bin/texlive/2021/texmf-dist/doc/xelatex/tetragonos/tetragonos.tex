\documentclass[a4paper]{ctexart}
% 20190115
% Author: Yuchang Yang < yang.yc.allium@gmail.com >
\usepackage{metalogo}
\usepackage{fancyvrb}
\usepackage{changepage}
\usepackage{color}
\definecolor{darkmiku}{RGB}{19, 149, 139}
\CTEXsetup[format={\Large\bfseries}]{section}

\title{tetragonos宏包\footnote{Github 地址:\texttt{https://github.com/Mikumikunisiteageru/tetragonos}}}
\author{杨宇昌\footnote{电子邮箱:\texttt{yang.yc.allium@gmail.com}}}
\date{2019年1月16日\qquad v1}

\begin{document}

\maketitle

\begingroup
	\CJKfontspec{IPAPMincho}
	\large
	\centerline{四角い地球を丸くしてる!}
	\hfill --- 初音ミク、\kern1ex TOKOTOKO(西沢さん\mbox{P})\hbox{}\par
	\vskip1em
\endgroup

\section{简介}
tetragonos 是一个用于查询汉字四角号码的\XeLaTeX{}宏包。\par
宏包名称取自希腊语,tetra者四也,gon者角也。\par
	
\section{用法}
tetragonos 宏包定义了宏 \verb|\getTG|。\texttt{\textbackslash getTG\{\textit{<string>}\}} 可展开成汉字字符串 \texttt{\textit{<string>}} 中第一个汉字的带附号的新版四角号码。举个简单的例子,\verb|\getTG{汉}| 与 \verb|\getTG{汉字}| 都会展开成“汉”字的四角号码,即 37140。\par
	
\section{四角索引范例}
四角号码常用来为汉字字词编写索引。文件 tetragonos-example.tex 中即示范了 tetragonos 宏包与 glossaries 宏包的联合应用。该文件从《广东新语》中摘录了一段文字,并为其中出现的所有植物名称建立了首字四角号码索引。该文件中的关键内容如下:
	
\begingroup
	\begin{adjustwidth}{3em}{3em}
	\footnotesize
	\begin{Verbatim}[numbers=left,numbersep=1ex,gobble=2,firstnumber=1,formatcom=\color{darkmiku}]
		% arara: clean: {files: [tetragonos-example.glsdefs, tetragonos-example.TG-*]}
		% arara: xelatex
		% arara: makeglossaries
		% arara: xelatex
		% arara: xelatex
	\end{Verbatim}
	指示 arara 运行时先清空历史索引条目,之后依次执行 \XeLaTeX{}、makeglossaries 等各程序。编译时直接输入 \verb|arara tetragonos-example| 即可。
	\begin{Verbatim}[numbers=left,numbersep=1ex,gobble=2,firstnumber=28,formatcom=\color{darkmiku}]
		\usepackage{tetragonos}
		\usepackage[nopostdot,nomain]{glossaries}
		\newglossary*{TG}{Tegragonos}
		\makeglossaries
	\end{Verbatim}
	引入 tetragonos 与 glossaries 宏包后,新建好索引,准备登记四角号码与汉字条目。
	\begin{Verbatim}[numbers=left,numbersep=1ex,gobble=2,firstnumber=32,formatcom=\color{darkmiku}]
		\def\syntaxTG#1#2#3#4#5{\textbf{#1#2#3#4\textsubscript{#5}}}
	\end{Verbatim}
	定义四角号码的格式,使其整体加粗,并令附号(即附角号码)作下标。
	\begin{Verbatim}[numbers=left,numbersep=1ex,gobble=2,firstnumber=33,tabsize=4,formatcom=\color{darkmiku}]
		\newcommand{\addTG}[1]{%
			\newglossaryentry{#1}{
				type=TG, 
				name={\syntaxTG#1}, 
				description={\nopostdesc}
			}%
		}
	\end{Verbatim}
	定义宏 \verb|\addTG|,建立四角号码条目,在索引中位于第一级。
	\begin{Verbatim}[numbers=left,numbersep=1ex,gobble=2,firstnumber=40,tabsize=4,formatcom=\color{darkmiku}]
		\newcommand{\anchorTG}[1]{%
			\edef\theTG{\getTG{#1}}%
			\expandafter\addTG\expandafter{\theTG\expandafter}%
			\newglossaryentry{#1}{
				type=TG, 
				name={\textmd{#1}}, 
				parent={\theTG}, 
				sort={\theTG}, 
				description={\hfill}
			}%
			\glsadd{#1}%
		}
	\end{Verbatim}
	定义宏 \verb|\anchorTG|,建立汉字(词)条目,居于第二级,是其首字四角号码对应的条目的子条目。\verb|\anchorTG| 只插入锚点,登记汉字(词)条目与当前页码,不在正文中产生可视效应。这是照顾一些在正文与索引中有不同字(词)形的条目的需要,毕竟在 \TeX{} 里合总比分容易。glossaries 宏包中的 \verb|\newglossaryentry| 命令会自动忽略登记过的条目,因此不必特意对条目判重。\verb|\expandafter| 强迫 \verb|\theTG| 在送入 \verb|\addTG| 前完全展平成四角号码。
	\begin{Verbatim}[numbers=left,numbersep=1ex,gobble=2,firstnumber=52,formatcom=\color{darkmiku}]
		\newcommand{\cc}[1]{#1\anchorTG{#1}}
	\end{Verbatim}
	\verb|\cc| 则定义为显示完汉字(词)后马上把它原样加入索引的命令。
	\begin{Verbatim}[numbers=left,numbersep=1ex,gobble=2,firstnumber=68,tabsize=4,formatcom=\color{darkmiku}]
		\begin{multicols}{4}
			\renewcommand{\glossarysection}[2][]{}
			\renewcommand*{\glsgroupskip}{}
			\setlength{\glstreeindent}{7pt}
			\printglossary[type=TG,style=tree,title=]
		\end{multicols}
	\end{Verbatim}
	四角号码索引分成多个纵栏,采用 tree 样式,第一级的四角号码均无缩进,第二级的汉字(词)排于各自所属的四角号码之后,统一缩进值可根据需要自由调整。
	\end{adjustwidth}
\endgroup

\section{代码实现}

宏包主文件 tetragonos.sty 的全部内容如下:

\begingroup
	\begin{adjustwidth}{3em}{3em}
	\footnotesize
	\begin{Verbatim}[numbers=left,numbersep=1ex,gobble=2,formatcom=\color{darkmiku}]
		%% tetragonos.sty
		%% Copyright 2019 Yuchang Yang < yang.yc.allium@gmail.com >
		%
		% This work may be distributed and/or modified under the
		% conditions of the LaTeX Project Public License, either version 1.3
		% of this license or (at your option) any later version.
		% The latest version of this license is in
		%   http://www.latex-project.org/lppl.txt
		% and version 1.3 or later is part of all distributions of LaTeX
		% version 2005/12/01 or later.
		%
		% This work has the LPPL maintenance status `maintained'.
		% 
		% The Current Maintainer of this work is Yuchang Yang.
		%
		% This work consists of the files tetragonos.sty and tetragonos-database.def
		% and the associated example file tetragonos-example.tex.
		%%
		\NeedsTeXFormat{LaTeX2e}
		\ProvidesPackage{tetragonos}[2019/01/15 v1 package tetragonos]
		\newcommand{\saveTG}[2]{\expandafter\gdef\csname TG-#1\endcsname{#2}}
		\newcommand{\loadTG}[1]{\csname TG-#1\endcsname}
	\end{Verbatim}
	宏 \verb|\saveTG| 与 \verb|\loadTG| 实际上实现了一个索引为汉字字符的数组的写入与读出。
	\begin{Verbatim}[numbers=left,numbersep=1ex,gobble=2,firstnumber=last,formatcom=\color{darkmiku}]
		\makeatletter
		\newcommand{\get@first@char}[1]{\loadTG{#1}\iffalse}
		\newcommand{\getTG}[1]{\get@first@char#1\fi}
		\makeatother
	\end{Verbatim}
	宏 \verb|\getTG| 利用 \TeX{} 的宏机制配合 \verb|\get@first@char|,实现了抓取参数中的第一个字符并查询其四角号码的功能。
	\begin{Verbatim}[numbers=left,numbersep=1ex,gobble=2,firstnumber=last,formatcom=\color{darkmiku}]
		%% tetragonos-database.def
%% Copyright 2019 Yuchang Yang < yang.yc.allium@gmail.com >
%
% This work may be distributed and/or modified under the
% conditions of the LaTeX Project Public License, either version 1.3
% of this license or (at your option) any later version.
% The latest version of this license is in
%   http://www.latex-project.org/lppl.txt
% and version 1.3 or later is part of all distributions of LaTeX
% version 2005/12/01 or later.
%
% This work has the LPPL maintenance status `maintained'.
% 
% The Current Maintainer of this work is Yuchang Yang.
%
% This work consists of the files tetragonos.sty and tetragonos-database.def
% and the associated example file tetragonos-example.tex
%%
%% Original dictionary prepared by wangyanhan at
%% http://bbs.unispim.com/forum.php?mod=viewthread&tid=31674
%%
\ProvidesFile{tetragonos-database.def}[2019/01/14 v1 tetragonos database]
\saveTG{亠}{00000}
\saveTG{弯}{00027}
\saveTG{亪}{00037}
\saveTG{疒}{00100}
\saveTG{韲}{00101}
\saveTG{𣁇}{00102}
\saveTG{𩐐}{00102}
\saveTG{𣁪}{00102}
\saveTG{𥃐}{00102}
\saveTG{𥃂}{00102}
\saveTG{𥂶}{00102}
\saveTG{𥂡}{00102}
\saveTG{𥂬}{00102}
\saveTG{𥃀}{00102}
\saveTG{𥁃}{00102}
\saveTG{𠆊}{00102}
\saveTG{𣁫}{00102}
\saveTG{衁}{00102}
\saveTG{齍}{00102}
\saveTG{亹}{00102}
\saveTG{斖}{00102}
\saveTG{𡔏}{00104}
\saveTG{䜃}{00104}
\saveTG{㙜}{00104}
\saveTG{𡔞}{00104}
\saveTG{𡑔}{00104}
\saveTG{𪤜}{00104}
\saveTG{㻾}{00104}
\saveTG{㙶}{00104}
\saveTG{𡎲}{00104}
\saveTG{𥫍}{00104}
\saveTG{𩫊}{00104}
\saveTG{𡓩}{00104}
\saveTG{堃}{00104}
\saveTG{壅}{00104}
\saveTG{主}{00104}
\saveTG{𤣴}{00104}
\saveTG{𪋻}{00104}
\saveTG{𠅴}{00104}
\saveTG{𨤫}{00105}
\saveTG{𩫍}{00105}
\saveTG{𥪿}{00105}
\saveTG{𥪽}{00105}
\saveTG{童}{00105}
\saveTG{亶}{00106}
\saveTG{𤹼}{00107}
\saveTG{𫒣}{00108}
\saveTG{𨮲}{00108}
\saveTG{𠅉}{00108}
\saveTG{立}{00108}
\saveTG{𨬇}{00109}
\saveTG{銮}{00109}
\saveTG{䥆}{00109}
\saveTG{𤶼}{00110}
\saveTG{𤻡}{00110}
\saveTG{𤷣}{00110}
\saveTG{𤶙}{00111}
\saveTG{痱}{00111}
\saveTG{癧}{00111}
\saveTG{痄}{00111}
\saveTG{症}{00111}
\saveTG{㾚}{00111}
\saveTG{𤷀}{00111}
\saveTG{𤶨}{00111}
\saveTG{㾤}{00112}
\saveTG{𤵢}{00112}
\saveTG{𤻸}{00112}
\saveTG{㾃}{00112}
\saveTG{𤹻}{00112}
\saveTG{𤼓}{00112}
\saveTG{𤻜}{00112}
\saveTG{𤷪}{00112}
\saveTG{𤶰}{00112}
\saveTG{𤸙}{00112}
\saveTG{𤸱}{00112}
\saveTG{𤷷}{00112}
\saveTG{𤵿}{00112}
\saveTG{𤹡}{00112}
\saveTG{㿖}{00112}
\saveTG{𤶻}{00112}
\saveTG{𤼎}{00112}
\saveTG{𤷨}{00112}
\saveTG{𤶢}{00112}
\saveTG{𤸩}{00112}
\saveTG{𤵚}{00112}
\saveTG{𤵁}{00112}
\saveTG{𤺙}{00112}
\saveTG{𤹺}{00112}
\saveTG{𤸸}{00112}
\saveTG{𤶍}{00112}
\saveTG{𤻤}{00112}
\saveTG{疕}{00112}
\saveTG{疪}{00112}
\saveTG{瘪}{00112}
\saveTG{瘥}{00112}
\saveTG{疮}{00112}
\saveTG{疵}{00112}
\saveTG{痥}{00112}
\saveTG{疘}{00112}
\saveTG{𤻷}{00112}
\saveTG{𤷄}{00112}
\saveTG{疽}{00112}
\saveTG{痲}{00112}
\saveTG{痆}{00112}
\saveTG{疱}{00112}
\saveTG{癙}{00112}
\saveTG{瘟}{00112}
\saveTG{痖}{00112}
\saveTG{瘂}{00112}
\saveTG{疣}{00112}
\saveTG{𤺞}{00112}
\saveTG{痉}{00112}
\saveTG{痙}{00112}
\saveTG{瘣}{00113}
\saveTG{𤹊}{00113}
\saveTG{𤷮}{00114}
\saveTG{𤵴}{00114}
\saveTG{𥩣}{00114}
\saveTG{㾏}{00114}
\saveTG{㿀}{00114}
\saveTG{𤶶}{00114}
\saveTG{𤶲}{00114}
\saveTG{㾠}{00114}
\saveTG{㾮}{00114}
\saveTG{𤶄}{00114}
\saveTG{𪏜}{00114}
\saveTG{𤷏}{00114}
\saveTG{𤶚}{00114}
\saveTG{𤷉}{00114}
\saveTG{𤺷}{00114}
\saveTG{瘞}{00114}
\saveTG{𤵞}{00114}
\saveTG{𤷁}{00114}
\saveTG{癍}{00114}
\saveTG{痓}{00114}
\saveTG{痝}{00114}
\saveTG{癦}{00114}
\saveTG{疟}{00114}
\saveTG{瘧}{00114}
\saveTG{痊}{00114}
\saveTG{瘗}{00114}
\saveTG{疰}{00114}
\saveTG{㽵}{00114}
\saveTG{𤴶}{00114}
\saveTG{痤}{00114}
\saveTG{𤶜}{00114}
\saveTG{𤵙}{00115}
\saveTG{𤸰}{00115}
\saveTG{㿈}{00115}
\saveTG{𤻕}{00115}
\saveTG{𤺮}{00115}
\saveTG{𤸇}{00115}
\saveTG{𨾨}{00115}
\saveTG{𩁉}{00115}
\saveTG{𤷬}{00115}
\saveTG{㾖}{00115}
\saveTG{𤺄}{00115}
\saveTG{㢆}{00115}
\saveTG{痽}{00115}
\saveTG{癃}{00115}
\saveTG{瘽}{00115}
\saveTG{癯}{00115}
\saveTG{瘫}{00115}
\saveTG{癱}{00115}
\saveTG{癕}{00115}
\saveTG{癰}{00115}
\saveTG{瘇}{00115}
\saveTG{癨}{00115}
\saveTG{𤼐}{00115}
\saveTG{𤹩}{00115}
\saveTG{㿚}{00115}
\saveTG{㿑}{00115}
\saveTG{痷}{00116}
\saveTG{疸}{00116}
\saveTG{𥫐}{00116}
\saveTG{𤸧}{00116}
\saveTG{𤹪}{00116}
\saveTG{𤺺}{00116}
\saveTG{㾴}{00116}
\saveTG{𤴸}{00117}
\saveTG{𪎠}{00117}
\saveTG{𤵀}{00117}
\saveTG{𤹍}{00117}
\saveTG{𤵔}{00117}
\saveTG{𤷊}{00117}
\saveTG{𤶁}{00117}
\saveTG{㿛}{00117}
\saveTG{𤴰}{00117}
\saveTG{𤷡}{00117}
\saveTG{𤺝}{00117}
\saveTG{𤶃}{00117}
\saveTG{𤶏}{00117}
\saveTG{𤴴}{00117}
\saveTG{𤴺}{00117}
\saveTG{㽸}{00117}
\saveTG{𤹙}{00117}
\saveTG{𤵻}{00117}
\saveTG{㾷}{00117}
\saveTG{痲}{00117}
\saveTG{𤵦}{00117}
\saveTG{𤹈}{00117}
\saveTG{㾌}{00117}
\saveTG{㾩}{00117}
\saveTG{𥪑}{00117}
\saveTG{𥪐}{00117}
\saveTG{𤷅}{00117}
\saveTG{𤵯}{00117}
\saveTG{𥩚}{00117}
\saveTG{𠅰}{00117}
\saveTG{𠅙}{00117}
\saveTG{𤵾}{00117}
\saveTG{𤴱}{00117}
\saveTG{𤵩}{00117}
\saveTG{㾱}{00117}
\saveTG{𪱯}{00117}
\saveTG{𤹃}{00117}
\saveTG{𤷋}{00117}
\saveTG{𤷂}{00117}
\saveTG{𤶛}{00117}
\saveTG{𤴪}{00117}
\saveTG{㽶}{00117}
\saveTG{𤷧}{00117}
\saveTG{𤶋}{00117}
\saveTG{𤵎}{00117}
\saveTG{𤼃}{00117}
\saveTG{𤴦}{00117}
\saveTG{𤴯}{00117}
\saveTG{𤼘}{00117}
\saveTG{𪊤}{00117}
\saveTG{疙}{00117}
\saveTG{疤}{00117}
\saveTG{疯}{00117}
\saveTG{瘋}{00117}
\saveTG{痜}{00117}
\saveTG{疶}{00117}
\saveTG{𢈇}{00117}
\saveTG{𤴥}{00117}
\saveTG{𧮼}{00117}
\saveTG{𪽺}{00117}
\saveTG{𤶮}{00117}
\saveTG{竝}{00118}
\saveTG{瘎}{00118}
\saveTG{痘}{00118}
\saveTG{𫁪}{00118}
\saveTG{𤺌}{00118}
\saveTG{𤷾}{00118}
\saveTG{𤸳}{00118}
\saveTG{𤷟}{00118}
\saveTG{㾣}{00119}
\saveTG{𤸨}{00119}
\saveTG{𤵛}{00119}
\saveTG{㾐}{00120}
\saveTG{𤴮}{00120}
\saveTG{㽱}{00120}
\saveTG{𤷯}{00120}
\saveTG{𤸹}{00120}
\saveTG{𤺨}{00120}
\saveTG{㽼}{00120}
\saveTG{𠅁}{00120}
\saveTG{痸}{00120}
\saveTG{瘌}{00120}
\saveTG{痢}{00120}
\saveTG{𪽯}{00120}
\saveTG{𤷫}{00120}
\saveTG{𤸥}{00121}
\saveTG{𥪜}{00121}
\saveTG{𤺊}{00121}
\saveTG{𤴾}{00121}
\saveTG{𤵈}{00121}
\saveTG{㾨}{00121}
\saveTG{𤻖}{00121}
\saveTG{痹}{00121}
\saveTG{疴}{00121}
\saveTG{痾}{00121}
\saveTG{瘉}{00121}
\saveTG{𤴫}{00121}
\saveTG{疔}{00121}
\saveTG{𤺗}{00121}
\saveTG{𤹧}{00121}
\saveTG{𤻝}{00121}
\saveTG{瘆}{00122}
\saveTG{瘮}{00122}
\saveTG{㾟}{00122}
\saveTG{𤺬}{00122}
\saveTG{𤺋}{00122}
\saveTG{疹}{00122}
\saveTG{瘳}{00122}
\saveTG{癠}{00123}
\saveTG{𤹀}{00127}
\saveTG{𤸎}{00127}
\saveTG{𤶹}{00127}
\saveTG{𤺼}{00127}
\saveTG{𤵆}{00127}
\saveTG{𤶇}{00127}
\saveTG{𤵐}{00127}
\saveTG{𤵠}{00127}
\saveTG{𤷭}{00127}
\saveTG{𤵼}{00127}
\saveTG{𤷱}{00127}
\saveTG{𤹞}{00127}
\saveTG{𤹟}{00127}
\saveTG{𤺯}{00127}
\saveTG{𤹔}{00127}
\saveTG{㿕}{00127}
\saveTG{𤺛}{00127}
\saveTG{𤺖}{00127}
\saveTG{𤺎}{00127}
\saveTG{𤴬}{00127}
\saveTG{𤷽}{00127}
\saveTG{𤺜}{00127}
\saveTG{㾸}{00127}
\saveTG{𤹽}{00127}
\saveTG{㾫}{00127}
\saveTG{𤺉}{00127}
\saveTG{𠅻}{00127}
\saveTG{𪽶}{00127}
\saveTG{𤻟}{00127}
\saveTG{𤸶}{00127}
\saveTG{𤺹}{00127}
\saveTG{𤼒}{00127}
\saveTG{𤸾}{00127}
\saveTG{㾍}{00127}
\saveTG{𤻪}{00127}
\saveTG{㽲}{00127}
\saveTG{𤻈}{00127}
\saveTG{㾶}{00127}
\saveTG{𤼑}{00127}
\saveTG{𤻨}{00127}
\saveTG{𤹢}{00127}
\saveTG{𤵕}{00127}
\saveTG{𤹌}{00127}
\saveTG{𤸼}{00127}
\saveTG{𤹷}{00127}
\saveTG{𤻉}{00127}
\saveTG{𤼡}{00127}
\saveTG{𤺻}{00127}
\saveTG{𤴽}{00127}
\saveTG{㿃}{00127}
\saveTG{𤸝}{00127}
\saveTG{𤹰}{00127}
\saveTG{𥪟}{00127}
\saveTG{𢉑}{00127}
\saveTG{𤷥}{00127}
\saveTG{𤵗}{00127}
\saveTG{𤴹}{00127}
\saveTG{𤺿}{00127}
\saveTG{𤷤}{00127}
\saveTG{㾙}{00127}
\saveTG{𤸷}{00127}
\saveTG{㾓}{00127}
\saveTG{瘸}{00127}
\saveTG{𤺐}{00127}
\saveTG{㿜}{00127}
\saveTG{𪽩}{00127}
\saveTG{𤵶}{00127}
\saveTG{𤷸}{00127}
\saveTG{𤺇}{00127}
\saveTG{𤶸}{00127}
\saveTG{𤹓}{00127}
\saveTG{𤻶}{00127}
\saveTG{㾿}{00127}
\saveTG{𪽱}{00127}
\saveTG{𤺩}{00127}
\saveTG{𤻋}{00127}
\saveTG{𤸀}{00127}
\saveTG{𤹗}{00127}
\saveTG{𤷔}{00127}
\saveTG{𤵏}{00127}
\saveTG{𤵇}{00127}
\saveTG{𤻞}{00127}
\saveTG{𤷛}{00127}
\saveTG{𤸮}{00127}
\saveTG{𤻚}{00127}
\saveTG{𤷗}{00127}
\saveTG{痭}{00127}
\saveTG{癟}{00127}
\saveTG{病}{00127}
\saveTG{瘍}{00127}
\saveTG{瘹}{00127}
\saveTG{疿}{00127}
\saveTG{瘑}{00127}
\saveTG{瘠}{00127}
\saveTG{痀}{00127}
\saveTG{痨}{00127}
\saveTG{癆}{00127}
\saveTG{疠}{00127}
\saveTG{疬}{00127}
\saveTG{𤸒}{00127}
\saveTG{疗}{00127}
\saveTG{瘺}{00127}
\saveTG{鸾}{00127}
\saveTG{癵}{00127}
\saveTG{疓}{00127}
\saveTG{痡}{00127}
\saveTG{痑}{00127}
\saveTG{痛}{00127}
\saveTG{痌}{00127}
\saveTG{痏}{00127}
\saveTG{痫}{00127}
\saveTG{癇}{00127}
\saveTG{癎}{00127}
\saveTG{痟}{00127}
\saveTG{疞}{00127}
\saveTG{癘}{00127}
\saveTG{痬}{00127}
\saveTG{痈}{00127}
\saveTG{㾡}{00127}
\saveTG{𪽹}{00127}
\saveTG{㾺}{00127}
\saveTG{㾰}{00127}
\saveTG{疖}{00127}
\saveTG{癤}{00127}
\saveTG{疡}{00127}
\saveTG{𤸜}{00128}
\saveTG{𤸋}{00128}
\saveTG{疥}{00128}
\saveTG{𤵌}{00129}
\saveTG{痧}{00129}
\saveTG{𤴩}{00130}
\saveTG{𤵂}{00130}
\saveTG{𤸌}{00131}
\saveTG{𤶊}{00131}
\saveTG{𤸐}{00131}
\saveTG{𢌇}{00131}
\saveTG{𤸲}{00131}
\saveTG{𤹛}{00131}
\saveTG{𤸕}{00131}
\saveTG{𤹂}{00131}
\saveTG{𤼚}{00131}
\saveTG{𤻹}{00131}
\saveTG{𤻠}{00131}
\saveTG{𤼜}{00131}
\saveTG{𤻮}{00131}
\saveTG{𤼠}{00131}
\saveTG{𧏼}{00131}
\saveTG{𧖁}{00131}
\saveTG{䘇}{00131}
\saveTG{𧕔}{00131}
\saveTG{𩫲}{00131}
\saveTG{𧕕}{00131}
\saveTG{𧓂}{00131}
\saveTG{𧖊}{00131}
\saveTG{𢌖}{00131}
\saveTG{𤻩}{00131}
\saveTG{𤵄}{00131}
\saveTG{𤐅}{00131}
\saveTG{䇊}{00131}
\saveTG{癋}{00131}
\saveTG{癄}{00131}
\saveTG{疜}{00131}
\saveTG{痣}{00131}
\saveTG{𤷜}{00131}
\saveTG{𢇗}{00131}
\saveTG{𤹒}{00131}
\saveTG{𤻰}{00131}
\saveTG{㾀}{00131}
\saveTG{𤵋}{00131}
\saveTG{𤎉}{00131}
\saveTG{𧓚}{00131}
\saveTG{㾯}{00132}
\saveTG{𤺸}{00132}
\saveTG{𤸉}{00132}
\saveTG{𤶝}{00132}
\saveTG{𤼝}{00132}
\saveTG{𤸁}{00132}
\saveTG{𤵑}{00132}
\saveTG{𤻄}{00132}
\saveTG{癏}{00132}
\saveTG{𤸄}{00132}
\saveTG{𤸖}{00132}
\saveTG{𤼛}{00132}
\saveTG{𤹯}{00132}
\saveTG{𤶒}{00132}
\saveTG{𤻌}{00132}
\saveTG{㾉}{00132}
\saveTG{𤸤}{00132}
\saveTG{𤸬}{00132}
\saveTG{㾗}{00132}
\saveTG{疺}{00132}
\saveTG{瘛}{00132}
\saveTG{痕}{00132}
\saveTG{癑}{00132}
\saveTG{痃}{00132}
\saveTG{癢}{00132}
\saveTG{癊}{00132}
\saveTG{癒}{00132}
\saveTG{痮}{00132}
\saveTG{瘬}{00132}
\saveTG{瘃}{00132}
\saveTG{𤹐}{00132}
\saveTG{𤷶}{00132}
\saveTG{㿆}{00132}
\saveTG{㽷}{00132}
\saveTG{𤷴}{00132}
\saveTG{𧙜}{00132}
\saveTG{𪽫}{00132}
\saveTG{𤺀}{00132}
\saveTG{𤺟}{00133}
\saveTG{疼}{00133}
\saveTG{𤺫}{00133}
\saveTG{瘀}{00133}
\saveTG{𢈅}{00133}
\saveTG{𤹏}{00133}
\saveTG{𤹎}{00133}
\saveTG{𪽵}{00133}
\saveTG{㾼}{00133}
\saveTG{㽿}{00133}
\saveTG{𤻗}{00133}
\saveTG{𪽪}{00134}
\saveTG{𤵘}{00134}
\saveTG{𤻘}{00134}
\saveTG{𤼉}{00134}
\saveTG{𤷞}{00134}
\saveTG{𤹨}{00135}
\saveTG{𩁈}{00135}
\saveTG{𤶉}{00136}
\saveTG{𧕢}{00136}
\saveTG{𤼂}{00136}
\saveTG{𧎿}{00136}
\saveTG{𧉶}{00136}
\saveTG{𧊯}{00136}
\saveTG{𧐠}{00136}
\saveTG{䖟}{00136}
\saveTG{𧋲}{00136}
\saveTG{䗸}{00136}
\saveTG{痋}{00136}
\saveTG{蛮}{00136}
\saveTG{𤹕}{00136}
\saveTG{瘙}{00136}
\saveTG{螡}{00136}
\saveTG{癔}{00136}
\saveTG{蝱}{00136}
\saveTG{𤸛}{00136}
\saveTG{𤻱}{00136}
\saveTG{𤹸}{00136}
\saveTG{𤹿}{00136}
\saveTG{瘜}{00136}
\saveTG{𤸊}{00136}
\saveTG{蚉}{00136}
\saveTG{𠔳}{00137}
\saveTG{㾾}{00137}
\saveTG{𤻑}{00137}
\saveTG{㾽}{00137}
\saveTG{𢇫}{00137}
\saveTG{癮}{00137}
\saveTG{瘾}{00137}
\saveTG{𤹑}{00138}
\saveTG{𤺑}{00138}
\saveTG{𪽷}{00138}
\saveTG{𤼔}{00138}
\saveTG{瘱}{00138}
\saveTG{𤺱}{00138}
\saveTG{𢊒}{00139}
\saveTG{𤼣}{00139}
\saveTG{𤸽}{00139}
\saveTG{𤺣}{00139}
\saveTG{㾵}{00140}
\saveTG{𥩛}{00140}
\saveTG{𤼄}{00140}
\saveTG{疛}{00140}
\saveTG{𤴧}{00140}
\saveTG{}{00140}
\saveTG{𤵒}{00140}
\saveTG{疩}{00141}
\saveTG{癖}{00141}
\saveTG{疨}{00141}
\saveTG{痔}{00141}
\saveTG{𤴲}{00141}
\saveTG{㽳}{00141}
\saveTG{㡰}{00141}
\saveTG{𤶴}{00141}
\saveTG{𥫉}{00141}
\saveTG{𤸿}{00141}
\saveTG{𤻂}{00141}
\saveTG{𤻁}{00141}
\saveTG{㾕}{00141}
\saveTG{𤸓}{00142}
\saveTG{疷}{00142}
\saveTG{𤷖}{00142}
\saveTG{𤵱}{00142}
\saveTG{疧}{00142}
\saveTG{𤹉}{00142}
\saveTG{𤻫}{00143}
\saveTG{㾈}{00143}
\saveTG{𥩸}{00143}
\saveTG{㿒}{00143}
\saveTG{𤹵}{00143}
\saveTG{𤷙}{00143}
\saveTG{𤸟}{00143}
\saveTG{𤹬}{00143}
\saveTG{𤸵}{00143}
\saveTG{𤹘}{00143}
\saveTG{𢋠}{00143}
\saveTG{𤶡}{00143}
\saveTG{𤴵}{00143}
\saveTG{𤵽}{00144}
\saveTG{𤶦}{00144}
\saveTG{瘘}{00144}
\saveTG{痿}{00144}
\saveTG{𤹴}{00144}
\saveTG{㾳}{00144}
\saveTG{𪽳}{00144}
\saveTG{𤸞}{00144}
\saveTG{瘻}{00144}
\saveTG{癭}{00144}
\saveTG{瘿}{00144}
\saveTG{㽴}{00145}
\saveTG{㾘}{00145}
\saveTG{𤵵}{00145}
\saveTG{𤷒}{00146}
\saveTG{𥪮}{00146}
\saveTG{𤷘}{00146}
\saveTG{痺}{00146}
\saveTG{瘴}{00146}
\saveTG{𪽨}{00147}
\saveTG{㿄}{00147}
\saveTG{𤻙}{00147}
\saveTG{𤸦}{00147}
\saveTG{𤴳}{00147}
\saveTG{𤶐}{00147}
\saveTG{𤶖}{00147}
\saveTG{𤻅}{00147}
\saveTG{𤹥}{00147}
\saveTG{𤶽}{00147}
\saveTG{𤶿}{00147}
\saveTG{𤷺}{00147}
\saveTG{𤶣}{00147}
\saveTG{𤻏}{00147}
\saveTG{𤹆}{00147}
\saveTG{䇏}{00147}
\saveTG{𢇽}{00147}
\saveTG{𢇳}{00147}
\saveTG{𪊎}{00147}
\saveTG{𤷵}{00147}
\saveTG{𤺴}{00147}
\saveTG{𤶯}{00147}
\saveTG{疲}{00147}
\saveTG{㽹}{00147}
\saveTG{㾛}{00147}
\saveTG{𤷹}{00147}
\saveTG{𤵬}{00147}
\saveTG{㽻}{00147}
\saveTG{𤹹}{00147}
\saveTG{𤻢}{00147}
\saveTG{𤸑}{00147}
\saveTG{𤹫}{00147}
\saveTG{𤵟}{00147}
\saveTG{𤹱}{00147}
\saveTG{𤵤}{00147}
\saveTG{𤵷}{00147}
\saveTG{𤹋}{00147}
\saveTG{𤷎}{00147}
\saveTG{𤶌}{00147}
\saveTG{𤸃}{00147}
\saveTG{𤶟}{00147}
\saveTG{𤶤}{00147}
\saveTG{瘢}{00147}
\saveTG{癜}{00147}
\saveTG{癈}{00147}
\saveTG{癁}{00147}
\saveTG{痵}{00147}
\saveTG{瘕}{00147}
\saveTG{瘦}{00147}
\saveTG{痩}{00147}
\saveTG{痠}{00147}
\saveTG{痚}{00147}
\saveTG{疫}{00147}
\saveTG{𤷌}{00147}
\saveTG{𤵓}{00147}
\saveTG{𤺅}{00147}
\saveTG{𤺶}{00147}
\saveTG{𤴨}{00147}
\saveTG{㽺}{00147}
\saveTG{㾥}{00147}
\saveTG{㾆}{00147}
\saveTG{𤵧}{00147}
\saveTG{𤶀}{00148}
\saveTG{𤻒}{00148}
\saveTG{𪋌}{00148}
\saveTG{𤵍}{00148}
\saveTG{瘁}{00148}
\saveTG{瘷}{00148}
\saveTG{癓}{00148}
\saveTG{癥}{00148}
\saveTG{𢋳}{00148}
\saveTG{㾲}{00148}
\saveTG{𤺲}{00148}
\saveTG{𤻺}{00148}
\saveTG{㿂}{00148}
\saveTG{𤺍}{00148}
\saveTG{𤷠}{00148}
\saveTG{㿁}{00148}
\saveTG{𤺓}{00148}
\saveTG{𤵣}{00149}
\saveTG{𤹣}{00149}
\saveTG{𤵡}{00149}
\saveTG{𤺃}{00149}
\saveTG{𤵃}{00150}
\saveTG{𤺽}{00151}
\saveTG{癣}{00151}
\saveTG{𤼋}{00151}
\saveTG{痒}{00151}
\saveTG{癬}{00151}
\saveTG{㿏}{00151}
\saveTG{𤻛}{00151}
\saveTG{癴}{00152}
\saveTG{瘒}{00152}
\saveTG{𤶪}{00152}
\saveTG{𤸪}{00152}
\saveTG{𤻀}{00152}
\saveTG{㿍}{00152}
\saveTG{𤸻}{00152}
\saveTG{㾅}{00152}
\saveTG{𢊏}{00152}
\saveTG{𤻦}{00153}
\saveTG{𤻯}{00153}
\saveTG{𪽭}{00153}
\saveTG{𤷇}{00153}
\saveTG{𤹚}{00153}
\saveTG{𤼕}{00153}
\saveTG{𤻵}{00153}
\saveTG{𤵳}{00153}
\saveTG{𤷃}{00153}
\saveTG{𤻻}{00153}
\saveTG{𤺁}{00153}
\saveTG{𤹾}{00154}
\saveTG{𤶞}{00154}
\saveTG{𤵸}{00154}
\saveTG{𤹝}{00154}
\saveTG{𤹖}{00154}
\saveTG{𤵭}{00156}
\saveTG{𤸆}{00156}
\saveTG{𤶱}{00156}
\saveTG{𤻾}{00156}
\saveTG{𢊵}{00156}
\saveTG{瘅}{00156}
\saveTG{癉}{00156}
\saveTG{㾝}{00156}
\saveTG{𤵺}{00156}
\saveTG{𤴿}{00157}
\saveTG{𤵝}{00157}
\saveTG{𤸚}{00157}
\saveTG{𤺭}{00157}
\saveTG{痗}{00157}
\saveTG{𤷰}{00158}
\saveTG{𤺳}{00159}
\saveTG{痐}{00160}
\saveTG{𤷼}{00160}
\saveTG{㾒}{00160}
\saveTG{𤶓}{00160}
\saveTG{𢈺}{00160}
\saveTG{𤹦}{00160}
\saveTG{痴}{00160}
\saveTG{痼}{00160}
\saveTG{痂}{00160}
\saveTG{𤶑}{00160}
\saveTG{𤵖}{00160}
\saveTG{㾇}{00160}
\saveTG{𤹁}{00160}
\saveTG{𤶳}{00161}
\saveTG{𤺦}{00161}
\saveTG{㿊}{00161}
\saveTG{𤺵}{00161}
\saveTG{𤺥}{00161}
\saveTG{㾦}{00161}
\saveTG{㾑}{00161}
\saveTG{𤺪}{00161}
\saveTG{瘄}{00161}
\saveTG{瘩}{00161}
\saveTG{癚}{00161}
\saveTG{癗}{00161}
\saveTG{痦}{00161}
\saveTG{痁}{00161}
\saveTG{瘖}{00161}
\saveTG{𫁥}{00161}
\saveTG{𤷐}{00161}
\saveTG{㿇}{00161}
\saveTG{𤶘}{00161}
\saveTG{𤻣}{00161}
\saveTG{𤻭}{00161}
\saveTG{𢈪}{00161}
\saveTG{𤺒}{00161}
\saveTG{𤵹}{00161}
\saveTG{癅}{00162}
\saveTG{㾪}{00162}
\saveTG{𤷻}{00162}
\saveTG{𪊞}{00162}
\saveTG{瘤}{00162}
\saveTG{𤹜}{00162}
\saveTG{𤵪}{00162}
\saveTG{㾬}{00162}
\saveTG{𤻳}{00162}
\saveTG{𤷳}{00162}
\saveTG{𢋡}{00163}
\saveTG{𤷩}{00163}
\saveTG{𤹇}{00163}
\saveTG{㾂}{00163}
\saveTG{𤻼}{00163}
\saveTG{𪎢}{00164}
\saveTG{𤶂}{00164}
\saveTG{𤸅}{00164}
\saveTG{𤷑}{00164}
\saveTG{瘔}{00164}
\saveTG{痻}{00164}
\saveTG{瘏}{00164}
\saveTG{𤸠}{00164}
\saveTG{㽽}{00164}
\saveTG{𤶾}{00164}
\saveTG{𤻔}{00164}
\saveTG{𤻃}{00164}
\saveTG{𤺈}{00164}
\saveTG{𤶈}{00164}
\saveTG{㾞}{00164}
\saveTG{𤸈}{00164}
\saveTG{㾄}{00165}
\saveTG{癐}{00166}
\saveTG{𤸔}{00166}
\saveTG{㾔}{00166}
\saveTG{㿔}{00166}
\saveTG{㿘}{00166}
\saveTG{𢊴}{00166}
\saveTG{𤺧}{00166}
\saveTG{𤼁}{00167}
\saveTG{𤶷}{00167}
\saveTG{瘡}{00167}
\saveTG{𤹲}{00168}
\saveTG{𤶕}{00168}
\saveTG{𪊿}{00168}
\saveTG{𤼅}{00168}
\saveTG{𤺏}{00169}
\saveTG{痞}{00169}
\saveTG{𢊠}{00169}
\saveTG{𤼍}{00169}
\saveTG{𩠾}{00169}
\saveTG{𩐓}{00172}
\saveTG{𤹳}{00172}
\saveTG{癌}{00172}
\saveTG{疝}{00172}
\saveTG{㽾}{00172}
\saveTG{𤸢}{00172}
\saveTG{𤵅}{00172}
\saveTG{𤸣}{00172}
\saveTG{𤷝}{00172}
\saveTG{𤶵}{00172}
\saveTG{𤶆}{00172}
\saveTG{𤻽}{00172}
\saveTG{𤺠}{00172}
\saveTG{𢇾}{00174}
\saveTG{疳}{00174}
\saveTG{𠅢}{00174}
\saveTG{𤵊}{00175}
\saveTG{痯}{00177}
\saveTG{𪽮}{00177}
\saveTG{𤵰}{00177}
\saveTG{𤻥}{00177}
\saveTG{㾎}{00177}
\saveTG{疭}{00180}
\saveTG{𤴭}{00180}
\saveTG{瘲}{00181}
\saveTG{𤷢}{00181}
\saveTG{𤼈}{00181}
\saveTG{𤷍}{00181}
\saveTG{𤺆}{00181}
\saveTG{𤼌}{00181}
\saveTG{𢈎}{00181}
\saveTG{瘨}{00181}
\saveTG{癡}{00181}
\saveTG{痶}{00181}
\saveTG{𤺤}{00182}
\saveTG{𤺘}{00182}
\saveTG{𤶧}{00182}
\saveTG{𤶺}{00182}
\saveTG{𤺰}{00182}
\saveTG{𤴼}{00182}
\saveTG{𤷓}{00182}
\saveTG{𤻊}{00182}
\saveTG{𤻐}{00182}
\saveTG{𤻍}{00182}
\saveTG{𤼏}{00182}
\saveTG{𤸺}{00182}
\saveTG{𢋆}{00182}
\saveTG{癫}{00182}
\saveTG{瘚}{00182}
\saveTG{癞}{00182}
\saveTG{瘶}{00182}
\saveTG{痍}{00182}
\saveTG{𤼗}{00182}
\saveTG{痎}{00182}
\saveTG{𤻇}{00182}
\saveTG{𤸭}{00182}
\saveTG{㿅}{00182}
\saveTG{𥩲}{00182}
\saveTG{𪽰}{00182}
\saveTG{㿓}{00182}
\saveTG{𨄬}{00182}
\saveTG{𤼢}{00182}
\saveTG{𤶗}{00184}
\saveTG{𤶥}{00184}
\saveTG{𤶅}{00184}
\saveTG{瘈}{00184}
\saveTG{瘯}{00184}
\saveTG{瘊}{00184}
\saveTG{痪}{00184}
\saveTG{瘓}{00184}
\saveTG{疾}{00184}
\saveTG{瘼}{00184}
\saveTG{𤼤}{00184}
\saveTG{𤹶}{00184}
\saveTG{𤺾}{00184}
\saveTG{𤼇}{00184}
\saveTG{𤴻}{00184}
\saveTG{𤵉}{00184}
\saveTG{𤸂}{00184}
\saveTG{𤸗}{00184}
\saveTG{㿙}{00184}
\saveTG{𤷿}{00184}
\saveTG{𤸡}{00185}
\saveTG{𤵨}{00185}
\saveTG{疦}{00185}
\saveTG{𤺕}{00186}
\saveTG{𢊼}{00186}
\saveTG{㿎}{00186}
\saveTG{𤸫}{00186}
\saveTG{𤺂}{00186}
\saveTG{𤼟}{00186}
\saveTG{𤼊}{00186}
\saveTG{㿌}{00186}
\saveTG{癲}{00186}
\saveTG{癀}{00186}
\saveTG{癪}{00186}
\saveTG{癩}{00186}
\saveTG{疻}{00186}
\saveTG{𢊷}{00186}
\saveTG{𤺔}{00186}
\saveTG{𤹠}{00186}
\saveTG{㿉}{00186}
\saveTG{𪎯}{00186}
\saveTG{𤻆}{00186}
\saveTG{𤸘}{00186}
\saveTG{㿗}{00186}
\saveTG{疚}{00187}
\saveTG{瘐}{00187}
\saveTG{𪽴}{00187}
\saveTG{㿐}{00187}
\saveTG{㾜}{00188}
\saveTG{㾭}{00189}
\saveTG{𤵮}{00189}
\saveTG{疢}{00189}
\saveTG{痰}{00189}
\saveTG{瘵}{00191}
\saveTG{𤷈}{00191}
\saveTG{𤵫}{00191}
\saveTG{𤺡}{00191}
\saveTG{𤸏}{00191}
\saveTG{𤻎}{00191}
\saveTG{䴩}{00191}
\saveTG{瘭}{00191}
\saveTG{癝}{00191}
\saveTG{𤺚}{00193}
\saveTG{𤸴}{00193}
\saveTG{𤸍}{00193}
\saveTG{𤼙}{00193}
\saveTG{瘰}{00193}
\saveTG{癳}{00193}
\saveTG{𤵥}{00194}
\saveTG{㾁}{00194}
\saveTG{𤻲}{00194}
\saveTG{𤹅}{00194}
\saveTG{𤻧}{00194}
\saveTG{𤶫}{00194}
\saveTG{𤶭}{00194}
\saveTG{𤹄}{00194}
\saveTG{𤸯}{00194}
\saveTG{㾢}{00194}
\saveTG{𤷕}{00194}
\saveTG{𤺢}{00194}
\saveTG{𢊲}{00194}
\saveTG{癛}{00194}
\saveTG{痳}{00194}
\saveTG{𤶔}{00194}
\saveTG{𤷲}{00194}
\saveTG{𤴷}{00194}
\saveTG{㾋}{00194}
\saveTG{㾻}{00194}
\saveTG{㾹}{00194}
\saveTG{㿋}{00194}
\saveTG{𤶠}{00194}
\saveTG{𤻿}{00194}
\saveTG{𤼀}{00194}
\saveTG{𤷆}{00195}
\saveTG{𤶬}{00195}
\saveTG{㾧}{00195}
\saveTG{𤻬}{00195}
\saveTG{𢋯}{00195}
\saveTG{𤶎}{00195}
\saveTG{㾊}{00195}
\saveTG{𤷦}{00196}
\saveTG{療}{00196}
\saveTG{𤹭}{00198}
\saveTG{𤹮}{00198}
\saveTG{𤵲}{00198}
\saveTG{𤵜}{00198}
\saveTG{𤻴}{00199}
\saveTG{𤹤}{00199}
\saveTG{𪎭}{00199}
\saveTG{𤷚}{00199}
\saveTG{𤶩}{00199}
\saveTG{瘝}{00199}
\saveTG{广}{00200}
\saveTG{𠅘}{00201}
\saveTG{产}{00201}
\saveTG{亭}{00201}
\saveTG{𠅛}{00203}
\saveTG{𩫗}{00206}
\saveTG{亨}{00207}
\saveTG{庐}{00207}
\saveTG{夣}{00207}
\saveTG{𢇜}{00210}
\saveTG{𧑢}{00210}
\saveTG{𢉓}{00211}
\saveTG{𢉮}{00211}
\saveTG{㡸}{00211}
\saveTG{龐}{00211}
\saveTG{靡}{00211}
\saveTG{䨾}{00211}
\saveTG{𪋺}{00212}
\saveTG{𥂓}{00212}
\saveTG{𢈑}{00212}
\saveTG{𢌐}{00212}
\saveTG{𩐕}{00212}
\saveTG{𢌁}{00212}
\saveTG{庇}{00212}
\saveTG{庀}{00212}
\saveTG{充}{00212}
\saveTG{庛}{00212}
\saveTG{麆}{00212}
\saveTG{麤}{00212}
\saveTG{廅}{00212}
\saveTG{巟}{00212}
\saveTG{竟}{00212}
\saveTG{竞}{00212}
\saveTG{𢌔}{00212}
\saveTG{竸}{00212}
\saveTG{廐}{00212}
\saveTG{斍}{00212}
\saveTG{廤}{00212}
\saveTG{廲}{00212}
\saveTG{旈}{00212}
\saveTG{麍}{00212}
\saveTG{鹿}{00212}
\saveTG{廘}{00212}
\saveTG{廬}{00212}
\saveTG{麑}{00212}
\saveTG{庖}{00212}
\saveTG{麅}{00212}
\saveTG{庑}{00212}
\saveTG{兖}{00212}
\saveTG{兗}{00212}
\saveTG{㡹}{00212}
\saveTG{麀}{00212}
\saveTG{㢒}{00212}
\saveTG{競}{00212}
\saveTG{𪪎}{00212}
\saveTG{𪊵}{00212}
\saveTG{𪋳}{00212}
\saveTG{𢈵}{00212}
\saveTG{𢈀}{00212}
\saveTG{𢇚}{00212}
\saveTG{𢉺}{00212}
\saveTG{贏}{00212}
\saveTG{𢇹}{00212}
\saveTG{庒}{00213}
\saveTG{魔}{00213}
\saveTG{廆}{00213}
\saveTG{庣}{00213}
\saveTG{𢉗}{00214}
\saveTG{𠅬}{00214}
\saveTG{𠅨}{00214}
\saveTG{𢋓}{00214}
\saveTG{𢉉}{00214}
\saveTG{𢊃}{00214}
\saveTG{廛}{00214}
\saveTG{塵}{00214}
\saveTG{廃}{00214}
\saveTG{麾}{00214}
\saveTG{庞}{00214}
\saveTG{庬}{00214}
\saveTG{塺}{00214}
\saveTG{庄}{00214}
\saveTG{庢}{00214}
\saveTG{麈}{00214}
\saveTG{座}{00214}
\saveTG{𣁑}{00214}
\saveTG{𢇸}{00214}
\saveTG{𢈜}{00214}
\saveTG{𪪛}{00214}
\saveTG{𪊧}{00214}
\saveTG{𪋂}{00214}
\saveTG{𧍽}{00214}
\saveTG{𢇦}{00214}
\saveTG{𢋨}{00214}
\saveTG{𢉰}{00214}
\saveTG{𢉬}{00214}
\saveTG{𪜤}{00215}
\saveTG{䴤}{00215}
\saveTG{𢋘}{00215}
\saveTG{𤯿}{00215}
\saveTG{𨿶}{00215}
\saveTG{𨿤}{00215}
\saveTG{䧹}{00215}
\saveTG{𨾔}{00215}
\saveTG{𪋠}{00215}
\saveTG{𪋥}{00215}
\saveTG{𣄛}{00215}
\saveTG{產}{00215}
\saveTG{産}{00215}
\saveTG{離}{00215}
\saveTG{廑}{00215}
\saveTG{雍}{00215}
\saveTG{廱}{00215}
\saveTG{𢋊}{00215}
\saveTG{𩀪}{00215}
\saveTG{𢌄}{00215}
\saveTG{𢌏}{00215}
\saveTG{𢊛}{00215}
\saveTG{𪋇}{00215}
\saveTG{𢋬}{00215}
\saveTG{㢈}{00215}
\saveTG{㢕}{00215}
\saveTG{𢌈}{00215}
\saveTG{𨾒}{00215}
\saveTG{𢋃}{00216}
\saveTG{庵}{00216}
\saveTG{麠}{00216}
\saveTG{亱}{00216}
\saveTG{㡺}{00216}
\saveTG{𪊥}{00216}
\saveTG{𩀛}{00216}
\saveTG{𢋔}{00216}
\saveTG{𢇟}{00217}
\saveTG{𢈶}{00217}
\saveTG{𠆚}{00217}
\saveTG{𢈥}{00217}
\saveTG{庛}{00217}
\saveTG{𠆃}{00217}
\saveTG{𩫓}{00217}
\saveTG{𢈃}{00217}
\saveTG{𪋃}{00217}
\saveTG{𪊕}{00217}
\saveTG{𪊯}{00217}
\saveTG{𪋆}{00217}
\saveTG{𩱿}{00217}
\saveTG{𣨻}{00217}
\saveTG{𢇧}{00217}
\saveTG{𢇼}{00217}
\saveTG{𠑽}{00217}
\saveTG{𢉕}{00217}
\saveTG{𧫘}{00217}
\saveTG{𢊀}{00217}
\saveTG{𪎛}{00217}
\saveTG{𪓹}{00217}
\saveTG{𪎣}{00217}
\saveTG{𠅭}{00217}
\saveTG{𣩇}{00217}
\saveTG{𢌓}{00217}
\saveTG{𪗎}{00217}
\saveTG{㡯}{00217}
\saveTG{㢉}{00217}
\saveTG{𢊊}{00217}
\saveTG{㐣}{00217}
\saveTG{𪎜}{00217}
\saveTG{𢇠}{00217}
\saveTG{𢈄}{00217}
\saveTG{𪪝}{00217}
\saveTG{𪪔}{00217}
\saveTG{𪪘}{00217}
\saveTG{𥩕}{00217}
\saveTG{𠅟}{00217}
\saveTG{𢈐}{00217}
\saveTG{𤰑}{00217}
\saveTG{𦣔}{00217}
\saveTG{𠘺}{00217}
\saveTG{𧘊}{00217}
\saveTG{𪎒}{00217}
\saveTG{𪋅}{00217}
\saveTG{𪊚}{00217}
\saveTG{𪊍}{00217}
\saveTG{䴟}{00217}
\saveTG{䊨}{00217}
\saveTG{𡣍}{00217}
\saveTG{𦢼}{00217}
\saveTG{𦏞}{00217}
\saveTG{𦟀}{00217}
\saveTG{䇔}{00217}
\saveTG{𡑤}{00217}
\saveTG{𡳴}{00217}
\saveTG{𣎆}{00217}
\saveTG{𣜄}{00217}
\saveTG{𥢵}{00217}
\saveTG{𦆁}{00217}
\saveTG{𦝠}{00217}
\saveTG{𦣄}{00217}
\saveTG{𦣉}{00217}
\saveTG{𦣖}{00217}
\saveTG{𧝹}{00217}
\saveTG{𨏩}{00217}
\saveTG{𨭞}{00217}
\saveTG{𩼊}{00217}
\saveTG{𠅞}{00217}
\saveTG{𢈌}{00217}
\saveTG{𠅫}{00217}
\saveTG{𪎑}{00217}
\saveTG{𢋩}{00217}
\saveTG{𠙰}{00217}
\saveTG{𣃚}{00217}
\saveTG{庉}{00217}
\saveTG{亢}{00217}
\saveTG{蠃}{00217}
\saveTG{麂}{00217}
\saveTG{羸}{00217}
\saveTG{亮}{00217}
\saveTG{臝}{00217}
\saveTG{驘}{00217}
\saveTG{鸁}{00217}
\saveTG{髚}{00217}
\saveTG{赢}{00217}
\saveTG{嬴}{00217}
\saveTG{𪁚}{00217}
\saveTG{𥩗}{00217}
\saveTG{𥪰}{00217}
\saveTG{𠒋}{00217}
\saveTG{𢊩}{00217}
\saveTG{㐬}{00217}
\saveTG{𢇰}{00217}
\saveTG{𢉚}{00217}
\saveTG{𢊡}{00217}
\saveTG{𠅪}{00217}
\saveTG{𢌒}{00217}
\saveTG{䇂}{00218}
\saveTG{㡴}{00218}
\saveTG{㢄}{00218}
\saveTG{应}{00219}
\saveTG{鏖}{00219}
\saveTG{𢉅}{00219}
\saveTG{廁}{00220}
\saveTG{𢉨}{00220}
\saveTG{𢈱}{00220}
\saveTG{𠅜}{00220}
\saveTG{㢌}{00221}
\saveTG{𡿫}{00221}
\saveTG{𢇡}{00221}
\saveTG{𥩧}{00221}
\saveTG{𪯪}{00221}
\saveTG{𢋛}{00221}
\saveTG{𪎧}{00221}
\saveTG{庍}{00221}
\saveTG{廝}{00221}
\saveTG{庁}{00221}
\saveTG{𢈷}{00221}
\saveTG{𢇮}{00222}
\saveTG{𢊌}{00222}
\saveTG{廖}{00222}
\saveTG{廫}{00222}
\saveTG{彦}{00222}
\saveTG{彥}{00222}
\saveTG{序}{00222}
\saveTG{𢈨}{00222}
\saveTG{𢊥}{00222}
\saveTG{𦠕}{00223}
\saveTG{𪗌}{00223}
\saveTG{齊}{00223}
\saveTG{齌}{00223}
\saveTG{齎}{00223}
\saveTG{齏}{00223}
\saveTG{麡}{00223}
\saveTG{𧓉}{00223}
\saveTG{齋}{00223}
\saveTG{𪗋}{00223}
\saveTG{𩹵}{00223}
\saveTG{𪗉}{00223}
\saveTG{𦠃}{00223}
\saveTG{䶒}{00223}
\saveTG{𪗏}{00223}
\saveTG{𪗈}{00223}
\saveTG{𪗇}{00223}
\saveTG{𪗍}{00223}
\saveTG{斎}{00224}
\saveTG{𪯠}{00224}
\saveTG{斉}{00224}
\saveTG{齑}{00224}
\saveTG{齐}{00224}
\saveTG{𠅥}{00226}
\saveTG{𢅈}{00227}
\saveTG{𢂋}{00227}
\saveTG{𢇺}{00227}
\saveTG{㢐}{00227}
\saveTG{㢁}{00227}
\saveTG{㡻}{00227}
\saveTG{𢉠}{00227}
\saveTG{𩦵}{00227}
\saveTG{𪪢}{00227}
\saveTG{𢉞}{00227}
\saveTG{𢊯}{00227}
\saveTG{𪯨}{00227}
\saveTG{𦚏}{00227}
\saveTG{𠹧}{00227}
\saveTG{𢊑}{00227}
\saveTG{𢉿}{00227}
\saveTG{𢂙}{00227}
\saveTG{𢉀}{00227}
\saveTG{𧥛}{00227}
\saveTG{𠅈}{00227}
\saveTG{𩫎}{00227}
\saveTG{𩫒}{00227}
\saveTG{𩫚}{00227}
\saveTG{𪊒}{00227}
\saveTG{𪋉}{00227}
\saveTG{𪋙}{00227}
\saveTG{𪊫}{00227}
\saveTG{𪋣}{00227}
\saveTG{𪋯}{00227}
\saveTG{𪊛}{00227}
\saveTG{𪊹}{00227}
\saveTG{𪊮}{00227}
\saveTG{㠵}{00227}
\saveTG{𦘻}{00227}
\saveTG{𢂠}{00227}
\saveTG{𢑸}{00227}
\saveTG{䯧}{00227}
\saveTG{𢋚}{00227}
\saveTG{𦓡}{00227}
\saveTG{䳸}{00227}
\saveTG{𪎔}{00227}
\saveTG{𢇱}{00227}
\saveTG{𠜡}{00227}
\saveTG{𩾦}{00227}
\saveTG{𩪠}{00227}
\saveTG{𢋥}{00227}
\saveTG{𢋄}{00227}
\saveTG{𪈲}{00227}
\saveTG{𦡚}{00227}
\saveTG{𦡈}{00227}
\saveTG{㢚}{00227}
\saveTG{𩫌}{00227}
\saveTG{𪎪}{00227}
\saveTG{𢈋}{00227}
\saveTG{𧘯}{00227}
\saveTG{𧙚}{00227}
\saveTG{𧘉}{00227}
\saveTG{𢂪}{00227}
\saveTG{𠆈}{00227}
\saveTG{𪪌}{00227}
\saveTG{𢈓}{00227}
\saveTG{𢇴}{00227}
\saveTG{𦙟}{00227}
\saveTG{䯢}{00227}
\saveTG{𡅟}{00227}
\saveTG{𡄚}{00227}
\saveTG{𡃬}{00227}
\saveTG{𦚾}{00227}
\saveTG{𢌅}{00227}
\saveTG{䐡}{00227}
\saveTG{𢊁}{00227}
\saveTG{𢊶}{00227}
\saveTG{𠞧}{00227}
\saveTG{𪗅}{00227}
\saveTG{𪔉}{00227}
\saveTG{㢋}{00227}
\saveTG{𪗆}{00227}
\saveTG{𢋿}{00227}
\saveTG{𩃽}{00227}
\saveTG{㢊}{00227}
\saveTG{𢊈}{00227}
\saveTG{𪎕}{00227}
\saveTG{㢗}{00227}
\saveTG{𩒩}{00227}
\saveTG{𢉼}{00227}
\saveTG{𢈭}{00227}
\saveTG{𦚍}{00227}
\saveTG{𢉒}{00227}
\saveTG{𪎩}{00227}
\saveTG{旁}{00227}
\saveTG{庯}{00227}
\saveTG{廍}{00227}
\saveTG{离}{00227}
\saveTG{廗}{00227}
\saveTG{帝}{00227}
\saveTG{啇}{00227}
\saveTG{方}{00227}
\saveTG{腐}{00227}
\saveTG{市}{00227}
\saveTG{高}{00227}
\saveTG{膏}{00227}
\saveTG{髙}{00227}
\saveTG{肓}{00227}
\saveTG{脔}{00227}
\saveTG{廓}{00227}
\saveTG{廊}{00227}
\saveTG{廟}{00227}
\saveTG{庈}{00227}
\saveTG{商}{00227}
\saveTG{席}{00227}
\saveTG{裔}{00227}
\saveTG{帟}{00227}
\saveTG{鹰}{00227}
\saveTG{膺}{00227}
\saveTG{鷹}{00227}
\saveTG{育}{00227}
\saveTG{庸}{00227}
\saveTG{庽}{00227}
\saveTG{斋}{00227}
\saveTG{廌}{00227}
\saveTG{𪪗}{00227}
\saveTG{㢙}{00227}
\saveTG{𠾃}{00227}
\saveTG{𦛄}{00227}
\saveTG{𢉢}{00227}
\saveTG{𢉋}{00227}
\saveTG{𢉁}{00227}
\saveTG{𢋽}{00227}
\saveTG{𢂚}{00227}
\saveTG{𢉥}{00227}
\saveTG{𢈉}{00227}
\saveTG{𠆏}{00228}
\saveTG{𪪑}{00228}
\saveTG{𠆂}{00228}
\saveTG{䵉}{00228}
\saveTG{㢏}{00228}
\saveTG{庎}{00228}
\saveTG{𫜗}{00229}
\saveTG{卞}{00230}
\saveTG{亦}{00230}
\saveTG{応}{00230}
\saveTG{𠅯}{00230}
\saveTG{𠅄}{00230}
\saveTG{𪊳}{00230}
\saveTG{𢋭}{00231}
\saveTG{広}{00231}
\saveTG{𢇨}{00231}
\saveTG{𢉍}{00231}
\saveTG{𪊱}{00231}
\saveTG{𢇝}{00231}
\saveTG{𣃡}{00231}
\saveTG{䗪}{00231}
\saveTG{𢌆}{00231}
\saveTG{𢋜}{00231}
\saveTG{𢋙}{00231}
\saveTG{𧖇}{00231}
\saveTG{𢉖}{00231}
\saveTG{麃}{00231}
\saveTG{爢}{00231}
\saveTG{廳}{00231}
\saveTG{廡}{00231}
\saveTG{應}{00231}
\saveTG{𢉐}{00231}
\saveTG{𧲊}{00232}
\saveTG{𪊘}{00232}
\saveTG{𪋁}{00232}
\saveTG{𪋱}{00232}
\saveTG{𧘝}{00232}
\saveTG{𢉭}{00232}
\saveTG{𠆟}{00232}
\saveTG{𢊜}{00232}
\saveTG{𢊔}{00232}
\saveTG{𢈕}{00232}
\saveTG{𢉝}{00232}
\saveTG{𩞁}{00232}
\saveTG{𪋚}{00232}
\saveTG{𪋪}{00232}
\saveTG{𩞇}{00232}
\saveTG{𢉛}{00232}
\saveTG{𢉟}{00232}
\saveTG{㡾}{00232}
\saveTG{㡵}{00232}
\saveTG{㢃}{00232}
\saveTG{𣷖}{00232}
\saveTG{𤔨}{00232}
\saveTG{麎}{00232}
\saveTG{広}{00232}
\saveTG{豪}{00232}
\saveTG{麼}{00232}
\saveTG{𩫕}{00232}
\saveTG{庅}{00232}
\saveTG{庡}{00232}
\saveTG{豙}{00232}
\saveTG{廕}{00232}
\saveTG{𢈫}{00232}
\saveTG{𧜟}{00232}
\saveTG{𢌀}{00232}
\saveTG{𢇤}{00232}
\saveTG{𣱵}{00232}
\saveTG{𢊇}{00232}
\saveTG{𪊬}{00232}
\saveTG{𢊭}{00232}
\saveTG{𧱏}{00232}
\saveTG{㐮}{00232}
\saveTG{麽}{00232}
\saveTG{𩫞}{00232}
\saveTG{庝}{00233}
\saveTG{懬}{00233}
\saveTG{𤫹}{00233}
\saveTG{𫋡}{00233}
\saveTG{𢊧}{00236}
\saveTG{𢊕}{00236}
\saveTG{㢜}{00236}
\saveTG{螷}{00236}
\saveTG{廰}{00236}
\saveTG{蠯}{00236}
\saveTG{𠆁}{00237}
\saveTG{㢘}{00237}
\saveTG{𢈘}{00237}
\saveTG{𢊅}{00237}
\saveTG{廉}{00237}
\saveTG{亷}{00237}
\saveTG{庶}{00237}
\saveTG{𢌍}{00237}
\saveTG{𢞢}{00238}
\saveTG{㥷}{00238}
\saveTG{懬}{00238}
\saveTG{𢈳}{00238}
\saveTG{𢈸}{00238}
\saveTG{𪋝}{00239}
\saveTG{㦄}{00239}
\saveTG{𢇭}{00239}
\saveTG{𪋗}{00239}
\saveTG{𪊓}{00240}
\saveTG{𢋒}{00240}
\saveTG{廚}{00240}
\saveTG{府}{00240}
\saveTG{𪪍}{00240}
\saveTG{𪊙}{00240}
\saveTG{𪪐}{00240}
\saveTG{𪎙}{00241}
\saveTG{𢊢}{00241}
\saveTG{廦}{00241}
\saveTG{庰}{00241}
\saveTG{麛}{00241}
\saveTG{麝}{00241}
\saveTG{庭}{00241}
\saveTG{庌}{00241}
\saveTG{庤}{00241}
\saveTG{𢇥}{00241}
\saveTG{𢋇}{00241}
\saveTG{㡿}{00241}
\saveTG{𢉊}{00241}
\saveTG{㢑}{00241}
\saveTG{𢋖}{00241}
\saveTG{𢇛}{00241}
\saveTG{𦗕}{00241}
\saveTG{𢈯}{00241}
\saveTG{𪊌}{00241}
\saveTG{𪪜}{00242}
\saveTG{底}{00242}
\saveTG{㡳}{00242}
\saveTG{𠆙}{00242}
\saveTG{𪜣}{00243}
\saveTG{𢊍}{00243}
\saveTG{𠅵}{00243}
\saveTG{𢇘}{00244}
\saveTG{㢅}{00244}
\saveTG{𢇩}{00244}
\saveTG{𡡉}{00244}
\saveTG{𡢦}{00244}
\saveTG{𢇷}{00244}
\saveTG{𡢜}{00244}
\saveTG{廔}{00244}
\saveTG{廮}{00244}
\saveTG{𢊤}{00244}
\saveTG{𢋉}{00244}
\saveTG{𢈒}{00244}
\saveTG{𢈔}{00245}
\saveTG{𠅚}{00245}
\saveTG{𢉆}{00245}
\saveTG{庳}{00246}
\saveTG{𪋟}{00246}
\saveTG{㢓}{00246}
\saveTG{麞}{00246}
\saveTG{𪋛}{00246}
\saveTG{𪪖}{00247}
\saveTG{𢇬}{00247}
\saveTG{𢉏}{00247}
\saveTG{𧛌}{00247}
\saveTG{𢉎}{00247}
\saveTG{𢈧}{00247}
\saveTG{𢌂}{00247}
\saveTG{𢊿}{00247}
\saveTG{𢋌}{00247}
\saveTG{𢋺}{00247}
\saveTG{𢋪}{00247}
\saveTG{𢈹}{00247}
\saveTG{𢈡}{00247}
\saveTG{𢌉}{00247}
\saveTG{𢊋}{00247}
\saveTG{𪊴}{00247}
\saveTG{庱}{00247}
\saveTG{度}{00247}
\saveTG{廢}{00247}
\saveTG{废}{00247}
\saveTG{庋}{00247}
\saveTG{庪}{00247}
\saveTG{麚}{00247}
\saveTG{廄}{00247}
\saveTG{廏}{00247}
\saveTG{慶}{00247}
\saveTG{廈}{00247}
\saveTG{廀}{00247}
\saveTG{廋}{00247}
\saveTG{庨}{00247}
\saveTG{夜}{00247}
\saveTG{黀}{00247}
\saveTG{𢈾}{00247}
\saveTG{𢉌}{00247}
\saveTG{𢇯}{00247}
\saveTG{𣄥}{00247}
\saveTG{𢌎}{00247}
\saveTG{𢈖}{00247}
\saveTG{𢈲}{00247}
\saveTG{𪋄}{00247}
\saveTG{𢋶}{00247}
\saveTG{𢋤}{00247}
\saveTG{𢊄}{00247}
\saveTG{𢇪}{00247}
\saveTG{𢋣}{00247}
\saveTG{廠}{00248}
\saveTG{𩫛}{00248}
\saveTG{𢊰}{00248}
\saveTG{𢋻}{00248}
\saveTG{廒}{00248}
\saveTG{𢈰}{00248}
\saveTG{𪋹}{00248}
\saveTG{𢋮}{00248}
\saveTG{𢈼}{00248}
\saveTG{𢋋}{00248}
\saveTG{𢈿}{00251}
\saveTG{𢋼}{00251}
\saveTG{廯}{00251}
\saveTG{庠}{00251}
\saveTG{𢋑}{00251}
\saveTG{𢳁}{00252}
\saveTG{𪊗}{00252}
\saveTG{廨}{00252}
\saveTG{摩}{00252}
\saveTG{𢳅}{00252}
\saveTG{𪎮}{00252}
\saveTG{麙}{00253}
\saveTG{𪊜}{00253}
\saveTG{𪎝}{00253}
\saveTG{㡷}{00253}
\saveTG{𢇙}{00253}
\saveTG{㡲}{00253}
\saveTG{𢈽}{00253}
\saveTG{𠆝}{00253}
\saveTG{𪊩}{00254}
\saveTG{𪋜}{00254}
\saveTG{𢈦}{00254}
\saveTG{库}{00254}
\saveTG{𢈢}{00256}
\saveTG{庫}{00256}
\saveTG{庘}{00256}
\saveTG{㡼}{00256}
\saveTG{𢈤}{00256}
\saveTG{𢉦}{00256}
\saveTG{𨌱}{00256}
\saveTG{𨍗}{00256}
\saveTG{㡽}{00257}
\saveTG{犘}{00259}
\saveTG{𢈛}{00260}
\saveTG{廂}{00260}
\saveTG{𪪚}{00260}
\saveTG{𢇞}{00260}
\saveTG{𢇶}{00260}
\saveTG{𢇵}{00260}
\saveTG{𢊂}{00261}
\saveTG{𥊹}{00261}
\saveTG{𪊦}{00261}
\saveTG{𪊣}{00261}
\saveTG{𪋢}{00261}
\saveTG{䴥}{00261}
\saveTG{𪋓}{00261}
\saveTG{𪊼}{00261}
\saveTG{𪊽}{00261}
\saveTG{𪎗}{00261}
\saveTG{𢊫}{00261}
\saveTG{𩐴}{00261}
\saveTG{𢉩}{00261}
\saveTG{𢈈}{00261}
\saveTG{店}{00261}
\saveTG{庴}{00261}
\saveTG{麕}{00261}
\saveTG{麢}{00261}
\saveTG{廧}{00261}
\saveTG{噟}{00261}
\saveTG{譍}{00261}
\saveTG{𪊨}{00261}
\saveTG{𧧑}{00261}
\saveTG{𢋀}{00261}
\saveTG{䴦}{00261}
\saveTG{䜆}{00261}
\saveTG{𢉲}{00261}
\saveTG{㢇}{00261}
\saveTG{𫜎}{00261}
\saveTG{𢋾}{00261}
\saveTG{𪊪}{00261}
\saveTG{𫜖}{00261}
\saveTG{𢋦}{00261}
\saveTG{𢊉}{00261}
\saveTG{𢈚}{00262}
\saveTG{𠷗}{00262}
\saveTG{𪊰}{00262}
\saveTG{𢊺}{00262}
\saveTG{𢈆}{00262}
\saveTG{𢈣}{00262}
\saveTG{𢋧}{00262}
\saveTG{𢈞}{00262}
\saveTG{磨}{00262}
\saveTG{𢇣}{00262}
\saveTG{廇}{00262}
\saveTG{𢈬}{00264}
\saveTG{𪊺}{00264}
\saveTG{𢉽}{00264}
\saveTG{𢉃}{00264}
\saveTG{𢉜}{00264}
\saveTG{𪋏}{00264}
\saveTG{𪋰}{00264}
\saveTG{𢉷}{00264}
\saveTG{𢋵}{00264}
\saveTG{麔}{00264}
\saveTG{麐}{00264}
\saveTG{庿}{00264}
\saveTG{廜}{00264}
\saveTG{庮}{00264}
\saveTG{𢉳}{00264}
\saveTG{𢋂}{00264}
\saveTG{𢉧}{00264}
\saveTG{𢉱}{00264}
\saveTG{庙}{00265}
\saveTG{唐}{00265}
\saveTG{廥}{00266}
\saveTG{𩫖}{00266}
\saveTG{𪋦}{00266}
\saveTG{𠅠}{00266}
\saveTG{麿}{00266}
\saveTG{麏}{00267}
\saveTG{𪎟}{00267}
\saveTG{𪪒}{00267}
\saveTG{𥉵}{00267}
\saveTG{𢉻}{00267}
\saveTG{𢈁}{00268}
\saveTG{𣋴}{00268}
\saveTG{𢊓}{00269}
\saveTG{𪪓}{00269}
\saveTG{𢋢}{00269}
\saveTG{㢖}{00269}
\saveTG{黁}{00269}
\saveTG{麘}{00269}
\saveTG{𢋁}{00271}
\saveTG{𢇢}{00272}
\saveTG{𢉵}{00272}
\saveTG{𢉾}{00272}
\saveTG{𪙱}{00272}
\saveTG{𢇿}{00272}
\saveTG{𢌕}{00272}
\saveTG{𪋎}{00272}
\saveTG{𢉈}{00272}
\saveTG{𤯌}{00274}
\saveTG{𢈊}{00274}
\saveTG{𤯎}{00274}
\saveTG{𢊘}{00274}
\saveTG{𪋕}{00277}
\saveTG{𠅡}{00277}
\saveTG{𢉂}{00277}
\saveTG{𦉨}{00277}
\saveTG{㢎}{00277}
\saveTG{㢂}{00277}
\saveTG{𠅏}{00277}
\saveTG{𡱀}{00277}
\saveTG{𪪏}{00277}
\saveTG{𢊞}{00278}
\saveTG{𪎞}{00280}
\saveTG{庂}{00280}
\saveTG{𪪕}{00280}
\saveTG{𢊎}{00281}
\saveTG{𠆍}{00281}
\saveTG{𢉴}{00281}
\saveTG{𪊻}{00281}
\saveTG{𪋮}{00281}
\saveTG{㢞}{00281}
\saveTG{𢋸}{00281}
\saveTG{𢊐}{00281}
\saveTG{廙}{00281}
\saveTG{𢋲}{00281}
\saveTG{庼}{00282}
\saveTG{𢈩}{00282}
\saveTG{𢊚}{00282}
\saveTG{𪪧}{00282}
\saveTG{𢉄}{00282}
\saveTG{𢉇}{00282}
\saveTG{𨇭}{00282}
\saveTG{𢋅}{00282}
\saveTG{𢌃}{00282}
\saveTG{𢊦}{00282}
\saveTG{赓}{00282}
\saveTG{㢔}{00282}
\saveTG{}{00282}
\saveTG{𢒲}{00282}
\saveTG{廞}{00282}
\saveTG{𢊗}{00284}
\saveTG{𪪙}{00284}
\saveTG{𢉶}{00284}
\saveTG{䴠}{00284}
\saveTG{𪋐}{00284}
\saveTG{𪋧}{00284}
\saveTG{𪋬}{00284}
\saveTG{𪊏}{00284}
\saveTG{𢈟}{00284}
\saveTG{𪊢}{00284}
\saveTG{庆}{00284}
\saveTG{麌}{00284}
\saveTG{𢇻}{00284}
\saveTG{𢉡}{00284}
\saveTG{𢈻}{00285}
\saveTG{𪊐}{00285}
\saveTG{𢈮}{00285}
\saveTG{㢍}{00285}
\saveTG{𢊾}{00286}
\saveTG{𠆄}{00286}
\saveTG{𢊟}{00286}
\saveTG{𢋍}{00286}
\saveTG{𢊹}{00286}
\saveTG{𢋐}{00286}
\saveTG{𢋫}{00286}
\saveTG{𧷫}{00286}
\saveTG{黂}{00286}
\saveTG{賡}{00286}
\saveTG{廣}{00286}
\saveTG{廭}{00286}
\saveTG{廎}{00286}
\saveTG{𪎬}{00286}
\saveTG{㢛}{00286}
\saveTG{㡶}{00286}
\saveTG{𢉯}{00286}
\saveTG{𢊮}{00286}
\saveTG{𪊾}{00286}
\saveTG{𧸛}{00286}
\saveTG{𢋷}{00286}
\saveTG{𢋞}{00286}
\saveTG{𢊱}{00286}
\saveTG{庹}{00287}
\saveTG{庾}{00287}
\saveTG{㡱}{00287}
\saveTG{𢈏}{00287}
\saveTG{庚}{00287}
\saveTG{庻}{00287}
\saveTG{𢈙}{00288}
\saveTG{𢋗}{00289}
\saveTG{𢉸}{00289}
\saveTG{𪎖}{00289}
\saveTG{㸏}{00289}
\saveTG{𢉙}{00289}
\saveTG{𢉘}{00289}
\saveTG{𢊽}{00289}
\saveTG{𢊝}{00289}
\saveTG{𢊬}{00291}
\saveTG{𢊣}{00291}
\saveTG{𢋎}{00291}
\saveTG{𢋕}{00291}
\saveTG{𢉔}{00291}
\saveTG{廪}{00291}
\saveTG{𦄐}{00293}
\saveTG{縻}{00293}
\saveTG{緳}{00293}
\saveTG{𢊨}{00293}
\saveTG{𢋏}{00294}
\saveTG{𢉤}{00294}
\saveTG{𢉣}{00294}
\saveTG{𢌋}{00294}
\saveTG{𣒷}{00294}
\saveTG{𢈍}{00294}
\saveTG{𥹺}{00294}
\saveTG{𢋹}{00294}
\saveTG{麇}{00294}
\saveTG{廩}{00294}
\saveTG{麜}{00294}
\saveTG{麻}{00294}
\saveTG{糜}{00294}
\saveTG{穈}{00294}
\saveTG{麋}{00294}
\saveTG{庺}{00294}
\saveTG{庩}{00294}
\saveTG{床}{00294}
\saveTG{𢉪}{00294}
\saveTG{𢊖}{00294}
\saveTG{䴢}{00294}
\saveTG{𢋰}{00294}
\saveTG{𢋝}{00294}
\saveTG{𢊸}{00294}
\saveTG{㢝}{00294}
\saveTG{𢌑}{00294}
\saveTG{𢇲}{00294}
\saveTG{庥}{00294}
\saveTG{𢊆}{00294}
\saveTG{𢋟}{00294}
\saveTG{𢉹}{00294}
\saveTG{𢋈}{00294}
\saveTG{𢊳}{00295}
\saveTG{𪋊}{00295}
\saveTG{㢀}{00295}
\saveTG{𢈠}{00295}
\saveTG{𪋔}{00296}
\saveTG{𢈴}{00296}
\saveTG{𡮎}{00296}
\saveTG{𠅽}{00296}
\saveTG{麖}{00296}
\saveTG{庲}{00298}
\saveTG{𢊻}{00298}
\saveTG{𪋈}{00298}
\saveTG{𢈝}{00299}
\saveTG{𡮰}{00299}
\saveTG{𪋩}{00299}
\saveTG{康}{00299}
\saveTG{䴪}{00299}
\saveTG{𪋵}{00299}
\saveTG{𨘰}{00308}
\saveTG{𨖓}{00308}
\saveTG{鴍}{00327}
\saveTG{𪇻}{00327}
\saveTG{𩥊}{00327}
\saveTG{𪀰}{00327}
\saveTG{鵉}{00327}
\saveTG{忘}{00331}
\saveTG{𪐰}{00331}
\saveTG{𢥗}{00331}
\saveTG{𢜖}{00331}
\saveTG{烹}{00332}
\saveTG{𢝃}{00332}
\saveTG{𪬥}{00332}
\saveTG{恋}{00333}
\saveTG{𢢦}{00333}
\saveTG{𢣫}{00333}
\saveTG{𪫳}{00333}
\saveTG{𪸿}{00334}
\saveTG{忞}{00334}
\saveTG{𢣑}{00334}
\saveTG{𣁆}{00334}
\saveTG{𪭀}{00334}
\saveTG{𢜆}{00334}
\saveTG{𢥺}{00334}
\saveTG{𤉿}{00336}
\saveTG{𤉽}{00336}
\saveTG{𧮗}{00336}
\saveTG{𩵳}{00336}
\saveTG{𠆠}{00336}
\saveTG{𢛫}{00336}
\saveTG{𢡃}{00336}
\saveTG{恴}{00336}
\saveTG{意}{00336}
\saveTG{悥}{00336}
\saveTG{𠆐}{00336}
\saveTG{𠅤}{00336}
\saveTG{𢚟}{00337}
\saveTG{𤍍}{00337}
\saveTG{𡿮}{00337}
\saveTG{𢞱}{00338}
\saveTG{𢣽}{00338}
\saveTG{𠆡}{00339}
\saveTG{𢠓}{00339}
\saveTG{𤓥}{00339}
\saveTG{𣁅}{00400}
\saveTG{文}{00400}
\saveTG{𦎧}{00401}
\saveTG{𨐋}{00401}
\saveTG{辛}{00401}
\saveTG{𠆆}{00401}
\saveTG{𦕺}{00401}
\saveTG{斊}{00401}
\saveTG{𠦸}{00403}
\saveTG{䘚}{00403}
\saveTG{𠅋}{00403}
\saveTG{率}{00403}
\saveTG{妄}{00404}
\saveTG{妾}{00404}
\saveTG{娈}{00404}
\saveTG{㛳}{00404}
\saveTG{𩎮}{00405}
\saveTG{章}{00406}
\saveTG{𠆀}{00406}
\saveTG{𣁌}{00406}
\saveTG{𠅖}{00406}
\saveTG{𠅷}{00406}
\saveTG{𩫏}{00406}
\saveTG{変}{00407}
\saveTG{孪}{00407}
\saveTG{享}{00407}
\saveTG{斈}{00407}
\saveTG{孶}{00407}
\saveTG{变}{00407}
\saveTG{﨎}{00407}
\saveTG{𩫃}{00407}
\saveTG{𠅝}{00407}
\saveTG{𠆎}{00407}
\saveTG{𩏉}{00407}
\saveTG{𤕚}{00408}
\saveTG{交}{00408}
\saveTG{卒}{00408}
\saveTG{𡕨}{00409}
\saveTG{𠅦}{00412}
\saveTG{𩙷}{00413}
\saveTG{𨿡}{00415}
\saveTG{𨿼}{00415}
\saveTG{𠆞}{00416}
\saveTG{亴}{00417}
\saveTG{𠅂}{00417}
\saveTG{𤰎}{00427}
\saveTG{𣁠}{00427}
\saveTG{𩫱}{00427}
\saveTG{𪯢}{00440}
\saveTG{𢌮}{00440}
\saveTG{𣁕}{00440}
\saveTG{辯}{00441}
\saveTG{辫}{00441}
\saveTG{辡}{00441}
\saveTG{辮}{00441}
\saveTG{竎}{00441}
\saveTG{㵷}{00441}
\saveTG{𨐵}{00441}
\saveTG{𪊑}{00441}
\saveTG{𤀲}{00441}
\saveTG{𥌊}{00441}
\saveTG{𨐱}{00441}
\saveTG{𠆉}{00441}
\saveTG{㦚}{00441}
\saveTG{㸤}{00441}
\saveTG{𨐰}{00441}
\saveTG{瓣}{00441}
\saveTG{辬}{00441}
\saveTG{辦}{00441}
\saveTG{辧}{00441}
\saveTG{辨}{00441}
\saveTG{辩}{00441}
\saveTG{𠆓}{00442}
\saveTG{弃}{00443}
\saveTG{弈}{00443}
\saveTG{𩓗}{00443}
\saveTG{𫌲}{00446}
\saveTG{𢍓}{00446}
\saveTG{𪯩}{00447}
\saveTG{㝇}{00447}
\saveTG{𠆜}{00447}
\saveTG{𢌸}{00448}
\saveTG{𡣹}{00448}
\saveTG{㚆}{00476}
\saveTG{𦎫}{00501}
\saveTG{𨐌}{00501}
\saveTG{𪺮}{00502}
\saveTG{𥩝}{00502}
\saveTG{𢶜}{00502}
\saveTG{𤚸}{00502}
\saveTG{挛}{00502}
\saveTG{牽}{00503}
\saveTG{㪯}{00504}
\saveTG{𣁄}{00504}
\saveTG{𠅊}{00506}
\saveTG{𥩫}{00506}
\saveTG{𨌤}{00506}
\saveTG{𩫂}{00507}
\saveTG{𤙺}{00508}
\saveTG{𩁛}{00515}
\saveTG{𠂅}{00517}
\saveTG{𢤴}{00551}
\saveTG{𦏝}{00551}
\saveTG{𠱗}{00557}
\saveTG{𥪠}{00562}
\saveTG{𥃰}{00600}
\saveTG{𣅀}{00600}
\saveTG{亩}{00600}
\saveTG{𠅓}{00600}
\saveTG{㐭}{00600}
\saveTG{音}{00601}
\saveTG{𠅿}{00601}
\saveTG{𤰡}{00601}
\saveTG{𧥜}{00601}
\saveTG{𣁎}{00601}
\saveTG{𧨟}{00601}
\saveTG{𣋿}{00601}
\saveTG{㽫}{00601}
\saveTG{𠆒}{00601}
\saveTG{𠆋}{00601}
\saveTG{𠆖}{00601}
\saveTG{盲}{00601}
\saveTG{吂}{00601}
\saveTG{咅}{00601}
\saveTG{暜}{00601}
\saveTG{言}{00601}
\saveTG{訁}{00601}
\saveTG{啻}{00602}
\saveTG{𠱫}{00602}
\saveTG{𥗱}{00602}
\saveTG{𠆗}{00602}
\saveTG{𥆕}{00602}
\saveTG{畜}{00603}
\saveTG{𤰸}{00603}
\saveTG{𪗐}{00603}
\saveTG{𧜊}{00603}
\saveTG{𨢞}{00604}
\saveTG{㖱}{00604}
\saveTG{𪯣}{00604}
\saveTG{吝}{00604}
\saveTG{𠲩}{00604}
\saveTG{𪊲}{00604}
\saveTG{㖖}{00604}
\saveTG{𠲯}{00605}
\saveTG{𠶷}{00605}
\saveTG{亯}{00606}
\saveTG{𥄈}{00608}
\saveTG{㕻}{00609}
\saveTG{𨏯}{00609}
\saveTG{㐯}{00609}
\saveTG{註}{00614}
\saveTG{𧨉}{00614}
\saveTG{𧮓}{00615}
\saveTG{𧮛}{00615}
\saveTG{誰}{00615}
\saveTG{𧬤}{00615}
\saveTG{𨿦}{00615}
\saveTG{𩀖}{00615}
\saveTG{𫍑}{00615}
\saveTG{譠}{00616}
\saveTG{㖜}{00617}
\saveTG{𧫙}{00617}
\saveTG{𧨆}{00617}
\saveTG{𧧢}{00617}
\saveTG{𧫥}{00617}
\saveTG{𧦑}{00617}
\saveTG{𧧉}{00617}
\saveTG{𧦰}{00618}
\saveTG{𧪻}{00618}
\saveTG{𥪼}{00621}
\saveTG{𤾕}{00621}
\saveTG{竒}{00621}
\saveTG{諪}{00621}
\saveTG{𥩬}{00621}
\saveTG{諺}{00622}
\saveTG{𦤘}{00626}
\saveTG{謫}{00627}
\saveTG{朚}{00627}
\saveTG{謪}{00627}
\saveTG{謧}{00627}
\saveTG{謞}{00627}
\saveTG{訪}{00627}
\saveTG{諦}{00627}
\saveTG{𥩮}{00627}
\saveTG{謗}{00627}
\saveTG{𥩭}{00627}
\saveTG{𧨑}{00627}
\saveTG{𧧩}{00630}
\saveTG{譙}{00631}
\saveTG{䜇}{00632}
\saveTG{譹}{00632}
\saveTG{詃}{00632}
\saveTG{譲}{00632}
\saveTG{讓}{00632}
\saveTG{誸}{00632}
\saveTG{讁}{00632}
\saveTG{𧫀}{00632}
\saveTG{𧮨}{00632}
\saveTG{𧭧}{00633}
\saveTG{譩}{00636}
\saveTG{䪰}{00636}
\saveTG{謶}{00637}
\saveTG{𧧮}{00637}
\saveTG{䜞}{00637}
\saveTG{譧}{00637}
\saveTG{𫌴}{00640}
\saveTG{䛨}{00641}
\saveTG{𧩔}{00641}
\saveTG{𧩯}{00641}
\saveTG{𧭺}{00641}
\saveTG{𧭫}{00641}
\saveTG{𧨱}{00641}
\saveTG{𫍗}{00641}
\saveTG{𧩕}{00644}
\saveTG{𧧄}{00644}
\saveTG{𧨐}{00644}
\saveTG{𧦣}{00644}
\saveTG{𧫱}{00646}
\saveTG{諄}{00647}
\saveTG{謢}{00647}
\saveTG{𧩧}{00647}
\saveTG{𧩽}{00647}
\saveTG{𧩅}{00647}
\saveTG{𩐟}{00648}
\saveTG{詨}{00648}
\saveTG{誶}{00648}
\saveTG{𠆛}{00656}
\saveTG{讟}{00661}
\saveTG{誩}{00661}
\saveTG{譶}{00661}
\saveTG{𧮊}{00661}
\saveTG{𣞇}{00661}
\saveTG{𧪄}{00661}
\saveTG{𧮦}{00661}
\saveTG{𧩵}{00661}
\saveTG{䪭}{00661}
\saveTG{𥩶}{00661}
\saveTG{𧭘}{00661}
\saveTG{諳}{00661}
\saveTG{𧭛}{00661}
\saveTG{𥗧}{00662}
\saveTG{𧬍}{00662}
\saveTG{䛸}{00662}
\saveTG{𨑂}{00664}
\saveTG{𨐼}{00664}
\saveTG{𧫨}{00667}
\saveTG{𧮠}{00669}
\saveTG{𧧣}{00672}
\saveTG{該}{00682}
\saveTG{𧫟}{00684}
\saveTG{𧪠}{00684}
\saveTG{𧫰}{00685}
\saveTG{𠆌}{00686}
\saveTG{𧭰}{00686}
\saveTG{𩑈}{00686}
\saveTG{𧩉}{00687}
\saveTG{𧫽}{00689}
\saveTG{𧫼}{00694}
\saveTG{𧨯}{00694}
\saveTG{諒}{00696}
\saveTG{亡}{00710}
\saveTG{𪜥}{00712}
\saveTG{𧙌}{00712}
\saveTG{𠅸}{00712}
\saveTG{亳}{00714}
\saveTG{毫}{00714}
\saveTG{𥪖}{00714}
\saveTG{𠅾}{00715}
\saveTG{竜}{00716}
\saveTG{𪓖}{00717}
\saveTG{䯩}{00717}
\saveTG{甕}{00717}
\saveTG{𣬙}{00717}
\saveTG{㐔}{00717}
\saveTG{𠅔}{00717}
\saveTG{𤣥}{00722}
\saveTG{𧚍}{00729}
\saveTG{𧚹}{00732}
\saveTG{𧜫}{00732}
\saveTG{𧛨}{00732}
\saveTG{𧘠}{00732}
\saveTG{𧛫}{00732}
\saveTG{𧘎}{00732}
\saveTG{𧙩}{00732}
\saveTG{𧚽}{00732}
\saveTG{𧙙}{00732}
\saveTG{𧚞}{00732}
\saveTG{𧞗}{00732}
\saveTG{𧙎}{00732}
\saveTG{𧙃}{00732}
\saveTG{𧝥}{00732}
\saveTG{𧚤}{00732}
\saveTG{𧘨}{00732}
\saveTG{𠅕}{00732}
\saveTG{𧚰}{00732}
\saveTG{𧘘}{00732}
\saveTG{𧘙}{00732}
\saveTG{𧞻}{00732}
\saveTG{𧝻}{00732}
\saveTG{𧝾}{00732}
\saveTG{䘱}{00732}
\saveTG{𧘼}{00732}
\saveTG{𧙪}{00732}
\saveTG{𧝕}{00732}
\saveTG{𧙏}{00732}
\saveTG{𧜪}{00732}
\saveTG{𧘮}{00732}
\saveTG{𩛚}{00732}
\saveTG{𧞂}{00732}
\saveTG{𧜍}{00732}
\saveTG{𩝦}{00732}
\saveTG{𧘳}{00732}
\saveTG{𧙰}{00732}
\saveTG{𧙵}{00732}
\saveTG{𧙦}{00732}
\saveTG{𧜈}{00732}
\saveTG{𧜸}{00732}
\saveTG{𤇯}{00732}
\saveTG{𧝢}{00732}
\saveTG{𧘭}{00732}
\saveTG{𧛙}{00732}
\saveTG{𧜯}{00732}
\saveTG{褱}{00732}
\saveTG{𧝺}{00732}
\saveTG{𧛿}{00732}
\saveTG{哀}{00732}
\saveTG{褒}{00732}
\saveTG{袌}{00732}
\saveTG{裒}{00732}
\saveTG{裦}{00732}
\saveTG{襃}{00732}
\saveTG{袲}{00732}
\saveTG{衰}{00732}
\saveTG{裵}{00732}
\saveTG{衮}{00732}
\saveTG{袞}{00732}
\saveTG{裹}{00732}
\saveTG{裏}{00732}
\saveTG{袤}{00732}
\saveTG{褭}{00732}
\saveTG{襄}{00732}
\saveTG{亵}{00732}
\saveTG{衺}{00732}
\saveTG{褻}{00732}
\saveTG{褎}{00732}
\saveTG{褏}{00732}
\saveTG{玄}{00732}
\saveTG{玆}{00732}
\saveTG{衣}{00732}
\saveTG{裛}{00732}
\saveTG{饔}{00732}
\saveTG{袬}{00732}
\saveTG{袠}{00732}
\saveTG{衷}{00732}
\saveTG{𫋴}{00732}
\saveTG{𧟛}{00732}
\saveTG{𧜏}{00732}
\saveTG{𧛁}{00732}
\saveTG{𧙬}{00732}
\saveTG{𧙴}{00732}
\saveTG{𧚾}{00732}
\saveTG{𧜉}{00732}
\saveTG{𧞚}{00732}
\saveTG{𧚱}{00732}
\saveTG{𧘩}{00732}
\saveTG{𠆘}{00732}
\saveTG{裏}{00732}
\saveTG{𧘦}{00732}
\saveTG{𧝑}{00732}
\saveTG{𧙉}{00732}
\saveTG{𧘫}{00732}
\saveTG{䙝}{00732}
\saveTG{𧚌}{00732}
\saveTG{𧞯}{00732}
\saveTG{褢}{00732}
\saveTG{𣱅}{00747}
\saveTG{𣫭}{00757}
\saveTG{𡻤}{00771}
\saveTG{𡵡}{00772}
\saveTG{峦}{00772}
\saveTG{罋}{00772}
\saveTG{㐫}{00772}
\saveTG{𪙯}{00772}
\saveTG{𡵍}{00772}
\saveTG{𪚎}{00772}
\saveTG{𪙔}{00772}
\saveTG{𦦏}{00773}
\saveTG{𤼆}{00793}
\saveTG{六}{00800}
\saveTG{𥪢}{00801}
\saveTG{𠆑}{00801}
\saveTG{𩇰}{00801}
\saveTG{𥫆}{00801}
\saveTG{𠔚}{00801}
\saveTG{𡖍}{00802}
\saveTG{𨀶}{00802}
\saveTG{𨃱}{00802}
\saveTG{𪎫}{00802}
\saveTG{𠅆}{00802}
\saveTG{龰}{00802}
\saveTG{亥}{00802}
\saveTG{𠅲}{00804}
\saveTG{𩫀}{00804}
\saveTG{𠅌}{00804}
\saveTG{𠆇}{00804}
\saveTG{𧦂}{00804}
\saveTG{䯨}{00804}
\saveTG{𠅑}{00804}
\saveTG{㐪}{00804}
\saveTG{奕}{00804}
\saveTG{𡙩}{00804}
\saveTG{𥏉}{00804}
\saveTG{𧷔}{00806}
\saveTG{𧴻}{00806}
\saveTG{𧷞}{00806}
\saveTG{𧴨}{00806}
\saveTG{𧷮}{00806}
\saveTG{賌}{00806}
\saveTG{𧶜}{00806}
\saveTG{𤆴}{00809}
\saveTG{焤}{00809}
\saveTG{𤌾}{00809}
\saveTG{𥫈}{00814}
\saveTG{𠅧}{00846}
\saveTG{䝺}{00864}
\saveTG{𠅃}{00880}
\saveTG{𥜘}{00901}
\saveTG{禀}{00901}
\saveTG{𪨂}{00901}
\saveTG{𥿿}{00903}
\saveTG{𥾻}{00903}
\saveTG{紊}{00903}
\saveTG{𠆅}{00904}
\saveTG{稟}{00904}
\saveTG{椉}{00904}
\saveTG{槀}{00904}
\saveTG{稁}{00904}
\saveTG{稾}{00904}
\saveTG{栾}{00904}
\saveTG{杗}{00904}
\saveTG{棄}{00904}
\saveTG{亲}{00904}
\saveTG{𠅺}{00904}
\saveTG{𣖈}{00904}
\saveTG{𣠛}{00904}
\saveTG{𥠤}{00904}
\saveTG{𣕏}{00904}
\saveTG{𣓀}{00904}
\saveTG{𣖇}{00904}
\saveTG{㪰}{00904}
\saveTG{𢊪}{00904}
\saveTG{𣐽}{00905}
\saveTG{亰}{00906}
\saveTG{京}{00906}
\saveTG{𣁃}{00907}
\saveTG{禀}{00911}
\saveTG{雜}{00915}
\saveTG{𠅅}{00940}
\saveTG{𪲪}{00946}
\saveTG{𣚏}{00946}
\saveTG{𧛉}{00948}
\saveTG{𠅼}{00985}
\saveTG{𦇨}{00993}
\saveTG{𣡥}{00994}
\saveTG{𪎦}{00994}
\saveTG{𥾁}{01026}
\saveTG{𪚠}{01102}
\saveTG{𡔕}{01104}
\saveTG{壟}{01104}
\saveTG{𣥢}{01112}
\saveTG{雿}{01113}
\saveTG{𥩨}{01114}
\saveTG{竰}{01115}
\saveTG{𥩰}{01117}
\saveTG{㼿}{01117}
\saveTG{𥩼}{01117}
\saveTG{𤮜}{01117}
\saveTG{𥩤}{01126}
\saveTG{䇕}{01127}
\saveTG{𩦍}{01127}
\saveTG{竵}{01127}
\saveTG{𥫀}{01127}
\saveTG{𥫇}{01127}
\saveTG{蠪}{01136}
\saveTG{𤼖}{01136}
\saveTG{𥪌}{01138}
\saveTG{竨}{01146}
\saveTG{𥪺}{01155}
\saveTG{站}{01160}
\saveTG{𥫌}{01162}
\saveTG{竡}{01162}
\saveTG{䇉}{01162}
\saveTG{𠃎}{01170}
\saveTG{颤}{01182}
\saveTG{𥪙}{01186}
\saveTG{𩕉}{01186}
\saveTG{顫}{01186}
\saveTG{𪱮}{01186}
\saveTG{㡣}{01211}
\saveTG{龍}{01211}
\saveTG{龘}{01211}
\saveTG{龖}{01211}
\saveTG{𣄬}{01212}
\saveTG{𦣭}{01212}
\saveTG{𥫏}{01214}
\saveTG{㼚}{01217}
\saveTG{㼾}{01217}
\saveTG{𤮢}{01217}
\saveTG{㼺}{01217}
\saveTG{𤮊}{01217}
\saveTG{𪚥}{01217}
\saveTG{𩣐}{01217}
\saveTG{甋}{01217}
\saveTG{𤭕}{01217}
\saveTG{𪩧}{01217}
\saveTG{瓬}{01217}
\saveTG{𪊔}{01221}
\saveTG{𪎱}{01227}
\saveTG{𪚞}{01227}
\saveTG{𢄫}{01227}
\saveTG{𪚑}{01227}
\saveTG{𠅹}{01231}
\saveTG{䴫}{01232}
\saveTG{𧞮}{01232}
\saveTG{𪗑}{01241}
\saveTG{𪗒}{01241}
\saveTG{𢌊}{01244}
\saveTG{𢾊}{01247}
\saveTG{𣀫}{01247}
\saveTG{𪎘}{01247}
\saveTG{敲}{01247}
\saveTG{𢾬}{01247}
\saveTG{𢾖}{01247}
\saveTG{𣀯}{01247}
\saveTG{𢿇}{01247}
\saveTG{𣀃}{01247}
\saveTG{𪋶}{01261}
\saveTG{𣃷}{01262}
\saveTG{𡖵}{01262}
\saveTG{𧱳}{01264}
\saveTG{颃}{01282}
\saveTG{颜}{01282}
\saveTG{頏}{01286}
\saveTG{顡}{01286}
\saveTG{顔}{01286}
\saveTG{𩔍}{01286}
\saveTG{𩕺}{01286}
\saveTG{𩕴}{01286}
\saveTG{䫕}{01286}
\saveTG{𩓲}{01286}
\saveTG{顏}{01286}
\saveTG{𩔼}{01286}
\saveTG{𩒘}{01286}
\saveTG{䫯}{01286}
\saveTG{𩒊}{01286}
\saveTG{𣄩}{01286}
\saveTG{𣄫}{01286}
\saveTG{䯪}{01286}
\saveTG{𩖆}{01286}
\saveTG{𩫢}{01286}
\saveTG{𩔴}{01286}
\saveTG{𩕅}{01286}
\saveTG{𩔶}{01286}
\saveTG{𩕲}{01286}
\saveTG{𩔽}{01286}
\saveTG{𩔣}{01286}
\saveTG{𩕇}{01286}
\saveTG{𣄌}{01286}
\saveTG{𩒯}{01286}
\saveTG{𣄔}{01291}
\saveTG{䴨}{01296}
\saveTG{𤭉}{01317}
\saveTG{驡}{01327}
\saveTG{鸗}{01327}
\saveTG{𢤲}{01331}
\saveTG{𢥹}{01338}
\saveTG{𢥮}{01338}
\saveTG{戅}{01338}
\saveTG{𢥭}{01342}
\saveTG{聾}{01401}
\saveTG{𪚛}{01401}
\saveTG{𪎁}{01407}
\saveTG{𪚢}{01413}
\saveTG{𩫧}{01416}
\saveTG{𤮩}{01417}
\saveTG{𤭢}{01417}
\saveTG{虠}{01417}
\saveTG{𡦨}{01417}
\saveTG{𣮢}{01417}
\saveTG{𤭞}{01417}
\saveTG{}{01417}
\saveTG{𨰯}{01419}
\saveTG{𥫒}{01436}
\saveTG{𪚜}{01441}
\saveTG{龏}{01441}
\saveTG{𦪿}{01447}
\saveTG{𣦮}{01452}
\saveTG{𩕆}{01486}
\saveTG{顇}{01486}
\saveTG{贑}{01486}
\saveTG{頝}{01486}
\saveTG{𩓉}{01486}
\saveTG{𣁡}{01491}
\saveTG{𨐣}{01494}
\saveTG{𢸭}{01502}
\saveTG{𡃡}{01601}
\saveTG{𧮩}{01601}
\saveTG{讋}{01601}
\saveTG{礱}{01602}
\saveTG{訨}{01610}
\saveTG{証}{01611}
\saveTG{讈}{01611}
\saveTG{誹}{01611}
\saveTG{誆}{01611}
\saveTG{𧨊}{01611}
\saveTG{𧦩}{01611}
\saveTG{誙}{01612}
\saveTG{𩑊}{01612}
\saveTG{䪫}{01612}
\saveTG{譃}{01612}
\saveTG{𧬩}{01612}
\saveTG{䪦}{01612}
\saveTG{訌}{01612}
\saveTG{謯}{01612}
\saveTG{𧥮}{01612}
\saveTG{䛩}{01612}
\saveTG{𧬃}{01612}
\saveTG{𧩄}{01612}
\saveTG{𧧰}{01612}
\saveTG{𧥶}{01614}
\saveTG{諲}{01614}
\saveTG{𧫔}{01614}
\saveTG{𧩆}{01614}
\saveTG{𧩦}{01614}
\saveTG{誑}{01614}
\saveTG{誈}{01614}
\saveTG{謔}{01614}
\saveTG{𧦅}{01614}
\saveTG{𫍁}{01615}
\saveTG{讍}{01616}
\saveTG{謳}{01616}
\saveTG{𩐛}{01617}
\saveTG{䛃}{01617}
\saveTG{詎}{01617}
\saveTG{𧭣}{01617}
\saveTG{諕}{01617}
\saveTG{𧩗}{01617}
\saveTG{瓿}{01617}
\saveTG{𧦠}{01617}
\saveTG{𧭌}{01617}
\saveTG{𧧀}{01617}
\saveTG{𧥭}{01617}
\saveTG{𣁔}{01617}
\saveTG{𧯹}{01618}
\saveTG{誣}{01618}
\saveTG{䛠}{01618}
\saveTG{訂}{01620}
\saveTG{訶}{01620}
\saveTG{謌}{01621}
\saveTG{𧬐}{01621}
\saveTG{𧫌}{01621}
\saveTG{𧭡}{01622}
\saveTG{𧨪}{01622}
\saveTG{𧨷}{01622}
\saveTG{𧨂}{01626}
\saveTG{𧭉}{01627}
\saveTG{譳}{01627}
\saveTG{謣}{01627}
\saveTG{𧦯}{01627}
\saveTG{𧩂}{01627}
\saveTG{𧦉}{01627}
\saveTG{𧥦}{01627}
\saveTG{𧦿}{01627}
\saveTG{䛔}{01627}
\saveTG{韴}{01627}
\saveTG{䛿}{01627}
\saveTG{𧬭}{01627}
\saveTG{𧭱}{01631}
\saveTG{𧧎}{01631}
\saveTG{𧦚}{01631}
\saveTG{䜑}{01631}
\saveTG{𧬞}{01631}
\saveTG{𧥼}{01631}
\saveTG{𧭲}{01631}
\saveTG{䛫}{01632}
\saveTG{𧬷}{01632}
\saveTG{𧭜}{01632}
\saveTG{誫}{01632}
\saveTG{諑}{01632}
\saveTG{𧪊}{01636}
\saveTG{詽}{01640}
\saveTG{𧥪}{01640}
\saveTG{訝}{01640}
\saveTG{訮}{01640}
\saveTG{訏}{01640}
\saveTG{訐}{01640}
\saveTG{誀}{01640}
\saveTG{𩐮}{01641}
\saveTG{𧨢}{01641}
\saveTG{𧦡}{01641}
\saveTG{讘}{01641}
\saveTG{𧧝}{01642}
\saveTG{𧬟}{01645}
\saveTG{𧨒}{01645}
\saveTG{譚}{01646}
\saveTG{𧨳}{01646}
\saveTG{䜡}{01647}
\saveTG{𢾑}{01647}
\saveTG{㪗}{01647}
\saveTG{𢿪}{01647}
\saveTG{𧪤}{01648}
\saveTG{謼}{01649}
\saveTG{評}{01649}
\saveTG{𧥯}{01650}
\saveTG{𧬨}{01652}
\saveTG{𧫅}{01652}
\saveTG{𧭖}{01653}
\saveTG{詀}{01660}
\saveTG{語}{01661}
\saveTG{𧪽}{01661}
\saveTG{譖}{01661}
\saveTG{𧨼}{01662}
\saveTG{𧫓}{01662}
\saveTG{𧩤}{01662}
\saveTG{𧦳}{01662}
\saveTG{𧨿}{01662}
\saveTG{𧧍}{01664}
\saveTG{𦧨}{01664}
\saveTG{𩑆}{01664}
\saveTG{諨}{01666}
\saveTG{䜜}{01668}
\saveTG{𪘙}{01672}
\saveTG{𧩐}{01681}
\saveTG{𧫪}{01681}
\saveTG{𧪌}{01682}
\saveTG{䚶}{01684}
\saveTG{𩐵}{01686}
\saveTG{𫖙}{01686}
\saveTG{𩐳}{01686}
\saveTG{䫓}{01686}
\saveTG{䜖}{01686}
\saveTG{𧩬}{01686}
\saveTG{𧨗}{01689}
\saveTG{𩐡}{01691}
\saveTG{𧧌}{01691}
\saveTG{謤}{01691}
\saveTG{𧬎}{01694}
\saveTG{謜}{01696}
\saveTG{𣥊}{01712}
\saveTG{𤮨}{01717}
\saveTG{𪷹}{01726}
\saveTG{𧟟}{01732}
\saveTG{襲}{01732}
\saveTG{𣀮}{01747}
\saveTG{𣀤}{01747}
\saveTG{𡾩}{01772}
\saveTG{𩑹}{01786}
\saveTG{龔}{01801}
\saveTG{𨇘}{01802}
\saveTG{𪚔}{01804}
\saveTG{龑}{01804}
\saveTG{𠅮}{01817}
\saveTG{颏}{01882}
\saveTG{頦}{01886}
\saveTG{龒}{01901}
\saveTG{㰍}{01904}
\saveTG{𩫡}{01916}
\saveTG{𢀮}{01917}
\saveTG{𣄵}{01917}
\saveTG{𣄶}{01917}
\saveTG{𢽫}{01947}
\saveTG{㔊}{02100}
\saveTG{𠞑}{02100}
\saveTG{𠟍}{02100}
\saveTG{𩙼}{02113}
\saveTG{氃}{02114}
\saveTG{竓}{02114}
\saveTG{氈}{02114}
\saveTG{𣯞}{02117}
\saveTG{𣯓}{02117}
\saveTG{𥪪}{02118}
\saveTG{竳}{02118}
\saveTG{𦚎}{02121}
\saveTG{𥪄}{02127}
\saveTG{端}{02127}
\saveTG{𥩜}{02127}
\saveTG{竬}{02127}
\saveTG{𫋉}{02136}
\saveTG{竏}{02140}
\saveTG{𥪍}{02144}
\saveTG{𠘯}{02170}
\saveTG{𥪧}{02181}
\saveTG{𥪦}{02184}
\saveTG{𥪕}{02195}
\saveTG{𠜢}{02200}
\saveTG{𠝡}{02200}
\saveTG{𠚳}{02200}
\saveTG{𠞟}{02200}
\saveTG{𠝇}{02200}
\saveTG{𠞥}{02200}
\saveTG{𠠒}{02200}
\saveTG{剷}{02200}
\saveTG{𠛍}{02200}
\saveTG{剫}{02200}
\saveTG{剂}{02200}
\saveTG{剤}{02200}
\saveTG{劑}{02200}
\saveTG{劆}{02200}
\saveTG{劘}{02200}
\saveTG{𪟍}{02200}
\saveTG{𠞶}{02200}
\saveTG{𠠟}{02200}
\saveTG{𪟒}{02200}
\saveTG{𠠎}{02200}
\saveTG{𩫔}{02214}
\saveTG{𣯵}{02217}
\saveTG{𣯟}{02217}
\saveTG{𩫁}{02217}
\saveTG{𣬵}{02217}
\saveTG{㲥}{02217}
\saveTG{㫂}{02221}
\saveTG{𪯲}{02221}
\saveTG{𣂳}{02221}
\saveTG{𢒌}{02222}
\saveTG{𢒹}{02222}
\saveTG{𢒺}{02222}
\saveTG{㣑}{02222}
\saveTG{𪋸}{02227}
\saveTG{}{02227}
\saveTG{𤔟}{02230}
\saveTG{𠦳}{02240}
\saveTG{𪋋}{02241}
\saveTG{𪊶}{02241}
\saveTG{䴧}{02244}
\saveTG{𧞥}{02244}
\saveTG{𣄁}{02247}
\saveTG{䴣}{02249}
\saveTG{𪎚}{02250}
\saveTG{𩠪}{02262}
\saveTG{旙}{02269}
\saveTG{𪋒}{02269}
\saveTG{𩡑}{02269}
\saveTG{𩡕}{02269}
\saveTG{𪋡}{02285}
\saveTG{𩕝}{02286}
\saveTG{𪋫}{02295}
\saveTG{𨔄}{02302}
\saveTG{𢣪}{02332}
\saveTG{𪬴}{02332}
\saveTG{刘}{02400}
\saveTG{𠞩}{02400}
\saveTG{𠜅}{02400}
\saveTG{彣}{02402}
\saveTG{㜪}{02404}
\saveTG{乵}{02410}
\saveTG{𪧃}{02415}
\saveTG{㲔}{02417}
\saveTG{㲞}{02417}
\saveTG{乵}{02417}
\saveTG{𨐐}{02417}
\saveTG{𨐸}{02418}
\saveTG{𧰄}{02418}
\saveTG{𧯺}{02418}
\saveTG{彰}{02422}
\saveTG{𠡖}{02427}
\saveTG{𨐏}{02427}
\saveTG{𡣎}{02442}
\saveTG{𣀦}{02447}
\saveTG{𨐿}{02486}
\saveTG{𤗔}{02527}
\saveTG{訠}{02600}
\saveTG{𠝒}{02600}
\saveTG{𧥞}{02600}
\saveTG{𧩲}{02600}
\saveTG{𠞌}{02600}
\saveTG{𠟸}{02600}
\saveTG{訓}{02600}
\saveTG{𧧋}{02600}
\saveTG{剖}{02600}
\saveTG{訆}{02600}
\saveTG{詶}{02600}
\saveTG{誗}{02600}
\saveTG{𧨰}{02600}
\saveTG{𧧸}{02600}
\saveTG{䚯}{02600}
\saveTG{訿}{02610}
\saveTG{𧭯}{02612}
\saveTG{𧦺}{02612}
\saveTG{誂}{02613}
\saveTG{𫍄}{02613}
\saveTG{䜝}{02613}
\saveTG{𧪴}{02614}
\saveTG{䛘}{02614}
\saveTG{䚾}{02614}
\saveTG{𧬡}{02614}
\saveTG{𧧺}{02614}
\saveTG{託}{02614}
\saveTG{諥}{02615}
\saveTG{諈}{02615}
\saveTG{䜅}{02615}
\saveTG{𧭞}{02616}
\saveTG{𧥳}{02617}
\saveTG{𧧗}{02617}
\saveTG{𧧲}{02617}
\saveTG{䚰}{02617}
\saveTG{𧦖}{02617}
\saveTG{䚽}{02617}
\saveTG{𧧊}{02617}
\saveTG{䛢}{02617}
\saveTG{䚹}{02617}
\saveTG{謕}{02617}
\saveTG{𣮒}{02617}
\saveTG{𧪚}{02618}
\saveTG{證}{02618}
\saveTG{𣃎}{02621}
\saveTG{𧪥}{02621}
\saveTG{訢}{02621}
\saveTG{𧬜}{02621}
\saveTG{𩐙}{02621}
\saveTG{𧫄}{02622}
\saveTG{𧨘}{02622}
\saveTG{䚲}{02622}
\saveTG{譑}{02627}
\saveTG{𧮄}{02627}
\saveTG{譌}{02627}
\saveTG{𧫶}{02627}
\saveTG{𩑁}{02627}
\saveTG{讗}{02627}
\saveTG{𧪙}{02627}
\saveTG{𧦻}{02627}
\saveTG{諯}{02627}
\saveTG{𧮋}{02627}
\saveTG{𧦓}{02627}
\saveTG{誘}{02627}
\saveTG{𧫲}{02627}
\saveTG{𧦋}{02630}
\saveTG{畆}{02630}
\saveTG{𧦐}{02630}
\saveTG{𤱔}{02631}
\saveTG{𧭙}{02632}
\saveTG{𤬥}{02632}
\saveTG{𤬃}{02633}
\saveTG{𧦼}{02633}
\saveTG{𧦟}{02637}
\saveTG{讔}{02637}
\saveTG{𧨩}{02638}
\saveTG{詆}{02640}
\saveTG{誕}{02641}
\saveTG{誔}{02641}
\saveTG{訴}{02641}
\saveTG{𫌽}{02644}
\saveTG{諉}{02644}
\saveTG{䛵}{02647}
\saveTG{諼}{02647}
\saveTG{詜}{02647}
\saveTG{𧬋}{02647}
\saveTG{䛀}{02647}
\saveTG{𧦄}{02647}
\saveTG{𧦝}{02649}
\saveTG{𣅲}{02650}
\saveTG{𧦌}{02650}
\saveTG{譏}{02653}
\saveTG{𧩒}{02657}
\saveTG{𧦎}{02660}
\saveTG{𧦶}{02661}
\saveTG{𧩢}{02661}
\saveTG{詬}{02661}
\saveTG{詣}{02661}
\saveTG{諧}{02662}
\saveTG{𧩣}{02662}
\saveTG{𧨥}{02663}
\saveTG{話}{02664}
\saveTG{䛡}{02664}
\saveTG{諙}{02664}
\saveTG{䛻}{02664}
\saveTG{譒}{02669}
\saveTG{誻}{02669}
\saveTG{𧪑}{02669}
\saveTG{訩}{02670}
\saveTG{訕}{02670}
\saveTG{𧬿}{02672}
\saveTG{謡}{02672}
\saveTG{𧦆}{02672}
\saveTG{詘}{02672}
\saveTG{韷}{02672}
\saveTG{𧪗}{02672}
\saveTG{䛽}{02677}
\saveTG{𧧆}{02677}
\saveTG{謟}{02677}
\saveTG{𧦈}{02677}
\saveTG{訞}{02684}
\saveTG{𧨞}{02684}
\saveTG{謑}{02684}
\saveTG{𧭎}{02685}
\saveTG{𩑀}{02685}
\saveTG{𧫇}{02686}
\saveTG{䜠}{02686}
\saveTG{𧪝}{02691}
\saveTG{𧧈}{02693}
\saveTG{𫌾}{02693}
\saveTG{𧪾}{02693}
\saveTG{𧭥}{02694}
\saveTG{訸}{02694}
\saveTG{𩑃}{02695}
\saveTG{𧬬}{02695}
\saveTG{𧭄}{02695}
\saveTG{䜈}{02695}
\saveTG{𩑄}{02695}
\saveTG{𧬠}{02699}
\saveTG{𠟼}{02700}
\saveTG{𠞬}{02700}
\saveTG{𠛑}{02700}
\saveTG{𠜆}{02700}
\saveTG{𣯡}{02717}
\saveTG{㲤}{02717}
\saveTG{𣰶}{02717}
\saveTG{𫍥}{02717}
\saveTG{𤣧}{02718}
\saveTG{瓤}{02730}
\saveTG{𪿁}{02765}
\saveTG{𠜇}{02800}
\saveTG{刻}{02800}
\saveTG{剠}{02900}
\saveTG{𠟇}{02900}
\saveTG{𣮭}{02917}
\saveTG{𣂺}{02921}
\saveTG{新}{02921}
\saveTG{𤗟}{02927}
\saveTG{𪺆}{02927}
\saveTG{𨐻}{02932}
\saveTG{𥩖}{03100}
\saveTG{𨭉}{03109}
\saveTG{𥩟}{03112}
\saveTG{竩}{03112}
\saveTG{竚}{03121}
\saveTG{鹫}{03127}
\saveTG{𥪀}{03127}
\saveTG{竤}{03132}
\saveTG{𥩦}{03140}
\saveTG{𫁢}{03141}
\saveTG{𥪣}{03142}
\saveTG{𥩽}{03143}
\saveTG{竣}{03147}
\saveTG{𢨀}{03150}
\saveTG{𢨅}{03150}
\saveTG{𥩡}{03150}
\saveTG{䇅}{03150}
\saveTG{𢨒}{03150}
\saveTG{𥩱}{03150}
\saveTG{竢}{03184}
\saveTG{𥫎}{03186}
\saveTG{𥪆}{03188}
\saveTG{𥪗}{03191}
\saveTG{𠅣}{03214}
\saveTG{𩀻}{03215}
\saveTG{𢅛}{03217}
\saveTG{𡯶}{03217}
\saveTG{𩫦}{03222}
\saveTG{𣄎}{03243}
\saveTG{𡕧}{03247}
\saveTG{𪊝}{03247}
\saveTG{𢧳}{03250}
\saveTG{𩫅}{03250}
\saveTG{旘}{03250}
\saveTG{𢧪}{03250}
\saveTG{𣄏}{03268}
\saveTG{𤡎}{03284}
\saveTG{𩫈}{03284}
\saveTG{旀}{03290}
\saveTG{鷲}{03327}
\saveTG{𤏅}{03331}
\saveTG{𪹱}{03338}
\saveTG{斒}{03427}
\saveTG{斏}{03432}
\saveTG{斌}{03440}
\saveTG{𩫯}{03443}
\saveTG{𢨊}{03451}
\saveTG{𩫨}{03452}
\saveTG{𨐭}{03489}
\saveTG{𨐑}{03517}
\saveTG{𧦨}{03600}
\saveTG{訃}{03600}
\saveTG{訫}{03600}
\saveTG{𤱈}{03600}
\saveTG{𧭻}{03601}
\saveTG{䛑}{03604}
\saveTG{訧}{03612}
\saveTG{誼}{03612}
\saveTG{𧧼}{03612}
\saveTG{詑}{03612}
\saveTG{謐}{03612}
\saveTG{𧫡}{03614}
\saveTG{詫}{03614}
\saveTG{諠}{03616}
\saveTG{䛷}{03617}
\saveTG{𧨎}{03617}
\saveTG{𧫾}{03617}
\saveTG{𧬯}{03617}
\saveTG{𧭈}{03621}
\saveTG{詝}{03621}
\saveTG{謲}{03622}
\saveTG{𧨜}{03627}
\saveTG{𧭠}{03627}
\saveTG{𧫐}{03627}
\saveTG{䛪}{03627}
\saveTG{誧}{03627}
\saveTG{諞}{03627}
\saveTG{謆}{03627}
\saveTG{誏}{03632}
\saveTG{䜗}{03635}
\saveTG{䜢}{03636}
\saveTG{𧩰}{03637}
\saveTG{試}{03640}
\saveTG{𧪹}{03641}
\saveTG{𫖘}{03641}
\saveTG{𤱶}{03641}
\saveTG{𧩇}{03647}
\saveTG{𧩮}{03647}
\saveTG{誜}{03647}
\saveTG{詙}{03647}
\saveTG{𧧕}{03647}
\saveTG{𧨠}{03647}
\saveTG{𧬵}{03648}
\saveTG{𧫒}{03648}
\saveTG{諴}{03650}
\saveTG{識}{03650}
\saveTG{戠}{03650}
\saveTG{𧬣}{03650}
\saveTG{𧨺}{03650}
\saveTG{𧥾}{03650}
\saveTG{䛋}{03650}
\saveTG{誡}{03650}
\saveTG{𢧂}{03650}
\saveTG{讖}{03650}
\saveTG{誠}{03650}
\saveTG{誐}{03650}
\saveTG{𧧐}{03651}
\saveTG{䜟}{03651}
\saveTG{𫍔}{03651}
\saveTG{諓}{03653}
\saveTG{𫍎}{03654}
\saveTG{𧧟}{03654}
\saveTG{𧪖}{03658}
\saveTG{𧭶}{03659}
\saveTG{𧫳}{03659}
\saveTG{詒}{03660}
\saveTG{𧮎}{03661}
\saveTG{𪡳}{03661}
\saveTG{𩐼}{03662}
\saveTG{䛮}{03664}
\saveTG{𧬙}{03666}
\saveTG{讅}{03669}
\saveTG{𧨨}{03672}
\saveTG{䛎}{03680}
\saveTG{𧮈}{03681}
\saveTG{諚}{03681}
\saveTG{𤟞}{03684}
\saveTG{𧩈}{03684}
\saveTG{誒}{03684}
\saveTG{讞}{03684}
\saveTG{𫍒}{03686}
\saveTG{𡫫}{03686}
\saveTG{𧮡}{03686}
\saveTG{𧭂}{03691}
\saveTG{誴}{03691}
\saveTG{詠}{03692}
\saveTG{訹}{03694}
\saveTG{𧧷}{03699}
\saveTG{蹵}{03801}
\saveTG{赟}{03802}
\saveTG{贇}{03806}
\saveTG{𢨞}{03850}
\saveTG{𥏛}{03850}
\saveTG{𤎷}{03850}
\saveTG{𧸠}{03865}
\saveTG{𤎼}{03897}
\saveTG{𧬀}{03903}
\saveTG{就}{03912}
\saveTG{𡰔}{03917}
\saveTG{𡰜}{03917}
\saveTG{𦁋}{03950}
\saveTG{竔}{04100}
\saveTG{竍}{04100}
\saveTG{𥃌}{04102}
\saveTG{䇆}{04103}
\saveTG{𣂂}{04103}
\saveTG{𨪣}{04109}
\saveTG{𥪯}{04117}
\saveTG{𥪘}{04118}
\saveTG{𥪫}{04123}
\saveTG{𠢻}{04127}
\saveTG{𥪹}{04127}
\saveTG{𦒭}{04127}
\saveTG{勯}{04127}
\saveTG{勭}{04127}
\saveTG{竑}{04132}
\saveTG{𧏛}{04133}
\saveTG{𥩳}{04143}
\saveTG{𥩾}{04147}
\saveTG{𪗓}{04147}
\saveTG{皽}{04147}
\saveTG{攱}{04147}
\saveTG{𫁨}{04157}
\saveTG{𥩪}{04160}
\saveTG{䇎}{04161}
\saveTG{𥪤}{04164}
\saveTG{𥩩}{04170}
\saveTG{𥪓}{04181}
\saveTG{𤼞}{04182}
\saveTG{𠅩}{04187}
\saveTG{𥪂}{04188}
\saveTG{𣂅}{04203}
\saveTG{𣂉}{04203}
\saveTG{𣂆}{04203}
\saveTG{𣁼}{04203}
\saveTG{𣂁}{04203}
\saveTG{𩫄}{04203}
\saveTG{𩫆}{04204}
\saveTG{𠡌}{04212}
\saveTG{𠡤}{04212}
\saveTG{𠡜}{04212}
\saveTG{𠢖}{04212}
\saveTG{𧦕}{04217}
\saveTG{𣃾}{04217}
\saveTG{𦓄}{04217}
\saveTG{旑}{04221}
\saveTG{𣄄}{04221}
\saveTG{𦞍}{04227}
\saveTG{𠢗}{04227}
\saveTG{𫜋}{04227}
\saveTG{劥}{04227}
\saveTG{𠢹}{04227}
\saveTG{𪊭}{04231}
\saveTG{𢻘}{04247}
\saveTG{𪊷}{04247}
\saveTG{𣃣}{04247}
\saveTG{㿶}{04247}
\saveTG{𥀱}{04247}
\saveTG{𡦵}{04247}
\saveTG{𥀆}{04247}
\saveTG{𡦱}{04247}
\saveTG{𥀔}{04247}
\saveTG{𩫋}{04247}
\saveTG{𫀥}{04261}
\saveTG{𧁹}{04264}
\saveTG{𪋑}{04264}
\saveTG{𡙴}{04280}
\saveTG{麒}{04281}
\saveTG{𪯻}{04282}
\saveTG{𪎰}{04286}
\saveTG{𪫼}{04330}
\saveTG{𪭂}{04332}
\saveTG{対}{04400}
\saveTG{効}{04427}
\saveTG{𠠼}{04427}
\saveTG{𨐎}{04440}
\saveTG{㿰}{04447}
\saveTG{𢻓}{04447}
\saveTG{𨐟}{04464}
\saveTG{𨐡}{04464}
\saveTG{𡦡}{04464}
\saveTG{𩫭}{04464}
\saveTG{𨐮}{04482}
\saveTG{𪏆}{04486}
\saveTG{𨐥}{04556}
\saveTG{謝}{04600}
\saveTG{討}{04600}
\saveTG{譵}{04600}
\saveTG{𫖗}{04600}
\saveTG{詂}{04600}
\saveTG{計}{04600}
\saveTG{䚵}{04603}
\saveTG{𧨽}{04603}
\saveTG{䛂}{04604}
\saveTG{訛}{04610}
\saveTG{𧩳}{04611}
\saveTG{𩐺}{04612}
\saveTG{𧪞}{04612}
\saveTG{訦}{04612}
\saveTG{訑}{04612}
\saveTG{読}{04612}
\saveTG{譊}{04612}
\saveTG{詵}{04612}
\saveTG{謊}{04612}
\saveTG{𧭏}{04614}
\saveTG{𧭩}{04614}
\saveTG{𧫠}{04614}
\saveTG{詿}{04614}
\saveTG{誮}{04614}
\saveTG{𧫴}{04615}
\saveTG{讙}{04615}
\saveTG{謹}{04615}
\saveTG{𧬇}{04616}
\saveTG{𧥩}{04617}
\saveTG{𧨙}{04617}
\saveTG{詍}{04617}
\saveTG{訅}{04617}
\saveTG{䛳}{04617}
\saveTG{𧦭}{04617}
\saveTG{𧥨}{04617}
\saveTG{𧪡}{04617}
\saveTG{䪧}{04617}
\saveTG{𧨔}{04617}
\saveTG{𧨀}{04618}
\saveTG{諶}{04618}
\saveTG{𧨌}{04618}
\saveTG{𧪆}{04621}
\saveTG{䛴}{04621}
\saveTG{𧦞}{04627}
\saveTG{𠣂}{04627}
\saveTG{𠡧}{04627}
\saveTG{𧪮}{04627}
\saveTG{𧮉}{04627}
\saveTG{䜏}{04627}
\saveTG{𧩭}{04627}
\saveTG{𧩊}{04627}
\saveTG{𧫚}{04627}
\saveTG{𧬺}{04627}
\saveTG{𧬝}{04627}
\saveTG{𧨄}{04627}
\saveTG{䛥}{04627}
\saveTG{𧩫}{04627}
\saveTG{譪}{04627}
\saveTG{勏}{04627}
\saveTG{誇}{04627}
\saveTG{諵}{04627}
\saveTG{訥}{04627}
\saveTG{詴}{04627}
\saveTG{誵}{04627}
\saveTG{詏}{04627}
\saveTG{𪟠}{04627}
\saveTG{䜕}{04627}
\saveTG{𧫩}{04627}
\saveTG{𧮧}{04627}
\saveTG{𧭪}{04628}
\saveTG{𧨃}{04631}
\saveTG{𧫉}{04631}
\saveTG{𧮚}{04631}
\saveTG{讌}{04631}
\saveTG{讛}{04631}
\saveTG{誌}{04631}
\saveTG{𧪉}{04632}
\saveTG{𧭊}{04632}
\saveTG{𧫁}{04632}
\saveTG{𧮇}{04632}
\saveTG{䜔}{04632}
\saveTG{詓}{04632}
\saveTG{𧪋}{04634}
\saveTG{𧬻}{04635}
\saveTG{訤}{04640}
\saveTG{𫌳}{04640}
\saveTG{𧬖}{04640}
\saveTG{𧨫}{04641}
\saveTG{𧬕}{04641}
\saveTG{䛭}{04641}
\saveTG{譸}{04641}
\saveTG{詩}{04641}
\saveTG{𧪇}{04642}
\saveTG{𧬏}{04644}
\saveTG{𧩃}{04644}
\saveTG{𢻕}{04647}
\saveTG{䚳}{04647}
\saveTG{𧪶}{04647}
\saveTG{𧮁}{04647}
\saveTG{誖}{04647}
\saveTG{詖}{04647}
\saveTG{護}{04647}
\saveTG{頀}{04647}
\saveTG{誟}{04647}
\saveTG{䪬}{04647}
\saveTG{𧪐}{04649}
\saveTG{𫖛}{04654}
\saveTG{譁}{04654}
\saveTG{諽}{04656}
\saveTG{諱}{04656}
\saveTG{𧧳}{04658}
\saveTG{諸}{04660}
\saveTG{詁}{04660}
\saveTG{詰}{04661}
\saveTG{譆}{04661}
\saveTG{𧮂}{04661}
\saveTG{𧫭}{04661}
\saveTG{𧭽}{04661}
\saveTG{誥}{04661}
\saveTG{諎}{04661}
\saveTG{譇}{04664}
\saveTG{諾}{04664}
\saveTG{詌}{04670}
\saveTG{𫍈}{04677}
\saveTG{𡘯}{04680}
\saveTG{𪥘}{04680}
\saveTG{𧮖}{04681}
\saveTG{諆}{04681}
\saveTG{𩐠}{04681}
\saveTG{謓}{04681}
\saveTG{𧬧}{04684}
\saveTG{𧨶}{04684}
\saveTG{𧫗}{04684}
\saveTG{𩐻}{04684}
\saveTG{謨}{04684}
\saveTG{韺}{04685}
\saveTG{讀}{04686}
\saveTG{讚}{04686}
\saveTG{䜁}{04687}
\saveTG{䛟}{04688}
\saveTG{𧧵}{04688}
\saveTG{詼}{04689}
\saveTG{𧮃}{04689}
\saveTG{諃}{04690}
\saveTG{䛙}{04690}
\saveTG{𧭝}{04691}
\saveTG{𩐱}{04694}
\saveTG{𩐷}{04694}
\saveTG{𧩜}{04694}
\saveTG{𧬗}{04694}
\saveTG{䜓}{04694}
\saveTG{諜}{04694}
\saveTG{謀}{04694}
\saveTG{𧬔}{04694}
\saveTG{䜍}{04696}
\saveTG{誺}{04698}
\saveTG{䜉}{04699}
\saveTG{𧩚}{04699}
\saveTG{𥪞}{04712}
\saveTG{𪥛}{04718}
\saveTG{䜦}{04727}
\saveTG{𠡆}{04727}
\saveTG{勷}{04727}
\saveTG{䜩}{04731}
\saveTG{𤣦}{04731}
\saveTG{𥀶}{04747}
\saveTG{䜤}{04781}
\saveTG{劾}{04827}
\saveTG{𢻉}{04847}
\saveTG{𡬱}{04903}
\saveTG{𦄠}{04903}
\saveTG{𡰗}{04917}
\saveTG{勍}{04927}
\saveTG{𠡽}{04927}
\saveTG{𥀏}{04947}
\saveTG{𫖎}{04956}
\saveTG{𣕟}{04990}
\saveTG{𧪸}{05047}
\saveTG{塾}{05104}
\saveTG{𠗩}{05106}
\saveTG{靖}{05127}
\saveTG{𥪁}{05127}
\saveTG{竱}{05143}
\saveTG{𥩺}{05182}
\saveTG{螤}{05186}
\saveTG{竦}{05196}
\saveTG{䇐}{05199}
\saveTG{𤜘}{05202}
\saveTG{𪊟}{05212}
\saveTG{𪽅}{05212}
\saveTG{𣄟}{05212}
\saveTG{𩫟}{05214}
\saveTG{𪚚}{05216}
\saveTG{𠅶}{05217}
\saveTG{𩪿}{05217}
\saveTG{𧓣}{05231}
\saveTG{𫇄}{05232}
\saveTG{旔}{05240}
\saveTG{𣄂}{05281}
\saveTG{𣃝}{05282}
\saveTG{𣄜}{05286}
\saveTG{𪰂}{05286}
\saveTG{𪎡}{05292}
\saveTG{𩫤}{05294}
\saveTG{熟}{05331}
\saveTG{㦁}{05333}
\saveTG{𨐶}{05416}
\saveTG{㝄}{05417}
\saveTG{孰}{05417}
\saveTG{𨐺}{05432}
\saveTG{𩫰}{05446}
\saveTG{𩫠}{05482}
\saveTG{𡥹}{05482}
\saveTG{𤒆}{05489}
\saveTG{𡦦}{05489}
\saveTG{𣖛}{05490}
\saveTG{辣}{05496}
\saveTG{𨐤}{05496}
\saveTG{𡦚}{05496}
\saveTG{𩍷}{05506}
\saveTG{𩍯}{05582}
\saveTG{𧥸}{05600}
\saveTG{𧥹}{05600}
\saveTG{䃞}{05601}
\saveTG{𫌼}{05606}
\saveTG{䛖}{05606}
\saveTG{訷}{05606}
\saveTG{訲}{05606}
\saveTG{𧨴}{05607}
\saveTG{𧧪}{05607}
\saveTG{𧪪}{05612}
\saveTG{𧥫}{05617}
\saveTG{訙}{05617}
\saveTG{訰}{05617}
\saveTG{𧬹}{05618}
\saveTG{𧭵}{05619}
\saveTG{䛣}{05627}
\saveTG{𩐚}{05627}
\saveTG{䛍}{05627}
\saveTG{請}{05627}
\saveTG{謰}{05630}
\saveTG{𧦹}{05630}
\saveTG{譨}{05632}
\saveTG{諘}{05632}
\saveTG{譓}{05633}
\saveTG{䛱}{05636}
\saveTG{譴}{05637}
\saveTG{譿}{05637}
\saveTG{𧪯}{05637}
\saveTG{讉}{05638}
\saveTG{𧫻}{05639}
\saveTG{謱}{05644}
\saveTG{𧧅}{05645}
\saveTG{䛕}{05646}
\saveTG{講}{05647}
\saveTG{𧦾}{05647}
\saveTG{𧦦}{05647}
\saveTG{𩐸}{05657}
\saveTG{𧪿}{05657}
\saveTG{䛆}{05660}
\saveTG{𧧥}{05665}
\saveTG{䜊}{05666}
\saveTG{譛}{05668}
\saveTG{𩐽}{05677}
\saveTG{詇}{05680}
\saveTG{𧥱}{05680}
\saveTG{訣}{05680}
\saveTG{詄}{05680}
\saveTG{誱}{05681}
\saveTG{𧨸}{05681}
\saveTG{𧩻}{05684}
\saveTG{謮}{05686}
\saveTG{讃}{05686}
\saveTG{䜋}{05686}
\saveTG{誅}{05690}
\saveTG{誄}{05690}
\saveTG{𧧒}{05692}
\saveTG{䛾}{05693}
\saveTG{䛶}{05694}
\saveTG{謋}{05694}
\saveTG{𩐫}{05696}
\saveTG{誎}{05696}
\saveTG{諫}{05696}
\saveTG{諌}{05696}
\saveTG{讲}{05705}
\saveTG{𤄯}{05712}
\saveTG{𥪝}{05716}
\saveTG{䜨}{05736}
\saveTG{𪞸}{05740}
\saveTG{𠅎}{05782}
\saveTG{𠅒}{05790}
\saveTG{𨄡}{05802}
\saveTG{𡙰}{05804}
\saveTG{𤍨}{05809}
\saveTG{𨽷}{05899}
\saveTG{𫖜}{06022}
\saveTG{𩑂}{06027}
\saveTG{嗶}{06054}
\saveTG{𤽞}{06102}
\saveTG{䚒}{06117}
\saveTG{竭}{06127}
\saveTG{𥪔}{06127}
\saveTG{𫁩}{06127}
\saveTG{𥪎}{06142}
\saveTG{䇑}{06145}
\saveTG{𥪳}{06147}
\saveTG{䇍}{06182}
\saveTG{𥪩}{06186}
\saveTG{𠵩}{06195}
\saveTG{覫}{06212}
\saveTG{𩲋}{06217}
\saveTG{𩲌}{06217}
\saveTG{𧠰}{06217}
\saveTG{𩳁}{06217}
\saveTG{𩫬}{06221}
\saveTG{𪎥}{06227}
\saveTG{𡃯}{06227}
\saveTG{𣃫}{06227}
\saveTG{𩫘}{06227}
\saveTG{𫘷}{06240}
\saveTG{髜}{06240}
\saveTG{𩫜}{06244}
\saveTG{𩫮}{06245}
\saveTG{𩫝}{06245}
\saveTG{麣}{06248}
\saveTG{𡅄}{06256}
\saveTG{𣎘}{06260}
\saveTG{𣄗}{06281}
\saveTG{𩫣}{06286}
\saveTG{𪯾}{06286}
\saveTG{髞}{06294}
\saveTG{𨙎}{06309}
\saveTG{𢥿}{06338}
\saveTG{𢥫}{06338}
\saveTG{𪬳}{06386}
\saveTG{𥫊}{06412}
\saveTG{𩫩}{06417}
\saveTG{𣁐}{06417}
\saveTG{𧠷}{06417}
\saveTG{𥫑}{06430}
\saveTG{𢥔}{06430}
\saveTG{𧡗}{06441}
\saveTG{𥫖}{06444}
\saveTG{𩫫}{06445}
\saveTG{𩫪}{06445}
\saveTG{𡦟}{06445}
\saveTG{𨐜}{06445}
\saveTG{𥫕}{06447}
\saveTG{嚲}{06456}
\saveTG{䯬}{06456}
\saveTG{𧬁}{06499}
\saveTG{𧥵}{06600}
\saveTG{𧥣}{06600}
\saveTG{䛛}{06600}
\saveTG{𧦤}{06600}
\saveTG{訵}{06600}
\saveTG{䛜}{06602}
\saveTG{詯}{06602}
\saveTG{詚}{06610}
\saveTG{𧨤}{06612}
\saveTG{𧪍}{06612}
\saveTG{誢}{06612}
\saveTG{詋}{06612}
\saveTG{謉}{06613}
\saveTG{𧨚}{06614}
\saveTG{𧫍}{06614}
\saveTG{𧬼}{06614}
\saveTG{諻}{06614}
\saveTG{韹}{06614}
\saveTG{謃}{06615}
\saveTG{郶}{06617}
\saveTG{𧪫}{06617}
\saveTG{𧭼}{06617}
\saveTG{𢈗}{06617}
\saveTG{䛰}{06617}
\saveTG{𧨛}{06617}
\saveTG{𧡱}{06617}
\saveTG{𧬌}{06621}
\saveTG{𧨬}{06621}
\saveTG{𧪒}{06623}
\saveTG{𧩑}{06627}
\saveTG{𧭮}{06627}
\saveTG{䛲}{06627}
\saveTG{𧬒}{06627}
\saveTG{𩐪}{06627}
\saveTG{𧪓}{06627}
\saveTG{𧩎}{06627}
\saveTG{𧬴}{06627}
\saveTG{𧭇}{06627}
\saveTG{諹}{06627}
\saveTG{謁}{06627}
\saveTG{謂}{06627}
\saveTG{諤}{06627}
\saveTG{𧪛}{06627}
\saveTG{䛺}{06628}
\saveTG{𧭬}{06628}
\saveTG{𧪩}{06630}
\saveTG{𧪳}{06630}
\saveTG{謥}{06630}
\saveTG{諰}{06630}
\saveTG{𧪎}{06631}
\saveTG{𧭴}{06632}
\saveTG{譞}{06632}
\saveTG{𧪏}{06632}
\saveTG{䜙}{06633}
\saveTG{𧭑}{06633}
\saveTG{𧮅}{06633}
\saveTG{䜚}{06639}
\saveTG{𧨓}{06640}
\saveTG{諀}{06640}
\saveTG{𧮞}{06640}
\saveTG{諿}{06641}
\saveTG{譯}{06641}
\saveTG{䛞}{06641}
\saveTG{𧩙}{06641}
\saveTG{𧬳}{06641}
\saveTG{𧮆}{06644}
\saveTG{𧫤}{06645}
\saveTG{𩑇}{06647}
\saveTG{𧮜}{06647}
\saveTG{謖}{06647}
\saveTG{謾}{06647}
\saveTG{𧫝}{06647}
\saveTG{䜂}{06648}
\saveTG{讝}{06648}
\saveTG{䛅}{06650}
\saveTG{譂}{06656}
\saveTG{誯}{06660}
\saveTG{讄}{06660}
\saveTG{𧦢}{06671}
\saveTG{𧧾}{06680}
\saveTG{䛊}{06680}
\saveTG{諟}{06681}
\saveTG{䛤}{06682}
\saveTG{𧬮}{06684}
\saveTG{誤}{06684}
\saveTG{𧩍}{06685}
\saveTG{韻}{06686}
\saveTG{𧬂}{06686}
\saveTG{𧪼}{06686}
\saveTG{𥪾}{06686}
\saveTG{𧫖}{06693}
\saveTG{𧮢}{06693}
\saveTG{譟}{06694}
\saveTG{課}{06694}
\saveTG{𧮔}{06694}
\saveTG{𩐿}{06696}
\saveTG{𧭤}{06699}
\saveTG{𧪟}{06699}
\saveTG{𥉩}{06710}
\saveTG{𪽞}{06710}
\saveTG{𩲉}{06717}
\saveTG{𤣨}{06727}
\saveTG{𪞻}{06784}
\saveTG{𠅳}{06786}
\saveTG{親}{06912}
\saveTG{𧡿}{06917}
\saveTG{𧗛}{07102}
\saveTG{𥂣}{07102}
\saveTG{䀍}{07102}
\saveTG{望}{07104}
\saveTG{𡏅}{07104}
\saveTG{𥩿}{07108}
\saveTG{飒}{07110}
\saveTG{颯}{07110}
\saveTG{竌}{07110}
\saveTG{𪛀}{07112}
\saveTG{𥩢}{07112}
\saveTG{𥩹}{07112}
\saveTG{䇈}{07114}
\saveTG{𪓼}{07117}
\saveTG{𪚹}{07117}
\saveTG{𥩥}{07117}
\saveTG{䇃}{07117}
\saveTG{𥩙}{07117}
\saveTG{翊}{07120}
\saveTG{竘}{07120}
\saveTG{𦒜}{07121}
\saveTG{𫁞}{07121}
\saveTG{𥩘}{07121}
\saveTG{𦒍}{07121}
\saveTG{𥩞}{07123}
\saveTG{𩿢}{07127}
\saveTG{𥪃}{07127}
\saveTG{𥪥}{07127}
\saveTG{𥪒}{07127}
\saveTG{鴗}{07127}
\saveTG{鸇}{07127}
\saveTG{鹯}{07127}
\saveTG{𨚪}{07127}
\saveTG{𫁡}{07127}
\saveTG{𫁣}{07127}
\saveTG{𨝯}{07127}
\saveTG{䇋}{07127}
\saveTG{𨟐}{07127}
\saveTG{䴀}{07127}
\saveTG{竧}{07140}
\saveTG{竮}{07141}
\saveTG{𠗏}{07141}
\saveTG{𩫙}{07141}
\saveTG{𫁠}{07144}
\saveTG{殶}{07147}
\saveTG{竐}{07147}
\saveTG{䇇}{07147}
\saveTG{𣪪}{07147}
\saveTG{𥩯}{07147}
\saveTG{𥪏}{07147}
\saveTG{竫}{07157}
\saveTG{𫁦}{07164}
\saveTG{𥪊}{07172}
\saveTG{𧗜}{07172}
\saveTG{𡎿}{07174}
\saveTG{𥪅}{07182}
\saveTG{𫜌}{07182}
\saveTG{𣢦}{07182}
\saveTG{斔}{07187}
\saveTG{𥪲}{07193}
\saveTG{𥪬}{07199}
\saveTG{𥪋}{07199}
\saveTG{𪋴}{07210}
\saveTG{𪋖}{07210}
\saveTG{𩙒}{07210}
\saveTG{𦒮}{07212}
\saveTG{𪚤}{07214}
\saveTG{𢁆}{07217}
\saveTG{髛}{07217}
\saveTG{㫉}{07217}
\saveTG{𪊡}{07217}
\saveTG{𪊋}{07217}
\saveTG{𦫨}{07217}
\saveTG{䒍}{07217}
\saveTG{𤮇}{07217}
\saveTG{𪊖}{07221}
\saveTG{𫆡}{07221}
\saveTG{𣃗}{07221}
\saveTG{𦐄}{07221}
\saveTG{𪥀}{07223}
\saveTG{𥪈}{07223}
\saveTG{𢊙}{07226}
\saveTG{𪈠}{07227}
\saveTG{𪈾}{07227}
\saveTG{𪇛}{07227}
\saveTG{𪇉}{07227}
\saveTG{𩩇}{07227}
\saveTG{𪟑}{07227}
\saveTG{鄽}{07227}
\saveTG{邡}{07227}
\saveTG{鴋}{07227}
\saveTG{鄜}{07227}
\saveTG{鹒}{07227}
\saveTG{邟}{07227}
\saveTG{鄗}{07227}
\saveTG{鶮}{07227}
\saveTG{邝}{07227}
\saveTG{鄺}{07227}
\saveTG{鶶}{07227}
\saveTG{鵺}{07227}
\saveTG{鄘}{07227}
\saveTG{鷛}{07227}
\saveTG{鹧}{07227}
\saveTG{鷓}{07227}
\saveTG{鶙}{07227}
\saveTG{𫑰}{07227}
\saveTG{𨜷}{07227}
\saveTG{𫜆}{07227}
\saveTG{𨝏}{07227}
\saveTG{𪉉}{07227}
\saveTG{𠂫}{07227}
\saveTG{𨝧}{07227}
\saveTG{𨞻}{07227}
\saveTG{𨚃}{07227}
\saveTG{䣌}{07227}
\saveTG{𨟖}{07227}
\saveTG{𨝎}{07227}
\saveTG{𣃶}{07227}
\saveTG{𫘵}{07227}
\saveTG{𪈗}{07227}
\saveTG{𪅑}{07227}
\saveTG{𪄱}{07227}
\saveTG{鄌}{07227}
\saveTG{𪄲}{07227}
\saveTG{𪃛}{07227}
\saveTG{𪆿}{07227}
\saveTG{𪃒}{07227}
\saveTG{𪇵}{07227}
\saveTG{𪅆}{07227}
\saveTG{𪁇}{07227}
\saveTG{𪁈}{07227}
\saveTG{𪇮}{07227}
\saveTG{䲳}{07227}
\saveTG{𪋞}{07227}
\saveTG{𪀞}{07227}
\saveTG{𪁂}{07227}
\saveTG{𪂑}{07227}
\saveTG{𪁱}{07227}
\saveTG{鶊}{07227}
\saveTG{𢒶}{07232}
\saveTG{𥫄}{07232}
\saveTG{𦗤}{07242}
\saveTG{𣪢}{07247}
\saveTG{毃}{07247}
\saveTG{毅}{07247}
\saveTG{𠮈}{07247}
\saveTG{𪵏}{07247}
\saveTG{𣫖}{07247}
\saveTG{𣫚}{07247}
\saveTG{𡦔}{07247}
\saveTG{𣫛}{07247}
\saveTG{𣄤}{07263}
\saveTG{𫜍}{07264}
\saveTG{𧬥}{07270}
\saveTG{𣄣}{07281}
\saveTG{𣣎}{07282}
\saveTG{歒}{07282}
\saveTG{𣤤}{07282}
\saveTG{㱂}{07282}
\saveTG{𪴩}{07282}
\saveTG{歊}{07282}
\saveTG{㰠}{07282}
\saveTG{𠨛}{07320}
\saveTG{𪬫}{07321}
\saveTG{𨚳}{07327}
\saveTG{𪅪}{07327}
\saveTG{鷾}{07327}
\saveTG{𪬆}{07334}
\saveTG{戆}{07338}
\saveTG{戇}{07338}
\saveTG{嬜}{07404}
\saveTG{𣫡}{07407}
\saveTG{𩐾}{07407}
\saveTG{𩖣}{07410}
\saveTG{𨐦}{07410}
\saveTG{𣁈}{07412}
\saveTG{𡔊}{07414}
\saveTG{𠙇}{07417}
\saveTG{𠆔}{07417}
\saveTG{斓}{07420}
\saveTG{𣁚}{07420}
\saveTG{斕}{07420}
\saveTG{𦐑}{07421}
\saveTG{𠞻}{07422}
\saveTG{𨐗}{07426}
\saveTG{𪅂}{07427}
\saveTG{𪉃}{07427}
\saveTG{𨟍}{07427}
\saveTG{𨞥}{07427}
\saveTG{𨟞}{07427}
\saveTG{𫜂}{07427}
\saveTG{䴔}{07427}
\saveTG{}{07427}
\saveTG{鹑}{07427}
\saveTG{𣁘}{07427}
\saveTG{𪅄}{07427}
\saveTG{𪁽}{07427}
\saveTG{𪈃}{07427}
\saveTG{𪈋}{07427}
\saveTG{𪆾}{07427}
\saveTG{鶉}{07427}
\saveTG{郭}{07427}
\saveTG{郊}{07427}
\saveTG{鵁}{07427}
\saveTG{鳼}{07427}
\saveTG{鄣}{07427}
\saveTG{𪈷}{07427}
\saveTG{𪆝}{07427}
\saveTG{𨐪}{07427}
\saveTG{𦑋}{07431}
\saveTG{𡣼}{07441}
\saveTG{𠮒}{07447}
\saveTG{𥫓}{07447}
\saveTG{竷}{07447}
\saveTG{𥪴}{07457}
\saveTG{𫁯}{07477}
\saveTG{𪟼}{07481}
\saveTG{㰵}{07482}
\saveTG{赣}{07482}
\saveTG{㪱}{07484}
\saveTG{𥫔}{07486}
\saveTG{贛}{07486}
\saveTG{𦏧}{07510}
\saveTG{𫛅}{07527}
\saveTG{𧫈}{07601}
\saveTG{唘}{07604}
\saveTG{䤗}{07604}
\saveTG{𧥤}{07607}
\saveTG{𫌹}{07610}
\saveTG{訊}{07610}
\saveTG{諷}{07610}
\saveTG{訉}{07610}
\saveTG{𧬽}{07610}
\saveTG{𧩠}{07610}
\saveTG{𧪀}{07611}
\saveTG{譅}{07611}
\saveTG{䜛}{07612}
\saveTG{𧧫}{07612}
\saveTG{𧧓}{07612}
\saveTG{䚼}{07612}
\saveTG{𥃚}{07612}
\saveTG{䜀}{07612}
\saveTG{詛}{07612}
\saveTG{誽}{07612}
\saveTG{詭}{07612}
\saveTG{䛼}{07612}
\saveTG{𧩌}{07612}
\saveTG{𧪧}{07612}
\saveTG{𧨏}{07612}
\saveTG{𧫺}{07612}
\saveTG{𧦒}{07612}
\saveTG{𧭸}{07612}
\saveTG{讒}{07613}
\saveTG{𧫢}{07614}
\saveTG{𧩖}{07614}
\saveTG{記}{07617}
\saveTG{𧧘}{07617}
\saveTG{䛌}{07617}
\saveTG{譝}{07617}
\saveTG{𩐜}{07617}
\saveTG{𧨕}{07617}
\saveTG{䛄}{07617}
\saveTG{䛏}{07617}
\saveTG{𧥬}{07617}
\saveTG{詢}{07620}
\saveTG{韵}{07620}
\saveTG{譋}{07620}
\saveTG{𧥟}{07620}
\saveTG{諊}{07620}
\saveTG{訽}{07620}
\saveTG{讕}{07620}
\saveTG{詷}{07620}
\saveTG{調}{07620}
\saveTG{訋}{07620}
\saveTG{詞}{07620}
\saveTG{謿}{07620}
\saveTG{讇}{07620}
\saveTG{𧥴}{07620}
\saveTG{訒}{07620}
\saveTG{誷}{07620}
\saveTG{詗}{07620}
\saveTG{詾}{07620}
\saveTG{詡}{07620}
\saveTG{𧦷}{07621}
\saveTG{𥪇}{07621}
\saveTG{𧦃}{07621}
\saveTG{𧨋}{07621}
\saveTG{𪡍}{07621}
\saveTG{𧪜}{07621}
\saveTG{䚴}{07621}
\saveTG{𧩝}{07621}
\saveTG{𧥺}{07621}
\saveTG{𦟁}{07621}
\saveTG{𡂯}{07621}
\saveTG{𧮍}{07621}
\saveTG{𧦍}{07622}
\saveTG{𧮑}{07622}
\saveTG{𧮕}{07622}
\saveTG{𫅬}{07622}
\saveTG{𧩘}{07622}
\saveTG{𧬶}{07622}
\saveTG{𧬱}{07622}
\saveTG{謬}{07622}
\saveTG{䪨}{07623}
\saveTG{𫍍}{07623}
\saveTG{𧧜}{07624}
\saveTG{𧦘}{07624}
\saveTG{𧦛}{07624}
\saveTG{𩐤}{07626}
\saveTG{𩐝}{07626}
\saveTG{𠹪}{07626}
\saveTG{𧧦}{07626}
\saveTG{𧨝}{07626}
\saveTG{𧬘}{07626}
\saveTG{𧥰}{07627}
\saveTG{鶕}{07627}
\saveTG{𧨖}{07627}
\saveTG{𧬚}{07627}
\saveTG{𧪦}{07627}
\saveTG{䚮}{07627}
\saveTG{部}{07627}
\saveTG{謅}{07627}
\saveTG{諝}{07627}
\saveTG{䪮}{07627}
\saveTG{𫍘}{07627}
\saveTG{謻}{07627}
\saveTG{鄐}{07627}
\saveTG{諣}{07627}
\saveTG{譎}{07627}
\saveTG{誦}{07627}
\saveTG{䳝}{07627}
\saveTG{誃}{07627}
\saveTG{𧮏}{07627}
\saveTG{䛬}{07627}
\saveTG{𩐯}{07627}
\saveTG{𧪱}{07627}
\saveTG{𧩛}{07627}
\saveTG{𧫞}{07627}
\saveTG{𪆡}{07627}
\saveTG{𧩀}{07627}
\saveTG{𧪣}{07631}
\saveTG{𧩷}{07631}
\saveTG{誋}{07631}
\saveTG{𩐹}{07632}
\saveTG{𧩪}{07632}
\saveTG{𧬛}{07632}
\saveTG{𫍌}{07632}
\saveTG{𧩺}{07632}
\saveTG{詪}{07632}
\saveTG{認}{07632}
\saveTG{䛹}{07632}
\saveTG{𧩓}{07632}
\saveTG{𧪅}{07634}
\saveTG{𧭋}{07634}
\saveTG{韼}{07635}
\saveTG{𧭟}{07635}
\saveTG{𧬪}{07636}
\saveTG{𧪲}{07637}
\saveTG{𧫎}{07638}
\saveTG{諏}{07640}
\saveTG{諔}{07640}
\saveTG{詉}{07640}
\saveTG{𧫮}{07641}
\saveTG{𩐦}{07641}
\saveTG{謘}{07641}
\saveTG{𧫿}{07643}
\saveTG{訍}{07643}
\saveTG{𧩱}{07644}
\saveTG{𧧚}{07644}
\saveTG{𧧔}{07644}
\saveTG{𧪕}{07647}
\saveTG{𧦮}{07647}
\saveTG{譭}{07647}
\saveTG{讂}{07647}
\saveTG{誛}{07647}
\saveTG{訯}{07647}
\saveTG{設}{07647}
\saveTG{謏}{07647}
\saveTG{諁}{07647}
\saveTG{𧫂}{07647}
\saveTG{𠭇}{07647}
\saveTG{𧪂}{07647}
\saveTG{𧭦}{07647}
\saveTG{𧫫}{07647}
\saveTG{𧦀}{07647}
\saveTG{䛉}{07647}
\saveTG{𧨮}{07647}
\saveTG{𧧛}{07650}
\saveTG{諢}{07652}
\saveTG{韸}{07654}
\saveTG{𧧽}{07654}
\saveTG{𧧭}{07654}
\saveTG{䛁}{07655}
\saveTG{𧮘}{07656}
\saveTG{諍}{07657}
\saveTG{𧬦}{07657}
\saveTG{𩐨}{07657}
\saveTG{䜄}{07659}
\saveTG{䛇}{07661}
\saveTG{譫}{07661}
\saveTG{𩐶}{07661}
\saveTG{詺}{07662}
\saveTG{䚿}{07662}
\saveTG{謵}{07662}
\saveTG{𧪭}{07662}
\saveTG{𩐲}{07662}
\saveTG{韶}{07662}
\saveTG{詔}{07662}
\saveTG{𧭷}{07663}
\saveTG{詻}{07664}
\saveTG{䛯}{07664}
\saveTG{𧮒}{07664}
\saveTG{𧬅}{07664}
\saveTG{𩐩}{07667}
\saveTG{𧨡}{07667}
\saveTG{諮}{07668}
\saveTG{𧧖}{07670}
\saveTG{𧨵}{07672}
\saveTG{𧬲}{07672}
\saveTG{𧧱}{07672}
\saveTG{誳}{07672}
\saveTG{謠}{07672}
\saveTG{䛝}{07672}
\saveTG{𧦫}{07677}
\saveTG{諂}{07677}
\saveTG{𧧠}{07677}
\saveTG{畝}{07680}
\saveTG{𩐗}{07681}
\saveTG{譔}{07681}
\saveTG{譺}{07681}
\saveTG{𧩿}{07681}
\saveTG{𫍖}{07681}
\saveTG{歆}{07682}
\saveTG{𧭍}{07682}
\saveTG{𣢿}{07682}
\saveTG{𣣱}{07682}
\saveTG{㰴}{07682}
\saveTG{𧩨}{07684}
\saveTG{䜒}{07684}
\saveTG{𧩩}{07684}
\saveTG{𧩶}{07684}
\saveTG{𧭾}{07686}
\saveTG{謴}{07686}
\saveTG{𩑅}{07686}
\saveTG{諛}{07687}
\saveTG{訳}{07687}
\saveTG{𩐬}{07689}
\saveTG{𧩴}{07691}
\saveTG{𧫕}{07691}
\saveTG{𧦽}{07692}
\saveTG{𧨇}{07694}
\saveTG{𧨾}{07694}
\saveTG{𧧨}{07694}
\saveTG{𧧇}{07694}
\saveTG{𧨹}{07699}
\saveTG{䜧}{07712}
\saveTG{𠨑}{07720}
\saveTG{𨟚}{07727}
\saveTG{䲻}{07727}
\saveTG{邙}{07727}
\saveTG{𪜟}{07731}
\saveTG{𦫋}{07732}
\saveTG{氓}{07747}
\saveTG{𫍮}{07772}
\saveTG{𣤸}{07782}
\saveTG{𣤽}{07782}
\saveTG{𣢅}{07782}
\saveTG{𠛳}{07821}
\saveTG{𠜨}{07824}
\saveTG{郂}{07827}
\saveTG{㱾}{07847}
\saveTG{𦐤}{07871}
\saveTG{欬}{07882}
\saveTG{𩘁}{07910}
\saveTG{鶁}{07927}
\saveTG{𪆩}{07927}
\saveTG{𨛌}{07927}
\saveTG{𡗋}{07927}
\saveTG{𨜎}{07927}
\saveTG{𨛧}{07927}
\saveTG{𡃝}{08027}
\saveTG{𣦤}{08102}
\saveTG{𥂚}{08102}
\saveTG{𥂦}{08102}
\saveTG{𡒱}{08104}
\saveTG{𡌼}{08104}
\saveTG{墪}{08104}
\saveTG{𡒋}{08104}
\saveTG{𥪱}{08108}
\saveTG{𨰩}{08109}
\saveTG{鐜}{08109}
\saveTG{𠖯}{08117}
\saveTG{䇄}{08117}
\saveTG{𣃩}{08117}
\saveTG{𫁬}{08127}
\saveTG{㫇}{08127}
\saveTG{竕}{08127}
\saveTG{}{08127}
\saveTG{𥩷}{08132}
\saveTG{𩚷}{08132}
\saveTG{竛}{08132}
\saveTG{𧏅}{08136}
\saveTG{䗐}{08136}
\saveTG{𧌱}{08136}
\saveTG{䗠}{08136}
\saveTG{𧐈}{08136}
\saveTG{𥪸}{08143}
\saveTG{𥩵}{08144}
\saveTG{竴}{08146}
\saveTG{𥪚}{08147}
\saveTG{𥫃}{08153}
\saveTG{𥩻}{08161}
\saveTG{𥫁}{08166}
\saveTG{竲}{08166}
\saveTG{𥪶}{08168}
\saveTG{𥪉}{08168}
\saveTG{㫃}{08200}
\saveTG{𪯕}{08211}
\saveTG{施}{08212}
\saveTG{旎}{08212}
\saveTG{旒}{08212}
\saveTG{㫌}{08212}
\saveTG{旐}{08213}
\saveTG{𩙴}{08213}
\saveTG{𣄀}{08214}
\saveTG{旄}{08214}
\saveTG{𣄒}{08214}
\saveTG{𣃯}{08214}
\saveTG{旌}{08215}
\saveTG{𩀥}{08215}
\saveTG{𣄢}{08215}
\saveTG{旜}{08216}
\saveTG{𣄓}{08217}
\saveTG{斻}{08217}
\saveTG{旊}{08217}
\saveTG{𣄖}{08217}
\saveTG{𣃿}{08217}
\saveTG{𣃢}{08217}
\saveTG{𣄙}{08217}
\saveTG{𠙟}{08217}
\saveTG{𣃲}{08217}
\saveTG{𣄑}{08217}
\saveTG{斺}{08220}
\saveTG{𧱘}{08220}
\saveTG{𣃘}{08220}
\saveTG{旖}{08221}
\saveTG{𣄨}{08221}
\saveTG{㫊}{08221}
\saveTG{㫅}{08221}
\saveTG{旂}{08221}
\saveTG{膐}{08227}
\saveTG{𣃰}{08227}
\saveTG{膂}{08227}
\saveTG{𢄧}{08227}
\saveTG{𩩘}{08227}
\saveTG{𣄘}{08227}
\saveTG{𣄊}{08227}
\saveTG{𣄇}{08227}
\saveTG{𣃦}{08227}
\saveTG{𢄲}{08227}
\saveTG{𣃽}{08227}
\saveTG{旓}{08227}
\saveTG{斾}{08227}
\saveTG{旆}{08227}
\saveTG{𣃭}{08228}
\saveTG{𣄦}{08231}
\saveTG{𣄆}{08232}
\saveTG{𣃠}{08232}
\saveTG{旍}{08232}
\saveTG{旅}{08232}
\saveTG{旞}{08233}
\saveTG{於}{08233}
\saveTG{𣄚}{08233}
\saveTG{𣄧}{08238}
\saveTG{𪯔}{08240}
\saveTG{𪯐}{08240}
\saveTG{放}{08240}
\saveTG{䵇}{08240}
\saveTG{𣀊}{08240}
\saveTG{敵}{08240}
\saveTG{㪣}{08240}
\saveTG{𢾛}{08240}
\saveTG{𢼜}{08240}
\saveTG{𢾅}{08240}
\saveTG{𪰅}{08242}
\saveTG{𩫑}{08244}
\saveTG{𣃬}{08244}
\saveTG{㫏}{08244}
\saveTG{𣃼}{08244}
\saveTG{旇}{08247}
\saveTG{斿}{08247}
\saveTG{旃}{08247}
\saveTG{𪯽}{08247}
\saveTG{𪯳}{08247}
\saveTG{㫋}{08247}
\saveTG{𣃮}{08247}
\saveTG{𣃹}{08247}
\saveTG{㫍}{08248}
\saveTG{𣄞}{08253}
\saveTG{㫎}{08256}
\saveTG{𣃵}{08256}
\saveTG{𣃻}{08256}
\saveTG{𣃧}{08260}
\saveTG{𣃳}{08260}
\saveTG{𣃴}{08260}
\saveTG{𣄕}{08261}
\saveTG{𪯸}{08262}
\saveTG{𣄉}{08264}
\saveTG{旝}{08266}
\saveTG{𣃺}{08268}
\saveTG{旛}{08269}
\saveTG{𣄐}{08277}
\saveTG{旟}{08281}
\saveTG{旋}{08281}
\saveTG{旗}{08281}
\saveTG{𣄠}{08284}
\saveTG{族}{08284}
\saveTG{𪯶}{08284}
\saveTG{𣄝}{08286}
\saveTG{𣃨}{08288}
\saveTG{𣄪}{08289}
\saveTG{𣄡}{08289}
\saveTG{𣃥}{08290}
\saveTG{㫆}{08290}
\saveTG{旚}{08291}
\saveTG{𪊸}{08294}
\saveTG{𨔆}{08303}
\saveTG{䢟}{08304}
\saveTG{𪆃}{08327}
\saveTG{鷟}{08327}
\saveTG{𤍋}{08331}
\saveTG{憝}{08334}
\saveTG{𢥲}{08334}
\saveTG{𩺯}{08336}
\saveTG{𢽯}{08340}
\saveTG{𡭅}{08342}
\saveTG{𪍧}{08407}
\saveTG{𡥮}{08407}
\saveTG{𨐷}{08411}
\saveTG{𣁥}{08412}
\saveTG{𪯧}{08419}
\saveTG{𨐩}{08437}
\saveTG{效}{08440}
\saveTG{敦}{08440}
\saveTG{𡟕}{08441}
\saveTG{𡝶}{08448}
\saveTG{𦍢}{08451}
\saveTG{亸}{08456}
\saveTG{𦏎}{08461}
\saveTG{𪯤}{08461}
\saveTG{𧹄}{08486}
\saveTG{𧹉}{08486}
\saveTG{𢳈}{08502}
\saveTG{撉}{08502}
\saveTG{𢥦}{08534}
\saveTG{𠓬}{08600}
\saveTG{畒}{08600}
\saveTG{䃦}{08601}
\saveTG{㫈}{08603}
\saveTG{䁈}{08604}
\saveTG{𧧻}{08611}
\saveTG{詐}{08611}
\saveTG{𧪘}{08612}
\saveTG{𧭗}{08612}
\saveTG{說}{08612}
\saveTG{説}{08612}
\saveTG{諩}{08612}
\saveTG{𧮤}{08612}
\saveTG{𧦧}{08612}
\saveTG{𧨦}{08612}
\saveTG{旕}{08612}
\saveTG{詤}{08612}
\saveTG{諡}{08612}
\saveTG{謚}{08612}
\saveTG{詮}{08614}
\saveTG{𪵦}{08617}
\saveTG{𧪢}{08617}
\saveTG{訖}{08617}
\saveTG{𧩹}{08617}
\saveTG{𧭚}{08617}
\saveTG{𧥷}{08617}
\saveTG{𧭀}{08620}
\saveTG{𧪺}{08620}
\saveTG{䚸}{08620}
\saveTG{諭}{08621}
\saveTG{𧪈}{08621}
\saveTG{診}{08622}
\saveTG{論}{08627}
\saveTG{讑}{08627}
\saveTG{䪩}{08627}
\saveTG{𧬈}{08627}
\saveTG{䚷}{08627}
\saveTG{訡}{08627}
\saveTG{譾}{08627}
\saveTG{𧭕}{08627}
\saveTG{謭}{08627}
\saveTG{訜}{08627}
\saveTG{𧭆}{08627}
\saveTG{𩐭}{08630}
\saveTG{譕}{08631}
\saveTG{𧧡}{08631}
\saveTG{𧪁}{08631}
\saveTG{𧧴}{08631}
\saveTG{訟}{08632}
\saveTG{詅}{08632}
\saveTG{𧬫}{08632}
\saveTG{諗}{08632}
\saveTG{𧫛}{08632}
\saveTG{𧩟}{08633}
\saveTG{譢}{08633}
\saveTG{謙}{08637}
\saveTG{𩑉}{08637}
\saveTG{䜘}{08640}
\saveTG{𢽒}{08640}
\saveTG{㪟}{08640}
\saveTG{𧩼}{08640}
\saveTG{𢾚}{08640}
\saveTG{𪯙}{08640}
\saveTG{敨}{08640}
\saveTG{譤}{08640}
\saveTG{許}{08640}
\saveTG{譀}{08640}
\saveTG{譈}{08640}
\saveTG{謸}{08640}
\saveTG{䚺}{08640}
\saveTG{誁}{08641}
\saveTG{譐}{08646}
\saveTG{𧫸}{08647}
\saveTG{𧫑}{08647}
\saveTG{詳}{08651}
\saveTG{議}{08653}
\saveTG{誨}{08657}
\saveTG{𧪃}{08661}
\saveTG{譗}{08661}
\saveTG{譜}{08661}
\saveTG{詥}{08661}
\saveTG{𧬆}{08661}
\saveTG{𩐥}{08661}
\saveTG{𩐧}{08662}
\saveTG{誝}{08662}
\saveTG{𧫧}{08664}
\saveTG{譄}{08666}
\saveTG{譮}{08666}
\saveTG{謒}{08667}
\saveTG{𧬢}{08668}
\saveTG{䛦}{08668}
\saveTG{𧦁}{08680}
\saveTG{𫍕}{08681}
\saveTG{𩐰}{08682}
\saveTG{𧬄}{08684}
\saveTG{𧧤}{08684}
\saveTG{䛈}{08684}
\saveTG{譣}{08686}
\saveTG{𧧂}{08690}
\saveTG{𧦜}{08690}
\saveTG{𧧶}{08694}
\saveTG{乻}{08717}
\saveTG{玈}{08732}
\saveTG{𧝗}{08732}
\saveTG{𩞤}{08732}
\saveTG{𩜢}{08732}
\saveTG{𣀩}{08740}
\saveTG{𢻬}{08740}
\saveTG{𫍶}{08740}
\saveTG{𪯒}{08740}
\saveTG{𡼖}{08772}
\saveTG{𪞼}{08782}
\saveTG{䞄}{08806}
\saveTG{𢼵}{08840}
\saveTG{𪯛}{08903}
\saveTG{𣓰}{08904}
\saveTG{𪯜}{08940}
\saveTG{𢽮}{08940}
\saveTG{𢽺}{08940}
\saveTG{𢾉}{08940}
\saveTG{𥆵}{09044}
\saveTG{竗}{09120}
\saveTG{𫁭}{09127}
\saveTG{䇌}{09127}
\saveTG{髝}{09227}
\saveTG{𤒼}{09228}
\saveTG{𪰃}{09250}
\saveTG{𪋷}{09257}
\saveTG{𪋲}{09257}
\saveTG{麟}{09259}
\saveTG{𣃛}{09280}
\saveTG{𡮇}{09400}
\saveTG{𪦍}{09417}
\saveTG{䚱}{09600}
\saveTG{𧫆}{09617}
\saveTG{𩐣}{09617}
\saveTG{𧧯}{09617}
\saveTG{䛧}{09619}
\saveTG{訬}{09620}
\saveTG{𧩡}{09622}
\saveTG{誚}{09627}
\saveTG{䜎}{09627}
\saveTG{讜}{09631}
\saveTG{謎}{09639}
\saveTG{𧫵}{09644}
\saveTG{詊}{09650}
\saveTG{𧬸}{09661}
\saveTG{𪱐}{09662}
\saveTG{譡}{09666}
\saveTG{𡙑}{09684}
\saveTG{𧭢}{09686}
\saveTG{𧭓}{09689}
\saveTG{談}{09689}
\saveTG{詸}{09694}
\saveTG{玅}{09720}
\saveTG{𡭹}{09820}
\saveTG{一}{10000}
\saveTG{覀}{10006}
\saveTG{丂}{10027}
\saveTG{𢀔}{10027}
\saveTG{𠀒}{10027}
\saveTG{𡗙}{10027}
\saveTG{丐}{10027}
\saveTG{亏}{10027}
\saveTG{丏}{10027}
\saveTG{雩}{10027}
\saveTG{亐}{10027}
\saveTG{𢍼}{10043}
\saveTG{𠄞}{10100}
\saveTG{𠄟}{10100}
\saveTG{𠄠}{10100}
\saveTG{𤴔}{10100}
\saveTG{二}{10100}
\saveTG{亖}{10101}
\saveTG{𢀚}{10101}
\saveTG{三}{10101}
\saveTG{歪}{10101}
\saveTG{正}{10101}
\saveTG{盃}{10102}
\saveTG{霊}{10102}
\saveTG{亞}{10102}
\saveTG{亙}{10102}
\saveTG{工}{10102}
\saveTG{互}{10102}
\saveTG{巠}{10102}
\saveTG{五}{10102}
\saveTG{盂}{10102}
\saveTG{㱏}{10102}
\saveTG{𨤏}{10102}
\saveTG{𠄨}{10102}
\saveTG{𠀜}{10102}
\saveTG{𩂳}{10102}
\saveTG{𣅯}{10102}
\saveTG{𣥄}{10102}
\saveTG{㠪}{10102}
\saveTG{㿼}{10102}
\saveTG{亚}{10102}
\saveTG{𩄕}{10102}
\saveTG{𩄝}{10102}
\saveTG{𧟬}{10102}
\saveTG{𠀝}{10102}
\saveTG{𩐄}{10102}
\saveTG{𣥔}{10102}
\saveTG{𥁆}{10102}
\saveTG{𥁁}{10102}
\saveTG{𧖬}{10102}
\saveTG{𥁄}{10102}
\saveTG{𢀑}{10102}
\saveTG{㿻}{10102}
\saveTG{𩄅}{10102}
\saveTG{𠁆}{10102}
\saveTG{𥁶}{10102}
\saveTG{䀃}{10102}
\saveTG{𠁏}{10102}
\saveTG{𧟠}{10102}
\saveTG{𠄸}{10102}
\saveTG{𠄣}{10102}
\saveTG{𠀕}{10102}
\saveTG{𪾌}{10102}
\saveTG{𨣓}{10102}
\saveTG{𩆅}{10102}
\saveTG{䪞}{10102}
\saveTG{𡗶}{10102}
\saveTG{𢒄}{10102}
\saveTG{玉}{10103}
\saveTG{璽}{10103}
\saveTG{琧}{10103}
\saveTG{𠄮}{10103}
\saveTG{𡔈}{10104}
\saveTG{𡉤}{10104}
\saveTG{至}{10104}
\saveTG{垩}{10104}
\saveTG{堊}{10104}
\saveTG{坖}{10104}
\saveTG{坙}{10104}
\saveTG{玊}{10104}
\saveTG{王}{10104}
\saveTG{垔}{10104}
\saveTG{𩂊}{10104}
\saveTG{𢀓}{10104}
\saveTG{𤣩}{10104}
\saveTG{𠄡}{10104}
\saveTG{𡉐}{10104}
\saveTG{𡉙}{10104}
\saveTG{𣄱}{10104}
\saveTG{𡋻}{10104}
\saveTG{𩆔}{10104}
\saveTG{𤫊}{10104}
\saveTG{𩄇}{10104}
\saveTG{𤣪}{10104}
\saveTG{𡊝}{10104}
\saveTG{㘸}{10104}
\saveTG{𡉝}{10104}
\saveTG{𩂐}{10104}
\saveTG{𡋲}{10104}
\saveTG{𡊹}{10104}
\saveTG{𧟿}{10104}
\saveTG{𫕞}{10104}
\saveTG{𡿱}{10104}
\saveTG{壐}{10104}
\saveTG{𩆜}{10104}
\saveTG{𡌥}{10104}
\saveTG{𧟪}{10104}
\saveTG{𡊞}{10104}
\saveTG{𫕶}{10104}
\saveTG{𡊵}{10104}
\saveTG{𡎄}{10104}
\saveTG{𪤬}{10104}
\saveTG{𩇎}{10104}
\saveTG{噩}{10104}
\saveTG{𪡅}{10104}
\saveTG{𠄭}{10104}
\saveTG{𡍂}{10104}
\saveTG{𠄹}{10104}
\saveTG{𡉊}{10104}
\saveTG{𦤳}{10105}
\saveTG{𩃒}{10105}
\saveTG{𩄆}{10105}
\saveTG{𤯚}{10105}
\saveTG{亜}{10105}
\saveTG{𤯔}{10105}
\saveTG{噩}{10106}
\saveTG{𧟡}{10106}
\saveTG{𠀥}{10106}
\saveTG{𠄵}{10106}
\saveTG{亘}{10106}
\saveTG{畺}{10106}
\saveTG{𩄟}{10108}
\saveTG{𫁮}{10108}
\saveTG{豆}{10108}
\saveTG{靊}{10108}
\saveTG{𥩠}{10108}
\saveTG{巫}{10108}
\saveTG{霻}{10108}
\saveTG{靈}{10108}
\saveTG{雴}{10108}
\saveTG{霯}{10108}
\saveTG{𩈓}{10109}
\saveTG{𨥴}{10109}
\saveTG{𨮪}{10109}
\saveTG{𧟴}{10109}
\saveTG{丕}{10109}
\saveTG{霏}{10111}
\saveTG{靋}{10111}
\saveTG{霃}{10112}
\saveTG{𫕩}{10112}
\saveTG{靋}{10112}
\saveTG{疏}{10112}
\saveTG{珫}{10112}
\saveTG{璄}{10112}
\saveTG{琉}{10112}
\saveTG{巰}{10112}
\saveTG{巯}{10112}
\saveTG{雿}{10113}
\saveTG{𤤛}{10114}
\saveTG{𤥦}{10114}
\saveTG{𤪮}{10114}
\saveTG{霪}{10114}
\saveTG{𩅵}{10114}
\saveTG{霔}{10114}
\saveTG{𦥏}{10114}
\saveTG{䨙}{10114}
\saveTG{䨟}{10114}
\saveTG{𨿊}{10115}
\saveTG{𩅚}{10115}
\saveTG{𩆸}{10115}
\saveTG{𨾖}{10115}
\saveTG{𨿋}{10115}
\saveTG{𤫚}{10115}
\saveTG{𨾆}{10115}
\saveTG{𦑏}{10115}
\saveTG{𫕷}{10115}
\saveTG{𨿉}{10115}
\saveTG{䨪}{10115}
\saveTG{𤩔}{10115}
\saveTG{𨾽}{10115}
\saveTG{琟}{10115}
\saveTG{𨿏}{10115}
\saveTG{𨾊}{10115}
\saveTG{𨤻}{10115}
\saveTG{𫘰}{10116}
\saveTG{璮}{10116}
\saveTG{𪛈}{10117}
\saveTG{𠀢}{10117}
\saveTG{𪚦}{10117}
\saveTG{𠁘}{10117}
\saveTG{𩂻}{10117}
\saveTG{𩂷}{10117}
\saveTG{𩃱}{10117}
\saveTG{𩄬}{10117}
\saveTG{𩂸}{10117}
\saveTG{𠀘}{10117}
\saveTG{𩆛}{10117}
\saveTG{𤨞}{10117}
\saveTG{𤦻}{10117}
\saveTG{疏}{10117}
\saveTG{𪻑}{10117}
\saveTG{𫕡}{10117}
\saveTG{𩃰}{10117}
\saveTG{𡏡}{10117}
\saveTG{霃}{10117}
\saveTG{𩄳}{10118}
\saveTG{𤤔}{10118}
\saveTG{霮}{10118}
\saveTG{䨿}{10118}
\saveTG{翋}{10118}
\saveTG{𩃜}{10118}
\saveTG{𩄼}{10121}
\saveTG{𤦺}{10121}
\saveTG{𤧟}{10121}
\saveTG{𪻾}{10122}
\saveTG{𫕪}{10122}
\saveTG{璾}{10123}
\saveTG{𩆤}{10127}
\saveTG{𠁌}{10127}
\saveTG{𠀭}{10127}
\saveTG{霘}{10127}
\saveTG{靎}{10127}
\saveTG{璃}{10127}
\saveTG{霶}{10127}
\saveTG{霠}{10127}
\saveTG{𤨬}{10127}
\saveTG{𤧭}{10127}
\saveTG{𤧼}{10127}
\saveTG{𤤁}{10127}
\saveTG{𦏺}{10127}
\saveTG{𩄋}{10127}
\saveTG{𤧛}{10127}
\saveTG{㻪}{10127}
\saveTG{㻽}{10127}
\saveTG{𤧿}{10127}
\saveTG{𦏳}{10127}
\saveTG{𪻥}{10127}
\saveTG{𩃎}{10127}
\saveTG{𩄘}{10127}
\saveTG{𩃇}{10127}
\saveTG{𩅳}{10127}
\saveTG{𩅕}{10127}
\saveTG{䨒}{10127}
\saveTG{𫕴}{10127}
\saveTG{㻙}{10127}
\saveTG{𩅠}{10127}
\saveTG{𩂧}{10127}
\saveTG{霈}{10127}
\saveTG{𤨭}{10127}
\saveTG{𩃨}{10127}
\saveTG{𤥂}{10130}
\saveTG{玣}{10130}
\saveTG{𧓑}{10131}
\saveTG{𤧠}{10131}
\saveTG{𧔁}{10131}
\saveTG{𩃍}{10131}
\saveTG{𩃚}{10131}
\saveTG{璡}{10131}
\saveTG{𤪗}{10132}
\saveTG{𤪿}{10132}
\saveTG{𩄂}{10132}
\saveTG{𩆼}{10132}
\saveTG{䨧}{10132}
\saveTG{𧟺}{10132}
\saveTG{瓌}{10132}
\saveTG{霐}{10132}
\saveTG{霗}{10132}
\saveTG{瓋}{10132}
\saveTG{玹}{10132}
\saveTG{瓖}{10132}
\saveTG{𤩻}{10135}
\saveTG{𧉗}{10136}
\saveTG{䗞}{10136}
\saveTG{𧌙}{10136}
\saveTG{𩃘}{10136}
\saveTG{𧋠}{10136}
\saveTG{𧈫}{10136}
\saveTG{𩃪}{10136}
\saveTG{蚕}{10136}
\saveTG{蠒}{10136}
\saveTG{蟸}{10136}
\saveTG{蠠}{10136}
\saveTG{蝁}{10136}
\saveTG{𧒜}{10136}
\saveTG{䘉}{10136}
\saveTG{虿}{10136}
\saveTG{𧋔}{10136}
\saveTG{𪼦}{10136}
\saveTG{𤨫}{10137}
\saveTG{𪼥}{10137}
\saveTG{玟}{10140}
\saveTG{𫕦}{10140}
\saveTG{𩅅}{10141}
\saveTG{㻭}{10141}
\saveTG{𫕵}{10141}
\saveTG{䨕}{10141}
\saveTG{𩇋}{10141}
\saveTG{𤧻}{10141}
\saveTG{䙵}{10141}
\saveTG{𦐹}{10141}
\saveTG{𤨘}{10141}
\saveTG{𨐕}{10141}
\saveTG{𩆥}{10142}
\saveTG{𤥿}{10142}
\saveTG{𩆩}{10143}
\saveTG{䨴}{10143}
\saveTG{𩅿}{10143}
\saveTG{𪧽}{10143}
\saveTG{𤥡}{10144}
\saveTG{𤥑}{10144}
\saveTG{𫕭}{10146}
\saveTG{䨵}{10146}
\saveTG{璋}{10146}
\saveTG{𤨼}{10146}
\saveTG{𪼌}{10147}
\saveTG{𩆀}{10147}
\saveTG{𦕪}{10147}
\saveTG{䝄}{10147}
\saveTG{𧰤}{10147}
\saveTG{𩃁}{10147}
\saveTG{𤥥}{10147}
\saveTG{𩆓}{10147}
\saveTG{𪻿}{10147}
\saveTG{𣦞}{10147}
\saveTG{琗}{10148}
\saveTG{珓}{10148}
\saveTG{𩄗}{10149}
\saveTG{𧯱}{10151}
\saveTG{𩆧}{10151}
\saveTG{𢑻}{10152}
\saveTG{𩆷}{10153}
\saveTG{𩆭}{10153}
\saveTG{𢑶}{10153}
\saveTG{𩆨}{10154}
\saveTG{𩅴}{10156}
\saveTG{𠽚}{10156}
\saveTG{䨰}{10156}
\saveTG{𤲾}{10160}
\saveTG{琂}{10161}
\saveTG{琣}{10161}
\saveTG{𤭵}{10161}
\saveTG{䜾}{10161}
\saveTG{霑}{10161}
\saveTG{𩅪}{10161}
\saveTG{𩆎}{10162}
\saveTG{𩃅}{10162}
\saveTG{𩃦}{10163}
\saveTG{𦧆}{10164}
\saveTG{露}{10164}
\saveTG{𩅻}{10164}
\saveTG{瑭}{10165}
\saveTG{𩅹}{10167}
\saveTG{𩆍}{10172}
\saveTG{𧟾}{10172}
\saveTG{䨮}{10177}
\saveTG{雪}{10177}
\saveTG{霟}{10181}
\saveTG{𪻞}{10182}
\saveTG{䨏}{10182}
\saveTG{𤫓}{10184}
\saveTG{𪼁}{10184}
\saveTG{𪽬}{10185}
\saveTG{𩆯}{10186}
\saveTG{㼅}{10186}
\saveTG{䨬}{10194}
\saveTG{霂}{10194}
\saveTG{𪼛}{10194}
\saveTG{琼}{10196}
\saveTG{翞}{10196}
\saveTG{丅}{10200}
\saveTG{丁}{10200}
\saveTG{丆}{10200}
\saveTG{亍}{10201}
\saveTG{𩁷}{10201}
\saveTG{严}{10201}
\saveTG{𡗣}{10201}
\saveTG{𫕨}{10202}
\saveTG{𩅙}{10202}
\saveTG{𩁺}{10202}
\saveTG{𢆺}{10203}
\saveTG{𡔝}{10207}
\saveTG{𢨥}{10207}
\saveTG{𩆠}{10207}
\saveTG{𩂙}{10207}
\saveTG{𩂇}{10207}
\saveTG{歹}{10207}
\saveTG{戸}{10207}
\saveTG{𡗘}{10208}
\saveTG{𠀰}{10209}
\saveTG{𠀣}{10211}
\saveTG{𩂖}{10211}
\saveTG{靂}{10211}
\saveTG{靇}{10211}
\saveTG{龗}{10211}
\saveTG{𩑋}{10212}
\saveTG{𠄱}{10212}
\saveTG{𩆴}{10212}
\saveTG{覔}{10212}
\saveTG{霓}{10212}
\saveTG{死}{10212}
\saveTG{兀}{10212}
\saveTG{元}{10212}
\saveTG{𣩸}{10212}
\saveTG{兀}{10212}
\saveTG{𠑶}{10212}
\saveTG{𩃾}{10213}
\saveTG{𩄁}{10213}
\saveTG{𩄩}{10213}
\saveTG{𫕤}{10214}
\saveTG{𩄌}{10214}
\saveTG{𩈪}{10214}
\saveTG{霳}{10215}
\saveTG{靃}{10215}
\saveTG{𩆄}{10215}
\saveTG{𣨫}{10215}
\saveTG{䝑}{10215}
\saveTG{𩁖}{10215}
\saveTG{𩁙}{10215}
\saveTG{𩁃}{10215}
\saveTG{𨾿}{10215}
\saveTG{𩁝}{10215}
\saveTG{䂌}{10215}
\saveTG{霍}{10215}
\saveTG{霾}{10215}
\saveTG{雡}{10215}
\saveTG{𢀢}{10216}
\saveTG{𩐘}{10216}
\saveTG{𢀯}{10217}
\saveTG{𠁈}{10217}
\saveTG{𠁉}{10217}
\saveTG{𩄏}{10217}
\saveTG{𠀸}{10217}
\saveTG{䨌}{10217}
\saveTG{䨘}{10217}
\saveTG{䨔}{10217}
\saveTG{𩵂}{10217}
\saveTG{𩅄}{10217}
\saveTG{𣨀}{10217}
\saveTG{𧟣}{10217}
\saveTG{𧟽}{10217}
\saveTG{䴡}{10217}
\saveTG{𩳮}{10217}
\saveTG{𠑺}{10217}
\saveTG{𣩜}{10217}
\saveTG{𣩏}{10217}
\saveTG{㱻}{10217}
\saveTG{𩂵}{10217}
\saveTG{𠀡}{10217}
\saveTG{𫌞}{10217}
\saveTG{𫌒}{10217}
\saveTG{𩃥}{10217}
\saveTG{𫅮}{10217}
\saveTG{𣦂}{10217}
\saveTG{𩗏}{10217}
\saveTG{𩂒}{10217}
\saveTG{𩁸}{10217}
\saveTG{𩇑}{10217}
\saveTG{𠙁}{10217}
\saveTG{𠘶}{10217}
\saveTG{𠘲}{10217}
\saveTG{𩖲}{10217}
\saveTG{𠀺}{10217}
\saveTG{颪}{10217}
\saveTG{覔}{10217}
\saveTG{㱞}{10218}
\saveTG{㱷}{10218}
\saveTG{𩆱}{10219}
\saveTG{𤰱}{10220}
\saveTG{𠶘}{10220}
\saveTG{丌}{10220}
\saveTG{亓}{10221}
\saveTG{𢀘}{10221}
\saveTG{𧗩}{10221}
\saveTG{𩂋}{10221}
\saveTG{𣨉}{10221}
\saveTG{𢂇}{10221}
\saveTG{𩇐}{10221}
\saveTG{𦓔}{10222}
\saveTG{𣨹}{10222}
\saveTG{𩂶}{10222}
\saveTG{雺}{10222}
\saveTG{𧱱}{10222}
\saveTG{霽}{10223}
\saveTG{𢁴}{10223}
\saveTG{𩂢}{10223}
\saveTG{𤤰}{10224}
\saveTG{霁}{10224}
\saveTG{𩆹}{10227}
\saveTG{䨜}{10227}
\saveTG{𩃴}{10227}
\saveTG{𩆆}{10227}
\saveTG{䨳}{10227}
\saveTG{𩃮}{10227}
\saveTG{𩇂}{10227}
\saveTG{𩂭}{10227}
\saveTG{𩂂}{10227}
\saveTG{䨞}{10227}
\saveTG{𫕯}{10227}
\saveTG{𫕝}{10227}
\saveTG{𢏭}{10227}
\saveTG{𩂌}{10227}
\saveTG{𩃞}{10227}
\saveTG{𠸲}{10227}
\saveTG{𣨧}{10227}
\saveTG{𫕟}{10227}
\saveTG{𧱿}{10227}
\saveTG{䨖}{10227}
\saveTG{𩄑}{10227}
\saveTG{𠀛}{10227}
\saveTG{𧦸}{10227}
\saveTG{𦘼}{10227}
\saveTG{𩰬}{10227}
\saveTG{𢄉}{10227}
\saveTG{𩆚}{10227}
\saveTG{𠠢}{10227}
\saveTG{𦟆}{10227}
\saveTG{𠀷}{10227}
\saveTG{𠁐}{10227}
\saveTG{𤘊}{10227}
\saveTG{𩇃}{10227}
\saveTG{𩅭}{10227}
\saveTG{𩅶}{10227}
\saveTG{𣩀}{10227}
\saveTG{𫔬}{10227}
\saveTG{𢑎}{10227}
\saveTG{𪟊}{10227}
\saveTG{𩂪}{10227}
\saveTG{𩱐}{10227}
\saveTG{𣍡}{10227}
\saveTG{𠀹}{10227}
\saveTG{𦙖}{10227}
\saveTG{𫕧}{10227}
\saveTG{𦙂}{10227}
\saveTG{甭}{10227}
\saveTG{丙}{10227}
\saveTG{豴}{10227}
\saveTG{鬲}{10227}
\saveTG{而}{10227}
\saveTG{爾}{10227}
\saveTG{雱}{10227}
\saveTG{雰}{10227}
\saveTG{靍}{10227}
\saveTG{靏}{10227}
\saveTG{丽}{10227}
\saveTG{雳}{10227}
\saveTG{両}{10227}
\saveTG{兩}{10227}
\saveTG{霛}{10227}
\saveTG{霧}{10227}
\saveTG{万}{10227}
\saveTG{需}{10227}
\saveTG{霄}{10227}
\saveTG{襾}{10227}
\saveTG{雨}{10227}
\saveTG{霱}{10227}
\saveTG{帀}{10227}
\saveTG{霌}{10227}
\saveTG{𨳌}{10227}
\saveTG{𧖔}{10227}
\saveTG{𤌎}{10227}
\saveTG{䨦}{10227}
\saveTG{𡕌}{10227}
\saveTG{𪟀}{10227}
\saveTG{𪜁}{10227}
\saveTG{𢀰}{10227}
\saveTG{𣃟}{10227}
\saveTG{𩫇}{10227}
\saveTG{𧟩}{10227}
\saveTG{𣃪}{10227}
\saveTG{𠱩}{10227}
\saveTG{𠀑}{10227}
\saveTG{两}{10227}
\saveTG{𢁳}{10227}
\saveTG{𣩅}{10227}
\saveTG{𣃙}{10227}
\saveTG{𢐊}{10227}
\saveTG{𢎷}{10227}
\saveTG{𢀪}{10227}
\saveTG{𪻟}{10227}
\saveTG{𨳔}{10227}
\saveTG{𩅡}{10227}
\saveTG{䨝}{10227}
\saveTG{𩆾}{10227}
\saveTG{𩄨}{10227}
\saveTG{𩆳}{10227}
\saveTG{㫄}{10227}
\saveTG{𨳏}{10227}
\saveTG{𨴙}{10227}
\saveTG{𨶄}{10227}
\saveTG{𦝽}{10227}
\saveTG{𡂦}{10227}
\saveTG{𢄀}{10227}
\saveTG{𪼀}{10227}
\saveTG{𫔙}{10227}
\saveTG{䨶}{10227}
\saveTG{𢅍}{10228}
\saveTG{奡}{10228}
\saveTG{𡗤}{10228}
\saveTG{𩇁}{10228}
\saveTG{下}{10230}
\saveTG{𠂪}{10230}
\saveTG{𩂚}{10231}
\saveTG{䨹}{10231}
\saveTG{䨎}{10231}
\saveTG{雫}{10231}
\saveTG{𨠊}{10232}
\saveTG{𣳳}{10232}
\saveTG{㱢}{10232}
\saveTG{𣲮}{10232}
\saveTG{𣩹}{10232}
\saveTG{豖}{10232}
\saveTG{霢}{10232}
\saveTG{霥}{10232}
\saveTG{靀}{10232}
\saveTG{震}{10232}
\saveTG{豕}{10232}
\saveTG{弦}{10232}
\saveTG{𩅽}{10232}
\saveTG{𩅲}{10232}
\saveTG{𩆬}{10232}
\saveTG{𩇀}{10232}
\saveTG{𩄖}{10232}
\saveTG{𧰳}{10232}
\saveTG{𣩽}{10232}
\saveTG{𣱳}{10232}
\saveTG{𧰧}{10232}
\saveTG{𩅥}{10232}
\saveTG{𩂡}{10232}
\saveTG{𣸞}{10232}
\saveTG{𧰬}{10232}
\saveTG{𨑇}{10232}
\saveTG{𢐿}{10232}
\saveTG{𩆰}{10232}
\saveTG{𩂍}{10232}
\saveTG{𩅧}{10232}
\saveTG{𩄡}{10237}
\saveTG{𩆌}{10237}
\saveTG{䨸}{10237}
\saveTG{𠪕}{10237}
\saveTG{𩆝}{10239}
\saveTG{䰚}{10240}
\saveTG{𢐻}{10241}
\saveTG{𠙚}{10241}
\saveTG{𧲉}{10241}
\saveTG{㱸}{10241}
\saveTG{㢹}{10241}
\saveTG{霹}{10241}
\saveTG{霨}{10241}
\saveTG{𣨜}{10242}
\saveTG{𠅗}{10242}
\saveTG{𩂠}{10243}
\saveTG{𩃐}{10243}
\saveTG{𩂛}{10243}
\saveTG{𧱙}{10244}
\saveTG{㢺}{10244}
\saveTG{𩅋}{10245}
\saveTG{𠁅}{10245}
\saveTG{𧟱}{10247}
\saveTG{𧟵}{10247}
\saveTG{𩕂}{10247}
\saveTG{𠄦}{10247}
\saveTG{䨷}{10247}
\saveTG{𩂹}{10247}
\saveTG{𩄸}{10247}
\saveTG{𪫄}{10247}
\saveTG{𧱮}{10247}
\saveTG{𧱯}{10247}
\saveTG{𣨲}{10247}
\saveTG{雭}{10247}
\saveTG{䨱}{10247}
\saveTG{覆}{10247}
\saveTG{霞}{10247}
\saveTG{弴}{10247}
\saveTG{𧠃}{10247}
\saveTG{𧠂}{10248}
\saveTG{𩅓}{10248}
\saveTG{𧠀}{10248}
\saveTG{𩅢}{10248}
\saveTG{𪺨}{10248}
\saveTG{𣨛}{10248}
\saveTG{覈}{10248}
\saveTG{霰}{10248}
\saveTG{霺}{10248}
\saveTG{霚}{10248}
\saveTG{𧲀}{10252}
\saveTG{𩆪}{10253}
\saveTG{𩂟}{10254}
\saveTG{𩄐}{10254}
\saveTG{𩄦}{10254}
\saveTG{𩄴}{10256}
\saveTG{𩄛}{10256}
\saveTG{䨩}{10256}
\saveTG{𩂑}{10257}
\saveTG{𫕰}{10258}
\saveTG{𣨌}{10261}
\saveTG{殕}{10261}
\saveTG{𩃫}{10262}
\saveTG{𧟸}{10262}
\saveTG{𩅗}{10264}
\saveTG{𩅬}{10264}
\saveTG{霿}{10264}
\saveTG{𧱵}{10265}
\saveTG{𩄎}{10267}
\saveTG{𢀞}{10274}
\saveTG{𩅎}{10277}
\saveTG{𩃿}{10277}
\saveTG{𩰶}{10282}
\saveTG{豥}{10282}
\saveTG{𩃺}{10284}
\saveTG{覄}{10284}
\saveTG{𩅺}{10286}
\saveTG{彍}{10286}
\saveTG{𥎥}{10289}
\saveTG{霡}{10292}
\saveTG{𢏸}{10293}
\saveTG{𢏲}{10294}
\saveTG{𩃕}{10294}
\saveTG{𩅩}{10294}
\saveTG{𩂯}{10294}
\saveTG{𣨚}{10294}
\saveTG{𢑀}{10294}
\saveTG{䨯}{10295}
\saveTG{𡰏}{10296}
\saveTG{𣨣}{10296}
\saveTG{弶}{10296}
\saveTG{𣦶}{10297}
\saveTG{𨗒}{10301}
\saveTG{零}{10302}
\saveTG{覂}{10302}
\saveTG{𫕲}{10302}
\saveTG{𨖕}{10302}
\saveTG{𩆖}{10302}
\saveTG{𩄮}{10303}
\saveTG{𩅆}{10304}
\saveTG{𩅛}{10305}
\saveTG{䨤}{10306}
\saveTG{𠁛}{10306}
\saveTG{𡚑}{10306}
\saveTG{䨨}{10307}
\saveTG{𩆡}{10307}
\saveTG{𩄫}{10307}
\saveTG{𩄲}{10309}
\saveTG{𩅘}{10309}
\saveTG{𩁎}{10315}
\saveTG{𠀦}{10320}
\saveTG{𨠩}{10326}
\saveTG{䲶}{10327}
\saveTG{𩖔}{10327}
\saveTG{𩾏}{10327}
\saveTG{𩿳}{10327}
\saveTG{鴌}{10327}
\saveTG{焉}{10327}
\saveTG{𤈅}{10330}
\saveTG{𢙏}{10330}
\saveTG{𢘫}{10331}
\saveTG{𢙂}{10331}
\saveTG{䵤}{10331}
\saveTG{䵡}{10331}
\saveTG{㤪}{10331}
\saveTG{炁}{10331}
\saveTG{惡}{10331}
\saveTG{恶}{10331}
\saveTG{𪸓}{10331}
\saveTG{悪}{10331}
\saveTG{𢖶}{10331}
\saveTG{𢗏}{10331}
\saveTG{𩄈}{10331}
\saveTG{忢}{10331}
\saveTG{忈}{10331}
\saveTG{𩄰}{10332}
\saveTG{䨚}{10332}
\saveTG{恧}{10332}
\saveTG{𩂈}{10332}
\saveTG{𢛆}{10333}
\saveTG{𤉄}{10333}
\saveTG{𩃯}{10333}
\saveTG{𢝊}{10333}
\saveTG{忑}{10333}
\saveTG{𩂓}{10334}
\saveTG{𢟜}{10334}
\saveTG{𤍱}{10334}
\saveTG{𢗬}{10334}
\saveTG{𢘑}{10336}
\saveTG{𢤏}{10336}
\saveTG{𩅔}{10336}
\saveTG{𢙯}{10336}
\saveTG{𢚧}{10336}
\saveTG{𢜊}{10336}
\saveTG{𩵶}{10336}
\saveTG{𩸖}{10336}
\saveTG{𢙣}{10336}
\saveTG{𢣭}{10337}
\saveTG{㤁}{10338}
\saveTG{𢞘}{10338}
\saveTG{慐}{10338}
\saveTG{忝}{10338}
\saveTG{𢡰}{10338}
\saveTG{𢗫}{10339}
\saveTG{㶨}{10339}
\saveTG{𩅂}{10343}
\saveTG{𩆵}{10351}
\saveTG{𨀂}{10382}
\saveTG{𩇍}{10393}
\saveTG{𪬧}{10396}
\saveTG{廴}{10400}
\saveTG{于}{10400}
\saveTG{𡕒}{10400}
\saveTG{𣁨}{10400}
\saveTG{干}{10400}
\saveTG{耳}{10400}
\saveTG{雯}{10400}
\saveTG{霆}{10401}
\saveTG{𩂀}{10401}
\saveTG{𦕂}{10401}
\saveTG{𩃀}{10401}
\saveTG{𩈞}{10401}
\saveTG{𤬧}{10401}
\saveTG{𢀝}{10401}
\saveTG{𩁹}{10401}
\saveTG{𣍙}{10403}
\saveTG{霎}{10404}
\saveTG{耍}{10404}
\saveTG{婱}{10404}
\saveTG{要}{10404}
\saveTG{𡛌}{10404}
\saveTG{霋}{10404}
\saveTG{嫑}{10404}
\saveTG{孁}{10404}
\saveTG{𧟹}{10406}
\saveTG{𧟻}{10406}
\saveTG{𠀼}{10406}
\saveTG{覃}{10406}
\saveTG{𧥡}{10406}
\saveTG{𩅈}{10406}
\saveTG{𡕠}{10407}
\saveTG{𠭍}{10407}
\saveTG{夏}{10407}
\saveTG{𩃃}{10407}
\saveTG{𡕿}{10407}
\saveTG{𠮕}{10407}
\saveTG{夒}{10407}
\saveTG{䨗}{10407}
\saveTG{𡕢}{10407}
\saveTG{𩀝}{10407}
\saveTG{䨥}{10407}
\saveTG{𩆿}{10407}
\saveTG{𢙴}{10407}
\saveTG{𩂏}{10407}
\saveTG{𠬷}{10407}
\saveTG{𡕾}{10407}
\saveTG{𩅍}{10407}
\saveTG{𩁾}{10407}
\saveTG{𩃡}{10407}
\saveTG{䨫}{10407}
\saveTG{憂}{10407}
\saveTG{𩃣}{10408}
\saveTG{𩂎}{10408}
\saveTG{𠀱}{10408}
\saveTG{𠦓}{10408}
\saveTG{㔻}{10409}
\saveTG{𢆓}{10409}
\saveTG{平}{10409}
\saveTG{雽}{10409}
\saveTG{𦔮}{10412}
\saveTG{𦗓}{10412}
\saveTG{旡}{10412}
\saveTG{无}{10412}
\saveTG{䬠}{10413}
\saveTG{䧲}{10415}
\saveTG{𨾌}{10415}
\saveTG{𩀄}{10415}
\saveTG{𩀽}{10415}
\saveTG{𨾑}{10415}
\saveTG{𨾐}{10415}
\saveTG{𡦜}{10415}
\saveTG{雃}{10415}
\saveTG{𩀳}{10415}
\saveTG{𦓑}{10417}
\saveTG{𩅀}{10417}
\saveTG{𡯊}{10417}
\saveTG{𩃄}{10417}
\saveTG{𩅱}{10417}
\saveTG{𩇊}{10417}
\saveTG{𪜂}{10417}
\saveTG{聤}{10421}
\saveTG{𩃋}{10422}
\saveTG{霩}{10427}
\saveTG{𠢨}{10427}
\saveTG{𠢘}{10427}
\saveTG{𠢊}{10427}
\saveTG{𠀅}{10427}
\saveTG{𪪇}{10427}
\saveTG{𦗽}{10427}
\saveTG{𦗍}{10427}
\saveTG{𦗑}{10427}
\saveTG{雾}{10427}
\saveTG{𦗐}{10432}
\saveTG{𧚠}{10432}
\saveTG{𡦉}{10440}
\saveTG{𪦹}{10440}
\saveTG{开}{10440}
\saveTG{𢛤}{10441}
\saveTG{𩁻}{10441}
\saveTG{𡚬}{10441}
\saveTG{聶}{10441}
\saveTG{𢍈}{10441}
\saveTG{𢍮}{10441}
\saveTG{𩂽}{10441}
\saveTG{𡦍}{10441}
\saveTG{𡛵}{10441}
\saveTG{𩂦}{10441}
\saveTG{弄}{10441}
\saveTG{𧟨}{10442}
\saveTG{𩂫}{10442}
\saveTG{𥒇}{10442}
\saveTG{𢍐}{10442}
\saveTG{𧟤}{10442}
\saveTG{𡛍}{10443}
\saveTG{𩆕}{10444}
\saveTG{𩈽}{10444}
\saveTG{𢍗}{10444}
\saveTG{𩈾}{10444}
\saveTG{𩂜}{10445}
\saveTG{𧟥}{10446}
\saveTG{𢍦}{10446}
\saveTG{𩃩}{10446}
\saveTG{𢍞}{10446}
\saveTG{𥐧}{10446}
\saveTG{𩄽}{10446}
\saveTG{𩇈}{10447}
\saveTG{𩂤}{10447}
\saveTG{𩄢}{10447}
\saveTG{𠕂}{10447}
\saveTG{孬}{10447}
\saveTG{聂}{10447}
\saveTG{再}{10447}
\saveTG{𡖃}{10447}
\saveTG{𩂔}{10447}
\saveTG{𡛎}{10447}
\saveTG{𦖒}{10448}
\saveTG{𩅁}{10448}
\saveTG{霵}{10453}
\saveTG{𦖢}{10461}
\saveTG{𩂰}{10463}
\saveTG{𡥟}{10471}
\saveTG{𪦸}{10480}
\saveTG{孩}{10482}
\saveTG{𢌌}{10486}
\saveTG{𦘅}{10486}
\saveTG{𩄿}{10488}
\saveTG{㸦}{10500}
\saveTG{𥓨}{10502}
\saveTG{𤘔}{10502}
\saveTG{𪭔}{10503}
\saveTG{𩃔}{10503}
\saveTG{𩂅}{10503}
\saveTG{戛}{10503}
\saveTG{戞}{10503}
\saveTG{𢦌}{10503}
\saveTG{𩃢}{10503}
\saveTG{㸴}{10506}
\saveTG{𩂝}{10506}
\saveTG{𩂼}{10506}
\saveTG{更}{10506}
\saveTG{鞷}{10506}
\saveTG{𩅦}{10506}
\saveTG{䨣}{10506}
\saveTG{𫖌}{10506}
\saveTG{𩂘}{10506}
\saveTG{𠄙}{10507}
\saveTG{𩃳}{10508}
\saveTG{𤘮}{10509}
\saveTG{䰟}{10513}
\saveTG{𩄔}{10513}
\saveTG{𩃧}{10517}
\saveTG{𩳧}{10517}
\saveTG{覉}{10521}
\saveTG{𣩁}{10521}
\saveTG{𠄽}{10521}
\saveTG{𩆺}{10521}
\saveTG{𧟶}{10527}
\saveTG{霸}{10527}
\saveTG{覇}{10527}
\saveTG{覊}{10527}
\saveTG{𩄤}{10527}
\saveTG{𫕬}{10527}
\saveTG{𫕳}{10527}
\saveTG{𧟳}{10527}
\saveTG{𩂕}{10527}
\saveTG{𩃑}{10527}
\saveTG{𫕮}{10527}
\saveTG{𩃛}{10527}
\saveTG{𫕫}{10532}
\saveTG{𧠄}{10541}
\saveTG{𩃂}{10548}
\saveTG{䨍}{10551}
\saveTG{𧟲}{10552}
\saveTG{𩂨}{10557}
\saveTG{𠕅}{10557}
\saveTG{𩆋}{10574}
\saveTG{𠀯}{10600}
\saveTG{𣄼}{10600}
\saveTG{矿}{10600}
\saveTG{覀}{10600}
\saveTG{𠮛}{10600}
\saveTG{𠄿}{10601}
\saveTG{𥐖}{10601}
\saveTG{𠵥}{10601}
\saveTG{𥄪}{10601}
\saveTG{𣇩}{10601}
\saveTG{𤲢}{10601}
\saveTG{𣄮}{10601}
\saveTG{𣅨}{10601}
\saveTG{𠺞}{10601}
\saveTG{𣈆}{10601}
\saveTG{𩅨}{10601}
\saveTG{𠳄}{10601}
\saveTG{𠷡}{10601}
\saveTG{𩇄}{10601}
\saveTG{𩄒}{10601}
\saveTG{晋}{10601}
\saveTG{雷}{10601}
\saveTG{霅}{10601}
\saveTG{吾}{10601}
\saveTG{𢀛}{10601}
\saveTG{𫌺}{10601}
\saveTG{𧨈}{10601}
\saveTG{𠄰}{10601}
\saveTG{䨐}{10601}
\saveTG{晉}{10601}
\saveTG{𩉆}{10602}
\saveTG{𩄯}{10602}
\saveTG{𩅸}{10602}
\saveTG{𤾄}{10602}
\saveTG{𩂥}{10602}
\saveTG{䨓}{10602}
\saveTG{𣍌}{10602}
\saveTG{百}{10602}
\saveTG{石}{10602}
\saveTG{雼}{10602}
\saveTG{霤}{10602}
\saveTG{面}{10602}
\saveTG{靣}{10602}
\saveTG{霫}{10602}
\saveTG{𥑟}{10602}
\saveTG{𩈲}{10602}
\saveTG{𠮴}{10602}
\saveTG{𡘒}{10603}
\saveTG{𩄉}{10603}
\saveTG{𩄙}{10603}
\saveTG{㐁}{10604}
\saveTG{酉}{10604}
\saveTG{西}{10604}
\saveTG{𠀬}{10604}
\saveTG{𤘉}{10604}
\saveTG{𩆘}{10605}
\saveTG{𠰶}{10605}
\saveTG{畐}{10606}
\saveTG{𩅑}{10606}
\saveTG{𡈧}{10606}
\saveTG{𠁞}{10606}
\saveTG{𩆻}{10606}
\saveTG{𩆒}{10606}
\saveTG{𩆞}{10606}
\saveTG{𩂿}{10607}
\saveTG{𤲇}{10607}
\saveTG{畱}{10607}
\saveTG{𩄄}{10608}
\saveTG{𣉩}{10608}
\saveTG{昋}{10608}
\saveTG{䂖}{10608}
\saveTG{吞}{10608}
\saveTG{𠀮}{10608}
\saveTG{𤰳}{10608}
\saveTG{𣅆}{10608}
\saveTG{𩅮}{10608}
\saveTG{𡈯}{10609}
\saveTG{𣆇}{10609}
\saveTG{𥔡}{10609}
\saveTG{𠴅}{10609}
\saveTG{否}{10609}
\saveTG{𤰺}{10609}
\saveTG{𥄓}{10609}
\saveTG{㫘}{10609}
\saveTG{𧥽}{10611}
\saveTG{𧭳}{10611}
\saveTG{釄}{10611}
\saveTG{𠁙}{10611}
\saveTG{䃺}{10611}
\saveTG{𥒝}{10612}
\saveTG{䃙}{10612}
\saveTG{硫}{10612}
\saveTG{酼}{10612}
\saveTG{𥗯}{10612}
\saveTG{醯}{10612}
\saveTG{𥐞}{10612}
\saveTG{𥒹}{10614}
\saveTG{䂯}{10614}
\saveTG{砫}{10614}
\saveTG{𩀦}{10615}
\saveTG{䃥}{10615}
\saveTG{𨣒}{10615}
\saveTG{𥗭}{10615}
\saveTG{碓}{10615}
\saveTG{醀}{10615}
\saveTG{𩀿}{10615}
\saveTG{𨣚}{10616}
\saveTG{䃪}{10616}
\saveTG{砊}{10617}
\saveTG{䃷}{10617}
\saveTG{𡕋}{10617}
\saveTG{𨢷}{10617}
\saveTG{𥕚}{10617}
\saveTG{砬}{10618}
\saveTG{可}{10620}
\saveTG{𠾳}{10621}
\saveTG{𪜄}{10621}
\saveTG{碠}{10621}
\saveTG{𩃤}{10621}
\saveTG{𩃟}{10621}
\saveTG{㪽}{10621}
\saveTG{哥}{10621}
\saveTG{𥑈}{10621}
\saveTG{𥔎}{10621}
\saveTG{𥖭}{10623}
\saveTG{𨠨}{10624}
\saveTG{𥔷}{10624}
\saveTG{𥓯}{10627}
\saveTG{𨢐}{10627}
\saveTG{𨢓}{10627}
\saveTG{𧪨}{10627}
\saveTG{𣅰}{10627}
\saveTG{𨠺}{10627}
\saveTG{𥕮}{10627}
\saveTG{𩅇}{10627}
\saveTG{𨣧}{10627}
\saveTG{𩃼}{10627}
\saveTG{𣈇}{10627}
\saveTG{靄}{10627}
\saveTG{磅}{10627}
\saveTG{碲}{10627}
\saveTG{醨}{10627}
\saveTG{碻}{10627}
\saveTG{奣}{10627}
\saveTG{霷}{10627}
\saveTG{𥕐}{10627}
\saveTG{𨢶}{10630}
\saveTG{𥑃}{10630}
\saveTG{硛}{10630}
\saveTG{𪿖}{10631}
\saveTG{𥕁}{10631}
\saveTG{䩌}{10631}
\saveTG{𥗡}{10631}
\saveTG{礁}{10631}
\saveTG{醮}{10631}
\saveTG{砿}{10632}
\saveTG{䃶}{10632}
\saveTG{醸}{10632}
\saveTG{磙}{10632}
\saveTG{𥑴}{10632}
\saveTG{𥗝}{10632}
\saveTG{䃵}{10632}
\saveTG{釀}{10632}
\saveTG{𥕦}{10632}
\saveTG{𥖾}{10632}
\saveTG{醷}{10636}
\saveTG{𥖝}{10637}
\saveTG{砇}{10640}
\saveTG{𩄊}{10641}
\saveTG{𨣳}{10641}
\saveTG{𨣶}{10641}
\saveTG{礔}{10641}
\saveTG{𨐚}{10641}
\saveTG{𨐓}{10641}
\saveTG{𪿢}{10641}
\saveTG{𩆑}{10642}
\saveTG{𥓵}{10642}
\saveTG{𨠪}{10643}
\saveTG{𩄺}{10644}
\saveTG{𥕞}{10646}
\saveTG{𩂣}{10647}
\saveTG{𩁦}{10647}
\saveTG{𥖂}{10647}
\saveTG{醇}{10647}
\saveTG{䂭}{10648}
\saveTG{𨠦}{10648}
\saveTG{𨣑}{10648}
\saveTG{礮}{10648}
\saveTG{碎}{10648}
\saveTG{醉}{10648}
\saveTG{𪿫}{10648}
\saveTG{𫕛}{10652}
\saveTG{𨣴}{10652}
\saveTG{𩉑}{10652}
\saveTG{𥗂}{10652}
\saveTG{𨡤}{10660}
\saveTG{䂴}{10661}
\saveTG{𩂩}{10661}
\saveTG{𨡄}{10661}
\saveTG{𥕒}{10661}
\saveTG{𩈴}{10661}
\saveTG{𩆢}{10661}
\saveTG{䤃}{10661}
\saveTG{碚}{10661}
\saveTG{靐}{10661}
\saveTG{霝}{10661}
\saveTG{醅}{10661}
\saveTG{𤾩}{10662}
\saveTG{𥗲}{10662}
\saveTG{𡕎}{10662}
\saveTG{𩄷}{10662}
\saveTG{𩄣}{10662}
\saveTG{𩆗}{10662}
\saveTG{礳}{10662}
\saveTG{磊}{10662}
\saveTG{𧟮}{10664}
\saveTG{𣆛}{10664}
\saveTG{醣}{10665}
\saveTG{磄}{10665}
\saveTG{𩅟}{10666}
\saveTG{靁}{10666}
\saveTG{醕}{10666}
\saveTG{𩇓}{10666}
\saveTG{𩆮}{10668}
\saveTG{𩆁}{10668}
\saveTG{𨠯}{10672}
\saveTG{𨠳}{10682}
\saveTG{𠁓}{10682}
\saveTG{硋}{10682}
\saveTG{𩇅}{10684}
\saveTG{𥓷}{10685}
\saveTG{礦}{10686}
\saveTG{𨣽}{10691}
\saveTG{𫑾}{10691}
\saveTG{𠀿}{10692}
\saveTG{醿}{10693}
\saveTG{䩋}{10694}
\saveTG{磼}{10694}
\saveTG{醾}{10694}
\saveTG{𨣿}{10694}
\saveTG{𨟼}{10694}
\saveTG{𦧢}{10694}
\saveTG{𩇆}{10694}
\saveTG{𪿳}{10694}
\saveTG{䣼}{10696}
\saveTG{䃄}{10696}
\saveTG{𥕎}{10699}
\saveTG{𠀂}{10710}
\saveTG{𡂧}{10710}
\saveTG{㐙}{10711}
\saveTG{䜳}{10711}
\saveTG{㐚}{10711}
\saveTG{𩁿}{10711}
\saveTG{𢀒}{10711}
\saveTG{㐏}{10711}
\saveTG{𠒞}{10712}
\saveTG{𩇌}{10712}
\saveTG{𩁶}{10712}
\saveTG{雹}{10712}
\saveTG{𩃓}{10714}
\saveTG{𠃗}{10714}
\saveTG{𩆇}{10714}
\saveTG{䨋}{10714}
\saveTG{𤮌}{10714}
\saveTG{𩃗}{10714}
\saveTG{雮}{10714}
\saveTG{𩂬}{10715}
\saveTG{𩅐}{10715}
\saveTG{𠀻}{10715}
\saveTG{𩄱}{10715}
\saveTG{𠄲}{10715}
\saveTG{𩄀}{10715}
\saveTG{𪘮}{10715}
\saveTG{𦣽}{10716}
\saveTG{電}{10716}
\saveTG{鼋}{10716}
\saveTG{𠄾}{10716}
\saveTG{𩂄}{10717}
\saveTG{𪓿}{10717}
\saveTG{𪕢}{10717}
\saveTG{𠄢}{10717}
\saveTG{𪥉}{10717}
\saveTG{乤}{10717}
\saveTG{乭}{10717}
\saveTG{瓦}{10717}
\saveTG{黿}{10717}
\saveTG{𫕠}{10717}
\saveTG{㐉}{10717}
\saveTG{𩅯}{10717}
\saveTG{𩇇}{10717}
\saveTG{𠀀}{10717}
\saveTG{㐓}{10717}
\saveTG{𤮮}{10717}
\saveTG{𪓣}{10717}
\saveTG{𩃊}{10717}
\saveTG{𩆈}{10717}
\saveTG{䶘}{10718}
\saveTG{𡂤}{10718}
\saveTG{䙴}{10718}
\saveTG{䨢}{10718}
\saveTG{㔿}{10720}
\saveTG{𤮂}{10726}
\saveTG{𩁽}{10727}
\saveTG{霭}{10727}
\saveTG{𩄹}{10727}
\saveTG{𠣣}{10727}
\saveTG{𡴈}{10727}
\saveTG{𩃬}{10727}
\saveTG{𠄳}{10727}
\saveTG{𩂾}{10727}
\saveTG{䨠}{10727}
\saveTG{𩂞}{10727}
\saveTG{𩅜}{10727}
\saveTG{𠀾}{10731}
\saveTG{𩅣}{10731}
\saveTG{䨺}{10731}
\saveTG{𠫔}{10731}
\saveTG{𢆰}{10731}
\saveTG{𨱥}{10732}
\saveTG{𨱼}{10732}
\saveTG{䬩}{10732}
\saveTG{𥘖}{10732}
\saveTG{雲}{10732}
\saveTG{丟}{10732}
\saveTG{𩆶}{10732}
\saveTG{𧙑}{10732}
\saveTG{𩂱}{10732}
\saveTG{云}{10732}
\saveTG{𣯍}{10741}
\saveTG{𥪵}{10744}
\saveTG{𠮎}{10747}
\saveTG{𩰥}{10747}
\saveTG{𢀣}{10747}
\saveTG{𣱉}{10747}
\saveTG{霉}{10757}
\saveTG{𫕢}{10757}
\saveTG{𣫱}{10757}
\saveTG{𩄥}{10772}
\saveTG{𡹅}{10772}
\saveTG{𤣶}{10772}
\saveTG{䨡}{10772}
\saveTG{𩂗}{10772}
\saveTG{𠚇}{10772}
\saveTG{𤘍}{10772}
\saveTG{𠚑}{10772}
\saveTG{画}{10772}
\saveTG{𠚃}{10772}
\saveTG{雸}{10774}
\saveTG{𦉣}{10774}
\saveTG{𡔥}{10774}
\saveTG{𤮻}{10774}
\saveTG{𩃹}{10774}
\saveTG{丣}{10777}
\saveTG{𤘈}{10777}
\saveTG{𩅒}{10777}
\saveTG{𠆣}{10800}
\saveTG{𩄚}{10801}
\saveTG{趸}{10801}
\saveTG{兲}{10801}
\saveTG{𠔬}{10801}
\saveTG{𠔐}{10801}
\saveTG{𠇬}{10801}
\saveTG{𠁒}{10801}
\saveTG{霬}{10801}
\saveTG{贡}{10802}
\saveTG{贾}{10802}
\saveTG{页}{10802}
\saveTG{𤴓}{10802}
\saveTG{𤴤}{10802}
\saveTG{𤴕}{10802}
\saveTG{𩃈}{10802}
\saveTG{𧾸}{10802}
\saveTG{𫏔}{10802}
\saveTG{䨑}{10802}
\saveTG{𫕥}{10802}
\saveTG{𧟧}{10803}
\saveTG{𩅏}{10803}
\saveTG{㺯}{10804}
\saveTG{𩅌}{10804}
\saveTG{𩃉}{10804}
\saveTG{𪥇}{10804}
\saveTG{𡘀}{10804}
\saveTG{𩄻}{10804}
\saveTG{𩃙}{10804}
\saveTG{奀}{10804}
\saveTG{耎}{10804}
\saveTG{天}{10804}
\saveTG{𩂉}{10804}
\saveTG{雵}{10805}
\saveTG{𩃖}{10805}
\saveTG{𠁍}{10805}
\saveTG{𩂃}{10805}
\saveTG{霙}{10805}
\saveTG{𩆃}{10806}
\saveTG{𧵹}{10806}
\saveTG{靌}{10806}
\saveTG{賈}{10806}
\saveTG{霣}{10806}
\saveTG{貢}{10806}
\saveTG{𩄾}{10806}
\saveTG{賷}{10806}
\saveTG{𧴳}{10806}
\saveTG{𧵔}{10806}
\saveTG{𩇉}{10806}
\saveTG{𩂲}{10806}
\saveTG{頁}{10806}
\saveTG{𩆫}{10807}
\saveTG{𤊆}{10809}
\saveTG{𩅊}{10809}
\saveTG{𩄪}{10809}
\saveTG{㶮}{10809}
\saveTG{䎡}{10809}
\saveTG{㶾}{10809}
\saveTG{𤊗}{10809}
\saveTG{𩃏}{10809}
\saveTG{𩉈}{10809}
\saveTG{𤈨}{10809}
\saveTG{烎}{10809}
\saveTG{灭}{10809}
\saveTG{𧟼}{10814}
\saveTG{𩀁}{10815}
\saveTG{𩀋}{10815}
\saveTG{𩆣}{10817}
\saveTG{𩆙}{10817}
\saveTG{𠀔}{10820}
\saveTG{𩅰}{10822}
\saveTG{𪳈}{10841}
\saveTG{䙲}{10846}
\saveTG{𩆉}{10853}
\saveTG{𧵲}{10861}
\saveTG{𧶝}{10862}
\saveTG{𧶪}{10862}
\saveTG{𧶏}{10866}
\saveTG{𩅉}{10882}
\saveTG{𡙎}{10884}
\saveTG{𩅼}{10886}
\saveTG{𩖏}{10886}
\saveTG{𩇒}{10889}
\saveTG{𤆱}{10892}
\saveTG{㶣}{10894}
\saveTG{𤇈}{10896}
\saveTG{䙳}{10896}
\saveTG{㶪}{10899}
\saveTG{不}{10900}
\saveTG{𤓯}{10900}
\saveTG{𡭕}{10900}
\saveTG{𣎴}{10900}
\saveTG{示}{10901}
\saveTG{票}{10901}
\saveTG{𩂆}{10901}
\saveTG{𩅃}{10901}
\saveTG{𡭚}{10901}
\saveTG{𫀀}{10901}
\saveTG{汞}{10902}
\saveTG{泵}{10902}
\saveTG{𩄜}{10903}
\saveTG{𥾟}{10903}
\saveTG{𠀚}{10903}
\saveTG{𩆏}{10904}
\saveTG{栗}{10904}
\saveTG{粟}{10904}
\saveTG{𣡷}{10904}
\saveTG{𩄓}{10904}
\saveTG{𩃵}{10904}
\saveTG{𥤋}{10904}
\saveTG{𩆐}{10904}
\saveTG{𣐂}{10904}
\saveTG{𣏏}{10904}
\saveTG{𣠞}{10904}
\saveTG{雬}{10904}
\saveTG{𠁃}{10905}
\saveTG{𢆚}{10905}
\saveTG{𦓩}{10905}
\saveTG{𠁋}{10905}
\saveTG{𩂴}{10905}
\saveTG{𡘸}{10905}
\saveTG{𩂺}{10906}
\saveTG{𥓜}{10908}
\saveTG{𫇏}{10908}
\saveTG{𩃲}{10909}
\saveTG{沗}{10909}
\saveTG{𩃆}{10912}
\saveTG{𣒳}{10914}
\saveTG{𣠁}{10915}
\saveTG{䊲}{10915}
\saveTG{𥜆}{10917}
\saveTG{𩄃}{10917}
\saveTG{䨛}{10921}
\saveTG{霦}{10922}
\saveTG{𩆽}{10927}
\saveTG{𩆲}{10927}
\saveTG{𩃻}{10927}
\saveTG{𩄵}{10927}
\saveTG{𩃭}{10931}
\saveTG{𩅫}{10932}
\saveTG{𫕱}{10941}
\saveTG{𣑘}{10941}
\saveTG{䅇}{10942}
\saveTG{𩅤}{10942}
\saveTG{𣑨}{10944}
\saveTG{䂞}{10946}
\saveTG{𩄧}{10948}
\saveTG{𩃝}{10948}
\saveTG{𩂮}{10949}
\saveTG{𩇏}{10953}
\saveTG{𩄭}{10961}
\saveTG{霜}{10961}
\saveTG{𩄶}{10982}
\saveTG{𥜟}{10986}
\saveTG{𤧦}{10989}
\saveTG{𩄍}{10989}
\saveTG{霖}{10994}
\saveTG{𤾓}{10994}
\saveTG{𩄞}{10994}
\saveTG{𠄻}{10994}
\saveTG{韭}{11101}
\saveTG{𫉶}{11102}
\saveTG{㐀}{11102}
\saveTG{𠀌}{11102}
\saveTG{瓕}{11103}
\saveTG{㻗}{11104}
\saveTG{𤥷}{11104}
\saveTG{𡍳}{11104}
\saveTG{㻹}{11104}
\saveTG{𡌦}{11104}
\saveTG{𤦅}{11104}
\saveTG{𡏵}{11104}
\saveTG{𤤘}{11104}
\saveTG{𡓭}{11104}
\saveTG{𦍮}{11104}
\saveTG{𦍟}{11107}
\saveTG{}{11107}
\saveTG{𧰓}{11108}
\saveTG{𧰛}{11108}
\saveTG{𧄼}{11108}
\saveTG{𨨖}{11109}
\saveTG{𨪱}{11109}
\saveTG{瓏}{11111}
\saveTG{非}{11111}
\saveTG{琲}{11111}
\saveTG{𤩩}{11111}
\saveTG{𩇽}{11111}
\saveTG{𩇦}{11111}
\saveTG{𤤃}{11111}
\saveTG{瓑}{11111}
\saveTG{玒}{11112}
\saveTG{玩}{11112}
\saveTG{瓐}{11112}
\saveTG{豇}{11112}
\saveTG{羾}{11112}
\saveTG{㺿}{11112}
\saveTG{𤦩}{11112}
\saveTG{𤨛}{11112}
\saveTG{𤼸}{11112}
\saveTG{㠭}{11112}
\saveTG{𧯬}{11112}
\saveTG{臦}{11112}
\saveTG{珏}{11113}
\saveTG{斑}{11114}
\saveTG{臸}{11114}
\saveTG{玨}{11114}
\saveTG{𧰁}{11114}
\saveTG{𨌥}{11114}
\saveTG{𤤻}{11114}
\saveTG{𤦦}{11114}
\saveTG{𪻮}{11114}
\saveTG{𤦐}{11114}
\saveTG{𤧕}{11114}
\saveTG{𤦵}{11114}
\saveTG{𤤴}{11114}
\saveTG{𤥇}{11114}
\saveTG{𤪾}{11114}
\saveTG{班}{11114}
\saveTG{𪻵}{11115}
\saveTG{𪼡}{11116}
\saveTG{𪻘}{11116}
\saveTG{疆}{11116}
\saveTG{虣}{11117}
\saveTG{珁}{11117}
\saveTG{瓨}{11117}
\saveTG{琥}{11117}
\saveTG{甄}{11117}
\saveTG{㤍}{11117}
\saveTG{𤭎}{11117}
\saveTG{𤮹}{11117}
\saveTG{𡿩}{11117}
\saveTG{䎎}{11117}
\saveTG{𤩭}{11117}
\saveTG{𤫟}{11117}
\saveTG{𤤞}{11117}
\saveTG{𤤍}{11117}
\saveTG{𩇧}{11117}
\saveTG{𡔓}{11117}
\saveTG{㼧}{11117}
\saveTG{𤫣}{11117}
\saveTG{𤭓}{11117}
\saveTG{𤭾}{11117}
\saveTG{㽅}{11117}
\saveTG{𪻡}{11118}
\saveTG{𤫩}{11118}
\saveTG{㺽}{11119}
\saveTG{𤥗}{11120}
\saveTG{𤫄}{11120}
\saveTG{玎}{11120}
\saveTG{珂}{11120}
\saveTG{𧯫}{11120}
\saveTG{𤤾}{11121}
\saveTG{珩}{11121}
\saveTG{𤦷}{11121}
\saveTG{㻉}{11122}
\saveTG{㺮}{11127}
\saveTG{㻬}{11127}
\saveTG{𤩡}{11127}
\saveTG{𤤝}{11127}
\saveTG{𤫗}{11127}
\saveTG{㺰}{11127}
\saveTG{𥒿}{11127}
\saveTG{𡋫}{11127}
\saveTG{瑪}{11127}
\saveTG{𤪲}{11127}
\saveTG{𤫦}{11127}
\saveTG{𦒃}{11127}
\saveTG{𤩋}{11127}
\saveTG{翡}{11127}
\saveTG{巧}{11127}
\saveTG{瓀}{11127}
\saveTG{瑡}{11127}
\saveTG{玙}{11127}
\saveTG{𤪙}{11127}
\saveTG{𦒤}{11127}
\saveTG{𩈂}{11131}
\saveTG{𤩙}{11131}
\saveTG{𧖑}{11131}
\saveTG{𤤈}{11131}
\saveTG{𪴹}{11131}
\saveTG{𤥨}{11132}
\saveTG{璩}{11132}
\saveTG{琢}{11132}
\saveTG{𤪴}{11134}
\saveTG{璱}{11134}
\saveTG{𧌩}{11136}
\saveTG{𧕿}{11136}
\saveTG{𧓐}{11136}
\saveTG{𧊪}{11136}
\saveTG{𧕜}{11136}
\saveTG{蜚}{11136}
\saveTG{蝅}{11136}
\saveTG{蠶}{11136}
\saveTG{蠺}{11136}
\saveTG{𧓊}{11136}
\saveTG{𧕽}{11136}
\saveTG{𩢳}{11137}
\saveTG{珥}{11140}
\saveTG{玗}{11140}
\saveTG{𪼗}{11140}
\saveTG{𦏹}{11140}
\saveTG{玡}{11140}
\saveTG{玕}{11140}
\saveTG{𤨹}{11141}
\saveTG{𤧶}{11142}
\saveTG{𤫕}{11142}
\saveTG{𪼅}{11142}
\saveTG{𤧋}{11142}
\saveTG{𤧄}{11144}
\saveTG{𤣿}{11144}
\saveTG{𪻠}{11144}
\saveTG{𤩁}{11145}
\saveTG{𤥆}{11145}
\saveTG{㻼}{11146}
\saveTG{琸}{11146}
\saveTG{𦥍}{11147}
\saveTG{𪻶}{11147}
\saveTG{㺳}{11147}
\saveTG{𢾲}{11147}
\saveTG{𢿤}{11147}
\saveTG{𦤺}{11147}
\saveTG{㪪}{11147}
\saveTG{䜴}{11147}
\saveTG{𤥙}{11147}
\saveTG{瓇}{11147}
\saveTG{𤫑}{11147}
\saveTG{𤫘}{11147}
\saveTG{㻯}{11149}
\saveTG{玶}{11149}
\saveTG{𤨇}{11152}
\saveTG{𡿑}{11157}
\saveTG{玷}{11160}
\saveTG{瑨}{11161}
\saveTG{𤩾}{11161}
\saveTG{𤨁}{11161}
\saveTG{珸}{11161}
\saveTG{𤤿}{11162}
\saveTG{𣶊}{11162}
\saveTG{璢}{11162}
\saveTG{𤤟}{11162}
\saveTG{𤥒}{11164}
\saveTG{𤫢}{11164}
\saveTG{𪤠}{11164}
\saveTG{𦑞}{11166}
\saveTG{𪼧}{11167}
\saveTG{㻸}{11167}
\saveTG{𤪽}{11168}
\saveTG{𤪻}{11168}
\saveTG{璿}{11168}
\saveTG{𤩅}{11168}
\saveTG{𤪷}{11172}
\saveTG{𦑣}{11177}
\saveTG{𤦣}{11181}
\saveTG{顼}{11182}
\saveTG{项}{11182}
\saveTG{颈}{11182}
\saveTG{𤥩}{11182}
\saveTG{𤤇}{11184}
\saveTG{瑌}{11184}
\saveTG{𤮵}{11186}
\saveTG{𩕙}{11186}
\saveTG{䫍}{11186}
\saveTG{𩒞}{11186}
\saveTG{𫖨}{11186}
\saveTG{䫦}{11186}
\saveTG{𩓩}{11186}
\saveTG{𣦓}{11186}
\saveTG{項}{11186}
\saveTG{𩒢}{11186}
\saveTG{頭}{11186}
\saveTG{頸}{11186}
\saveTG{頙}{11186}
\saveTG{頨}{11186}
\saveTG{𩑠}{11186}
\saveTG{𩑧}{11186}
\saveTG{𤃡}{11186}
\saveTG{𩔮}{11186}
\saveTG{𩒤}{11186}
\saveTG{䫷}{11186}
\saveTG{䪹}{11186}
\saveTG{𤦹}{11186}
\saveTG{頊}{11186}
\saveTG{𦤹}{11190}
\saveTG{环}{11190}
\saveTG{𢑽}{11191}
\saveTG{𤨧}{11191}
\saveTG{𤩰}{11194}
\saveTG{𪼬}{11194}
\saveTG{䥸}{11194}
\saveTG{𨯞}{11194}
\saveTG{璖}{11194}
\saveTG{瑮}{11194}
\saveTG{𤩗}{11194}
\saveTG{𪼆}{11196}
\saveTG{𪻴}{11202}
\saveTG{𢒣}{11202}
\saveTG{琴}{11207}
\saveTG{𪚗}{11211}
\saveTG{𧲖}{11211}
\saveTG{㒬}{11211}
\saveTG{尫}{11211}
\saveTG{𧱠}{11211}
\saveTG{𩇲}{11212}
\saveTG{𩰹}{11212}
\saveTG{㒮}{11212}
\saveTG{𧟦}{11212}
\saveTG{𥍢}{11212}
\saveTG{𧱖}{11212}
\saveTG{𢏠}{11212}
\saveTG{弫}{11212}
\saveTG{弬}{11212}
\saveTG{麗}{11212}
\saveTG{殌}{11212}
\saveTG{弳}{11212}
\saveTG{麉}{11212}
\saveTG{𫌕}{11212}
\saveTG{㱹}{11212}
\saveTG{𢐸}{11212}
\saveTG{𢎸}{11212}
\saveTG{𣥠}{11212}
\saveTG{𣧐}{11212}
\saveTG{𩐁}{11212}
\saveTG{𩖎}{11212}
\saveTG{㱺}{11212}
\saveTG{豗}{11213}
\saveTG{𢑃}{11214}
\saveTG{𣨂}{11214}
\saveTG{𥍻}{11214}
\saveTG{㢾}{11214}
\saveTG{𥜬}{11214}
\saveTG{}{11214}
\saveTG{𨾢}{11215}
\saveTG{雈}{11215}
\saveTG{𢑆}{11215}
\saveTG{𧱂}{11216}
\saveTG{𩰵}{11216}
\saveTG{殭}{11216}
\saveTG{𡯝}{11216}
\saveTG{𠒓}{11216}
\saveTG{彊}{11216}
\saveTG{彄}{11216}
\saveTG{𧇘}{11217}
\saveTG{𪋘}{11217}
\saveTG{𢏯}{11217}
\saveTG{䶬}{11217}
\saveTG{䰛}{11217}
\saveTG{㼗}{11217}
\saveTG{𢐝}{11217}
\saveTG{𢏍}{11217}
\saveTG{弡}{11217}
\saveTG{𤭃}{11217}
\saveTG{𢏉}{11218}
\saveTG{𥎛}{11218}
\saveTG{𢬫}{11220}
\saveTG{𧰩}{11220}
\saveTG{𠀩}{11220}
\saveTG{卝}{11220}
\saveTG{顨}{11221}
\saveTG{彁}{11221}
\saveTG{戼}{11221}
\saveTG{𩇪}{11221}
\saveTG{𥎘}{11221}
\saveTG{𣧤}{11226}
\saveTG{𩇱}{11227}
\saveTG{𣩙}{11227}
\saveTG{𪱟}{11227}
\saveTG{𣧁}{11227}
\saveTG{𤩟}{11227}
\saveTG{𦠮}{11227}
\saveTG{𩇷}{11227}
\saveTG{𪪻}{11227}
\saveTG{㣂}{11227}
\saveTG{𨪖}{11227}
\saveTG{𣧰}{11227}
\saveTG{㱛}{11227}
\saveTG{㱙}{11227}
\saveTG{𨳇}{11227}
\saveTG{𠀙}{11227}
\saveTG{𩰴}{11227}
\saveTG{𦓠}{11227}
\saveTG{𩱇}{11227}
\saveTG{𠁬}{11227}
\saveTG{𤰈}{11227}
\saveTG{𢑋}{11227}
\saveTG{𢅓}{11227}
\saveTG{𠄥}{11227}
\saveTG{𫉾}{11227}
\saveTG{𤰊}{11227}
\saveTG{𦝫}{11227}
\saveTG{𦇂}{11227}
\saveTG{𢃻}{11227}
\saveTG{𨴚}{11227}
\saveTG{𧱽}{11227}
\saveTG{㢪}{11227}
\saveTG{𢐰}{11227}
\saveTG{𢁓}{11227}
\saveTG{𫉐}{11227}
\saveTG{𩇴}{11227}
\saveTG{𥎖}{11227}
\saveTG{彌}{11227}
\saveTG{鬵}{11227}
\saveTG{𦙼}{11227}
\saveTG{𠞢}{11227}
\saveTG{𤦠}{11228}
\saveTG{臩}{11228}
\saveTG{𠧥}{11230}
\saveTG{𣩤}{11231}
\saveTG{𢏁}{11231}
\saveTG{豩}{11232}
\saveTG{張}{11232}
\saveTG{𧲎}{11232}
\saveTG{𧲏}{11232}
\saveTG{𤦗}{11232}
\saveTG{𢏾}{11232}
\saveTG{𥎔}{11236}
\saveTG{弙}{11240}
\saveTG{𣦿}{11240}
\saveTG{㢨}{11240}
\saveTG{豜}{11240}
\saveTG{豣}{11240}
\saveTG{弭}{11240}
\saveTG{𤘆}{11240}
\saveTG{䝕}{11241}
\saveTG{𣧹}{11242}
\saveTG{𡭃}{11243}
\saveTG{𢐖}{11244}
\saveTG{𧱩}{11245}
\saveTG{𤽩}{11246}
\saveTG{𠭼}{11247}
\saveTG{𧱰}{11247}
\saveTG{𢾠}{11247}
\saveTG{𣀺}{11247}
\saveTG{㪌}{11247}
\saveTG{𢾿}{11247}
\saveTG{𣀷}{11247}
\saveTG{𣧏}{11247}
\saveTG{𣀑}{11247}
\saveTG{㪬}{11247}
\saveTG{𢼿}{11247}
\saveTG{𢽪}{11247}
\saveTG{㢭}{11247}
\saveTG{𣧒}{11247}
\saveTG{𦲘}{11247}
\saveTG{𢏊}{11249}
\saveTG{𢧻}{11252}
\saveTG{𣩪}{11252}
\saveTG{𦹋}{11253}
\saveTG{𢧖}{11253}
\saveTG{𠁖}{11256}
\saveTG{𥎑}{11261}
\saveTG{𧲙}{11261}
\saveTG{㣅}{11261}
\saveTG{𤘇}{11262}
\saveTG{𣩚}{11262}
\saveTG{㢶}{11262}
\saveTG{𢐈}{11262}
\saveTG{𢐤}{11262}
\saveTG{𢐀}{11264}
\saveTG{𢐜}{11264}
\saveTG{𧁌}{11264}
\saveTG{𪯝}{11264}
\saveTG{𢐡}{11264}
\saveTG{𧲍}{11272}
\saveTG{𣨇}{11277}
\saveTG{𢑇}{11277}
\saveTG{㱜}{11282}
\saveTG{顶}{11282}
\saveTG{顽}{11282}
\saveTG{颥}{11282}
\saveTG{预}{11282}
\saveTG{𪉝}{11282}
\saveTG{𢎾}{11284}
\saveTG{𩱄}{11284}
\saveTG{䫱}{11286}
\saveTG{𩑕}{11286}
\saveTG{𩒐}{11286}
\saveTG{𣪀}{11286}
\saveTG{頂}{11286}
\saveTG{頑}{11286}
\saveTG{顟}{11286}
\saveTG{顬}{11286}
\saveTG{預}{11286}
\saveTG{𥎀}{11286}
\saveTG{䪵}{11286}
\saveTG{𧶃}{11286}
\saveTG{𩒼}{11286}
\saveTG{䪲}{11286}
\saveTG{𩱊}{11289}
\saveTG{𢐕}{11291}
\saveTG{㣄}{11294}
\saveTG{豲}{11296}
\saveTG{𤧂}{11302}
\saveTG{𢐘}{11321}
\saveTG{𩧕}{11327}
\saveTG{𪂫}{11327}
\saveTG{𪂏}{11327}
\saveTG{䲾}{11327}
\saveTG{𢡵}{11331}
\saveTG{𢜈}{11331}
\saveTG{𢣓}{11331}
\saveTG{悲}{11331}
\saveTG{𤨝}{11334}
\saveTG{瑟}{11334}
\saveTG{𢚸}{11334}
\saveTG{𩻊}{11336}
\saveTG{𩺁}{11336}
\saveTG{𩼭}{11336}
\saveTG{𩇻}{11337}
\saveTG{𤍟}{11339}
\saveTG{𢾧}{11347}
\saveTG{䫮}{11386}
\saveTG{斐}{11400}
\saveTG{𪯥}{11400}
\saveTG{𢌛}{11401}
\saveTG{𪪲}{11401}
\saveTG{𠦞}{11402}
\saveTG{𪪬}{11404}
\saveTG{婯}{11404}
\saveTG{婓}{11404}
\saveTG{廼}{11406}
\saveTG{𢌨}{11406}
\saveTG{𢌫}{11406}
\saveTG{𡖂}{11407}
\saveTG{蘷}{11407}
\saveTG{𩇫}{11407}
\saveTG{耻}{11410}
\saveTG{聇}{11411}
\saveTG{𪚒}{11411}
\saveTG{䏊}{11411}
\saveTG{𦖕}{11411}
\saveTG{𩇳}{11412}
\saveTG{𦗮}{11412}
\saveTG{𦘊}{11412}
\saveTG{兓}{11412}
\saveTG{孲}{11412}
\saveTG{𦘐}{11412}
\saveTG{𦔸}{11412}
\saveTG{𢌩}{11414}
\saveTG{𦖧}{11414}
\saveTG{𤳾}{11416}
\saveTG{𦖖}{11417}
\saveTG{𤮱}{11417}
\saveTG{㼞}{11417}
\saveTG{㼛}{11417}
\saveTG{𡥚}{11420}
\saveTG{𦕼}{11420}
\saveTG{耵}{11420}
\saveTG{𢆊}{11420}
\saveTG{𡦘}{11421}
\saveTG{聏}{11427}
\saveTG{𦘎}{11427}
\saveTG{𠣃}{11427}
\saveTG{𫆋}{11427}
\saveTG{𤨍}{11427}
\saveTG{𩡵}{11427}
\saveTG{𦔲}{11427}
\saveTG{孺}{11427}
\saveTG{𩙻}{11431}
\saveTG{𦗂}{11432}
\saveTG{耺}{11432}
\saveTG{𦕷}{11432}
\saveTG{𧰪}{11432}
\saveTG{聑}{11440}
\saveTG{幵}{11440}
\saveTG{𢏘}{11440}
\saveTG{𪚟}{11440}
\saveTG{𦕆}{11440}
\saveTG{𦖩}{11440}
\saveTG{𢍴}{11441}
\saveTG{𢆛}{11441}
\saveTG{㜷}{11442}
\saveTG{𡠈}{11442}
\saveTG{㚽}{11442}
\saveTG{𢍳}{11446}
\saveTG{𡡜}{11446}
\saveTG{𦗡}{11446}
\saveTG{𢽄}{11447}
\saveTG{𣀄}{11447}
\saveTG{𢽖}{11447}
\saveTG{𦔼}{11447}
\saveTG{𢽁}{11447}
\saveTG{𡦌}{11461}
\saveTG{䎸}{11461}
\saveTG{𢌧}{11461}
\saveTG{𦕒}{11462}
\saveTG{𦕩}{11464}
\saveTG{𨠧}{11464}
\saveTG{𡥢}{11472}
\saveTG{顸}{11482}
\saveTG{颞}{11482}
\saveTG{颒}{11482}
\saveTG{颋}{11482}
\saveTG{𦔿}{11484}
\saveTG{頇}{11486}
\saveTG{𩒸}{11486}
\saveTG{𩑓}{11486}
\saveTG{𩒖}{11486}
\saveTG{𩔘}{11486}
\saveTG{𩑬}{11486}
\saveTG{頲}{11486}
\saveTG{顳}{11486}
\saveTG{頮}{11486}
\saveTG{𩑳}{11486}
\saveTG{𦗩}{11489}
\saveTG{𦕝}{11491}
\saveTG{䏇}{11491}
\saveTG{㹃}{11501}
\saveTG{𦍋}{11502}
\saveTG{𩇮}{11502}
\saveTG{芈}{11502}
\saveTG{羋}{11502}
\saveTG{揅}{11502}
\saveTG{𤧆}{11503}
\saveTG{辈}{11504}
\saveTG{𩍻}{11506}
\saveTG{輩}{11506}
\saveTG{㹕}{11508}
\saveTG{𦸑}{11537}
\saveTG{𩑫}{11586}
\saveTG{䪳}{11586}
\saveTG{𥐫}{11600}
\saveTG{䛒}{11601}
\saveTG{𥇖}{11601}
\saveTG{𩇬}{11601}
\saveTG{𩇸}{11601}
\saveTG{㬜}{11601}
\saveTG{𥍊}{11601}
\saveTG{𧃼}{11601}
\saveTG{𣊔}{11601}
\saveTG{朁}{11601}
\saveTG{酽}{11601}
\saveTG{𥄕}{11602}
\saveTG{𦤄}{11602}
\saveTG{𥕗}{11602}
\saveTG{𨣖}{11602}
\saveTG{𥅝}{11604}
\saveTG{𩇵}{11604}
\saveTG{𨣾}{11604}
\saveTG{𥋚}{11605}
\saveTG{砋}{11610}
\saveTG{𥐵}{11611}
\saveTG{礰}{11611}
\saveTG{礲}{11611}
\saveTG{𪿨}{11611}
\saveTG{𥖋}{11612}
\saveTG{𥓚}{11612}
\saveTG{𥕰}{11612}
\saveTG{𥘃}{11612}
\saveTG{𨠣}{11612}
\saveTG{䣿}{11612}
\saveTG{𨣷}{11612}
\saveTG{砈}{11612}
\saveTG{𩈡}{11612}
\saveTG{𨣙}{11612}
\saveTG{𥔦}{11612}
\saveTG{𥑗}{11612}
\saveTG{𨠂}{11612}
\saveTG{𥔂}{11612}
\saveTG{𥑣}{11612}
\saveTG{𨡖}{11612}
\saveTG{酛}{11612}
\saveTG{矹}{11612}
\saveTG{翫}{11612}
\saveTG{砸}{11612}
\saveTG{硜}{11612}
\saveTG{矼}{11612}
\saveTG{䃁}{11612}
\saveTG{𥐳}{11612}
\saveTG{𥓟}{11612}
\saveTG{𨠸}{11612}
\saveTG{𨤍}{11612}
\saveTG{𥕑}{11612}
\saveTG{釃}{11612}
\saveTG{砡}{11613}
\saveTG{䃌}{11614}
\saveTG{𥐰}{11614}
\saveTG{𥒓}{11614}
\saveTG{𣍒}{11614}
\saveTG{𥔌}{11614}
\saveTG{𥗃}{11615}
\saveTG{礭}{11615}
\saveTG{𥒑}{11615}
\saveTG{𥕠}{11615}
\saveTG{礓}{11616}
\saveTG{𥕥}{11616}
\saveTG{𥕇}{11616}
\saveTG{醧}{11616}
\saveTG{砙}{11617}
\saveTG{𤮷}{11617}
\saveTG{𥑉}{11617}
\saveTG{𥑭}{11617}
\saveTG{䣰}{11617}
\saveTG{𥖬}{11617}
\saveTG{𩈎}{11617}
\saveTG{𪿔}{11617}
\saveTG{𨣬}{11617}
\saveTG{𨠛}{11617}
\saveTG{𤭻}{11617}
\saveTG{𣄹}{11617}
\saveTG{𤭑}{11617}
\saveTG{㽌}{11617}
\saveTG{㼣}{11617}
\saveTG{𤮚}{11617}
\saveTG{𤮸}{11617}
\saveTG{𫑶}{11618}
\saveTG{𥗅}{11618}
\saveTG{𨠙}{11619}
\saveTG{𥑜}{11619}
\saveTG{𩈔}{11620}
\saveTG{酠}{11620}
\saveTG{砢}{11620}
\saveTG{矴}{11620}
\saveTG{酊}{11620}
\saveTG{𨢾}{11621}
\saveTG{𥐡}{11621}
\saveTG{𥒼}{11622}
\saveTG{𥊄}{11627}
\saveTG{𨢌}{11627}
\saveTG{𨟺}{11627}
\saveTG{𥕷}{11627}
\saveTG{𪿒}{11627}
\saveTG{䃻}{11627}
\saveTG{𩤤}{11627}
\saveTG{}{11627}
\saveTG{磭}{11627}
\saveTG{茍}{11627}
\saveTG{砺}{11627}
\saveTG{礪}{11627}
\saveTG{碼}{11627}
\saveTG{醹}{11627}
\saveTG{礝}{11627}
\saveTG{酾}{11627}
\saveTG{𡦀}{11627}
\saveTG{䃒}{11627}
\saveTG{𥓠}{11627}
\saveTG{𫑺}{11627}
\saveTG{𥖅}{11631}
\saveTG{𨠀}{11631}
\saveTG{𥐯}{11631}
\saveTG{醵}{11632}
\saveTG{𥗗}{11632}
\saveTG{䂻}{11632}
\saveTG{𥗾}{11632}
\saveTG{𪿯}{11632}
\saveTG{硺}{11632}
\saveTG{𥐺}{11632}
\saveTG{𥖟}{11632}
\saveTG{酝}{11632}
\saveTG{𩈛}{11632}
\saveTG{𥖼}{11636}
\saveTG{𥔕}{11636}
\saveTG{𥔮}{11640}
\saveTG{𢆖}{11640}
\saveTG{𩈅}{11640}
\saveTG{𥒀}{11640}
\saveTG{酐}{11640}
\saveTG{矸}{11640}
\saveTG{研}{11640}
\saveTG{酑}{11640}
\saveTG{砑}{11640}
\saveTG{硏}{11640}
\saveTG{𥒔}{11641}
\saveTG{𥔈}{11641}
\saveTG{硦}{11641}
\saveTG{䣵}{11642}
\saveTG{𡭋}{11643}
\saveTG{𥔖}{11644}
\saveTG{𨠴}{11644}
\saveTG{䂽}{11646}
\saveTG{磹}{11646}
\saveTG{硬}{11646}
\saveTG{醰}{11646}
\saveTG{㪮}{11647}
\saveTG{𣀀}{11647}
\saveTG{𢾇}{11647}
\saveTG{𢾵}{11647}
\saveTG{𢿟}{11647}
\saveTG{㪃}{11647}
\saveTG{㪊}{11647}
\saveTG{𥖪}{11647}
\saveTG{𢿩}{11647}
\saveTG{𥔹}{11647}
\saveTG{𥌱}{11647}
\saveTG{𨠃}{11647}
\saveTG{𨠟}{11649}
\saveTG{砰}{11649}
\saveTG{𥕕}{11649}
\saveTG{𪿪}{11650}
\saveTG{䤏}{11652}
\saveTG{𨣱}{11653}
\saveTG{𧂝}{11653}
\saveTG{𨣌}{11656}
\saveTG{硵}{11660}
\saveTG{砧}{11660}
\saveTG{磠}{11660}
\saveTG{酟}{11660}
\saveTG{𥕉}{11661}
\saveTG{礌}{11661}
\saveTG{醽}{11661}
\saveTG{礵}{11661}
\saveTG{䨻}{11661}
\saveTG{𥒾}{11661}
\saveTG{䃡}{11661}
\saveTG{𥔨}{11662}
\saveTG{𩈳}{11662}
\saveTG{䩇}{11662}
\saveTG{𣗁}{11662}
\saveTG{𥗉}{11662}
\saveTG{𥗐}{11662}
\saveTG{𩉖}{11662}
\saveTG{砳}{11662}
\saveTG{䤄}{11662}
\saveTG{皕}{11662}
\saveTG{𨡞}{11662}
\saveTG{𥕝}{11662}
\saveTG{硒}{11664}
\saveTG{𦧥}{11664}
\saveTG{𦧚}{11664}
\saveTG{𦧵}{11664}
\saveTG{𠠦}{11666}
\saveTG{疈}{11666}
\saveTG{㽬}{11666}
\saveTG{𥔁}{11666}
\saveTG{𨤀}{11666}
\saveTG{䤐}{11667}
\saveTG{𥗊}{11672}
\saveTG{𥓾}{11677}
\saveTG{𣣧}{11682}
\saveTG{𥓢}{11682}
\saveTG{䂵}{11682}
\saveTG{𥕲}{11682}
\saveTG{𫖳}{11682}
\saveTG{硕}{11682}
\saveTG{𩇼}{11684}
\saveTG{𩅖}{11684}
\saveTG{碝}{11684}
\saveTG{頵}{11686}
\saveTG{䫊}{11686}
\saveTG{𥊾}{11686}
\saveTG{𩕸}{11686}
\saveTG{𩒾}{11686}
\saveTG{𨢣}{11686}
\saveTG{𩑸}{11686}
\saveTG{碽}{11686}
\saveTG{碩}{11686}
\saveTG{䫬}{11686}
\saveTG{𩔞}{11686}
\saveTG{𩓑}{11686}
\saveTG{碵}{11686}
\saveTG{𩔁}{11686}
\saveTG{𩖊}{11686}
\saveTG{𥖿}{11686}
\saveTG{𥗳}{11686}
\saveTG{𨟷}{11690}
\saveTG{𥐴}{11690}
\saveTG{}{11691}
\saveTG{磦}{11691}
\saveTG{醥}{11691}
\saveTG{𩈀}{11694}
\saveTG{磲}{11694}
\saveTG{𥓼}{11694}
\saveTG{甅}{11711}
\saveTG{㐟}{11711}
\saveTG{𪼷}{11711}
\saveTG{𪓯}{11712}
\saveTG{𦓓}{11712}
\saveTG{琵}{11712}
\saveTG{𤮑}{11713}
\saveTG{𧰦}{11713}
\saveTG{𤭖}{11713}
\saveTG{靟}{11714}
\saveTG{䩁}{11715}
\saveTG{瓸}{11716}
\saveTG{𤭗}{11717}
\saveTG{𨛬}{11717}
\saveTG{𩇯}{11717}
\saveTG{䨽}{11717}
\saveTG{𪓔}{11717}
\saveTG{𤧲}{11717}
\saveTG{𤬮}{11717}
\saveTG{琶}{11717}
\saveTG{𤧰}{11717}
\saveTG{𧝣}{11723}
\saveTG{𢏗}{11727}
\saveTG{𢐯}{11727}
\saveTG{𤣫}{11727}
\saveTG{𤮄}{11727}
\saveTG{𢐓}{11727}
\saveTG{𩇔}{11731}
\saveTG{餥}{11732}
\saveTG{裴}{11732}
\saveTG{𩜕}{11732}
\saveTG{𧞙}{11732}
\saveTG{𧞕}{11732}
\saveTG{𩛬}{11736}
\saveTG{𢻽}{11747}
\saveTG{𢾸}{11747}
\saveTG{𡵾}{11772}
\saveTG{𡵽}{11772}
\saveTG{𠙿}{11772}
\saveTG{𡶪}{11772}
\saveTG{𤕁}{11777}
\saveTG{戼}{11777}
\saveTG{䪱}{11786}
\saveTG{𩔓}{11786}
\saveTG{𩒔}{11786}
\saveTG{顄}{11786}
\saveTG{珡}{11801}
\saveTG{𥤾}{11802}
\saveTG{𧱾}{11804}
\saveTG{奜}{11804}
\saveTG{猆}{11804}
\saveTG{𧶻}{11806}
\saveTG{𧵭}{11806}
\saveTG{𩕧}{11806}
\saveTG{𤦡}{11807}
\saveTG{羐}{11807}
\saveTG{𦍚}{11809}
\saveTG{𩇭}{11809}
\saveTG{𤐱}{11809}
\saveTG{燹}{11809}
\saveTG{燛}{11809}
\saveTG{㼲}{11817}
\saveTG{𤮁}{11817}
\saveTG{𠤜}{11817}
\saveTG{𨅪}{11841}
\saveTG{䩀}{11861}
\saveTG{𧷩}{11861}
\saveTG{𧹁}{11861}
\saveTG{𧷋}{11864}
\saveTG{𡚌}{11884}
\saveTG{𩔊}{11886}
\saveTG{𩒏}{11886}
\saveTG{𩖈}{11886}
\saveTG{𤋂}{11896}
\saveTG{𫉿}{11901}
\saveTG{𣓢}{11904}
\saveTG{琹}{11904}
\saveTG{𣗜}{11904}
\saveTG{𤦲}{11904}
\saveTG{𣕓}{11904}
\saveTG{𣓁}{11904}
\saveTG{棐}{11904}
\saveTG{䊙}{11904}
\saveTG{栞}{11904}
\saveTG{𣝚}{11904}
\saveTG{𤩘}{11905}
\saveTG{𩇹}{11905}
\saveTG{𢆞}{11905}
\saveTG{𥜠}{11912}
\saveTG{𥜦}{11914}
\saveTG{𤬭}{11917}
\saveTG{𣙬}{11917}
\saveTG{㼼}{11917}
\saveTG{𥜸}{11926}
\saveTG{𧟫}{11926}
\saveTG{𫊃}{11927}
\saveTG{𢿏}{11947}
\saveTG{𥜹}{11947}
\saveTG{𥜯}{11947}
\saveTG{𢼗}{11947}
\saveTG{𥺟}{11949}
\saveTG{𥎁}{11962}
\saveTG{𥜧}{11963}
\saveTG{𩒴}{11986}
\saveTG{𩔛}{11986}
\saveTG{𩑢}{11986}
\saveTG{𩒜}{11986}
\saveTG{顠}{11986}
\saveTG{𧲟}{11991}
\saveTG{祘}{11991}
\saveTG{𩰊}{12000}
\saveTG{飞}{12013}
\saveTG{刯}{12100}
\saveTG{刭}{12100}
\saveTG{剄}{12100}
\saveTG{𠜉}{12100}
\saveTG{𠛙}{12100}
\saveTG{𠚧}{12100}
\saveTG{𠠱}{12100}
\saveTG{𠞯}{12100}
\saveTG{㻝}{12100}
\saveTG{𤥅}{12100}
\saveTG{𠜲}{12100}
\saveTG{𠝙}{12100}
\saveTG{𠞐}{12100}
\saveTG{𤤫}{12100}
\saveTG{㺫}{12100}
\saveTG{㺩}{12100}
\saveTG{到}{12100}
\saveTG{琍}{12100}
\saveTG{𢀙}{12100}
\saveTG{㔁}{12100}
\saveTG{𠀓}{12100}
\saveTG{玔}{12100}
\saveTG{㓚}{12100}
\saveTG{剅}{12100}
\saveTG{剕}{12100}
\saveTG{丠}{12101}
\saveTG{𥃄}{12102}
\saveTG{泴}{12102}
\saveTG{𠤢}{12102}
\saveTG{㘳}{12104}
\saveTG{𡌑}{12104}
\saveTG{𡋭}{12104}
\saveTG{𦥊}{12104}
\saveTG{𡒐}{12104}
\saveTG{𡏸}{12104}
\saveTG{坔}{12104}
\saveTG{型}{12104}
\saveTG{𣹛}{12104}
\saveTG{𫚓}{12106}
\saveTG{𣇇}{12106}
\saveTG{登}{12108}
\saveTG{𨰱}{12109}
\saveTG{𨥗}{12109}
\saveTG{銐}{12109}
\saveTG{𢀗}{12110}
\saveTG{北}{12110}
\saveTG{玼}{12110}
\saveTG{玭}{12110}
\saveTG{玌}{12110}
\saveTG{𠃖}{12110}
\saveTG{𤦄}{12111}
\saveTG{𪵮}{12111}
\saveTG{𥁟}{12112}
\saveTG{𣹉}{12112}
\saveTG{𤪩}{12112}
\saveTG{𤧏}{12112}
\saveTG{𦐭}{12113}
\saveTG{珧}{12113}
\saveTG{𢑮}{12113}
\saveTG{𩙱}{12113}
\saveTG{𠃧}{12113}
\saveTG{毭}{12114}
\saveTG{㻔}{12115}
\saveTG{璀}{12115}
\saveTG{㼃}{12116}
\saveTG{𦒦}{12116}
\saveTG{𤧜}{12117}
\saveTG{𡐜}{12117}
\saveTG{𣯰}{12117}
\saveTG{㲪}{12117}
\saveTG{𤨖}{12117}
\saveTG{𣬾}{12117}
\saveTG{𤩫}{12117}
\saveTG{𠃡}{12117}
\saveTG{𤨾}{12117}
\saveTG{𢀫}{12117}
\saveTG{𣴝}{12117}
\saveTG{𧯝}{12117}
\saveTG{𤣯}{12117}
\saveTG{璒}{12118}
\saveTG{𤃶}{12118}
\saveTG{𤧸}{12118}
\saveTG{𧰐}{12118}
\saveTG{𣃆}{12121}
\saveTG{𪻢}{12121}
\saveTG{𪻩}{12121}
\saveTG{𤤆}{12121}
\saveTG{𧯞}{12121}
\saveTG{𣂪}{12121}
\saveTG{𣂼}{12121}
\saveTG{㣉}{12122}
\saveTG{𤩐}{12122}
\saveTG{濷}{12127}
\saveTG{𦒓}{12127}
\saveTG{𤩝}{12127}
\saveTG{𤨈}{12127}
\saveTG{琇}{12127}
\saveTG{𩧮}{12127}
\saveTG{䴕}{12127}
\saveTG{}{12127}
\saveTG{㼇}{12127}
\saveTG{瑀}{12127}
\saveTG{瑞}{12127}
\saveTG{瓗}{12127}
\saveTG{琌}{12127}
\saveTG{𤣺}{12130}
\saveTG{㴇}{12130}
\saveTG{𧒈}{12131}
\saveTG{䗺}{12131}
\saveTG{𤪠}{12131}
\saveTG{㺨}{12131}
\saveTG{𤣵}{12131}
\saveTG{𧊿}{12135}
\saveTG{𧊢}{12136}
\saveTG{𧎕}{12136}
\saveTG{𧉥}{12136}
\saveTG{𤼭}{12136}
\saveTG{蜑}{12136}
\saveTG{𧑧}{12136}
\saveTG{𦕑}{12140}
\saveTG{𤣳}{12140}
\saveTG{𤥻}{12141}
\saveTG{𦐿}{12141}
\saveTG{珽}{12141}
\saveTG{珳}{12142}
\saveTG{𦐊}{12147}
\saveTG{𦑛}{12147}
\saveTG{𣱌}{12147}
\saveTG{𤪞}{12147}
\saveTG{𪼠}{12147}
\saveTG{𣼔}{12147}
\saveTG{𤤣}{12147}
\saveTG{璦}{12147}
\saveTG{㻑}{12147}
\saveTG{琈}{12147}
\saveTG{瑗}{12147}
\saveTG{翪}{12147}
\saveTG{瑷}{12147}
\saveTG{𤤵}{12150}
\saveTG{𣧙}{12150}
\saveTG{璣}{12153}
\saveTG{𪼇}{12154}
\saveTG{㼄}{12156}
\saveTG{𤪫}{12156}
\saveTG{𤩥}{12157}
\saveTG{𤤓}{12157}
\saveTG{㻈}{12161}
\saveTG{瑎}{12162}
\saveTG{瑙}{12162}
\saveTG{𤦊}{12163}
\saveTG{䎓}{12163}
\saveTG{琘}{12164}
\saveTG{𤩣}{12167}
\saveTG{䪤}{12169}
\saveTG{𤧘}{12169}
\saveTG{璠}{12169}
\saveTG{𡷆}{12170}
\saveTG{𪻬}{12172}
\saveTG{𤤼}{12172}
\saveTG{瑶}{12172}
\saveTG{瑫}{12177}
\saveTG{𪻤}{12181}
\saveTG{𤪵}{12182}
\saveTG{㻄}{12182}
\saveTG{㰯}{12182}
\saveTG{𤨊}{12184}
\saveTG{璞}{12185}
\saveTG{𤨏}{12186}
\saveTG{瓆}{12186}
\saveTG{𤨲}{12191}
\saveTG{𡑙}{12194}
\saveTG{𤧮}{12194}
\saveTG{𣡝}{12194}
\saveTG{𤥾}{12194}
\saveTG{瓅}{12194}
\saveTG{璅}{12194}
\saveTG{𤤤}{12194}
\saveTG{𡏫}{12194}
\saveTG{}{12194}
\saveTG{𤩶}{12195}
\saveTG{𤦋}{12197}
\saveTG{𠜝}{12200}
\saveTG{𦓎}{12200}
\saveTG{𠚵}{12200}
\saveTG{𠟺}{12200}
\saveTG{𠠰}{12200}
\saveTG{𠛥}{12200}
\saveTG{𠟵}{12200}
\saveTG{𣧿}{12200}
\saveTG{𠛖}{12200}
\saveTG{𣨸}{12200}
\saveTG{𠝃}{12200}
\saveTG{𠚥}{12200}
\saveTG{㔇}{12200}
\saveTG{𢏄}{12200}
\saveTG{剢}{12200}
\saveTG{劀}{12200}
\saveTG{剹}{12200}
\saveTG{列}{12200}
\saveTG{刓}{12200}
\saveTG{引}{12200}
\saveTG{𠠫}{12200}
\saveTG{𠠛}{12200}
\saveTG{𢎞}{12210}
\saveTG{𡰁}{12211}
\saveTG{兠}{12211}
\saveTG{𠚯}{12212}
\saveTG{𣧭}{12212}
\saveTG{𢒇}{12212}
\saveTG{𣩶}{12212}
\saveTG{𧱓}{12212}
\saveTG{𣨮}{12214}
\saveTG{氄}{12214}
\saveTG{𣰻}{12215}
\saveTG{𣭲}{12215}
\saveTG{𣰄}{12215}
\saveTG{𨾸}{12215}
\saveTG{𣩑}{12215}
\saveTG{𢏴}{12215}
\saveTG{𥍽}{12215}
\saveTG{𩅞}{12215}
\saveTG{𣭅}{12216}
\saveTG{䝓}{12216}
\saveTG{𣧇}{12217}
\saveTG{𩰷}{12217}
\saveTG{𪶂}{12217}
\saveTG{𠤥}{12217}
\saveTG{𠤦}{12217}
\saveTG{𥧐}{12217}
\saveTG{𪊠}{12217}
\saveTG{𤼫}{12217}
\saveTG{𢐋}{12217}
\saveTG{卍}{12217}
\saveTG{凳}{12217}
\saveTG{𠙩}{12217}
\saveTG{𣱻}{12217}
\saveTG{𣰿}{12217}
\saveTG{𤘑}{12218}
\saveTG{𣩟}{12218}
\saveTG{𥎇}{12218}
\saveTG{㱯}{12218}
\saveTG{𣨋}{12221}
\saveTG{𥍭}{12221}
\saveTG{㱤}{12221}
\saveTG{歽}{12221}
\saveTG{𣂕}{12221}
\saveTG{𣂡}{12221}
\saveTG{𣂘}{12221}
\saveTG{𣩠}{12221}
\saveTG{𪿆}{12221}
\saveTG{𢏂}{12221}
\saveTG{彲}{12222}
\saveTG{彨}{12222}
\saveTG{𢒴}{12222}
\saveTG{𥍪}{12222}
\saveTG{𦓘}{12222}
\saveTG{𢒔}{12222}
\saveTG{𢒓}{12222}
\saveTG{㱶}{12222}
\saveTG{耏}{12222}
\saveTG{𢎺}{12223}
\saveTG{𪵃}{12223}
\saveTG{𧲚}{12227}
\saveTG{𤂒}{12227}
\saveTG{𢂖}{12227}
\saveTG{㢷}{12227}
\saveTG{𢐃}{12227}
\saveTG{𢐟}{12227}
\saveTG{背}{12227}
\saveTG{𥍧}{12227}
\saveTG{𣨊}{12227}
\saveTG{𧲑}{12227}
\saveTG{䝎}{12227}
\saveTG{㡂}{12227}
\saveTG{𧲄}{12227}
\saveTG{𤘐}{12227}
\saveTG{𢐒}{12227}
\saveTG{𥙏}{12228}
\saveTG{𣷘}{12230}
\saveTG{𧩁}{12230}
\saveTG{弘}{12230}
\saveTG{𣸐}{12230}
\saveTG{弧}{12230}
\saveTG{𣹳}{12230}
\saveTG{𣻣}{12230}
\saveTG{弘}{12231}
\saveTG{𠚦}{12231}
\saveTG{𢐩}{12231}
\saveTG{㵘}{12232}
\saveTG{张}{12234}
\saveTG{歼}{12240}
\saveTG{弤}{12240}
\saveTG{𪪼}{12241}
\saveTG{𢏳}{12241}
\saveTG{𩱛}{12242}
\saveTG{𤪀}{12242}
\saveTG{𣨅}{12243}
\saveTG{㢻}{12244}
\saveTG{㱣}{12244}
\saveTG{𥍺}{12247}
\saveTG{𪵁}{12247}
\saveTG{𢏫}{12247}
\saveTG{𥀅}{12247}
\saveTG{𪫁}{12247}
\saveTG{𣧶}{12247}
\saveTG{鬷}{12247}
\saveTG{弢}{12247}
\saveTG{發}{12247}
\saveTG{𣧃}{12247}
\saveTG{殍}{12247}
\saveTG{𤼲}{12248}
\saveTG{𣧯}{12249}
\saveTG{𢏰}{12257}
\saveTG{𧫯}{12261}
\saveTG{𥍾}{12262}
\saveTG{𧱥}{12263}
\saveTG{𤼰}{12263}
\saveTG{𥍿}{12264}
\saveTG{𤘏}{12264}
\saveTG{殙}{12264}
\saveTG{𢐠}{12269}
\saveTG{𣧈}{12270}
\saveTG{㢫}{12270}
\saveTG{𣧑}{12270}
\saveTG{𥎐}{12272}
\saveTG{𣨡}{12272}
\saveTG{𣧽}{12272}
\saveTG{𣧪}{12272}
\saveTG{𧰹}{12272}
\saveTG{𤥍}{12282}
\saveTG{殀}{12284}
\saveTG{豯}{12284}
\saveTG{𤼵}{12284}
\saveTG{𤪟}{12285}
\saveTG{𧵜}{12286}
\saveTG{𩒆}{12286}
\saveTG{𧶢}{12286}
\saveTG{𢐏}{12287}
\saveTG{𣎐}{12288}
\saveTG{𣲦}{12290}
\saveTG{𫌌}{12291}
\saveTG{𣨒}{12293}
\saveTG{𢑈}{12293}
\saveTG{𢑄}{12293}
\saveTG{𣩓}{12294}
\saveTG{𣩫}{12295}
\saveTG{𠟨}{12300}
\saveTG{𠟢}{12300}
\saveTG{𠟦}{12300}
\saveTG{𪧾}{12315}
\saveTG{𠃞}{12317}
\saveTG{𩢾}{12327}
\saveTG{𫛊}{12327}
\saveTG{𪂲}{12327}
\saveTG{鴷}{12327}
\saveTG{𢥍}{12330}
\saveTG{烈}{12330}
\saveTG{㤠}{12330}
\saveTG{𢞵}{12331}
\saveTG{𢘠}{12331}
\saveTG{𢝵}{12331}
\saveTG{𢗨}{12333}
\saveTG{𢙰}{12333}
\saveTG{𢡲}{12336}
\saveTG{鮤}{12336}
\saveTG{愻}{12339}
\saveTG{𤤐}{12341}
\saveTG{𠠨}{12400}
\saveTG{䏀}{12400}
\saveTG{𦖇}{12400}
\saveTG{䎺}{12400}
\saveTG{㓴}{12400}
\saveTG{𠛬}{12400}
\saveTG{刵}{12400}
\saveTG{刊}{12400}
\saveTG{刑}{12400}
\saveTG{𠜘}{12400}
\saveTG{𡥁}{12400}
\saveTG{癷}{12401}
\saveTG{廷}{12401}
\saveTG{延}{12401}
\saveTG{㢟}{12401}
\saveTG{廵}{12403}
\saveTG{姴}{12404}
\saveTG{𢌙}{12404}
\saveTG{𢌜}{12405}
\saveTG{𤼦}{12407}
\saveTG{𡕶}{12407}
\saveTG{𢌚}{12407}
\saveTG{𤼬}{12407}
\saveTG{癹}{12407}
\saveTG{𤼧}{12407}
\saveTG{𦕿}{12410}
\saveTG{𡥑}{12410}
\saveTG{孔}{12410}
\saveTG{耴}{12410}
\saveTG{𦕭}{12412}
\saveTG{𡦒}{12412}
\saveTG{𦕖}{12412}
\saveTG{𣄺}{12412}
\saveTG{聎}{12413}
\saveTG{飝}{12413}
\saveTG{𪧄}{12413}
\saveTG{飛}{12413}
\saveTG{発}{12414}
\saveTG{毦}{12414}
\saveTG{𦔽}{12414}
\saveTG{𦖋}{12415}
\saveTG{𡥿}{12415}
\saveTG{𦗨}{12415}
\saveTG{𡥊}{12415}
\saveTG{𥦚}{12417}
\saveTG{㝂}{12421}
\saveTG{𦗝}{12421}
\saveTG{𣂖}{12421}
\saveTG{𦕄}{12421}
\saveTG{𦕶}{12421}
\saveTG{形}{12422}
\saveTG{𢌘}{12422}
\saveTG{聥}{12427}
\saveTG{劽}{12427}
\saveTG{𡥶}{12427}
\saveTG{𤼶}{12427}
\saveTG{孤}{12430}
\saveTG{𦔷}{12432}
\saveTG{𧏄}{12436}
\saveTG{𫀁}{12440}
\saveTG{𤼷}{12441}
\saveTG{𦕣}{12441}
\saveTG{𤼪}{12443}
\saveTG{𤼴}{12443}
\saveTG{聨}{12443}
\saveTG{𦖀}{12447}
\saveTG{𦖸}{12448}
\saveTG{䎽}{12464}
\saveTG{𦕾}{12464}
\saveTG{聒}{12464}
\saveTG{𦔺}{12470}
\saveTG{聯}{12472}
\saveTG{聉}{12472}
\saveTG{𥨏}{12478}
\saveTG{𦕯}{12484}
\saveTG{聫}{12484}
\saveTG{聧}{12484}
\saveTG{𡦫}{12486}
\saveTG{𦕱}{12493}
\saveTG{聮}{12493}
\saveTG{孫}{12493}
\saveTG{䏈}{12493}
\saveTG{𦘈}{12494}
\saveTG{𦗔}{12494}
\saveTG{𠤖}{12501}
\saveTG{𤘿}{12501}
\saveTG{軰}{12506}
\saveTG{𨋉}{12506}
\saveTG{𤼳}{12508}
\saveTG{𡦶}{12514}
\saveTG{䂰}{12600}
\saveTG{䃗}{12600}
\saveTG{𥒁}{12600}
\saveTG{𨡘}{12600}
\saveTG{刟}{12600}
\saveTG{酬}{12600}
\saveTG{矵}{12600}
\saveTG{副}{12600}
\saveTG{硎}{12600}
\saveTG{𠟗}{12600}
\saveTG{㔆}{12600}
\saveTG{㔏}{12600}
\saveTG{𥓫}{12600}
\saveTG{𥒂}{12600}
\saveTG{𥗌}{12600}
\saveTG{㓦}{12600}
\saveTG{𥐣}{12600}
\saveTG{𥒻}{12600}
\saveTG{𥓑}{12600}
\saveTG{𧥿}{12601}
\saveTG{𧦴}{12601}
\saveTG{𥅮}{12602}
\saveTG{㽝}{12602}
\saveTG{㴅}{12602}
\saveTG{沯}{12602}
\saveTG{𣳉}{12603}
\saveTG{𨠆}{12604}
\saveTG{呇}{12609}
\saveTG{畓}{12609}
\saveTG{沓}{12609}
\saveTG{砒}{12610}
\saveTG{𥐕}{12610}
\saveTG{𨡬}{12612}
\saveTG{䂣}{12612}
\saveTG{䂪}{12612}
\saveTG{𫖁}{12612}
\saveTG{䩃}{12612}
\saveTG{𥓱}{12612}
\saveTG{䂷}{12612}
\saveTG{𨟵}{12612}
\saveTG{䃬}{12613}
\saveTG{𥕬}{12614}
\saveTG{醗}{12614}
\saveTG{𨠲}{12614}
\saveTG{矺}{12614}
\saveTG{𥓕}{12614}
\saveTG{酕}{12614}
\saveTG{𥖌}{12614}
\saveTG{𣮿}{12615}
\saveTG{𥗿}{12615}
\saveTG{𥐽}{12615}
\saveTG{𥕹}{12615}
\saveTG{𥗦}{12615}
\saveTG{𥔧}{12615}
\saveTG{磪}{12615}
\saveTG{硾}{12615}
\saveTG{䃳}{12616}
\saveTG{𨢂}{12617}
\saveTG{𨡦}{12617}
\saveTG{䣥}{12617}
\saveTG{𣯮}{12617}
\saveTG{𣬸}{12617}
\saveTG{𥑽}{12617}
\saveTG{硙}{12617}
\saveTG{䣴}{12617}
\saveTG{磃}{12617}
\saveTG{𨢉}{12618}
\saveTG{磑}{12618}
\saveTG{磴}{12618}
\saveTG{𥕶}{12621}
\saveTG{𥕌}{12621}
\saveTG{𨠇}{12621}
\saveTG{㪼}{12621}
\saveTG{斫}{12621}
\saveTG{碋}{12621}
\saveTG{𥓸}{12622}
\saveTG{𥕱}{12622}
\saveTG{𥒱}{12622}
\saveTG{𥑓}{12627}
\saveTG{𥔗}{12627}
\saveTG{𥔱}{12627}
\saveTG{磮}{12627}
\saveTG{磞}{12627}
\saveTG{礄}{12627}
\saveTG{酳}{12627}
\saveTG{硶}{12627}
\saveTG{硚}{12628}
\saveTG{𥒩}{12630}
\saveTG{𥑔}{12630}
\saveTG{𩈕}{12630}
\saveTG{𥒜}{12630}
\saveTG{𥒛}{12631}
\saveTG{醺}{12631}
\saveTG{砭}{12632}
\saveTG{𥕈}{12632}
\saveTG{𩉒}{12632}
\saveTG{𨠋}{12633}
\saveTG{𨢑}{12637}
\saveTG{𥖵}{12637}
\saveTG{𥒗}{12639}
\saveTG{砥}{12640}
\saveTG{𦧝}{12641}
\saveTG{䂨}{12641}
\saveTG{硸}{12641}
\saveTG{硟}{12641}
\saveTG{𣳖}{12642}
\saveTG{𥓔}{12644}
\saveTG{𨡌}{12644}
\saveTG{釂}{12646}
\saveTG{醱}{12647}
\saveTG{𥔺}{12647}
\saveTG{𦧦}{12647}
\saveTG{𨤊}{12647}
\saveTG{𣱇}{12647}
\saveTG{𨟾}{12647}
\saveTG{𨠏}{12647}
\saveTG{𥓻}{12647}
\saveTG{𥘂}{12647}
\saveTG{酻}{12647}
\saveTG{𥒫}{12647}
\saveTG{𤿽}{12647}
\saveTG{𥔛}{12647}
\saveTG{𠺥}{12647}
\saveTG{𥖦}{12647}
\saveTG{𨡩}{12647}
\saveTG{𪿿}{12648}
\saveTG{酹}{12649}
\saveTG{𨢙}{12652}
\saveTG{磯}{12653}
\saveTG{䤒}{12658}
\saveTG{𥒖}{12661}
\saveTG{酯}{12661}
\saveTG{𦧟}{12662}
\saveTG{𨡍}{12662}
\saveTG{碯}{12662}
\saveTG{䃈}{12662}
\saveTG{𥔠}{12663}
\saveTG{䂿}{12663}
\saveTG{𥓲}{12663}
\saveTG{𥔥}{12663}
\saveTG{𩈙}{12664}
\saveTG{䣶}{12664}
\saveTG{碈}{12664}
\saveTG{碷}{12664}
\saveTG{磻}{12669}
\saveTG{𥔜}{12669}
\saveTG{酗}{12670}
\saveTG{𥐢}{12670}
\saveTG{𥒤}{12672}
\saveTG{础}{12672}
\saveTG{磘}{12672}
\saveTG{𥔿}{12677}
\saveTG{𣣻}{12682}
\saveTG{𥗢}{12682}
\saveTG{磎}{12684}
\saveTG{䤆}{12684}
\saveTG{醭}{12685}
\saveTG{礩}{12686}
\saveTG{𨢖}{12686}
\saveTG{𥕫}{12686}
\saveTG{碳}{12689}
\saveTG{𨤁}{12689}
\saveTG{砅}{12690}
\saveTG{𥓬}{12692}
\saveTG{𥒶}{12693}
\saveTG{𨣺}{12694}
\saveTG{𫌚}{12694}
\saveTG{𥕆}{12694}
\saveTG{𥡰}{12694}
\saveTG{䤕}{12694}
\saveTG{酥}{12694}
\saveTG{礫}{12694}
\saveTG{砾}{12694}
\saveTG{𥕘}{12694}
\saveTG{䃯}{12694}
\saveTG{𥓽}{12694}
\saveTG{𩈒}{12694}
\saveTG{礏}{12695}
\saveTG{𨢪}{12695}
\saveTG{𥔝}{12695}
\saveTG{𠟙}{12700}
\saveTG{瓱}{12711}
\saveTG{𦣮}{12712}
\saveTG{𤼥}{12712}
\saveTG{𠫕}{12713}
\saveTG{瓪}{12714}
\saveTG{瓩}{12714}
\saveTG{𣬏}{12717}
\saveTG{𦫔}{12717}
\saveTG{𣲯}{12717}
\saveTG{㼱}{12717}
\saveTG{氹}{12719}
\saveTG{𣂱}{12721}
\saveTG{𤼮}{12727}
\saveTG{𢎜}{12727}
\saveTG{裂}{12732}
\saveTG{𩚾}{12732}
\saveTG{𤼼}{12747}
\saveTG{靉}{12747}
\saveTG{𨲔}{12747}
\saveTG{叆}{12747}
\saveTG{𣳗}{12750}
\saveTG{𨲌}{12757}
\saveTG{𤼻}{12772}
\saveTG{𡷒}{12772}
\saveTG{䶛}{12772}
\saveTG{𪚀}{12772}
\saveTG{𪚁}{12772}
\saveTG{𡶳}{12772}
\saveTG{㳫}{12777}
\saveTG{𠞖}{12800}
\saveTG{冀}{12801}
\saveTG{𤳓}{12801}
\saveTG{𩙺}{12801}
\saveTG{癶}{12802}
\saveTG{𤼯}{12802}
\saveTG{𣱸}{12803}
\saveTG{𤼩}{12804}
\saveTG{𥎨}{12804}
\saveTG{癸}{12804}
\saveTG{烮}{12809}
\saveTG{氼}{12809}
\saveTG{𩑍}{12810}
\saveTG{䂡}{12812}
\saveTG{𥎲}{12817}
\saveTG{𩑘}{12822}
\saveTG{𤬉}{12833}
\saveTG{𤼺}{12847}
\saveTG{𤼹}{12848}
\saveTG{𧶙}{12862}
\saveTG{𣺪}{12866}
\saveTG{𣶷}{12889}
\saveTG{𦄳}{12893}
\saveTG{𢑲}{12893}
\saveTG{𠝱}{12900}
\saveTG{㔌}{12900}
\saveTG{𠟕}{12900}
\saveTG{𠞉}{12900}
\saveTG{𠞨}{12900}
\saveTG{刴}{12900}
\saveTG{剽}{12900}
\saveTG{剥}{12900}
\saveTG{㔄}{12900}
\saveTG{𨡊}{12900}
\saveTG{水}{12900}
\saveTG{𠝌}{12900}
\saveTG{𥙊}{12901}
\saveTG{𪟪}{12901}
\saveTG{𣛁}{12904}
\saveTG{𣔽}{12904}
\saveTG{𣕰}{12904}
\saveTG{𤼨}{12904}
\saveTG{䉾}{12904}
\saveTG{礼}{12910}
\saveTG{𣮪}{12915}
\saveTG{𣯼}{12917}
\saveTG{𨠐}{12917}
\saveTG{彯}{12922}
\saveTG{瓢}{12930}
\saveTG{𥢝}{12932}
\saveTG{𦄵}{12939}
\saveTG{𣱊}{12941}
\saveTG{𥞥}{12942}
\saveTG{𨣥}{12947}
\saveTG{𣝭}{12962}
\saveTG{𨢝}{12977}
\saveTG{𥜪}{12986}
\saveTG{沝}{12990}
\saveTG{淼}{12992}
\saveTG{𥽹}{12994}
\saveTG{𢦐}{13050}
\saveTG{㺪}{13100}
\saveTG{𪻒}{13100}
\saveTG{㱐}{13101}
\saveTG{𥂫}{13102}
\saveTG{𣱲}{13102}
\saveTG{珌}{13104}
\saveTG{𨫣}{13109}
\saveTG{豌}{13112}
\saveTG{𤦌}{13112}
\saveTG{𤧺}{13112}
\saveTG{玧}{13112}
\saveTG{琓}{13112}
\saveTG{琬}{13112}
\saveTG{珑}{13114}
\saveTG{𪼍}{13116}
\saveTG{瑄}{13116}
\saveTG{𦒨}{13117}
\saveTG{𪼝}{13117}
\saveTG{𤪥}{13121}
\saveTG{𧰗}{13121}
\saveTG{𤨥}{13121}
\saveTG{𤨵}{13122}
\saveTG{𤦓}{13122}
\saveTG{㻞}{13127}
\saveTG{𦑮}{13127}
\saveTG{䎍}{13127}
\saveTG{㻆}{13127}
\saveTG{𦑷}{13127}
\saveTG{𦒊}{13127}
\saveTG{𤨚}{13131}
\saveTG{𤥠}{13131}
\saveTG{𪼮}{13132}
\saveTG{琅}{13132}
\saveTG{𤤯}{13132}
\saveTG{𤨎}{13132}
\saveTG{𪼨}{13132}
\saveTG{𪼭}{13136}
\saveTG{𤩨}{13138}
\saveTG{弐}{13140}
\saveTG{玳}{13140}
\saveTG{𦏵}{13140}
\saveTG{珷}{13140}
\saveTG{武}{13140}
\saveTG{𪻝}{13141}
\saveTG{瑏}{13141}
\saveTG{𪻙}{13141}
\saveTG{𤧵}{13143}
\saveTG{䎔}{13143}
\saveTG{𦐗}{13144}
\saveTG{㺹}{13144}
\saveTG{𤥃}{13144}
\saveTG{𤤒}{13144}
\saveTG{𦑁}{13147}
\saveTG{𦥀}{13147}
\saveTG{㻐}{13147}
\saveTG{琙}{13150}
\saveTG{珹}{13150}
\saveTG{戤}{13150}
\saveTG{臹}{13150}
\saveTG{珴}{13150}
\saveTG{珬}{13150}
\saveTG{瑊}{13150}
\saveTG{𤪚}{13151}
\saveTG{㦱}{13151}
\saveTG{㵶}{13151}
\saveTG{𢦰}{13151}
\saveTG{𦑌}{13151}
\saveTG{琖}{13153}
\saveTG{䎒}{13153}
\saveTG{𡍌}{13153}
\saveTG{䎉}{13153}
\saveTG{𤥟}{13156}
\saveTG{㻘}{13156}
\saveTG{𪼞}{13156}
\saveTG{䎀}{13157}
\saveTG{𤨟}{13159}
\saveTG{珆}{13160}
\saveTG{𤦮}{13161}
\saveTG{𪻽}{13164}
\saveTG{瑢}{13168}
\saveTG{𤪓}{13168}
\saveTG{𤩷}{13168}
\saveTG{𤪺}{13169}
\saveTG{琯}{13177}
\saveTG{瑸}{13181}
\saveTG{琔}{13181}
\saveTG{㺴}{13184}
\saveTG{𪼖}{13184}
\saveTG{㻠}{13184}
\saveTG{瓛}{13184}
\saveTG{𤩽}{13184}
\saveTG{𤫖}{13186}
\saveTG{𤫞}{13186}
\saveTG{璸}{13186}
\saveTG{璌}{13186}
\saveTG{𤧪}{13189}
\saveTG{琮}{13191}
\saveTG{㺷}{13194}
\saveTG{𤥤}{13194}
\saveTG{𦥅}{13194}
\saveTG{球}{13199}
\saveTG{𢎡}{13200}
\saveTG{𧱡}{13212}
\saveTG{𧰰}{13212}
\saveTG{𥎣}{13212}
\saveTG{𣨩}{13212}
\saveTG{殧}{13212}
\saveTG{𫀏}{13212}
\saveTG{兘}{13216}
\saveTG{𩴒}{13217}
\saveTG{𢏋}{13217}
\saveTG{𣧗}{13217}
\saveTG{㱧}{13217}
\saveTG{𢏿}{13217}
\saveTG{豟}{13217}
\saveTG{𤘓}{13221}
\saveTG{𢏟}{13227}
\saveTG{𡬘}{13227}
\saveTG{𥙳}{13227}
\saveTG{𣨈}{13227}
\saveTG{𧱚}{13227}
\saveTG{豧}{13227}
\saveTG{鬴}{13227}
\saveTG{䳮}{13231}
\saveTG{𢐵}{13231}
\saveTG{𩱝}{13232}
\saveTG{𩱘}{13232}
\saveTG{𥍫}{13232}
\saveTG{強}{13236}
\saveTG{𣧆}{13240}
\saveTG{㱰}{13241}
\saveTG{𧱦}{13242}
\saveTG{𧱹}{13242}
\saveTG{𣧧}{13244}
\saveTG{𩰺}{13247}
\saveTG{𢏤}{13247}
\saveTG{𥍬}{13247}
\saveTG{𩰽}{13247}
\saveTG{𦓢}{13248}
\saveTG{𩰼}{13249}
\saveTG{残}{13250}
\saveTG{𢏌}{13250}
\saveTG{𦐂}{13250}
\saveTG{殲}{13250}
\saveTG{𩰭}{13250}
\saveTG{𥍥}{13250}
\saveTG{𢨌}{13250}
\saveTG{𢦧}{13250}
\saveTG{戮}{13250}
\saveTG{𣧌}{13250}
\saveTG{㦷}{13250}
\saveTG{𥎕}{13250}
\saveTG{𧲘}{13250}
\saveTG{殱}{13250}
\saveTG{𤘋}{13250}
\saveTG{戳}{13250}
\saveTG{殲}{13251}
\saveTG{𣧵}{13251}
\saveTG{𣨤}{13251}
\saveTG{𢧈}{13253}
\saveTG{殘}{13253}
\saveTG{𣧡}{13257}
\saveTG{殆}{13260}
\saveTG{𣩐}{13262}
\saveTG{𥎆}{13265}
\saveTG{𪵂}{13272}
\saveTG{𣨭}{13277}
\saveTG{殡}{13281}
\saveTG{㱨}{13282}
\saveTG{𢏹}{13282}
\saveTG{𤣅}{13284}
\saveTG{𢏦}{13284}
\saveTG{𢏓}{13284}
\saveTG{𤣉}{13284}
\saveTG{𤟶}{13284}
\saveTG{𢐛}{13285}
\saveTG{𫌐}{13285}
\saveTG{殥}{13286}
\saveTG{𪫂}{13286}
\saveTG{𣩵}{13286}
\saveTG{殯}{13286}
\saveTG{䝋}{13291}
\saveTG{㢱}{13291}
\saveTG{𧱍}{13294}
\saveTG{殏}{13299}
\saveTG{𩰻}{13299}
\saveTG{𡯼}{13317}
\saveTG{𫛁}{13327}
\saveTG{𣸒}{13341}
\saveTG{𤡾}{13384}
\saveTG{恥}{13400}
\saveTG{𦖶}{13400}
\saveTG{𪫵}{13400}
\saveTG{𢌠}{13402}
\saveTG{𪎍}{13402}
\saveTG{𡠥}{13404}
\saveTG{䎵}{13404}
\saveTG{聜}{13412}
\saveTG{𦗃}{13412}
\saveTG{𥧪}{13412}
\saveTG{𦖑}{13412}
\saveTG{䏄}{13414}
\saveTG{聹}{13421}
\saveTG{聍}{13421}
\saveTG{𦗞}{13421}
\saveTG{𦗰}{13421}
\saveTG{𦖓}{13427}
\saveTG{𦖶}{13431}
\saveTG{𦕟}{13432}
\saveTG{𦕹}{13434}
\saveTG{𫆒}{13437}
\saveTG{𣦏}{13440}
\saveTG{䏁}{13441}
\saveTG{𦖪}{13447}
\saveTG{職}{13450}
\saveTG{聝}{13450}
\saveTG{𢧎}{13450}
\saveTG{𢧙}{13450}
\saveTG{𦖱}{13450}
\saveTG{𦕧}{13450}
\saveTG{𪦷}{13450}
\saveTG{𦖎}{13450}
\saveTG{聀}{13450}
\saveTG{𦕵}{13452}
\saveTG{孡}{13460}
\saveTG{聢}{13481}
\saveTG{孮}{13491}
\saveTG{聺}{13491}
\saveTG{𦗋}{13491}
\saveTG{𢟛}{13527}
\saveTG{𠧬}{13600}
\saveTG{𠧙}{13600}
\saveTG{𨠉}{13600}
\saveTG{𥐚}{13600}
\saveTG{𥑖}{13604}
\saveTG{𠷠}{13605}
\saveTG{}{13607}
\saveTG{𥓗}{13611}
\saveTG{醡}{13611}
\saveTG{砣}{13612}
\saveTG{硿}{13612}
\saveTG{酡}{13612}
\saveTG{𨣮}{13612}
\saveTG{𨣘}{13612}
\saveTG{䤉}{13612}
\saveTG{𩈭}{13612}
\saveTG{䩊}{13612}
\saveTG{𥐶}{13612}
\saveTG{𩉊}{13612}
\saveTG{𥐸}{13612}
\saveTG{𩈊}{13612}
\saveTG{碗}{13612}
\saveTG{𥕅}{13612}
\saveTG{𥒈}{13614}
\saveTG{硥}{13614}
\saveTG{碹}{13616}
\saveTG{𨠄}{13617}
\saveTG{𦧊}{13617}
\saveTG{𨠻}{13617}
\saveTG{𪿜}{13617}
\saveTG{砨}{13617}
\saveTG{𦧑}{13617}
\saveTG{醦}{13622}
\saveTG{𩈼}{13622}
\saveTG{碜}{13622}
\saveTG{磣}{13622}
\saveTG{𥔪}{13623}
\saveTG{𨢃}{13624}
\saveTG{𥓰}{13627}
\saveTG{𥖽}{13627}
\saveTG{𨢺}{13627}
\saveTG{酺}{13627}
\saveTG{碥}{13627}
\saveTG{𨡗}{13627}
\saveTG{𥑢}{13627}
\saveTG{𥒰}{13627}
\saveTG{䩉}{13627}
\saveTG{𨠕}{13632}
\saveTG{𩉌}{13632}
\saveTG{𩈗}{13632}
\saveTG{酿}{13632}
\saveTG{硠}{13632}
\saveTG{𪿭}{13632}
\saveTG{硡}{13632}
\saveTG{𥕃}{13633}
\saveTG{𨣝}{13635}
\saveTG{䃭}{13635}
\saveTG{𨣼}{13636}
\saveTG{𩉄}{13638}
\saveTG{𨣔}{13638}
\saveTG{𥕺}{13638}
\saveTG{㢦}{13640}
\saveTG{𢎄}{13640}
\saveTG{𨠍}{13640}
\saveTG{碔}{13640}
\saveTG{𨠔}{13641}
\saveTG{磗}{13642}
\saveTG{酧}{13642}
\saveTG{𨣪}{13642}
\saveTG{酦}{13647}
\saveTG{𩈥}{13647}
\saveTG{酸}{13647}
\saveTG{𥑕}{13647}
\saveTG{𨡻}{13647}
\saveTG{䤇}{13647}
\saveTG{𥔉}{13647}
\saveTG{䣮}{13647}
\saveTG{𥔑}{13647}
\saveTG{戩}{13650}
\saveTG{醎}{13650}
\saveTG{磩}{13650}
\saveTG{䂸}{13650}
\saveTG{戨}{13650}
\saveTG{䂝}{13650}
\saveTG{碱}{13650}
\saveTG{𥖘}{13650}
\saveTG{戬}{13650}
\saveTG{𥔃}{13650}
\saveTG{𥓉}{13650}
\saveTG{𥑳}{13650}
\saveTG{䃱}{13650}
\saveTG{𥒎}{13650}
\saveTG{𥓛}{13650}
\saveTG{𥖙}{13650}
\saveTG{𥗜}{13650}
\saveTG{𩉔}{13650}
\saveTG{𢦱}{13650}
\saveTG{㦻}{13650}
\saveTG{䣬}{13650}
\saveTG{䣹}{13650}
\saveTG{𢦪}{13650}
\saveTG{𢧃}{13650}
\saveTG{硪}{13650}
\saveTG{䤘}{13651}
\saveTG{𨣲}{13651}
\saveTG{䃸}{13651}
\saveTG{𥒪}{13652}
\saveTG{𨠽}{13652}
\saveTG{𥖔}{13653}
\saveTG{碊}{13653}
\saveTG{醆}{13653}
\saveTG{𨠤}{13654}
\saveTG{𪿘}{13661}
\saveTG{𨢳}{13661}
\saveTG{𥕯}{13662}
\saveTG{𨢲}{13662}
\saveTG{碦}{13664}
\saveTG{磍}{13665}
\saveTG{𦧮}{13665}
\saveTG{𨡥}{13665}
\saveTG{䃔}{13666}
\saveTG{𪿮}{13668}
\saveTG{𨢧}{13672}
\saveTG{𥖚}{13672}
\saveTG{𪿲}{13674}
\saveTG{𩈬}{13677}
\saveTG{碇}{13681}
\saveTG{𥑫}{13682}
\saveTG{𥒲}{13684}
\saveTG{𩉇}{13684}
\saveTG{䃐}{13684}
\saveTG{𨡧}{13685}
\saveTG{𥖶}{13686}
\saveTG{礗}{13686}
\saveTG{䃰}{13691}
\saveTG{𩉐}{13691}
\saveTG{碂}{13691}
\saveTG{𨠼}{13694}
\saveTG{𥑡}{13694}
\saveTG{𥒬}{13694}
\saveTG{𥒸}{13699}
\saveTG{𢖮}{13711}
\saveTG{𢎁}{13740}
\saveTG{𩃷}{13742}
\saveTG{𩃶}{13744}
\saveTG{𤐦}{13809}
\saveTG{𤆲}{13809}
\saveTG{貮}{13840}
\saveTG{𢨙}{13850}
\saveTG{戣}{13850}
\saveTG{𪼱}{13864}
\saveTG{䝾}{13864}
\saveTG{𪲠}{13904}
\saveTG{𠁇}{13912}
\saveTG{䣲}{13944}
\saveTG{𣕥}{13945}
\saveTG{𠦁}{14000}
\saveTG{𣁬}{14000}
\saveTG{𥂘}{14102}
\saveTG{𥂹}{14102}
\saveTG{𥂰}{14102}
\saveTG{𥂧}{14102}
\saveTG{𤤕}{14103}
\saveTG{𡬵}{14103}
\saveTG{㺶}{14103}
\saveTG{㪷}{14103}
\saveTG{𡎂}{14104}
\saveTG{𪣿}{14104}
\saveTG{𥪛}{14108}
\saveTG{𤣰}{14110}
\saveTG{𤥧}{14112}
\saveTG{𡏖}{14112}
\saveTG{𩅷}{14112}
\saveTG{𤧔}{14112}
\saveTG{𤫧}{14112}
\saveTG{瓂}{14112}
\saveTG{珯}{14112}
\saveTG{珗}{14112}
\saveTG{}{14113}
\saveTG{𤧊}{14114}
\saveTG{𤪣}{14114}
\saveTG{𪻧}{14114}
\saveTG{𪻗}{14114}
\saveTG{𢑫}{14114}
\saveTG{珪}{14114}
\saveTG{璂}{14114}
\saveTG{𤩖}{14115}
\saveTG{瓘}{14115}
\saveTG{瑾}{14115}
\saveTG{𪼘}{14116}
\saveTG{𤤙}{14117}
\saveTG{𤩊}{14117}
\saveTG{𤧌}{14117}
\saveTG{𪻔}{14117}
\saveTG{𦒒}{14117}
\saveTG{𤪝}{14117}
\saveTG{𣦥}{14117}
\saveTG{𤥣}{14117}
\saveTG{𪼒}{14117}
\saveTG{𤨰}{14117}
\saveTG{𢑝}{14117}
\saveTG{𤧈}{14117}
\saveTG{𤦙}{14117}
\saveTG{𤧳}{14117}
\saveTG{𤤜}{14117}
\saveTG{𠡔}{14117}
\saveTG{𦑎}{14117}
\saveTG{𦐕}{14117}
\saveTG{玴}{14117}
\saveTG{𤩌}{14118}
\saveTG{㻣}{14118}
\saveTG{㻳}{14120}
\saveTG{𤫨}{14121}
\saveTG{𤨦}{14121}
\saveTG{琦}{14121}
\saveTG{𤨀}{14121}
\saveTG{𤨯}{14121}
\saveTG{𪻸}{14121}
\saveTG{𣩄}{14122}
\saveTG{𠡕}{14127}
\saveTG{𤤬}{14127}
\saveTG{𣻀}{14127}
\saveTG{功}{14127}
\saveTG{㻟}{14127}
\saveTG{𤨳}{14127}
\saveTG{𤦈}{14127}
\saveTG{劲}{14127}
\saveTG{勁}{14127}
\saveTG{玏}{14127}
\saveTG{珕}{14127}
\saveTG{璊}{14127}
\saveTG{𪻳}{14127}
\saveTG{𤨕}{14127}
\saveTG{勐}{14127}
\saveTG{琋}{14127}
\saveTG{𤥉}{14127}
\saveTG{珛}{14127}
\saveTG{璓}{14127}
\saveTG{𤩲}{14127}
\saveTG{𠣋}{14127}
\saveTG{𤨮}{14127}
\saveTG{𤨸}{14127}
\saveTG{㻤}{14127}
\saveTG{𠡚}{14127}
\saveTG{𦐌}{14131}
\saveTG{聽}{14131}
\saveTG{𤥴}{14131}
\saveTG{𤣾}{14131}
\saveTG{𡐉}{14131}
\saveTG{𣶨}{14131}
\saveTG{𤫇}{14131}
\saveTG{𪼄}{14132}
\saveTG{珐}{14132}
\saveTG{𤨂}{14132}
\saveTG{𤩼}{14132}
\saveTG{𤧍}{14132}
\saveTG{瓍}{14132}
\saveTG{𤪑}{14132}
\saveTG{𤩿}{14132}
\saveTG{琺}{14132}
\saveTG{𤫃}{14133}
\saveTG{瓙}{14134}
\saveTG{琏}{14135}
\saveTG{𤪼}{14135}
\saveTG{㼀}{14135}
\saveTG{𧐵}{14136}
\saveTG{𤦧}{14137}
\saveTG{𤪶}{14138}
\saveTG{𤧴}{14138}
\saveTG{𤣷}{14140}
\saveTG{𡚽}{14140}
\saveTG{𢑒}{14140}
\saveTG{璹}{14141}
\saveTG{𤦫}{14142}
\saveTG{𤨅}{14143}
\saveTG{𣻞}{14143}
\saveTG{𪻚}{14144}
\saveTG{翍}{14147}
\saveTG{玻}{14147}
\saveTG{豉}{14147}
\saveTG{珔}{14147}
\saveTG{瓁}{14147}
\saveTG{翄}{14147}
\saveTG{臶}{14147}
\saveTG{𤥝}{14147}
\saveTG{𤿻}{14147}
\saveTG{䜵}{14147}
\saveTG{𤿿}{14147}
\saveTG{𪤶}{14147}
\saveTG{𤪍}{14153}
\saveTG{璍}{14154}
\saveTG{瑋}{14156}
\saveTG{𪻕}{14160}
\saveTG{㴌}{14160}
\saveTG{㵈}{14160}
\saveTG{琽}{14160}
\saveTG{𤦘}{14161}
\saveTG{䜺}{14161}
\saveTG{𤩠}{14161}
\saveTG{𤨑}{14161}
\saveTG{𤥐}{14161}
\saveTG{𤪘}{14161}
\saveTG{𪻷}{14164}
\saveTG{𣾻}{14164}
\saveTG{𪶖}{14164}
\saveTG{𤨃}{14168}
\saveTG{𤪜}{14168}
\saveTG{玵}{14170}
\saveTG{𤪬}{14181}
\saveTG{璴}{14181}
\saveTG{珙}{14181}
\saveTG{琪}{14181}
\saveTG{瑱}{14181}
\saveTG{𪼊}{14182}
\saveTG{瓒}{14182}
\saveTG{𧯯}{14184}
\saveTG{𤩸}{14184}
\saveTG{𣨘}{14184}
\saveTG{𪻨}{14184}
\saveTG{𣩎}{14184}
\saveTG{𤨢}{14185}
\saveTG{瑛}{14185}
\saveTG{𦒅}{14185}
\saveTG{𣨽}{14185}
\saveTG{𤩵}{14186}
\saveTG{𪼩}{14186}
\saveTG{璜}{14186}
\saveTG{㱵}{14186}
\saveTG{𤩳}{14186}
\saveTG{瓄}{14186}
\saveTG{瓚}{14186}
\saveTG{𠓕}{14186}
\saveTG{𤧙}{14187}
\saveTG{𤥵}{14188}
\saveTG{𪼐}{14188}
\saveTG{𤪃}{14189}
\saveTG{𦐔}{14190}
\saveTG{琳}{14190}
\saveTG{𪻦}{14191}
\saveTG{𪻜}{14192}
\saveTG{𤪌}{14193}
\saveTG{𤩒}{14194}
\saveTG{𤧀}{14194}
\saveTG{𪣆}{14194}
\saveTG{𤨷}{14194}
\saveTG{㻡}{14194}
\saveTG{𫇑}{14194}
\saveTG{瑹}{14194}
\saveTG{𤨓}{14194}
\saveTG{𧰉}{14196}
\saveTG{璙}{14196}
\saveTG{𢑬}{14198}
\saveTG{𧯲}{14198}
\saveTG{琜}{14198}
\saveTG{𤨶}{14199}
\saveTG{𠦇}{14200}
\saveTG{耐}{14200}
\saveTG{弣}{14200}
\saveTG{𩰮}{14203}
\saveTG{㪴}{14203}
\saveTG{𫌗}{14203}
\saveTG{𥛖}{14203}
\saveTG{𠧍}{14204}
\saveTG{𢒥}{14207}
\saveTG{㝴}{14210}
\saveTG{𡉹}{14210}
\saveTG{𣧅}{14210}
\saveTG{𢎳}{14210}
\saveTG{䝅}{14210}
\saveTG{𡰞}{14211}
\saveTG{殑}{14212}
\saveTG{殖}{14212}
\saveTG{𠠺}{14212}
\saveTG{𢏶}{14212}
\saveTG{弛}{14212}
\saveTG{𣩱}{14212}
\saveTG{𣩲}{14214}
\saveTG{𣒮}{14214}
\saveTG{𩰳}{14214}
\saveTG{𧱀}{14214}
\saveTG{𥎚}{14214}
\saveTG{䝔}{14215}
\saveTG{𥎊}{14215}
\saveTG{殣}{14215}
\saveTG{殗}{14216}
\saveTG{𣦾}{14217}
\saveTG{𠣉}{14217}
\saveTG{𣩦}{14217}
\saveTG{㱡}{14217}
\saveTG{𢏡}{14217}
\saveTG{𢎩}{14217}
\saveTG{𢏬}{14218}
\saveTG{殪}{14218}
\saveTG{豷}{14218}
\saveTG{𤘌}{14221}
\saveTG{𧱺}{14221}
\saveTG{㱦}{14221}
\saveTG{𥍰}{14227}
\saveTG{𠣊}{14227}
\saveTG{㔝}{14227}
\saveTG{𤬲}{14227}
\saveTG{𣧦}{14227}
\saveTG{𣦺}{14227}
\saveTG{𣧻}{14227}
\saveTG{㱝}{14227}
\saveTG{𣧥}{14227}
\saveTG{𠢯}{14227}
\saveTG{𢐳}{14227}
\saveTG{𢐶}{14227}
\saveTG{𠣀}{14227}
\saveTG{𢎭}{14227}
\saveTG{䝐}{14227}
\saveTG{𧱫}{14227}
\saveTG{㣁}{14227}
\saveTG{𣨼}{14227}
\saveTG{𢑁}{14227}
\saveTG{勠}{14227}
\saveTG{劢}{14227}
\saveTG{豨}{14227}
\saveTG{勈}{14227}
\saveTG{𠡁}{14227}
\saveTG{殢}{14227}
\saveTG{𧲒}{14228}
\saveTG{𧰯}{14230}
\saveTG{𣨠}{14230}
\saveTG{𧹚}{14231}
\saveTG{㢬}{14231}
\saveTG{𢏐}{14231}
\saveTG{𧲈}{14232}
\saveTG{𪠞}{14234}
\saveTG{𣩷}{14236}
\saveTG{𥎢}{14239}
\saveTG{𧰿}{14240}
\saveTG{𣻤}{14240}
\saveTG{𡠢}{14240}
\saveTG{𡞒}{14240}
\saveTG{𥎂}{14241}
\saveTG{𧰸}{14242}
\saveTG{䰙}{14247}
\saveTG{𣲰}{14247}
\saveTG{𢎼}{14247}
\saveTG{𥀫}{14247}
\saveTG{㱟}{14247}
\saveTG{𤿏}{14247}
\saveTG{𥀘}{14247}
\saveTG{㢰}{14247}
\saveTG{㱥}{14247}
\saveTG{𦡟}{14247}
\saveTG{𪫀}{14256}
\saveTG{𩰯}{14260}
\saveTG{㱠}{14260}
\saveTG{𢏆}{14260}
\saveTG{豬}{14260}
\saveTG{𥍱}{14261}
\saveTG{矠}{14261}
\saveTG{𠫆}{14263}
\saveTG{𠞺}{14264}
\saveTG{𢑳}{14264}
\saveTG{𤮾}{14270}
\saveTG{𤘎}{14277}
\saveTG{𣧂}{14280}
\saveTG{𣨓}{14281}
\saveTG{𪜝}{14281}
\saveTG{𪵆}{14281}
\saveTG{㣀}{14281}
\saveTG{𩄠}{14281}
\saveTG{豮}{14282}
\saveTG{㱩}{14284}
\saveTG{𥍼}{14285}
\saveTG{𣪁}{14286}
\saveTG{𢑊}{14286}
\saveTG{殰}{14286}
\saveTG{豶}{14286}
\saveTG{彉}{14286}
\saveTG{殎}{14288}
\saveTG{𧱉}{14289}
\saveTG{𣐫}{14290}
\saveTG{𣖦}{14294}
\saveTG{殜}{14294}
\saveTG{弽}{14294}
\saveTG{𣩣}{14294}
\saveTG{𢐇}{14294}
\saveTG{𦜪}{14294}
\saveTG{𣩨}{14294}
\saveTG{𦜫}{14294}
\saveTG{𣨴}{14294}
\saveTG{𢏼}{14294}
\saveTG{𣩢}{14296}
\saveTG{䮛}{14300}
\saveTG{𪑤}{14331}
\saveTG{𤎂}{14332}
\saveTG{𢤔}{14332}
\saveTG{𤐿}{14336}
\saveTG{𤋕}{14338}
\saveTG{彟}{14347}
\saveTG{彠}{14347}
\saveTG{𤍰}{14347}
\saveTG{𢌗}{14400}
\saveTG{𠡣}{14402}
\saveTG{𠡀}{14402}
\saveTG{耽}{14412}
\saveTG{𦖅}{14412}
\saveTG{𦕳}{14412}
\saveTG{𦗆}{14415}
\saveTG{𦖈}{14415}
\saveTG{𠖙}{14417}
\saveTG{𪪅}{14417}
\saveTG{𡦮}{14418}
\saveTG{𦖊}{14421}
\saveTG{𡦖}{14427}
\saveTG{𦔳}{14427}
\saveTG{㔜}{14427}
\saveTG{𫆊}{14427}
\saveTG{𦖔}{14427}
\saveTG{𦖁}{14427}
\saveTG{𦖯}{14427}
\saveTG{聈}{14427}
\saveTG{聼}{14431}
\saveTG{耾}{14432}
\saveTG{聴}{14436}
\saveTG{𣀳}{14440}
\saveTG{𡥃}{14440}
\saveTG{𤿐}{14447}
\saveTG{𤿰}{14447}
\saveTG{𪔠}{14447}
\saveTG{𪦶}{14447}
\saveTG{𤿊}{14447}
\saveTG{𢆮}{14451}
\saveTG{𦗉}{14454}
\saveTG{㝆}{14461}
\saveTG{聐}{14461}
\saveTG{聕}{14461}
\saveTG{𦖿}{14461}
\saveTG{𫇜}{14464}
\saveTG{𦕠}{14481}
\saveTG{𦗀}{14481}
\saveTG{聗}{14488}
\saveTG{𠡅}{14527}
\saveTG{𩈃}{14600}
\saveTG{䂤}{14600}
\saveTG{酙}{14600}
\saveTG{酎}{14600}
\saveTG{𧦪}{14601}
\saveTG{𥑙}{14601}
\saveTG{𥓺}{14602}
\saveTG{𥐟}{14603}
\saveTG{㪶}{14603}
\saveTG{𠷱}{14604}
\saveTG{硭}{14610}
\saveTG{𨡈}{14610}
\saveTG{𥕜}{14612}
\saveTG{𥔾}{14612}
\saveTG{𥑰}{14612}
\saveTG{𥑻}{14612}
\saveTG{䃊}{14612}
\saveTG{𥔙}{14612}
\saveTG{𥔐}{14612}
\saveTG{𩈉}{14612}
\saveTG{𪿤}{14612}
\saveTG{𨡿}{14612}
\saveTG{𨡝}{14612}
\saveTG{䣾}{14612}
\saveTG{𨢠}{14612}
\saveTG{𦧇}{14612}
\saveTG{磽}{14612}
\saveTG{酖}{14612}
\saveTG{醢}{14612}
\saveTG{磕}{14612}
\saveTG{礚}{14612}
\saveTG{醘}{14612}
\saveTG{礷}{14612}
\saveTG{硓}{14612}
\saveTG{醓}{14612}
\saveTG{酰}{14612}
\saveTG{酏}{14612}
\saveTG{𥐨}{14612}
\saveTG{𥑀}{14612}
\saveTG{𨣫}{14613}
\saveTG{硅}{14614}
\saveTG{硴}{14614}
\saveTG{𤱡}{14614}
\saveTG{𥒒}{14614}
\saveTG{𥒦}{14614}
\saveTG{𥒃}{14614}
\saveTG{𥕛}{14614}
\saveTG{𥓪}{14614}
\saveTG{醛}{14614}
\saveTG{𥘁}{14615}
\saveTG{礶}{14615}
\saveTG{𨢜}{14615}
\saveTG{𩈯}{14615}
\saveTG{確}{14615}
\saveTG{𡂬}{14615}
\saveTG{𩈿}{14615}
\saveTG{醃}{14616}
\saveTG{𥗪}{14616}
\saveTG{硽}{14616}
\saveTG{碴}{14616}
\saveTG{𩈷}{14617}
\saveTG{𥑄}{14617}
\saveTG{𨠬}{14617}
\saveTG{𪿬}{14617}
\saveTG{𥐜}{14617}
\saveTG{䤁}{14618}
\saveTG{𥒴}{14618}
\saveTG{碪}{14618}
\saveTG{䂗}{14620}
\saveTG{碕}{14621}
\saveTG{𥘄}{14622}
\saveTG{𥗀}{14622}
\saveTG{䂶}{14627}
\saveTG{𠡳}{14627}
\saveTG{𩈏}{14627}
\saveTG{𥑑}{14627}
\saveTG{䤍}{14627}
\saveTG{𨡯}{14627}
\saveTG{𢃆}{14627}
\saveTG{𥗔}{14627}
\saveTG{𥖈}{14627}
\saveTG{𥓿}{14627}
\saveTG{𥑹}{14627}
\saveTG{𨡽}{14627}
\saveTG{𨡚}{14627}
\saveTG{𨡜}{14627}
\saveTG{𨡂}{14627}
\saveTG{䃎}{14627}
\saveTG{磡}{14627}
\saveTG{勔}{14627}
\saveTG{劭}{14627}
\saveTG{酭}{14627}
\saveTG{劯}{14627}
\saveTG{礍}{14627}
\saveTG{𨢗}{14627}
\saveTG{𥖣}{14627}
\saveTG{𥗺}{14627}
\saveTG{𥖰}{14627}
\saveTG{䣦}{14627}
\saveTG{𥗠}{14627}
\saveTG{𥑿}{14627}
\saveTG{𥕾}{14629}
\saveTG{}{14630}
\saveTG{𥐪}{14630}
\saveTG{𨟳}{14630}
\saveTG{酞}{14630}
\saveTG{醼}{14631}
\saveTG{𥒺}{14631}
\saveTG{𥗕}{14631}
\saveTG{𠳊}{14631}
\saveTG{硳}{14631}
\saveTG{砝}{14631}
\saveTG{䙶}{14632}
\saveTG{𥔵}{14632}
\saveTG{𨣭}{14632}
\saveTG{礞}{14632}
\saveTG{}{14632}
\saveTG{砝}{14632}
\saveTG{𥗨}{14632}
\saveTG{䤓}{14632}
\saveTG{䃮}{14635}
\saveTG{𥔸}{14638}
\saveTG{}{14638}
\saveTG{𥑾}{14640}
\saveTG{砹}{14640}
\saveTG{礟}{14640}
\saveTG{𨟿}{14640}
\saveTG{䂚}{14640}
\saveTG{䣧}{14640}
\saveTG{醻}{14641}
\saveTG{𥖲}{14641}
\saveTG{𣓉}{14641}
\saveTG{䤊}{14642}
\saveTG{礴}{14642}
\saveTG{𤾋}{14642}
\saveTG{𦧌}{14642}
\saveTG{礡}{14642}
\saveTG{𥐿}{14643}
\saveTG{𪿚}{14643}
\saveTG{𥓈}{14644}
\saveTG{𥓳}{14647}
\saveTG{𢻊}{14647}
\saveTG{𥕪}{14647}
\saveTG{𥑂}{14647}
\saveTG{𫑴}{14647}
\saveTG{𨡆}{14647}
\saveTG{𨢀}{14647}
\saveTG{𨠜}{14647}
\saveTG{𤿢}{14647}
\saveTG{酵}{14647}
\saveTG{碐}{14647}
\saveTG{破}{14647}
\saveTG{硣}{14647}
\saveTG{䩅}{14647}
\saveTG{𥕊}{14648}
\saveTG{砗}{14650}
\saveTG{𨤃}{14651}
\saveTG{𥗘}{14652}
\saveTG{礣}{14653}
\saveTG{䩏}{14653}
\saveTG{𨣄}{14654}
\saveTG{𧠁}{14654}
\saveTG{𥔬}{14656}
\saveTG{𦧹}{14658}
\saveTG{醏}{14660}
\saveTG{𥑮}{14660}
\saveTG{𥑛}{14660}
\saveTG{酤}{14660}
\saveTG{𥔽}{14661}
\saveTG{醋}{14661}
\saveTG{𨢔}{14661}
\saveTG{𨢍}{14661}
\saveTG{硞}{14661}
\saveTG{𪡬}{14661}
\saveTG{礂}{14661}
\saveTG{碏}{14661}
\saveTG{酷}{14661}
\saveTG{硈}{14661}
\saveTG{𥔇}{14662}
\saveTG{𥓩}{14662}
\saveTG{䃴}{14664}
\saveTG{𥖛}{14664}
\saveTG{𨣍}{14664}
\saveTG{䤀}{14664}
\saveTG{𨡱}{14664}
\saveTG{𪿾}{14669}
\saveTG{酣}{14670}
\saveTG{𥑠}{14670}
\saveTG{䣫}{14670}
\saveTG{䃆}{14681}
\saveTG{𪿰}{14681}
\saveTG{磌}{14681}
\saveTG{硔}{14681}
\saveTG{礎}{14681}
\saveTG{𨡨}{14681}
\saveTG{𥕩}{14682}
\saveTG{𨢟}{14682}
\saveTG{䣭}{14683}
\saveTG{𨢢}{14684}
\saveTG{𥕓}{14684}
\saveTG{磢}{14684}
\saveTG{碤}{14685}
\saveTG{𥖀}{14686}
\saveTG{𨤆}{14686}
\saveTG{礸}{14686}
\saveTG{磺}{14686}
\saveTG{𨠿}{14688}
\saveTG{硤}{14688}
\saveTG{醂}{14690}
\saveTG{碄}{14690}
\saveTG{礤}{14691}
\saveTG{𥖜}{14691}
\saveTG{𩈶}{14691}
\saveTG{𩈫}{14691}
\saveTG{𨠹}{14694}
\saveTG{䤂}{14694}
\saveTG{𥗹}{14694}
\saveTG{𥖢}{14694}
\saveTG{𨢬}{14694}
\saveTG{碟}{14694}
\saveTG{𨢕}{14694}
\saveTG{䩍}{14696}
\saveTG{𨣀}{14696}
\saveTG{𥕴}{14696}
\saveTG{䤑}{14696}
\saveTG{䂾}{14698}
\saveTG{𫑹}{14698}
\saveTG{瓧}{14710}
\saveTG{𤮳}{14711}
\saveTG{𪜐}{14711}
\saveTG{𪜒}{14716}
\saveTG{𩅾}{14717}
\saveTG{𩅝}{14717}
\saveTG{𨱧}{14727}
\saveTG{𩃠}{14727}
\saveTG{动}{14727}
\saveTG{𤮠}{14732}
\saveTG{𩝕}{14732}
\saveTG{𠦊}{14740}
\saveTG{妀}{14740}
\saveTG{𢻗}{14747}
\saveTG{𥀈}{14747}
\saveTG{𤭚}{14761}
\saveTG{𡻧}{14772}
\saveTG{巭}{14805}
\saveTG{𫎿}{14808}
\saveTG{𤉎}{14809}
\saveTG{𩓸}{14894}
\saveTG{𧷱}{14894}
\saveTG{𦀧}{14903}
\saveTG{𡬽}{14903}
\saveTG{櫫}{14904}
\saveTG{𥙝}{14917}
\saveTG{勡}{14927}
\saveTG{𠢢}{14927}
\saveTG{𥍲}{14927}
\saveTG{𠢜}{14927}
\saveTG{𥜣}{14927}
\saveTG{𥙼}{14931}
\saveTG{𢿖}{14940}
\saveTG{𠁩}{15021}
\saveTG{𧷤}{15086}
\saveTG{玤}{15100}
\saveTG{𤣸}{15100}
\saveTG{𤤭}{15106}
\saveTG{𤥭}{15106}
\saveTG{㻀}{15106}
\saveTG{䎈}{15106}
\saveTG{𤤺}{15106}
\saveTG{翀}{15106}
\saveTG{珅}{15106}
\saveTG{𠁨}{15106}
\saveTG{珒}{15107}
\saveTG{𤦥}{15107}
\saveTG{𤦯}{15107}
\saveTG{珄}{15110}
\saveTG{璶}{15112}
\saveTG{𩐔}{15112}
\saveTG{𤮍}{15114}
\saveTG{𤤀}{15117}
\saveTG{𤧎}{15122}
\saveTG{𦑊}{15127}
\saveTG{𤦭}{15127}
\saveTG{𤤖}{15127}
\saveTG{璛}{15127}
\saveTG{㺻}{15127}
\saveTG{靕}{15127}
\saveTG{玮}{15127}
\saveTG{翴}{15130}
\saveTG{璉}{15130}
\saveTG{琎}{15130}
\saveTG{𦑐}{15131}
\saveTG{㚂}{15132}
\saveTG{璤}{15133}
\saveTG{𦒎}{15133}
\saveTG{𧈬}{15136}
\saveTG{䎚}{15137}
\saveTG{𤪳}{15137}
\saveTG{瑼}{15143}
\saveTG{㻲}{15144}
\saveTG{𧰃}{15144}
\saveTG{珃}{15147}
\saveTG{𣼚}{15147}
\saveTG{𤤏}{15147}
\saveTG{𤧣}{15147}
\saveTG{𤪆}{15152}
\saveTG{瑇}{15157}
\saveTG{𦐘}{15157}
\saveTG{𧯵}{15158}
\saveTG{琫}{15158}
\saveTG{𤤧}{15160}
\saveTG{𦥌}{15161}
\saveTG{𤥢}{15162}
\saveTG{𡐋}{15166}
\saveTG{瑃}{15168}
\saveTG{𦒄}{15174}
\saveTG{㻰}{15177}
\saveTG{𣻛}{15177}
\saveTG{翐}{15180}
\saveTG{玦}{15180}
\saveTG{玞}{15180}
\saveTG{𦥃}{15181}
\saveTG{琠}{15181}
\saveTG{𦑈}{15181}
\saveTG{𪻺}{15182}
\saveTG{𦐍}{15182}
\saveTG{𤤥}{15182}
\saveTG{𤤠}{15182}
\saveTG{𦥇}{15184}
\saveTG{璳}{15186}
\saveTG{璝}{15186}
\saveTG{𧶷}{15186}
\saveTG{瓉}{15186}
\saveTG{珠}{15190}
\saveTG{𣥵}{15192}
\saveTG{㻷}{15192}
\saveTG{𦐣}{15192}
\saveTG{𤤹}{15192}
\saveTG{𤧉}{15193}
\saveTG{𤨄}{15193}
\saveTG{臻}{15194}
\saveTG{瑧}{15194}
\saveTG{㻧}{15194}
\saveTG{䜹}{15196}
\saveTG{𫇎}{15196}
\saveTG{瑓}{15196}
\saveTG{疎}{15196}
\saveTG{𢑧}{15196}
\saveTG{㻋}{15196}
\saveTG{𤦪}{15196}
\saveTG{𣓺}{15196}
\saveTG{𣦚}{15196}
\saveTG{㻖}{15199}
\saveTG{𤂰}{15199}
\saveTG{𪼃}{15199}
\saveTG{𣧘}{15202}
\saveTG{𣧨}{15202}
\saveTG{𠄯}{15203}
\saveTG{𦓕}{15206}
\saveTG{𣦰}{15206}
\saveTG{𫙄}{15206}
\saveTG{肂}{15207}
\saveTG{𩰫}{15207}
\saveTG{𧱇}{15207}
\saveTG{殅}{15210}
\saveTG{𩇖}{15212}
\saveTG{虺}{15213}
\saveTG{𥎓}{15216}
\saveTG{豘}{15217}
\saveTG{𢎫}{15217}
\saveTG{𪨇}{15218}
\saveTG{𠒺}{15218}
\saveTG{𢄥}{15227}
\saveTG{𪿽}{15227}
\saveTG{彇}{15227}
\saveTG{𧖓}{15231}
\saveTG{甤}{15231}
\saveTG{𢐪}{15232}
\saveTG{𫋊}{15236}
\saveTG{𣨃}{15236}
\saveTG{䂈}{15236}
\saveTG{融}{15236}
\saveTG{𢌤}{15237}
\saveTG{𥍹}{15240}
\saveTG{𣩘}{15243}
\saveTG{䝏}{15244}
\saveTG{𩱃}{15247}
\saveTG{㘱}{15247}
\saveTG{𫌖}{15256}
\saveTG{𣨞}{15258}
\saveTG{䏾}{15258}
\saveTG{𣩳}{15261}
\saveTG{𢏣}{15265}
\saveTG{𣩒}{15266}
\saveTG{殃}{15280}
\saveTG{𤩍}{15280}
\saveTG{𢏵}{15281}
\saveTG{𣧞}{15282}
\saveTG{𣧎}{15282}
\saveTG{㱮}{15282}
\saveTG{𪩺}{15282}
\saveTG{𪺧}{15282}
\saveTG{𢎹}{15282}
\saveTG{𧱅}{15282}
\saveTG{}{15282}
\saveTG{𥎝}{15282}
\saveTG{𧱪}{15284}
\saveTG{𣩬}{15286}
\saveTG{𣧴}{15286}
\saveTG{殨}{15286}
\saveTG{䂎}{15286}
\saveTG{𥎞}{15286}
\saveTG{㱴}{15286}
\saveTG{𥎍}{15286}
\saveTG{𣧣}{15290}
\saveTG{殊}{15290}
\saveTG{𣩊}{15294}
\saveTG{殝}{15294}
\saveTG{殐}{15296}
\saveTG{𫑽}{15296}
\saveTG{㱫}{15296}
\saveTG{殔}{15299}
\saveTG{𨖊}{15308}
\saveTG{䎴}{15400}
\saveTG{建}{15400}
\saveTG{𤯳}{15401}
\saveTG{𧈰}{15403}
\saveTG{𪪁}{15405}
\saveTG{廸}{15406}
\saveTG{𦕏}{15406}
\saveTG{䎶}{15406}
\saveTG{𪎌}{15409}
\saveTG{𤯩}{15412}
\saveTG{孻}{15412}
\saveTG{聙}{15427}
\saveTG{𡦗}{15427}
\saveTG{𦕚}{15427}
\saveTG{聘}{15427}
\saveTG{𦗳}{15432}
\saveTG{𫆓}{15432}
\saveTG{𡦕}{15443}
\saveTG{聃}{15447}
\saveTG{𪫊}{15458}
\saveTG{䏆}{15466}
\saveTG{𥓹}{15480}
\saveTG{𦔾}{15480}
\saveTG{𦖌}{15481}
\saveTG{聩}{15482}
\saveTG{聵}{15486}
\saveTG{𦏙}{15486}
\saveTG{𡥛}{15490}
\saveTG{䎷}{15490}
\saveTG{𦕜}{15490}
\saveTG{𡥤}{15490}
\saveTG{𦕽}{15496}
\saveTG{甦}{15501}
\saveTG{𩎳}{15507}
\saveTG{𨡁}{15517}
\saveTG{𩈈}{15600}
\saveTG{𥐹}{15600}
\saveTG{䂜}{15600}
\saveTG{𥕙}{15602}
\saveTG{𥔴}{15602}
\saveTG{砷}{15606}
\saveTG{硨}{15606}
\saveTG{𥒏}{15606}
\saveTG{𥒅}{15606}
\saveTG{𥑏}{15606}
\saveTG{硉}{15607}
\saveTG{𥑥}{15612}
\saveTG{𨠠}{15612}
\saveTG{𥑯}{15612}
\saveTG{醠}{15612}
\saveTG{硗}{15612}
\saveTG{砘}{15617}
\saveTG{䣨}{15617}
\saveTG{䣩}{15617}
\saveTG{𫑳}{15617}
\saveTG{𥐠}{15617}
\saveTG{醴}{15618}
\saveTG{𨠓}{15627}
\saveTG{砩}{15627}
\saveTG{碃}{15627}
\saveTG{䣪}{15627}
\saveTG{砵}{15630}
\saveTG{𪿛}{15631}
\saveTG{䃩}{15632}
\saveTG{𪿦}{15632}
\saveTG{醲}{15632}
\saveTG{砖}{15632}
\saveTG{䃛}{15635}
\saveTG{𪿞}{15643}
\saveTG{磚}{15643}
\saveTG{𥕍}{15644}
\saveTG{䃀}{15644}
\saveTG{䃓}{15647}
\saveTG{𪿵}{15654}
\saveTG{𥗇}{15656}
\saveTG{碡}{15657}
\saveTG{𥑤}{15660}
\saveTG{𥕢}{15666}
\saveTG{醩}{15666}
\saveTG{𨤇}{15669}
\saveTG{砆}{15680}
\saveTG{酜}{15680}
\saveTG{砄}{15680}
\saveTG{硖}{15680}
\saveTG{𥓐}{15681}
\saveTG{碘}{15681}
\saveTG{𥑇}{15682}
\saveTG{𫖃}{15682}
\saveTG{𥑞}{15682}
\saveTG{碛}{15682}
\saveTG{𥑷}{15686}
\saveTG{𨢦}{15686}
\saveTG{𨣵}{15686}
\saveTG{磧}{15686}
\saveTG{𨣜}{15686}
\saveTG{靧}{15686}
\saveTG{𫖀}{15690}
\saveTG{𥑶}{15690}
\saveTG{𥑘}{15690}
\saveTG{𩈐}{15690}
\saveTG{𨠝}{15690}
\saveTG{砞}{15690}
\saveTG{硃}{15690}
\saveTG{𨡏}{15691}
\saveTG{䣷}{15692}
\saveTG{𥔫}{15694}
\saveTG{磔}{15694}
\saveTG{}{15696}
\saveTG{𥓝}{15696}
\saveTG{𩃸}{15702}
\saveTG{瓲}{15711}
\saveTG{𤬸}{15712}
\saveTG{𤮃}{15712}
\saveTG{𪼹}{15713}
\saveTG{𩃌}{15717}
\saveTG{霕}{15717}
\saveTG{𧝚}{15732}
\saveTG{靆}{15739}
\saveTG{叇}{15739}
\saveTG{𠄴}{15752}
\saveTG{𩆦}{15766}
\saveTG{𤮤}{15782}
\saveTG{𪗫}{15782}
\saveTG{靅}{15786}
\saveTG{𪜞}{15794}
\saveTG{霴}{15799}
\saveTG{𥙍}{15906}
\saveTG{𦘞}{15907}
\saveTG{𫃉}{15917}
\saveTG{𥙌}{15927}
\saveTG{𣖓}{15942}
\saveTG{𣜇}{15986}
\saveTG{𤰓}{16000}
\saveTG{𥖉}{16096}
\saveTG{𤥪}{16100}
\saveTG{𪼓}{16100}
\saveTG{㺺}{16100}
\saveTG{𤤦}{16100}
\saveTG{㻁}{16100}
\saveTG{𤤨}{16100}
\saveTG{𤥳}{16100}
\saveTG{㻒}{16100}
\saveTG{𤧇}{16100}
\saveTG{珈}{16100}
\saveTG{珚}{16100}
\saveTG{𦕻}{16102}
\saveTG{𦐚}{16102}
\saveTG{珀}{16102}
\saveTG{聖}{16104}
\saveTG{𤧥}{16104}
\saveTG{䥒}{16109}
\saveTG{𦤾}{16110}
\saveTG{覴}{16112}
\saveTG{𤥛}{16112}
\saveTG{瑥}{16112}
\saveTG{現}{16112}
\saveTG{琨}{16112}
\saveTG{覡}{16112}
\saveTG{瑰}{16113}
\saveTG{𤭯}{16113}
\saveTG{𤩏}{16114}
\saveTG{𤨱}{16114}
\saveTG{𦑠}{16114}
\saveTG{𤪯}{16114}
\saveTG{珵}{16114}
\saveTG{瑝}{16114}
\saveTG{㼈}{16115}
\saveTG{理}{16115}
\saveTG{瑆}{16115}
\saveTG{𧡀}{16117}
\saveTG{𪾂}{16117}
\saveTG{𤪄}{16117}
\saveTG{䚈}{16117}
\saveTG{䰱}{16117}
\saveTG{䚖}{16117}
\saveTG{𧠫}{16117}
\saveTG{𧢯}{16117}
\saveTG{𤨆}{16117}
\saveTG{𧢮}{16117}
\saveTG{𤮟}{16117}
\saveTG{𪼚}{16117}
\saveTG{䶍}{16121}
\saveTG{𤫝}{16127}
\saveTG{𤦛}{16127}
\saveTG{㻦}{16127}
\saveTG{㻛}{16127}
\saveTG{𧰂}{16127}
\saveTG{㻿}{16127}
\saveTG{𤦝}{16127}
\saveTG{𦑼}{16127}
\saveTG{𤥰}{16127}
\saveTG{𦑀}{16127}
\saveTG{𤁗}{16127}
\saveTG{瑒}{16127}
\saveTG{琄}{16127}
\saveTG{琾}{16128}
\saveTG{璁}{16130}
\saveTG{𤨒}{16130}
\saveTG{𤨌}{16130}
\saveTG{𤄋}{16130}
\saveTG{𤪹}{16132}
\saveTG{𤧖}{16132}
\saveTG{𤨔}{16132}
\saveTG{環}{16132}
\saveTG{𤫛}{16132}
\saveTG{䝃}{16133}
\saveTG{𤫐}{16133}
\saveTG{𧓮}{16135}
\saveTG{豍}{16140}
\saveTG{琕}{16140}
\saveTG{琝}{16140}
\saveTG{𤥚}{16141}
\saveTG{𪼢}{16141}
\saveTG{𢎔}{16142}
\saveTG{瓔}{16144}
\saveTG{𤦑}{16147}
\saveTG{𤩛}{16147}
\saveTG{𤪛}{16148}
\saveTG{𤫠}{16148}
\saveTG{玾}{16150}
\saveTG{㻫}{16154}
\saveTG{𧰆}{16154}
\saveTG{𤩧}{16156}
\saveTG{𫆐}{16156}
\saveTG{𤩜}{16160}
\saveTG{琩}{16160}
\saveTG{瓃}{16160}
\saveTG{瑁}{16160}
\saveTG{𤫥}{16174}
\saveTG{𧵼}{16180}
\saveTG{珼}{16180}
\saveTG{𦒖}{16181}
\saveTG{珿}{16181}
\saveTG{瑅}{16181}
\saveTG{𧯩}{16182}
\saveTG{䎌}{16182}
\saveTG{䜻}{16182}
\saveTG{𦑧}{16182}
\saveTG{㻍}{16184}
\saveTG{𦑰}{16186}
\saveTG{瑔}{16192}
\saveTG{𤫤}{16193}
\saveTG{𤂽}{16193}
\saveTG{𤥯}{16194}
\saveTG{𤦸}{16194}
\saveTG{𪻼}{16194}
\saveTG{𤪒}{16194}
\saveTG{璪}{16194}
\saveTG{𤪁}{16196}
\saveTG{璟}{16196}
\saveTG{𦑶}{16199}
\saveTG{𢏎}{16200}
\saveTG{𪪺}{16200}
\saveTG{𢐚}{16200}
\saveTG{𥎙}{16200}
\saveTG{𢐑}{16200}
\saveTG{𣧝}{16200}
\saveTG{𥍡}{16202}
\saveTG{𣨔}{16202}
\saveTG{𠒜}{16211}
\saveTG{𩱽}{16211}
\saveTG{𧠱}{16212}
\saveTG{𧱗}{16212}
\saveTG{𧱟}{16212}
\saveTG{𧇙}{16212}
\saveTG{覼}{16212}
\saveTG{豱}{16212}
\saveTG{殟}{16212}
\saveTG{䰰}{16213}
\saveTG{𩴲}{16213}
\saveTG{𩴢}{16213}
\saveTG{䰭}{16213}
\saveTG{𥎡}{16215}
\saveTG{𨾣}{16215}
\saveTG{𣩿}{16215}
\saveTG{𣨾}{16215}
\saveTG{䚕}{16217}
\saveTG{𨞯}{16217}
\saveTG{𥘱}{16217}
\saveTG{𩲡}{16217}
\saveTG{𣨆}{16217}
\saveTG{𪵄}{16217}
\saveTG{𩳿}{16217}
\saveTG{𣩴}{16217}
\saveTG{𧠅}{16217}
\saveTG{𨛛}{16217}
\saveTG{㱱}{16217}
\saveTG{𧢋}{16217}
\saveTG{𪖕}{16221}
\saveTG{𪿇}{16227}
\saveTG{𥍣}{16227}
\saveTG{𣨵}{16227}
\saveTG{𣨟}{16227}
\saveTG{𣄋}{16227}
\saveTG{弲}{16227}
\saveTG{𥍴}{16227}
\saveTG{𥎋}{16230}
\saveTG{𪵿}{16232}
\saveTG{𧱨}{16232}
\saveTG{㱬}{16232}
\saveTG{𤃄}{16232}
\saveTG{𦒠}{16232}
\saveTG{彋}{16232}
\saveTG{𣩯}{16233}
\saveTG{强}{16236}
\saveTG{𢏥}{16241}
\saveTG{殬}{16241}
\saveTG{𣨨}{16247}
\saveTG{矡}{16247}
\saveTG{𪪾}{16247}
\saveTG{𣩡}{16247}
\saveTG{彏}{16247}
\saveTG{𫀤}{16248}
\saveTG{彃}{16254}
\saveTG{彈}{16256}
\saveTG{殫}{16256}
\saveTG{𣨕}{16262}
\saveTG{𢅦}{16282}
\saveTG{𤭍}{16282}
\saveTG{殒}{16282}
\saveTG{𢐂}{16282}
\saveTG{殠}{16284}
\saveTG{𣩥}{16286}
\saveTG{𧲆}{16286}
\saveTG{殞}{16286}
\saveTG{𣨁}{16292}
\saveTG{𥚌}{16294}
\saveTG{𣨪}{16294}
\saveTG{𥎅}{16299}
\saveTG{䂍}{16299}
\saveTG{𧲐}{16299}
\saveTG{𪇚}{16327}
\saveTG{𢙢}{16330}
\saveTG{𢦉}{16331}
\saveTG{𢦋}{16339}
\saveTG{𤲌}{16400}
\saveTG{廹}{16400}
\saveTG{𡥌}{16400}
\saveTG{𣅄}{16400}
\saveTG{廻}{16400}
\saveTG{廽}{16400}
\saveTG{𦔻}{16400}
\saveTG{𢌪}{16401}
\saveTG{𪪱}{16401}
\saveTG{𢌥}{16401}
\saveTG{𪦺}{16402}
\saveTG{𡥵}{16412}
\saveTG{𡥴}{16412}
\saveTG{𪧁}{16412}
\saveTG{𣌃}{16412}
\saveTG{𦖃}{16412}
\saveTG{聭}{16413}
\saveTG{𫆅}{16414}
\saveTG{𦕸}{16415}
\saveTG{𦖤}{16415}
\saveTG{䙹}{16417}
\saveTG{𫌟}{16417}
\saveTG{𧠍}{16417}
\saveTG{𨛿}{16417}
\saveTG{𨚆}{16417}
\saveTG{𦘂}{16427}
\saveTG{𪟤}{16427}
\saveTG{𦖛}{16427}
\saveTG{𦕮}{16427}
\saveTG{勥}{16427}
\saveTG{𦖻}{16430}
\saveTG{聰}{16430}
\saveTG{𦗣}{16431}
\saveTG{䏉}{16433}
\saveTG{𦗻}{16436}
\saveTG{𡞺}{16440}
\saveTG{聛}{16440}
\saveTG{𡠝}{16441}
\saveTG{孾}{16444}
\saveTG{孭}{16480}
\saveTG{职}{16480}
\saveTG{𦖍}{16494}
\saveTG{𦗵}{16494}
\saveTG{犟}{16503}
\saveTG{𥕏}{16600}
\saveTG{䂩}{16600}
\saveTG{𥑆}{16600}
\saveTG{𥔆}{16600}
\saveTG{𫑵}{16600}
\saveTG{碅}{16600}
\saveTG{硱}{16600}
\saveTG{硘}{16600}
\saveTG{謽}{16601}
\saveTG{硇}{16602}
\saveTG{砶}{16602}
\saveTG{𥒕}{16602}
\saveTG{𥓁}{16602}
\saveTG{𨠘}{16602}
\saveTG{碧}{16602}
\saveTG{𥑲}{16610}
\saveTG{𩈍}{16610}
\saveTG{𨠚}{16610}
\saveTG{䃂}{16612}
\saveTG{𥗒}{16612}
\saveTG{𥔋}{16612}
\saveTG{醖}{16612}
\saveTG{覠}{16612}
\saveTG{𥓣}{16612}
\saveTG{醌}{16612}
\saveTG{靦}{16612}
\saveTG{磇}{16612}
\saveTG{覗}{16612}
\saveTG{𥓀}{16612}
\saveTG{硯}{16612}
\saveTG{醞}{16612}
\saveTG{醜}{16613}
\saveTG{磈}{16613}
\saveTG{𥗬}{16614}
\saveTG{酲}{16614}
\saveTG{𥗴}{16615}
\saveTG{𩉗}{16615}
\saveTG{𩉙}{16615}
\saveTG{醒}{16615}
\saveTG{䃏}{16615}
\saveTG{𥓄}{16615}
\saveTG{𥗫}{16615}
\saveTG{𥕄}{16615}
\saveTG{𨛦}{16617}
\saveTG{䙼}{16617}
\saveTG{𧢇}{16617}
\saveTG{𨢄}{16617}
\saveTG{𧢥}{16617}
\saveTG{𩵀}{16617}
\saveTG{𨡠}{16627}
\saveTG{𥔲}{16627}
\saveTG{碭}{16627}
\saveTG{碣}{16627}
\saveTG{𦧭}{16627}
\saveTG{䃇}{16627}
\saveTG{𪿠}{16627}
\saveTG{䣺}{16627}
\saveTG{𥖠}{16627}
\saveTG{𥓘}{16627}
\saveTG{𥔘}{16627}
\saveTG{𥔅}{16628}
\saveTG{𨡾}{16630}
\saveTG{𨢨}{16630}
\saveTG{𨡹}{16632}
\saveTG{𥘀}{16632}
\saveTG{𦌺}{16632}
\saveTG{𥔯}{16632}
\saveTG{碨}{16632}
\saveTG{礘}{16633}
\saveTG{𤄥}{16633}
\saveTG{䃨}{16636}
\saveTG{碑}{16640}
\saveTG{礋}{16641}
\saveTG{𨣠}{16641}
\saveTG{碍}{16641}
\saveTG{醳}{16641}
\saveTG{𥕣}{16641}
\saveTG{𨢼}{16642}
\saveTG{𨡟}{16644}
\saveTG{𨡕}{16645}
\saveTG{𥕸}{16647}
\saveTG{𨣅}{16647}
\saveTG{𨢥}{16647}
\saveTG{礹}{16648}
\saveTG{釅}{16648}
\saveTG{𥑐}{16650}
\saveTG{𨢡}{16654}
\saveTG{𩉁}{16656}
\saveTG{𦧴}{16656}
\saveTG{磾}{16656}
\saveTG{礧}{16660}
\saveTG{𥓥}{16660}
\saveTG{𥗟}{16661}
\saveTG{𥖊}{16662}
\saveTG{𥖓}{16662}
\saveTG{𥓅}{16662}
\saveTG{醍}{16681}
\saveTG{碮}{16681}
\saveTG{𥒭}{16682}
\saveTG{𩈤}{16682}
\saveTG{𦧧}{16682}
\saveTG{𨡺}{16682}
\saveTG{𩈸}{16684}
\saveTG{𨣈}{16686}
\saveTG{磒}{16686}
\saveTG{𪿧}{16692}
\saveTG{𦧲}{16693}
\saveTG{磥}{16693}
\saveTG{𥗼}{16693}
\saveTG{𥖨}{16694}
\saveTG{𥓖}{16694}
\saveTG{䂺}{16694}
\saveTG{𥔟}{16694}
\saveTG{𥗋}{16699}
\saveTG{䤖}{16699}
\saveTG{𣄽}{16710}
\saveTG{𠮞}{16710}
\saveTG{乪}{16710}
\saveTG{瓼}{16711}
\saveTG{魂}{16713}
\saveTG{𪼿}{16714}
\saveTG{𦉦}{16772}
\saveTG{𡼮}{16796}
\saveTG{㠬}{16805}
\saveTG{覝}{16812}
\saveTG{𧠣}{16817}
\saveTG{𧡫}{16817}
\saveTG{𫖧}{16896}
\saveTG{𤨴}{16903}
\saveTG{𥞣}{16904}
\saveTG{𠀶}{16908}
\saveTG{𥜩}{16912}
\saveTG{䚄}{16917}
\saveTG{𧢀}{16917}
\saveTG{𧢄}{16917}
\saveTG{𩲚}{16917}
\saveTG{𨞮}{16917}
\saveTG{𣟮}{16947}
\saveTG{𥛴}{16981}
\saveTG{乛}{17000}
\saveTG{}{17010}
\saveTG{}{17020}
\saveTG{𠃍}{17020}
\saveTG{𠄎}{17027}
\saveTG{𨞖}{17027}
\saveTG{弓}{17027}
\saveTG{乁}{17030}
\saveTG{𠽗}{17032}
\saveTG{𡬢}{17041}
\saveTG{𩐆}{17101}
\saveTG{𠄪}{17101}
\saveTG{}{17101}
\saveTG{孟}{17102}
\saveTG{卫}{17102}
\saveTG{盈}{17102}
\saveTG{盁}{17102}
\saveTG{盄}{17102}
\saveTG{𢀖}{17102}
\saveTG{𥍝}{17102}
\saveTG{𣥒}{17102}
\saveTG{丑}{17102}
\saveTG{𥁂}{17102}
\saveTG{𥁛}{17102}
\saveTG{𥂂}{17102}
\saveTG{𥁪}{17102}
\saveTG{𥁝}{17102}
\saveTG{㿿}{17102}
\saveTG{𥁥}{17102}
\saveTG{𥂔}{17102}
\saveTG{𥂨}{17102}
\saveTG{𪾞}{17102}
\saveTG{𥁏}{17102}
\saveTG{𥁀}{17102}
\saveTG{𥁜}{17102}
\saveTG{丒}{17102}
\saveTG{弖}{17102}
\saveTG{盝}{17102}
\saveTG{𡒍}{17104}
\saveTG{𡏍}{17104}
\saveTG{𡋆}{17104}
\saveTG{𡔛}{17104}
\saveTG{𡉔}{17104}
\saveTG{𡎳}{17104}
\saveTG{𤪡}{17104}
\saveTG{𤣼}{17104}
\saveTG{𨥲}{17104}
\saveTG{埾}{17104}
\saveTG{亟}{17104}
\saveTG{𡤽}{17104}
\saveTG{𥻫}{17104}
\saveTG{𥸨}{17104}
\saveTG{𡍶}{17104}
\saveTG{𤦀}{17104}
\saveTG{𥔊}{17104}
\saveTG{𡌯}{17104}
\saveTG{𡐲}{17104}
\saveTG{䍿}{17104}
\saveTG{𡏄}{17104}
\saveTG{䥐}{17104}
\saveTG{𤧬}{17104}
\saveTG{𤦟}{17104}
\saveTG{𨧎}{17104}
\saveTG{𡊼}{17104}
\saveTG{𠯉}{17106}
\saveTG{㺾}{17106}
\saveTG{𧰈}{17106}
\saveTG{疍}{17106}
\saveTG{𤣹}{17107}
\saveTG{𧯢}{17108}
\saveTG{翌}{17108}
\saveTG{𥪷}{17108}
\saveTG{𢎮}{17108}
\saveTG{丞}{17109}
\saveTG{銎}{17109}
\saveTG{𨦠}{17109}
\saveTG{𦑕}{17109}
\saveTG{珟}{17110}
\saveTG{虱}{17110}
\saveTG{珮}{17110}
\saveTG{玑}{17110}
\saveTG{巩}{17110}
\saveTG{𩙄}{17110}
\saveTG{𧒢}{17110}
\saveTG{𡈾}{17110}
\saveTG{𣽝}{17110}
\saveTG{𤦍}{17110}
\saveTG{㺲}{17110}
\saveTG{𤧑}{17110}
\saveTG{𢀜}{17110}
\saveTG{𤣲}{17110}
\saveTG{歰}{17111}
\saveTG{𣵬}{17111}
\saveTG{现}{17112}
\saveTG{觋}{17112}
\saveTG{璏}{17112}
\saveTG{珇}{17112}
\saveTG{玸}{17112}
\saveTG{𤦕}{17112}
\saveTG{𧖛}{17112}
\saveTG{𫇐}{17114}
\saveTG{𪼑}{17114}
\saveTG{𤦶}{17115}
\saveTG{𪻭}{17116}
\saveTG{𤤲}{17117}
\saveTG{𤩑}{17117}
\saveTG{䜸}{17117}
\saveTG{𦏸}{17117}
\saveTG{𥏰}{17117}
\saveTG{㻊}{17117}
\saveTG{𤤗}{17117}
\saveTG{𦤽}{17117}
\saveTG{𧢱}{17117}
\saveTG{𦐆}{17117}
\saveTG{㺬}{17117}
\saveTG{𤤌}{17117}
\saveTG{𤣱}{17117}
\saveTG{玘}{17117}
\saveTG{𦐙}{17117}
\saveTG{䎂}{17117}
\saveTG{𣾵}{17117}
\saveTG{𤧾}{17117}
\saveTG{𤥕}{17117}
\saveTG{𤩃}{17117}
\saveTG{𪼻}{17118}
\saveTG{瓓}{17120}
\saveTG{珋}{17120}
\saveTG{卭}{17120}
\saveTG{翑}{17120}
\saveTG{习}{17120}
\saveTG{珦}{17120}
\saveTG{珝}{17120}
\saveTG{珣}{17120}
\saveTG{玥}{17120}
\saveTG{㓛}{17120}
\saveTG{𢀕}{17120}
\saveTG{羽}{17120}
\saveTG{瑚}{17120}
\saveTG{玽}{17120}
\saveTG{琱}{17120}
\saveTG{玓}{17120}
\saveTG{刁}{17120}
\saveTG{𤤑}{17120}
\saveTG{}{17120}
\saveTG{𦐰}{17120}
\saveTG{𪻣}{17120}
\saveTG{𠀄}{17120}
\saveTG{𦏼}{17120}
\saveTG{𤥋}{17120}
\saveTG{䑒}{17121}
\saveTG{𤥊}{17121}
\saveTG{㻚}{17121}
\saveTG{𤣽}{17121}
\saveTG{𪼙}{17121}
\saveTG{𤨡}{17121}
\saveTG{𤤂}{17121}
\saveTG{𦐱}{17121}
\saveTG{𤄡}{17121}
\saveTG{𪻫}{17121}
\saveTG{𤤡}{17121}
\saveTG{𣴆}{17121}
\saveTG{𤤅}{17122}
\saveTG{璆}{17122}
\saveTG{𣥛}{17123}
\saveTG{𣷋}{17123}
\saveTG{𤤢}{17124}
\saveTG{𤤪}{17124}
\saveTG{𤦳}{17124}
\saveTG{𧯠}{17126}
\saveTG{𤩎}{17126}
\saveTG{𦑟}{17126}
\saveTG{𪻛}{17126}
\saveTG{𦐥}{17126}
\saveTG{𣵮}{17127}
\saveTG{邶}{17127}
\saveTG{玚}{17127}
\saveTG{鵄}{17127}
\saveTG{鄧}{17127}
\saveTG{郖}{17127}
\saveTG{翵}{17127}
\saveTG{郅}{17127}
\saveTG{鵛}{17127}
\saveTG{鄄}{17127}
\saveTG{璚}{17127}
\saveTG{瑯}{17127}
\saveTG{玛}{17127}
\saveTG{邳}{17127}
\saveTG{邛}{17127}
\saveTG{弱}{17127}
\saveTG{鶸}{17127}
\saveTG{瑦}{17127}
\saveTG{鵐}{17127}
\saveTG{鵡}{17127}
\saveTG{鹀}{17127}
\saveTG{琊}{17127}
\saveTG{瑘}{17127}
\saveTG{鵶}{17127}
\saveTG{鳿}{17127}
\saveTG{鴊}{17127}
\saveTG{䴒}{17127}
\saveTG{𨚑}{17127}
\saveTG{𪜜}{17127}
\saveTG{䲹}{17127}
\saveTG{𦖼}{17127}
\saveTG{䲨}{17127}
\saveTG{𩿸}{17127}
\saveTG{𡍯}{17127}
\saveTG{𨞉}{17127}
\saveTG{𢎥}{17127}
\saveTG{𡖩}{17127}
\saveTG{𨛃}{17127}
\saveTG{𨛆}{17127}
\saveTG{𨛄}{17127}
\saveTG{𨜈}{17127}
\saveTG{𨙺}{17127}
\saveTG{䣁}{17127}
\saveTG{𣹍}{17127}
\saveTG{𨚝}{17127}
\saveTG{𪼫}{17127}
\saveTG{𤩬}{17127}
\saveTG{𨚎}{17127}
\saveTG{𤥶}{17127}
\saveTG{𤧨}{17127}
\saveTG{䣆}{17127}
\saveTG{𤥀}{17127}
\saveTG{𤥌}{17127}
\saveTG{𤧗}{17127}
\saveTG{𧯨}{17127}
\saveTG{𤧢}{17127}
\saveTG{𤥸}{17127}
\saveTG{𨚣}{17127}
\saveTG{𤧡}{17127}
\saveTG{𨟢}{17127}
\saveTG{𤧱}{17127}
\saveTG{鹉}{17127}
\saveTG{𨟘}{17127}
\saveTG{𫑚}{17127}
\saveTG{𤩉}{17127}
\saveTG{𨞟}{17127}
\saveTG{𡤼}{17127}
\saveTG{𨚱}{17127}
\saveTG{𡌇}{17127}
\saveTG{𨝽}{17127}
\saveTG{}{17127}
\saveTG{𨟯}{17127}
\saveTG{𪁞}{17127}
\saveTG{𪇏}{17127}
\saveTG{𪂐}{17127}
\saveTG{𩿍}{17127}
\saveTG{𪄕}{17127}
\saveTG{𩿱}{17127}
\saveTG{𣱾}{17127}
\saveTG{𩾗}{17127}
\saveTG{𪀑}{17127}
\saveTG{𩿮}{17127}
\saveTG{𪃋}{17127}
\saveTG{𪀫}{17127}
\saveTG{䳾}{17127}
\saveTG{𤦇}{17127}
\saveTG{𤤚}{17127}
\saveTG{𤤊}{17127}
\saveTG{𪉒}{17127}
\saveTG{𦒔}{17127}
\saveTG{𠖖}{17127}
\saveTG{𦒇}{17127}
\saveTG{𤧓}{17127}
\saveTG{𤥈}{17127}
\saveTG{𤩈}{17127}
\saveTG{㶄}{17127}
\saveTG{𦒛}{17127}
\saveTG{𦒞}{17127}
\saveTG{㢧}{17127}
\saveTG{𧔨}{17131}
\saveTG{𠘑}{17131}
\saveTG{𧌣}{17131}
\saveTG{𤧽}{17132}
\saveTG{𤦏}{17132}
\saveTG{𤧚}{17132}
\saveTG{𤩪}{17132}
\saveTG{𣼢}{17132}
\saveTG{珢}{17132}
\saveTG{瑑}{17132}
\saveTG{𤫈}{17132}
\saveTG{𤧷}{17132}
\saveTG{𤤮}{17133}
\saveTG{璭}{17135}
\saveTG{㻱}{17135}
\saveTG{蟊}{17136}
\saveTG{蝨}{17136}
\saveTG{瑵}{17136}
\saveTG{蠿}{17136}
\saveTG{𧊡}{17136}
\saveTG{𫊼}{17136}
\saveTG{𧉨}{17136}
\saveTG{𧋳}{17136}
\saveTG{䖥}{17136}
\saveTG{𩸇}{17136}
\saveTG{𧎻}{17136}
\saveTG{𧕑}{17136}
\saveTG{𧈲}{17136}
\saveTG{蛋}{17136}
\saveTG{蛩}{17136}
\saveTG{𤧫}{17137}
\saveTG{珘}{17140}
\saveTG{㻓}{17140}
\saveTG{𤪨}{17140}
\saveTG{琡}{17140}
\saveTG{玬}{17140}
\saveTG{珊}{17140}
\saveTG{𤤶}{17141}
\saveTG{𤧅}{17141}
\saveTG{𤣭}{17141}
\saveTG{𦤶}{17142}
\saveTG{䜷}{17142}
\saveTG{𤨿}{17143}
\saveTG{𤩆}{17144}
\saveTG{𦐷}{17144}
\saveTG{𤦾}{17144}
\saveTG{璎}{17144}
\saveTG{𣸹}{17144}
\saveTG{璕}{17146}
\saveTG{𤩞}{17147}
\saveTG{𤧯}{17147}
\saveTG{瑖}{17147}
\saveTG{珉}{17147}
\saveTG{瓊}{17147}
\saveTG{瑕}{17147}
\saveTG{𦐏}{17147}
\saveTG{𠭱}{17147}
\saveTG{𢀤}{17147}
\saveTG{㺭}{17147}
\saveTG{𤩚}{17147}
\saveTG{𤩮}{17147}
\saveTG{𣪅}{17147}
\saveTG{𣪌}{17147}
\saveTG{𤪢}{17147}
\saveTG{𤤄}{17147}
\saveTG{𤩴}{17147}
\saveTG{㻵}{17147}
\saveTG{𣾮}{17147}
\saveTG{𡥆}{17147}
\saveTG{𣫟}{17147}
\saveTG{𪻏}{17147}
\saveTG{𤥜}{17147}
\saveTG{𤣻}{17147}
\saveTG{}{17147}
\saveTG{𤥘}{17147}
\saveTG{𤦞}{17147}
\saveTG{璻}{17148}
\saveTG{𪼟}{17152}
\saveTG{琿}{17152}
\saveTG{𤦖}{17152}
\saveTG{𣸏}{17152}
\saveTG{𪼰}{17152}
\saveTG{𡍅}{17153}
\saveTG{琒}{17154}
\saveTG{珲}{17154}
\saveTG{𣹼}{17155}
\saveTG{䜶}{17157}
\saveTG{琤}{17157}
\saveTG{𤦱}{17160}
\saveTG{𪼤}{17161}
\saveTG{𤩇}{17161}
\saveTG{𤪅}{17161}
\saveTG{䝀}{17162}
\saveTG{𤥁}{17162}
\saveTG{瑠}{17162}
\saveTG{𪼪}{17162}
\saveTG{𧰏}{17162}
\saveTG{𪴻}{17162}
\saveTG{玿}{17162}
\saveTG{𡅡}{17164}
\saveTG{𦐦}{17164}
\saveTG{臵}{17164}
\saveTG{琚}{17164}
\saveTG{珞}{17164}
\saveTG{璐}{17164}
\saveTG{瑉}{17164}
\saveTG{珺}{17167}
\saveTG{瑂}{17167}
\saveTG{𤦿}{17168}
\saveTG{𠨔}{17170}
\saveTG{彐}{17170}
\saveTG{𣽶}{17171}
\saveTG{𪻱}{17172}
\saveTG{㻕}{17172}
\saveTG{瑤}{17172}
\saveTG{𠃠}{17174}
\saveTG{𤤷}{17177}
\saveTG{𢑰}{17177}
\saveTG{𧯰}{17177}
\saveTG{𢑑}{17177}
\saveTG{𤦆}{17177}
\saveTG{𡖈}{17177}
\saveTG{𢁇}{17177}
\saveTG{𦫠}{17177}
\saveTG{玖}{17180}
\saveTG{𤦚}{17181}
\saveTG{𤫂}{17181}
\saveTG{𤫌}{17181}
\saveTG{𤩄}{17181}
\saveTG{璵}{17181}
\saveTG{𣢈}{17182}
\saveTG{𦤷}{17182}
\saveTG{𣢶}{17182}
\saveTG{㰳}{17182}
\saveTG{𣤲}{17182}
\saveTG{㺵}{17182}
\saveTG{𤪦}{17182}
\saveTG{𤁒}{17182}
\saveTG{𣼋}{17182}
\saveTG{𣤪}{17182}
\saveTG{𪻲}{17182}
\saveTG{歅}{17182}
\saveTG{𪼣}{17184}
\saveTG{翭}{17184}
\saveTG{瑍}{17184}
\saveTG{𤧃}{17184}
\saveTG{𤥺}{17184}
\saveTG{𤧝}{17184}
\saveTG{𦑚}{17184}
\saveTG{𤫋}{17186}
\saveTG{瑻}{17186}
\saveTG{瓎}{17186}
\saveTG{㻏}{17189}
\saveTG{㻮}{17191}
\saveTG{珎}{17192}
\saveTG{𤩦}{17193}
\saveTG{𤦰}{17193}
\saveTG{𤤸}{17194}
\saveTG{𣺉}{17194}
\saveTG{𤨪}{17194}
\saveTG{璨}{17194}
\saveTG{琛}{17194}
\saveTG{𤁼}{17194}
\saveTG{𤧁}{17194}
\saveTG{𤨺}{17194}
\saveTG{瑈}{17194}
\saveTG{䎑}{17199}
\saveTG{瓈}{17199}
\saveTG{琭}{17199}
\saveTG{翏}{17202}
\saveTG{予}{17202}
\saveTG{𢇊}{17203}
\saveTG{𡰣}{17207}
\saveTG{𢎗}{17207}
\saveTG{㚉}{17207}
\saveTG{𫑸}{17207}
\saveTG{𢐫}{17207}
\saveTG{夛}{17207}
\saveTG{了}{17207}
\saveTG{𦒑}{17210}
\saveTG{𩘷}{17210}
\saveTG{䂇}{17210}
\saveTG{𤬫}{17211}
\saveTG{𥘲}{17211}
\saveTG{𠒉}{17211}
\saveTG{𠒢}{17211}
\saveTG{卼}{17211}
\saveTG{𫅣}{17212}
\saveTG{𥍨}{17212}
\saveTG{𩱂}{17212}
\saveTG{𪪽}{17212}
\saveTG{𩰰}{17212}
\saveTG{𡰅}{17212}
\saveTG{䲫}{17212}
\saveTG{𢏞}{17212}
\saveTG{豠}{17212}
\saveTG{殂}{17212}
\saveTG{弪}{17212}
\saveTG{殈}{17212}
\saveTG{𤰏}{17214}
\saveTG{𣧊}{17214}
\saveTG{𨹸}{17214}
\saveTG{𡯅}{17214}
\saveTG{殛}{17214}
\saveTG{𧱎}{17215}
\saveTG{𣩰}{17215}
\saveTG{翟}{17215}
\saveTG{𥛠}{17216}
\saveTG{𩘗}{17216}
\saveTG{𢀶}{17217}
\saveTG{弝}{17217}
\saveTG{豝}{17217}
\saveTG{孒}{17217}
\saveTG{𩰸}{17217}
\saveTG{𩆊}{17217}
\saveTG{䨲}{17217}
\saveTG{𤦤}{17217}
\saveTG{𤫅}{17217}
\saveTG{𦒏}{17217}
\saveTG{𠑻}{17217}
\saveTG{𢏱}{17217}
\saveTG{𧱧}{17217}
\saveTG{𦫰}{17217}
\saveTG{𣎜}{17217}
\saveTG{𦫣}{17217}
\saveTG{𣧜}{17217}
\saveTG{𣦽}{17217}
\saveTG{𣧼}{17217}
\saveTG{𣳣}{17217}
\saveTG{𪵀}{17217}
\saveTG{𣣰}{17217}
\saveTG{𠙛}{17217}
\saveTG{𢃳}{17217}
\saveTG{𦢱}{17217}
\saveTG{𢎪}{17217}
\saveTG{𠙶}{17220}
\saveTG{刀}{17220}
\saveTG{𩙅}{17220}
\saveTG{翮}{17220}
\saveTG{豞}{17220}
\saveTG{𥍞}{17220}
\saveTG{弸}{17220}
\saveTG{殉}{17220}
\saveTG{𧰶}{17220}
\saveTG{䂆}{17220}
\saveTG{歾}{17220}
\saveTG{𠨦}{17220}
\saveTG{𧱁}{17220}
\saveTG{𠄛}{17220}
\saveTG{𣩃}{17220}
\saveTG{𣍣}{17221}
\saveTG{𣍤}{17221}
\saveTG{𨸍}{17221}
\saveTG{𦑅}{17221}
\saveTG{𫋶}{17221}
\saveTG{㐨}{17221}
\saveTG{𦒧}{17221}
\saveTG{𢎻}{17221}
\saveTG{𪠉}{17221}
\saveTG{𠚪}{17221}
\saveTG{𣨥}{17221}
\saveTG{㱚}{17221}
\saveTG{𦏲}{17222}
\saveTG{矛}{17222}
\saveTG{𣩞}{17222}
\saveTG{𣩍}{17222}
\saveTG{𣧔}{17223}
\saveTG{𣧀}{17223}
\saveTG{㢩}{17223}
\saveTG{𧲃}{17224}
\saveTG{𫙅}{17224}
\saveTG{𤤱}{17224}
\saveTG{𣧺}{17225}
\saveTG{𠻸}{17226}
\saveTG{𣧬}{17226}
\saveTG{𢏕}{17226}
\saveTG{𢑪}{17226}
\saveTG{𩱗}{17227}
\saveTG{𩱟}{17227}
\saveTG{𩱧}{17227}
\saveTG{𩱭}{17227}
\saveTG{𩱤}{17227}
\saveTG{𩱆}{17227}
\saveTG{𩱡}{17227}
\saveTG{𩰲}{17227}
\saveTG{𩱎}{17227}
\saveTG{𩱳}{17227}
\saveTG{𩱌}{17227}
\saveTG{𨺜}{17227}
\saveTG{𥍩}{17227}
\saveTG{𦚇}{17227}
\saveTG{𡍒}{17227}
\saveTG{𩱈}{17227}
\saveTG{䴓}{17227}
\saveTG{𣨺}{17227}
\saveTG{𤰄}{17227}
\saveTG{𠜦}{17227}
\saveTG{𠱛}{17227}
\saveTG{𢅜}{17227}
\saveTG{𥍤}{17227}
\saveTG{𪱝}{17227}
\saveTG{𦛯}{17227}
\saveTG{𢂜}{17227}
\saveTG{𢁨}{17227}
\saveTG{𢂫}{17227}
\saveTG{𥎠}{17227}
\saveTG{𠄕}{17227}
\saveTG{𪱛}{17227}
\saveTG{𦐯}{17227}
\saveTG{𩱠}{17227}
\saveTG{㢽}{17227}
\saveTG{𩱒}{17227}
\saveTG{𢐥}{17227}
\saveTG{𢐅}{17227}
\saveTG{𢐼}{17227}
\saveTG{𢐾}{17227}
\saveTG{䰜}{17227}
\saveTG{𢐌}{17227}
\saveTG{𢏺}{17227}
\saveTG{𢎨}{17227}
\saveTG{𢐆}{17227}
\saveTG{𢐮}{17227}
\saveTG{𢐨}{17227}
\saveTG{𢐬}{17227}
\saveTG{𢐺}{17227}
\saveTG{𢐲}{17227}
\saveTG{𠨓}{17227}
\saveTG{𢐐}{17227}
\saveTG{𢑍}{17227}
\saveTG{𢑌}{17227}
\saveTG{𢐷}{17227}
\saveTG{𢐁}{17227}
\saveTG{𢏝}{17227}
\saveTG{㣃}{17227}
\saveTG{𢐹}{17227}
\saveTG{𢑺}{17227}
\saveTG{乃}{17227}
\saveTG{鄁}{17227}
\saveTG{弼}{17227}
\saveTG{弻}{17227}
\saveTG{邴}{17227}
\saveTG{脀}{17227}
\saveTG{鸐}{17227}
\saveTG{殦}{17227}
\saveTG{鳭}{17227}
\saveTG{甬}{17227}
\saveTG{鸸}{17227}
\saveTG{鴯}{17227}
\saveTG{夃}{17227}
\saveTG{翯}{17227}
\saveTG{鸖}{17227}
\saveTG{鄠}{17227}
\saveTG{弜}{17227}
\saveTG{鄹}{17227}
\saveTG{鬻}{17227}
\saveTG{矞}{17227}
\saveTG{郦}{17227}
\saveTG{鹂}{17227}
\saveTG{刕}{17227}
\saveTG{酈}{17227}
\saveTG{鸝}{17227}
\saveTG{鄝}{17227}
\saveTG{鹨}{17227}
\saveTG{鷚}{17227}
\saveTG{務}{17227}
\saveTG{鸍}{17227}
\saveTG{鼐}{17227}
\saveTG{帬}{17227}
\saveTG{鳾}{17227}
\saveTG{鷸}{17227}
\saveTG{鹬}{17227}
\saveTG{邒}{17227}
\saveTG{脋}{17227}
\saveTG{胥}{17227}
\saveTG{鷊}{17227}
\saveTG{鹝}{17227}
\saveTG{粥}{17227}
\saveTG{邧}{17227}
\saveTG{帚}{17227}
\saveTG{𨟁}{17227}
\saveTG{𧐘}{17227}
\saveTG{𨛤}{17227}
\saveTG{𡕔}{17227}
\saveTG{𨜄}{17227}
\saveTG{𨙱}{17227}
\saveTG{𨟓}{17227}
\saveTG{𨜃}{17227}
\saveTG{䣓}{17227}
\saveTG{𨜥}{17227}
\saveTG{𨟫}{17227}
\saveTG{𣨳}{17227}
\saveTG{𨚌}{17227}
\saveTG{𣨱}{17227}
\saveTG{𣧍}{17227}
\saveTG{𣨍}{17227}
\saveTG{𧨣}{17227}
\saveTG{𠃓}{17227}
\saveTG{𨞩}{17227}
\saveTG{𠀲}{17227}
\saveTG{𩫉}{17227}
\saveTG{𧱶}{17227}
\saveTG{𢎱}{17227}
\saveTG{𢏧}{17227}
\saveTG{𢏜}{17227}
\saveTG{𢐴}{17227}
\saveTG{𪇴}{17227}
\saveTG{𪈘}{17227}
\saveTG{𪈛}{17227}
\saveTG{𦛆}{17227}
\saveTG{𩿎}{17227}
\saveTG{𩿗}{17227}
\saveTG{𪃱}{17227}
\saveTG{𪂞}{17227}
\saveTG{𪁥}{17227}
\saveTG{䲯}{17227}
\saveTG{𪄮}{17227}
\saveTG{}{17227}
\saveTG{𩿷}{17227}
\saveTG{𪈑}{17227}
\saveTG{䲮}{17227}
\saveTG{𪂕}{17227}
\saveTG{𪈊}{17227}
\saveTG{𪈹}{17227}
\saveTG{𪇾}{17227}
\saveTG{𣧓}{17227}
\saveTG{𩾼}{17227}
\saveTG{𦐟}{17227}
\saveTG{𦑖}{17227}
\saveTG{𪇯}{17227}
\saveTG{𠀵}{17227}
\saveTG{𩾚}{17227}
\saveTG{𪄏}{17227}
\saveTG{𫙇}{17227}
\saveTG{𫙆}{17227}
\saveTG{𪂪}{17227}
\saveTG{𠝧}{17227}
\saveTG{𩱓}{17227}
\saveTG{𩱥}{17227}
\saveTG{𩱜}{17227}
\saveTG{𩱨}{17227}
\saveTG{𩱍}{17227}
\saveTG{𩱚}{17227}
\saveTG{䰞}{17227}
\saveTG{𩱞}{17227}
\saveTG{𩱰}{17227}
\saveTG{𩱶}{17227}
\saveTG{𩱖}{17227}
\saveTG{𩱫}{17227}
\saveTG{𩱮}{17227}
\saveTG{𩱣}{17227}
\saveTG{𩱲}{17227}
\saveTG{𩱷}{17227}
\saveTG{𩱯}{17227}
\saveTG{𩱦}{17227}
\saveTG{𩱱}{17228}
\saveTG{𥯵}{17228}
\saveTG{𣩼}{17229}
\saveTG{𣩝}{17229}
\saveTG{聚}{17232}
\saveTG{𤔛}{17232}
\saveTG{䝒}{17232}
\saveTG{𣵳}{17232}
\saveTG{𢐄}{17232}
\saveTG{𢑙}{17232}
\saveTG{𢑨}{17232}
\saveTG{𣨶}{17232}
\saveTG{承}{17232}
\saveTG{𥎒}{17232}
\saveTG{𦓖}{17232}
\saveTG{𠄘}{17232}
\saveTG{𣷗}{17232}
\saveTG{豫}{17232}
\saveTG{𠹩}{17232}
\saveTG{𣧾}{17232}
\saveTG{豤}{17232}
\saveTG{𥍷}{17232}
\saveTG{㸧}{17232}
\saveTG{𢐣}{17232}
\saveTG{𤃻}{17232}
\saveTG{䂊}{17232}
\saveTG{𣧩}{17233}
\saveTG{𧴂}{17235}
\saveTG{𥎌}{17235}
\saveTG{𪫅}{17236}
\saveTG{𢐗}{17236}
\saveTG{𪠤}{17241}
\saveTG{𥍟}{17242}
\saveTG{𣩗}{17243}
\saveTG{𣧱}{17244}
\saveTG{𦐩}{17244}
\saveTG{𥎟}{17246}
\saveTG{𤥞}{17247}
\saveTG{𤘅}{17247}
\saveTG{及}{17247}
\saveTG{𧱕}{17247}
\saveTG{𦐅}{17247}
\saveTG{𣪃}{17247}
\saveTG{𣪣}{17247}
\saveTG{𧰱}{17247}
\saveTG{歿}{17247}
\saveTG{𤥱}{17247}
\saveTG{䝌}{17247}
\saveTG{𧱣}{17247}
\saveTG{𣧟}{17247}
\saveTG{豭}{17247}
\saveTG{殁}{17247}
\saveTG{豛}{17247}
\saveTG{𣪏}{17247}
\saveTG{𧱛}{17247}
\saveTG{𠀨}{17247}
\saveTG{𣧉}{17247}
\saveTG{𢎤}{17247}
\saveTG{𢎽}{17247}
\saveTG{㢯}{17247}
\saveTG{𥍯}{17247}
\saveTG{𧰵}{17247}
\saveTG{㱭}{17247}
\saveTG{𣩌}{17252}
\saveTG{𥎎}{17252}
\saveTG{𧰷}{17253}
\saveTG{𢑟}{17253}
\saveTG{𧱆}{17255}
\saveTG{𣨿}{17256}
\saveTG{䝍}{17256}
\saveTG{𦑩}{17256}
\saveTG{鬸}{17262}
\saveTG{弨}{17262}
\saveTG{㱪}{17264}
\saveTG{𣧳}{17264}
\saveTG{𥎈}{17264}
\saveTG{𢏷}{17272}
\saveTG{𣨢}{17272}
\saveTG{𢏈}{17277}
\saveTG{𣩆}{17280}
\saveTG{𣧸}{17280}
\saveTG{𥎗}{17281}
\saveTG{𣣌}{17281}
\saveTG{弞}{17282}
\saveTG{𪵇}{17282}
\saveTG{𣤩}{17282}
\saveTG{𣶙}{17282}
\saveTG{㰷}{17282}
\saveTG{𣧋}{17282}
\saveTG{㺼}{17282}
\saveTG{䂉}{17284}
\saveTG{𥍶}{17284}
\saveTG{㢿}{17284}
\saveTG{𠟹}{17286}
\saveTG{𥎉}{17286}
\saveTG{𣩔}{17286}
\saveTG{𨽠}{17286}
\saveTG{𩆂}{17286}
\saveTG{𥎤}{17289}
\saveTG{𣧢}{17292}
\saveTG{弥}{17292}
\saveTG{殩}{17294}
\saveTG{𣧷}{17294}
\saveTG{}{17302}
\saveTG{龴}{17302}
\saveTG{䎆}{17302}
\saveTG{𫑏}{17303}
\saveTG{𨓥}{17306}
\saveTG{𫅯}{17309}
\saveTG{𢟃}{17310}
\saveTG{𩘸}{17310}
\saveTG{𢟄}{17314}
\saveTG{𩥡}{17317}
\saveTG{刃}{17320}
\saveTG{𫅰}{17322}
\saveTG{𨛑}{17327}
\saveTG{䳱}{17327}
\saveTG{𩥦}{17327}
\saveTG{𪆀}{17327}
\saveTG{𠟀}{17327}
\saveTG{𫛉}{17327}
\saveTG{𪀛}{17327}
\saveTG{𪅴}{17327}
\saveTG{䴇}{17327}
\saveTG{𨝻}{17327}
\saveTG{𩢽}{17327}
\saveTG{𪈄}{17327}
\saveTG{𩢗}{17327}
\saveTG{𢎚}{17327}
\saveTG{鵋}{17327}
\saveTG{鄩}{17327}
\saveTG{鄢}{17327}
\saveTG{}{17327}
\saveTG{刅}{17330}
\saveTG{𢝭}{17331}
\saveTG{𢙷}{17331}
\saveTG{焏}{17331}
\saveTG{𤊅}{17331}
\saveTG{忌}{17331}
\saveTG{恐}{17331}
\saveTG{烝}{17331}
\saveTG{𤆔}{17331}
\saveTG{恿}{17332}
\saveTG{忍}{17332}
\saveTG{𤑜}{17332}
\saveTG{𢢧}{17332}
\saveTG{𤆨}{17332}
\saveTG{𢞉}{17332}
\saveTG{𢘅}{17332}
\saveTG{𢞔}{17332}
\saveTG{𪫥}{17332}
\saveTG{𢖫}{17332}
\saveTG{𤆕}{17332}
\saveTG{𢜍}{17332}
\saveTG{𤎘}{17332}
\saveTG{𢡆}{17333}
\saveTG{𤇏}{17333}
\saveTG{𤉋}{17333}
\saveTG{焣}{17334}
\saveTG{𢙎}{17334}
\saveTG{孞}{17334}
\saveTG{𢚣}{17334}
\saveTG{㤂}{17334}
\saveTG{𢚫}{17334}
\saveTG{焄}{17336}
\saveTG{𩷡}{17336}
\saveTG{𩱪}{17337}
\saveTG{𢗂}{17337}
\saveTG{𢦀}{17338}
\saveTG{㝷}{17341}
\saveTG{𡬶}{17341}
\saveTG{𪧹}{17341}
\saveTG{𦏽}{17342}
\saveTG{尋}{17346}
\saveTG{寻}{17347}
\saveTG{𡬞}{17347}
\saveTG{𦐐}{17400}
\saveTG{𠦘}{17401}
\saveTG{𠦕}{17401}
\saveTG{𠦦}{17401}
\saveTG{聓}{17401}
\saveTG{翆}{17401}
\saveTG{𦖜}{17401}
\saveTG{𦏴}{17401}
\saveTG{𢌟}{17402}
\saveTG{𢌡}{17402}
\saveTG{㢠}{17402}
\saveTG{𢌞}{17402}
\saveTG{𣍇}{17404}
\saveTG{翣}{17404}
\saveTG{娶}{17404}
\saveTG{𡝠}{17404}
\saveTG{𣍛}{17405}
\saveTG{翇}{17407}
\saveTG{𦐓}{17407}
\saveTG{孓}{17407}
\saveTG{孠}{17407}
\saveTG{孕}{17407}
\saveTG{子}{17407}
\saveTG{𡥍}{17407}
\saveTG{𠭈}{17407}
\saveTG{𠬶}{17407}
\saveTG{孑}{17407}
\saveTG{𡕪}{17407}
\saveTG{𠭪}{17407}
\saveTG{𠬛}{17407}
\saveTG{𦕔}{17407}
\saveTG{𡕡}{17407}
\saveTG{𪍫}{17407}
\saveTG{翠}{17408}
\saveTG{𪏬}{17408}
\saveTG{𦐜}{17409}
\saveTG{𩙝}{17410}
\saveTG{䎲}{17410}
\saveTG{𨐾}{17410}
\saveTG{𩙀}{17410}
\saveTG{𠃨}{17410}
\saveTG{卂}{17410}
\saveTG{𡤾}{17411}
\saveTG{孢}{17412}
\saveTG{聣}{17412}
\saveTG{㝃}{17412}
\saveTG{𡥠}{17412}
\saveTG{𦏚}{17414}
\saveTG{𡔀}{17414}
\saveTG{𢐽}{17414}
\saveTG{𢥀}{17414}
\saveTG{𠙞}{17417}
\saveTG{𦕍}{17417}
\saveTG{𨚁}{17417}
\saveTG{䎳}{17420}
\saveTG{𦕘}{17420}
\saveTG{𦕋}{17420}
\saveTG{䎻}{17420}
\saveTG{𡥱}{17420}
\saveTG{𦕅}{17420}
\saveTG{𦖐}{17420}
\saveTG{𦖡}{17420}
\saveTG{𦘁}{17420}
\saveTG{𦗬}{17420}
\saveTG{𦘑}{17420}
\saveTG{𩘔}{17420}
\saveTG{𫆀}{17420}
\saveTG{覅}{17420}
\saveTG{聊}{17420}
\saveTG{𦖉}{17420}
\saveTG{𦘍}{17420}
\saveTG{𦕀}{17420}
\saveTG{𦖳}{17421}
\saveTG{𦑃}{17421}
\saveTG{𫆍}{17421}
\saveTG{䍾}{17421}
\saveTG{㜿}{17421}
\saveTG{𦗖}{17422}
\saveTG{𪦼}{17423}
\saveTG{邢}{17427}
\saveTG{郉}{17427}
\saveTG{邗}{17427}
\saveTG{鳽}{17427}
\saveTG{𦞋}{17427}
\saveTG{𨜦}{17427}
\saveTG{𫜃}{17427}
\saveTG{𨟡}{17427}
\saveTG{𫛚}{17427}
\saveTG{𨝸}{17427}
\saveTG{𪅷}{17427}
\saveTG{𦖄}{17427}
\saveTG{𪂮}{17427}
\saveTG{𦖙}{17427}
\saveTG{郔}{17427}
\saveTG{𪂍}{17427}
\saveTG{䳩}{17427}
\saveTG{𢑕}{17427}
\saveTG{𡦅}{17427}
\saveTG{鳵}{17427}
\saveTG{鵈}{17427}
\saveTG{鳱}{17427}
\saveTG{鷣}{17427}
\saveTG{郰}{17427}
\saveTG{耶}{17427}
\saveTG{𪄂}{17427}
\saveTG{勇}{17427}
\saveTG{鄾}{17427}
\saveTG{𫑬}{17427}
\saveTG{邘}{17427}
\saveTG{𦗏}{17427}
\saveTG{𨚾}{17427}
\saveTG{𣄸}{17427}
\saveTG{𡥾}{17427}
\saveTG{𦕬}{17427}
\saveTG{𡥥}{17427}
\saveTG{𦗒}{17427}
\saveTG{𨛓}{17427}
\saveTG{𦕙}{17429}
\saveTG{𦖹}{17431}
\saveTG{聦}{17432}
\saveTG{𦕨}{17432}
\saveTG{𪪄}{17432}
\saveTG{𦖆}{17432}
\saveTG{𦖟}{17432}
\saveTG{𡥧}{17433}
\saveTG{𦔹}{17440}
\saveTG{𦕃}{17440}
\saveTG{取}{17440}
\saveTG{刄}{17440}
\saveTG{𦐧}{17441}
\saveTG{𢆙}{17441}
\saveTG{彛}{17442}
\saveTG{𦐒}{17442}
\saveTG{𡛺}{17442}
\saveTG{𡜹}{17442}
\saveTG{羿}{17442}
\saveTG{㜈}{17442}
\saveTG{𢌼}{17443}
\saveTG{𪧀}{17444}
\saveTG{𢍟}{17446}
\saveTG{𧲕}{17446}
\saveTG{𡦷}{17447}
\saveTG{𡦪}{17447}
\saveTG{𦩧}{17447}
\saveTG{孖}{17447}
\saveTG{䏂}{17447}
\saveTG{𦖚}{17447}
\saveTG{孨}{17447}
\saveTG{𦕛}{17447}
\saveTG{𡦰}{17447}
\saveTG{𣣀}{17447}
\saveTG{𦖲}{17447}
\saveTG{彞}{17449}
\saveTG{𡥓}{17450}
\saveTG{耼}{17450}
\saveTG{𢌢}{17454}
\saveTG{𦗇}{17456}
\saveTG{𦐪}{17456}
\saveTG{𢥙}{17458}
\saveTG{𦗯}{17461}
\saveTG{聸}{17461}
\saveTG{𦗎}{17461}
\saveTG{𡥙}{17462}
\saveTG{𦗗}{17462}
\saveTG{𫆢}{17470}
\saveTG{𡥦}{17477}
\saveTG{䏃}{17480}
\saveTG{𫆁}{17482}
\saveTG{𣤾}{17482}
\saveTG{㱊}{17482}
\saveTG{𣣺}{17482}
\saveTG{𦘀}{17482}
\saveTG{㰢}{17482}
\saveTG{㱌}{17482}
\saveTG{𡥼}{17486}
\saveTG{𦘋}{17486}
\saveTG{𦕓}{17491}
\saveTG{䏅}{17491}
\saveTG{𦕗}{17492}
\saveTG{𦇚}{17493}
\saveTG{𡦞}{17494}
\saveTG{𦕰}{17494}
\saveTG{䎼}{17499}
\saveTG{羣}{17501}
\saveTG{㧬}{17502}
\saveTG{𢬙}{17502}
\saveTG{𢪻}{17502}
\saveTG{𢮝}{17502}
\saveTG{𢫴}{17502}
\saveTG{𢱵}{17502}
\saveTG{𢭤}{17502}
\saveTG{翬}{17502}
\saveTG{𢦝}{17503}
\saveTG{𤚉}{17504}
\saveTG{翚}{17504}
\saveTG{𨋬}{17505}
\saveTG{𨋑}{17506}
\saveTG{𢑖}{17506}
\saveTG{鞏}{17506}
\saveTG{𫖈}{17506}
\saveTG{𫏵}{17506}
\saveTG{尹}{17507}
\saveTG{丮}{17510}
\saveTG{𢩦}{17510}
\saveTG{𡀪}{17511}
\saveTG{𩲧}{17513}
\saveTG{𠜭}{17517}
\saveTG{𦐃}{17521}
\saveTG{𠂐}{17527}
\saveTG{𨜉}{17527}
\saveTG{𨞗}{17527}
\saveTG{𫛀}{17527}
\saveTG{弔}{17527}
\saveTG{郠}{17527}
\saveTG{那}{17527}
\saveTG{𡱉}{17557}
\saveTG{䡗}{17561}
\saveTG{𨢎}{17580}
\saveTG{𪴲}{17582}
\saveTG{㧭}{17592}
\saveTG{𩊳}{17596}
\saveTG{圅}{17600}
\saveTG{𦐮}{17600}
\saveTG{𠮯}{17601}
\saveTG{𧦬}{17601}
\saveTG{𧩞}{17601}
\saveTG{𥕔}{17601}
\saveTG{𠝽}{17601}
\saveTG{𣇶}{17601}
\saveTG{𡃠}{17601}
\saveTG{䛐}{17601}
\saveTG{𥓆}{17602}
\saveTG{叾}{17602}
\saveTG{𣊃}{17602}
\saveTG{𥒽}{17602}
\saveTG{𥑱}{17602}
\saveTG{𫖂}{17602}
\saveTG{𥔩}{17602}
\saveTG{𥉠}{17602}
\saveTG{𪽖}{17602}
\saveTG{𤳕}{17602}
\saveTG{召}{17602}
\saveTG{習}{17602}
\saveTG{𡇶}{17603}
\saveTG{呄}{17604}
\saveTG{𡥄}{17604}
\saveTG{𡦏}{17604}
\saveTG{𡥨}{17604}
\saveTG{䀾}{17604}
\saveTG{孴}{17604}
\saveTG{𦑥}{17604}
\saveTG{君}{17607}
\saveTG{𣆁}{17608}
\saveTG{矶}{17610}
\saveTG{飁}{17610}
\saveTG{𠮸}{17610}
\saveTG{矾}{17610}
\saveTG{𧟢}{17610}
\saveTG{𥐲}{17610}
\saveTG{𥐥}{17610}
\saveTG{砜}{17610}
\saveTG{碸}{17610}
\saveTG{𦑇}{17611}
\saveTG{靤}{17612}
\saveTG{𨠁}{17612}
\saveTG{䣯}{17612}
\saveTG{𩈦}{17612}
\saveTG{𠭚}{17612}
\saveTG{䃲}{17612}
\saveTG{𪿓}{17612}
\saveTG{𪿱}{17612}
\saveTG{𨣕}{17612}
\saveTG{𨡼}{17612}
\saveTG{𨡇}{17612}
\saveTG{𥓋}{17612}
\saveTG{硊}{17612}
\saveTG{𥒌}{17612}
\saveTG{𥐗}{17612}
\saveTG{砠}{17612}
\saveTG{硁}{17612}
\saveTG{砚}{17612}
\saveTG{䩄}{17612}
\saveTG{砲}{17612}
\saveTG{𥒡}{17613}
\saveTG{𥑋}{17614}
\saveTG{𥔒}{17614}
\saveTG{𫍊}{17614}
\saveTG{𥔄}{17614}
\saveTG{𩈇}{17614}
\saveTG{䃘}{17614}
\saveTG{𥒮}{17615}
\saveTG{𨟱}{17617}
\saveTG{𩈌}{17617}
\saveTG{𥐱}{17617}
\saveTG{𪜘}{17617}
\saveTG{𪿗}{17617}
\saveTG{𥒵}{17617}
\saveTG{𦫥}{17617}
\saveTG{𪓱}{17617}
\saveTG{𥒍}{17617}
\saveTG{𨠥}{17617}
\saveTG{𨠶}{17617}
\saveTG{𨠖}{17617}
\saveTG{配}{17617}
\saveTG{巶}{17617}
\saveTG{𥑁}{17617}
\saveTG{𥖖}{17617}
\saveTG{𩈆}{17617}
\saveTG{𥐦}{17617}
\saveTG{𪕡}{17617}
\saveTG{𩈮}{17620}
\saveTG{𩈄}{17620}
\saveTG{𥗮}{17620}
\saveTG{𥑎}{17620}
\saveTG{䃃}{17620}
\saveTG{𥒉}{17620}
\saveTG{𥐾}{17620}
\saveTG{磵}{17620}
\saveTG{𥒘}{17620}
\saveTG{𥒚}{17620}
\saveTG{𥑪}{17620}
\saveTG{𥓮}{17620}
\saveTG{卲}{17620}
\saveTG{砌}{17620}
\saveTG{𥔀}{17620}
\saveTG{𥑩}{17620}
\saveTG{𥑨}{17620}
\saveTG{𥑚}{17620}
\saveTG{𥖆}{17620}
\saveTG{䃹}{17620}
\saveTG{𥐛}{17620}
\saveTG{𧥝}{17620}
\saveTG{𥔓}{17620}
\saveTG{酮}{17620}
\saveTG{司}{17620}
\saveTG{碉}{17620}
\saveTG{𥐩}{17620}
\saveTG{硐}{17620}
\saveTG{碙}{17620}
\saveTG{醐}{17620}
\saveTG{𥐝}{17620}
\saveTG{𥒙}{17620}
\saveTG{礀}{17620}
\saveTG{硼}{17620}
\saveTG{𥕡}{17620}
\saveTG{醄}{17620}
\saveTG{矽}{17620}
\saveTG{砽}{17620}
\saveTG{酌}{17620}
\saveTG{𥐼}{17620}
\saveTG{𨣇}{17621}
\saveTG{𪿸}{17621}
\saveTG{𦒆}{17621}
\saveTG{䎄}{17621}
\saveTG{𪃿}{17621}
\saveTG{𨡒}{17621}
\saveTG{䂬}{17621}
\saveTG{𦐸}{17621}
\saveTG{䂙}{17621}
\saveTG{𨟴}{17621}
\saveTG{𦑭}{17621}
\saveTG{𨟸}{17622}
\saveTG{醪}{17622}
\saveTG{磟}{17622}
\saveTG{䂛}{17623}
\saveTG{䃤}{17624}
\saveTG{𪿕}{17626}
\saveTG{䣳}{17626}
\saveTG{𨣉}{17626}
\saveTG{䣱}{17626}
\saveTG{𠣫}{17626}
\saveTG{𨚹}{17627}
\saveTG{𨟰}{17627}
\saveTG{𨚩}{17627}
\saveTG{𨝆}{17627}
\saveTG{𫑪}{17627}
\saveTG{𠷎}{17627}
\saveTG{𤾊}{17627}
\saveTG{𢐉}{17627}
\saveTG{𠯶}{17627}
\saveTG{㢴}{17627}
\saveTG{𪈆}{17627}
\saveTG{𪅽}{17627}
\saveTG{𩈺}{17627}
\saveTG{𥕟}{17627}
\saveTG{䃖}{17627}
\saveTG{䲽}{17627}
\saveTG{𥖧}{17627}
\saveTG{𪈝}{17627}
\saveTG{𪆼}{17627}
\saveTG{𪅳}{17627}
\saveTG{鶝}{17627}
\saveTG{𪁙}{17627}
\saveTG{𪂌}{17627}
\saveTG{𪄶}{17627}
\saveTG{𪀉}{17627}
\saveTG{𪀹}{17627}
\saveTG{𢑔}{17627}
\saveTG{䳂}{17627}
\saveTG{𪆇}{17627}
\saveTG{𠰟}{17627}
\saveTG{𢑜}{17627}
\saveTG{䤎}{17627}
\saveTG{𨠮}{17627}
\saveTG{𠭞}{17627}
\saveTG{𨠱}{17627}
\saveTG{𦒯}{17627}
\saveTG{𦒫}{17627}
\saveTG{𥓦}{17627}
\saveTG{𨣯}{17627}
\saveTG{𨢫}{17627}
\saveTG{𨣊}{17627}
\saveTG{𠼡}{17627}
\saveTG{𠾉}{17627}
\saveTG{砀}{17627}
\saveTG{磆}{17627}
\saveTG{郡}{17627}
\saveTG{鵘}{17627}
\saveTG{酃}{17627}
\saveTG{码}{17627}
\saveTG{确}{17627}
\saveTG{邵}{17627}
\saveTG{碿}{17627}
\saveTG{硧}{17627}
\saveTG{碢}{17627}
\saveTG{郚}{17627}
\saveTG{磶}{17627}
\saveTG{醑}{17627}
\saveTG{鄑}{17627}
\saveTG{䣟}{17627}
\saveTG{𥒷}{17627}
\saveTG{𨜧}{17627}
\saveTG{𩈩}{17627}
\saveTG{𥒥}{17627}
\saveTG{𥒢}{17627}
\saveTG{𥕖}{17627}
\saveTG{𨟕}{17627}
\saveTG{𨛔}{17627}
\saveTG{𨟪}{17627}
\saveTG{𨝶}{17627}
\saveTG{𪿣}{17627}
\saveTG{𠣷}{17629}
\saveTG{䤙}{17631}
\saveTG{𨣰}{17631}
\saveTG{𨢻}{17632}
\saveTG{𥑝}{17632}
\saveTG{𩈢}{17632}
\saveTG{𨢊}{17632}
\saveTG{𥗷}{17632}
\saveTG{𥗰}{17632}
\saveTG{𥗸}{17632}
\saveTG{𨤉}{17632}
\saveTG{𪿼}{17632}
\saveTG{𨡮}{17632}
\saveTG{𥖤}{17632}
\saveTG{硍}{17632}
\saveTG{碾}{17632}
\saveTG{𨙝}{17633}
\saveTG{䂢}{17633}
\saveTG{𨠌}{17633}
\saveTG{𥒣}{17634}
\saveTG{𨢭}{17636}
\saveTG{𥕨}{17636}
\saveTG{磓}{17637}
\saveTG{𥑬}{17640}
\saveTG{砃}{17640}
\saveTG{䂘}{17640}
\saveTG{酔}{17641}
\saveTG{𥅐}{17641}
\saveTG{𥑌}{17641}
\saveTG{𪿡}{17641}
\saveTG{砕}{17641}
\saveTG{𥑸}{17644}
\saveTG{𥐷}{17645}
\saveTG{𨠞}{17645}
\saveTG{𥖇}{17646}
\saveTG{䂥}{17647}
\saveTG{𥖒}{17647}
\saveTG{砐}{17647}
\saveTG{𥑊}{17647}
\saveTG{𩈑}{17647}
\saveTG{𨡉}{17647}
\saveTG{𥕋}{17647}
\saveTG{𥓭}{17647}
\saveTG{𩈝}{17647}
\saveTG{𨠡}{17647}
\saveTG{𣆸}{17647}
\saveTG{酘}{17647}
\saveTG{碫}{17647}
\saveTG{𨠈}{17647}
\saveTG{醙}{17647}
\saveTG{碬}{17647}
\saveTG{磤}{17647}
\saveTG{醊}{17647}
\saveTG{矷}{17647}
\saveTG{𠭖}{17647}
\saveTG{𠭿}{17647}
\saveTG{砓}{17647}
\saveTG{𪠧}{17647}
\saveTG{𪠬}{17647}
\saveTG{𣪾}{17647}
\saveTG{𠭵}{17647}
\saveTG{𥗙}{17647}
\saveTG{䃑}{17647}
\saveTG{𥖮}{17648}
\saveTG{砪}{17650}
\saveTG{𨡃}{17654}
\saveTG{𥗩}{17655}
\saveTG{𨡫}{17656}
\saveTG{䂫}{17657}
\saveTG{碀}{17657}
\saveTG{𥗻}{17661}
\saveTG{𦑾}{17661}
\saveTG{𥖷}{17661}
\saveTG{𥗵}{17662}
\saveTG{𨢇}{17662}
\saveTG{𥒊}{17662}
\saveTG{𦧱}{17662}
\saveTG{磖}{17662}
\saveTG{磂}{17662}
\saveTG{酩}{17662}
\saveTG{𥗆}{17663}
\saveTG{𥓶}{17664}
\saveTG{𠶱}{17664}
\saveTG{𪿩}{17664}
\saveTG{硌}{17664}
\saveTG{酪}{17664}
\saveTG{䃫}{17667}
\saveTG{𡃭}{17668}
\saveTG{𥓞}{17672}
\saveTG{𥓒}{17677}
\saveTG{𩈜}{17677}
\saveTG{𩈟}{17677}
\saveTG{𥑵}{17677}
\saveTG{𦧼}{17679}
\saveTG{𨣦}{17681}
\saveTG{𥔤}{17681}
\saveTG{礙}{17681}
\saveTG{礖}{17681}
\saveTG{𥗄}{17682}
\saveTG{䃢}{17682}
\saveTG{𩉏}{17682}
\saveTG{𣢯}{17682}
\saveTG{𣤬}{17682}
\saveTG{𣣄}{17682}
\saveTG{𣣂}{17682}
\saveTG{𣤊}{17682}
\saveTG{𨠅}{17682}
\saveTG{㰤}{17682}
\saveTG{𥒧}{17682}
\saveTG{欩}{17682}
\saveTG{歌}{17682}
\saveTG{砍}{17682}
\saveTG{碶}{17684}
\saveTG{礇}{17684}
\saveTG{𩈹}{17686}
\saveTG{𥗓}{17686}
\saveTG{𩉓}{17686}
\saveTG{礥}{17686}
\saveTG{𥐮}{17687}
\saveTG{𥔢}{17687}
\saveTG{𥖏}{17687}
\saveTG{䂹}{17689}
\saveTG{𨢵}{17691}
\saveTG{磜}{17691}
\saveTG{䂧}{17692}
\saveTG{𪿶}{17693}
\saveTG{磉}{17694}
\saveTG{𨢆}{17694}
\saveTG{𥓃}{17694}
\saveTG{𥔔}{17694}
\saveTG{醁}{17699}
\saveTG{碌}{17699}
\saveTG{𥗍}{17699}
\saveTG{𥓏}{17699}
\saveTG{乙}{17710}
\saveTG{厾}{17710}
\saveTG{𦐀}{17710}
\saveTG{𠃐}{17710}
\saveTG{𪼶}{17711}
\saveTG{𤬨}{17711}
\saveTG{𠃬}{17711}
\saveTG{𢬬}{17712}
\saveTG{𤭆}{17712}
\saveTG{𪓥}{17712}
\saveTG{卺}{17712}
\saveTG{𠃮}{17713}
\saveTG{毣}{17714}
\saveTG{𣬻}{17715}
\saveTG{𤮗}{17716}
\saveTG{𤭿}{17717}
\saveTG{𠃻}{17717}
\saveTG{𠝕}{17717}
\saveTG{𢀾}{17717}
\saveTG{㐒}{17717}
\saveTG{𨛾}{17717}
\saveTG{𨚡}{17717}
\saveTG{巹}{17717}
\saveTG{𤭡}{17717}
\saveTG{已}{17717}
\saveTG{己}{17717}
\saveTG{𢀷}{17717}
\saveTG{𪓦}{17717}
\saveTG{𢀵}{17717}
\saveTG{𤭀}{17717}
\saveTG{𤬳}{17717}
\saveTG{𢀿}{17717}
\saveTG{㼦}{17717}
\saveTG{𦏶}{17718}
\saveTG{𦑄}{17718}
\saveTG{𦑪}{17721}
\saveTG{𫑱}{17721}
\saveTG{𦒘}{17721}
\saveTG{𪉂}{17721}
\saveTG{𤬱}{17722}
\saveTG{䲰}{17723}
\saveTG{𢆱}{17723}
\saveTG{邔}{17727}
\saveTG{邷}{17727}
\saveTG{𢎘}{17727}
\saveTG{𩿺}{17727}
\saveTG{𩿒}{17727}
\saveTG{𦏻}{17727}
\saveTG{𩾠}{17727}
\saveTG{𪄷}{17727}
\saveTG{𢎝}{17727}
\saveTG{𢎙}{17727}
\saveTG{𢏖}{17727}
\saveTG{𨙬}{17727}
\saveTG{䢵}{17727}
\saveTG{𨚏}{17727}
\saveTG{𪆚}{17727}
\saveTG{𫚮}{17727}
\saveTG{𢎟}{17727}
\saveTG{𠃚}{17727}
\saveTG{𢏚}{17727}
\saveTG{𢎠}{17727}
\saveTG{𢎛}{17727}
\saveTG{𢎧}{17727}
\saveTG{𧚿}{17732}
\saveTG{裠}{17732}
\saveTG{䬸}{17732}
\saveTG{𩜒}{17732}
\saveTG{𪙕}{17732}
\saveTG{𧙭}{17732}
\saveTG{𨝍}{17737}
\saveTG{𣱏}{17742}
\saveTG{𤮒}{17744}
\saveTG{𣪐}{17747}
\saveTG{𣫮}{17757}
\saveTG{𪘸}{17771}
\saveTG{𦈱}{17772}
\saveTG{𡶕}{17772}
\saveTG{函}{17772}
\saveTG{㞯}{17772}
\saveTG{𪙲}{17772}
\saveTG{㞪}{17772}
\saveTG{𡻒}{17772}
\saveTG{𡴻}{17772}
\saveTG{𡵑}{17772}
\saveTG{㞼}{17772}
\saveTG{𠚔}{17772}
\saveTG{𦈩}{17772}
\saveTG{𠚚}{17772}
\saveTG{凾}{17772}
\saveTG{𦥖}{17777}
\saveTG{𤮥}{17782}
\saveTG{𣣖}{17782}
\saveTG{𣣁}{17782}
\saveTG{𣢇}{17782}
\saveTG{𨲒}{17788}
\saveTG{𨽸}{17799}
\saveTG{𪦾}{17801}
\saveTG{翨}{17801}
\saveTG{㢲}{17801}
\saveTG{疋}{17801}
\saveTG{翼}{17801}
\saveTG{㠱}{17801}
\saveTG{𢁉}{17801}
\saveTG{跫}{17801}
\saveTG{奦}{17804}
\saveTG{买}{17804}
\saveTG{𢑴}{17804}
\saveTG{𥎫}{17804}
\saveTG{㚑}{17804}
\saveTG{巬}{17805}
\saveTG{𦗘}{17805}
\saveTG{𦑯}{17805}
\saveTG{翜}{17808}
\saveTG{𤑨}{17809}
\saveTG{熃}{17809}
\saveTG{灵}{17809}
\saveTG{𤉅}{17809}
\saveTG{𪸍}{17809}
\saveTG{𪹃}{17809}
\saveTG{𤎧}{17809}
\saveTG{𤑵}{17809}
\saveTG{𢏅}{17809}
\saveTG{𠔣}{17814}
\saveTG{𢁄}{17817}
\saveTG{𪃉}{17827}
\saveTG{𪆲}{17827}
\saveTG{𨞵}{17827}
\saveTG{鄈}{17827}
\saveTG{䳫}{17827}
\saveTG{𨞈}{17827}
\saveTG{𨞝}{17827}
\saveTG{𢎴}{17827}
\saveTG{𪄌}{17827}
\saveTG{㠫}{17851}
\saveTG{𣤻}{17882}
\saveTG{𣣬}{17882}
\saveTG{𧹃}{17886}
\saveTG{㷅}{17894}
\saveTG{𡭖}{17897}
\saveTG{𦃤}{17902}
\saveTG{氶}{17902}
\saveTG{尕}{17902}
\saveTG{綤}{17903}
\saveTG{𥿩}{17903}
\saveTG{𧆉}{17903}
\saveTG{𦁫}{17903}
\saveTG{𣏍}{17904}
\saveTG{𣒊}{17904}
\saveTG{𪼏}{17904}
\saveTG{𢑿}{17904}
\saveTG{𢑯}{17904}
\saveTG{𡥀}{17904}
\saveTG{𣐱}{17904}
\saveTG{𥹏}{17904}
\saveTG{朶}{17904}
\saveTG{彚}{17904}
\saveTG{柔}{17904}
\saveTG{棸}{17904}
\saveTG{𣕠}{17904}
\saveTG{𣓔}{17904}
\saveTG{𣏻}{17904}
\saveTG{𣏗}{17904}
\saveTG{䅃}{17904}
\saveTG{𥞱}{17904}
\saveTG{𫀟}{17904}
\saveTG{𪲑}{17904}
\saveTG{𣎼}{17904}
\saveTG{𣑦}{17904}
\saveTG{𦖰}{17905}
\saveTG{𦑿}{17905}
\saveTG{录}{17909}
\saveTG{飄}{17910}
\saveTG{𪇃}{17910}
\saveTG{飘}{17910}
\saveTG{𠃥}{17910}
\saveTG{𥸦}{17910}
\saveTG{𫀆}{17912}
\saveTG{𥍳}{17914}
\saveTG{𪹂}{17914}
\saveTG{𠝗}{17916}
\saveTG{𫌬}{17917}
\saveTG{𠃌}{17920}
\saveTG{翲}{17920}
\saveTG{沀}{17922}
\saveTG{𥸥}{17924}
\saveTG{鹴}{17927}
\saveTG{𨝮}{17927}
\saveTG{𨟥}{17927}
\saveTG{𨚸}{17927}
\saveTG{𨚀}{17927}
\saveTG{𫛜}{17927}
\saveTG{𫛽}{17927}
\saveTG{𨝓}{17927}
\saveTG{鸘}{17927}
\saveTG{𨜙}{17927}
\saveTG{𥾋}{17927}
\saveTG{𪄃}{17927}
\saveTG{𥜞}{17927}
\saveTG{𪅃}{17927}
\saveTG{鴀}{17927}
\saveTG{鷅}{17927}
\saveTG{鶔}{17927}
\saveTG{𨜼}{17928}
\saveTG{𣕡}{17941}
\saveTG{𣖶}{17942}
\saveTG{𣏉}{17942}
\saveTG{䊄}{17949}
\saveTG{𢑤}{17956}
\saveTG{𣠳}{17961}
\saveTG{𣢾}{17982}
\saveTG{𠞠}{17985}
\saveTG{𢑘}{17994}
\saveTG{𡯈}{18010}
\saveTG{䃚}{18084}
\saveTG{玐}{18100}
\saveTG{𪻐}{18100}
\saveTG{𩐇}{18102}
\saveTG{𤤳}{18103}
\saveTG{堥}{18104}
\saveTG{𣮖}{18105}
\saveTG{䜼}{18108}
\saveTG{鍪}{18109}
\saveTG{𧯤}{18111}
\saveTG{𤧒}{18112}
\saveTG{𢑠}{18112}
\saveTG{玱}{18112}
\saveTG{瑳}{18112}
\saveTG{琷}{18112}
\saveTG{璼}{18112}
\saveTG{𦒁}{18112}
\saveTG{𤤩}{18112}
\saveTG{𤬻}{18113}
\saveTG{𣴶}{18114}
\saveTG{𤨩}{18114}
\saveTG{㻇}{18114}
\saveTG{𤥖}{18114}
\saveTG{𤦴}{18116}
\saveTG{𪼎}{18117}
\saveTG{𤣮}{18117}
\saveTG{𪼋}{18117}
\saveTG{㻢}{18117}
\saveTG{𤨜}{18118}
\saveTG{琻}{18119}
\saveTG{玠}{18120}
\saveTG{䜽}{18120}
\saveTG{𤭷}{18120}
\saveTG{瑐}{18121}
\saveTG{瑜}{18121}
\saveTG{珍}{18122}
\saveTG{𤨤}{18122}
\saveTG{𤥬}{18122}
\saveTG{𤪱}{18122}
\saveTG{𪼜}{18127}
\saveTG{𤦎}{18127}
\saveTG{𢐢}{18127}
\saveTG{𧟯}{18127}
\saveTG{𤨋}{18127}
\saveTG{鹜}{18127}
\saveTG{𪼔}{18127}
\saveTG{骛}{18127}
\saveTG{玪}{18127}
\saveTG{翂}{18127}
\saveTG{珶}{18127}
\saveTG{玢}{18127}
\saveTG{𧯪}{18127}
\saveTG{𤤽}{18130}
\saveTG{𤦬}{18130}
\saveTG{㶃}{18131}
\saveTG{𤧹}{18131}
\saveTG{㻅}{18131}
\saveTG{璑}{18131}
\saveTG{𤪂}{18132}
\saveTG{𤪪}{18132}
\saveTG{𤪊}{18132}
\saveTG{𫅤}{18132}
\saveTG{玜}{18132}
\saveTG{玲}{18132}
\saveTG{𤥽}{18133}
\saveTG{𤥼}{18133}
\saveTG{璲}{18133}
\saveTG{蝥}{18136}
\saveTG{㻩}{18137}
\saveTG{豏}{18137}
\saveTG{𤩀}{18140}
\saveTG{𪼂}{18140}
\saveTG{璬}{18140}
\saveTG{𤨉}{18140}
\saveTG{𤨨}{18140}
\saveTG{㻻}{18140}
\saveTG{𦒚}{18140}
\saveTG{𦒐}{18140}
\saveTG{𪻯}{18140}
\saveTG{𪻰}{18140}
\saveTG{𢾼}{18140}
\saveTG{璈}{18140}
\saveTG{璷}{18140}
\saveTG{攻}{18140}
\saveTG{璥}{18140}
\saveTG{玫}{18140}
\saveTG{𤪖}{18140}
\saveTG{玝}{18140}
\saveTG{政}{18140}
\saveTG{致}{18140}
\saveTG{㻂}{18144}
\saveTG{𤦽}{18147}
\saveTG{㻴}{18147}
\saveTG{珜}{18151}
\saveTG{𦍽}{18151}
\saveTG{𫅧}{18151}
\saveTG{㼁}{18153}
\saveTG{𤩺}{18155}
\saveTG{珻}{18157}
\saveTG{𪼯}{18157}
\saveTG{𤪕}{18157}
\saveTG{㻶}{18157}
\saveTG{珨}{18161}
\saveTG{𤩕}{18161}
\saveTG{𦐬}{18161}
\saveTG{𤩓}{18161}
\saveTG{琀}{18162}
\saveTG{㻥}{18164}
\saveTG{𤦜}{18164}
\saveTG{𦒗}{18166}
\saveTG{璔}{18166}
\saveTG{璯}{18166}
\saveTG{瑲}{18167}
\saveTG{𤥫}{18168}
\saveTG{珤}{18172}
\saveTG{𤫡}{18174}
\saveTG{𤪔}{18177}
\saveTG{瑽}{18181}
\saveTG{琁}{18181}
\saveTG{璇}{18181}
\saveTG{㻜}{18182}
\saveTG{𪼕}{18184}
\saveTG{𤧞}{18184}
\saveTG{㳽}{18190}
\saveTG{𪞡}{18191}
\saveTG{㻌}{18194}
\saveTG{𢎢}{18200}
\saveTG{𣦹}{18200}
\saveTG{𣱺}{18200}
\saveTG{𫎆}{18208}
\saveTG{𡯓}{18210}
\saveTG{𢎲}{18210}
\saveTG{𣧫}{18211}
\saveTG{尶}{18211}
\saveTG{𣩋}{18211}
\saveTG{𣨐}{18211}
\saveTG{㱲}{18212}
\saveTG{㢮}{18212}
\saveTG{𣩈}{18212}
\saveTG{𥍸}{18212}
\saveTG{𣴳}{18214}
\saveTG{𣨎}{18214}
\saveTG{𨤲}{18215}
\saveTG{𣯛}{18215}
\saveTG{𠒻}{18216}
\saveTG{𠄄}{18217}
\saveTG{𫅲}{18217}
\saveTG{𧱲}{18217}
\saveTG{𠄋}{18217}
\saveTG{殓}{18219}
\saveTG{𩱉}{18219}
\saveTG{殄}{18222}
\saveTG{𧱬}{18222}
\saveTG{㡔}{18224}
\saveTG{矜}{18227}
\saveTG{彅}{18227}
\saveTG{殇}{18227}
\saveTG{殤}{18227}
\saveTG{鬺}{18227}
\saveTG{𥍦}{18227}
\saveTG{𪵅}{18227}
\saveTG{𥍠}{18227}
\saveTG{𥎃}{18227}
\saveTG{𧲗}{18227}
\saveTG{𧱜}{18227}
\saveTG{𠢐}{18227}
\saveTG{𠝸}{18227}
\saveTG{矝}{18232}
\saveTG{飱}{18232}
\saveTG{㢳}{18232}
\saveTG{𢏀}{18232}
\saveTG{𧰻}{18232}
\saveTG{𪋭}{18232}
\saveTG{𧰾}{18232}
\saveTG{𣺶}{18232}
\saveTG{𣨝}{18233}
\saveTG{𢐎}{18237}
\saveTG{𥍵}{18238}
\saveTG{𢽰}{18240}
\saveTG{𫀙}{18240}
\saveTG{𪯖}{18240}
\saveTG{𢐍}{18240}
\saveTG{𢎿}{18240}
\saveTG{㢸}{18240}
\saveTG{敄}{18240}
\saveTG{攷}{18240}
\saveTG{𢽴}{18240}
\saveTG{𣷜}{18240}
\saveTG{𫋿}{18240}
\saveTG{𦞺}{18240}
\saveTG{𡠉}{18244}
\saveTG{䨁}{18245}
\saveTG{𦭇}{18247}
\saveTG{𦍙}{18251}
\saveTG{𢏙}{18251}
\saveTG{𤛆}{18251}
\saveTG{𥍮}{18254}
\saveTG{弹}{18256}
\saveTG{殚}{18256}
\saveTG{㢵}{18261}
\saveTG{𣩧}{18261}
\saveTG{}{18261}
\saveTG{𩠑}{18262}
\saveTG{𣩮}{18266}
\saveTG{𢐞}{18266}
\saveTG{䰝}{18266}
\saveTG{𠁚}{18266}
\saveTG{𧲅}{18266}
\saveTG{𥎄}{18267}
\saveTG{𥎏}{18268}
\saveTG{豵}{18281}
\saveTG{𢐔}{18282}
\saveTG{𧱷}{18282}
\saveTG{𢎬}{18283}
\saveTG{矤}{18284}
\saveTG{𧲌}{18285}
\saveTG{𦗹}{18286}
\saveTG{𪜇}{18286}
\saveTG{殮}{18286}
\saveTG{𢏏}{18290}
\saveTG{𣧠}{18290}
\saveTG{𣀗}{18294}
\saveTG{𥼅}{18294}
\saveTG{䢩}{18304}
\saveTG{𨖍}{18304}
\saveTG{𢾘}{18307}
\saveTG{騖}{18327}
\saveTG{鶩}{18327}
\saveTG{𩤇}{18327}
\saveTG{𪑢}{18331}
\saveTG{𢢭}{18332}
\saveTG{㥤}{18332}
\saveTG{㤵}{18334}
\saveTG{𢚢}{18334}
\saveTG{憨}{18334}
\saveTG{愗}{18334}
\saveTG{䱯}{18336}
\saveTG{𢿼}{18340}
\saveTG{𪪈}{18352}
\saveTG{𡤿}{18400}
\saveTG{㬲}{18401}
\saveTG{𢌝}{18402}
\saveTG{𢌣}{18402}
\saveTG{婺}{18404}
\saveTG{𢌦}{18404}
\saveTG{𡝡}{18404}
\saveTG{𪍓}{18407}
\saveTG{𫆌}{18412}
\saveTG{𠕱}{18412}
\saveTG{𦖮}{18412}
\saveTG{𦖬}{18412}
\saveTG{𣍏}{18412}
\saveTG{𦕫}{18412}
\saveTG{𦗅}{18413}
\saveTG{𫆂}{18414}
\saveTG{}{18417}
\saveTG{𡦎}{18417}
\saveTG{㜾}{18420}
\saveTG{𦖭}{18420}
\saveTG{聄}{18422}
\saveTG{聬}{18427}
\saveTG{𡥩}{18427}
\saveTG{聁}{18427}
\saveTG{耹}{18427}
\saveTG{𡕵}{18427}
\saveTG{𦕎}{18427}
\saveTG{𦕞}{18427}
\saveTG{䎾}{18427}
\saveTG{𪪆}{18430}
\saveTG{聆}{18432}
\saveTG{𦖺}{18432}
\saveTG{𪦿}{18432}
\saveTG{聡}{18433}
\saveTG{𦗄}{18434}
\saveTG{聪}{18436}
\saveTG{𦖾}{18437}
\saveTG{𦗈}{18440}
\saveTG{孜}{18440}
\saveTG{敢}{18440}
\saveTG{𣀋}{18440}
\saveTG{𦘇}{18440}
\saveTG{㪀}{18440}
\saveTG{𢼷}{18440}
\saveTG{𦗸}{18440}
\saveTG{攼}{18440}
\saveTG{𦗪}{18440}
\saveTG{聠}{18441}
\saveTG{𠭕}{18447}
\saveTG{𦗧}{18461}
\saveTG{𦗢}{18461}
\saveTG{𦕲}{18461}
\saveTG{𦖣}{18464}
\saveTG{𦖘}{18464}
\saveTG{𫆑}{18466}
\saveTG{𦗜}{18482}
\saveTG{联}{18484}
\saveTG{聅}{18484}
\saveTG{𦏃}{18501}
\saveTG{𦎦}{18501}
\saveTG{𦏓}{18503}
\saveTG{鞪}{18506}
\saveTG{𨍎}{18506}
\saveTG{𣀆}{18540}
\saveTG{𪯍}{18540}
\saveTG{𥐙}{18600}
\saveTG{𨡭}{18604}
\saveTG{暓}{18604}
\saveTG{瞀}{18604}
\saveTG{砟}{18611}
\saveTG{䩆}{18611}
\saveTG{酢}{18611}
\saveTG{𥔞}{18612}
\saveTG{𨣨}{18612}
\saveTG{醝}{18612}
\saveTG{𥕤}{18612}
\saveTG{磋}{18612}
\saveTG{砤}{18612}
\saveTG{砼}{18612}
\saveTG{碰}{18612}
\saveTG{礛}{18612}
\saveTG{𨢴}{18612}
\saveTG{䤈}{18612}
\saveTG{𨢸}{18612}
\saveTG{𨢘}{18612}
\saveTG{𨣎}{18612}
\saveTG{䂱}{18612}
\saveTG{𥗽}{18612}
\saveTG{𥓌}{18612}
\saveTG{}{18612}
\saveTG{𨤎}{18612}
\saveTG{𥔶}{18613}
\saveTG{硂}{18614}
\saveTG{酫}{18614}
\saveTG{䂳}{18614}
\saveTG{䩎}{18616}
\saveTG{𥐬}{18617}
\saveTG{𥑅}{18617}
\saveTG{矻}{18617}
\saveTG{𦧓}{18617}
\saveTG{𨠑}{18617}
\saveTG{𨤋}{18617}
\saveTG{𨣸}{18617}
\saveTG{𪿻}{18618}
\saveTG{碒}{18619}
\saveTG{硷}{18619}
\saveTG{𩈋}{18620}
\saveTG{䤅}{18620}
\saveTG{砎}{18620}
\saveTG{𨣣}{18621}
\saveTG{䂦}{18622}
\saveTG{𡦇}{18627}
\saveTG{䩂}{18627}
\saveTG{𥕀}{18627}
\saveTG{𥐤}{18627}
\saveTG{𨟹}{18627}
\saveTG{𨢩}{18627}
\saveTG{𦧈}{18627}
\saveTG{𨤄}{18627}
\saveTG{𨡪}{18627}
\saveTG{𥗖}{18627}
\saveTG{砏}{18627}
\saveTG{酚}{18627}
\saveTG{砛}{18627}
\saveTG{碖}{18627}
\saveTG{𥒨}{18630}
\saveTG{䂼}{18630}
\saveTG{𨡎}{18630}
\saveTG{𥕻}{18631}
\saveTG{𥖐}{18632}
\saveTG{䃍}{18632}
\saveTG{𨣢}{18632}
\saveTG{𩈖}{18632}
\saveTG{𨠎}{18632}
\saveTG{𨢁}{18632}
\saveTG{𠾥}{18632}
\saveTG{磁}{18632}
\saveTG{砱}{18632}
\saveTG{硹}{18632}
\saveTG{礢}{18632}
\saveTG{𩉋}{18633}
\saveTG{礠}{18633}
\saveTG{礈}{18633}
\saveTG{磀}{18634}
\saveTG{磏}{18637}
\saveTG{䃟}{18640}
\saveTG{䃝}{18640}
\saveTG{𥖻}{18640}
\saveTG{𥕵}{18640}
\saveTG{𥐭}{18640}
\saveTG{𥓴}{18640}
\saveTG{𠵦}{18640}
\saveTG{𪿥}{18640}
\saveTG{𪯘}{18640}
\saveTG{𢽤}{18640}
\saveTG{𨣟}{18640}
\saveTG{敔}{18640}
\saveTG{礅}{18640}
\saveTG{礉}{18640}
\saveTG{磝}{18640}
\saveTG{𢽏}{18640}
\saveTG{𩈚}{18641}
\saveTG{硑}{18641}
\saveTG{𪿺}{18641}
\saveTG{𫑼}{18643}
\saveTG{𥖃}{18644}
\saveTG{𥖁}{18646}
\saveTG{𥒞}{18651}
\saveTG{群}{18651}
\saveTG{𥗥}{18653}
\saveTG{礒}{18653}
\saveTG{𨣞}{18655}
\saveTG{𥔚}{18656}
\saveTG{䃅}{18656}
\saveTG{𨡙}{18656}
\saveTG{𪢦}{18657}
\saveTG{䩈}{18657}
\saveTG{酶}{18657}
\saveTG{𨣁}{18661}
\saveTG{𨣏}{18661}
\saveTG{磰}{18661}
\saveTG{硆}{18661}
\saveTG{𩈣}{18662}
\saveTG{䣻}{18662}
\saveTG{𥓂}{18662}
\saveTG{𥕼}{18664}
\saveTG{𥗁}{18664}
\saveTG{𫑻}{18666}
\saveTG{磳}{18666}
\saveTG{䤌}{18667}
\saveTG{硲}{18668}
\saveTG{𨣋}{18668}
\saveTG{𥖴}{18672}
\saveTG{𥒋}{18672}
\saveTG{磫}{18681}
\saveTG{𥖄}{18682}
\saveTG{䃠}{18682}
\saveTG{磸}{18684}
\saveTG{𨣆}{18684}
\saveTG{𥒆}{18684}
\saveTG{䂠}{18684}
\saveTG{𥖺}{18684}
\saveTG{礆}{18686}
\saveTG{醶}{18686}
\saveTG{𥑒}{18690}
\saveTG{酴}{18694}
\saveTG{硢}{18694}
\saveTG{瓰}{18712}
\saveTG{𤬯}{18712}
\saveTG{𤭅}{18714}
\saveTG{𠃺}{18716}
\saveTG{𪼼}{18716}
\saveTG{𤭙}{18716}
\saveTG{霼}{18717}
\saveTG{𨟮}{18717}
\saveTG{𪗟}{18717}
\saveTG{𪵣}{18717}
\saveTG{𢐧}{18727}
\saveTG{霒}{18727}
\saveTG{饏}{18732}
\saveTG{餮}{18732}
\saveTG{改}{18740}
\saveTG{𫅞}{18764}
\saveTG{嵍}{18772}
\saveTG{𪘯}{18772}
\saveTG{𪚐}{18772}
\saveTG{𨄝}{18802}
\saveTG{𨂣}{18802}
\saveTG{𠛇}{18804}
\saveTG{䪾}{18822}
\saveTG{𫖬}{18822}
\saveTG{𦕦}{18880}
\saveTG{𦖏}{18880}
\saveTG{𩒈}{18890}
\saveTG{𤋄}{18894}
\saveTG{𦁘}{18903}
\saveTG{䋷}{18903}
\saveTG{楘}{18904}
\saveTG{𣻊}{18909}
\saveTG{𫀩}{18940}
\saveTG{㮗}{18942}
\saveTG{𥠪}{18942}
\saveTG{𥔣}{18944}
\saveTG{珖}{19112}
\saveTG{𤨠}{19114}
\saveTG{㼆}{19114}
\saveTG{𤦔}{19117}
\saveTG{𪴽}{19120}
\saveTG{𤤉}{19120}
\saveTG{𤩂}{19127}
\saveTG{𧰎}{19127}
\saveTG{𣹝}{19127}
\saveTG{𣦎}{19127}
\saveTG{琑}{19127}
\saveTG{瑺}{19127}
\saveTG{𤫁}{19131}
\saveTG{𪼲}{19131}
\saveTG{珱}{19144}
\saveTG{𤥎}{19147}
\saveTG{𤫉}{19147}
\saveTG{璘}{19157}
\saveTG{𤪏}{19157}
\saveTG{璘}{19159}
\saveTG{𪻻}{19162}
\saveTG{𤥮}{19162}
\saveTG{𧋻}{19164}
\saveTG{𤫎}{19166}
\saveTG{璫}{19166}
\saveTG{珰}{19177}
\saveTG{𪻓}{19180}
\saveTG{𤧐}{19180}
\saveTG{琐}{19182}
\saveTG{瑣}{19186}
\saveTG{𤪸}{19186}
\saveTG{琰}{19189}
\saveTG{𤫙}{19189}
\saveTG{𢒅}{19192}
\saveTG{𤥄}{19194}
\saveTG{𤪤}{19194}
\saveTG{𣼓}{19196}
\saveTG{𣰲}{19215}
\saveTG{𩂁}{19217}
\saveTG{𥘅}{19217}
\saveTG{矟}{19227}
\saveTG{㢼}{19227}
\saveTG{𩦶}{19227}
\saveTG{弰}{19227}
\saveTG{𣧖}{19230}
\saveTG{弾}{19250}
\saveTG{𢏑}{19250}
\saveTG{𧲂}{19257}
\saveTG{𣄳}{19257}
\saveTG{𤆉}{19280}
\saveTG{𣧛}{19280}
\saveTG{𥎜}{19285}
\saveTG{𣨬}{19289}
\saveTG{𣧲}{19294}
\saveTG{㥑}{19332}
\saveTG{㷦}{19338}
\saveTG{𢞚}{19338}
\saveTG{孙}{19400}
\saveTG{𦕤}{19412}
\saveTG{𡦓}{19415}
\saveTG{𪎊}{19420}
\saveTG{𦕈}{19420}
\saveTG{𦡠}{19427}
\saveTG{𦗠}{19431}
\saveTG{𦘄}{19432}
\saveTG{𡥡}{19448}
\saveTG{𦗲}{19459}
\saveTG{𦗴}{19466}
\saveTG{耿}{19480}
\saveTG{䎿}{19480}
\saveTG{𦘆}{19486}
\saveTG{𤑶}{19489}
\saveTG{𦖠}{19489}
\saveTG{𥻴}{19494}
\saveTG{㪅}{19497}
\saveTG{𦣻}{19602}
\saveTG{𥉔}{19608}
\saveTG{𩉚}{19611}
\saveTG{𨠵}{19612}
\saveTG{硄}{19612}
\saveTG{𥗏}{19614}
\saveTG{𪿴}{19615}
\saveTG{𨡅}{19619}
\saveTG{砂}{19620}
\saveTG{𨡀}{19627}
\saveTG{𨣃}{19627}
\saveTG{𥕂}{19627}
\saveTG{磱}{19627}
\saveTG{硝}{19627}
\saveTG{𥓡}{19627}
\saveTG{𡆊}{19631}
\saveTG{醚}{19639}
\saveTG{𢼔}{19640}
\saveTG{𥑍}{19650}
\saveTG{礃}{19652}
\saveTG{磷}{19659}
\saveTG{𨡋}{19660}
\saveTG{𥗞}{19662}
\saveTG{𨤂}{19664}
\saveTG{𨡴}{19664}
\saveTG{礑}{19666}
\saveTG{𨣛}{19674}
\saveTG{𨡲}{19680}
\saveTG{𥔍}{19680}
\saveTG{𥗈}{19682}
\saveTG{𥔭}{19686}
\saveTG{𤑆}{19689}
\saveTG{醈}{19689}
\saveTG{𪹑}{19689}
\saveTG{砯}{19690}
\saveTG{𨣤}{19691}
\saveTG{礯}{19693}
\saveTG{𥒄}{19694}
\saveTG{𠃝}{19710}
\saveTG{𠃣}{19712}
\saveTG{𪼸}{19718}
\saveTG{䨭}{19727}
\saveTG{褧}{19732}
\saveTG{𤮪}{19744}
\saveTG{𦨆}{19757}
\saveTG{𤊇}{19809}
\saveTG{𡙼}{19827}
\saveTG{𩕼}{19858}
\saveTG{䫰}{19859}
\saveTG{𪳎}{19904}
\saveTG{丿}{20000}
\saveTG{亅}{20000}
\saveTG{丨}{20000}
\saveTG{龵}{20005}
\saveTG{𤖸}{20014}
\saveTG{𨾇}{20015}
\saveTG{𤖹}{20018}
\saveTG{𠃑}{20020}
\saveTG{𤗞}{20021}
\saveTG{𤓳}{20027}
\saveTG{𠃀}{20027}
\saveTG{牓}{20027}
\saveTG{牗}{20027}
\saveTG{𤗫}{20027}
\saveTG{𤗒}{20027}
\saveTG{𤔁}{20029}
\saveTG{𤗺}{20041}
\saveTG{𤗪}{20043}
\saveTG{𤗈}{20044}
\saveTG{𫃵}{20044}
\saveTG{𤗏}{20061}
\saveTG{𤗙}{20072}
\saveTG{𤗊}{20094}
\saveTG{丄}{20100}
\saveTG{纩}{20100}
\saveTG{𠂗}{20100}
\saveTG{纟}{20101}
\saveTG{𣥋}{20101}
\saveTG{𥁢}{20102}
\saveTG{𩐖}{20102}
\saveTG{𠂛}{20102}
\saveTG{盉}{20102}
\saveTG{𫀧}{20102}
\saveTG{𢀩}{20102}
\saveTG{𣥆}{20102}
\saveTG{𢒉}{20102}
\saveTG{𡏕}{20104}
\saveTG{𡐼}{20104}
\saveTG{𡔅}{20104}
\saveTG{𤫔}{20104}
\saveTG{𡍴}{20104}
\saveTG{𤨐}{20104}
\saveTG{𤔊}{20104}
\saveTG{㸒}{20104}
\saveTG{𡋧}{20104}
\saveTG{𡓱}{20104}
\saveTG{𡈼}{20104}
\saveTG{壬}{20104}
\saveTG{玍}{20104}
\saveTG{𡍺}{20104}
\saveTG{重}{20105}
\saveTG{垂}{20105}
\saveTG{埀}{20105}
\saveTG{𩠺}{20106}
\saveTG{𧰌}{20108}
\saveTG{𨭑}{20109}
\saveTG{𥝍}{20109}
\saveTG{䀉}{20110}
\saveTG{𩇺}{20111}
\saveTG{𣦴}{20112}
\saveTG{乖}{20112}
\saveTG{统}{20112}
\saveTG{𩁅}{20115}
\saveTG{𩀡}{20115}
\saveTG{䧳}{20115}
\saveTG{𨿽}{20115}
\saveTG{𨿿}{20115}
\saveTG{𩁲}{20115}
\saveTG{缠}{20115}
\saveTG{雏}{20115}
\saveTG{雌}{20115}
\saveTG{维}{20115}
\saveTG{𨿠}{20115}
\saveTG{鳣}{20116}
\saveTG{𧗔}{20117}
\saveTG{𫀰}{20117}
\saveTG{𪚼}{20117}
\saveTG{𤓸}{20117}
\saveTG{䲞}{20118}
\saveTG{𩁵}{20121}
\saveTG{鲚}{20124}
\saveTG{𫄰}{20127}
\saveTG{𫘶}{20127}
\saveTG{𫛯}{20127}
\saveTG{鳙}{20127}
\saveTG{缡}{20127}
\saveTG{缟}{20127}
\saveTG{鳑}{20127}
\saveTG{鲂}{20127}
\saveTG{缔}{20127}
\saveTG{纺}{20127}
\saveTG{𧥘}{20127}
\saveTG{𣥚}{20128}
\saveTG{𧕲}{20131}
\saveTG{}{20132}
\saveTG{缞}{20132}
\saveTG{𤰂}{20132}
\saveTG{𧖞}{20135}
\saveTG{蠥}{20136}
\saveTG{蚃}{20136}
\saveTG{𫄷}{20136}
\saveTG{𧉕}{20136}
\saveTG{䖝}{20136}
\saveTG{𧉷}{20136}
\saveTG{䗽}{20136}
\saveTG{𧒁}{20136}
\saveTG{纹}{20140}
\saveTG{𦈞}{20141}
\saveTG{㙑}{20142}
\saveTG{𤯰}{20143}
\saveTG{𫄴}{20143}
\saveTG{𤤋}{20143}
\saveTG{𨿻}{20145}
\saveTG{𫋬}{20146}
\saveTG{𧖝}{20146}
\saveTG{}{20147}
\saveTG{㱖}{20148}
\saveTG{绞}{20148}
\saveTG{鲛}{20148}
\saveTG{𠂜}{20172}
\saveTG{𪺍}{20174}
\saveTG{𤔌}{20174}
\saveTG{𦈜}{20194}
\saveTG{𨤹}{20194}
\saveTG{鲸}{20196}
\saveTG{䲈}{20199}
\saveTG{𠆲}{20200}
\saveTG{㓲}{20200}
\saveTG{亻}{20200}
\saveTG{𫀦}{20202}
\saveTG{彳}{20202}
\saveTG{彡}{20202}
\saveTG{乡}{20202}
\saveTG{𢆯}{20203}
\saveTG{𥝘}{20207}
\saveTG{爳}{20207}
\saveTG{戶}{20207}
\saveTG{𢖥}{20211}
\saveTG{覓}{20212}
\saveTG{觅}{20212}
\saveTG{𩲠}{20212}
\saveTG{𤔓}{20212}
\saveTG{𪧍}{20212}
\saveTG{𣩖}{20212}
\saveTG{𠑠}{20212}
\saveTG{𧴈}{20212}
\saveTG{𧠳}{20212}
\saveTG{傹}{20212}
\saveTG{𩴔}{20212}
\saveTG{禿}{20212}
\saveTG{𪜍}{20213}
\saveTG{𩴧}{20213}
\saveTG{𤔞}{20213}
\saveTG{𠉫}{20214}
\saveTG{𤕸}{20214}
\saveTG{𧣓}{20214}
\saveTG{𪝊}{20214}
\saveTG{往}{20214}
\saveTG{侂}{20214}
\saveTG{𩴐}{20214}
\saveTG{住}{20214}
\saveTG{𪖿}{20215}
\saveTG{𨾷}{20215}
\saveTG{𧣽}{20215}
\saveTG{𩀅}{20215}
\saveTG{𨾼}{20215}
\saveTG{𩁨}{20215}
\saveTG{𩁋}{20215}
\saveTG{㒕}{20215}
\saveTG{𡗌}{20215}
\saveTG{𧴗}{20215}
\saveTG{𧳞}{20215}
\saveTG{𨿒}{20215}
\saveTG{雥}{20215}
\saveTG{儺}{20215}
\saveTG{傩}{20215}
\saveTG{雦}{20215}
\saveTG{隹}{20215}
\saveTG{讎}{20215}
\saveTG{雠}{20215}
\saveTG{雔}{20215}
\saveTG{𩀕}{20215}
\saveTG{𩁗}{20215}
\saveTG{㒧}{20215}
\saveTG{𨿂}{20215}
\saveTG{徸}{20215}
\saveTG{𨿖}{20215}
\saveTG{僮}{20215}
\saveTG{倠}{20215}
\saveTG{𩀌}{20215}
\saveTG{𨾪}{20215}
\saveTG{儃}{20216}
\saveTG{𤕄}{20216}
\saveTG{㣶}{20216}
\saveTG{𧪵}{20216}
\saveTG{𪝂}{20216}
\saveTG{𣬜}{20217}
\saveTG{𠑖}{20217}
\saveTG{𧲪}{20217}
\saveTG{𠘹}{20217}
\saveTG{凣}{20217}
\saveTG{𡦹}{20217}
\saveTG{秃}{20217}
\saveTG{𩳝}{20217}
\saveTG{伉}{20217}
\saveTG{𤔆}{20217}
\saveTG{𠾄}{20217}
\saveTG{𧇠}{20217}
\saveTG{𪞳}{20217}
\saveTG{侂}{20217}
\saveTG{𠉆}{20217}
\saveTG{𠍨}{20217}
\saveTG{𡁗}{20217}
\saveTG{𪜎}{20217}
\saveTG{𢓔}{20218}
\saveTG{𠊔}{20218}
\saveTG{𠇸}{20218}
\saveTG{位}{20218}
\saveTG{傡}{20218}
\saveTG{𧤞}{20220}
\saveTG{𠋣}{20221}
\saveTG{停}{20221}
\saveTG{𤔃}{20221}
\saveTG{䨊}{20221}
\saveTG{豸}{20222}
\saveTG{𫜡}{20222}
\saveTG{偐}{20222}
\saveTG{儕}{20223}
\saveTG{㑪}{20224}
\saveTG{𩴚}{20224}
\saveTG{侪}{20224}
\saveTG{𤔔}{20227}
\saveTG{𤔬}{20227}
\saveTG{𤔦}{20227}
\saveTG{爲}{20227}
\saveTG{𥜻}{20227}
\saveTG{㒀}{20227}
\saveTG{𠏛}{20227}
\saveTG{𠊓}{20227}
\saveTG{𤔰}{20227}
\saveTG{𧤤}{20227}
\saveTG{𧤪}{20227}
\saveTG{𧥏}{20227}
\saveTG{𪵗}{20227}
\saveTG{𧤹}{20227}
\saveTG{𪖰}{20227}
\saveTG{㑺}{20227}
\saveTG{𨶊}{20227}
\saveTG{𠆼}{20227}
\saveTG{𧔰}{20227}
\saveTG{𧕣}{20227}
\saveTG{𨿇}{20227}
\saveTG{𠏰}{20227}
\saveTG{𩀟}{20227}
\saveTG{𢅌}{20227}
\saveTG{𧤍}{20227}
\saveTG{㔎}{20227}
\saveTG{𦠋}{20227}
\saveTG{𦠬}{20227}
\saveTG{𤖑}{20227}
\saveTG{𠉸}{20227}
\saveTG{𧴁}{20227}
\saveTG{𧴄}{20227}
\saveTG{喬}{20227}
\saveTG{隽}{20227}
\saveTG{雋}{20227}
\saveTG{儁}{20227}
\saveTG{秀}{20227}
\saveTG{牅}{20227}
\saveTG{禹}{20227}
\saveTG{俼}{20227}
\saveTG{傍}{20227}
\saveTG{觹}{20227}
\saveTG{𢖇}{20227}
\saveTG{𠌯}{20227}
\saveTG{𠎗}{20227}
\saveTG{𩀬}{20227}
\saveTG{𦙔}{20227}
\saveTG{傐}{20227}
\saveTG{徬}{20227}
\saveTG{币}{20227}
\saveTG{傭}{20227}
\saveTG{偙}{20227}
\saveTG{仿}{20227}
\saveTG{彷}{20227}
\saveTG{𦟝}{20227}
\saveTG{𤔹}{20228}
\saveTG{乔}{20228}
\saveTG{㑊}{20230}
\saveTG{𤓹}{20231}
\saveTG{𪨬}{20231}
\saveTG{軈}{20231}
\saveTG{僬}{20231}
\saveTG{儦}{20231}
\saveTG{𢖐}{20231}
\saveTG{𧣙}{20231}
\saveTG{𧥍}{20231}
\saveTG{㒣}{20231}
\saveTG{㣹}{20231}
\saveTG{𧳓}{20232}
\saveTG{𠐝}{20232}
\saveTG{𣩻}{20232}
\saveTG{𥝸}{20232}
\saveTG{𠐦}{20232}
\saveTG{儫}{20232}
\saveTG{儴}{20232}
\saveTG{忀}{20232}
\saveTG{伭}{20232}
\saveTG{依}{20232}
\saveTG{偯}{20232}
\saveTG{𧥉}{20232}
\saveTG{𪺏}{20232}
\saveTG{𧱌}{20232}
\saveTG{𠍡}{20232}
\saveTG{𢯣}{20232}
\saveTG{𪷣}{20232}
\saveTG{㒅}{20232}
\saveTG{㒟}{20232}
\saveTG{𠐤}{20232}
\saveTG{𠎵}{20232}
\saveTG{𣲛}{20232}
\saveTG{𨤜}{20232}
\saveTG{𠌮}{20234}
\saveTG{𠐥}{20236}
\saveTG{億}{20236}
\saveTG{𣁗}{20240}
\saveTG{𠄒}{20240}
\saveTG{𪫋}{20240}
\saveTG{伩}{20240}
\saveTG{俯}{20240}
\saveTG{僻}{20241}
\saveTG{觪}{20241}
\saveTG{𨐔}{20241}
\saveTG{𨐙}{20241}
\saveTG{𨐲}{20241}
\saveTG{䑀}{20241}
\saveTG{𨐍}{20241}
\saveTG{辭}{20241}
\saveTG{𢕾}{20241}
\saveTG{𢓫}{20241}
\saveTG{𥞽}{20241}
\saveTG{𠉄}{20241}
\saveTG{𠊴}{20241}
\saveTG{𢕃}{20242}
\saveTG{𢕑}{20243}
\saveTG{𪖶}{20243}
\saveTG{𪻋}{20243}
\saveTG{𠌭}{20243}
\saveTG{倿}{20244}
\saveTG{侫}{20244}
\saveTG{𧳛}{20244}
\saveTG{傽}{20246}
\saveTG{𢕔}{20246}
\saveTG{𥀧}{20247}
\saveTG{𠋩}{20247}
\saveTG{𢕒}{20247}
\saveTG{𩰨}{20247}
\saveTG{𡖀}{20247}
\saveTG{𤔿}{20247}
\saveTG{𠭧}{20247}
\saveTG{𠊒}{20247}
\saveTG{𠉷}{20247}
\saveTG{𧣦}{20248}
\saveTG{佼}{20248}
\saveTG{倅}{20248}
\saveTG{𢔙}{20248}
\saveTG{䚝}{20248}
\saveTG{𧳚}{20248}
\saveTG{𪝬}{20248}
\saveTG{𧣿}{20251}
\saveTG{舜}{20252}
\saveTG{𠍉}{20253}
\saveTG{𨏜}{20256}
\saveTG{𢕎}{20256}
\saveTG{𪜢}{20261}
\saveTG{𪺞}{20261}
\saveTG{𩐞}{20261}
\saveTG{𠋭}{20261}
\saveTG{𠐜}{20261}
\saveTG{𠒷}{20261}
\saveTG{𧦇}{20261}
\saveTG{信}{20261}
\saveTG{倍}{20261}
\saveTG{偣}{20261}
\saveTG{𪊁}{20262}
\saveTG{𢕮}{20262}
\saveTG{傗}{20263}
\saveTG{傏}{20265}
\saveTG{𠊑}{20269}
\saveTG{𪙼}{20272}
\saveTG{𠆾}{20280}
\saveTG{𢓌}{20280}
\saveTG{侅}{20282}
\saveTG{㚊}{20282}
\saveTG{㑵}{20284}
\saveTG{𤖏}{20284}
\saveTG{𠂚}{20284}
\saveTG{𠋈}{20284}
\saveTG{𠌿}{20285}
\saveTG{儣}{20286}
\saveTG{𦓣}{20286}
\saveTG{𧥌}{20286}
\saveTG{𤏀}{20289}
\saveTG{𧤇}{20294}
\saveTG{𪎓}{20294}
\saveTG{𠍱}{20294}
\saveTG{𠋆}{20294}
\saveTG{𠏟}{20294}
\saveTG{𠉪}{20294}
\saveTG{𧤀}{20296}
\saveTG{𧤚}{20296}
\saveTG{倞}{20296}
\saveTG{躿}{20299}
\saveTG{𡿨}{20300}
\saveTG{乏}{20302}
\saveTG{𤾤}{20302}
\saveTG{鯍}{20312}
\saveTG{𩶃}{20314}
\saveTG{𩼼}{20314}
\saveTG{𪆏}{20315}
\saveTG{鵻}{20315}
\saveTG{𩽀}{20315}
\saveTG{䱦}{20315}
\saveTG{𨿐}{20315}
\saveTG{䧷}{20315}
\saveTG{𫕜}{20315}
\saveTG{𡿾}{20315}
\saveTG{𩻡}{20315}
\saveTG{𩻏}{20315}
\saveTG{鱣}{20316}
\saveTG{𩺮}{20317}
\saveTG{𩹗}{20317}
\saveTG{𩶶}{20317}
\saveTG{䰶}{20317}
\saveTG{𫙤}{20317}
\saveTG{魧}{20317}
\saveTG{𩷠}{20317}
\saveTG{𩺿}{20317}
\saveTG{𩶞}{20317}
\saveTG{𩶘}{20318}
\saveTG{𩻑}{20318}
\saveTG{𨖧}{20319}
\saveTG{乊}{20320}
\saveTG{𩹇}{20321}
\saveTG{鱭}{20323}
\saveTG{鱅}{20327}
\saveTG{𩺋}{20327}
\saveTG{穒}{20327}
\saveTG{𤔡}{20327}
\saveTG{䲱}{20327}
\saveTG{𩺫}{20327}
\saveTG{𩾭}{20327}
\saveTG{𪆳}{20327}
\saveTG{𪂣}{20327}
\saveTG{䱱}{20327}
\saveTG{𪅟}{20327}
\saveTG{魴}{20327}
\saveTG{鰟}{20327}
\saveTG{鶭}{20327}
\saveTG{鰝}{20327}
\saveTG{𩷉}{20330}
\saveTG{𢠀}{20331}
\saveTG{㦌}{20331}
\saveTG{𤋱}{20331}
\saveTG{𪒷}{20331}
\saveTG{𪆄}{20331}
\saveTG{𩽁}{20331}
\saveTG{𢛧}{20331}
\saveTG{𪹮}{20331}
\saveTG{𤒏}{20331}
\saveTG{𤓬}{20331}
\saveTG{𢣘}{20331}
\saveTG{𤆬}{20331}
\saveTG{熏}{20331}
\saveTG{焦}{20331}
\saveTG{𢗩}{20331}
\saveTG{𢛲}{20331}
\saveTG{𩽱}{20331}
\saveTG{𢡺}{20332}
\saveTG{鮌}{20332}
\saveTG{㥋}{20333}
\saveTG{𤇠}{20334}
\saveTG{忎}{20334}
\saveTG{𪺚}{20336}
\saveTG{𢘗}{20336}
\saveTG{𩸡}{20336}
\saveTG{𩼔}{20337}
\saveTG{𩼖}{20337}
\saveTG{㥯}{20337}
\saveTG{𢚩}{20337}
\saveTG{𥢯}{20338}
\saveTG{𢘳}{20339}
\saveTG{𤇕}{20339}
\saveTG{悉}{20339}
\saveTG{魰}{20340}
\saveTG{𪇊}{20341}
\saveTG{𩼎}{20341}
\saveTG{𪇷}{20341}
\saveTG{𪄩}{20341}
\saveTG{𩷔}{20341}
\saveTG{𩺽}{20342}
\saveTG{鯳}{20342}
\saveTG{𪂰}{20343}
\saveTG{𩸅}{20343}
\saveTG{鯜}{20344}
\saveTG{𡬳}{20346}
\saveTG{鱆}{20346}
\saveTG{𩽧}{20347}
\saveTG{𪂎}{20347}
\saveTG{𩽉}{20347}
\saveTG{鯙}{20347}
\saveTG{𩹹}{20347}
\saveTG{鮫}{20348}
\saveTG{𪁉}{20348}
\saveTG{䱣}{20348}
\saveTG{寽}{20349}
\saveTG{𩹎}{20361}
\saveTG{𪬽}{20361}
\saveTG{𩸬}{20361}
\saveTG{𩹱}{20363}
\saveTG{𩹶}{20365}
\saveTG{𩼤}{20394}
\saveTG{鯨}{20396}
\saveTG{鱇}{20399}
\saveTG{𠂌}{20400}
\saveTG{千}{20400}
\saveTG{䎹}{20401}
\saveTG{𪟻}{20401}
\saveTG{𠦼}{20401}
\saveTG{𦖗}{20401}
\saveTG{𥝝}{20401}
\saveTG{隼}{20401}
\saveTG{𠧐}{20402}
\saveTG{䄹}{20402}
\saveTG{秊}{20402}
\saveTG{秊}{20402}
\saveTG{𠧎}{20403}
\saveTG{𤔭}{20403}
\saveTG{𤕍}{20403}
\saveTG{𤕀}{20404}
\saveTG{𡠓}{20404}
\saveTG{𡞅}{20404}
\saveTG{𠂟}{20404}
\saveTG{委}{20404}
\saveTG{妥}{20404}
\saveTG{𠃇}{20406}
\saveTG{𠧏}{20406}
\saveTG{𨾏}{20407}
\saveTG{𡦯}{20407}
\saveTG{𤔂}{20407}
\saveTG{隻}{20407}
\saveTG{爰}{20407}
\saveTG{孼}{20407}
\saveTG{𡥞}{20407}
\saveTG{雙}{20407}
\saveTG{受}{20407}
\saveTG{𩀱}{20407}
\saveTG{𤔐}{20407}
\saveTG{𤔫}{20407}
\saveTG{𪺒}{20407}
\saveTG{𡕼}{20407}
\saveTG{𦥛}{20407}
\saveTG{𠬪}{20407}
\saveTG{𠮗}{20407}
\saveTG{爱}{20407}
\saveTG{愛}{20407}
\saveTG{叐}{20407}
\saveTG{孚}{20407}
\saveTG{季}{20407}
\saveTG{𤓴}{20407}
\saveTG{𤔣}{20407}
\saveTG{𤔪}{20407}
\saveTG{𤓿}{20407}
\saveTG{𠂿}{20408}
\saveTG{𠦬}{20408}
\saveTG{乎}{20409}
\saveTG{𨿮}{20411}
\saveTG{𢀠}{20412}
\saveTG{艈}{20412}
\saveTG{𨈫}{20414}
\saveTG{𨾰}{20415}
\saveTG{𨿓}{20415}
\saveTG{艟}{20415}
\saveTG{𩀹}{20415}
\saveTG{𨿚}{20415}
\saveTG{雛}{20415}
\saveTG{𨿵}{20415}
\saveTG{𦪉}{20415}
\saveTG{𥃛}{20416}
\saveTG{䡀}{20416}
\saveTG{𡕬}{20417}
\saveTG{𨈢}{20417}
\saveTG{航}{20417}
\saveTG{𦪇}{20417}
\saveTG{䠴}{20418}
\saveTG{𠂷}{20418}
\saveTG{𨉬}{20421}
\saveTG{艩}{20423}
\saveTG{𨉲}{20427}
\saveTG{𠂍}{20427}
\saveTG{𡕽}{20427}
\saveTG{𦨾}{20427}
\saveTG{𢆲}{20427}
\saveTG{𠢏}{20427}
\saveTG{艕}{20427}
\saveTG{舫}{20427}
\saveTG{舷}{20432}
\saveTG{𦪅}{20441}
\saveTG{𥝯}{20441}
\saveTG{辤}{20441}
\saveTG{𠆕}{20442}
\saveTG{𧞠}{20443}
\saveTG{𢍒}{20443}
\saveTG{𨤝}{20444}
\saveTG{𢍃}{20444}
\saveTG{𤕇}{20444}
\saveTG{䑾}{20447}
\saveTG{𨤚}{20447}
\saveTG{䨇}{20447}
\saveTG{爯}{20447}
\saveTG{艔}{20447}
\saveTG{艭}{20447}
\saveTG{𡦧}{20448}
\saveTG{𢍏}{20449}
\saveTG{𦩜}{20461}
\saveTG{𦪀}{20465}
\saveTG{𡴘}{20472}
\saveTG{䠹}{20482}
\saveTG{𨊕}{20486}
\saveTG{𤔠}{20488}
\saveTG{手}{20500}
\saveTG{}{20500}
\saveTG{犫}{20501}
\saveTG{犨}{20501}
\saveTG{𢹭}{20502}
\saveTG{𤓾}{20502}
\saveTG{𠦚}{20502}
\saveTG{𠡦}{20502}
\saveTG{𤔒}{20503}
\saveTG{𢆍}{20506}
\saveTG{𠂝}{20506}
\saveTG{𨋟}{20506}
\saveTG{爭}{20507}
\saveTG{𢪐}{20508}
\saveTG{䄵}{20509}
\saveTG{牤}{20510}
\saveTG{𡾵}{20511}
\saveTG{魋}{20511}
\saveTG{魑}{20512}
\saveTG{兎}{20513}
\saveTG{𤙵}{20515}
\saveTG{𨿍}{20515}
\saveTG{㹌}{20515}
\saveTG{犝}{20515}
\saveTG{𤙫}{20516}
\saveTG{𢷆}{20516}
\saveTG{牨}{20517}
\saveTG{㹍}{20527}
\saveTG{𤚰}{20527}
\saveTG{𤚢}{20527}
\saveTG{犒}{20527}
\saveTG{𢎣}{20527}
\saveTG{𤛑}{20527}
\saveTG{牥}{20527}
\saveTG{𢲤}{20527}
\saveTG{犥}{20531}
\saveTG{𤜄}{20532}
\saveTG{𠂯}{20540}
\saveTG{𢪖}{20540}
\saveTG{𤜃}{20541}
\saveTG{𤚃}{20541}
\saveTG{𤛄}{20542}
\saveTG{𡦛}{20547}
\saveTG{𤚡}{20547}
\saveTG{犉}{20547}
\saveTG{𤔤}{20548}
\saveTG{𤜀}{20551}
\saveTG{掱}{20552}
\saveTG{犃}{20561}
\saveTG{𢮏}{20561}
\saveTG{𤛅}{20563}
\saveTG{𤚫}{20565}
\saveTG{𤚊}{20594}
\saveTG{𤚒}{20596}
\saveTG{㹁}{20596}
\saveTG{𥄥}{20601}
\saveTG{𣬧}{20601}
\saveTG{讐}{20601}
\saveTG{售}{20601}
\saveTG{㘜}{20602}
\saveTG{噕}{20602}
\saveTG{𨿞}{20603}
\saveTG{稥}{20603}
\saveTG{𨣩}{20604}
\saveTG{𣅌}{20604}
\saveTG{舌}{20604}
\saveTG{𨤓}{20604}
\saveTG{𤔱}{20605}
\saveTG{看}{20605}
\saveTG{呑}{20608}
\saveTG{𨤗}{20608}
\saveTG{𠱦}{20608}
\saveTG{㕿}{20609}
\saveTG{𤱜}{20609}
\saveTG{𠳅}{20609}
\saveTG{𩠼}{20609}
\saveTG{𩡠}{20609}
\saveTG{𪏰}{20609}
\saveTG{番}{20609}
\saveTG{香}{20609}
\saveTG{𥄃}{20609}
\saveTG{㸓}{20609}
\saveTG{𪽋}{20609}
\saveTG{㸔}{20609}
\saveTG{𠴦}{20614}
\saveTG{雒}{20615}
\saveTG{𩀢}{20615}
\saveTG{𩀣}{20615}
\saveTG{𨿅}{20615}
\saveTG{𤽼}{20615}
\saveTG{𩀷}{20615}
\saveTG{𨿴}{20615}
\saveTG{𩀊}{20615}
\saveTG{䧼}{20615}
\saveTG{𥫂}{20615}
\saveTG{雊}{20615}
\saveTG{𨾱}{20615}
\saveTG{𩀓}{20615}
\saveTG{𪉺}{20615}
\saveTG{𤾝}{20617}
\saveTG{𪋍}{20617}
\saveTG{䴚}{20617}
\saveTG{𥢿}{20621}
\saveTG{𤔄}{20621}
\saveTG{𪊆}{20623}
\saveTG{皜}{20627}
\saveTG{皫}{20631}
\saveTG{𪊊}{20632}
\saveTG{𪊉}{20632}
\saveTG{𤳼}{20634}
\saveTG{𤳯}{20634}
\saveTG{𦧷}{20637}
\saveTG{𦤩}{20637}
\saveTG{𪽼}{20640}
\saveTG{𠮮}{20640}
\saveTG{辞}{20641}
\saveTG{辝}{20641}
\saveTG{𡖺}{20642}
\saveTG{𠾹}{20648}
\saveTG{皎}{20648}
\saveTG{𤕈}{20661}
\saveTG{𤚵}{20661}
\saveTG{舙}{20664}
\saveTG{𠤌}{20664}
\saveTG{馫}{20669}
\saveTG{𤳲}{20685}
\saveTG{㗮}{20685}
\saveTG{𫘂}{20694}
\saveTG{𩡌}{20694}
\saveTG{𩡥}{20696}
\saveTG{𡶴}{20701}
\saveTG{𠃊}{20710}
\saveTG{乇}{20710}
\saveTG{乚}{20710}
\saveTG{𠃋}{20710}
\saveTG{𣭉}{20711}
\saveTG{氊}{20711}
\saveTG{㲙}{20711}
\saveTG{𪵡}{20711}
\saveTG{𡊥}{20712}
\saveTG{㲝}{20712}
\saveTG{𣯤}{20712}
\saveTG{𣯖}{20712}
\saveTG{𣯊}{20712}
\saveTG{𠨟}{20712}
\saveTG{𫗴}{20712}
\saveTG{𠄔}{20713}
\saveTG{毛}{20714}
\saveTG{𢩥}{20714}
\saveTG{毳}{20714}
\saveTG{𩀘}{20715}
\saveTG{𪙞}{20715}
\saveTG{𡹐}{20715}
\saveTG{㠉}{20715}
\saveTG{𡙶}{20715}
\saveTG{嵼}{20715}
\saveTG{雝}{20715}
\saveTG{𣮩}{20716}
\saveTG{𪙵}{20716}
\saveTG{毰}{20716}
\saveTG{𨤖}{20717}
\saveTG{𤮲}{20717}
\saveTG{㐍}{20717}
\saveTG{乥}{20717}
\saveTG{乯}{20717}
\saveTG{𠄏}{20717}
\saveTG{𪼽}{20717}
\saveTG{𡹕}{20717}
\saveTG{𡵀}{20717}
\saveTG{𡼂}{20717}
\saveTG{𠃔}{20717}
\saveTG{𪗜}{20717}
\saveTG{𪗯}{20717}
\saveTG{𡵻}{20717}
\saveTG{𩡆}{20717}
\saveTG{𡶧}{20718}
\saveTG{𣮘}{20719}
\saveTG{嵉}{20721}
\saveTG{﨑}{20721}
\saveTG{嵃}{20722}
\saveTG{𡽟}{20722}
\saveTG{𫜮}{20722}
\saveTG{齴}{20722}
\saveTG{䶩}{20723}
\saveTG{𪗊}{20723}
\saveTG{㠃}{20727}
\saveTG{𠌶}{20727}
\saveTG{𡽉}{20727}
\saveTG{𠂞}{20727}
\saveTG{𠂠}{20727}
\saveTG{𤔴}{20727}
\saveTG{𡼕}{20727}
\saveTG{𤓳}{20727}
\saveTG{嵭}{20727}
\saveTG{崹}{20727}
\saveTG{嵪}{20727}
\saveTG{𥝙}{20727}
\saveTG{𩡜}{20727}
\saveTG{㟾}{20727}
\saveTG{𪨨}{20730}
\saveTG{厶}{20730}
\saveTG{𣌡}{20731}
\saveTG{𨾜}{20731}
\saveTG{𡾌}{20731}
\saveTG{𡾶}{20731}
\saveTG{𪗰}{20731}
\saveTG{𪺕}{20731}
\saveTG{𠂓}{20731}
\saveTG{嶕}{20731}
\saveTG{丢}{20732}
\saveTG{𩞠}{20732}
\saveTG{𩟷}{20732}
\saveTG{𠂻}{20732}
\saveTG{饻}{20732}
\saveTG{㠡}{20732}
\saveTG{𧘰}{20732}
\saveTG{𫗵}{20732}
\saveTG{㠤}{20732}
\saveTG{幺}{20732}
\saveTG{么}{20732}
\saveTG{𡾝}{20732}
\saveTG{𦫌}{20732}
\saveTG{𨿈}{20733}
\saveTG{𡻠}{20734}
\saveTG{㞶}{20740}
\saveTG{辥}{20741}
\saveTG{辪}{20741}
\saveTG{𪙘}{20742}
\saveTG{𠅱}{20742}
\saveTG{𡸝}{20743}
\saveTG{𣮍}{20744}
\saveTG{爵}{20746}
\saveTG{𡾘}{20746}
\saveTG{嶂}{20746}
\saveTG{𠭡}{20747}
\saveTG{崞}{20747}
\saveTG{𩰣}{20747}
\saveTG{𡾼}{20747}
\saveTG{𨤑}{20747}
\saveTG{饺}{20748}
\saveTG{峧}{20748}
\saveTG{齩}{20748}
\saveTG{𪘧}{20748}
\saveTG{𫜪}{20748}
\saveTG{𤔸}{20748}
\saveTG{崪}{20748}
\saveTG{𡾉}{20752}
\saveTG{𪘎}{20761}
\saveTG{𡺈}{20761}
\saveTG{㟝}{20761}
\saveTG{嵣}{20765}
\saveTG{𡻫}{20766}
\saveTG{𠚘}{20770}
\saveTG{屲}{20772}
\saveTG{𡹁}{20772}
\saveTG{𪙖}{20772}
\saveTG{𡸁}{20772}
\saveTG{䍃}{20772}
\saveTG{岙}{20772}
\saveTG{㟀}{20772}
\saveTG{𦉥}{20772}
\saveTG{𡷩}{20772}
\saveTG{𠚏}{20772}
\saveTG{𦉈}{20772}
\saveTG{𠚟}{20773}
\saveTG{𦈼}{20774}
\saveTG{𥞌}{20774}
\saveTG{𪏽}{20774}
\saveTG{𠚡}{20777}
\saveTG{𦥲}{20777}
\saveTG{𠓳}{20777}
\saveTG{𦥫}{20777}
\saveTG{𤔘}{20777}
\saveTG{臿}{20777}
\saveTG{舀}{20777}
\saveTG{峐}{20782}
\saveTG{𡾇}{20786}
\saveTG{𡻥}{20794}
\saveTG{㠎}{20794}
\saveTG{𡾭}{20794}
\saveTG{𡹡}{20796}
\saveTG{嵻}{20799}
\saveTG{䧶}{20800}
\saveTG{辵}{20801}
\saveTG{𨅴}{20802}
\saveTG{𨁢}{20802}
\saveTG{𡔣}{20804}
\saveTG{𤓻}{20804}
\saveTG{𡗮}{20804}
\saveTG{奚}{20804}
\saveTG{夭}{20804}
\saveTG{𨾎}{20804}
\saveTG{𡗞}{20804}
\saveTG{𤕃}{20804}
\saveTG{𨾗}{20804}
\saveTG{𠂕}{20805}
\saveTG{𧸬}{20806}
\saveTG{𦈆}{20806}
\saveTG{𧶓}{20806}
\saveTG{𤊙}{20809}
\saveTG{㸈}{20809}
\saveTG{𤓪}{20809}
\saveTG{𤈚}{20809}
\saveTG{秂}{20809}
\saveTG{𤒂}{20815}
\saveTG{𨾤}{20815}
\saveTG{雞}{20815}
\saveTG{𨾘}{20815}
\saveTG{𠅐}{20817}
\saveTG{𤒣}{20817}
\saveTG{䌒}{20817}
\saveTG{𦂃}{20821}
\saveTG{𧺆}{20824}
\saveTG{𥿹}{20830}
\saveTG{𤓏}{20833}
\saveTG{𥏔}{20841}
\saveTG{𥝚}{20849}
\saveTG{𣬱}{20881}
\saveTG{𪠠}{20882}
\saveTG{𪏪}{20886}
\saveTG{𠂹}{20888}
\saveTG{𨐹}{20894}
\saveTG{𤐮}{20898}
\saveTG{𤒞}{20898}
\saveTG{𤇫}{20899}
\saveTG{爫}{20900}
\saveTG{𥞆}{20901}
\saveTG{䄟}{20901}
\saveTG{乘}{20901}
\saveTG{𤔍}{20902}
\saveTG{乑}{20902}
\saveTG{𨤙}{20903}
\saveTG{𦅽}{20903}
\saveTG{𦃃}{20903}
\saveTG{系}{20903}
\saveTG{𠂲}{20904}
\saveTG{𥞫}{20904}
\saveTG{釆}{20904}
\saveTG{采}{20904}
\saveTG{禾}{20904}
\saveTG{𤓽}{20904}
\saveTG{櫱}{20904}
\saveTG{𤔥}{20904}
\saveTG{𣗷}{20904}
\saveTG{𥝩}{20904}
\saveTG{𨤘}{20904}
\saveTG{㯔}{20904}
\saveTG{雧}{20904}
\saveTG{集}{20904}
\saveTG{𥼛}{20904}
\saveTG{糱}{20904}
\saveTG{乗}{20905}
\saveTG{𣖄}{20905}
\saveTG{𦓤}{20905}
\saveTG{𣏮}{20905}
\saveTG{秉}{20907}
\saveTG{𥟅}{20907}
\saveTG{黍}{20909}
\saveTG{𦇺}{20911}
\saveTG{綂}{20912}
\saveTG{統}{20912}
\saveTG{𫃮}{20914}
\saveTG{纒}{20914}
\saveTG{紸}{20914}
\saveTG{纏}{20914}
\saveTG{𦀐}{20914}
\saveTG{維}{20915}
\saveTG{䧽}{20915}
\saveTG{穜}{20915}
\saveTG{𦅅}{20915}
\saveTG{𩁟}{20915}
\saveTG{稚}{20915}
\saveTG{𥽽}{20915}
\saveTG{𥽀}{20915}
\saveTG{𨾲}{20915}
\saveTG{𨿥}{20915}
\saveTG{繵}{20916}
\saveTG{䆄}{20916}
\saveTG{䌴}{20917}
\saveTG{𪐎}{20917}
\saveTG{𦁙}{20917}
\saveTG{䌱}{20917}
\saveTG{䋁}{20917}
\saveTG{䅊}{20917}
\saveTG{𥿼}{20917}
\saveTG{𥝕}{20917}
\saveTG{秔}{20917}
\saveTG{𫃡}{20918}
\saveTG{𦂶}{20921}
\saveTG{𥠣}{20921}
\saveTG{𦆲}{20922}
\saveTG{纃}{20923}
\saveTG{穧}{20923}
\saveTG{𦄴}{20924}
\saveTG{緕}{20924}
\saveTG{縭}{20927}
\saveTG{縞}{20927}
\saveTG{稿}{20927}
\saveTG{締}{20927}
\saveTG{黐}{20927}
\saveTG{縍}{20927}
\saveTG{𦀠}{20927}
\saveTG{䋭}{20927}
\saveTG{𦇽}{20927}
\saveTG{𥡲}{20927}
\saveTG{𩁴}{20927}
\saveTG{紡}{20927}
\saveTG{䄱}{20927}
\saveTG{𩫥}{20927}
\saveTG{𦄢}{20927}
\saveTG{𥞑}{20927}
\saveTG{𪐏}{20927}
\saveTG{䅭}{20927}
\saveTG{𥡦}{20927}
\saveTG{𥾼}{20927}
\saveTG{𥠒}{20927}
\saveTG{𦆉}{20927}
\saveTG{𦆈}{20927}
\saveTG{𥾽}{20930}
\saveTG{穛}{20931}
\saveTG{𣟼}{20931}
\saveTG{𦅃}{20931}
\saveTG{穮}{20931}
\saveTG{穰}{20932}
\saveTG{𥤂}{20932}
\saveTG{𦄣}{20932}
\saveTG{𦆪}{20932}
\saveTG{縗}{20932}
\saveTG{絋}{20932}
\saveTG{穣}{20932}
\saveTG{纕}{20932}
\saveTG{絃}{20932}
\saveTG{𦃳}{20933}
\saveTG{繶}{20936}
\saveTG{䆂}{20937}
\saveTG{𥣹}{20937}
\saveTG{𦆆}{20937}
\saveTG{紋}{20940}
\saveTG{𨐠}{20941}
\saveTG{䌟}{20941}
\saveTG{𦀓}{20941}
\saveTG{𥤁}{20941}
\saveTG{𦁹}{20942}
\saveTG{𥡒}{20942}
\saveTG{𣁝}{20942}
\saveTG{𦂊}{20942}
\saveTG{𥡢}{20943}
\saveTG{𦈁}{20943}
\saveTG{繂}{20943}
\saveTG{𣙼}{20944}
\saveTG{𦁉}{20944}
\saveTG{𥟋}{20944}
\saveTG{𥿺}{20944}
\saveTG{𤚮}{20946}
\saveTG{綧}{20947}
\saveTG{𥟆}{20947}
\saveTG{𦂀}{20947}
\saveTG{稕}{20947}
\saveTG{𫄆}{20948}
\saveTG{綷}{20948}
\saveTG{稡}{20948}
\saveTG{絞}{20948}
\saveTG{𥞯}{20949}
\saveTG{𣒧}{20951}
\saveTG{𦇑}{20952}
\saveTG{縴}{20953}
\saveTG{𦀮}{20954}
\saveTG{𥞞}{20960}
\saveTG{䅧}{20961}
\saveTG{𦂺}{20961}
\saveTG{䋨}{20961}
\saveTG{稖}{20961}
\saveTG{𦁦}{20962}
\saveTG{稸}{20963}
\saveTG{䅯}{20965}
\saveTG{䌅}{20967}
\saveTG{絯}{20982}
\saveTG{𥞨}{20982}
\saveTG{𦆌}{20984}
\saveTG{𦀫}{20984}
\saveTG{𦆍}{20985}
\saveTG{纊}{20986}
\saveTG{穬}{20986}
\saveTG{𦁶}{20987}
\saveTG{𫄊}{20991}
\saveTG{𤔳}{20993}
\saveTG{𦃟}{20993}
\saveTG{𦄝}{20994}
\saveTG{䌕}{20994}
\saveTG{𦀾}{20994}
\saveTG{䌖}{20994}
\saveTG{𦇘}{20994}
\saveTG{𦁛}{20994}
\saveTG{䢄}{20994}
\saveTG{穕}{20994}
\saveTG{𦂠}{20996}
\saveTG{稤}{20996}
\saveTG{綡}{20996}
\saveTG{䅫}{20996}
\saveTG{穅}{20999}
\saveTG{𪚖}{21011}
\saveTG{𤗋}{21011}
\saveTG{𤘃}{21012}
\saveTG{𪜈}{21014}
\saveTG{𦤸}{21014}
\saveTG{𤗑}{21017}
\saveTG{𤭘}{21017}
\saveTG{𤗕}{21017}
\saveTG{㸞}{21027}
\saveTG{𪟽}{21027}
\saveTG{𤖶}{21027}
\saveTG{𤗩}{21027}
\saveTG{师}{21027}
\saveTG{𤗦}{21027}
\saveTG{𤖳}{21049}
\saveTG{𤗭}{21049}
\saveTG{𤗚}{21066}
\saveTG{顺}{21082}
\saveTG{順}{21086}
\saveTG{𤗤}{21086}
\saveTG{頥}{21086}
\saveTG{𩑎}{21086}
\saveTG{𤖯}{21090}
\saveTG{𤕬}{21100}
\saveTG{止}{21100}
\saveTG{上}{21100}
\saveTG{𢏮}{21100}
\saveTG{𥂷}{21102}
\saveTG{𥃎}{21102}
\saveTG{𥃍}{21102}
\saveTG{𣦄}{21102}
\saveTG{盨}{21102}
\saveTG{𥃏}{21102}
\saveTG{𣥕}{21102}
\saveTG{𪚍}{21102}
\saveTG{𣥌}{21102}
\saveTG{𣥗}{21102}
\saveTG{𣦐}{21102}
\saveTG{𩐃}{21102}
\saveTG{𪉟}{21102}
\saveTG{𧖭}{21102}
\saveTG{𨡰}{21102}
\saveTG{𪾔}{21102}
\saveTG{𡋬}{21104}
\saveTG{𪟫}{21104}
\saveTG{𡒴}{21104}
\saveTG{𡑸}{21104}
\saveTG{𡒿}{21104}
\saveTG{𡍏}{21104}
\saveTG{𡌓}{21104}
\saveTG{墍}{21104}
\saveTG{𣰼}{21105}
\saveTG{曁}{21106}
\saveTG{𪴶}{21106}
\saveTG{鲈}{21107}
\saveTG{䇓}{21108}
\saveTG{𨧔}{21109}
\saveTG{𨦷}{21109}
\saveTG{𨨾}{21109}
\saveTG{绯}{21111}
\saveTG{歮}{21111}
\saveTG{鲱}{21111}
\saveTG{𣦙}{21112}
\saveTG{頾}{21112}
\saveTG{红}{21112}
\saveTG{𦈗}{21112}
\saveTG{𣦕}{21112}
\saveTG{𣥖}{21112}
\saveTG{𫚉}{21112}
\saveTG{𦈊}{21112}
\saveTG{绖}{21114}
\saveTG{𦈑}{21114}
\saveTG{𫚢}{21114}
\saveTG{𣦟}{21115}
\saveTG{缰}{21116}
\saveTG{𤭮}{21117}
\saveTG{𤮘}{21117}
\saveTG{𤮔}{21117}
\saveTG{歫}{21117}
\saveTG{甀}{21117}
\saveTG{𣥓}{21117}
\saveTG{𤯮}{21117}
\saveTG{𫚚}{21117}
\saveTG{𫄞}{21119}
\saveTG{鲄}{21120}
\saveTG{绗}{21121}
\saveTG{}{21121}
\saveTG{𧯶}{21126}
\saveTG{𩢑}{21127}
\saveTG{鲕}{21127}
\saveTG{鲡}{21127}
\saveTG{卐}{21127}
\saveTG{与}{21127}
\saveTG{𫚕}{21127}
\saveTG{𦈡}{21127}
\saveTG{𫚎}{21127}
\saveTG{𩣷}{21127}
\saveTG{𩦛}{21127}
\saveTG{𧖹}{21132}
\saveTG{𨭭}{21132}
\saveTG{纭}{21132}
\saveTG{𧉘}{21136}
\saveTG{𧔪}{21136}
\saveTG{𧍻}{21136}
\saveTG{𧍢}{21136}
\saveTG{𧏍}{21136}
\saveTG{衈}{21140}
\saveTG{𤂅}{21140}
\saveTG{纡}{21140}
\saveTG{𣥮}{21141}
\saveTG{𡋪}{21142}
\saveTG{䌺}{21142}
\saveTG{缛}{21143}
\saveTG{𧗈}{21143}
\saveTG{𤯨}{21144}
\saveTG{𫚣}{21145}
\saveTG{𧗌}{21145}
\saveTG{缏}{21146}
\saveTG{绠}{21146}
\saveTG{绰}{21146}
\saveTG{𪉖}{21146}
\saveTG{鲠}{21146}
\saveTG{𩾎}{21147}
\saveTG{敱}{21147}
\saveTG{𢼟}{21147}
\saveTG{𢽃}{21147}
\saveTG{𢾫}{21147}
\saveTG{𢻸}{21147}
\saveTG{𪯉}{21147}
\saveTG{㪓}{21147}
\saveTG{𣦩}{21147}
\saveTG{𦈙}{21147}
\saveTG{𣵵}{21147}
\saveTG{𡎱}{21148}
\saveTG{鲆}{21149}
\saveTG{𪴼}{21153}
\saveTG{鲇}{21160}
\saveTG{缙}{21161}
\saveTG{缅}{21162}
\saveTG{𦈋}{21164}
\saveTG{䲤}{21164}
\saveTG{𠏹}{21166}
\saveTG{𪉙}{21166}
\saveTG{䌿}{21166}
\saveTG{鲾}{21166}
\saveTG{𩐅}{21174}
\saveTG{𠧕}{21177}
\saveTG{鳕}{21177}
\saveTG{𠫽}{21181}
\saveTG{鳜}{21182}
\saveTG{缬}{21182}
\saveTG{𫄳}{21182}
\saveTG{𩖕}{21182}
\saveTG{𩑽}{21186}
\saveTG{𤁩}{21186}
\saveTG{𩔇}{21186}
\saveTG{𩕎}{21186}
\saveTG{頉}{21186}
\saveTG{頿}{21186}
\saveTG{顗}{21186}
\saveTG{𪉣}{21186}
\saveTG{𩓭}{21186}
\saveTG{𩓣}{21186}
\saveTG{𫖠}{21186}
\saveTG{𩑙}{21186}
\saveTG{衃}{21190}
\saveTG{鳔}{21191}
\saveTG{缥}{21191}
\saveTG{𤕫}{21200}
\saveTG{㣜}{21200}
\saveTG{步}{21201}
\saveTG{俨}{21201}
\saveTG{𣥿}{21202}
\saveTG{𪉷}{21206}
\saveTG{𣥶}{21206}
\saveTG{卢}{21207}
\saveTG{歺}{21207}
\saveTG{㱑}{21207}
\saveTG{歩}{21209}
\saveTG{𨘩}{21209}
\saveTG{仩}{21210}
\saveTG{仁}{21210}
\saveTG{儮}{21211}
\saveTG{𠒑}{21211}
\saveTG{征}{21211}
\saveTG{䰧}{21211}
\saveTG{𤖋}{21211}
\saveTG{𩳯}{21211}
\saveTG{𧇢}{21211}
\saveTG{𧆳}{21211}
\saveTG{𧆬}{21211}
\saveTG{㑌}{21211}
\saveTG{𩳀}{21211}
\saveTG{𩳍}{21211}
\saveTG{佂}{21211}
\saveTG{㒎}{21211}
\saveTG{𠍣}{21211}
\saveTG{徘}{21211}
\saveTG{徰}{21211}
\saveTG{俳}{21211}
\saveTG{仨}{21211}
\saveTG{徿}{21211}
\saveTG{儱}{21211}
\saveTG{𩲼}{21211}
\saveTG{𠋗}{21212}
\saveTG{𠇈}{21212}
\saveTG{𣦛}{21212}
\saveTG{𧲦}{21212}
\saveTG{𧴆}{21212}
\saveTG{𧴎}{21212}
\saveTG{𧈀}{21212}
\saveTG{虗}{21212}
\saveTG{䖈}{21212}
\saveTG{𤖥}{21212}
\saveTG{𤖦}{21212}
\saveTG{䰢}{21212}
\saveTG{𠒣}{21212}
\saveTG{𠈄}{21212}
\saveTG{𪫒}{21212}
\saveTG{𧇁}{21212}
\saveTG{𩳟}{21212}
\saveTG{𩵄}{21212}
\saveTG{𪗁}{21212}
\saveTG{㒑}{21212}
\saveTG{𢿅}{21212}
\saveTG{𪫘}{21212}
\saveTG{𧳔}{21212}
\saveTG{𦞑}{21212}
\saveTG{𥃈}{21212}
\saveTG{𧣰}{21212}
\saveTG{𩥢}{21212}
\saveTG{𩴶}{21212}
\saveTG{𩲬}{21212}
\saveTG{𪖩}{21212}
\saveTG{𢕖}{21212}
\saveTG{仾}{21212}
\saveTG{𠐳}{21212}
\saveTG{𧴠}{21212}
\saveTG{𤖢}{21212}
\saveTG{𢖙}{21212}
\saveTG{伌}{21212}
\saveTG{仜}{21212}
\saveTG{鼿}{21212}
\saveTG{俓}{21212}
\saveTG{徑}{21212}
\saveTG{盧}{21212}
\saveTG{儷}{21212}
\saveTG{伍}{21212}
\saveTG{虚}{21212}
\saveTG{俹}{21212}
\saveTG{齇}{21212}
\saveTG{𧢸}{21212}
\saveTG{𧣠}{21212}
\saveTG{𧗸}{21212}
\saveTG{𠌨}{21212}
\saveTG{虘}{21212}
\saveTG{𩳭}{21213}
\saveTG{𫙋}{21213}
\saveTG{𩴯}{21213}
\saveTG{𠂧}{21213}
\saveTG{𩳤}{21213}
\saveTG{𩲑}{21213}
\saveTG{𩳨}{21214}
\saveTG{𡖧}{21214}
\saveTG{𠏂}{21214}
\saveTG{㣆}{21214}
\saveTG{𢔉}{21214}
\saveTG{虍}{21214}
\saveTG{𫊟}{21214}
\saveTG{𩴘}{21214}
\saveTG{𠊎}{21214}
\saveTG{𠍝}{21214}
\saveTG{𢓯}{21214}
\saveTG{𢆋}{21214}
\saveTG{䝙}{21214}
\saveTG{侄}{21214}
\saveTG{𠇽}{21214}
\saveTG{伛}{21214}
\saveTG{躽}{21214}
\saveTG{偃}{21214}
\saveTG{彺}{21214}
\saveTG{仼}{21214}
\saveTG{躯}{21214}
\saveTG{虐}{21214}
\saveTG{𠋿}{21214}
\saveTG{俇}{21214}
\saveTG{𩵊}{21214}
\saveTG{𧈄}{21214}
\saveTG{𩳹}{21214}
\saveTG{𠍾}{21214}
\saveTG{𠇤}{21214}
\saveTG{𪖣}{21214}
\saveTG{𤥹}{21214}
\saveTG{雐}{21215}
\saveTG{軅}{21215}
\saveTG{𨿳}{21215}
\saveTG{𠐶}{21215}
\saveTG{𨿃}{21215}
\saveTG{僵}{21216}
\saveTG{軀}{21216}
\saveTG{傴}{21216}
\saveTG{貙}{21216}
\saveTG{𩳜}{21216}
\saveTG{𨡛}{21216}
\saveTG{𩵁}{21216}
\saveTG{𩲦}{21216}
\saveTG{𩳌}{21216}
\saveTG{𠋧}{21216}
\saveTG{𠈗}{21216}
\saveTG{䚙}{21216}
\saveTG{𢕓}{21216}
\saveTG{貆}{21216}
\saveTG{俿}{21217}
\saveTG{甔}{21217}
\saveTG{虎}{21217}
\saveTG{齀}{21217}
\saveTG{佢}{21217}
\saveTG{佤}{21217}
\saveTG{虛}{21217}
\saveTG{甗}{21217}
\saveTG{𦨞}{21217}
\saveTG{𤆛}{21217}
\saveTG{䝞}{21217}
\saveTG{𢇒}{21217}
\saveTG{𧆧}{21217}
\saveTG{𠑷}{21217}
\saveTG{𦒕}{21217}
\saveTG{𩲨}{21217}
\saveTG{㡇}{21217}
\saveTG{𤖊}{21217}
\saveTG{𠨜}{21217}
\saveTG{𧢾}{21217}
\saveTG{𧥖}{21217}
\saveTG{𧥗}{21217}
\saveTG{𧣾}{21217}
\saveTG{𠿻}{21217}
\saveTG{𢓆}{21217}
\saveTG{𪖺}{21217}
\saveTG{𠐿}{21217}
\saveTG{㑙}{21217}
\saveTG{㐾}{21217}
\saveTG{𠐚}{21217}
\saveTG{㐳}{21217}
\saveTG{𠐢}{21217}
\saveTG{𪝷}{21217}
\saveTG{𠌐}{21217}
\saveTG{𠌰}{21217}
\saveTG{𠤐}{21217}
\saveTG{僊}{21217}
\saveTG{𤮙}{21217}
\saveTG{𤮣}{21217}
\saveTG{𤮧}{21217}
\saveTG{虤}{21217}
\saveTG{𤮆}{21217}
\saveTG{𤮝}{21217}
\saveTG{㼷}{21217}
\saveTG{𤮰}{21217}
\saveTG{𧣒}{21217}
\saveTG{𧣂}{21217}
\saveTG{𧥕}{21217}
\saveTG{𤭶}{21217}
\saveTG{𠌩}{21217}
\saveTG{𣲊}{21217}
\saveTG{𧲽}{21217}
\saveTG{侸}{21218}
\saveTG{𩳰}{21218}
\saveTG{𠏺}{21218}
\saveTG{䫥}{21218}
\saveTG{𧆲}{21218}
\saveTG{䖒}{21218}
\saveTG{𫏰}{21218}
\saveTG{𧢽}{21218}
\saveTG{𩲹}{21218}
\saveTG{豾}{21219}
\saveTG{伾}{21219}
\saveTG{𢓖}{21219}
\saveTG{牁}{21220}
\saveTG{𢂦}{21220}
\saveTG{𧢴}{21220}
\saveTG{𠄚}{21220}
\saveTG{㣔}{21220}
\saveTG{仃}{21220}
\saveTG{何}{21220}
\saveTG{𧗠}{21221}
\saveTG{𧗶}{21221}
\saveTG{𧗹}{21221}
\saveTG{𧘂}{21221}
\saveTG{𧘄}{21221}
\saveTG{𧗝}{21221}
\saveTG{𧘀}{21221}
\saveTG{𧗺}{21221}
\saveTG{𧗞}{21221}
\saveTG{𧗟}{21221}
\saveTG{𧗳}{21221}
\saveTG{䘕}{21221}
\saveTG{𧗵}{21221}
\saveTG{𧗿}{21221}
\saveTG{𤖄}{21221}
\saveTG{𠧣}{21221}
\saveTG{𠨆}{21221}
\saveTG{𣥍}{21221}
\saveTG{衝}{21221}
\saveTG{衜}{21221}
\saveTG{衟}{21221}
\saveTG{行}{21221}
\saveTG{衡}{21221}
\saveTG{鵆}{21221}
\saveTG{衚}{21221}
\saveTG{衎}{21221}
\saveTG{衑}{21221}
\saveTG{衖}{21221}
\saveTG{衐}{21221}
\saveTG{術}{21221}
\saveTG{衛}{21221}
\saveTG{衞}{21221}
\saveTG{衔}{21221}
\saveTG{銜}{21221}
\saveTG{衒}{21221}
\saveTG{衙}{21221}
\saveTG{衍}{21221}
\saveTG{𢖍}{21221}
\saveTG{衘}{21221}
\saveTG{䘖}{21221}
\saveTG{𢔖}{21221}
\saveTG{𧗤}{21221}
\saveTG{𢔬}{21221}
\saveTG{𧗾}{21221}
\saveTG{䘘}{21221}
\saveTG{䚘}{21221}
\saveTG{𧗫}{21221}
\saveTG{𧗻}{21221}
\saveTG{𢕋}{21221}
\saveTG{𧗱}{21221}
\saveTG{𧗮}{21221}
\saveTG{䘙}{21221}
\saveTG{𢖅}{21221}
\saveTG{𢕅}{21221}
\saveTG{𧗼}{21221}
\saveTG{𧗲}{21221}
\saveTG{𧗬}{21221}
\saveTG{𧗽}{21221}
\saveTG{䘗}{21221}
\saveTG{衕}{21221}
\saveTG{𧗣}{21221}
\saveTG{𧘃}{21221}
\saveTG{䡓}{21221}
\saveTG{𧘁}{21221}
\saveTG{𧗧}{21221}
\saveTG{𧊔}{21221}
\saveTG{𧗯}{21221}
\saveTG{𧗰}{21221}
\saveTG{𧘆}{21221}
\saveTG{衢}{21221}
\saveTG{𧗡}{21221}
\saveTG{𧗪}{21221}
\saveTG{𧗦}{21221}
\saveTG{𢕁}{21221}
\saveTG{衏}{21221}
\saveTG{𧘅}{21221}
\saveTG{街}{21221}
\saveTG{𢖋}{21221}
\saveTG{𧗷}{21221}
\saveTG{𧗥}{21221}
\saveTG{𢕥}{21221}
\saveTG{𢖡}{21221}
\saveTG{𧗢}{21221}
\saveTG{𧗭}{21221}
\saveTG{𧗴}{21221}
\saveTG{𢔮}{21221}
\saveTG{𢕵}{21221}
\saveTG{𧗨}{21221}
\saveTG{𢓁}{21221}
\saveTG{𫋯}{21221}
\saveTG{𫋱}{21221}
\saveTG{𫋮}{21221}
\saveTG{𫋰}{21221}
\saveTG{衠}{21221}
\saveTG{𫋭}{21221}
\saveTG{徏}{21221}
\saveTG{𠉡}{21222}
\saveTG{𠧻}{21222}
\saveTG{𠧳}{21222}
\saveTG{𠌹}{21222}
\saveTG{𣥩}{21222}
\saveTG{禼}{21224}
\saveTG{𫊢}{21224}
\saveTG{𪝏}{21226}
\saveTG{𠧽}{21227}
\saveTG{𥜽}{21227}
\saveTG{𣥪}{21227}
\saveTG{𠧲}{21227}
\saveTG{𠧜}{21227}
\saveTG{𠨁}{21227}
\saveTG{𠧿}{21227}
\saveTG{𠨂}{21227}
\saveTG{𩡽}{21227}
\saveTG{𧤜}{21227}
\saveTG{𧢵}{21227}
\saveTG{𪠡}{21227}
\saveTG{𤜳}{21227}
\saveTG{𢔸}{21227}
\saveTG{𠍛}{21227}
\saveTG{𢔚}{21227}
\saveTG{𠆮}{21227}
\saveTG{𠇆}{21227}
\saveTG{𠈓}{21227}
\saveTG{㑂}{21227}
\saveTG{𠇐}{21227}
\saveTG{𪝜}{21227}
\saveTG{𩣈}{21227}
\saveTG{𠉝}{21227}
\saveTG{𩢮}{21227}
\saveTG{𪝰}{21227}
\saveTG{𩪨}{21227}
\saveTG{𦢖}{21227}
\saveTG{𧳲}{21227}
\saveTG{𧴓}{21227}
\saveTG{𦜨}{21227}
\saveTG{𧆪}{21227}
\saveTG{𧈃}{21227}
\saveTG{𣥈}{21227}
\saveTG{𠨈}{21227}
\saveTG{𦙡}{21227}
\saveTG{𣥑}{21227}
\saveTG{𧣆}{21227}
\saveTG{𠂱}{21227}
\saveTG{𩨱}{21227}
\saveTG{𪝥}{21227}
\saveTG{𧣅}{21227}
\saveTG{𧈉}{21227}
\saveTG{𧆿}{21227}
\saveTG{𧆞}{21227}
\saveTG{𧲞}{21227}
\saveTG{𤔧}{21227}
\saveTG{𧆴}{21227}
\saveTG{𧆜}{21227}
\saveTG{𧇊}{21227}
\saveTG{𣥯}{21227}
\saveTG{𢃌}{21227}
\saveTG{𢆈}{21227}
\saveTG{𢖨}{21227}
\saveTG{𠆬}{21227}
\saveTG{𠋛}{21227}
\saveTG{𢎰}{21227}
\saveTG{𩤌}{21227}
\saveTG{𠛤}{21227}
\saveTG{𣥴}{21227}
\saveTG{𧈂}{21227}
\saveTG{𠤛}{21227}
\saveTG{𣦒}{21227}
\saveTG{𠋊}{21227}
\saveTG{儞}{21227}
\saveTG{鼑}{21227}
\saveTG{𠉴}{21227}
\saveTG{侕}{21227}
\saveTG{膚}{21227}
\saveTG{肯}{21227}
\saveTG{肻}{21227}
\saveTG{虧}{21227}
\saveTG{俪}{21227}
\saveTG{俩}{21227}
\saveTG{倆}{21227}
\saveTG{虏}{21227}
\saveTG{虜}{21227}
\saveTG{傌}{21227}
\saveTG{卨}{21227}
\saveTG{傿}{21227}
\saveTG{鬳}{21227}
\saveTG{𦟡}{21227}
\saveTG{𥤃}{21227}
\saveTG{𢄼}{21227}
\saveTG{儒}{21227}
\saveTG{𠬙}{21227}
\saveTG{𪩵}{21227}
\saveTG{𫚈}{21227}
\saveTG{𠨄}{21227}
\saveTG{𥹟}{21227}
\saveTG{𠧼}{21227}
\saveTG{𠧰}{21227}
\saveTG{𣦵}{21227}
\saveTG{𠍩}{21228}
\saveTG{𣬄}{21228}
\saveTG{𢕱}{21229}
\saveTG{佧}{21231}
\saveTG{𠧗}{21231}
\saveTG{𠎩}{21231}
\saveTG{𢓎}{21231}
\saveTG{𠎉}{21231}
\saveTG{𠑢}{21231}
\saveTG{𠍍}{21231}
\saveTG{𤣝}{21231}
\saveTG{𧆹}{21231}
\saveTG{𧆸}{21231}
\saveTG{𧣎}{21231}
\saveTG{虑}{21231}
\saveTG{僫}{21231}
\saveTG{卡}{21231}
\saveTG{豦}{21232}
\saveTG{𧆝}{21232}
\saveTG{𧴘}{21232}
\saveTG{𧙾}{21232}
\saveTG{𢓻}{21232}
\saveTG{伝}{21232}
\saveTG{躼}{21232}
\saveTG{侲}{21232}
\saveTG{𠑚}{21232}
\saveTG{𣻯}{21232}
\saveTG{𧣨}{21232}
\saveTG{𠧶}{21232}
\saveTG{倀}{21232}
\saveTG{𠏡}{21232}
\saveTG{𤖜}{21232}
\saveTG{𧲝}{21232}
\saveTG{𧲔}{21232}
\saveTG{𠑓}{21232}
\saveTG{𠍪}{21233}
\saveTG{𧤶}{21233}
\saveTG{𧤲}{21233}
\saveTG{𠧹}{21233}
\saveTG{虙}{21234}
\saveTG{𫊣}{21236}
\saveTG{儢}{21236}
\saveTG{慮}{21236}
\saveTG{𠏊}{21237}
\saveTG{𪇿}{21237}
\saveTG{𢞋}{21238}
\saveTG{𠍴}{21238}
\saveTG{豻}{21240}
\saveTG{𠋵}{21240}
\saveTG{佴}{21240}
\saveTG{𠆻}{21240}
\saveTG{伢}{21240}
\saveTG{𠎋}{21240}
\saveTG{𪔆}{21240}
\saveTG{㐵}{21240}
\saveTG{虔}{21240}
\saveTG{𧤥}{21240}
\saveTG{仠}{21240}
\saveTG{鼾}{21240}
\saveTG{㑝}{21241}
\saveTG{𢔭}{21241}
\saveTG{處}{21241}
\saveTG{𢆌}{21241}
\saveTG{䖉}{21241}
\saveTG{𦚺}{21241}
\saveTG{𠐪}{21241}
\saveTG{𤖁}{21241}
\saveTG{𧣐}{21242}
\saveTG{𠌘}{21242}
\saveTG{𧆵}{21242}
\saveTG{㒤}{21242}
\saveTG{𧆶}{21242}
\saveTG{傉}{21243}
\saveTG{𧲨}{21244}
\saveTG{𢓄}{21244}
\saveTG{偠}{21244}
\saveTG{𠉔}{21244}
\saveTG{𧟰}{21244}
\saveTG{𠏕}{21244}
\saveTG{佞}{21244}
\saveTG{𧣲}{21245}
\saveTG{䖏}{21245}
\saveTG{𠇢}{21245}
\saveTG{𧳝}{21246}
\saveTG{便}{21246}
\saveTG{𢍷}{21246}
\saveTG{𢕯}{21246}
\saveTG{𠊼}{21246}
\saveTG{倬}{21246}
\saveTG{𢔄}{21246}
\saveTG{𧣺}{21246}
\saveTG{㪥}{21247}
\saveTG{㪭}{21247}
\saveTG{𢿊}{21247}
\saveTG{𣀞}{21247}
\saveTG{𢿑}{21247}
\saveTG{𢼾}{21247}
\saveTG{㪜}{21247}
\saveTG{𢼨}{21247}
\saveTG{𠧘}{21247}
\saveTG{𣥰}{21247}
\saveTG{𢖒}{21247}
\saveTG{𢕶}{21247}
\saveTG{𠊳}{21247}
\saveTG{𪝐}{21247}
\saveTG{𪝫}{21247}
\saveTG{𢽥}{21247}
\saveTG{𥀡}{21247}
\saveTG{𠈯}{21247}
\saveTG{𠮘}{21247}
\saveTG{𧤧}{21247}
\saveTG{𪝺}{21247}
\saveTG{𥜷}{21247}
\saveTG{𠑍}{21247}
\saveTG{侢}{21247}
\saveTG{敽}{21247}
\saveTG{優}{21247}
\saveTG{𢿝}{21247}
\saveTG{𢾈}{21247}
\saveTG{𢾍}{21247}
\saveTG{𢿢}{21247}
\saveTG{𢿮}{21247}
\saveTG{𢽩}{21247}
\saveTG{𧆛}{21247}
\saveTG{𡥋}{21247}
\saveTG{𧇆}{21247}
\saveTG{𧣀}{21247}
\saveTG{𢽝}{21247}
\saveTG{𠆸}{21247}
\saveTG{𢓉}{21247}
\saveTG{𢿉}{21247}
\saveTG{𧤿}{21247}
\saveTG{䖍}{21248}
\saveTG{𠑛}{21248}
\saveTG{𧢽}{21248}
\saveTG{𧲺}{21249}
\saveTG{䝣}{21249}
\saveTG{虖}{21249}
\saveTG{伻}{21249}
\saveTG{𧆠}{21252}
\saveTG{𢕺}{21252}
\saveTG{𧴖}{21253}
\saveTG{𧇱}{21253}
\saveTG{歳}{21253}
\saveTG{歲}{21253}
\saveTG{䖗}{21256}
\saveTG{𧆥}{21256}
\saveTG{𠐡}{21256}
\saveTG{僲}{21256}
\saveTG{価}{21260}
\saveTG{佔}{21260}
\saveTG{𠇅}{21260}
\saveTG{𫊝}{21261}
\saveTG{僭}{21261}
\saveTG{𪜵}{21261}
\saveTG{𧆨}{21261}
\saveTG{𢓲}{21261}
\saveTG{𤕻}{21261}
\saveTG{𠇙}{21261}
\saveTG{𧭭}{21261}
\saveTG{𠑫}{21261}
\saveTG{俉}{21261}
\saveTG{𧳎}{21261}
\saveTG{𢓕}{21262}
\saveTG{𠐞}{21262}
\saveTG{𠑵}{21262}
\saveTG{𠐧}{21262}
\saveTG{佰}{21262}
\saveTG{貊}{21262}
\saveTG{𧲸}{21262}
\saveTG{佦}{21262}
\saveTG{偭}{21262}
\saveTG{𫖄}{21262}
\saveTG{𧣔}{21262}
\saveTG{𡖞}{21262}
\saveTG{𫊡}{21262}
\saveTG{𢔲}{21262}
\saveTG{𪖚}{21262}
\saveTG{𠊜}{21262}
\saveTG{𧈘}{21263}
\saveTG{𧈔}{21263}
\saveTG{𧇕}{21263}
\saveTG{𧇄}{21263}
\saveTG{𧇗}{21264}
\saveTG{徆}{21264}
\saveTG{𪖼}{21264}
\saveTG{𠉦}{21264}
\saveTG{𡶼}{21264}
\saveTG{𨟻}{21264}
\saveTG{𠈇}{21264}
\saveTG{𧆭}{21265}
\saveTG{𠍚}{21266}
\saveTG{偪}{21266}
\saveTG{𧇪}{21268}
\saveTG{𧆱}{21268}
\saveTG{𧇩}{21268}
\saveTG{𧇖}{21268}
\saveTG{𧇋}{21268}
\saveTG{𥈠}{21268}
\saveTG{𠏻}{21268}
\saveTG{𧳏}{21269}
\saveTG{俖}{21269}
\saveTG{𪙿}{21272}
\saveTG{𧇏}{21272}
\saveTG{𠎣}{21272}
\saveTG{𧆣}{21272}
\saveTG{𢖎}{21277}
\saveTG{𢔣}{21277}
\saveTG{𫊞}{21277}
\saveTG{𧆾}{21281}
\saveTG{徙}{21281}
\saveTG{虡}{21281}
\saveTG{𠊤}{21281}
\saveTG{𧆩}{21281}
\saveTG{𧇽}{21281}
\saveTG{𤖈}{21281}
\saveTG{𧇀}{21281}
\saveTG{颛}{21282}
\saveTG{倾}{21282}
\saveTG{侦}{21282}
\saveTG{须}{21282}
\saveTG{颓}{21282}
\saveTG{𢒩}{21282}
\saveTG{频}{21282}
\saveTG{𠎮}{21282}
\saveTG{㑔}{21282}
\saveTG{𫖰}{21282}
\saveTG{𢓺}{21282}
\saveTG{𧤼}{21282}
\saveTG{颅}{21282}
\saveTG{}{21282}
\saveTG{𤕺}{21284}
\saveTG{𥎼}{21284}
\saveTG{𪝭}{21284}
\saveTG{偄}{21284}
\saveTG{虞}{21284}
\saveTG{𢓍}{21284}
\saveTG{偵}{21286}
\saveTG{顓}{21286}
\saveTG{𧵰}{21286}
\saveTG{𪝖}{21286}
\saveTG{𩒌}{21286}
\saveTG{𩕣}{21286}
\saveTG{䫩}{21286}
\saveTG{䐓}{21286}
\saveTG{䫉}{21286}
\saveTG{𩒧}{21286}
\saveTG{𩔺}{21286}
\saveTG{𩕚}{21286}
\saveTG{𩕷}{21286}
\saveTG{𩕨}{21286}
\saveTG{顪}{21286}
\saveTG{𩔤}{21286}
\saveTG{𩕘}{21286}
\saveTG{𩓯}{21286}
\saveTG{𣾢}{21286}
\saveTG{𩕢}{21286}
\saveTG{𩓤}{21286}
\saveTG{𩓼}{21286}
\saveTG{䫈}{21286}
\saveTG{𩓅}{21286}
\saveTG{𩕓}{21286}
\saveTG{䫂}{21286}
\saveTG{𩑮}{21286}
\saveTG{𩕰}{21286}
\saveTG{𩒙}{21286}
\saveTG{𩕬}{21286}
\saveTG{𢔤}{21286}
\saveTG{䫵}{21286}
\saveTG{𢖤}{21286}
\saveTG{𩔔}{21286}
\saveTG{𩳾}{21286}
\saveTG{𩔅}{21286}
\saveTG{𠐾}{21286}
\saveTG{㑯}{21286}
\saveTG{𠑯}{21286}
\saveTG{𠑘}{21286}
\saveTG{𪝽}{21286}
\saveTG{𠐺}{21286}
\saveTG{䫛}{21286}
\saveTG{𠐬}{21286}
\saveTG{𩑭}{21286}
\saveTG{𩔂}{21286}
\saveTG{𤖤}{21286}
\saveTG{頻}{21286}
\saveTG{顀}{21286}
\saveTG{價}{21286}
\saveTG{顱}{21286}
\saveTG{傾}{21286}
\saveTG{頹}{21286}
\saveTG{頺}{21286}
\saveTG{頽}{21286}
\saveTG{頠}{21286}
\saveTG{須}{21286}
\saveTG{𠎌}{21289}
\saveTG{伓}{21290}
\saveTG{𠇣}{21291}
\saveTG{𧴋}{21291}
\saveTG{㑐}{21291}
\saveTG{僄}{21291}
\saveTG{徱}{21291}
\saveTG{𧇴}{21291}
\saveTG{觨}{21292}
\saveTG{𠋾}{21292}
\saveTG{𠍞}{21293}
\saveTG{僳}{21294}
\saveTG{傈}{21294}
\saveTG{𤡫}{21294}
\saveTG{𧇥}{21294}
\saveTG{𧇃}{21294}
\saveTG{𠍲}{21294}
\saveTG{𥟃}{21294}
\saveTG{𠍙}{21294}
\saveTG{𥹐}{21294}
\saveTG{䝠}{21296}
\saveTG{傆}{21296}
\saveTG{𨓑}{21302}
\saveTG{𨕭}{21305}
\saveTG{𫙔}{21307}
\saveTG{𩷵}{21308}
\saveTG{䱮}{21311}
\saveTG{𪁹}{21311}
\saveTG{鯡}{21311}
\saveTG{鱋}{21312}
\saveTG{鮿}{21312}
\saveTG{䱍}{21312}
\saveTG{𩸵}{21312}
\saveTG{𣦑}{21312}
\saveTG{𩾬}{21312}
\saveTG{䱭}{21312}
\saveTG{𪀤}{21312}
\saveTG{𣦘}{21312}
\saveTG{𩷏}{21312}
\saveTG{𩻩}{21312}
\saveTG{䱌}{21312}
\saveTG{魭}{21312}
\saveTG{𩸋}{21312}
\saveTG{𩶝}{21312}
\saveTG{𪈒}{21312}
\saveTG{𩽏}{21312}
\saveTG{魟}{21312}
\saveTG{魱}{21312}
\saveTG{鱺}{21312}
\saveTG{鱸}{21312}
\saveTG{𩺧}{21314}
\saveTG{鰋}{21314}
\saveTG{𩵭}{21314}
\saveTG{𪀒}{21314}
\saveTG{𩶪}{21314}
\saveTG{𩷗}{21314}
\saveTG{𩻖}{21314}
\saveTG{𩺡}{21314}
\saveTG{䱳}{21315}
\saveTG{𪈩}{21315}
\saveTG{鱷}{21316}
\saveTG{𩺱}{21316}
\saveTG{䳼}{21316}
\saveTG{鰸}{21316}
\saveTG{𩽙}{21316}
\saveTG{䲔}{21316}
\saveTG{䱎}{21316}
\saveTG{鮔}{21317}
\saveTG{𩵪}{21317}
\saveTG{𪈳}{21317}
\saveTG{䳖}{21317}
\saveTG{𩺺}{21317}
\saveTG{𩵡}{21317}
\saveTG{𩶏}{21317}
\saveTG{𪀏}{21317}
\saveTG{虢}{21317}
\saveTG{鯱}{21317}
\saveTG{𪆖}{21318}
\saveTG{䱏}{21318}
\saveTG{魾}{21319}
\saveTG{𪀇}{21319}
\saveTG{䰳}{21320}
\saveTG{魺}{21320}
\saveTG{鴚}{21320}
\saveTG{𩵧}{21321}
\saveTG{䲗}{21321}
\saveTG{𩷖}{21322}
\saveTG{𩽶}{21326}
\saveTG{䱴}{21326}
\saveTG{𩹺}{21327}
\saveTG{𩶁}{21327}
\saveTG{𩵿}{21327}
\saveTG{𩿐}{21327}
\saveTG{𪆦}{21327}
\saveTG{𫛌}{21327}
\saveTG{𩺂}{21327}
\saveTG{䲐}{21327}
\saveTG{鮞}{21327}
\saveTG{鱱}{21327}
\saveTG{鰢}{21327}
\saveTG{鷌}{21327}
\saveTG{鱬}{21327}
\saveTG{魳}{21327}
\saveTG{鰤}{21327}
\saveTG{鶳}{21327}
\saveTG{𩾳}{21327}
\saveTG{忐}{21331}
\saveTG{𢙡}{21331}
\saveTG{𩻺}{21331}
\saveTG{𩵷}{21331}
\saveTG{𤑅}{21331}
\saveTG{𨊘}{21331}
\saveTG{𪫠}{21331}
\saveTG{惩}{21331}
\saveTG{𫙠}{21332}
\saveTG{𩸕}{21332}
\saveTG{𩷩}{21332}
\saveTG{㦣}{21332}
\saveTG{𢚨}{21332}
\saveTG{𢗓}{21332}
\saveTG{𢚎}{21332}
\saveTG{𤎴}{21332}
\saveTG{𩻆}{21332}
\saveTG{𤏹}{21332}
\saveTG{𢘢}{21332}
\saveTG{愆}{21332}
\saveTG{鱁}{21333}
\saveTG{𢝐}{21334}
\saveTG{𢛂}{21334}
\saveTG{㤐}{21336}
\saveTG{𤑰}{21336}
\saveTG{𥡤}{21336}
\saveTG{𤐸}{21336}
\saveTG{𤓜}{21336}
\saveTG{点}{21336}
\saveTG{𢤧}{21338}
\saveTG{𩼲}{21338}
\saveTG{𢥨}{21338}
\saveTG{𢙤}{21339}
\saveTG{𩵟}{21340}
\saveTG{𩾝}{21340}
\saveTG{鰬}{21340}
\saveTG{䱓}{21341}
\saveTG{𩽪}{21342}
\saveTG{𡭑}{21342}
\saveTG{𩶀}{21342}
\saveTG{𩹐}{21342}
\saveTG{𩽔}{21343}
\saveTG{𪀺}{21344}
\saveTG{𩸷}{21344}
\saveTG{𩷹}{21346}
\saveTG{𪂱}{21346}
\saveTG{鯾}{21346}
\saveTG{鯁}{21346}
\saveTG{鱏}{21346}
\saveTG{𢽨}{21347}
\saveTG{𢻭}{21347}
\saveTG{𢾓}{21347}
\saveTG{𩽇}{21347}
\saveTG{𩺇}{21348}
\saveTG{鮃}{21349}
\saveTG{𩶨}{21349}
\saveTG{鱥}{21352}
\saveTG{鱥}{21353}
\saveTG{𫚄}{21356}
\saveTG{𩵯}{21357}
\saveTG{鮎}{21360}
\saveTG{𩺏}{21361}
\saveTG{鱩}{21361}
\saveTG{鯃}{21361}
\saveTG{𩽊}{21362}
\saveTG{𩹠}{21362}
\saveTG{𩉃}{21362}
\saveTG{鮖}{21362}
\saveTG{𩺦}{21363}
\saveTG{鯂}{21364}
\saveTG{𫙘}{21364}
\saveTG{𫚳}{21364}
\saveTG{𦧰}{21364}
\saveTG{鰏}{21366}
\saveTG{𩻛}{21367}
\saveTG{䱖}{21377}
\saveTG{鱈}{21377}
\saveTG{鱖}{21382}
\saveTG{𩹓}{21384}
\saveTG{𩔨}{21386}
\saveTG{𩹰}{21386}
\saveTG{𩕖}{21386}
\saveTG{𩹸}{21386}
\saveTG{𫚆}{21386}
\saveTG{𫖤}{21386}
\saveTG{𩕞}{21386}
\saveTG{顦}{21386}
\saveTG{顖}{21386}
\saveTG{頱}{21386}
\saveTG{𩵣}{21390}
\saveTG{鰾}{21391}
\saveTG{鮛}{21391}
\saveTG{䱈}{21391}
\saveTG{𪆫}{21394}
\saveTG{𠤏}{21400}
\saveTG{𠦀}{21401}
\saveTG{𫆎}{21401}
\saveTG{𣬂}{21401}
\saveTG{𦪩}{21402}
\saveTG{𠦋}{21402}
\saveTG{嬃}{21404}
\saveTG{𡜟}{21404}
\saveTG{𡠋}{21404}
\saveTG{媭}{21404}
\saveTG{㔬}{21406}
\saveTG{颦}{21406}
\saveTG{𩖓}{21406}
\saveTG{𩔾}{21406}
\saveTG{𪉲}{21406}
\saveTG{𪉫}{21406}
\saveTG{卓}{21406}
\saveTG{顰}{21406}
\saveTG{𠧾}{21407}
\saveTG{𡥣}{21407}
\saveTG{攴}{21407}
\saveTG{𠧔}{21407}
\saveTG{舻}{21407}
\saveTG{𠮅}{21407}
\saveTG{𢻛}{21407}
\saveTG{𠦮}{21408}
\saveTG{䡁}{21411}
\saveTG{𦩋}{21411}
\saveTG{𦪽}{21411}
\saveTG{𨊛}{21412}
\saveTG{𠓄}{21412}
\saveTG{舡}{21412}
\saveTG{艫}{21412}
\saveTG{𩆟}{21412}
\saveTG{䑭}{21412}
\saveTG{𦫄}{21412}
\saveTG{𦫊}{21412}
\saveTG{𦪡}{21412}
\saveTG{𦨝}{21412}
\saveTG{𦩒}{21412}
\saveTG{𠄀}{21414}
\saveTG{𨉉}{21414}
\saveTG{𨊑}{21416}
\saveTG{𩈠}{21416}
\saveTG{𤭟}{21417}
\saveTG{𤭂}{21417}
\saveTG{㼰}{21417}
\saveTG{𤮭}{21417}
\saveTG{𪩨}{21417}
\saveTG{𨈚}{21417}
\saveTG{𦩕}{21417}
\saveTG{甁}{21417}
\saveTG{𨈤}{21417}
\saveTG{瓾}{21417}
\saveTG{䑢}{21417}
\saveTG{𦪍}{21418}
\saveTG{𦨍}{21420}
\saveTG{舸}{21420}
\saveTG{𦨵}{21421}
\saveTG{𩶷}{21421}
\saveTG{𦨘}{21421}
\saveTG{䑰}{21422}
\saveTG{𣦖}{21422}
\saveTG{𩱅}{21427}
\saveTG{艣}{21427}
\saveTG{駂}{21427}
\saveTG{𠧌}{21427}
\saveTG{𨉿}{21427}
\saveTG{𣦃}{21427}
\saveTG{𫇛}{21427}
\saveTG{𧇧}{21427}
\saveTG{𨉸}{21427}
\saveTG{𥜾}{21427}
\saveTG{𠬣}{21427}
\saveTG{𦨠}{21431}
\saveTG{𫇟}{21431}
\saveTG{𦫆}{21431}
\saveTG{𨉎}{21432}
\saveTG{𨉼}{21433}
\saveTG{𩽷}{21437}
\saveTG{𪜨}{21440}
\saveTG{𢆒}{21441}
\saveTG{𣥎}{21441}
\saveTG{𣥭}{21441}
\saveTG{𢆢}{21441}
\saveTG{𧯭}{21441}
\saveTG{𪥤}{21441}
\saveTG{𡛗}{21441}
\saveTG{𢆩}{21441}
\saveTG{𢌱}{21442}
\saveTG{𢡬}{21442}
\saveTG{𨊞}{21442}
\saveTG{𡤫}{21442}
\saveTG{𨊈}{21446}
\saveTG{䑲}{21446}
\saveTG{𨉔}{21446}
\saveTG{𠧺}{21447}
\saveTG{𢽅}{21447}
\saveTG{歬}{21447}
\saveTG{㪏}{21447}
\saveTG{𨈙}{21447}
\saveTG{𢿎}{21447}
\saveTG{㪕}{21447}
\saveTG{𢼻}{21447}
\saveTG{𣬓}{21448}
\saveTG{𦫃}{21461}
\saveTG{𨉥}{21462}
\saveTG{𨊚}{21462}
\saveTG{𦧐}{21464}
\saveTG{𦨯}{21464}
\saveTG{舾}{21464}
\saveTG{𦩡}{21466}
\saveTG{𪉞}{21466}
\saveTG{𡴃}{21472}
\saveTG{𠧫}{21476}
\saveTG{艝}{21477}
\saveTG{𨊝}{21482}
\saveTG{𦪘}{21482}
\saveTG{𨉊}{21484}
\saveTG{頯}{21486}
\saveTG{顊}{21486}
\saveTG{䫧}{21486}
\saveTG{䫋}{21486}
\saveTG{𡡓}{21486}
\saveTG{𦩼}{21486}
\saveTG{䪻}{21486}
\saveTG{𩓀}{21486}
\saveTG{𩓿}{21486}
\saveTG{𩔃}{21486}
\saveTG{𩓚}{21486}
\saveTG{𩔋}{21486}
\saveTG{𩒟}{21486}
\saveTG{𩑦}{21486}
\saveTG{𩖌}{21486}
\saveTG{𩒶}{21486}
\saveTG{䫌}{21486}
\saveTG{𩓖}{21486}
\saveTG{𦨗}{21487}
\saveTG{𩺤}{21494}
\saveTG{𡤍}{21496}
\saveTG{𢮠}{21502}
\saveTG{𤜂}{21502}
\saveTG{𠍬}{21506}
\saveTG{𨏞}{21506}
\saveTG{𣥷}{21506}
\saveTG{魖}{21511}
\saveTG{𤮞}{21511}
\saveTG{𤜆}{21511}
\saveTG{𤛏}{21512}
\saveTG{魉}{21512}
\saveTG{魎}{21512}
\saveTG{𤜙}{21512}
\saveTG{𤙆}{21512}
\saveTG{𪺰}{21512}
\saveTG{牼}{21512}
\saveTG{𩙵}{21513}
\saveTG{𤚕}{21514}
\saveTG{𤜍}{21515}
\saveTG{𤜅}{21515}
\saveTG{𤛐}{21516}
\saveTG{㹔}{21516}
\saveTG{㸿}{21517}
\saveTG{𤜎}{21517}
\saveTG{𪮀}{21517}
\saveTG{魒}{21519}
\saveTG{𤘹}{21519}
\saveTG{𪺓}{21520}
\saveTG{牱}{21520}
\saveTG{𤘖}{21520}
\saveTG{𤙀}{21527}
\saveTG{𤚴}{21527}
\saveTG{犡}{21527}
\saveTG{𩣣}{21527}
\saveTG{㹘}{21527}
\saveTG{𤜏}{21527}
\saveTG{𤛟}{21527}
\saveTG{𫘎}{21527}
\saveTG{𪺵}{21531}
\saveTG{𤘳}{21531}
\saveTG{𤙦}{21532}
\saveTG{𤙐}{21534}
\saveTG{㸩}{21540}
\saveTG{𤚳}{21540}
\saveTG{𤙢}{21541}
\saveTG{𤚾}{21543}
\saveTG{𤚘}{21544}
\saveTG{𤚷}{21546}
\saveTG{𤙴}{21546}
\saveTG{𤛾}{21547}
\saveTG{𤚨}{21547}
\saveTG{𤛽}{21547}
\saveTG{𤚿}{21547}
\saveTG{㹛}{21547}
\saveTG{𤘴}{21547}
\saveTG{𤘾}{21549}
\saveTG{拜}{21550}
\saveTG{𢪙}{21550}
\saveTG{𣥟}{21551}
\saveTG{𤛫}{21551}
\saveTG{牾}{21561}
\saveTG{𤚛}{21562}
\saveTG{牺}{21564}
\saveTG{𢬣}{21564}
\saveTG{𩒹}{21578}
\saveTG{𩓾}{21578}
\saveTG{𠤧}{21582}
\saveTG{𤛦}{21582}
\saveTG{𤘠}{21584}
\saveTG{𩒰}{21586}
\saveTG{𩑩}{21586}
\saveTG{𩓞}{21586}
\saveTG{𩓻}{21586}
\saveTG{𪺘}{21586}
\saveTG{𤘢}{21590}
\saveTG{𢪓}{21592}
\saveTG{㹉}{21596}
\saveTG{𠧧}{21600}
\saveTG{𠧸}{21600}
\saveTG{𠧚}{21600}
\saveTG{𠧪}{21600}
\saveTG{㔽}{21600}
\saveTG{𠝘}{21600}
\saveTG{鹵}{21600}
\saveTG{占}{21600}
\saveTG{卥}{21600}
\saveTG{卤}{21600}
\saveTG{卣}{21600}
\saveTG{𣥤}{21601}
\saveTG{讆}{21601}
\saveTG{讏}{21601}
\saveTG{𣦗}{21601}
\saveTG{𢒷}{21601}
\saveTG{𧭹}{21601}
\saveTG{𠮳}{21601}
\saveTG{𣥜}{21601}
\saveTG{㫖}{21601}
\saveTG{𫍐}{21601}
\saveTG{𥓇}{21602}
\saveTG{𠲻}{21602}
\saveTG{𤽤}{21602}
\saveTG{𣦉}{21602}
\saveTG{𣥸}{21604}
\saveTG{𣊑}{21604}
\saveTG{䖜}{21607}
\saveTG{𪘭}{21607}
\saveTG{睿}{21608}
\saveTG{𧮲}{21608}
\saveTG{𥅵}{21608}
\saveTG{𣑅}{21609}
\saveTG{𣑾}{21609}
\saveTG{𠄩}{21610}
\saveTG{𩇾}{21611}
\saveTG{𤾅}{21611}
\saveTG{馡}{21611}
\saveTG{𤿅}{21612}
\saveTG{𤴅}{21612}
\saveTG{㿨}{21612}
\saveTG{𩐏}{21612}
\saveTG{𩠡}{21612}
\saveTG{𩠝}{21612}
\saveTG{𥓤}{21612}
\saveTG{𥌠}{21612}
\saveTG{𤿂}{21614}
\saveTG{皬}{21615}
\saveTG{瓵}{21617}
\saveTG{㼟}{21617}
\saveTG{㽃}{21617}
\saveTG{𤭇}{21617}
\saveTG{𤽾}{21617}
\saveTG{𪿑}{21617}
\saveTG{𪵤}{21617}
\saveTG{𤭧}{21617}
\saveTG{𤱹}{21620}
\saveTG{㱒}{21621}
\saveTG{䭲}{21621}
\saveTG{𤽗}{21627}
\saveTG{𩢠}{21627}
\saveTG{𩡋}{21628}
\saveTG{𪿟}{21632}
\saveTG{𤾨}{21632}
\saveTG{舔}{21638}
\saveTG{𥝾}{21640}
\saveTG{𤽂}{21640}
\saveTG{𤳺}{21642}
\saveTG{𪐒}{21642}
\saveTG{𦖞}{21642}
\saveTG{𩡝}{21646}
\saveTG{𤽐}{21647}
\saveTG{𢽿}{21647}
\saveTG{𩡘}{21647}
\saveTG{𢾃}{21647}
\saveTG{𢿥}{21647}
\saveTG{𢼛}{21647}
\saveTG{敁}{21647}
\saveTG{𢼤}{21647}
\saveTG{㪫}{21647}
\saveTG{𢿁}{21647}
\saveTG{𢼒}{21647}
\saveTG{𣥧}{21661}
\saveTG{𤾻}{21661}
\saveTG{𪉜}{21662}
\saveTG{𠨋}{21662}
\saveTG{𥕿}{21662}
\saveTG{𦧖}{21664}
\saveTG{𪊂}{21668}
\saveTG{𫖭}{21682}
\saveTG{頶}{21686}
\saveTG{𩔲}{21686}
\saveTG{𥍎}{21686}
\saveTG{𩔻}{21686}
\saveTG{䫠}{21686}
\saveTG{䫇}{21686}
\saveTG{𩒎}{21686}
\saveTG{䪿}{21686}
\saveTG{𩕵}{21686}
\saveTG{𤾶}{21686}
\saveTG{𩕏}{21686}
\saveTG{𩒨}{21686}
\saveTG{䫁}{21686}
\saveTG{頢}{21686}
\saveTG{𩑻}{21686}
\saveTG{頟}{21686}
\saveTG{𩒵}{21686}
\saveTG{頕}{21686}
\saveTG{䪷}{21686}
\saveTG{䜭}{21688}
\saveTG{𤾛}{21691}
\saveTG{䫝}{21696}
\saveTG{𪨷}{21701}
\saveTG{𠧒}{21710}
\saveTG{巄}{21711}
\saveTG{毴}{21711}
\saveTG{𣰜}{21711}
\saveTG{𣰵}{21711}
\saveTG{𪙫}{21712}
\saveTG{屸}{21712}
\saveTG{𡷯}{21712}
\saveTG{𪘰}{21712}
\saveTG{𪙾}{21712}
\saveTG{䶥}{21712}
\saveTG{㠊}{21712}
\saveTG{𡷨}{21712}
\saveTG{㠠}{21712}
\saveTG{𣰃}{21712}
\saveTG{𪙽}{21712}
\saveTG{屼}{21712}
\saveTG{岏}{21712}
\saveTG{𡹄}{21712}
\saveTG{旣}{21712}
\saveTG{㠣}{21712}
\saveTG{𪗚}{21712}
\saveTG{𡷍}{21712}
\saveTG{𠒄}{21712}
\saveTG{𣰹}{21713}
\saveTG{峌}{21714}
\saveTG{𡹪}{21714}
\saveTG{𣯄}{21714}
\saveTG{𣭿}{21714}
\saveTG{𣯋}{21714}
\saveTG{𪗻}{21714}
\saveTG{𣥅}{21714}
\saveTG{𪘽}{21714}
\saveTG{𪘬}{21714}
\saveTG{崕}{21714}
\saveTG{𡹶}{21714}
\saveTG{𣭞}{21714}
\saveTG{岖}{21714}
\saveTG{𡾜}{21715}
\saveTG{𣯦}{21715}
\saveTG{𣮀}{21715}
\saveTG{𣮚}{21715}
\saveTG{𠧱}{21716}
\saveTG{𡶅}{21716}
\saveTG{𣯗}{21716}
\saveTG{𠧠}{21716}
\saveTG{𠧟}{21716}
\saveTG{𠧴}{21716}
\saveTG{𣭏}{21716}
\saveTG{𣮻}{21716}
\saveTG{𣮗}{21716}
\saveTG{𪙛}{21716}
\saveTG{𣰑}{21716}
\saveTG{𠨅}{21716}
\saveTG{峘}{21716}
\saveTG{嶇}{21716}
\saveTG{毢}{21716}
\saveTG{毡}{21716}
\saveTG{𤬷}{21717}
\saveTG{𡵊}{21717}
\saveTG{䖓}{21717}
\saveTG{𪘛}{21717}
\saveTG{𫗟}{21717}
\saveTG{𣬅}{21717}
\saveTG{䶙}{21717}
\saveTG{㼘}{21717}
\saveTG{𡵐}{21717}
\saveTG{𡿕}{21717}
\saveTG{𡼟}{21717}
\saveTG{𤮯}{21717}
\saveTG{岠}{21717}
\saveTG{𣯪}{21718}
\saveTG{𦫤}{21718}
\saveTG{饾}{21718}
\saveTG{𠃂}{21719}
\saveTG{𨸿}{21719}
\saveTG{岯}{21719}
\saveTG{𡴵}{21720}
\saveTG{饤}{21720}
\saveTG{𪗞}{21721}
\saveTG{𩥓}{21723}
\saveTG{䶗}{21726}
\saveTG{𡹣}{21726}
\saveTG{㞹}{21726}
\saveTG{𪙚}{21727}
\saveTG{𠧤}{21727}
\saveTG{𩣂}{21727}
\saveTG{𡹻}{21727}
\saveTG{𪙺}{21727}
\saveTG{𪙥}{21727}
\saveTG{𠧷}{21727}
\saveTG{峏}{21727}
\saveTG{巁}{21727}
\saveTG{嶿}{21727}
\saveTG{師}{21727}
\saveTG{嶀}{21727}
\saveTG{屿}{21727}
\saveTG{}{21727}
\saveTG{𠀴}{21730}
\saveTG{𡵮}{21731}
\saveTG{𡶛}{21731}
\saveTG{𢇌}{21731}
\saveTG{𢆹}{21731}
\saveTG{峠}{21731}
\saveTG{𫌍}{21732}
\saveTG{𪘝}{21732}
\saveTG{𨼽}{21732}
\saveTG{𩞚}{21732}
\saveTG{𩜾}{21732}
\saveTG{𧙨}{21732}
\saveTG{𣥏}{21732}
\saveTG{𡾅}{21736}
\saveTG{𫗞}{21740}
\saveTG{屽}{21740}
\saveTG{饵}{21740}
\saveTG{岍}{21740}
\saveTG{齖}{21740}
\saveTG{岈}{21740}
\saveTG{𪗙}{21740}
\saveTG{㟁}{21741}
\saveTG{𡼆}{21741}
\saveTG{岈}{21742}
\saveTG{𡶨}{21742}
\saveTG{崾}{21744}
\saveTG{𡷟}{21744}
\saveTG{𪗛}{21744}
\saveTG{峺}{21746}
\saveTG{𡾂}{21747}
\saveTG{𡵨}{21747}
\saveTG{𢼍}{21747}
\saveTG{𣀎}{21747}
\saveTG{𪗡}{21747}
\saveTG{𢽭}{21747}
\saveTG{𢽣}{21747}
\saveTG{敮}{21747}
\saveTG{巎}{21747}
\saveTG{𡺷}{21747}
\saveTG{𢾆}{21747}
\saveTG{𢾱}{21747}
\saveTG{𢼌}{21747}
\saveTG{𠭔}{21747}
\saveTG{𢻲}{21747}
\saveTG{岼}{21749}
\saveTG{𡻻}{21749}
\saveTG{𪗥}{21750}
\saveTG{𡼛}{21753}
\saveTG{岾}{21760}
\saveTG{龉}{21761}
\saveTG{峿}{21761}
\saveTG{𪘚}{21761}
\saveTG{𡺽}{21761}
\saveTG{齬}{21761}
\saveTG{𪗦}{21762}
\saveTG{𡶌}{21762}
\saveTG{𥇨}{21768}
\saveTG{𪕊}{21771}
\saveTG{𣥫}{21772}
\saveTG{𣥼}{21772}
\saveTG{𡵩}{21772}
\saveTG{𣦋}{21772}
\saveTG{𡾐}{21772}
\saveTG{𣦭}{21772}
\saveTG{𪙹}{21772}
\saveTG{齒}{21772}
\saveTG{𠚝}{21772}
\saveTG{歯}{21772}
\saveTG{齿}{21772}
\saveTG{𠤔}{21774}
\saveTG{𣬃}{21777}
\saveTG{嶥}{21782}
\saveTG{顷}{21782}
\saveTG{𫖮}{21782}
\saveTG{𡽣}{21784}
\saveTG{𫗬}{21784}
\saveTG{𩓟}{21785}
\saveTG{𩔪}{21786}
\saveTG{𩑒}{21786}
\saveTG{䫜}{21786}
\saveTG{𩑴}{21786}
\saveTG{頧}{21786}
\saveTG{𫖝}{21786}
\saveTG{崸}{21786}
\saveTG{𡺭}{21786}
\saveTG{𩔙}{21786}
\saveTG{𩑗}{21786}
\saveTG{頃}{21786}
\saveTG{𩑑}{21786}
\saveTG{𡽹}{21786}
\saveTG{𡼔}{21786}
\saveTG{𩑨}{21786}
\saveTG{𪚉}{21786}
\saveTG{𩖁}{21786}
\saveTG{䪼}{21786}
\saveTG{𠂾}{21790}
\saveTG{㟽}{21791}
\saveTG{𦅸}{21792}
\saveTG{㟳}{21794}
\saveTG{㟲}{21796}
\saveTG{仧}{21801}
\saveTG{躗}{21801}
\saveTG{𠤤}{21801}
\saveTG{躛}{21801}
\saveTG{歨}{21801}
\saveTG{𨑛}{21802}
\saveTG{贞}{21802}
\saveTG{𠤕}{21804}
\saveTG{奌}{21804}
\saveTG{𠧵}{21804}
\saveTG{𡚋}{21804}
\saveTG{𡙊}{21804}
\saveTG{𡙉}{21804}
\saveTG{𪟩}{21804}
\saveTG{𡙾}{21804}
\saveTG{𦁾}{21805}
\saveTG{𧷁}{21806}
\saveTG{贙}{21806}
\saveTG{貞}{21806}
\saveTG{𠧛}{21807}
\saveTG{𨙁}{21808}
\saveTG{𤓞}{21808}
\saveTG{𤉭}{21809}
\saveTG{𤈭}{21809}
\saveTG{𤌺}{21809}
\saveTG{𠧯}{21812}
\saveTG{𩽦}{21814}
\saveTG{𠧝}{21817}
\saveTG{𨇐}{21826}
\saveTG{𩣎}{21827}
\saveTG{𠧖}{21827}
\saveTG{𧶀}{21831}
\saveTG{䲣}{21840}
\saveTG{𣦌}{21841}
\saveTG{𦗁}{21842}
\saveTG{𡙏}{21846}
\saveTG{𪉗}{21846}
\saveTG{𢾋}{21847}
\saveTG{𣀶}{21847}
\saveTG{𢿡}{21847}
\saveTG{𥜱}{21847}
\saveTG{𥎩}{21848}
\saveTG{𨣐}{21864}
\saveTG{𣦁}{21881}
\saveTG{𣦊}{21881}
\saveTG{颎}{21882}
\saveTG{𩓎}{21886}
\saveTG{𩖂}{21886}
\saveTG{𩒠}{21886}
\saveTG{顚}{21886}
\saveTG{𩓙}{21886}
\saveTG{熲}{21886}
\saveTG{𫖣}{21886}
\saveTG{𩓈}{21886}
\saveTG{𩔆}{21886}
\saveTG{䫣}{21886}
\saveTG{𤎵}{21892}
\saveTG{𤓤}{21894}
\saveTG{𤈱}{21896}
\saveTG{𤈴}{21896}
\saveTG{𤐧}{21896}
\saveTG{𥘣}{21901}
\saveTG{𣦅}{21901}
\saveTG{𥘈}{21901}
\saveTG{尗}{21901}
\saveTG{澃}{21902}
\saveTG{𦅓}{21902}
\saveTG{𥾄}{21904}
\saveTG{𥹄}{21904}
\saveTG{𥼒}{21904}
\saveTG{槩}{21904}
\saveTG{𣡼}{21904}
\saveTG{𥻆}{21904}
\saveTG{𣖴}{21904}
\saveTG{𨤐}{21904}
\saveTG{𣏔}{21904}
\saveTG{榘}{21904}
\saveTG{𠨇}{21904}
\saveTG{𣟉}{21904}
\saveTG{𣗴}{21904}
\saveTG{𣓨}{21904}
\saveTG{㮚}{21904}
\saveTG{桌}{21904}
\saveTG{緋}{21911}
\saveTG{𪐖}{21911}
\saveTG{𥟍}{21911}
\saveTG{䆍}{21911}
\saveTG{穲}{21912}
\saveTG{纚}{21912}
\saveTG{纑}{21912}
\saveTG{稏}{21912}
\saveTG{穊}{21912}
\saveTG{𦄒}{21912}
\saveTG{𦅟}{21912}
\saveTG{䊺}{21912}
\saveTG{𫄖}{21912}
\saveTG{𥤀}{21912}
\saveTG{𦇔}{21912}
\saveTG{絚}{21912}
\saveTG{紅}{21912}
\saveTG{經}{21912}
\saveTG{𥚋}{21912}
\saveTG{𥠶}{21912}
\saveTG{䅉}{21912}
\saveTG{𦃄}{21912}
\saveTG{𦀍}{21912}
\saveTG{䋊}{21912}
\saveTG{䊼}{21912}
\saveTG{𥤞}{21912}
\saveTG{䋗}{21912}
\saveTG{𥾬}{21912}
\saveTG{𦃶}{21912}
\saveTG{𥡧}{21912}
\saveTG{𪏳}{21912}
\saveTG{緪}{21912}
\saveTG{𤪇}{21914}
\saveTG{秷}{21914}
\saveTG{絰}{21914}
\saveTG{䌳}{21914}
\saveTG{𦃷}{21914}
\saveTG{𦀇}{21914}
\saveTG{𦀰}{21914}
\saveTG{𤨗}{21914}
\saveTG{𦄕}{21914}
\saveTG{𦁪}{21914}
\saveTG{𦆑}{21914}
\saveTG{𦁩}{21914}
\saveTG{𦁽}{21914}
\saveTG{𥿁}{21914}
\saveTG{緸}{21914}
\saveTG{𥤊}{21915}
\saveTG{緾}{21915}
\saveTG{𥠍}{21915}
\saveTG{𪐌}{21916}
\saveTG{絙}{21916}
\saveTG{縆}{21916}
\saveTG{䌔}{21916}
\saveTG{繮}{21916}
\saveTG{𦇤}{21916}
\saveTG{𥾳}{21917}
\saveTG{䋌}{21917}
\saveTG{䌬}{21917}
\saveTG{秬}{21917}
\saveTG{𥝪}{21917}
\saveTG{甈}{21917}
\saveTG{𤮎}{21917}
\saveTG{𥾕}{21917}
\saveTG{𡰈}{21917}
\saveTG{㼡}{21917}
\saveTG{𧈊}{21917}
\saveTG{𦆡}{21917}
\saveTG{𦂬}{21917}
\saveTG{𦁲}{21917}
\saveTG{𤭽}{21917}
\saveTG{𦆠}{21918}
\saveTG{秠}{21919}
\saveTG{䋔}{21919}
\saveTG{糽}{21920}
\saveTG{𥾎}{21920}
\saveTG{絎}{21921}
\saveTG{𥞟}{21921}
\saveTG{𥞧}{21921}
\saveTG{𥝮}{21921}
\saveTG{𥾦}{21921}
\saveTG{䋍}{21926}
\saveTG{䋪}{21926}
\saveTG{𥞍}{21926}
\saveTG{𩻞}{21927}
\saveTG{𢀋}{21927}
\saveTG{𦅏}{21927}
\saveTG{䊸}{21927}
\saveTG{𥣭}{21927}
\saveTG{𦆨}{21927}
\saveTG{𦂲}{21927}
\saveTG{䌤}{21927}
\saveTG{𦇯}{21927}
\saveTG{穪}{21927}
\saveTG{𩢻}{21927}
\saveTG{繻}{21927}
\saveTG{緉}{21927}
\saveTG{𥢠}{21927}
\saveTG{𥝑}{21927}
\saveTG{𥿣}{21927}
\saveTG{𦄀}{21927}
\saveTG{穤}{21927}
\saveTG{𥿶}{21927}
\saveTG{𥾝}{21927}
\saveTG{𦃔}{21927}
\saveTG{䋑}{21927}
\saveTG{𥝜}{21927}
\saveTG{𥝽}{21927}
\saveTG{𦁥}{21927}
\saveTG{𣗱}{21930}
\saveTG{𦇟}{21931}
\saveTG{繧}{21931}
\saveTG{𦇼}{21931}
\saveTG{𥢚}{21931}
\saveTG{𦁢}{21932}
\saveTG{紜}{21932}
\saveTG{秐}{21932}
\saveTG{䌢}{21932}
\saveTG{𥢴}{21932}
\saveTG{𦂥}{21932}
\saveTG{𦁄}{21932}
\saveTG{𦄖}{21933}
\saveTG{𦆄}{21934}
\saveTG{𥡱}{21936}
\saveTG{𪐇}{21936}
\saveTG{𥡷}{21937}
\saveTG{䋬}{21938}
\saveTG{𦁔}{21938}
\saveTG{紆}{21940}
\saveTG{秆}{21940}
\saveTG{𦅙}{21940}
\saveTG{𥾍}{21940}
\saveTG{䄨}{21940}
\saveTG{䅍}{21941}
\saveTG{𥜂}{21942}
\saveTG{䋙}{21942}
\saveTG{䄰}{21942}
\saveTG{𥟫}{21942}
\saveTG{䌰}{21942}
\saveTG{䅶}{21943}
\saveTG{縟}{21943}
\saveTG{䄯}{21944}
\saveTG{䌁}{21944}
\saveTG{𠤡}{21944}
\saveTG{𫃫}{21944}
\saveTG{稉}{21946}
\saveTG{𦄹}{21946}
\saveTG{𥢏}{21946}
\saveTG{𦅰}{21946}
\saveTG{緶}{21946}
\saveTG{綆}{21946}
\saveTG{綽}{21946}
\saveTG{𦂰}{21947}
\saveTG{𦄱}{21947}
\saveTG{𦇉}{21947}
\saveTG{敊}{21947}
\saveTG{纋}{21947}
\saveTG{𢼲}{21947}
\saveTG{𦷻}{21947}
\saveTG{𢿆}{21947}
\saveTG{𥾲}{21947}
\saveTG{𢿹}{21947}
\saveTG{𢾦}{21947}
\saveTG{𥾵}{21947}
\saveTG{𣀝}{21947}
\saveTG{𥤙}{21947}
\saveTG{𥣯}{21947}
\saveTG{𦆷}{21947}
\saveTG{䌄}{21947}
\saveTG{𦁭}{21947}
\saveTG{𥠛}{21948}
\saveTG{𦇸}{21948}
\saveTG{𦂇}{21948}
\saveTG{秤}{21949}
\saveTG{𥾙}{21950}
\saveTG{𥾹}{21950}
\saveTG{𦅵}{21952}
\saveTG{𥢈}{21953}
\saveTG{𥣫}{21953}
\saveTG{𦇪}{21953}
\saveTG{穢}{21953}
\saveTG{黏}{21960}
\saveTG{秥}{21960}
\saveTG{縉}{21961}
\saveTG{𦇢}{21961}
\saveTG{䌮}{21961}
\saveTG{𦀡}{21961}
\saveTG{𥟊}{21961}
\saveTG{𦆇}{21961}
\saveTG{䄷}{21962}
\saveTG{絔}{21962}
\saveTG{𥢋}{21962}
\saveTG{𦇒}{21962}
\saveTG{緬}{21962}
\saveTG{𪉻}{21962}
\saveTG{𦂄}{21962}
\saveTG{𥿕}{21962}
\saveTG{綇}{21964}
\saveTG{䄼}{21964}
\saveTG{䄽}{21964}
\saveTG{絤}{21964}
\saveTG{𦆙}{21964}
\saveTG{𥟁}{21964}
\saveTG{䵗}{21966}
\saveTG{䋹}{21966}
\saveTG{稫}{21966}
\saveTG{𥞶}{21969}
\saveTG{𫃽}{21974}
\saveTG{𢀊}{21981}
\saveTG{䆊}{21981}
\saveTG{縰}{21981}
\saveTG{颍}{21982}
\saveTG{𦁡}{21982}
\saveTG{颕}{21982}
\saveTG{颖}{21982}
\saveTG{𥣘}{21984}
\saveTG{稬}{21984}
\saveTG{緛}{21984}
\saveTG{𦆺}{21986}
\saveTG{𩕛}{21986}
\saveTG{𥣶}{21986}
\saveTG{緽}{21986}
\saveTG{纐}{21986}
\saveTG{頴}{21986}
\saveTG{顈}{21986}
\saveTG{纈}{21986}
\saveTG{潁}{21986}
\saveTG{穎}{21986}
\saveTG{𩓄}{21986}
\saveTG{𩓰}{21986}
\saveTG{𩒛}{21986}
\saveTG{𦅐}{21986}
\saveTG{𩓵}{21986}
\saveTG{𩓬}{21986}
\saveTG{𥢆}{21986}
\saveTG{䅡}{21986}
\saveTG{𩓇}{21986}
\saveTG{𩓪}{21986}
\saveTG{䋶}{21986}
\saveTG{𦄼}{21986}
\saveTG{𦇖}{21986}
\saveTG{𦇦}{21986}
\saveTG{㯋}{21986}
\saveTG{𤊣}{21989}
\saveTG{𥝣}{21990}
\saveTG{紑}{21990}
\saveTG{縹}{21991}
\saveTG{𠨌}{21991}
\saveTG{𥟡}{21991}
\saveTG{䅺}{21991}
\saveTG{𦂘}{21994}
\saveTG{𦄽}{21994}
\saveTG{𥠲}{21994}
\saveTG{𦃊}{21994}
\saveTG{𥢕}{21994}
\saveTG{䌚}{21994}
\saveTG{縓}{21996}
\saveTG{𠚩}{22000}
\saveTG{𠂁}{22000}
\saveTG{𣶒}{22000}
\saveTG{𢍖}{22000}
\saveTG{𤰃}{22000}
\saveTG{𪺤}{22000}
\saveTG{𠂵}{22000}
\saveTG{𠂦}{22000}
\saveTG{𤰆}{22000}
\saveTG{𠛁}{22000}
\saveTG{𤖺}{22000}
\saveTG{𫂱}{22000}
\saveTG{𫃛}{22000}
\saveTG{𫎎}{22000}
\saveTG{𩰍}{22000}
\saveTG{刂}{22000}
\saveTG{川}{22000}
\saveTG{𩰎}{22001}
\saveTG{𩰜}{22001}
\saveTG{𩰘}{22001}
\saveTG{𩰒}{22001}
\saveTG{𩰔}{22001}
\saveTG{𠂎}{22001}
\saveTG{𩰖}{22001}
\saveTG{𩰚}{22001}
\saveTG{𩰝}{22002}
\saveTG{𣥂}{22002}
\saveTG{𩰟}{22003}
\saveTG{𩰛}{22004}
\saveTG{𩰑}{22004}
\saveTG{𩰙}{22008}
\saveTG{𩰏}{22008}
\saveTG{𩰗}{22008}
\saveTG{𩰓}{22008}
\saveTG{𩰞}{22009}
\saveTG{龻}{22009}
\saveTG{𩰐}{22009}
\saveTG{胤}{22010}
\saveTG{儿}{22010}
\saveTG{㸠}{22013}
\saveTG{𤗯}{22015}
\saveTG{𤗢}{22017}
\saveTG{㧌}{22017}
\saveTG{㲏}{22017}
\saveTG{𠃕}{22017}
\saveTG{𦚯}{22017}
\saveTG{𤖨}{22017}
\saveTG{𤗅}{22017}
\saveTG{𣂔}{22021}
\saveTG{𤗐}{22027}
\saveTG{彎}{22027}
\saveTG{𦙍}{22027}
\saveTG{片}{22027}
\saveTG{𣱱}{22030}
\saveTG{𤖴}{22041}
\saveTG{版}{22047}
\saveTG{𤖿}{22047}
\saveTG{𤗳}{22047}
\saveTG{㸟}{22061}
\saveTG{𤗹}{22069}
\saveTG{𠞇}{22070}
\saveTG{𤗠}{22072}
\saveTG{牐}{22077}
\saveTG{𥆘}{22080}
\saveTG{𤗵}{22085}
\saveTG{𤖱}{22094}
\saveTG{㸣}{22095}
\saveTG{鲗}{22100}
\saveTG{纠}{22100}
\saveTG{剴}{22100}
\saveTG{𠠂}{22100}
\saveTG{𠠞}{22100}
\saveTG{𪟏}{22100}
\saveTG{}{22100}
\saveTG{}{22100}
\saveTG{劙}{22100}
\saveTG{𠝶}{22100}
\saveTG{𠜄}{22100}
\saveTG{𠟞}{22100}
\saveTG{𠝍}{22100}
\saveTG{𠞏}{22100}
\saveTG{𠝤}{22100}
\saveTG{𠛆}{22100}
\saveTG{𠟁}{22100}
\saveTG{𠜐}{22100}
\saveTG{𠜶}{22100}
\saveTG{𪟎}{22100}
\saveTG{𠜸}{22100}
\saveTG{纼}{22100}
\saveTG{些}{22101}
\saveTG{丝}{22101}
\saveTG{峜}{22101}
\saveTG{亗}{22101}
\saveTG{嵳}{22102}
\saveTG{𥂽}{22102}
\saveTG{𡳿}{22102}
\saveTG{𡶦}{22102}
\saveTG{𡹝}{22102}
\saveTG{𦯥}{22102}
\saveTG{𪨻}{22102}
\saveTG{𣦱}{22102}
\saveTG{𥂑}{22102}
\saveTG{㞢}{22102}
\saveTG{𡿡}{22102}
\saveTG{𣦣}{22102}
\saveTG{𡽓}{22102}
\saveTG{𧗓}{22102}
\saveTG{𠄶}{22102}
\saveTG{𡺦}{22102}
\saveTG{𡽾}{22102}
\saveTG{𡶽}{22103}
\saveTG{㞷}{22104}
\saveTG{𡉚}{22104}
\saveTG{𨩹}{22104}
\saveTG{𨥢}{22104}
\saveTG{𨥣}{22104}
\saveTG{㘹}{22104}
\saveTG{𡷅}{22104}
\saveTG{𤨽}{22104}
\saveTG{𡒽}{22104}
\saveTG{峑}{22104}
\saveTG{峚}{22104}
\saveTG{坒}{22104}
\saveTG{𦤴}{22104}
\saveTG{𡉰}{22104}
\saveTG{𡎨}{22104}
\saveTG{𡉸}{22104}
\saveTG{𡾙}{22104}
\saveTG{𡐪}{22104}
\saveTG{𡴍}{22104}
\saveTG{𡴠}{22104}
\saveTG{𤫜}{22104}
\saveTG{𡋢}{22104}
\saveTG{𡍑}{22104}
\saveTG{𡼄}{22104}
\saveTG{𡹭}{22104}
\saveTG{埊}{22104}
\saveTG{𤪎}{22104}
\saveTG{𡓚}{22104}
\saveTG{𪩅}{22104}
\saveTG{𪨽}{22104}
\saveTG{𨤪}{22105}
\saveTG{𨤦}{22105}
\saveTG{𤯓}{22105}
\saveTG{𣆟}{22106}
\saveTG{𫚖}{22106}
\saveTG{𧯛}{22108}
\saveTG{𧯽}{22108}
\saveTG{𡾿}{22108}
\saveTG{𡸈}{22108}
\saveTG{豈}{22108}
\saveTG{岦}{22108}
\saveTG{豐}{22108}
\saveTG{𡸃}{22109}
\saveTG{𨧳}{22109}
\saveTG{𨨬}{22109}
\saveTG{鋫}{22109}
\saveTG{鑾}{22109}
\saveTG{崟}{22109}
\saveTG{鈭}{22109}
\saveTG{纰}{22110}
\saveTG{此}{22110}
\saveTG{㠑}{22111}
\saveTG{𩲵}{22111}
\saveTG{𧾭}{22112}
\saveTG{𢇅}{22113}
\saveTG{𪨱}{22113}
\saveTG{}{22113}
\saveTG{纴}{22114}
\saveTG{嶳}{22114}
\saveTG{𩾋}{22115}
\saveTG{𨤶}{22115}
\saveTG{缍}{22115}
\saveTG{歱}{22115}
\saveTG{𣮟}{22115}
\saveTG{𣥃}{22117}
\saveTG{𫋪}{22117}
\saveTG{𡾾}{22117}
\saveTG{𣥡}{22117}
\saveTG{𣬺}{22117}
\saveTG{䶰}{22117}
\saveTG{𫄜}{22117}
\saveTG{𠘽}{22117}
\saveTG{𡷲}{22117}
\saveTG{𪛂}{22117}
\saveTG{𥫋}{22118}
\saveTG{𠞚}{22120}
\saveTG{𣂠}{22121}
\saveTG{𣂸}{22121}
\saveTG{𣂾}{22121}
\saveTG{𡽻}{22121}
\saveTG{㟡}{22121}
\saveTG{㟵}{22121}
\saveTG{㞬}{22121}
\saveTG{𣃔}{22121}
\saveTG{𫚩}{22122}
\saveTG{𦐉}{22127}
\saveTG{𦑴}{22127}
\saveTG{𪉈}{22127}
\saveTG{𡼍}{22127}
\saveTG{𡾕}{22127}
\saveTG{𨦜}{22127}
\saveTG{𫄹}{22127}
\saveTG{𦐳}{22127}
\saveTG{𦏿}{22127}
\saveTG{𣦜}{22127}
\saveTG{𤯪}{22127}
\saveTG{巋}{22127}
\saveTG{鸶}{22127}
\saveTG{绣}{22127}
\saveTG{嵡}{22127}
\saveTG{}{22127}
\saveTG{𡼜}{22127}
\saveTG{𧔮}{22131}
\saveTG{𧔗}{22131}
\saveTG{𧖕}{22131}
\saveTG{𫄸}{22131}
\saveTG{𤬤}{22131}
\saveTG{𧔯}{22131}
\saveTG{𧔻}{22131}
\saveTG{}{22131}
\saveTG{𡾁}{22132}
\saveTG{𡻔}{22132}
\saveTG{𪨺}{22132}
\saveTG{衇}{22132}
\saveTG{𧌿}{22136}
\saveTG{蠻}{22136}
\saveTG{䖽}{22136}
\saveTG{螚}{22136}
\saveTG{䘅}{22136}
\saveTG{𧏇}{22136}
\saveTG{䖪}{22136}
\saveTG{𧖀}{22136}
\saveTG{𧑿}{22136}
\saveTG{蚩}{22136}
\saveTG{蠫}{22136}
\saveTG{𡿽}{22137}
\saveTG{𦈠}{22137}
\saveTG{纸}{22140}
\saveTG{纤}{22140}
\saveTG{𫄧}{22141}
\saveTG{𦈈}{22141}
\saveTG{𥃑}{22142}
\saveTG{𨰜}{22143}
\saveTG{𡽵}{22143}
\saveTG{𡺃}{22143}
\saveTG{𡣢}{22144}
\saveTG{𡽂}{22144}
\saveTG{𥡭}{22144}
\saveTG{绥}{22144}
\saveTG{𪾝}{22147}
\saveTG{𨥌}{22147}
\saveTG{𡹉}{22147}
\saveTG{𪔥}{22147}
\saveTG{𪔫}{22147}
\saveTG{崶}{22147}
\saveTG{绶}{22147}
\saveTG{缓}{22147}
\saveTG{𪩕}{22147}
\saveTG{𡻶}{22148}
\saveTG{𡸄}{22148}
\saveTG{𡾺}{22151}
\saveTG{𧗇}{22153}
\saveTG{𧗒}{22153}
\saveTG{𪩞}{22153}
\saveTG{𢨂}{22153}
\saveTG{𡸼}{22156}
\saveTG{𧰙}{22158}
\saveTG{鲘}{22161}
\saveTG{𡽞}{22161}
\saveTG{}{22161}
\saveTG{𡻂}{22162}
\saveTG{𣥾}{22162}
\saveTG{鲻}{22163}
\saveTG{缁}{22163}
\saveTG{𪝻}{22164}
\saveTG{𡽘}{22164}
\saveTG{𦈏}{22164}
\saveTG{濌}{22169}
\saveTG{绌}{22172}
\saveTG{鳐}{22172}
\saveTG{𡽎}{22174}
\saveTG{𨫞}{22181}
\saveTG{𡷑}{22182}
\saveTG{嶔}{22182}
\saveTG{崁}{22182}
\saveTG{𡾣}{22186}
\saveTG{鲧}{22193}
\saveTG{𥠟}{22194}
\saveTG{𧖲}{22194}
\saveTG{缫}{22194}
\saveTG{稣}{22194}
\saveTG{𥠭}{22194}
\saveTG{𨒠}{22194}
\saveTG{䌽}{22194}
\saveTG{𨭔}{22194}
\saveTG{𧆮}{22200}
\saveTG{𠛱}{22200}
\saveTG{𧆢}{22200}
\saveTG{𠝉}{22200}
\saveTG{𠋴}{22200}
\saveTG{𠝓}{22200}
\saveTG{𠉳}{22200}
\saveTG{𠎼}{22200}
\saveTG{𠊏}{22200}
\saveTG{𠐘}{22200}
\saveTG{𠇁}{22200}
\saveTG{𠟠}{22200}
\saveTG{侧}{22200}
\saveTG{側}{22200}
\saveTG{剼}{22200}
\saveTG{𠆯}{22200}
\saveTG{𠞤}{22200}
\saveTG{𠠌}{22200}
\saveTG{𢓟}{22200}
\saveTG{𠝨}{22200}
\saveTG{𠞾}{22200}
\saveTG{𠟲}{22200}
\saveTG{𢔯}{22200}
\saveTG{𠜎}{22200}
\saveTG{𠛻}{22200}
\saveTG{𠞜}{22200}
\saveTG{𪝵}{22200}
\saveTG{𠛷}{22200}
\saveTG{𠋙}{22200}
\saveTG{劌}{22200}
\saveTG{𠜊}{22200}
\saveTG{制}{22200}
\saveTG{劓}{22200}
\saveTG{鼼}{22200}
\saveTG{侀}{22200}
\saveTG{刎}{22200}
\saveTG{觓}{22200}
\saveTG{剻}{22200}
\saveTG{爿}{22200}
\saveTG{俐}{22200}
\saveTG{例}{22200}
\saveTG{劇}{22200}
\saveTG{𠟧}{22200}
\saveTG{刿}{22200}
\saveTG{倒}{22200}
\saveTG{剶}{22200}
\saveTG{剬}{22200}
\saveTG{𠛫}{22200}
\saveTG{㓩}{22200}
\saveTG{𠛠}{22200}
\saveTG{㔒}{22200}
\saveTG{𠝦}{22200}
\saveTG{𠛕}{22200}
\saveTG{𠜿}{22200}
\saveTG{𠟭}{22200}
\saveTG{𠛯}{22200}
\saveTG{𠜔}{22200}
\saveTG{㓺}{22200}
\saveTG{䖌}{22200}
\saveTG{𠊖}{22200}
\saveTG{𡺣}{22201}
\saveTG{𠤟}{22201}
\saveTG{𠂳}{22201}
\saveTG{𠤞}{22201}
\saveTG{𢫶}{22201}
\saveTG{𡴶}{22201}
\saveTG{𠨲}{22201}
\saveTG{𠩃}{22201}
\saveTG{屵}{22201}
\saveTG{㟥}{22202}
\saveTG{嵾}{22202}
\saveTG{𡻪}{22202}
\saveTG{𠯃}{22206}
\saveTG{𡼘}{22207}
\saveTG{岑}{22207}
\saveTG{岁}{22207}
\saveTG{㱔}{22207}
\saveTG{𣧄}{22207}
\saveTG{𡷉}{22207}
\saveTG{𡿪}{22207}
\saveTG{𡵯}{22209}
\saveTG{仳}{22210}
\saveTG{𧲠}{22210}
\saveTG{𧳿}{22210}
\saveTG{佌}{22210}
\saveTG{亂}{22210}
\saveTG{豼}{22210}
\saveTG{𡶂}{22211}
\saveTG{𢓊}{22211}
\saveTG{𡿐}{22211}
\saveTG{𠉃}{22211}
\saveTG{𩴂}{22211}
\saveTG{𩲊}{22211}
\saveTG{𩴝}{22211}
\saveTG{巃}{22211}
\saveTG{岝}{22211}
\saveTG{𦯤}{22211}
\saveTG{𧲰}{22212}
\saveTG{𧣜}{22212}
\saveTG{𪖛}{22212}
\saveTG{𧱻}{22212}
\saveTG{𩴭}{22212}
\saveTG{崺}{22212}
\saveTG{𦯦}{22212}
\saveTG{𦲬}{22212}
\saveTG{嶤}{22212}
\saveTG{𩴡}{22212}
\saveTG{兇}{22212}
\saveTG{𩴞}{22212}
\saveTG{𢔗}{22212}
\saveTG{𠎷}{22212}
\saveTG{𧳌}{22212}
\saveTG{𠏪}{22212}
\saveTG{毙}{22212}
\saveTG{彪}{22212}
\saveTG{儠}{22212}
\saveTG{能}{22212}
\saveTG{峞}{22212}
\saveTG{𠊦}{22212}
\saveTG{𠒕}{22213}
\saveTG{𩲒}{22213}
\saveTG{𠋉}{22213}
\saveTG{佻}{22213}
\saveTG{𩳕}{22214}
\saveTG{𧇯}{22214}
\saveTG{𩳎}{22214}
\saveTG{𡸌}{22214}
\saveTG{𠍮}{22214}
\saveTG{仛}{22214}
\saveTG{任}{22214}
\saveTG{毵}{22214}
\saveTG{毿}{22214}
\saveTG{崖}{22214}
\saveTG{䏶}{22214}
\saveTG{𦨁}{22214}
\saveTG{𡷜}{22214}
\saveTG{𩲔}{22214}
\saveTG{𢔎}{22214}
\saveTG{𤔕}{22214}
\saveTG{𢔷}{22214}
\saveTG{𢔺}{22214}
\saveTG{𢓩}{22214}
\saveTG{𠈺}{22214}
\saveTG{𠉰}{22214}
\saveTG{𩳡}{22214}
\saveTG{𠌠}{22214}
\saveTG{𡿎}{22215}
\saveTG{𢕘}{22215}
\saveTG{嶐}{22215}
\saveTG{𢔝}{22215}
\saveTG{𧰜}{22215}
\saveTG{𧳮}{22215}
\saveTG{㣫}{22215}
\saveTG{𨤨}{22215}
\saveTG{𡺉}{22215}
\saveTG{𤝵}{22215}
\saveTG{𢓸}{22215}
\saveTG{𠍧}{22215}
\saveTG{偅}{22215}
\saveTG{倕}{22215}
\saveTG{催}{22215}
\saveTG{崔}{22215}
\saveTG{嵟}{22215}
\saveTG{𡾪}{22216}
\saveTG{𩲆}{22216}
\saveTG{𢅇}{22217}
\saveTG{𪩑}{22217}
\saveTG{𡷱}{22217}
\saveTG{𣯯}{22217}
\saveTG{𪨿}{22217}
\saveTG{𣭫}{22217}
\saveTG{𣭡}{22217}
\saveTG{𠇔}{22217}
\saveTG{𢇖}{22217}
\saveTG{𡺮}{22217}
\saveTG{𡵂}{22217}
\saveTG{𡵷}{22217}
\saveTG{𡒑}{22217}
\saveTG{㞩}{22217}
\saveTG{𡶟}{22217}
\saveTG{𢀐}{22217}
\saveTG{𠃲}{22217}
\saveTG{𩴋}{22217}
\saveTG{𢀇}{22217}
\saveTG{𠈊}{22217}
\saveTG{𠘩}{22217}
\saveTG{𢆻}{22217}
\saveTG{𠙳}{22217}
\saveTG{𡸔}{22217}
\saveTG{𩙟}{22217}
\saveTG{𪈽}{22217}
\saveTG{岚}{22217}
\saveTG{嵐}{22217}
\saveTG{嶏}{22217}
\saveTG{凭}{22217}
\saveTG{傂}{22217}
\saveTG{𣦷}{22217}
\saveTG{郺}{22217}
\saveTG{𠒒}{22217}
\saveTG{𡵵}{22217}
\saveTG{𧢳}{22217}
\saveTG{𩲀}{22217}
\saveTG{𪖐}{22217}
\saveTG{𪖑}{22217}
\saveTG{𢒽}{22217}
\saveTG{㣧}{22217}
\saveTG{𠏢}{22217}
\saveTG{𠉗}{22217}
\saveTG{𠉥}{22217}
\saveTG{𠋃}{22217}
\saveTG{𦙌}{22217}
\saveTG{𦡳}{22217}
\saveTG{㠕}{22217}
\saveTG{𡽧}{22217}
\saveTG{𪩏}{22217}
\saveTG{𡵧}{22217}
\saveTG{𡵉}{22217}
\saveTG{𡷹}{22217}
\saveTG{𡶢}{22217}
\saveTG{𡸢}{22217}
\saveTG{𪩠}{22217}
\saveTG{𤕷}{22217}
\saveTG{𧥇}{22217}
\saveTG{𡖅}{22217}
\saveTG{𫆱}{22217}
\saveTG{𪖾}{22217}
\saveTG{𢓗}{22217}
\saveTG{𢕀}{22217}
\saveTG{𠙭}{22217}
\saveTG{𢀈}{22217}
\saveTG{㑷}{22217}
\saveTG{𪝿}{22217}
\saveTG{齆}{22217}
\saveTG{𠍕}{22217}
\saveTG{𠎺}{22217}
\saveTG{𠈁}{22217}
\saveTG{𩀗}{22217}
\saveTG{𠒴}{22217}
\saveTG{𩴆}{22217}
\saveTG{𩴪}{22217}
\saveTG{𡿳}{22217}
\saveTG{𢀂}{22217}
\saveTG{𡸏}{22218}
\saveTG{𤣖}{22218}
\saveTG{僜}{22218}
\saveTG{𩲓}{22218}
\saveTG{㒥}{22218}
\saveTG{𩲷}{22219}
\saveTG{𦠔}{22221}
\saveTG{㒋}{22221}
\saveTG{𧣊}{22221}
\saveTG{𦠟}{22221}
\saveTG{𦬟}{22221}
\saveTG{𣂵}{22221}
\saveTG{㑜}{22221}
\saveTG{𠌲}{22221}
\saveTG{𪨳}{22221}
\saveTG{𡾧}{22221}
\saveTG{𡵱}{22221}
\saveTG{𡻼}{22221}
\saveTG{伒}{22221}
\saveTG{斨}{22221}
\saveTG{嵛}{22221}
\saveTG{𧈎}{22222}
\saveTG{𢒱}{22222}
\saveTG{𨤅}{22222}
\saveTG{𧈇}{22222}
\saveTG{𡽦}{22222}
\saveTG{虨}{22222}
\saveTG{𢒪}{22222}
\saveTG{㣐}{22222}
\saveTG{𠎎}{22222}
\saveTG{𧣭}{22222}
\saveTG{𠉠}{22222}
\saveTG{𡽐}{22222}
\saveTG{𢒜}{22222}
\saveTG{𠆺}{22223}
\saveTG{𢀅}{22223}
\saveTG{𡸍}{22224}
\saveTG{𨶼}{22226}
\saveTG{𠕓}{22227}
\saveTG{𢋱}{22227}
\saveTG{𢄆}{22227}
\saveTG{𦮅}{22227}
\saveTG{𡴊}{22227}
\saveTG{㠿}{22227}
\saveTG{𪱫}{22227}
\saveTG{𡻺}{22227}
\saveTG{𩨮}{22227}
\saveTG{𡸺}{22227}
\saveTG{𡸹}{22227}
\saveTG{𡻎}{22227}
\saveTG{嶲}{22227}
\saveTG{𦙄}{22227}
\saveTG{𦓛}{22227}
\saveTG{𧥅}{22227}
\saveTG{𧤗}{22227}
\saveTG{觿}{22227}
\saveTG{觽}{22227}
\saveTG{𢀉}{22227}
\saveTG{㒞}{22227}
\saveTG{𠈩}{22227}
\saveTG{𦚮}{22227}
\saveTG{𠋐}{22227}
\saveTG{𠤚}{22227}
\saveTG{𩪾}{22227}
\saveTG{𦴘}{22227}
\saveTG{𡿲}{22227}
\saveTG{㡩}{22227}
\saveTG{𡻾}{22227}
\saveTG{𠻭}{22227}
\saveTG{𡴅}{22227}
\saveTG{𪔂}{22227}
\saveTG{㞧}{22227}
\saveTG{𡶤}{22227}
\saveTG{𡺥}{22227}
\saveTG{𡸑}{22227}
\saveTG{𡸿}{22227}
\saveTG{𢃷}{22227}
\saveTG{𡶝}{22227}
\saveTG{𦞄}{22227}
\saveTG{𡶀}{22227}
\saveTG{𡵫}{22227}
\saveTG{𤕰}{22227}
\saveTG{𤖎}{22227}
\saveTG{𥝄}{22227}
\saveTG{𪖯}{22227}
\saveTG{𢀏}{22227}
\saveTG{𠊵}{22227}
\saveTG{㑕}{22227}
\saveTG{𦇵}{22227}
\saveTG{㞣}{22227}
\saveTG{𡴚}{22227}
\saveTG{𡴒}{22227}
\saveTG{𡸐}{22227}
\saveTG{㠐}{22227}
\saveTG{𡵳}{22227}
\saveTG{𢕪}{22227}
\saveTG{𠍓}{22227}
\saveTG{侺}{22227}
\saveTG{𡾱}{22227}
\saveTG{𢓵}{22227}
\saveTG{𠉑}{22227}
\saveTG{𦣏}{22227}
\saveTG{𠠪}{22227}
\saveTG{崗}{22227}
\saveTG{偝}{22227}
\saveTG{崩}{22227}
\saveTG{傰}{22227}
\saveTG{巛}{22227}
\saveTG{胔}{22227}
\saveTG{屶}{22227}
\saveTG{鼎}{22227}
\saveTG{峝}{22227}
\saveTG{偳}{22227}
\saveTG{耑}{22227}
\saveTG{僞}{22227}
\saveTG{岗}{22227}
\saveTG{巂}{22227}
\saveTG{雟}{22227}
\saveTG{儶}{22227}
\saveTG{僑}{22227}
\saveTG{峛}{22227}
\saveTG{臠}{22227}
\saveTG{崙}{22227}
\saveTG{嵩}{22227}
\saveTG{貒}{22227}
\saveTG{峟}{22227}
\saveTG{偊}{22227}
\saveTG{歶}{22227}
\saveTG{觜}{22227}
\saveTG{𡴥}{22227}
\saveTG{𣇓}{22227}
\saveTG{𡹺}{22227}
\saveTG{𡵅}{22227}
\saveTG{𡺩}{22227}
\saveTG{𡼁}{22227}
\saveTG{𦜗}{22227}
\saveTG{𧤊}{22227}
\saveTG{𦜾}{22227}
\saveTG{𢕷}{22227}
\saveTG{𠝯}{22227}
\saveTG{𠜣}{22227}
\saveTG{𥞿}{22227}
\saveTG{𥜿}{22227}
\saveTG{𥝁}{22227}
\saveTG{𠉱}{22227}
\saveTG{𧂥}{22227}
\saveTG{侨}{22228}
\saveTG{岕}{22228}
\saveTG{𫌯}{22228}
\saveTG{𤕯}{22230}
\saveTG{𧲲}{22230}
\saveTG{䝖}{22230}
\saveTG{俬}{22230}
\saveTG{觚}{22230}
\saveTG{仏}{22230}
\saveTG{㲻}{22230}
\saveTG{𠊥}{22230}
\saveTG{𪨮}{22231}
\saveTG{𠇓}{22231}
\saveTG{𣥬}{22231}
\saveTG{𪜴}{22231}
\saveTG{𠋷}{22231}
\saveTG{𪨢}{22231}
\saveTG{𣂭}{22231}
\saveTG{𡺙}{22232}
\saveTG{𣲂}{22232}
\saveTG{𥣤}{22232}
\saveTG{𡷷}{22232}
\saveTG{仫}{22232}
\saveTG{𡺒}{22232}
\saveTG{𧰨}{22232}
\saveTG{𠇊}{22232}
\saveTG{𨑆}{22232}
\saveTG{𧤷}{22232}
\saveTG{𡷰}{22232}
\saveTG{𤬘}{22233}
\saveTG{𠇗}{22233}
\saveTG{𤬝}{22233}
\saveTG{𤬛}{22233}
\saveTG{𤬌}{22233}
\saveTG{𤣆}{22233}
\saveTG{𤬈}{22233}
\saveTG{𤫻}{22233}
\saveTG{𪼴}{22234}
\saveTG{𢕆}{22234}
\saveTG{𠌤}{22234}
\saveTG{𡾯}{22234}
\saveTG{伥}{22234}
\saveTG{𠎻}{22236}
\saveTG{𧳻}{22236}
\saveTG{嵹}{22236}
\saveTG{㒚}{22237}
\saveTG{𠍥}{22237}
\saveTG{𠇖}{22237}
\saveTG{㣰}{22238}
\saveTG{僁}{22239}
\saveTG{𥟦}{22239}
\saveTG{𠉟}{22240}
\saveTG{觝}{22240}
\saveTG{低}{22240}
\saveTG{仟}{22240}
\saveTG{觗}{22240}
\saveTG{彽}{22240}
\saveTG{倂}{22241}
\saveTG{㠔}{22241}
\saveTG{岸}{22241}
\saveTG{侹}{22241}
\saveTG{𠈰}{22241}
\saveTG{𠊀}{22241}
\saveTG{𪜲}{22241}
\saveTG{𡵥}{22241}
\saveTG{𡺰}{22241}
\saveTG{㿫}{22242}
\saveTG{𡺵}{22242}
\saveTG{嶈}{22242}
\saveTG{𪪋}{22243}
\saveTG{𧇛}{22243}
\saveTG{𡖲}{22244}
\saveTG{倭}{22244}
\saveTG{俀}{22244}
\saveTG{躷}{22244}
\saveTG{𧳭}{22244}
\saveTG{𧤋}{22244}
\saveTG{𣨙}{22244}
\saveTG{𥟿}{22244}
\saveTG{𢓰}{22244}
\saveTG{㣦}{22244}
\saveTG{巐}{22246}
\saveTG{𣥨}{22247}
\saveTG{𧰍}{22247}
\saveTG{䚨}{22247}
\saveTG{𧤔}{22247}
\saveTG{𢔏}{22247}
\saveTG{𠎆}{22247}
\saveTG{𤿙}{22247}
\saveTG{㑧}{22247}
\saveTG{𠐄}{22247}
\saveTG{𣬆}{22247}
\saveTG{𥀍}{22247}
\saveTG{㣪}{22247}
\saveTG{𠋠}{22247}
\saveTG{㠅}{22247}
\saveTG{𠈶}{22247}
\saveTG{𠈦}{22247}
\saveTG{𡺲}{22247}
\saveTG{𡺿}{22247}
\saveTG{㣭}{22247}
\saveTG{𠊂}{22247}
\saveTG{僾}{22247}
\saveTG{偁}{22247}
\saveTG{仮}{22247}
\saveTG{俘}{22247}
\saveTG{後}{22247}
\saveTG{儍}{22247}
\saveTG{嶎}{22247}
\saveTG{𪝈}{22247}
\saveTG{岌}{22247}
\saveTG{㟼}{22248}
\saveTG{𡽄}{22248}
\saveTG{𡽪}{22248}
\saveTG{𡸅}{22248}
\saveTG{𪝇}{22248}
\saveTG{嶶}{22248}
\saveTG{巌}{22248}
\saveTG{巖}{22248}
\saveTG{峳}{22248}
\saveTG{𠇼}{22249}
\saveTG{𢪇}{22250}
\saveTG{㐿}{22250}
\saveTG{𡾻}{22251}
\saveTG{嵗}{22252}
\saveTG{𡶷}{22252}
\saveTG{𡽖}{22252}
\saveTG{僢}{22252}
\saveTG{巀}{22253}
\saveTG{𡻕}{22253}
\saveTG{崴}{22253}
\saveTG{嵅}{22253}
\saveTG{㡬}{22253}
\saveTG{𡶔}{22253}
\saveTG{嶻}{22253}
\saveTG{𡾠}{22253}
\saveTG{𡸤}{22253}
\saveTG{𥢲}{22253}
\saveTG{僟}{22253}
\saveTG{嵗}{22253}
\saveTG{𡷫}{22253}
\saveTG{𡹷}{22254}
\saveTG{𨍃}{22256}
\saveTG{𣦬}{22257}
\saveTG{𡼵}{22257}
\saveTG{嵂}{22257}
\saveTG{𡻷}{22259}
\saveTG{𢖌}{22259}
\saveTG{𪩈}{22259}
\saveTG{𥅱}{22260}
\saveTG{𤖣}{22260}
\saveTG{𧲿}{22261}
\saveTG{𡹼}{22261}
\saveTG{𡻓}{22261}
\saveTG{𧳧}{22262}
\saveTG{𡗎}{22262}
\saveTG{𧳦}{22262}
\saveTG{𢔡}{22262}
\saveTG{偕}{22262}
\saveTG{𡾷}{22262}
\saveTG{𠉤}{22263}
\saveTG{𪺟}{22263}
\saveTG{佸}{22264}
\saveTG{𠈲}{22264}
\saveTG{偱}{22264}
\saveTG{𦧬}{22264}
\saveTG{𣨯}{22264}
\saveTG{𠉣}{22264}
\saveTG{循}{22264}
\saveTG{𦧠}{22264}
\saveTG{𧱸}{22267}
\saveTG{𪝾}{22268}
\saveTG{僠}{22269}
\saveTG{𢇍}{22270}
\saveTG{仙}{22270}
\saveTG{傜}{22272}
\saveTG{崫}{22272}
\saveTG{徭}{22272}
\saveTG{𧳀}{22272}
\saveTG{𣦧}{22272}
\saveTG{𣻮}{22272}
\saveTG{𦛁}{22272}
\saveTG{㑁}{22272}
\saveTG{貀}{22272}
\saveTG{㑎}{22272}
\saveTG{𡺇}{22277}
\saveTG{偛}{22277}
\saveTG{𠋯}{22277}
\saveTG{岿}{22277}
\saveTG{𥀺}{22277}
\saveTG{𡰮}{22277}
\saveTG{㑟}{22281}
\saveTG{嵸}{22281}
\saveTG{𦹼}{22282}
\saveTG{嶡}{22282}
\saveTG{𡸛}{22282}
\saveTG{𧲶}{22284}
\saveTG{徯}{22284}
\saveTG{傒}{22284}
\saveTG{仸}{22284}
\saveTG{𠋳}{22284}
\saveTG{𡻬}{22284}
\saveTG{𢓽}{22284}
\saveTG{𠊾}{22284}
\saveTG{𣧕}{22284}
\saveTG{嶽}{22284}
\saveTG{巚}{22284}
\saveTG{貕}{22284}
\saveTG{𢖃}{22285}
\saveTG{𧴌}{22285}
\saveTG{僕}{22285}
\saveTG{𧴇}{22285}
\saveTG{𡽠}{22286}
\saveTG{𡿅}{22286}
\saveTG{𡾫}{22286}
\saveTG{嵿}{22286}
\saveTG{儨}{22286}
\saveTG{𧶑}{22286}
\saveTG{𤖓}{22286}
\saveTG{𡷳}{22287}
\saveTG{𠋌}{22289}
\saveTG{㟜}{22292}
\saveTG{緜}{22293}
\saveTG{㒡}{22293}
\saveTG{𢖟}{22293}
\saveTG{係}{22293}
\saveTG{𠐴}{22293}
\saveTG{倸}{22294}
\saveTG{偨}{22294}
\saveTG{䵈}{22294}
\saveTG{𠍂}{22294}
\saveTG{𡿛}{22294}
\saveTG{𠌀}{22294}
\saveTG{𤾾}{22294}
\saveTG{䑈}{22294}
\saveTG{𪜰}{22294}
\saveTG{𩡄}{22294}
\saveTG{𥡻}{22294}
\saveTG{𡿌}{22294}
\saveTG{觻}{22294}
\saveTG{𠊻}{22294}
\saveTG{䐆}{22294}
\saveTG{𤕽}{22294}
\saveTG{𧈈}{22295}
\saveTG{㑿}{22295}
\saveTG{㒒}{22295}
\saveTG{㟶}{22296}
\saveTG{𤖉}{22297}
\saveTG{𡽋}{22299}
\saveTG{𡻚}{22299}
\saveTG{魝}{22300}
\saveTG{鯯}{22300}
\saveTG{鯏}{22300}
\saveTG{劁}{22300}
\saveTG{䱨}{22300}
\saveTG{𩶽}{22300}
\saveTG{𠛏}{22300}
\saveTG{𩵙}{22300}
\saveTG{𠜖}{22300}
\saveTG{𫙒}{22300}
\saveTG{𠞎}{22300}
\saveTG{𩷤}{22300}
\saveTG{𠞹}{22300}
\saveTG{鰂}{22300}
\saveTG{𠞆}{22300}
\saveTG{𠞸}{22300}
\saveTG{𩾧}{22300}
\saveTG{鯻}{22300}
\saveTG{𨘳}{22302}
\saveTG{㞮}{22302}
\saveTG{𡴔}{22302}
\saveTG{𡹢}{22302}
\saveTG{岺}{22302}
\saveTG{𡶞}{22303}
\saveTG{𨓁}{22303}
\saveTG{𡽅}{22305}
\saveTG{㞫}{22307}
\saveTG{𠄂}{22310}
\saveTG{𠃿}{22310}
\saveTG{魮}{22310}
\saveTG{𫚅}{22312}
\saveTG{䮭}{22312}
\saveTG{鱲}{22312}
\saveTG{鮡}{22313}
\saveTG{𩹉}{22313}
\saveTG{魠}{22314}
\saveTG{𫙡}{22314}
\saveTG{魹}{22314}
\saveTG{䱰}{22315}
\saveTG{𩸫}{22315}
\saveTG{𩺖}{22315}
\saveTG{㲕}{22315}
\saveTG{𣯺}{22315}
\saveTG{𫚭}{22316}
\saveTG{𩹣}{22317}
\saveTG{𩸼}{22317}
\saveTG{𪁦}{22317}
\saveTG{𩿘}{22317}
\saveTG{𩾐}{22317}
\saveTG{𩷋}{22317}
\saveTG{䰲}{22317}
\saveTG{𩸐}{22317}
\saveTG{𩷴}{22317}
\saveTG{𪇋}{22317}
\saveTG{𩷙}{22317}
\saveTG{鷈}{22317}
\saveTG{𩷇}{22317}
\saveTG{𩷁}{22317}
\saveTG{㲬}{22317}
\saveTG{𩶆}{22317}
\saveTG{𩺛}{22317}
\saveTG{䱺}{22318}
\saveTG{𫙼}{22318}
\saveTG{䲉}{22321}
\saveTG{𪃴}{22321}
\saveTG{䰺}{22321}
\saveTG{䱿}{22321}
\saveTG{𫚀}{22321}
\saveTG{𩻬}{22322}
\saveTG{䱑}{22323}
\saveTG{𪁐}{22327}
\saveTG{𩼿}{22327}
\saveTG{𪀻}{22327}
\saveTG{𩾣}{22327}
\saveTG{𫘖}{22327}
\saveTG{𩷒}{22327}
\saveTG{鴜}{22327}
\saveTG{嶌}{22327}
\saveTG{鸞}{22327}
\saveTG{𫙾}{22327}
\saveTG{魸}{22327}
\saveTG{鷥}{22327}
\saveTG{𩸗}{22327}
\saveTG{𩻟}{22327}
\saveTG{鱎}{22327}
\saveTG{𩷜}{22327}
\saveTG{岃}{22327}
\saveTG{𤉐}{22327}
\saveTG{𪈡}{22327}
\saveTG{𩽨}{22327}
\saveTG{𩽌}{22327}
\saveTG{鵀}{22327}
\saveTG{𤉥}{22330}
\saveTG{𢛁}{22330}
\saveTG{𤉩}{22330}
\saveTG{悡}{22330}
\saveTG{𤉉}{22330}
\saveTG{𡿦}{22330}
\saveTG{𢡹}{22331}
\saveTG{𤋤}{22331}
\saveTG{𤋞}{22331}
\saveTG{嶣}{22331}
\saveTG{恁}{22331}
\saveTG{態}{22331}
\saveTG{熊}{22331}
\saveTG{䵩}{22331}
\saveTG{𡿧}{22331}
\saveTG{𪁯}{22331}
\saveTG{𡽮}{22331}
\saveTG{𢗍}{22331}
\saveTG{㶵}{22331}
\saveTG{𢙌}{22331}
\saveTG{𡴦}{22331}
\saveTG{𡼡}{22331}
\saveTG{𩺉}{22331}
\saveTG{𢛋}{22332}
\saveTG{𫙧}{22332}
\saveTG{𩻔}{22332}
\saveTG{𩻴}{22332}
\saveTG{煭}{22332}
\saveTG{𢇁}{22333}
\saveTG{𢥁}{22333}
\saveTG{𩸰}{22333}
\saveTG{𫛈}{22333}
\saveTG{䱄}{22333}
\saveTG{𡽸}{22334}
\saveTG{毖}{22334}
\saveTG{䱥}{22336}
\saveTG{䱘}{22336}
\saveTG{𦻇}{22336}
\saveTG{𢠞}{22336}
\saveTG{𢤌}{22336}
\saveTG{𢠾}{22336}
\saveTG{𩹒}{22336}
\saveTG{𩷀}{22336}
\saveTG{𩡢}{22336}
\saveTG{鮆}{22336}
\saveTG{崽}{22336}
\saveTG{𡾢}{22336}
\saveTG{𪀐}{22337}
\saveTG{𢘿}{22337}
\saveTG{𩶟}{22337}
\saveTG{𠨀}{22337}
\saveTG{𡿭}{22337}
\saveTG{𡹸}{22337}
\saveTG{𩹘}{22337}
\saveTG{巜}{22337}
\saveTG{𢚏}{22338}
\saveTG{𡼶}{22338}
\saveTG{𢥩}{22339}
\saveTG{𢢹}{22339}
\saveTG{戀}{22339}
\saveTG{𩵞}{22340}
\saveTG{𩶅}{22341}
\saveTG{𪀍}{22341}
\saveTG{鯅}{22341}
\saveTG{鮾}{22344}
\saveTG{鯘}{22344}
\saveTG{魬}{22347}
\saveTG{鱫}{22347}
\saveTG{䱐}{22347}
\saveTG{䲬}{22347}
\saveTG{𡬸}{22347}
\saveTG{𩸿}{22347}
\saveTG{𩸣}{22347}
\saveTG{𡦄}{22347}
\saveTG{鱍}{22347}
\saveTG{𩷘}{22347}
\saveTG{𩼕}{22347}
\saveTG{鰀}{22347}
\saveTG{鯚}{22347}
\saveTG{鯼}{22347}
\saveTG{𩶈}{22349}
\saveTG{𧎶}{22353}
\saveTG{𫙰}{22356}
\saveTG{𡿠}{22357}
\saveTG{𪅹}{22358}
\saveTG{鮜}{22361}
\saveTG{鮨}{22361}
\saveTG{𪁆}{22361}
\saveTG{𩸶}{22362}
\saveTG{𩺢}{22362}
\saveTG{鯔}{22363}
\saveTG{𩷽}{22363}
\saveTG{鱕}{22369}
\saveTG{鰩}{22372}
\saveTG{𩹴}{22377}
\saveTG{䳶}{22384}
\saveTG{𩸭}{22384}
\saveTG{𩹍}{22384}
\saveTG{𩽄}{22386}
\saveTG{嶺}{22386}
\saveTG{𩻃}{22391}
\saveTG{𩶚}{22392}
\saveTG{𤬖}{22393}
\saveTG{鯀}{22393}
\saveTG{𩷌}{22394}
\saveTG{鱳}{22394}
\saveTG{𩻌}{22394}
\saveTG{穌}{22394}
\saveTG{𩼋}{22395}
\saveTG{䲃}{22395}
\saveTG{𠝣}{22400}
\saveTG{𠞃}{22400}
\saveTG{𠠤}{22400}
\saveTG{𠠥}{22400}
\saveTG{𠜱}{22400}
\saveTG{𩇿}{22400}
\saveTG{㞵}{22400}
\saveTG{𡢛}{22400}
\saveTG{𡷁}{22400}
\saveTG{屮}{22400}
\saveTG{劖}{22400}
\saveTG{刋}{22400}
\saveTG{𨈧}{22400}
\saveTG{𦨉}{22400}
\saveTG{𨉃}{22401}
\saveTG{𨊟}{22401}
\saveTG{𡴟}{22401}
\saveTG{㞿}{22401}
\saveTG{𡻛}{22401}
\saveTG{𡺾}{22401}
\saveTG{𡵃}{22401}
\saveTG{𡴖}{22401}
\saveTG{𡽕}{22401}
\saveTG{毕}{22401}
\saveTG{峷}{22401}
\saveTG{斚}{22402}
\saveTG{𣥥}{22402}
\saveTG{𡚕}{22404}
\saveTG{𡢮}{22404}
\saveTG{崣}{22404}
\saveTG{孌}{22404}
\saveTG{姕}{22404}
\saveTG{妛}{22404}
\saveTG{𪦥}{22404}
\saveTG{𡺅}{22405}
\saveTG{𡿼}{22406}
\saveTG{𦸶}{22406}
\saveTG{𡴐}{22406}
\saveTG{𡴙}{22406}
\saveTG{𡥎}{22407}
\saveTG{𡦣}{22407}
\saveTG{𡹨}{22407}
\saveTG{𡦙}{22407}
\saveTG{𡾤}{22407}
\saveTG{𡶻}{22407}
\saveTG{𡥐}{22407}
\saveTG{𡥏}{22407}
\saveTG{𡥬}{22407}
\saveTG{㜽}{22407}
\saveTG{𡥝}{22407}
\saveTG{𡼇}{22407}
\saveTG{𡶣}{22407}
\saveTG{㚇}{22407}
\saveTG{孿}{22407}
\saveTG{嵏}{22407}
\saveTG{𡵢}{22407}
\saveTG{𠮓}{22407}
\saveTG{𡷮}{22407}
\saveTG{𠭏}{22407}
\saveTG{𡵇}{22407}
\saveTG{𡹃}{22407}
\saveTG{㟬}{22407}
\saveTG{𠭆}{22407}
\saveTG{𠬢}{22407}
\saveTG{𣀵}{22407}
\saveTG{𢆼}{22407}
\saveTG{𡸠}{22407}
\saveTG{𡴨}{22408}
\saveTG{變}{22408}
\saveTG{𡴝}{22408}
\saveTG{崒}{22408}
\saveTG{𡘶}{22408}
\saveTG{㪻}{22409}
\saveTG{卛}{22409}
\saveTG{𥬏}{22409}
\saveTG{𠃶}{22410}
\saveTG{舭}{22410}
\saveTG{乳}{22410}
\saveTG{𠄉}{22411}
\saveTG{𠛦}{22412}
\saveTG{𨈬}{22412}
\saveTG{𦪯}{22412}
\saveTG{𩙸}{22413}
\saveTG{巍}{22413}
\saveTG{𦕢}{22414}
\saveTG{𨉤}{22414}
\saveTG{𡴂}{22414}
\saveTG{𨉡}{22415}
\saveTG{𨉢}{22415}
\saveTG{㲗}{22415}
\saveTG{𡦢}{22415}
\saveTG{𦩰}{22415}
\saveTG{𦪴}{22416}
\saveTG{𦩴}{22416}
\saveTG{𦨎}{22417}
\saveTG{䠷}{22417}
\saveTG{𠃵}{22417}
\saveTG{𡼈}{22417}
\saveTG{𦨊}{22417}
\saveTG{𣱂}{22417}
\saveTG{𣭹}{22417}
\saveTG{𣰖}{22417}
\saveTG{𨈥}{22417}
\saveTG{𨊉}{22417}
\saveTG{𨽱}{22417}
\saveTG{𡷝}{22417}
\saveTG{𡵈}{22417}
\saveTG{𪩜}{22417}
\saveTG{𡸊}{22417}
\saveTG{𡽷}{22417}
\saveTG{𠮖}{22417}
\saveTG{䑬}{22417}
\saveTG{𡿢}{22417}
\saveTG{艠}{22418}
\saveTG{䠽}{22418}
\saveTG{𣂲}{22421}
\saveTG{𣂹}{22421}
\saveTG{𣂚}{22421}
\saveTG{𣂣}{22421}
\saveTG{㪿}{22421}
\saveTG{𣂟}{22421}
\saveTG{𢒒}{22422}
\saveTG{𦪟}{22422}
\saveTG{䑣}{22422}
\saveTG{𢒛}{22422}
\saveTG{𡼅}{22427}
\saveTG{𠂣}{22427}
\saveTG{𦖂}{22427}
\saveTG{𡻙}{22427}
\saveTG{𦨽}{22427}
\saveTG{𦪞}{22427}
\saveTG{屴}{22427}
\saveTG{𠢛}{22427}
\saveTG{𨉘}{22432}
\saveTG{㼏}{22433}
\saveTG{𠦅}{22440}
\saveTG{𣬊}{22441}
\saveTG{㟖}{22441}
\saveTG{𠙢}{22441}
\saveTG{𦨢}{22441}
\saveTG{幷}{22441}
\saveTG{艇}{22441}
\saveTG{𡷢}{22441}
\saveTG{𨉈}{22441}
\saveTG{𦱹}{22442}
\saveTG{𦷶}{22442}
\saveTG{𡝧}{22442}
\saveTG{𦽱}{22442}
\saveTG{𦶛}{22442}
\saveTG{𦱠}{22442}
\saveTG{𦿑}{22442}
\saveTG{𢇂}{22443}
\saveTG{𥏶}{22444}
\saveTG{𦱴}{22444}
\saveTG{𦬧}{22444}
\saveTG{𡼃}{22444}
\saveTG{㟺}{22445}
\saveTG{𢍬}{22446}
\saveTG{𡴬}{22447}
\saveTG{㝈}{22447}
\saveTG{𦪑}{22447}
\saveTG{𪩝}{22447}
\saveTG{𡚴}{22447}
\saveTG{𡸘}{22447}
\saveTG{𢌲}{22447}
\saveTG{茻}{22447}
\saveTG{艀}{22447}
\saveTG{艸}{22447}
\saveTG{舨}{22447}
\saveTG{𦩮}{22447}
\saveTG{艐}{22447}
\saveTG{𦱶}{22447}
\saveTG{𦫹}{22447}
\saveTG{芔}{22447}
\saveTG{𢆕}{22447}
\saveTG{𡾄}{22448}
\saveTG{𡽫}{22448}
\saveTG{𡷓}{22448}
\saveTG{𡾸}{22448}
\saveTG{𢍶}{22449}
\saveTG{𪩇}{22449}
\saveTG{嶯}{22453}
\saveTG{幾}{22453}
\saveTG{𦩿}{22457}
\saveTG{𦩽}{22462}
\saveTG{𡿹}{22463}
\saveTG{𨈸}{22464}
\saveTG{𡶥}{22467}
\saveTG{𦪖}{22469}
\saveTG{𨊄}{22469}
\saveTG{𩡊}{22469}
\saveTG{舢}{22470}
\saveTG{𦩾}{22472}
\saveTG{𩶌}{22472}
\saveTG{𦨥}{22472}
\saveTG{䠳}{22472}
\saveTG{𦩹}{22477}
\saveTG{𨽫}{22481}
\saveTG{𦩶}{22484}
\saveTG{𨉕}{22484}
\saveTG{𩔹}{22486}
\saveTG{𨉒}{22489}
\saveTG{𦆰}{22493}
\saveTG{𫆇}{22494}
\saveTG{𠜕}{22500}
\saveTG{㸪}{22500}
\saveTG{𢩬}{22500}
\saveTG{㸨}{22500}
\saveTG{𤛜}{22500}
\saveTG{𡴀}{22500}
\saveTG{𠛃}{22500}
\saveTG{犁}{22500}
\saveTG{𤘤}{22501}
\saveTG{𦍧}{22501}
\saveTG{㧘}{22502}
\saveTG{𢆡}{22502}
\saveTG{𤙲}{22502}
\saveTG{掣}{22502}
\saveTG{攣}{22502}
\saveTG{𦱧}{22503}
\saveTG{𡿶}{22503}
\saveTG{𡴇}{22504}
\saveTG{𪨶}{22504}
\saveTG{峯}{22504}
\saveTG{崋}{22504}
\saveTG{𨋡}{22506}
\saveTG{𨌗}{22506}
\saveTG{㟦}{22506}
\saveTG{𨋅}{22506}
\saveTG{𩏹}{22506}
\saveTG{𡼤}{22506}
\saveTG{𡼬}{22506}
\saveTG{輋}{22506}
\saveTG{𡼒}{22506}
\saveTG{𨏶}{22506}
\saveTG{𡵞}{22507}
\saveTG{𡵄}{22507}
\saveTG{𡸵}{22507}
\saveTG{𡸪}{22507}
\saveTG{𡼸}{22507}
\saveTG{𡷏}{22507}
\saveTG{𡴑}{22508}
\saveTG{𡼀}{22508}
\saveTG{牝}{22510}
\saveTG{𢧲}{22511}
\saveTG{犣}{22512}
\saveTG{鬽}{22512}
\saveTG{鬿}{22512}
\saveTG{𩙲}{22513}
\saveTG{嵬}{22513}
\saveTG{𤙞}{22514}
\saveTG{𡾛}{22514}
\saveTG{𡿆}{22514}
\saveTG{牦}{22514}
\saveTG{𤛍}{22515}
\saveTG{𤚏}{22515}
\saveTG{魕}{22515}
\saveTG{嶊}{22515}
\saveTG{𡸲}{22517}
\saveTG{𢶟}{22517}
\saveTG{𤙔}{22517}
\saveTG{𤛲}{22517}
\saveTG{𤜓}{22517}
\saveTG{𤘥}{22517}
\saveTG{𪺩}{22517}
\saveTG{𢩷}{22517}
\saveTG{𣭣}{22517}
\saveTG{𤙍}{22517}
\saveTG{𪩗}{22521}
\saveTG{㸫}{22521}
\saveTG{崭}{22521}
\saveTG{嶄}{22521}
\saveTG{𤙪}{22522}
\saveTG{𣭰}{22525}
\saveTG{𤛨}{22527}
\saveTG{𡷠}{22527}
\saveTG{𢲒}{22527}
\saveTG{岪}{22527}
\saveTG{巈}{22527}
\saveTG{犞}{22527}
\saveTG{𪺭}{22528}
\saveTG{𤚍}{22531}
\saveTG{𡸚}{22535}
\saveTG{牴}{22540}
\saveTG{𤙾}{22541}
\saveTG{𤚠}{22542}
\saveTG{𤛵}{22542}
\saveTG{㸹}{22543}
\saveTG{𤘸}{22547}
\saveTG{𤚈}{22547}
\saveTG{𤙤}{22547}
\saveTG{䉟}{22547}
\saveTG{𢰿}{22550}
\saveTG{𢪒}{22550}
\saveTG{𢬛}{22550}
\saveTG{𤘶}{22550}
\saveTG{掰}{22550}
\saveTG{搿}{22550}
\saveTG{㠖}{22553}
\saveTG{𡽥}{22553}
\saveTG{𡻍}{22553}
\saveTG{𡽒}{22553}
\saveTG{𪭗}{22553}
\saveTG{峩}{22553}
\saveTG{𡷌}{22555}
\saveTG{㸸}{22561}
\saveTG{𪮒}{22565}
\saveTG{𤙿}{22572}
\saveTG{𡵣}{22573}
\saveTG{㟛}{22574}
\saveTG{𪮽}{22586}
\saveTG{㹒}{22589}
\saveTG{𤚀}{22594}
\saveTG{𡹍}{22600}
\saveTG{㓠}{22600}
\saveTG{𠜪}{22600}
\saveTG{㞱}{22600}
\saveTG{𡹏}{22600}
\saveTG{𡹤}{22600}
\saveTG{𡸋}{22600}
\saveTG{𠛎}{22600}
\saveTG{𠜯}{22600}
\saveTG{𠞔}{22600}
\saveTG{𠝲}{22600}
\saveTG{𠝪}{22600}
\saveTG{㓯}{22600}
\saveTG{𠰕}{22600}
\saveTG{刮}{22600}
\saveTG{舏}{22600}
\saveTG{峕}{22600}
\saveTG{刣}{22600}
\saveTG{𥢺}{22600}
\saveTG{𣉘}{22601}
\saveTG{𥉃}{22601}
\saveTG{𧬾}{22601}
\saveTG{𡽜}{22601}
\saveTG{𧦙}{22601}
\saveTG{𤰫}{22601}
\saveTG{𤯧}{22601}
\saveTG{𣋤}{22601}
\saveTG{旨}{22601}
\saveTG{𣋧}{22601}
\saveTG{辔}{22601}
\saveTG{𧥥}{22601}
\saveTG{𡺞}{22601}
\saveTG{𣅜}{22601}
\saveTG{𡿸}{22601}
\saveTG{嶜}{22601}
\saveTG{旹}{22601}
\saveTG{峇}{22601}
\saveTG{呰}{22601}
\saveTG{㫮}{22601}
\saveTG{眥}{22601}
\saveTG{訾}{22601}
\saveTG{訔}{22601}
\saveTG{砦}{22602}
\saveTG{皆}{22602}
\saveTG{㠄}{22602}
\saveTG{𠺤}{22602}
\saveTG{𥄾}{22602}
\saveTG{𩠐}{22602}
\saveTG{𩠹}{22602}
\saveTG{𡷚}{22602}
\saveTG{𡻦}{22602}
\saveTG{𣊅}{22602}
\saveTG{𡴗}{22602}
\saveTG{𡷂}{22602}
\saveTG{㟔}{22602}
\saveTG{𣊆}{22602}
\saveTG{岧}{22602}
\saveTG{岩}{22602}
\saveTG{𡴎}{22602}
\saveTG{𠱧}{22603}
\saveTG{𡸟}{22603}
\saveTG{𡿺}{22603}
\saveTG{甾}{22603}
\saveTG{𣌳}{22603}
\saveTG{𠰕}{22604}
\saveTG{𪿂}{22604}
\saveTG{𡺚}{22604}
\saveTG{𠱭}{22604}
\saveTG{𡿵}{22604}
\saveTG{𡷾}{22604}
\saveTG{䣸}{22604}
\saveTG{𠰛}{22604}
\saveTG{㟯}{22604}
\saveTG{峉}{22604}
\saveTG{𨡐}{22604}
\saveTG{峀}{22605}
\saveTG{𡿷}{22606}
\saveTG{𣋎}{22606}
\saveTG{𡼳}{22606}
\saveTG{𡽔}{22607}
\saveTG{𡾡}{22607}
\saveTG{崮}{22607}
\saveTG{㟒}{22607}
\saveTG{𡼫}{22608}
\saveTG{𡻲}{22608}
\saveTG{𣊨}{22608}
\saveTG{𡴜}{22608}
\saveTG{𡺊}{22608}
\saveTG{𡴏}{22608}
\saveTG{𡸱}{22608}
\saveTG{㘘}{22609}
\saveTG{𡅀}{22609}
\saveTG{曫}{22609}
\saveTG{矕}{22609}
\saveTG{轡}{22609}
\saveTG{皉}{22610}
\saveTG{乩}{22610}
\saveTG{臫}{22610}
\saveTG{乱}{22610}
\saveTG{癿}{22610}
\saveTG{乨}{22610}
\saveTG{𠃹}{22610}
\saveTG{㪾}{22612}
\saveTG{𪊈}{22612}
\saveTG{𦱢}{22612}
\saveTG{𡹽}{22612}
\saveTG{𫗿}{22613}
\saveTG{飜}{22613}
\saveTG{嶉}{22615}
\saveTG{𡽛}{22615}
\saveTG{皠}{22615}
\saveTG{𤽊}{22617}
\saveTG{𩡂}{22617}
\saveTG{𣬈}{22617}
\saveTG{㘒}{22617}
\saveTG{𪉚}{22617}
\saveTG{㲒}{22617}
\saveTG{𣰯}{22617}
\saveTG{㿞}{22617}
\saveTG{皑}{22617}
\saveTG{𤕋}{22617}
\saveTG{𪊃}{22617}
\saveTG{𡾖}{22617}
\saveTG{𡿣}{22617}
\saveTG{𤽪}{22617}
\saveTG{𤾢}{22618}
\saveTG{皚}{22618}
\saveTG{𣇘}{22620}
\saveTG{㟢}{22621}
\saveTG{嵜}{22621}
\saveTG{岢}{22621}
\saveTG{斪}{22621}
\saveTG{㣍}{22622}
\saveTG{𪽽}{22622}
\saveTG{㣌}{22622}
\saveTG{𨤌}{22626}
\saveTG{𡷧}{22627}
\saveTG{𪢡}{22627}
\saveTG{𥖱}{22627}
\saveTG{𤾽}{22627}
\saveTG{𤾡}{22627}
\saveTG{}{22627}
\saveTG{𡹌}{22627}
\saveTG{𦺂}{22627}
\saveTG{𩡟}{22627}
\saveTG{𥣉}{22627}
\saveTG{𦱡}{22627}
\saveTG{𪉱}{22627}
\saveTG{𡼞}{22627}
\saveTG{𤫱}{22633}
\saveTG{𦧔}{22633}
\saveTG{𤿃}{22640}
\saveTG{舐}{22640}
\saveTG{𨊃}{22641}
\saveTG{𡾦}{22641}
\saveTG{嵵}{22641}
\saveTG{𤳊}{22641}
\saveTG{𡼋}{22642}
\saveTG{䑛}{22643}
\saveTG{㟕}{22643}
\saveTG{皭}{22646}
\saveTG{𩡖}{22647}
\saveTG{𩡣}{22647}
\saveTG{𦤇}{22647}
\saveTG{皈}{22647}
\saveTG{皧}{22647}
\saveTG{馟}{22647}
\saveTG{畿}{22653}
\saveTG{𩠜}{22661}
\saveTG{啙}{22661}
\saveTG{䭬}{22662}
\saveTG{𡾏}{22662}
\saveTG{𦧗}{22664}
\saveTG{𡴡}{22665}
\saveTG{嵓}{22666}
\saveTG{𡾋}{22666}
\saveTG{𡴩}{22668}
\saveTG{𪩚}{22668}
\saveTG{𩡐}{22669}
\saveTG{皤}{22669}
\saveTG{𤽆}{22670}
\saveTG{𩡎}{22672}
\saveTG{䭯}{22672}
\saveTG{𪨸}{22682}
\saveTG{𡼭}{22682}
\saveTG{𦤟}{22684}
\saveTG{𤳤}{22684}
\saveTG{𤾣}{22685}
\saveTG{𡿃}{22686}
\saveTG{𡾆}{22686}
\saveTG{𦅹}{22693}
\saveTG{䌛}{22693}
\saveTG{𧪬}{22693}
\saveTG{皪}{22694}
\saveTG{𥢖}{22694}
\saveTG{𣈤}{22694}
\saveTG{𤾧}{22695}
\saveTG{𩡀}{22699}
\saveTG{𡶭}{22700}
\saveTG{𡷘}{22700}
\saveTG{𡺢}{22700}
\saveTG{㠒}{22700}
\saveTG{𡸉}{22700}
\saveTG{丩}{22700}
\saveTG{䶡}{22700}
\saveTG{𠝀}{22700}
\saveTG{𠂈}{22700}
\saveTG{𠜼}{22700}
\saveTG{𠝞}{22700}
\saveTG{𠠗}{22700}
\saveTG{剀}{22700}
\saveTG{峢}{22700}
\saveTG{𠄍}{22700}
\saveTG{刨}{22700}
\saveTG{𠞞}{22700}
\saveTG{㓰}{22700}
\saveTG{𪗿}{22700}
\saveTG{𠠚}{22700}
\saveTG{𠜷}{22700}
\saveTG{𣌇}{22701}
\saveTG{𡵘}{22707}
\saveTG{龇}{22710}
\saveTG{齓}{22710}
\saveTG{乢}{22710}
\saveTG{乣}{22710}
\saveTG{𠣶}{22710}
\saveTG{𦱺}{22710}
\saveTG{𡴧}{22710}
\saveTG{比}{22710}
\saveTG{匕}{22710}
\saveTG{𡶍}{22710}
\saveTG{齜}{22710}
\saveTG{𪟨}{22711}
\saveTG{㐠}{22711}
\saveTG{𦭫}{22711}
\saveTG{㲱}{22711}
\saveTG{𣯧}{22711}
\saveTG{𪨧}{22711}
\saveTG{𣰆}{22711}
\saveTG{𠥄}{22711}
\saveTG{崑}{22712}
\saveTG{㟟}{22712}
\saveTG{𡴼}{22712}
\saveTG{𠨤}{22712}
\saveTG{𣮁}{22712}
\saveTG{𡾑}{22712}
\saveTG{巤}{22712}
\saveTG{毝}{22712}
\saveTG{𢆴}{22712}
\saveTG{𢀀}{22712}
\saveTG{𧮌}{22712}
\saveTG{𦣟}{22712}
\saveTG{𪙸}{22712}
\saveTG{𣬟}{22712}
\saveTG{鬯}{22712}
\saveTG{𡶰}{22712}
\saveTG{𡴄}{22712}
\saveTG{𣮼}{22712}
\saveTG{岴}{22712}
\saveTG{𦫓}{22713}
\saveTG{㡭}{22713}
\saveTG{𣭛}{22713}
\saveTG{𪙂}{22713}
\saveTG{𢆸}{22713}
\saveTG{𡳾}{22714}
\saveTG{𡹑}{22714}
\saveTG{𣭤}{22714}
\saveTG{𣰀}{22714}
\saveTG{饪}{22714}
\saveTG{毞}{22714}
\saveTG{𣮄}{22714}
\saveTG{饦}{22714}
\saveTG{𣭁}{22715}
\saveTG{𣬡}{22715}
\saveTG{𣰈}{22715}
\saveTG{𡹛}{22715}
\saveTG{𡸾}{22715}
\saveTG{𡺍}{22715}
\saveTG{𡶁}{22715}
\saveTG{崜}{22715}
\saveTG{𪙷}{22716}
\saveTG{𦇷}{22716}
\saveTG{𡶆}{22716}
\saveTG{𡻮}{22717}
\saveTG{𪓓}{22717}
\saveTG{㽋}{22717}
\saveTG{𤬼}{22717}
\saveTG{岜}{22717}
\saveTG{岂}{22717}
\saveTG{邕}{22717}
\saveTG{𡵋}{22717}
\saveTG{𡸗}{22717}
\saveTG{𡴯}{22717}
\saveTG{𪨩}{22717}
\saveTG{𫜥}{22717}
\saveTG{𤭤}{22717}
\saveTG{𡶃}{22717}
\saveTG{𨸷}{22717}
\saveTG{𪘈}{22717}
\saveTG{㟅}{22717}
\saveTG{𩟜}{22717}
\saveTG{𣰏}{22717}
\saveTG{𣭑}{22717}
\saveTG{𣭀}{22717}
\saveTG{㲎}{22717}
\saveTG{𣰔}{22717}
\saveTG{𣰫}{22717}
\saveTG{𤕆}{22717}
\saveTG{𦉔}{22717}
\saveTG{𢀺}{22717}
\saveTG{𢁃}{22717}
\saveTG{嶝}{22718}
\saveTG{嵦}{22718}
\saveTG{䶣}{22718}
\saveTG{㠦}{22718}
\saveTG{𪨣}{22718}
\saveTG{𠫖}{22719}
\saveTG{𣮂}{22719}
\saveTG{𣰩}{22719}
\saveTG{𣮽}{22719}
\saveTG{𣂯}{22721}
\saveTG{斷}{22721}
\saveTG{嶃}{22721}
\saveTG{𫜭}{22721}
\saveTG{龂}{22721}
\saveTG{𣂢}{22721}
\saveTG{㫁}{22721}
\saveTG{岓}{22721}
\saveTG{断}{22721}
\saveTG{齭}{22721}
\saveTG{齗}{22721}
\saveTG{𡹂}{22721}
\saveTG{𣂶}{22721}
\saveTG{𢒕}{22722}
\saveTG{𣭪}{22726}
\saveTG{㠋}{22727}
\saveTG{𡼰}{22727}
\saveTG{𦦇}{22727}
\saveTG{𣬇}{22727}
\saveTG{㟨}{22727}
\saveTG{嵴}{22727}
\saveTG{嶠}{22727}
\saveTG{峁}{22727}
\saveTG{龋}{22727}
\saveTG{齲}{22727}
\saveTG{𢑉}{22727}
\saveTG{𡹗}{22727}
\saveTG{𡷴}{22727}
\saveTG{𡹫}{22727}
\saveTG{𡿞}{22727}
\saveTG{𡸎}{22727}
\saveTG{𢑂}{22727}
\saveTG{𠨫}{22727}
\saveTG{㞻}{22727}
\saveTG{𡼙}{22727}
\saveTG{𡾀}{22727}
\saveTG{𨺀}{22727}
\saveTG{𡷬}{22727}
\saveTG{𡿀}{22727}
\saveTG{𡶹}{22727}
\saveTG{岇}{22727}
\saveTG{峤}{22728}
\saveTG{𪗢}{22730}
\saveTG{𡵰}{22730}
\saveTG{厸}{22730}
\saveTG{𧘇}{22730}
\saveTG{𡵦}{22731}
\saveTG{𪨭}{22731}
\saveTG{𢆾}{22731}
\saveTG{𢇑}{22731}
\saveTG{𡵏}{22731}
\saveTG{𢆶}{22731}
\saveTG{𩚓}{22732}
\saveTG{𡾨}{22732}
\saveTG{飺}{22732}
\saveTG{峎}{22732}
\saveTG{崀}{22732}
\saveTG{嵔}{22732}
\saveTG{製}{22732}
\saveTG{𧟏}{22732}
\saveTG{䘡}{22732}
\saveTG{𧛵}{22732}
\saveTG{𧚲}{22732}
\saveTG{𡴁}{22732}
\saveTG{𡻌}{22732}
\saveTG{𧛬}{22732}
\saveTG{𧙁}{22732}
\saveTG{㠢}{22732}
\saveTG{𩝖}{22732}
\saveTG{𩞟}{22732}
\saveTG{𤬍}{22733}
\saveTG{𤫪}{22733}
\saveTG{𡶉}{22737}
\saveTG{嶾}{22737}
\saveTG{𫗰}{22737}
\saveTG{岻}{22740}
\saveTG{}{22741}
\saveTG{嶭}{22741}
\saveTG{𡶜}{22741}
\saveTG{𡸫}{22741}
\saveTG{𨻪}{22742}
\saveTG{𪘤}{22743}
\saveTG{𪚅}{22743}
\saveTG{𡹜}{22744}
\saveTG{㟎}{22744}
\saveTG{𫗪}{22744}
\saveTG{馁}{22744}
\saveTG{䐘}{22744}
\saveTG{嵕}{22747}
\saveTG{𩟾}{22747}
\saveTG{嵈}{22747}
\saveTG{饭}{22747}
\saveTG{岅}{22747}
\saveTG{𡽀}{22747}
\saveTG{㞴}{22747}
\saveTG{𡶗}{22747}
\saveTG{𪨾}{22747}
\saveTG{𧁇}{22747}
\saveTG{𡷶}{22748}
\saveTG{𡻀}{22754}
\saveTG{𡼠}{22754}
\saveTG{崢}{22757}
\saveTG{𡴕}{22757}
\saveTG{𡴋}{22757}
\saveTG{𡹆}{22757}
\saveTG{𡴛}{22757}
\saveTG{𪙧}{22758}
\saveTG{𪘇}{22761}
\saveTG{𪗷}{22761}
\saveTG{𡼴}{22761}
\saveTG{𣯐}{22762}
\saveTG{𡺓}{22762}
\saveTG{匘}{22762}
\saveTG{崰}{22763}
\saveTG{𪗽}{22764}
\saveTG{𪩄}{22764}
\saveTG{𪘢}{22764}
\saveTG{崏}{22764}
\saveTG{𡺗}{22765}
\saveTG{嶓}{22769}
\saveTG{崉}{22769}
\saveTG{𠁿}{22770}
\saveTG{𡴳}{22770}
\saveTG{𠙸}{22770}
\saveTG{𠚈}{22770}
\saveTG{𡺳}{22770}
\saveTG{𦣼}{22770}
\saveTG{𠙾}{22770}
\saveTG{𠁳}{22770}
\saveTG{𠚀}{22770}
\saveTG{𠙷}{22770}
\saveTG{𠙽}{22770}
\saveTG{𠚌}{22770}
\saveTG{𠚓}{22770}
\saveTG{𠙵}{22770}
\saveTG{𠚖}{22770}
\saveTG{𥃬}{22770}
\saveTG{𪗘}{22770}
\saveTG{𠚒}{22770}
\saveTG{𦥒}{22770}
\saveTG{𤮺}{22770}
\saveTG{𠚠}{22770}
\saveTG{𠙼}{22770}
\saveTG{𡻿}{22770}
\saveTG{𡷻}{22770}
\saveTG{𡺖}{22770}
\saveTG{𪞶}{22770}
\saveTG{丱}{22770}
\saveTG{𠚊}{22770}
\saveTG{𠚍}{22770}
\saveTG{𠚉}{22770}
\saveTG{𠙹}{22770}
\saveTG{𠙺}{22770}
\saveTG{𠚂}{22770}
\saveTG{𠚅}{22770}
\saveTG{𦣹}{22770}
\saveTG{𠚁}{22770}
\saveTG{㟗}{22770}
\saveTG{𠚄}{22770}
\saveTG{𠚕}{22770}
\saveTG{㓙}{22770}
\saveTG{豳}{22770}
\saveTG{凼}{22770}
\saveTG{凵}{22770}
\saveTG{凷}{22770}
\saveTG{山}{22770}
\saveTG{屾}{22770}
\saveTG{凶}{22770}
\saveTG{幽}{22770}
\saveTG{𪘥}{22771}
\saveTG{𪙬}{22771}
\saveTG{𠚛}{22772}
\saveTG{𡷈}{22772}
\saveTG{𪗶}{22772}
\saveTG{𡵤}{22772}
\saveTG{𡴸}{22772}
\saveTG{𪚄}{22772}
\saveTG{𠚋}{22772}
\saveTG{𢇇}{22772}
\saveTG{𪨲}{22772}
\saveTG{𣂬}{22772}
\saveTG{𡕜}{22772}
\saveTG{𡷼}{22772}
\saveTG{出}{22772}
\saveTG{岀}{22772}
\saveTG{饳}{22772}
\saveTG{巒}{22772}
\saveTG{𢇕}{22772}
\saveTG{𡶏}{22772}
\saveTG{𪙒}{22772}
\saveTG{𪗨}{22772}
\saveTG{𦈬}{22774}
\saveTG{𫄾}{22774}
\saveTG{𡶫}{22777}
\saveTG{𪘾}{22777}
\saveTG{𦦽}{22777}
\saveTG{𪗮}{22777}
\saveTG{𧀍}{22777}
\saveTG{𡺫}{22777}
\saveTG{𡽙}{22782}
\saveTG{𡼽}{22782}
\saveTG{嵌}{22782}
\saveTG{嵚}{22782}
\saveTG{嵠}{22784}
\saveTG{𡸒}{22784}
\saveTG{岆}{22784}
\saveTG{饫}{22784}
\saveTG{𫖩}{22786}
\saveTG{𪩌}{22786}
\saveTG{嵊}{22791}
\saveTG{繇}{22793}
\saveTG{𪘿}{22794}
\saveTG{𡾒}{22794}
\saveTG{𡸯}{22794}
\saveTG{𠃆}{22794}
\saveTG{嶫}{22795}
\saveTG{𡹴}{22795}
\saveTG{𪟇}{22800}
\saveTG{𠟣}{22800}
\saveTG{劗}{22800}
\saveTG{𠞿}{22800}
\saveTG{𪿍}{22800}
\saveTG{𠟮}{22800}
\saveTG{𠟓}{22800}
\saveTG{𡴆}{22800}
\saveTG{眞}{22801}
\saveTG{𠔞}{22801}
\saveTG{㠘}{22801}
\saveTG{𠔖}{22801}
\saveTG{𡻗}{22801}
\saveTG{𠔛}{22801}
\saveTG{𡹦}{22801}
\saveTG{𪩃}{22802}
\saveTG{𠇿}{22802}
\saveTG{𧿿}{22802}
\saveTG{𨃫}{22802}
\saveTG{赀}{22802}
\saveTG{赁}{22802}
\saveTG{𣢙}{22802}
\saveTG{𢇈}{22802}
\saveTG{㞺}{22804}
\saveTG{奱}{22804}
\saveTG{𡘌}{22804}
\saveTG{𡸦}{22804}
\saveTG{𡵗}{22804}
\saveTG{𦬦}{22804}
\saveTG{𨖪}{22804}
\saveTG{𡷪}{22804}
\saveTG{𡾗}{22805}
\saveTG{𤴗}{22805}
\saveTG{𡹈}{22805}
\saveTG{𡿍}{22806}
\saveTG{𡻃}{22806}
\saveTG{𧵠}{22806}
\saveTG{𧴺}{22806}
\saveTG{貲}{22806}
\saveTG{賃}{22806}
\saveTG{𧷓}{22806}
\saveTG{𧷷}{22806}
\saveTG{𧷵}{22806}
\saveTG{𧴲}{22806}
\saveTG{𨖁}{22806}
\saveTG{𧹂}{22806}
\saveTG{𡺋}{22806}
\saveTG{㞤}{22807}
\saveTG{㞥}{22807}
\saveTG{𦇥}{22807}
\saveTG{炭}{22809}
\saveTG{𤎁}{22809}
\saveTG{𤇲}{22809}
\saveTG{熋}{22809}
\saveTG{𤋴}{22809}
\saveTG{𡽽}{22809}
\saveTG{𤉞}{22809}
\saveTG{𤓖}{22809}
\saveTG{災}{22809}
\saveTG{𥡙}{22809}
\saveTG{𤎾}{22809}
\saveTG{𥏱}{22815}
\saveTG{𠤗}{22817}
\saveTG{𣮇}{22817}
\saveTG{𥐀}{22821}
\saveTG{𢒟}{22822}
\saveTG{𢒸}{22822}
\saveTG{𡺕}{22827}
\saveTG{𡿙}{22827}
\saveTG{崱}{22827}
\saveTG{𦓟}{22827}
\saveTG{𧹋}{22827}
\saveTG{𦤭}{22830}
\saveTG{𡘿}{22832}
\saveTG{𡗖}{22847}
\saveTG{𥐁}{22851}
\saveTG{𧴭}{22862}
\saveTG{𡼝}{22864}
\saveTG{㸋}{22869}
\saveTG{𡚘}{22869}
\saveTG{䌊}{22872}
\saveTG{𡴫}{22880}
\saveTG{嶷}{22881}
\saveTG{𡴪}{22881}
\saveTG{㠌}{22882}
\saveTG{巅}{22882}
\saveTG{𡺏}{22882}
\saveTG{𥟯}{22883}
\saveTG{𢇃}{22883}
\saveTG{𡽗}{22886}
\saveTG{巓}{22886}
\saveTG{巔}{22886}
\saveTG{𡵝}{22887}
\saveTG{𡸓}{22888}
\saveTG{𡺛}{22888}
\saveTG{𤎸}{22889}
\saveTG{𡹖}{22889}
\saveTG{𤇬}{22891}
\saveTG{𤉌}{22892}
\saveTG{𤑗}{22895}
\saveTG{𡼷}{22896}
\saveTG{𤎚}{22896}
\saveTG{𡵖}{22897}
\saveTG{𤈹}{22898}
\saveTG{𠠍}{22900}
\saveTG{𥟻}{22900}
\saveTG{𥢫}{22900}
\saveTG{𥠉}{22900}
\saveTG{𥞲}{22900}
\saveTG{㓿}{22900}
\saveTG{㓷}{22900}
\saveTG{𦃖}{22900}
\saveTG{䅀}{22900}
\saveTG{𠜫}{22900}
\saveTG{𫃺}{22900}
\saveTG{利}{22900}
\saveTG{㓽}{22900}
\saveTG{紖}{22900}
\saveTG{紃}{22900}
\saveTG{剰}{22900}
\saveTG{剩}{22900}
\saveTG{糾}{22900}
\saveTG{絒}{22900}
\saveTG{𠟫}{22900}
\saveTG{𦀑}{22900}
\saveTG{𦄈}{22900}
\saveTG{𠛴}{22900}
\saveTG{剝}{22900}
\saveTG{𠛩}{22900}
\saveTG{䌃}{22900}
\saveTG{𦀎}{22900}
\saveTG{𠟾}{22900}
\saveTG{剿}{22900}
\saveTG{𠞙}{22900}
\saveTG{祡}{22901}
\saveTG{𥚢}{22901}
\saveTG{𡻰}{22901}
\saveTG{𡮙}{22901}
\saveTG{㞸}{22901}
\saveTG{崇}{22901}
\saveTG{祟}{22901}
\saveTG{𡽿}{22901}
\saveTG{灓}{22902}
\saveTG{汖}{22902}
\saveTG{䋕}{22903}
\saveTG{𦀣}{22903}
\saveTG{𥾜}{22903}
\saveTG{𥿟}{22903}
\saveTG{𡻭}{22903}
\saveTG{𡽭}{22903}
\saveTG{𡿜}{22903}
\saveTG{糸}{22903}
\saveTG{𡼊}{22903}
\saveTG{紫}{22903}
\saveTG{糹}{22903}
\saveTG{粜}{22904}
\saveTG{岽}{22904}
\saveTG{樂}{22904}
\saveTG{梨}{22904}
\saveTG{欒}{22904}
\saveTG{栠}{22904}
\saveTG{𣎵}{22904}
\saveTG{𣐯}{22904}
\saveTG{𡸂}{22904}
\saveTG{㠍}{22904}
\saveTG{𣖃}{22904}
\saveTG{𡵬}{22904}
\saveTG{𥟸}{22904}
\saveTG{𥟜}{22904}
\saveTG{𡾚}{22904}
\saveTG{𣕐}{22904}
\saveTG{𡾹}{22904}
\saveTG{𡿈}{22904}
\saveTG{𡷵}{22904}
\saveTG{𣞏}{22904}
\saveTG{𡺼}{22904}
\saveTG{𥺋}{22904}
\saveTG{𣟾}{22904}
\saveTG{𣠋}{22904}
\saveTG{枈}{22904}
\saveTG{粊}{22904}
\saveTG{柴}{22904}
\saveTG{巢}{22904}
\saveTG{𡻝}{22905}
\saveTG{嶪}{22905}
\saveTG{𡹞}{22906}
\saveTG{𡾍}{22906}
\saveTG{嶚}{22906}
\saveTG{𡼩}{22906}
\saveTG{崬}{22906}
\saveTG{㴝}{22909}
\saveTG{秕}{22910}
\saveTG{紕}{22910}
\saveTG{糺}{22910}
\saveTG{乿}{22910}
\saveTG{紪}{22910}
\saveTG{𥠴}{22911}
\saveTG{𦇈}{22912}
\saveTG{𥿤}{22912}
\saveTG{𦅺}{22912}
\saveTG{繼}{22913}
\saveTG{絩}{22913}
\saveTG{𦁚}{22913}
\saveTG{𦂹}{22913}
\saveTG{秅}{22914}
\saveTG{紝}{22914}
\saveTG{𦀘}{22914}
\saveTG{𦁍}{22914}
\saveTG{𥢰}{22914}
\saveTG{秏}{22914}
\saveTG{秹}{22914}
\saveTG{絍}{22914}
\saveTG{嵀}{22914}
\saveTG{𥟰}{22914}
\saveTG{𡿂}{22915}
\saveTG{𡽯}{22915}
\saveTG{𪐈}{22915}
\saveTG{𡽰}{22915}
\saveTG{綞}{22915}
\saveTG{繀}{22915}
\saveTG{種}{22915}
\saveTG{䅜}{22915}
\saveTG{𦇮}{22915}
\saveTG{𡿏}{22915}
\saveTG{緟}{22915}
\saveTG{𦆻}{22916}
\saveTG{䅱}{22916}
\saveTG{𦃽}{22917}
\saveTG{𥝏}{22917}
\saveTG{𥾆}{22917}
\saveTG{𥾖}{22917}
\saveTG{𥾱}{22917}
\saveTG{㠇}{22917}
\saveTG{𥣼}{22917}
\saveTG{𥞕}{22917}
\saveTG{𢓝}{22917}
\saveTG{𥞝}{22917}
\saveTG{𥝓}{22917}
\saveTG{䌐}{22917}
\saveTG{𦄬}{22917}
\saveTG{𦃇}{22917}
\saveTG{𣰮}{22917}
\saveTG{𣮕}{22917}
\saveTG{䋃}{22917}
\saveTG{𥟉}{22917}
\saveTG{𣬖}{22917}
\saveTG{𥠱}{22917}
\saveTG{𣂫}{22921}
\saveTG{𣂛}{22921}
\saveTG{𪏴}{22921}
\saveTG{𥝹}{22921}
\saveTG{𥟝}{22921}
\saveTG{𥟞}{22921}
\saveTG{𦈀}{22921}
\saveTG{𥞉}{22921}
\saveTG{紤}{22921}
\saveTG{𢒭}{22922}
\saveTG{𦅈}{22922}
\saveTG{𢒦}{22922}
\saveTG{㣓}{22922}
\saveTG{𢒚}{22922}
\saveTG{彩}{22922}
\saveTG{絼}{22922}
\saveTG{𥝞}{22922}
\saveTG{𡽲}{22927}
\saveTG{𥢬}{22927}
\saveTG{𥣌}{22927}
\saveTG{𪐑}{22927}
\saveTG{𦅂}{22927}
\saveTG{𡹱}{22927}
\saveTG{𥠄}{22927}
\saveTG{𥞄}{22927}
\saveTG{䋳}{22927}
\saveTG{𦀦}{22927}
\saveTG{𣝜}{22927}
\saveTG{𡻣}{22927}
\saveTG{𥡇}{22927}
\saveTG{𦄟}{22927}
\saveTG{𦇜}{22927}
\saveTG{𪩋}{22927}
\saveTG{𣜱}{22927}
\saveTG{䅎}{22927}
\saveTG{𦅧}{22927}
\saveTG{繃}{22927}
\saveTG{秽}{22927}
\saveTG{穚}{22927}
\saveTG{繑}{22927}
\saveTG{峲}{22927}
\saveTG{綉}{22927}
\saveTG{纗}{22927}
\saveTG{𦀼}{22927}
\saveTG{𫁉}{22927}
\saveTG{𡸻}{22928}
\saveTG{𦁗}{22929}
\saveTG{𦀋}{22930}
\saveTG{紭}{22930}
\saveTG{私}{22930}
\saveTG{𥢾}{22931}
\saveTG{𫃹}{22931}
\saveTG{𥝠}{22931}
\saveTG{纁}{22931}
\saveTG{𦃾}{22932}
\saveTG{𦄘}{22932}
\saveTG{𦁟}{22932}
\saveTG{𤖼}{22932}
\saveTG{𥿯}{22932}
\saveTG{𦄭}{22932}
\saveTG{崧}{22932}
\saveTG{𤬇}{22933}
\saveTG{𦁣}{22933}
\saveTG{𡾳}{22936}
\saveTG{𥾭}{22937}
\saveTG{𦆸}{22937}
\saveTG{䌥}{22937}
\saveTG{穏}{22937}
\saveTG{穩}{22937}
\saveTG{纞}{22939}
\saveTG{䊹}{22940}
\saveTG{䄭}{22940}
\saveTG{秪}{22940}
\saveTG{秖}{22940}
\saveTG{紙}{22940}
\saveTG{綖}{22941}
\saveTG{𥢤}{22941}
\saveTG{𦂜}{22941}
\saveTG{𦃉}{22941}
\saveTG{綎}{22941}
\saveTG{𥿊}{22941}
\saveTG{𥟳}{22943}
\saveTG{𪱵}{22943}
\saveTG{緌}{22944}
\saveTG{綏}{22944}
\saveTG{𫃸}{22944}
\saveTG{䅑}{22944}
\saveTG{䅗}{22944}
\saveTG{𫄌}{22944}
\saveTG{穱}{22946}
\saveTG{𦇱}{22947}
\saveTG{𥣾}{22947}
\saveTG{𥤈}{22947}
\saveTG{𥣁}{22947}
\saveTG{𥿄}{22947}
\saveTG{𥿈}{22947}
\saveTG{𢾽}{22947}
\saveTG{嵙}{22947}
\saveTG{稱}{22947}
\saveTG{𦆔}{22947}
\saveTG{𦇻}{22947}
\saveTG{𦇌}{22947}
\saveTG{𦁳}{22947}
\saveTG{緵}{22947}
\saveTG{綬}{22947}
\saveTG{稯}{22947}
\saveTG{緩}{22947}
\saveTG{綒}{22947}
\saveTG{稃}{22947}
\saveTG{𥟛}{22947}
\saveTG{𦆛}{22947}
\saveTG{𡼹}{22948}
\saveTG{𡼦}{22948}
\saveTG{𪏹}{22949}
\saveTG{𥿙}{22950}
\saveTG{𦃱}{22951}
\saveTG{㭰}{22952}
\saveTG{嶘}{22953}
\saveTG{穖}{22953}
\saveTG{𡺄}{22953}
\saveTG{𥤑}{22954}
\saveTG{𥠐}{22956}
\saveTG{𫀱}{22961}
\saveTG{𥿻}{22961}
\saveTG{𥟠}{22961}
\saveTG{稭}{22962}
\saveTG{𦄃}{22962}
\saveTG{緇}{22963}
\saveTG{䅔}{22963}
\saveTG{𥞸}{22964}
\saveTG{𥟴}{22964}
\saveTG{絬}{22964}
\saveTG{秳}{22964}
\saveTG{緍}{22964}
\saveTG{𦅑}{22964}
\saveTG{䋸}{22964}
\saveTG{𡿘}{22966}
\saveTG{𡾽}{22966}
\saveTG{䅨}{22969}
\saveTG{𥢌}{22969}
\saveTG{稲}{22969}
\saveTG{繙}{22969}
\saveTG{𦇓}{22970}
\saveTG{𦂣}{22970}
\saveTG{秈}{22970}
\saveTG{𥠃}{22970}
\saveTG{𦇡}{22970}
\saveTG{𥞃}{22972}
\saveTG{𦤙}{22972}
\saveTG{絀}{22972}
\saveTG{𦂉}{22977}
\saveTG{稻}{22977}
\saveTG{縚}{22977}
\saveTG{䅤}{22977}
\saveTG{𦄮}{22977}
\saveTG{䵚}{22977}
\saveTG{𥠞}{22978}
\saveTG{𦽣}{22982}
\saveTG{縘}{22984}
\saveTG{𫃭}{22984}
\saveTG{秗}{22984}
\saveTG{𥣜}{22985}
\saveTG{纀}{22985}
\saveTG{𦄾}{22985}
\saveTG{穙}{22985}
\saveTG{𦅨}{22986}
\saveTG{𥣲}{22990}
\saveTG{𥡶}{22991}
\saveTG{𦃒}{22991}
\saveTG{㟤}{22991}
\saveTG{𦈄}{22993}
\saveTG{}{22993}
\saveTG{𦇹}{22993}
\saveTG{𫄕}{22993}
\saveTG{𦆕}{22993}
\saveTG{䜌}{22993}
\saveTG{𠬆}{22993}
\saveTG{𡾥}{22993}
\saveTG{絲}{22993}
\saveTG{𡮟}{22993}
\saveTG{綵}{22994}
\saveTG{秝}{22994}
\saveTG{纅}{22994}
\saveTG{繅}{22994}
\saveTG{𡹇}{22994}
\saveTG{𦂏}{22994}
\saveTG{𥿘}{22994}
\saveTG{𡽼}{22994}
\saveTG{𣞆}{22994}
\saveTG{𥟩}{22994}
\saveTG{𥟱}{22995}
\saveTG{𥣈}{22995}
\saveTG{䌜}{22995}
\saveTG{𥠗}{22995}
\saveTG{𢀌}{22995}
\saveTG{𣑼}{22997}
\saveTG{𡶿}{22997}
\saveTG{𥡝}{22997}
\saveTG{𪏭}{22998}
\saveTG{𫃿}{22999}
\saveTG{𪐐}{22999}
\saveTG{卜}{23000}
\saveTG{𤗇}{23012}
\saveTG{𤗍}{23017}
\saveTG{𦅜}{23021}
\saveTG{𢹨}{23022}
\saveTG{𤗲}{23022}
\saveTG{𠫫}{23027}
\saveTG{牑}{23027}
\saveTG{𤗃}{23027}
\saveTG{牖}{23027}
\saveTG{𤗀}{23032}
\saveTG{牔}{23042}
\saveTG{𢦤}{23050}
\saveTG{牋}{23053}
\saveTG{𤗂}{23099}
\saveTG{𪽁}{23100}
\saveTG{𠫰}{23101}
\saveTG{𢨑}{23101}
\saveTG{叄}{23101}
\saveTG{叁}{23101}
\saveTG{𠬅}{23101}
\saveTG{𠬄}{23101}
\saveTG{亝}{23101}
\saveTG{𥁐}{23102}
\saveTG{𥁱}{23102}
\saveTG{𥂒}{23102}
\saveTG{𥂥}{23102}
\saveTG{𥂏}{23102}
\saveTG{𥂐}{23102}
\saveTG{𠍜}{23102}
\saveTG{盀}{23102}
\saveTG{𡒥}{23102}
\saveTG{垡}{23104}
\saveTG{𡊏}{23104}
\saveTG{𠬁}{23104}
\saveTG{𡊄}{23104}
\saveTG{𡎥}{23104}
\saveTG{垒}{23104}
\saveTG{𫚑}{23104}
\saveTG{𤯢}{23104}
\saveTG{𡊅}{23104}
\saveTG{垈}{23104}
\saveTG{𡒉}{23104}
\saveTG{𠫶}{23105}
\saveTG{𠬋}{23105}
\saveTG{}{23112}
\saveTG{纻}{23112}
\saveTG{鱿}{23112}
\saveTG{鸵}{23112}
\saveTG{鲩}{23112}
\saveTG{𦈍}{23112}
\saveTG{}{23112}
\saveTG{𪣁}{23117}
\saveTG{𩽽}{23117}
\saveTG{鲹}{23122}
\saveTG{𧗋}{23122}
\saveTG{𠗿}{23122}
\saveTG{𠗒}{23127}
\saveTG{𫚙}{23127}
\saveTG{编}{23127}
\saveTG{鳊}{23127}
\saveTG{䘐}{23130}
\saveTG{𧒎}{23131}
\saveTG{䘑}{23132}
\saveTG{𫚬}{23132}
\saveTG{鳡}{23135}
\saveTG{𧌦}{23136}
\saveTG{䖸}{23136}
\saveTG{𧊇}{23136}
\saveTG{𫊲}{23136}
\saveTG{𧈧}{23136}
\saveTG{𢎍}{23141}
\saveTG{缚}{23142}
\saveTG{𩽾}{23144}
\saveTG{绂}{23147}
\saveTG{鲅}{23147}
\saveTG{䘒}{23147}
\saveTG{䳁}{23147}
\saveTG{缄}{23150}
\saveTG{绒}{23150}
\saveTG{线}{23150}
\saveTG{𪉇}{23151}
\saveTG{𩶢}{23152}
\saveTG{𢨡}{23153}
\saveTG{𤯫}{23155}
\saveTG{𧻂}{23157}
\saveTG{𦈚}{23159}
\saveTG{绐}{23160}
\saveTG{鲐}{23160}
\saveTG{缩}{23162}
\saveTG{绺}{23164}
\saveTG{䘔}{23164}
\saveTG{𫚦}{23168}
\saveTG{绾}{23177}
\saveTG{绽}{23181}
\saveTG{缤}{23181}
\saveTG{𫛣}{23182}
\saveTG{𤝶}{23184}
\saveTG{𤠷}{23184}
\saveTG{獃}{23184}
\saveTG{𫄫}{23184}
\saveTG{}{23186}
\saveTG{综}{23191}
\saveTG{𦈒}{23194}
\saveTG{𩾁}{23199}
\saveTG{𩳴}{23200}
\saveTG{伈}{23200}
\saveTG{仆}{23200}
\saveTG{外}{23200}
\saveTG{𢒮}{23202}
\saveTG{参}{23202}
\saveTG{參}{23202}
\saveTG{佖}{23204}
\saveTG{㣗}{23207}
\saveTG{𩳚}{23211}
\saveTG{𩲮}{23211}
\saveTG{𠫱}{23212}
\saveTG{𠒊}{23212}
\saveTG{𠫝}{23212}
\saveTG{㸜}{23212}
\saveTG{𠋪}{23212}
\saveTG{𩴑}{23212}
\saveTG{𧇒}{23212}
\saveTG{𧈚}{23212}
\saveTG{𥃓}{23212}
\saveTG{𩴄}{23212}
\saveTG{𩳐}{23212}
\saveTG{𠐅}{23212}
\saveTG{𠊙}{23212}
\saveTG{俒}{23212}
\saveTG{僦}{23212}
\saveTG{倥}{23212}
\saveTG{躻}{23212}
\saveTG{佗}{23212}
\saveTG{𩱻}{23212}
\saveTG{优}{23212}
\saveTG{允}{23212}
\saveTG{伫}{23212}
\saveTG{倇}{23212}
\saveTG{𠋤}{23214}
\saveTG{𧳑}{23214}
\saveTG{𩲭}{23214}
\saveTG{侘}{23214}
\saveTG{僿}{23214}
\saveTG{䏵}{23214}
\saveTG{𩴁}{23215}
\saveTG{僱}{23215}
\saveTG{𩁏}{23215}
\saveTG{𩲥}{23216}
\saveTG{𠊿}{23216}
\saveTG{𧤎}{23216}
\saveTG{𠘭}{23217}
\saveTG{𠫠}{23217}
\saveTG{𠙃}{23217}
\saveTG{𪗃}{23217}
\saveTG{𠎾}{23217}
\saveTG{𠫞}{23217}
\saveTG{𧣖}{23217}
\saveTG{𡖟}{23217}
\saveTG{𡰘}{23217}
\saveTG{𠎟}{23217}
\saveTG{𠎂}{23217}
\saveTG{𪫌}{23217}
\saveTG{𡰐}{23217}
\saveTG{𡖢}{23217}
\saveTG{𠫨}{23217}
\saveTG{𠈵}{23217}
\saveTG{𠑙}{23217}
\saveTG{𪝮}{23217}
\saveTG{𩴸}{23218}
\saveTG{偡}{23218}
\saveTG{𠑅}{23218}
\saveTG{𠐏}{23218}
\saveTG{𩴳}{23219}
\saveTG{𩳞}{23219}
\saveTG{𠫧}{23221}
\saveTG{𠫦}{23221}
\saveTG{㣷}{23221}
\saveTG{𠬐}{23221}
\saveTG{𠬘}{23221}
\saveTG{𥚪}{23221}
\saveTG{𢕗}{23221}
\saveTG{佇}{23221}
\saveTG{𠫸}{23221}
\saveTG{𧏝}{23221}
\saveTG{𪝣}{23221}
\saveTG{儜}{23221}
\saveTG{𠫼}{23221}
\saveTG{𢕕}{23222}
\saveTG{傪}{23222}
\saveTG{帒}{23227}
\saveTG{𢁔}{23227}
\saveTG{䏍}{23227}
\saveTG{𠫣}{23227}
\saveTG{𠌽}{23227}
\saveTG{𠬔}{23227}
\saveTG{𢒏}{23227}
\saveTG{𦓞}{23227}
\saveTG{𠃃}{23227}
\saveTG{𦚅}{23227}
\saveTG{𦙯}{23227}
\saveTG{𠊘}{23227}
\saveTG{𠏶}{23227}
\saveTG{𠌌}{23227}
\saveTG{徧}{23227}
\saveTG{俌}{23227}
\saveTG{偏}{23227}
\saveTG{傓}{23227}
\saveTG{𧴕}{23230}
\saveTG{𣲉}{23231}
\saveTG{𠏷}{23231}
\saveTG{𠍋}{23231}
\saveTG{𢔻}{23232}
\saveTG{𤂼}{23232}
\saveTG{傢}{23232}
\saveTG{俍}{23232}
\saveTG{躴}{23232}
\saveTG{𠇟}{23232}
\saveTG{𪵨}{23232}
\saveTG{𠍌}{23233}
\saveTG{𠎰}{23235}
\saveTG{𠏨}{23235}
\saveTG{𢖘}{23236}
\saveTG{㒄}{23238}
\saveTG{㒏}{23238}
\saveTG{𢎎}{23240}
\saveTG{𢍿}{23240}
\saveTG{𢓀}{23240}
\saveTG{侙}{23240}
\saveTG{代}{23240}
\saveTG{倵}{23240}
\saveTG{𧇭}{23241}
\saveTG{𠈸}{23241}
\saveTG{𤖟}{23241}
\saveTG{𧳵}{23242}
\saveTG{傅}{23242}
\saveTG{𢓭}{23242}
\saveTG{㑏}{23243}
\saveTG{𦓗}{23244}
\saveTG{𠊢}{23244}
\saveTG{侒}{23244}
\saveTG{𤕳}{23244}
\saveTG{𢀁}{23244}
\saveTG{𧲯}{23244}
\saveTG{𡖨}{23244}
\saveTG{𠊯}{23247}
\saveTG{㑓}{23247}
\saveTG{𪜸}{23247}
\saveTG{𢅢}{23247}
\saveTG{𠇞}{23247}
\saveTG{俊}{23247}
\saveTG{𠋢}{23247}
\saveTG{𠨢}{23247}
\saveTG{𠮍}{23247}
\saveTG{䝜}{23247}
\saveTG{𠍎}{23248}
\saveTG{㒃}{23248}
\saveTG{𠌗}{23248}
\saveTG{𠉛}{23250}
\saveTG{𠈙}{23250}
\saveTG{臧}{23250}
\saveTG{𢕽}{23250}
\saveTG{伐}{23250}
\saveTG{戯}{23250}
\saveTG{𠇘}{23250}
\saveTG{傤}{23250}
\saveTG{儎}{23250}
\saveTG{俄}{23250}
\saveTG{𧥄}{23250}
\saveTG{𢧁}{23250}
\saveTG{𢨕}{23250}
\saveTG{㑘}{23250}
\saveTG{𪝠}{23250}
\saveTG{傶}{23250}
\saveTG{侔}{23250}
\saveTG{𢧶}{23250}
\saveTG{戲}{23250}
\saveTG{𡾃}{23250}
\saveTG{軄}{23250}
\saveTG{戕}{23250}
\saveTG{戱}{23250}
\saveTG{𢨘}{23250}
\saveTG{𢖝}{23251}
\saveTG{𤖔}{23251}
\saveTG{𧣼}{23251}
\saveTG{𠏯}{23252}
\saveTG{𠈭}{23252}
\saveTG{𠬂}{23252}
\saveTG{俴}{23253}
\saveTG{㣤}{23253}
\saveTG{虥}{23253}
\saveTG{𧣴}{23253}
\saveTG{𤖆}{23253}
\saveTG{𠋰}{23253}
\saveTG{𧣱}{23253}
\saveTG{㣝}{23254}
\saveTG{𠈋}{23254}
\saveTG{𤖞}{23254}
\saveTG{𠋘}{23254}
\saveTG{𦠾}{23254}
\saveTG{𠏃}{23255}
\saveTG{𠐲}{23256}
\saveTG{𢧠}{23256}
\saveTG{𠊭}{23256}
\saveTG{𢔹}{23257}
\saveTG{𠏞}{23258}
\saveTG{𠏔}{23258}
\saveTG{𠑀}{23258}
\saveTG{佁}{23260}
\saveTG{𡖤}{23261}
\saveTG{佁}{23261}
\saveTG{𢕂}{23261}
\saveTG{𠍊}{23262}
\saveTG{倃}{23264}
\saveTG{偺}{23264}
\saveTG{僒}{23267}
\saveTG{𠌣}{23267}
\saveTG{傛}{23268}
\saveTG{𠏅}{23272}
\saveTG{𪚊}{23272}
\saveTG{㑻}{23272}
\saveTG{𠇇}{23274}
\saveTG{倌}{23277}
\saveTG{傧}{23281}
\saveTG{貁}{23282}
\saveTG{𠐻}{23282}
\saveTG{𠉵}{23282}
\saveTG{㑦}{23284}
\saveTG{𢓪}{23284}
\saveTG{伏}{23284}
\saveTG{俟}{23284}
\saveTG{狀}{23284}
\saveTG{𤟮}{23284}
\saveTG{𫜤}{23284}
\saveTG{𤢚}{23284}
\saveTG{𠊲}{23284}
\saveTG{𧳂}{23284}
\saveTG{𤟜}{23284}
\saveTG{獻}{23284}
\saveTG{𤣏}{23284}
\saveTG{𪝤}{23286}
\saveTG{𠋏}{23286}
\saveTG{軉}{23286}
\saveTG{儐}{23286}
\saveTG{𤠩}{23289}
\saveTG{徖}{23291}
\saveTG{倧}{23291}
\saveTG{㒍}{23293}
\saveTG{𠇲}{23294}
\saveTG{俕}{23294}
\saveTG{𧲷}{23294}
\saveTG{𠐟}{23296}
\saveTG{𤖡}{23296}
\saveTG{𤕾}{23299}
\saveTG{觩}{23299}
\saveTG{俅}{23299}
\saveTG{𩵽}{23300}
\saveTG{𩾽}{23300}
\saveTG{𪬤}{23304}
\saveTG{鮅}{23304}
\saveTG{魲}{23307}
\saveTG{𪆮}{23312}
\saveTG{魷}{23312}
\saveTG{鴕}{23312}
\saveTG{鯇}{23312}
\saveTG{𩸨}{23312}
\saveTG{鮀}{23312}
\saveTG{𪄓}{23314}
\saveTG{䳹}{23314}
\saveTG{𪄺}{23314}
\saveTG{鰘}{23314}
\saveTG{鰚}{23316}
\saveTG{𪃗}{23316}
\saveTG{𩶱}{23317}
\saveTG{𪂊}{23317}
\saveTG{𪁒}{23317}
\saveTG{𩸩}{23317}
\saveTG{𩶂}{23321}
\saveTG{鯵}{23322}
\saveTG{鰺}{23322}
\saveTG{𩼶}{23322}
\saveTG{𪁭}{23327}
\saveTG{𩣨}{23327}
\saveTG{𩺹}{23327}
\saveTG{鯿}{23327}
\saveTG{鯆}{23327}
\saveTG{鵞}{23327}
\saveTG{黛}{23331}
\saveTG{𢞤}{23332}
\saveTG{𩷕}{23332}
\saveTG{𩺼}{23332}
\saveTG{𩼆}{23332}
\saveTG{𩶙}{23332}
\saveTG{𢘋}{23334}
\saveTG{鱤}{23335}
\saveTG{𩶕}{23336}
\saveTG{𫙵}{23336}
\saveTG{怠}{23336}
\saveTG{𩻦}{23338}
\saveTG{然}{23338}
\saveTG{𠫵}{23338}
\saveTG{𩹧}{23338}
\saveTG{𤋣}{23338}
\saveTG{𢢃}{23338}
\saveTG{叅}{23338}
\saveTG{鮘}{23340}
\saveTG{𢎓}{23340}
\saveTG{鴏}{23340}
\saveTG{𪀸}{23341}
\saveTG{𩹲}{23343}
\saveTG{鮟}{23344}
\saveTG{𩹔}{23347}
\saveTG{𩼵}{23347}
\saveTG{鵔}{23347}
\saveTG{鮻}{23347}
\saveTG{鮁}{23347}
\saveTG{𢨓}{23350}
\saveTG{𫚂}{23350}
\saveTG{𢧏}{23350}
\saveTG{𢧝}{23350}
\saveTG{䰹}{23350}
\saveTG{鱡}{23350}
\saveTG{𩿙}{23350}
\saveTG{鯎}{23350}
\saveTG{鰔}{23350}
\saveTG{鰄}{23350}
\saveTG{𩻅}{23351}
\saveTG{𩽅}{23351}
\saveTG{䱛}{23351}
\saveTG{䱠}{23353}
\saveTG{𩽠}{23354}
\saveTG{𫙺}{23354}
\saveTG{𪀚}{23354}
\saveTG{𩶺}{23354}
\saveTG{𪁫}{23354}
\saveTG{䳘}{23355}
\saveTG{𩷦}{23355}
\saveTG{𩿠}{23357}
\saveTG{鮐}{23360}
\saveTG{𪆥}{23361}
\saveTG{𩿡}{23361}
\saveTG{𫙴}{23361}
\saveTG{𩹃}{23364}
\saveTG{鯦}{23364}
\saveTG{鰫}{23368}
\saveTG{𪁷}{23372}
\saveTG{𩸘}{23377}
\saveTG{鴥}{23382}
\saveTG{𩽜}{23382}
\saveTG{𫙕}{23382}
\saveTG{𩸎}{23382}
\saveTG{𩵥}{23384}
\saveTG{𩹏}{23384}
\saveTG{𩿁}{23384}
\saveTG{鮲}{23384}
\saveTG{𩻿}{23384}
\saveTG{𪆧}{23385}
\saveTG{𩼧}{23386}
\saveTG{鯮}{23391}
\saveTG{𩶄}{23394}
\saveTG{鯄}{23399}
\saveTG{处}{23400}
\saveTG{𨈒}{23400}
\saveTG{𠫘}{23400}
\saveTG{𪟿}{23400}
\saveTG{发}{23407}
\saveTG{舮}{23407}
\saveTG{夋}{23407}
\saveTG{𡥖}{23407}
\saveTG{𦩓}{23411}
\saveTG{𡴞}{23412}
\saveTG{舵}{23412}
\saveTG{𨿧}{23415}
\saveTG{𠫜}{23417}
\saveTG{𡹿}{23417}
\saveTG{𡻏}{23417}
\saveTG{䑱}{23417}
\saveTG{𨉝}{23417}
\saveTG{𦪻}{23417}
\saveTG{𨈷}{23417}
\saveTG{𨈣}{23417}
\saveTG{𦨩}{23417}
\saveTG{𫏫}{23417}
\saveTG{𦪌}{23421}
\saveTG{𨊓}{23421}
\saveTG{𨊎}{23421}
\saveTG{𨉗}{23424}
\saveTG{艑}{23427}
\saveTG{𫆶}{23427}
\saveTG{䑷}{23427}
\saveTG{䠻}{23427}
\saveTG{艆}{23432}
\saveTG{𢽂}{23432}
\saveTG{𦨬}{23432}
\saveTG{𦪳}{23432}
\saveTG{𦪜}{23438}
\saveTG{弁}{23440}
\saveTG{𦨒}{23440}
\saveTG{𢌹}{23441}
\saveTG{㛷}{23441}
\saveTG{𦨮}{23442}
\saveTG{𦤚}{23444}
\saveTG{𠦷}{23444}
\saveTG{𨈩}{23444}
\saveTG{𦩇}{23447}
\saveTG{𦪁}{23447}
\saveTG{䑹}{23447}
\saveTG{𠬎}{23447}
\saveTG{𡜀}{23448}
\saveTG{𨈟}{23450}
\saveTG{𦨜}{23450}
\saveTG{𢧛}{23450}
\saveTG{㦸}{23450}
\saveTG{艥}{23450}
\saveTG{𦨷}{23450}
\saveTG{𨉧}{23452}
\saveTG{𤛙}{23452}
\saveTG{𪟷}{23453}
\saveTG{𦩬}{23454}
\saveTG{𨉐}{23455}
\saveTG{𦩆}{23455}
\saveTG{𦩢}{23456}
\saveTG{𦪊}{23459}
\saveTG{𦪼}{23461}
\saveTG{𨊊}{23461}
\saveTG{𦪥}{23462}
\saveTG{䑿}{23462}
\saveTG{䒇}{23466}
\saveTG{𨉷}{23468}
\saveTG{𫇍}{23469}
\saveTG{𦨚}{23480}
\saveTG{𦩘}{23482}
\saveTG{𦩤}{23484}
\saveTG{𫆆}{23484}
\saveTG{𦩈}{23484}
\saveTG{䑸}{23491}
\saveTG{牟}{23500}
\saveTG{𠫚}{23501}
\saveTG{牮}{23504}
\saveTG{𠫹}{23506}
\saveTG{𨋿}{23506}
\saveTG{𠫺}{23506}
\saveTG{𨏾}{23506}
\saveTG{𠫿}{23506}
\saveTG{軬}{23508}
\saveTG{牻}{23514}
\saveTG{魃}{23514}
\saveTG{魆}{23515}
\saveTG{魊}{23515}
\saveTG{𤚗}{23516}
\saveTG{㸰}{23517}
\saveTG{𤘺}{23517}
\saveTG{𤘜}{23517}
\saveTG{𪺗}{23517}
\saveTG{𪺪}{23517}
\saveTG{犙}{23522}
\saveTG{犏}{23527}
\saveTG{𤙭}{23527}
\saveTG{𤙋}{23527}
\saveTG{𤜁}{23535}
\saveTG{𤛸}{23535}
\saveTG{𤛤}{23538}
\saveTG{𤘚}{23540}
\saveTG{𤚽}{23543}
\saveTG{牸}{23547}
\saveTG{𤚔}{23547}
\saveTG{㹑}{23548}
\saveTG{我}{23550}
\saveTG{牫}{23550}
\saveTG{𤚅}{23551}
\saveTG{𣫽}{23557}
\saveTG{犗}{23565}
\saveTG{㸻}{23584}
\saveTG{𤘲}{23584}
\saveTG{𤙷}{23591}
\saveTG{𤙠}{23599}
\saveTG{台}{23600}
\saveTG{𠧨}{23600}
\saveTG{𥑼}{23601}
\saveTG{𥡳}{23601}
\saveTG{𤱊}{23601}
\saveTG{𥡞}{23601}
\saveTG{𫌸}{23601}
\saveTG{諬}{23601}
\saveTG{䀤}{23602}
\saveTG{𤲙}{23603}
\saveTG{𠊝}{23603}
\saveTG{馝}{23604}
\saveTG{𨠰}{23604}
\saveTG{𨠒}{23604}
\saveTG{𨠢}{23604}
\saveTG{𠰰}{23604}
\saveTG{咎}{23604}
\saveTG{𥢑}{23604}
\saveTG{昝}{23604}
\saveTG{𡘞}{23605}
\saveTG{畚}{23608}
\saveTG{𤲛}{23608}
\saveTG{皖}{23612}
\saveTG{夞}{23612}
\saveTG{𦤤}{23615}
\saveTG{𠫮}{23617}
\saveTG{𤾂}{23617}
\saveTG{𤾱}{23621}
\saveTG{𤲑}{23621}
\saveTG{𪊄}{23635}
\saveTG{𤽭}{23647}
\saveTG{𫇊}{23647}
\saveTG{𪉼}{23647}
\saveTG{𥣣}{23647}
\saveTG{馛}{23647}
\saveTG{𠼷}{23650}
\saveTG{𢦯}{23650}
\saveTG{战}{23650}
\saveTG{鹹}{23650}
\saveTG{皒}{23650}
\saveTG{𪉠}{23650}
\saveTG{𢧽}{23650}
\saveTG{𪾀}{23651}
\saveTG{𤽷}{23652}
\saveTG{𤳀}{23653}
\saveTG{馢}{23653}
\saveTG{𢧗}{23653}
\saveTG{𦧩}{23656}
\saveTG{𢨟}{23656}
\saveTG{𦧞}{23662}
\saveTG{𩡔}{23665}
\saveTG{𪿷}{23667}
\saveTG{𤠪}{23684}
\saveTG{𤝡}{23684}
\saveTG{𩡍}{23684}
\saveTG{𫘄}{23686}
\saveTG{馪}{23686}
\saveTG{𤛡}{23693}
\saveTG{𫗣}{23704}
\saveTG{㠧}{23711}
\saveTG{𣭈}{23711}
\saveTG{𣭍}{23711}
\saveTG{}{23712}
\saveTG{𣭾}{23712}
\saveTG{崆}{23712}
\saveTG{毶}{23712}
\saveTG{𡹠}{23712}
\saveTG{𪘲}{23712}
\saveTG{㲑}{23712}
\saveTG{𣯶}{23712}
\saveTG{𪘫}{23712}
\saveTG{岮}{23712}
\saveTG{𣰂}{23713}
\saveTG{𡻜}{23714}
\saveTG{𣮬}{23714}
\saveTG{𦫘}{23714}
\saveTG{𪙜}{23714}
\saveTG{𤚣}{23714}
\saveTG{𣰷}{23715}
\saveTG{毪}{23715}
\saveTG{𣮏}{23715}
\saveTG{毧}{23715}
\saveTG{𪩐}{23715}
\saveTG{㲓}{23715}
\saveTG{𣬽}{23715}
\saveTG{奙}{23716}
\saveTG{𡺟}{23716}
\saveTG{𣯔}{23716}
\saveTG{𣭆}{23716}
\saveTG{𪗩}{23717}
\saveTG{𡵔}{23717}
\saveTG{𡷗}{23717}
\saveTG{䦾}{23717}
\saveTG{㟌}{23717}
\saveTG{𡼌}{23717}
\saveTG{𦩐}{23717}
\saveTG{𪵼}{23717}
\saveTG{𡸥}{23717}
\saveTG{𪗴}{23717}
\saveTG{𠫙}{23718}
\saveTG{𣭩}{23718}
\saveTG{𪵢}{23718}
\saveTG{𣮤}{23719}
\saveTG{𣰤}{23719}
\saveTG{毬}{23719}
\saveTG{𫜧}{23721}
\saveTG{㠁}{23722}
\saveTG{𫜫}{23727}
\saveTG{𡸰}{23727}
\saveTG{𡺂}{23727}
\saveTG{峬}{23727}
\saveTG{𡸕}{23727}
\saveTG{𡾈}{23727}
\saveTG{𫜦}{23727}
\saveTG{𫗦}{23727}
\saveTG{𠫬}{23731}
\saveTG{𪘋}{23731}
\saveTG{𠬀}{23731}
\saveTG{𠫛}{23731}
\saveTG{㕕}{23731}
\saveTG{㟍}{23732}
\saveTG{𫗨}{23732}
\saveTG{袋}{23732}
\saveTG{峵}{23732}
\saveTG{厽}{23732}
\saveTG{峅}{23740}
\saveTG{𢎕}{23740}
\saveTG{𡷎}{23741}
\saveTG{馎}{23742}
\saveTG{𪘑}{23742}
\saveTG{𪧡}{23743}
\saveTG{𪚂}{23743}
\saveTG{𪙍}{23743}
\saveTG{𪩉}{23743}
\saveTG{𡶖}{23744}
\saveTG{峖}{23744}
\saveTG{馂}{23747}
\saveTG{峻}{23747}
\saveTG{㟊}{23747}
\saveTG{饿}{23750}
\saveTG{峸}{23750}
\saveTG{𢧰}{23750}
\saveTG{𪩔}{23750}
\saveTG{𡶙}{23750}
\saveTG{䶪}{23750}
\saveTG{𪗠}{23750}
\saveTG{巇}{23750}
\saveTG{饯}{23750}
\saveTG{峨}{23750}
\saveTG{𪙻}{23750}
\saveTG{𡽴}{23751}
\saveTG{𡾟}{23751}
\saveTG{𡽱}{23751}
\saveTG{㟉}{23752}
\saveTG{𪘪}{23752}
\saveTG{㟞}{23753}
\saveTG{𨹇}{23753}
\saveTG{𪘐}{23755}
\saveTG{䶢}{23756}
\saveTG{㞽}{23757}
\saveTG{饴}{23760}
\saveTG{齝}{23760}
\saveTG{𡺀}{23764}
\saveTG{𪘺}{23764}
\saveTG{𫜯}{23765}
\saveTG{𪙏}{23765}
\saveTG{嵱}{23768}
\saveTG{𠙴}{23770}
\saveTG{岱}{23772}
\saveTG{嵆}{23772}
\saveTG{齾}{23772}
\saveTG{𡼿}{23772}
\saveTG{𡺴}{23772}
\saveTG{𪘕}{23772}
\saveTG{𦉧}{23774}
\saveTG{馆}{23777}
\saveTG{𠫻}{23778}
\saveTG{𡽆}{23781}
\saveTG{𩠆}{23782}
\saveTG{𡾰}{23782}
\saveTG{岤}{23782}
\saveTG{𥦶}{23782}
\saveTG{㟮}{23784}
\saveTG{𠤘}{23784}
\saveTG{巘}{23784}
\saveTG{𪚋}{23784}
\saveTG{𪩘}{23784}
\saveTG{𤟀}{23784}
\saveTG{𡸶}{23791}
\saveTG{㟈}{23799}
\saveTG{𠔰}{23801}
\saveTG{𠔩}{23801}
\saveTG{贠}{23802}
\saveTG{贷}{23802}
\saveTG{𠫯}{23802}
\saveTG{𨀑}{23802}
\saveTG{矣}{23804}
\saveTG{𡘾}{23804}
\saveTG{貸}{23806}
\saveTG{𧶣}{23806}
\saveTG{𧸾}{23806}
\saveTG{𥢅}{23806}
\saveTG{𧴵}{23806}
\saveTG{𧵏}{23806}
\saveTG{貵}{23806}
\saveTG{貟}{23806}
\saveTG{𤆃}{23809}
\saveTG{𤒝}{23809}
\saveTG{𤊜}{23809}
\saveTG{𤇰}{23809}
\saveTG{𤉷}{23809}
\saveTG{炱}{23809}
\saveTG{𦫑}{23832}
\saveTG{𠫪}{23845}
\saveTG{𢧾}{23850}
\saveTG{𤟙}{23884}
\saveTG{𤝀}{23884}
\saveTG{𥏑}{23887}
\saveTG{𥝒}{23900}
\saveTG{𫃚}{23900}
\saveTG{𥿂}{23900}
\saveTG{𥾠}{23900}
\saveTG{𥾾}{23900}
\saveTG{厼}{23900}
\saveTG{𥿝}{23903}
\saveTG{絫}{23903}
\saveTG{𥠑}{23904}
\saveTG{𠫡}{23904}
\saveTG{𥟲}{23904}
\saveTG{𪏺}{23904}
\saveTG{𣑡}{23904}
\saveTG{𣗆}{23904}
\saveTG{秘}{23904}
\saveTG{柋}{23904}
\saveTG{枲}{23904}
\saveTG{䋆}{23907}
\saveTG{𠫭}{23908}
\saveTG{𥡁}{23908}
\saveTG{㕘}{23908}
\saveTG{𫃰}{23911}
\saveTG{紌}{23912}
\saveTG{𦁈}{23912}
\saveTG{𥢱}{23912}
\saveTG{綄}{23912}
\saveTG{紽}{23912}
\saveTG{綩}{23912}
\saveTG{𥟚}{23914}
\saveTG{𥡫}{23914}
\saveTG{𦄔}{23914}
\saveTG{秺}{23914}
\saveTG{縇}{23916}
\saveTG{𥟶}{23917}
\saveTG{𦆼}{23917}
\saveTG{𥞒}{23917}
\saveTG{𥝲}{23917}
\saveTG{𫀶}{23917}
\saveTG{𥡔}{23917}
\saveTG{𥣗}{23921}
\saveTG{𥢃}{23921}
\saveTG{䅝}{23921}
\saveTG{紵}{23921}
\saveTG{𦆭}{23921}
\saveTG{縿}{23922}
\saveTG{穇}{23922}
\saveTG{䅟}{23922}
\saveTG{𠬊}{23922}
\saveTG{}{23922}
\saveTG{稨}{23927}
\saveTG{秿}{23927}
\saveTG{𦂪}{23927}
\saveTG{編}{23927}
\saveTG{䋠}{23927}
\saveTG{𥡨}{23927}
\saveTG{𦁊}{23930}
\saveTG{綋}{23931}
\saveTG{𦆥}{23931}
\saveTG{稂}{23932}
\saveTG{稼}{23932}
\saveTG{𦃲}{23932}
\saveTG{綋}{23932}
\saveTG{𦀬}{23932}
\saveTG{繈}{23936}
\saveTG{緿}{23936}
\saveTG{繎}{23938}
\saveTG{𦅒}{23938}
\saveTG{𥠩}{23938}
\saveTG{䄩}{23940}
\saveTG{𥾐}{23940}
\saveTG{縡}{23941}
\saveTG{𫄀}{23941}
\saveTG{𥿮}{23941}
\saveTG{縛}{23942}
\saveTG{𥿾}{23943}
\saveTG{㕖}{23943}
\saveTG{𥠵}{23943}
\saveTG{𥿋}{23944}
\saveTG{𥞬}{23944}
\saveTG{𥿽}{23944}
\saveTG{𥢔}{23946}
\saveTG{𫃣}{23947}
\saveTG{𦀷}{23947}
\saveTG{𦄎}{23947}
\saveTG{秡}{23947}
\saveTG{紱}{23947}
\saveTG{稄}{23947}
\saveTG{𥿵}{23948}
\saveTG{𫄅}{23950}
\saveTG{𥣩}{23950}
\saveTG{𥾮}{23950}
\saveTG{𦃎}{23950}
\saveTG{絾}{23950}
\saveTG{縬}{23950}
\saveTG{緘}{23950}
\saveTG{纎}{23950}
\saveTG{𢧡}{23950}
\saveTG{𢧍}{23950}
\saveTG{緎}{23950}
\saveTG{稶}{23950}
\saveTG{絨}{23950}
\saveTG{稢}{23950}
\saveTG{繊}{23950}
\saveTG{縅}{23950}
\saveTG{纖}{23950}
\saveTG{𢧼}{23950}
\saveTG{織}{23950}
\saveTG{𦄤}{23951}
\saveTG{𦅞}{23953}
\saveTG{䋐}{23953}
\saveTG{綫}{23953}
\saveTG{𥟥}{23953}
\saveTG{𢧹}{23954}
\saveTG{䄾}{23954}
\saveTG{𦄅}{23954}
\saveTG{𫄇}{23955}
\saveTG{𥢧}{23956}
\saveTG{𦄩}{23956}
\saveTG{𥠆}{23956}
\saveTG{紿}{23960}
\saveTG{秮}{23960}
\saveTG{𥠻}{23961}
\saveTG{䆌}{23961}
\saveTG{𥡴}{23961}
\saveTG{稽}{23961}
\saveTG{𦂨}{23962}
\saveTG{縮}{23962}
\saveTG{綹}{23964}
\saveTG{𦂦}{23964}
\saveTG{𦄲}{23964}
\saveTG{𦤬}{23965}
\saveTG{𦁖}{23965}
\saveTG{𠫩}{23965}
\saveTG{縖}{23965}
\saveTG{穃}{23968}
\saveTG{𨤛}{23968}
\saveTG{𫃻}{23968}
\saveTG{嵇}{23972}
\saveTG{䌏}{23972}
\saveTG{綰}{23977}
\saveTG{𥟓}{23977}
\saveTG{綻}{23981}
\saveTG{𥟐}{23982}
\saveTG{䋉}{23982}
\saveTG{綟}{23984}
\saveTG{𦂽}{23984}
\saveTG{紎}{23984}
\saveTG{絥}{23984}
\saveTG{𦅲}{23985}
\saveTG{繽}{23986}
\saveTG{𦆯}{23986}
\saveTG{縯}{23986}
\saveTG{穦}{23986}
\saveTG{綜}{23991}
\saveTG{𥟂}{23992}
\saveTG{𦅍}{23993}
\saveTG{絉}{23994}
\saveTG{𦂅}{23994}
\saveTG{秫}{23994}
\saveTG{𥣝}{23994}
\saveTG{𥟇}{23999}
\saveTG{絿}{23999}
\saveTG{𠄓}{24007}
\saveTG{𤗿}{24014}
\saveTG{𤗎}{24017}
\saveTG{𦁏}{24017}
\saveTG{𤖪}{24017}
\saveTG{𠠲}{24027}
\saveTG{帅}{24027}
\saveTG{㸢}{24027}
\saveTG{𤘁}{24032}
\saveTG{𤖩}{24040}
\saveTG{𤘀}{24043}
\saveTG{𤖷}{24047}
\saveTG{𤖲}{24060}
\saveTG{㟙}{24061}
\saveTG{𪺦}{24061}
\saveTG{𤗼}{24061}
\saveTG{牍}{24084}
\saveTG{𤘄}{24086}
\saveTG{牘}{24086}
\saveTG{𤗸}{24086}
\saveTG{𤗶}{24086}
\saveTG{𤗥}{24094}
\saveTG{𤗽}{24094}
\saveTG{牒}{24094}
\saveTG{鳚}{24100}
\saveTG{鲥}{24100}
\saveTG{𠂒}{24100}
\saveTG{鲋}{24100}
\saveTG{}{24100}
\saveTG{纣}{24100}
\saveTG{𩵝}{24102}
\saveTG{𥁒}{24102}
\saveTG{𥃊}{24102}
\saveTG{𪉶}{24102}
\saveTG{𣥲}{24102}
\saveTG{䌶}{24102}
\saveTG{𤯘}{24103}
\saveTG{𡋩}{24104}
\saveTG{𡉼}{24104}
\saveTG{𡌴}{24104}
\saveTG{𡐏}{24104}
\saveTG{銺}{24109}
\saveTG{𨧭}{24109}
\saveTG{靠}{24111}
\saveTG{𧗀}{24111}
\saveTG{}{24112}
\saveTG{豔}{24112}
\saveTG{豓}{24112}
\saveTG{鲑}{24114}
\saveTG{㓐}{24114}
\saveTG{绁}{24117}
\saveTG{𩻕}{24117}
\saveTG{𥝀}{24117}
\saveTG{绮}{24121}
\saveTG{纳}{24127}
\saveTG{鲔}{24127}
\saveTG{𦠫}{24127}
\saveTG{㔞}{24127}
\saveTG{𤯺}{24127}
\saveTG{𠡏}{24127}
\saveTG{𧖩}{24127}
\saveTG{䌾}{24127}
\saveTG{䚚}{24127}
\saveTG{𫄨}{24127}
\saveTG{動}{24127}
\saveTG{鲓}{24127}
\saveTG{绔}{24127}
\saveTG{鳓}{24127}
\saveTG{𡬪}{24130}
\saveTG{𤯻}{24132}
\saveTG{纮}{24132}
\saveTG{鲢}{24135}
\saveTG{𦈐}{24135}
\saveTG{𧑯}{24136}
\saveTG{𧎬}{24136}
\saveTG{𧎭}{24136}
\saveTG{𧖟}{24136}
\saveTG{𧎗}{24136}
\saveTG{𧕤}{24136}
\saveTG{𧊆}{24136}
\saveTG{𧍓}{24136}
\saveTG{𫄤}{24139}
\saveTG{𤯕}{24140}
\saveTG{歭}{24141}
\saveTG{𫚘}{24142}
\saveTG{𣥻}{24142}
\saveTG{鹱}{24147}
\saveTG{𤿥}{24147}
\saveTG{𣥣}{24147}
\saveTG{鳠}{24147}
\saveTG{𤯙}{24147}
\saveTG{㩾}{24147}
\saveTG{𢻨}{24147}
\saveTG{𢻩}{24147}
\saveTG{𢻞}{24147}
\saveTG{䵾}{24147}
\saveTG{皱}{24147}
\saveTG{歧}{24147}
\saveTG{鲮}{24147}
\saveTG{绫}{24147}
\saveTG{鲏}{24147}
\saveTG{衊}{24153}
\saveTG{缂}{24156}
\saveTG{绪}{24160}
\saveTG{鲒}{24161}
\saveTG{鳍}{24161}
\saveTG{结}{24161}
\saveTG{}{24161}
\saveTG{𫄱}{24161}
\saveTG{𦈘}{24161}
\saveTG{𣦈}{24163}
\saveTG{𣦡}{24164}
\saveTG{𤾑}{24170}
\saveTG{绀}{24170}
\saveTG{鲯}{24181}
\saveTG{缜}{24181}
\saveTG{鲼}{24182}
\saveTG{缵}{24182}
\saveTG{续}{24184}
\saveTG{绬}{24185}
\saveTG{𪏓}{24186}
\saveTG{𠗉}{24188}
\saveTG{}{24190}
\saveTG{𦈟}{24191}
\saveTG{鲽}{24194}
\saveTG{𪤘}{24194}
\saveTG{𫄬}{24194}
\saveTG{缭}{24196}
\saveTG{𣦨}{24198}
\saveTG{𨭁}{24199}
\saveTG{𢔿}{24200}
\saveTG{傠}{24200}
\saveTG{豺}{24200}
\saveTG{射}{24200}
\saveTG{斛}{24200}
\saveTG{㚈}{24200}
\saveTG{𢒾}{24200}
\saveTG{什}{24200}
\saveTG{𠦣}{24200}
\saveTG{斘}{24200}
\saveTG{付}{24200}
\saveTG{𠆫}{24202}
\saveTG{𧆯}{24202}
\saveTG{𧤳}{24202}
\saveTG{𪖘}{24203}
\saveTG{𧤰}{24203}
\saveTG{𧲣}{24203}
\saveTG{𠏮}{24203}
\saveTG{𠆱}{24203}
\saveTG{仪}{24203}
\saveTG{㐼}{24204}
\saveTG{𠉋}{24210}
\saveTG{化}{24210}
\saveTG{仕}{24210}
\saveTG{壯}{24210}
\saveTG{𦘳}{24210}
\saveTG{𡉟}{24210}
\saveTG{𢓂}{24210}
\saveTG{䰫}{24211}
\saveTG{𩳢}{24211}
\saveTG{𩴗}{24211}
\saveTG{𪟟}{24212}
\saveTG{值}{24212}
\saveTG{兟}{24212}
\saveTG{㓄}{24212}
\saveTG{𧈋}{24212}
\saveTG{𡖠}{24212}
\saveTG{躭}{24212}
\saveTG{他}{24212}
\saveTG{侁}{24212}
\saveTG{佬}{24212}
\saveTG{𩳣}{24212}
\saveTG{𠣈}{24212}
\saveTG{𠎴}{24212}
\saveTG{𪝙}{24212}
\saveTG{彵}{24212}
\saveTG{𩱹}{24212}
\saveTG{𤜣}{24212}
\saveTG{𤖙}{24212}
\saveTG{先}{24212}
\saveTG{徺}{24212}
\saveTG{僥}{24212}
\saveTG{𧲭}{24212}
\saveTG{徝}{24212}
\saveTG{佐}{24212}
\saveTG{𩲅}{24213}
\saveTG{𣁽}{24213}
\saveTG{𠈚}{24214}
\saveTG{𡖪}{24214}
\saveTG{儓}{24214}
\saveTG{佳}{24214}
\saveTG{𩴴}{24214}
\saveTG{𩲢}{24214}
\saveTG{㒗}{24214}
\saveTG{𤢬}{24214}
\saveTG{𧈙}{24214}
\saveTG{𤞇}{24214}
\saveTG{𠏠}{24214}
\saveTG{𪖢}{24214}
\saveTG{徍}{24214}
\saveTG{觟}{24214}
\saveTG{𩴾}{24215}
\saveTG{𩳷}{24215}
\saveTG{貛}{24215}
\saveTG{僅}{24215}
\saveTG{䚭}{24215}
\saveTG{傕}{24215}
\saveTG{𩳸}{24216}
\saveTG{𧆻}{24216}
\saveTG{𩲱}{24216}
\saveTG{𩴀}{24216}
\saveTG{𧴒}{24216}
\saveTG{俺}{24216}
\saveTG{齄}{24216}
\saveTG{値}{24216}
\saveTG{𢓧}{24217}
\saveTG{𡖐}{24217}
\saveTG{𠇳}{24217}
\saveTG{𩵉}{24217}
\saveTG{仇}{24217}
\saveTG{甝}{24217}
\saveTG{鼽}{24217}
\saveTG{伳}{24217}
\saveTG{𧣁}{24217}
\saveTG{𢔂}{24217}
\saveTG{𧳾}{24217}
\saveTG{𧆟}{24217}
\saveTG{𢔞}{24217}
\saveTG{𧢶}{24217}
\saveTG{𤕿}{24217}
\saveTG{𡗊}{24217}
\saveTG{𡖓}{24217}
\saveTG{𡖬}{24217}
\saveTG{𤞯}{24217}
\saveTG{𠓙}{24217}
\saveTG{𢓠}{24217}
\saveTG{𠆶}{24217}
\saveTG{𧡊}{24217}
\saveTG{𠑬}{24217}
\saveTG{𩲜}{24217}
\saveTG{𤕮}{24217}
\saveTG{𣦲}{24218}
\saveTG{𦝧}{24218}
\saveTG{𧹱}{24218}
\saveTG{𩵆}{24218}
\saveTG{𩴓}{24218}
\saveTG{𠊪}{24218}
\saveTG{𠍼}{24218}
\saveTG{偡}{24218}
\saveTG{𩴤}{24219}
\saveTG{𩲙}{24219}
\saveTG{𠇾}{24219}
\saveTG{𩴊}{24219}
\saveTG{𩳘}{24219}
\saveTG{𩲺}{24219}
\saveTG{𠍆}{24220}
\saveTG{觭}{24221}
\saveTG{躸}{24221}
\saveTG{㑸}{24221}
\saveTG{䝝}{24221}
\saveTG{倚}{24221}
\saveTG{徛}{24221}
\saveTG{𧲡}{24224}
\saveTG{豽}{24227}
\saveTG{儰}{24227}
\saveTG{俙}{24227}
\saveTG{勨}{24227}
\saveTG{俲}{24227}
\saveTG{倄}{24227}
\saveTG{侑}{24227}
\saveTG{偧}{24227}
\saveTG{𧱑}{24227}
\saveTG{𪤷}{24227}
\saveTG{𠢼}{24227}
\saveTG{䝡}{24227}
\saveTG{𧴝}{24227}
\saveTG{㔧}{24227}
\saveTG{𠢍}{24227}
\saveTG{𫎋}{24227}
\saveTG{𦚛}{24227}
\saveTG{𧤦}{24227}
\saveTG{觔}{24227}
\saveTG{𧃪}{24227}
\saveTG{𡖎}{24227}
\saveTG{𠡞}{24227}
\saveTG{㣮}{24227}
\saveTG{𫚸}{24227}
\saveTG{𠡢}{24227}
\saveTG{𠡝}{24227}
\saveTG{㒖}{24227}
\saveTG{𠑧}{24227}
\saveTG{𠏆}{24227}
\saveTG{𪝆}{24227}
\saveTG{𠡬}{24227}
\saveTG{㑲}{24227}
\saveTG{𪜾}{24227}
\saveTG{𦝭}{24227}
\saveTG{㑃}{24227}
\saveTG{𠌦}{24227}
\saveTG{𤖐}{24227}
\saveTG{𫌮}{24227}
\saveTG{𧤓}{24227}
\saveTG{𡖮}{24227}
\saveTG{𢓢}{24227}
\saveTG{𩱕}{24227}
\saveTG{𩱩}{24227}
\saveTG{𠎐}{24227}
\saveTG{𠊃}{24227}
\saveTG{𠉾}{24227}
\saveTG{𠋮}{24227}
\saveTG{𠇴}{24227}
\saveTG{𠑃}{24227}
\saveTG{𦚘}{24227}
\saveTG{侤}{24227}
\saveTG{𧳐}{24227}
\saveTG{𧣩}{24227}
\saveTG{𪖥}{24227}
\saveTG{𢓬}{24227}
\saveTG{𠍢}{24227}
\saveTG{𪝄}{24227}
\saveTG{𨿛}{24227}
\saveTG{𧤌}{24227}
\saveTG{𠊠}{24227}
\saveTG{𠉈}{24227}
\saveTG{𠋚}{24227}
\saveTG{𦝍}{24227}
\saveTG{備}{24227}
\saveTG{僃}{24227}
\saveTG{佈}{24227}
\saveTG{僀}{24227}
\saveTG{働}{24227}
\saveTG{伪}{24227}
\saveTG{侉}{24227}
\saveTG{偽}{24227}
\saveTG{勪}{24227}
\saveTG{勮}{24227}
\saveTG{仂}{24227}
\saveTG{勴}{24227}
\saveTG{儚}{24227}
\saveTG{𠐆}{24230}
\saveTG{𪜺}{24230}
\saveTG{俧}{24231}
\saveTG{䏻}{24231}
\saveTG{𪫍}{24231}
\saveTG{𪜻}{24231}
\saveTG{𠊁}{24231}
\saveTG{𠆽}{24231}
\saveTG{𠏖}{24231}
\saveTG{𢕉}{24231}
\saveTG{觾}{24231}
\saveTG{𠏽}{24231}
\saveTG{德}{24231}
\saveTG{𠐷}{24231}
\saveTG{侬}{24232}
\saveTG{𪝘}{24232}
\saveTG{徔}{24232}
\saveTG{𤄳}{24232}
\saveTG{𠏍}{24232}
\saveTG{佉}{24232}
\saveTG{𠐁}{24232}
\saveTG{𠏾}{24232}
\saveTG{𠑞}{24233}
\saveTG{𠎢}{24233}
\saveTG{𧴟}{24235}
\saveTG{㣵}{24235}
\saveTG{㒓}{24235}
\saveTG{𢕞}{24236}
\saveTG{徳}{24236}
\saveTG{𠌅}{24237}
\saveTG{𠉂}{24238}
\saveTG{𪝸}{24240}
\saveTG{𠏒}{24240}
\saveTG{𢖢}{24240}
\saveTG{𡛝}{24240}
\saveTG{𧤡}{24240}
\saveTG{𠍁}{24240}
\saveTG{㪇}{24240}
\saveTG{𪝅}{24240}
\saveTG{㚢}{24240}
\saveTG{妝}{24240}
\saveTG{侍}{24241}
\saveTG{軇}{24241}
\saveTG{待}{24241}
\saveTG{𢔛}{24241}
\saveTG{𠋽}{24241}
\saveTG{倖}{24241}
\saveTG{𧤱}{24241}
\saveTG{儔}{24241}
\saveTG{偫}{24241}
\saveTG{傇}{24241}
\saveTG{𠢅}{24242}
\saveTG{㣥}{24243}
\saveTG{𠍰}{24243}
\saveTG{倴}{24244}
\saveTG{𧇉}{24246}
\saveTG{𧇡}{24246}
\saveTG{𧤣}{24246}
\saveTG{𥀄}{24247}
\saveTG{𠍅}{24247}
\saveTG{彼}{24247}
\saveTG{㩼}{24247}
\saveTG{𢻤}{24247}
\saveTG{䰵}{24247}
\saveTG{𠑦}{24247}
\saveTG{𠍤}{24247}
\saveTG{侼}{24247}
\saveTG{㩺}{24247}
\saveTG{䝛}{24247}
\saveTG{𪤻}{24247}
\saveTG{𥀵}{24247}
\saveTG{𧤁}{24247}
\saveTG{𢔁}{24247}
\saveTG{𪌨}{24247}
\saveTG{𤿡}{24247}
\saveTG{佊}{24247}
\saveTG{皻}{24247}
\saveTG{侟}{24247}
\saveTG{倰}{24247}
\saveTG{伖}{24247}
\saveTG{侾}{24247}
\saveTG{伎}{24247}
\saveTG{𤿯}{24247}
\saveTG{𪖞}{24247}
\saveTG{𥀃}{24247}
\saveTG{𠍑}{24247}
\saveTG{𠐎}{24247}
\saveTG{㩻}{24247}
\saveTG{𣀁}{24247}
\saveTG{𧣄}{24247}
\saveTG{𤡶}{24248}
\saveTG{伡}{24250}
\saveTG{𦝯}{24251}
\saveTG{𠌾}{24253}
\saveTG{㒝}{24253}
\saveTG{𧤻}{24254}
\saveTG{𠎑}{24254}
\saveTG{}{24256}
\saveTG{𠍔}{24256}
\saveTG{𧤒}{24256}
\saveTG{𠌙}{24256}
\saveTG{偉}{24256}
\saveTG{徫}{24256}
\saveTG{𩏣}{24257}
\saveTG{𫌱}{24257}
\saveTG{估}{24260}
\saveTG{觰}{24260}
\saveTG{儲}{24260}
\saveTG{储}{24260}
\saveTG{偖}{24260}
\saveTG{𤖧}{24260}
\saveTG{㑬}{24260}
\saveTG{𠉊}{24260}
\saveTG{㑤}{24260}
\saveTG{㣨}{24260}
\saveTG{𧙗}{24260}
\saveTG{𧙖}{24260}
\saveTG{𣧮}{24260}
\saveTG{佑}{24260}
\saveTG{貓}{24260}
\saveTG{俈}{24261}
\saveTG{佶}{24261}
\saveTG{𤖠}{24261}
\saveTG{徣}{24261}
\saveTG{𪝦}{24261}
\saveTG{𠑏}{24261}
\saveTG{𪝌}{24261}
\saveTG{𠎲}{24261}
\saveTG{借}{24261}
\saveTG{𠏼}{24261}
\saveTG{㣟}{24261}
\saveTG{牆}{24261}
\saveTG{僖}{24261}
\saveTG{𠎸}{24261}
\saveTG{夡}{24261}
\saveTG{𧣗}{24262}
\saveTG{𧳯}{24263}
\saveTG{偌}{24264}
\saveTG{𧤮}{24264}
\saveTG{㒂}{24264}
\saveTG{𢔪}{24264}
\saveTG{𧤺}{24264}
\saveTG{𠍽}{24264}
\saveTG{傄}{24268}
\saveTG{𠉉}{24269}
\saveTG{𪖟}{24270}
\saveTG{佄}{24270}
\saveTG{𧣑}{24270}
\saveTG{𠏥}{24280}
\saveTG{㐲}{24280}
\saveTG{㣕}{24280}
\saveTG{徒}{24281}
\saveTG{供}{24281}
\saveTG{傎}{24281}
\saveTG{倛}{24281}
\saveTG{儊}{24281}
\saveTG{𧇫}{24281}
\saveTG{𢅬}{24281}
\saveTG{𡖾}{24281}
\saveTG{偾}{24282}
\saveTG{𠐑}{24282}
\saveTG{䶑}{24282}
\saveTG{𧼌}{24282}
\saveTG{𠏴}{24282}
\saveTG{㑀}{24283}
\saveTG{𠋲}{24283}
\saveTG{㣖}{24283}
\saveTG{𠋋}{24284}
\saveTG{𧴅}{24284}
\saveTG{貘}{24284}
\saveTG{傸}{24284}
\saveTG{𪝡}{24284}
\saveTG{𪝳}{24285}
\saveTG{偀}{24285}
\saveTG{傼}{24285}
\saveTG{僨}{24286}
\saveTG{觵}{24286}
\saveTG{僙}{24286}
\saveTG{𤖘}{24286}
\saveTG{𤖖}{24286}
\saveTG{𪏑}{24286}
\saveTG{𠏱}{24286}
\saveTG{𧴍}{24286}
\saveTG{𢖏}{24286}
\saveTG{儥}{24286}
\saveTG{㣴}{24286}
\saveTG{儹}{24286}
\saveTG{𠋟}{24287}
\saveTG{俠}{24288}
\saveTG{㣣}{24288}
\saveTG{𪖨}{24288}
\saveTG{㷇}{24289}
\saveTG{𠉻}{24289}
\saveTG{𠏗}{24289}
\saveTG{㑣}{24290}
\saveTG{牀}{24290}
\saveTG{休}{24290}
\saveTG{貅}{24290}
\saveTG{䝗}{24290}
\saveTG{㣩}{24290}
\saveTG{倷}{24291}
\saveTG{僸}{24291}
\saveTG{𠐐}{24291}
\saveTG{䖛}{24293}
\saveTG{𠋦}{24294}
\saveTG{偞}{24294}
\saveTG{𧤈}{24294}
\saveTG{僷}{24294}
\saveTG{𠎊}{24294}
\saveTG{牃}{24294}
\saveTG{𢆇}{24294}
\saveTG{䚢}{24294}
\saveTG{㑈}{24294}
\saveTG{僚}{24296}
\saveTG{䝤}{24296}
\saveTG{𧳟}{24298}
\saveTG{䚞}{24298}
\saveTG{𩳳}{24298}
\saveTG{徠}{24298}
\saveTG{倈}{24298}
\saveTG{鮒}{24300}
\saveTG{𫙏}{24300}
\saveTG{𨘴}{24301}
\saveTG{𩼷}{24303}
\saveTG{䲁}{24303}
\saveTG{𣁷}{24303}
\saveTG{𩵬}{24303}
\saveTG{𩿚}{24303}
\saveTG{𨒕}{24303}
\saveTG{𩷺}{24303}
\saveTG{𩸏}{24304}
\saveTG{魤}{24310}
\saveTG{𩵚}{24310}
\saveTG{鱙}{24312}
\saveTG{𫙫}{24312}
\saveTG{𩽭}{24312}
\saveTG{鰪}{24312}
\saveTG{鮱}{24312}
\saveTG{魫}{24312}
\saveTG{𪇫}{24313}
\saveTG{𩶦}{24314}
\saveTG{𪂚}{24314}
\saveTG{鮭}{24314}
\saveTG{鯥}{24314}
\saveTG{鱹}{24315}
\saveTG{𩾙}{24317}
\saveTG{𩵏}{24317}
\saveTG{𩵐}{24317}
\saveTG{𩶤}{24317}
\saveTG{𩽸}{24317}
\saveTG{𥤠}{24317}
\saveTG{𩸽}{24317}
\saveTG{𩸆}{24317}
\saveTG{𩵍}{24317}
\saveTG{𪠟}{24317}
\saveTG{𩺊}{24317}
\saveTG{𩺆}{24317}
\saveTG{𩷶}{24317}
\saveTG{𩸚}{24317}
\saveTG{𩻭}{24318}
\saveTG{𩸞}{24321}
\saveTG{𩽛}{24322}
\saveTG{鳨}{24327}
\saveTG{魶}{24327}
\saveTG{鰖}{24327}
\saveTG{鮪}{24327}
\saveTG{鯑}{24327}
\saveTG{勳}{24327}
\saveTG{𫙽}{24327}
\saveTG{𠢰}{24327}
\saveTG{𩼚}{24327}
\saveTG{𩹩}{24327}
\saveTG{𩼙}{24327}
\saveTG{𩺴}{24327}
\saveTG{𩶉}{24327}
\saveTG{𩺙}{24327}
\saveTG{𩶵}{24327}
\saveTG{𩹞}{24327}
\saveTG{𩷻}{24327}
\saveTG{𩵓}{24327}
\saveTG{䱂}{24327}
\saveTG{𠡻}{24327}
\saveTG{𪀂}{24327}
\saveTG{𠢴}{24327}
\saveTG{䲊}{24327}
\saveTG{𪀪}{24327}
\saveTG{𪂴}{24327}
\saveTG{𩾹}{24327}
\saveTG{𩺌}{24327}
\saveTG{𪇨}{24327}
\saveTG{𩹋}{24327}
\saveTG{鮳}{24327}
\saveTG{鮬}{24327}
\saveTG{鰳}{24327}
\saveTG{𩶩}{24328}
\saveTG{𩺳}{24329}
\saveTG{𩸌}{24329}
\saveTG{怤}{24330}
\saveTG{𩸜}{24331}
\saveTG{𩷓}{24331}
\saveTG{𤍹}{24331}
\saveTG{怹}{24331}
\saveTG{憄}{24331}
\saveTG{㥁}{24331}
\saveTG{𩽒}{24331}
\saveTG{𩼑}{24331}
\saveTG{𩼂}{24332}
\saveTG{𢣍}{24332}
\saveTG{䴌}{24332}
\saveTG{勲}{24332}
\saveTG{憅}{24332}
\saveTG{魼}{24332}
\saveTG{憊}{24332}
\saveTG{㷶}{24332}
\saveTG{𢤅}{24333}
\saveTG{𩼰}{24333}
\saveTG{㷱}{24333}
\saveTG{𢘂}{24334}
\saveTG{𩸔}{24336}
\saveTG{𩺔}{24336}
\saveTG{𩷮}{24337}
\saveTG{憇}{24337}
\saveTG{烋}{24339}
\saveTG{恷}{24339}
\saveTG{𩵲}{24340}
\saveTG{𡚷}{24340}
\saveTG{𩼃}{24340}
\saveTG{鰣}{24341}
\saveTG{𩺵}{24341}
\saveTG{𪈙}{24341}
\saveTG{𩼬}{24343}
\saveTG{𩹭}{24343}
\saveTG{𩶬}{24343}
\saveTG{𪀔}{24343}
\saveTG{䲖}{24343}
\saveTG{𩵾}{24347}
\saveTG{𤿈}{24347}
\saveTG{鮍}{24347}
\saveTG{鸌}{24347}
\saveTG{鱯}{24347}
\saveTG{鯪}{24347}
\saveTG{䱸}{24347}
\saveTG{𩷨}{24347}
\saveTG{𩷚}{24347}
\saveTG{𩸉}{24347}
\saveTG{𩵼}{24347}
\saveTG{𩼒}{24351}
\saveTG{𩽮}{24351}
\saveTG{𩹛}{24353}
\saveTG{鱴}{24353}
\saveTG{𩻮}{24354}
\saveTG{𫛓}{24354}
\saveTG{𪇅}{24356}
\saveTG{𨖲}{24356}
\saveTG{𩼘}{24356}
\saveTG{𩶮}{24357}
\saveTG{鮕}{24360}
\saveTG{鯺}{24360}
\saveTG{𩿵}{24360}
\saveTG{𪂯}{24360}
\saveTG{𩹬}{24360}
\saveTG{鯌}{24361}
\saveTG{䱜}{24361}
\saveTG{䲛}{24361}
\saveTG{鰭}{24361}
\saveTG{鮚}{24361}
\saveTG{鱚}{24361}
\saveTG{𩺗}{24361}
\saveTG{𪁄}{24361}
\saveTG{𪃞}{24364}
\saveTG{𪃲}{24364}
\saveTG{𩹜}{24364}
\saveTG{鰙}{24364}
\saveTG{𩼺}{24367}
\saveTG{魽}{24370}
\saveTG{䲦}{24380}
\saveTG{䱋}{24381}
\saveTG{鯕}{24381}
\saveTG{鶀}{24381}
\saveTG{鯐}{24381}
\saveTG{𩺘}{24381}
\saveTG{𩻁}{24384}
\saveTG{𩹅}{24385}
\saveTG{𩽆}{24386}
\saveTG{鱝}{24386}
\saveTG{鱑}{24386}
\saveTG{𩷟}{24388}
\saveTG{鮴}{24390}
\saveTG{𩵦}{24390}
\saveTG{䱞}{24391}
\saveTG{𩺚}{24391}
\saveTG{𩹟}{24391}
\saveTG{𫙢}{24394}
\saveTG{𩶼}{24394}
\saveTG{𩹮}{24394}
\saveTG{𪄦}{24394}
\saveTG{𩽃}{24394}
\saveTG{鰈}{24394}
\saveTG{𩻓}{24394}
\saveTG{𩻧}{24395}
\saveTG{𩻻}{24396}
\saveTG{鯠}{24398}
\saveTG{𩺲}{24399}
\saveTG{𦫵}{24400}
\saveTG{升}{24400}
\saveTG{华}{24401}
\saveTG{𦨑}{24402}
\saveTG{䑧}{24403}
\saveTG{𡬫}{24403}
\saveTG{舣}{24403}
\saveTG{𠦲}{24404}
\saveTG{姇}{24404}
\saveTG{娤}{24404}
\saveTG{孧}{24407}
\saveTG{圱}{24410}
\saveTG{勉}{24412}
\saveTG{𠠷}{24412}
\saveTG{𦪆}{24414}
\saveTG{𨊆}{24414}
\saveTG{𨊜}{24417}
\saveTG{𨊅}{24417}
\saveTG{𦪛}{24417}
\saveTG{𦩄}{24417}
\saveTG{𨊁}{24417}
\saveTG{𫇡}{24427}
\saveTG{𠣄}{24427}
\saveTG{𦼌}{24427}
\saveTG{㔓}{24427}
\saveTG{䠸}{24427}
\saveTG{艜}{24427}
\saveTG{舿}{24427}
\saveTG{𫆄}{24427}
\saveTG{䠵}{24430}
\saveTG{舦}{24430}
\saveTG{艨}{24432}
\saveTG{𦪭}{24435}
\saveTG{䒃}{24436}
\saveTG{𨈠}{24440}
\saveTG{𦩚}{24440}
\saveTG{弉}{24441}
\saveTG{𨉴}{24442}
\saveTG{𨉞}{24442}
\saveTG{𫇠}{24443}
\saveTG{𢍎}{24446}
\saveTG{皺}{24447}
\saveTG{𨈛}{24447}
\saveTG{艧}{24447}
\saveTG{𤿣}{24447}
\saveTG{𤿾}{24447}
\saveTG{𦨭}{24447}
\saveTG{𨈵}{24447}
\saveTG{皴}{24447}
\saveTG{𡞓}{24449}
\saveTG{𨊙}{24451}
\saveTG{𦪠}{24454}
\saveTG{𩹯}{24456}
\saveTG{𨉀}{24457}
\saveTG{䑩}{24460}
\saveTG{𨈰}{24460}
\saveTG{艢}{24461}
\saveTG{艁}{24461}
\saveTG{𦩳}{24464}
\saveTG{𦨟}{24474}
\saveTG{𡙦}{24480}
\saveTG{𦨐}{24480}
\saveTG{䑴}{24481}
\saveTG{舼}{24481}
\saveTG{𦪂}{24482}
\saveTG{𦪗}{24486}
\saveTG{䒈}{24486}
\saveTG{𪏒}{24486}
\saveTG{𨊇}{24486}
\saveTG{𦫅}{24486}
\saveTG{𦩀}{24488}
\saveTG{𨈼}{24494}
\saveTG{艓}{24494}
\saveTG{𦩑}{24498}
\saveTG{犐}{24500}
\saveTG{㸯}{24503}
\saveTG{𤛀}{24503}
\saveTG{𨌄}{24506}
\saveTG{牡}{24510}
\saveTG{𡊽}{24510}
\saveTG{魁}{24510}
\saveTG{𡓮}{24511}
\saveTG{牠}{24512}
\saveTG{犆}{24512}
\saveTG{鬾}{24514}
\saveTG{㹊}{24515}
\saveTG{㹏}{24515}
\saveTG{㹓}{24517}
\saveTG{𤘣}{24517}
\saveTG{𢴝}{24517}
\saveTG{𢴽}{24517}
\saveTG{𤙥}{24517}
\saveTG{𤘱}{24517}
\saveTG{𤛪}{24517}
\saveTG{𤙸}{24517}
\saveTG{魌}{24518}
\saveTG{犄}{24521}
\saveTG{𤛶}{24527}
\saveTG{𤙅}{24527}
\saveTG{𤙯}{24527}
\saveTG{犕}{24527}
\saveTG{牞}{24527}
\saveTG{劧}{24527}
\saveTG{劺}{24527}
\saveTG{𤙒}{24527}
\saveTG{𤛩}{24527}
\saveTG{𤚑}{24530}
\saveTG{𤛌}{24530}
\saveTG{𤙮}{24531}
\saveTG{𤜒}{24531}
\saveTG{𤘙}{24540}
\saveTG{特}{24541}
\saveTG{㹗}{24543}
\saveTG{𤚟}{24543}
\saveTG{𤛘}{24544}
\saveTG{𪺲}{24547}
\saveTG{㹀}{24547}
\saveTG{𤿪}{24547}
\saveTG{𤙎}{24547}
\saveTG{𤚆}{24547}
\saveTG{𤜐}{24551}
\saveTG{𤛻}{24551}
\saveTG{牯}{24560}
\saveTG{𤛷}{24561}
\saveTG{牿}{24561}
\saveTG{㸵}{24561}
\saveTG{𪺴}{24562}
\saveTG{𤚐}{24564}
\saveTG{𢱞}{24572}
\saveTG{𤛇}{24581}
\saveTG{犊}{24584}
\saveTG{𤛥}{24586}
\saveTG{犢}{24586}
\saveTG{𤘬}{24590}
\saveTG{𪮍}{24594}
\saveTG{𤽁}{24600}
\saveTG{𥑧}{24601}
\saveTG{𠯒}{24601}
\saveTG{告}{24601}
\saveTG{𧨻}{24601}
\saveTG{𣈪}{24602}
\saveTG{𡬹}{24603}
\saveTG{𥆴}{24603}
\saveTG{𣁳}{24603}
\saveTG{𡬲}{24603}
\saveTG{㬱}{24607}
\saveTG{𣈌}{24609}
\saveTG{𪉯}{24611}
\saveTG{𩡤}{24612}
\saveTG{皢}{24612}
\saveTG{𫗼}{24614}
\saveTG{㿥}{24615}
\saveTG{𫗾}{24616}
\saveTG{馣}{24616}
\saveTG{㖌}{24617}
\saveTG{㖚}{24617}
\saveTG{𠸛}{24617}
\saveTG{𣦳}{24618}
\saveTG{𤾯}{24627}
\saveTG{𠡪}{24627}
\saveTG{𩡙}{24627}
\saveTG{𤾼}{24627}
\saveTG{勓}{24627}
\saveTG{劰}{24627}
\saveTG{劬}{24627}
\saveTG{勂}{24627}
\saveTG{𠡇}{24627}
\saveTG{𪟚}{24627}
\saveTG{𤾘}{24627}
\saveTG{𤾸}{24627}
\saveTG{㔚}{24627}
\saveTG{勫}{24627}
\saveTG{𩡒}{24627}
\saveTG{𩡉}{24628}
\saveTG{𤳖}{24631}
\saveTG{𦧳}{24632}
\saveTG{𤾬}{24632}
\saveTG{𤾮}{24632}
\saveTG{𢽡}{24640}
\saveTG{𦦰}{24643}
\saveTG{㿧}{24643}
\saveTG{𤿩}{24647}
\saveTG{馞}{24647}
\saveTG{𢻇}{24647}
\saveTG{𤽑}{24647}
\saveTG{𦧉}{24647}
\saveTG{𤽴}{24647}
\saveTG{𤿟}{24647}
\saveTG{𤿝}{24647}
\saveTG{𤿘}{24647}
\saveTG{皣}{24654}
\saveTG{𩎞}{24656}
\saveTG{𤾹}{24658}
\saveTG{𤾴}{24658}
\saveTG{𦧒}{24660}
\saveTG{𠼑}{24661}
\saveTG{皓}{24661}
\saveTG{甜}{24670}
\saveTG{𪽀}{24670}
\saveTG{𤿀}{24686}
\saveTG{馩}{24686}
\saveTG{𪿙}{24689}
\saveTG{𩠽}{24690}
\saveTG{𪽻}{24690}
\saveTG{𪉴}{24694}
\saveTG{𦧤}{24694}
\saveTG{𤾔}{24694}
\saveTG{䑜}{24694}
\saveTG{𪗕}{24700}
\saveTG{𣁭}{24703}
\saveTG{㞳}{24703}
\saveTG{㠚}{24703}
\saveTG{𪩙}{24703}
\saveTG{𡵌}{24703}
\saveTG{𪞷}{24704}
\saveTG{㟐}{24710}
\saveTG{齔}{24710}
\saveTG{龀}{24710}
\saveTG{毨}{24711}
\saveTG{馌}{24712}
\saveTG{嶢}{24712}
\saveTG{𪵠}{24712}
\saveTG{㐥}{24712}
\saveTG{𪘡}{24712}
\saveTG{𪨥}{24712}
\saveTG{𣮲}{24712}
\saveTG{𡸜}{24712}
\saveTG{𡻊}{24712}
\saveTG{𣯣}{24712}
\saveTG{𣯩}{24712}
\saveTG{峔}{24712}
\saveTG{𣬯}{24713}
\saveTG{𣰥}{24713}
\saveTG{𡻸}{24714}
\saveTG{𣬢}{24714}
\saveTG{𪗹}{24714}
\saveTG{𣯏}{24714}
\saveTG{𡽩}{24714}
\saveTG{𣬼}{24714}
\saveTG{𣯜}{24714}
\saveTG{𨻏}{24714}
\saveTG{𣰾}{24715}
\saveTG{巏}{24715}
\saveTG{馑}{24715}
\saveTG{𣭟}{24715}
\saveTG{𪙟}{24715}
\saveTG{崦}{24716}
\saveTG{嵖}{24716}
\saveTG{馇}{24716}
\saveTG{𪙁}{24716}
\saveTG{𣯚}{24716}
\saveTG{𣭎}{24716}
\saveTG{𣭊}{24716}
\saveTG{𡸳}{24717}
\saveTG{𡹟}{24717}
\saveTG{齛}{24717}
\saveTG{䶧}{24717}
\saveTG{䶔}{24717}
\saveTG{㐛}{24717}
\saveTG{}{24718}
\saveTG{氁}{24718}
\saveTG{𣰬}{24718}
\saveTG{嵁}{24718}
\saveTG{𡏩}{24718}
\saveTG{㟇}{24718}
\saveTG{𣰁}{24719}
\saveTG{𣰙}{24719}
\saveTG{𪘠}{24719}
\saveTG{}{24721}
\saveTG{𢇎}{24721}
\saveTG{崎}{24721}
\saveTG{齮}{24721}
\saveTG{𡿥}{24722}
\saveTG{𡹰}{24724}
\saveTG{𨼕}{24727}
\saveTG{𪩂}{24727}
\saveTG{㟓}{24727}
\saveTG{𠨡}{24727}
\saveTG{𠡒}{24727}
\saveTG{𩟿}{24727}
\saveTG{𠢒}{24727}
\saveTG{𠠳}{24727}
\saveTG{𡴽}{24727}
\saveTG{𡽝}{24727}
\saveTG{𪘖}{24727}
\saveTG{𡽇}{24727}
\saveTG{𪘃}{24727}
\saveTG{崂}{24727}
\saveTG{𪙰}{24727}
\saveTG{𪘱}{24727}
\saveTG{岰}{24727}
\saveTG{幼}{24727}
\saveTG{㔘}{24727}
\saveTG{崤}{24727}
\saveTG{帥}{24727}
\saveTG{嶱}{24727}
\saveTG{嵽}{24727}
\saveTG{}{24727}
\saveTG{𡿋}{24731}
\saveTG{𡶱}{24731}
\saveTG{𡵓}{24731}
\saveTG{𩜔}{24732}
\saveTG{𧜻}{24732}
\saveTG{𧚒}{24732}
\saveTG{𩚻}{24732}
\saveTG{㠓}{24732}
\saveTG{裝}{24732}
\saveTG{𩼐}{24736}
\saveTG{𩠅}{24738}
\saveTG{𡾲}{24741}
\saveTG{㟆}{24741}
\saveTG{嶹}{24741}
\saveTG{崻}{24741}
\saveTG{峙}{24741}
\saveTG{𡿚}{24742}
\saveTG{𪘵}{24742}
\saveTG{㠨}{24743}
\saveTG{𡻄}{24743}
\saveTG{𪗺}{24743}
\saveTG{𫗤}{24744}
\saveTG{巕}{24744}
\saveTG{饽}{24747}
\saveTG{崚}{24747}
\saveTG{岥}{24747}
\saveTG{岐}{24747}
\saveTG{𡸨}{24747}
\saveTG{𢺵}{24747}
\saveTG{𡿒}{24747}
\saveTG{㠛}{24747}
\saveTG{㟑}{24747}
\saveTG{𤿷}{24747}
\saveTG{𡷸}{24747}
\saveTG{𥀰}{24747}
\saveTG{𤿎}{24747}
\saveTG{𡿟}{24747}
\saveTG{𡻴}{24747}
\saveTG{𥀉}{24747}
\saveTG{𡿄}{24751}
\saveTG{㠏}{24754}
\saveTG{𡻹}{24754}
\saveTG{𡺨}{24756}
\saveTG{𫗡}{24760}
\saveTG{岵}{24760}
\saveTG{齰}{24761}
\saveTG{𡼎}{24761}
\saveTG{㟷}{24761}
\saveTG{𫜬}{24761}
\saveTG{䶜}{24761}
\saveTG{𪗾}{24761}
\saveTG{𡺸}{24761}
\saveTG{峼}{24761}
\saveTG{𡺐}{24764}
\saveTG{𪗳}{24770}
\saveTG{𩠁}{24770}
\saveTG{𡶑}{24770}
\saveTG{𡼉}{24772}
\saveTG{𪘓}{24772}
\saveTG{𡷥}{24772}
\saveTG{㠂}{24780}
\saveTG{䶞}{24781}
\saveTG{𡸷}{24781}
\saveTG{𡶵}{24781}
\saveTG{嵮}{24781}
\saveTG{齻}{24781}
\saveTG{齼}{24781}
\saveTG{}{24781}
\saveTG{𡻟}{24784}
\saveTG{馍}{24784}
\saveTG{𪚇}{24786}
\saveTG{巑}{24786}
\saveTG{䶝}{24788}
\saveTG{𦦕}{24788}
\saveTG{𡻯}{24788}
\saveTG{峽}{24788}
\saveTG{崊}{24790}
\saveTG{𠁫}{24790}
\saveTG{𪘆}{24790}
\saveTG{𠚐}{24790}
\saveTG{齽}{24791}
\saveTG{𪨼}{24791}
\saveTG{𡿗}{24794}
\saveTG{𡼨}{24794}
\saveTG{嵘}{24794}
\saveTG{𡺑}{24794}
\saveTG{嶛}{24796}
\saveTG{𪘨}{24798}
\saveTG{𨝖}{24798}
\saveTG{崍}{24798}
\saveTG{𨖰}{24800}
\saveTG{𨑡}{24801}
\saveTG{货}{24802}
\saveTG{赞}{24802}
\saveTG{𤞛}{24804}
\saveTG{奊}{24804}
\saveTG{奘}{24804}
\saveTG{𪥝}{24804}
\saveTG{𫐳}{24805}
\saveTG{贊}{24806}
\saveTG{貨}{24806}
\saveTG{𤏛}{24809}
\saveTG{焋}{24809}
\saveTG{𤈢}{24809}
\saveTG{𪥣}{24812}
\saveTG{𤏝}{24814}
\saveTG{𫇌}{24815}
\saveTG{劮}{24827}
\saveTG{𦂿}{24827}
\saveTG{𦤯}{24827}
\saveTG{𣀹}{24840}
\saveTG{𡚰}{24840}
\saveTG{𡚮}{24840}
\saveTG{𡞩}{24840}
\saveTG{𤿴}{24847}
\saveTG{𠭹}{24847}
\saveTG{𡚥}{24851}
\saveTG{𩾀}{24860}
\saveTG{𦀽}{24861}
\saveTG{𧶯}{24881}
\saveTG{𩾉}{24888}
\saveTG{𤒧}{24889}
\saveTG{𦤠}{24889}
\saveTG{𤍙}{24889}
\saveTG{𦤧}{24891}
\saveTG{爒}{24896}
\saveTG{紂}{24900}
\saveTG{𥾅}{24900}
\saveTG{紏}{24900}
\saveTG{科}{24900}
\saveTG{紨}{24900}
\saveTG{䊷}{24902}
\saveTG{𡬟}{24903}
\saveTG{𡭫}{24903}
\saveTG{𡬧}{24903}
\saveTG{𦀹}{24903}
\saveTG{𦆝}{24903}
\saveTG{䌀}{24903}
\saveTG{䋽}{24903}
\saveTG{𫃼}{24903}
\saveTG{𥞂}{24903}
\saveTG{𥡿}{24903}
\saveTG{𦆹}{24903}
\saveTG{𥾷}{24904}
\saveTG{𢇔}{24906}
\saveTG{𥾘}{24910}
\saveTG{𥿥}{24910}
\saveTG{𥝟}{24910}
\saveTG{𦀜}{24910}
\saveTG{続}{24912}
\saveTG{𣠎}{24912}
\saveTG{紞}{24912}
\saveTG{𦂔}{24912}
\saveTG{䄬}{24912}
\saveTG{㞊}{24912}
\saveTG{穘}{24912}
\saveTG{繞}{24912}
\saveTG{稙}{24912}
\saveTG{絓}{24914}
\saveTG{𦇇}{24914}
\saveTG{䅅}{24914}
\saveTG{稑}{24914}
\saveTG{𥣑}{24915}
\saveTG{䌯}{24915}
\saveTG{䌍}{24915}
\saveTG{䅒}{24917}
\saveTG{䅄}{24917}
\saveTG{𫀲}{24917}
\saveTG{𥡃}{24917}
\saveTG{𥞠}{24917}
\saveTG{𥝴}{24917}
\saveTG{𦃑}{24917}
\saveTG{𦀈}{24917}
\saveTG{𦈂}{24917}
\saveTG{䊶}{24917}
\saveTG{𥿞}{24917}
\saveTG{𥟪}{24917}
\saveTG{𨤕}{24917}
\saveTG{𦁱}{24917}
\saveTG{𥟌}{24917}
\saveTG{䊵}{24917}
\saveTG{紲}{24917}
\saveTG{𦀴}{24917}
\saveTG{𦂼}{24918}
\saveTG{𥣸}{24918}
\saveTG{𦂂}{24921}
\saveTG{𦂭}{24921}
\saveTG{𥟏}{24921}
\saveTG{綺}{24921}
\saveTG{䵙}{24923}
\saveTG{𥞎}{24924}
\saveTG{𥢳}{24924}
\saveTG{𦄂}{24927}
\saveTG{𫃾}{24927}
\saveTG{𦇋}{24927}
\saveTG{𦂓}{24927}
\saveTG{絝}{24927}
\saveTG{糼}{24927}
\saveTG{䋻}{24927}
\saveTG{絺}{24927}
\saveTG{𨤞}{24927}
\saveTG{𠢶}{24927}
\saveTG{勦}{24927}
\saveTG{𦅶}{24927}
\saveTG{𦆐}{24927}
\saveTG{𦃜}{24927}
\saveTG{絠}{24927}
\saveTG{𦀖}{24927}
\saveTG{䄲}{24927}
\saveTG{𥤆}{24927}
\saveTG{𥡹}{24927}
\saveTG{𥢸}{24927}
\saveTG{𥡘}{24927}
\saveTG{𥿌}{24927}
\saveTG{𥾔}{24927}
\saveTG{𥿠}{24927}
\saveTG{𦃣}{24927}
\saveTG{𥠮}{24927}
\saveTG{𦇩}{24927}
\saveTG{稀}{24927}
\saveTG{納}{24927}
\saveTG{𦄶}{24928}
\saveTG{𥞋}{24931}
\saveTG{𥿭}{24931}
\saveTG{𩷧}{24931}
\saveTG{𦀗}{24931}
\saveTG{䌲}{24931}
\saveTG{綕}{24931}
\saveTG{𦁷}{24931}
\saveTG{𥿢}{24931}
\saveTG{𦄥}{24931}
\saveTG{紶}{24932}
\saveTG{𦆟}{24932}
\saveTG{𥣛}{24932}
\saveTG{秾}{24932}
\saveTG{紘}{24932}
\saveTG{繱}{24932}
\saveTG{𦅢}{24932}
\saveTG{𦅇}{24932}
\saveTG{𦇠}{24933}
\saveTG{繨}{24935}
\saveTG{纄}{24935}
\saveTG{𦇎}{24936}
\saveTG{𪐊}{24936}
\saveTG{𦇶}{24937}
\saveTG{𥡍}{24937}
\saveTG{𥿔}{24940}
\saveTG{𪏮}{24940}
\saveTG{𫃤}{24940}
\saveTG{𫄃}{24940}
\saveTG{𡤨}{24940}
\saveTG{䋂}{24940}
\saveTG{𦀏}{24940}
\saveTG{𦀻}{24940}
\saveTG{䅸}{24941}
\saveTG{𦁑}{24941}
\saveTG{緈}{24941}
\saveTG{秲}{24941}
\saveTG{縙}{24941}
\saveTG{穁}{24941}
\saveTG{𥢍}{24941}
\saveTG{𥢟}{24942}
\saveTG{䌧}{24943}
\saveTG{𦃀}{24943}
\saveTG{𫄋}{24943}
\saveTG{𥤏}{24943}
\saveTG{𣘳}{24943}
\saveTG{𦄫}{24944}
\saveTG{𪏾}{24944}
\saveTG{綍}{24947}
\saveTG{紴}{24947}
\saveTG{𦃯}{24947}
\saveTG{𥞳}{24947}
\saveTG{𥞘}{24947}
\saveTG{𥾣}{24947}
\saveTG{𢻃}{24947}
\saveTG{𢻙}{24947}
\saveTG{稜}{24947}
\saveTG{綾}{24947}
\saveTG{秛}{24947}
\saveTG{秓}{24947}
\saveTG{𥾤}{24947}
\saveTG{穫}{24947}
\saveTG{𦂡}{24951}
\saveTG{䆎}{24951}
\saveTG{𦇴}{24952}
\saveTG{䌩}{24953}
\saveTG{𥠏}{24953}
\saveTG{𦅠}{24954}
\saveTG{䅿}{24954}
\saveTG{緙}{24956}
\saveTG{稦}{24956}
\saveTG{緯}{24956}
\saveTG{緢}{24960}
\saveTG{緖}{24960}
\saveTG{緒}{24960}
\saveTG{𪏻}{24960}
\saveTG{𦁿}{24960}
\saveTG{𥿍}{24960}
\saveTG{𣗺}{24960}
\saveTG{𥟵}{24960}
\saveTG{秙}{24960}
\saveTG{𥢗}{24961}
\saveTG{結}{24961}
\saveTG{䌋}{24961}
\saveTG{𥣱}{24961}
\saveTG{穡}{24961}
\saveTG{𦆏}{24961}
\saveTG{𦆶}{24961}
\saveTG{秸}{24961}
\saveTG{穯}{24961}
\saveTG{穑}{24961}
\saveTG{𦁎}{24961}
\saveTG{稓}{24961}
\saveTG{繥}{24961}
\saveTG{𥞴}{24961}
\saveTG{繬}{24961}
\saveTG{𥤜}{24963}
\saveTG{䅦}{24964}
\saveTG{𦅁}{24964}
\saveTG{𦅷}{24964}
\saveTG{𥟾}{24964}
\saveTG{𫄈}{24964}
\saveTG{𦂍}{24964}
\saveTG{䅲}{24967}
\saveTG{𥾰}{24970}
\saveTG{𥞈}{24970}
\saveTG{紺}{24970}
\saveTG{𣙸}{24972}
\saveTG{𥞇}{24972}
\saveTG{𥝛}{24980}
\saveTG{稘}{24981}
\saveTG{稹}{24981}
\saveTG{𥣢}{24981}
\saveTG{𣞟}{24981}
\saveTG{綨}{24981}
\saveTG{縝}{24981}
\saveTG{𥟕}{24982}
\saveTG{𦆬}{24982}
\saveTG{𦃮}{24982}
\saveTG{𥡸}{24984}
\saveTG{𦄍}{24984}
\saveTG{縸}{24984}
\saveTG{𥠚}{24985}
\saveTG{緓}{24985}
\saveTG{䌙}{24986}
\saveTG{穔}{24986}
\saveTG{𦄗}{24986}
\saveTG{䅩}{24986}
\saveTG{𥢊}{24986}
\saveTG{纘}{24986}
\saveTG{續}{24986}
\saveTG{𦆋}{24986}
\saveTG{𦇣}{24986}
\saveTG{穳}{24986}
\saveTG{綊}{24988}
\saveTG{𥞵}{24988}
\saveTG{𥡠}{24988}
\saveTG{𦆖}{24989}
\saveTG{䊾}{24990}
\saveTG{𥝰}{24990}
\saveTG{綝}{24990}
\saveTG{𥟼}{24991}
\saveTG{䌝}{24991}
\saveTG{䌨}{24991}
\saveTG{䌭}{24991}
\saveTG{𫃢}{24992}
\saveTG{𦄨}{24993}
\saveTG{𦃝}{24993}
\saveTG{䌇}{24993}
\saveTG{𤚦}{24994}
\saveTG{𥿴}{24994}
\saveTG{𥠓}{24994}
\saveTG{緤}{24994}
\saveTG{𦁜}{24994}
\saveTG{𥢇}{24995}
\saveTG{繚}{24996}
\saveTG{䅘}{24998}
\saveTG{䋱}{24998}
\saveTG{牛}{25000}
\saveTG{𤗄}{25000}
\saveTG{牜}{25000}
\saveTG{𢺨}{25031}
\saveTG{𪺣}{25043}
\saveTG{𤗬}{25044}
\saveTG{𢲱}{25047}
\saveTG{𤗰}{25066}
\saveTG{𤗮}{25086}
\saveTG{𤗴}{25086}
\saveTG{𤗝}{25087}
\saveTG{㸡}{25092}
\saveTG{𤗣}{25094}
\saveTG{𤗗}{25096}
\saveTG{𤗛}{25096}
\saveTG{𣥱}{25100}
\saveTG{生}{25100}
\saveTG{𧗃}{25102}
\saveTG{𫚏}{25103}
\saveTG{𡋚}{25104}
\saveTG{㘶}{25104}
\saveTG{𡌅}{25104}
\saveTG{𣥞}{25105}
\saveTG{绅}{25106}
\saveTG{𠁪}{25106}
\saveTG{甡}{25110}
\saveTG{绕}{25112}
\saveTG{㽓}{25115}
\saveTG{甡}{25115}
\saveTG{鲀}{25117}
\saveTG{纨}{25117}
\saveTG{纯}{25117}
\saveTG{𥣨}{25117}
\saveTG{鳢}{25118}
\saveTG{𫚒}{25127}
\saveTG{𩇛}{25127}
\saveTG{𪉋}{25127}
\saveTG{𨨂}{25127}
\saveTG{绋}{25127}
\saveTG{鲭}{25127}
\saveTG{纬}{25127}
\saveTG{𩧶}{25127}
\saveTG{}{25127}
\saveTG{𧑤}{25131}
\saveTG{䌸}{25132}
\saveTG{𠘊}{25132}
\saveTG{𧍅}{25136}
\saveTG{缱}{25137}
\saveTG{𩾂}{25143}
\saveTG{𨥻}{25147}
\saveTG{𣥽}{25157}
\saveTG{𪽃}{25158}
\saveTG{䌷}{25160}
\saveTG{鲉}{25160}
\saveTG{}{25160}
\saveTG{𤾭}{25161}
\saveTG{𫚧}{25166}
\saveTG{}{25168}
\saveTG{䲠}{25168}
\saveTG{𨤴}{25181}
\saveTG{𫛞}{25182}
\saveTG{𡙇}{25182}
\saveTG{𧖫}{25182}
\saveTG{𡚆}{25182}
\saveTG{缋}{25182}
\saveTG{绩}{25182}
\saveTG{歵}{25186}
\saveTG{𧗏}{25186}
\saveTG{𤯡}{25192}
\saveTG{𫚐}{25192}
\saveTG{练}{25194}
\saveTG{}{25196}
\saveTG{𠗑}{25198}
\saveTG{𤃀}{25199}
\saveTG{𠘞}{25199}
\saveTG{𢓐}{25200}
\saveTG{仹}{25200}
\saveTG{件}{25200}
\saveTG{仗}{25200}
\saveTG{舛}{25200}
\saveTG{𧲤}{25202}
\saveTG{𧣈}{25202}
\saveTG{㣡}{25206}
\saveTG{𥟹}{25206}
\saveTG{仲}{25206}
\saveTG{使}{25206}
\saveTG{伸}{25206}
\saveTG{俥}{25206}
\saveTG{𨏧}{25206}
\saveTG{𧳡}{25206}
\saveTG{𧲬}{25206}
\saveTG{𧳅}{25206}
\saveTG{𢖀}{25206}
\saveTG{㑖}{25206}
\saveTG{𠈐}{25206}
\saveTG{律}{25207}
\saveTG{貄}{25207}
\saveTG{𦘖}{25207}
\saveTG{𠈞}{25207}
\saveTG{侓}{25207}
\saveTG{倳}{25207}
\saveTG{𫊠}{25210}
\saveTG{夝}{25210}
\saveTG{徃}{25210}
\saveTG{𠇷}{25210}
\saveTG{𩲗}{25210}
\saveTG{𩳂}{25211}
\saveTG{𤯷}{25212}
\saveTG{侥}{25212}
\saveTG{𩲍}{25212}
\saveTG{儘}{25212}
\saveTG{𤯴}{25212}
\saveTG{䖖}{25215}
\saveTG{𠌱}{25215}
\saveTG{𧇶}{25216}
\saveTG{𩳛}{25216}
\saveTG{𨌜}{25216}
\saveTG{䰠}{25216}
\saveTG{䰽}{25217}
\saveTG{𢓃}{25217}
\saveTG{𠌁}{25217}
\saveTG{𠌢}{25217}
\saveTG{𠌷}{25217}
\saveTG{伅}{25217}
\saveTG{䰡}{25218}
\saveTG{僼}{25218}
\saveTG{軆}{25218}
\saveTG{𩲴}{25218}
\saveTG{𩳒}{25219}
\saveTG{𩳅}{25219}
\saveTG{𩳆}{25219}
\saveTG{𤯝}{25227}
\saveTG{𠍶}{25227}
\saveTG{𧣘}{25227}
\saveTG{𧢿}{25227}
\saveTG{𪳓}{25227}
\saveTG{伟}{25227}
\saveTG{甧}{25227}
\saveTG{倩}{25227}
\saveTG{俜}{25227}
\saveTG{伂}{25227}
\saveTG{彿}{25227}
\saveTG{佛}{25227}
\saveTG{𨵣}{25227}
\saveTG{𧲁}{25227}
\saveTG{𢑩}{25227}
\saveTG{𠍟}{25227}
\saveTG{𠏐}{25227}
\saveTG{𠏬}{25227}
\saveTG{𢖊}{25227}
\saveTG{体}{25230}
\saveTG{僆}{25230}
\saveTG{躰}{25230}
\saveTG{𠐭}{25231}
\saveTG{㣸}{25231}
\saveTG{𢖧}{25231}
\saveTG{𤙕}{25231}
\saveTG{𠎱}{25231}
\saveTG{𤛚}{25232}
\saveTG{齈}{25232}
\saveTG{儂}{25232}
\saveTG{齉}{25232}
\saveTG{传}{25232}
\saveTG{𨑊}{25232}
\saveTG{儾}{25232}
\saveTG{𨾃}{25232}
\saveTG{俵}{25232}
\saveTG{僡}{25233}
\saveTG{触}{25236}
\saveTG{𧳃}{25236}
\saveTG{𠊞}{25236}
\saveTG{𧴊}{25236}
\saveTG{𠌼}{25236}
\saveTG{儙}{25237}
\saveTG{𪔊}{25237}
\saveTG{𠑌}{25238}
\saveTG{𠎝}{25238}
\saveTG{俦}{25240}
\saveTG{健}{25240}
\saveTG{徤}{25240}
\saveTG{𪫖}{25243}
\saveTG{傳}{25243}
\saveTG{軁}{25244}
\saveTG{僂}{25244}
\saveTG{貗}{25244}
\saveTG{𠉯}{25244}
\saveTG{𪖹}{25244}
\saveTG{𡖝}{25247}
\saveTG{傋}{25247}
\saveTG{𠉕}{25247}
\saveTG{𠎹}{25247}
\saveTG{𤠰}{25247}
\saveTG{𢓒}{25247}
\saveTG{𠇦}{25247}
\saveTG{𪜼}{25250}
\saveTG{𠎔}{25252}
\saveTG{𠑭}{25256}
\saveTG{𢔵}{25257}
\saveTG{𠉩}{25257}
\saveTG{俸}{25258}
\saveTG{𢔓}{25258}
\saveTG{伷}{25260}
\saveTG{㣙}{25260}
\saveTG{𧲹}{25260}
\saveTG{㑋}{25265}
\saveTG{傮}{25266}
\saveTG{僣}{25268}
\saveTG{𠏝}{25268}
\saveTG{偆}{25268}
\saveTG{𠑤}{25269}
\saveTG{𤕌}{25269}
\saveTG{𢓳}{25271}
\saveTG{𠌴}{25277}
\saveTG{䵻}{25277}
\saveTG{觖}{25280}
\saveTG{佚}{25280}
\saveTG{伕}{25280}
\saveTG{侠}{25280}
\saveTG{佒}{25280}
\saveTG{偼}{25281}
\saveTG{倢}{25281}
\saveTG{𪖮}{25281}
\saveTG{𤟃}{25281}
\saveTG{徢}{25281}
\saveTG{倎}{25281}
\saveTG{𠝰}{25282}
\saveTG{𧣧}{25282}
\saveTG{𧲱}{25282}
\saveTG{侇}{25282}
\saveTG{债}{25282}
\saveTG{㚍}{25286}
\saveTG{𢖑}{25286}
\saveTG{穨}{25286}
\saveTG{債}{25286}
\saveTG{僓}{25286}
\saveTG{𧷪}{25286}
\saveTG{𥣧}{25286}
\saveTG{儧}{25286}
\saveTG{𢓛}{25290}
\saveTG{俫}{25290}
\saveTG{㑍}{25290}
\saveTG{𠇱}{25290}
\saveTG{徕}{25290}
\saveTG{佅}{25290}
\saveTG{侏}{25290}
\saveTG{𠍷}{25292}
\saveTG{𢓣}{25292}
\saveTG{𤕹}{25292}
\saveTG{傃}{25293}
\saveTG{𤠚}{25293}
\saveTG{𠎀}{25294}
\saveTG{𪝨}{25294}
\saveTG{𧤠}{25294}
\saveTG{𧤛}{25294}
\saveTG{傑}{25294}
\saveTG{𠎖}{25294}
\saveTG{㑛}{25296}
\saveTG{觫}{25296}
\saveTG{倲}{25296}
\saveTG{㣱}{25296}
\saveTG{𠍀}{25296}
\saveTG{𡖯}{25296}
\saveTG{𤟈}{25296}
\saveTG{𢔅}{25296}
\saveTG{徚}{25296}
\saveTG{𠋖}{25296}
\saveTG{𧳣}{25296}
\saveTG{𤖂}{25296}
\saveTG{齂}{25299}
\saveTG{傣}{25299}
\saveTG{𨽽}{25299}
\saveTG{𧳙}{25299}
\saveTG{䰷}{25300}
\saveTG{𩵠}{25302}
\saveTG{𩿓}{25302}
\saveTG{䱁}{25303}
\saveTG{𩶳}{25305}
\saveTG{𩸲}{25306}
\saveTG{鰰}{25306}
\saveTG{𩵵}{25306}
\saveTG{𪀴}{25307}
\saveTG{鮏}{25310}
\saveTG{𫙬}{25312}
\saveTG{𩻽}{25316}
\saveTG{魨}{25317}
\saveTG{鱧}{25318}
\saveTG{鱐}{25327}
\saveTG{鮄}{25327}
\saveTG{鯖}{25327}
\saveTG{𪆭}{25327}
\saveTG{𩢫}{25327}
\saveTG{𩶍}{25327}
\saveTG{𪀾}{25327}
\saveTG{𩣏}{25327}
\saveTG{鰱}{25330}
\saveTG{𢛭}{25331}
\saveTG{𩶥}{25331}
\saveTG{𤇓}{25331}
\saveTG{𢘡}{25331}
\saveTG{䱪}{25332}
\saveTG{𩼅}{25332}
\saveTG{𪇌}{25332}
\saveTG{怣}{25338}
\saveTG{𩽎}{25338}
\saveTG{䲇}{25339}
\saveTG{𩻸}{25339}
\saveTG{鰎}{25340}
\saveTG{鱄}{25343}
\saveTG{𪅘}{25343}
\saveTG{𩸸}{25344}
\saveTG{𪅛}{25344}
\saveTG{䱾}{25344}
\saveTG{𩻥}{25347}
\saveTG{𩶎}{25347}
\saveTG{𩹝}{25354}
\saveTG{𩸮}{25358}
\saveTG{鮋}{25360}
\saveTG{𩺅}{25361}
\saveTG{𩻨}{25361}
\saveTG{鰽}{25366}
\saveTG{䲋}{25368}
\saveTG{鰆}{25368}
\saveTG{𤴎}{25369}
\saveTG{𪄻}{25377}
\saveTG{𩵩}{25380}
\saveTG{鴃}{25380}
\saveTG{䳀}{25382}
\saveTG{䱃}{25382}
\saveTG{鮧}{25382}
\saveTG{䱀}{25382}
\saveTG{𪃆}{25384}
\saveTG{𫙷}{25386}
\saveTG{鰿}{25386}
\saveTG{𩼜}{25386}
\saveTG{𩼱}{25386}
\saveTG{𫙛}{25390}
\saveTG{䱅}{25390}
\saveTG{鮢}{25390}
\saveTG{鮇}{25390}
\saveTG{𫙮}{25394}
\saveTG{𩻫}{25394}
\saveTG{鰊}{25396}
\saveTG{鯟}{25396}
\saveTG{𦪨}{25406}
\saveTG{舯}{25406}
\saveTG{𨎆}{25406}
\saveTG{𨈮}{25406}
\saveTG{𡕰}{25407}
\saveTG{𦨱}{25407}
\saveTG{𦩦}{25407}
\saveTG{𨉚}{25417}
\saveTG{𨽺}{25419}
\saveTG{𦪓}{25424}
\saveTG{䑶}{25427}
\saveTG{𦨡}{25427}
\saveTG{𠡩}{25427}
\saveTG{㔟}{25427}
\saveTG{𨉂}{25427}
\saveTG{𦪲}{25429}
\saveTG{𦪺}{25429}
\saveTG{𫏪}{25431}
\saveTG{艛}{25444}
\saveTG{𦩷}{25447}
\saveTG{𨈭}{25455}
\saveTG{𦩨}{25457}
\saveTG{舳}{25460}
\saveTG{艚}{25466}
\saveTG{𨉩}{25468}
\saveTG{𤴋}{25469}
\saveTG{𦩌}{25481}
\saveTG{䠿}{25486}
\saveTG{𦪒}{25486}
\saveTG{𦪸}{25496}
\saveTG{𤘧}{25500}
\saveTG{牪}{25500}
\saveTG{𢲝}{25502}
\saveTG{𤛕}{25502}
\saveTG{𢫰}{25502}
\saveTG{犇}{25505}
\saveTG{𨌾}{25506}
\saveTG{牲}{25510}
\saveTG{𤯶}{25512}
\saveTG{𤘛}{25517}
\saveTG{𤘫}{25517}
\saveTG{魅}{25519}
\saveTG{牬}{25527}
\saveTG{𤙂}{25527}
\saveTG{㸬}{25527}
\saveTG{𤙃}{25530}
\saveTG{𤜗}{25531}
\saveTG{犍}{25540}
\saveTG{𪭩}{25542}
\saveTG{𤛠}{25544}
\saveTG{𤘼}{25544}
\saveTG{𪺱}{25547}
\saveTG{𤛭}{25552}
\saveTG{𤚚}{25554}
\saveTG{牰}{25560}
\saveTG{𤙶}{25581}
\saveTG{𤙈}{25582}
\saveTG{𤚩}{25594}
\saveTG{𤙨}{25596}
\saveTG{吿}{25600}
\saveTG{𤯬}{25601}
\saveTG{𥓧}{25601}
\saveTG{𤚻}{25601}
\saveTG{眚}{25601}
\saveTG{𥑦}{25602}
\saveTG{𤯣}{25602}
\saveTG{𩠨}{25602}
\saveTG{𥅶}{25605}
\saveTG{𦧏}{25605}
\saveTG{𩡃}{25606}
\saveTG{𦧜}{25606}
\saveTG{𠱒}{25609}
\saveTG{𨡵}{25612}
\saveTG{𧦏}{25617}
\saveTG{𣠶}{25617}
\saveTG{𪴄}{25627}
\saveTG{𤽌}{25627}
\saveTG{皘}{25627}
\saveTG{𧑪}{25631}
\saveTG{𠳙}{25640}
\saveTG{𤽯}{25643}
\saveTG{舑}{25647}
\saveTG{䑙}{25647}
\saveTG{𪾃}{25656}
\saveTG{䭰}{25658}
\saveTG{𪉪}{25658}
\saveTG{㿤}{25668}
\saveTG{𤱽}{25680}
\saveTG{疀}{25681}
\saveTG{𤱾}{25682}
\saveTG{𦦅}{25684}
\saveTG{𪉮}{25684}
\saveTG{皟}{25686}
\saveTG{𩡞}{25686}
\saveTG{𫘁}{25686}
\saveTG{謮}{25686}
\saveTG{𤾀}{25687}
\saveTG{𤽜}{25690}
\saveTG{𩠿}{25690}
\saveTG{皌}{25690}
\saveTG{𩡦}{25694}
\saveTG{𡶒}{25702}
\saveTG{𡵜}{25702}
\saveTG{𨸥}{25705}
\saveTG{𡶈}{25706}
\saveTG{𩰡}{25706}
\saveTG{𡷖}{25706}
\saveTG{㞲}{25706}
\saveTG{𪗧}{25706}
\saveTG{𪘌}{25706}
\saveTG{𩰢}{25706}
\saveTG{齥}{25706}
\saveTG{峍}{25707}
\saveTG{𪜓}{25710}
\saveTG{𣬳}{25710}
\saveTG{峣}{25712}
\saveTG{𣮛}{25712}
\saveTG{𣯭}{25712}
\saveTG{饶}{25712}
\saveTG{𧈞}{25713}
\saveTG{𣭦}{25713}
\saveTG{𣯫}{25714}
\saveTG{毽}{25714}
\saveTG{𣭌}{25715}
\saveTG{𤯖}{25715}
\saveTG{饨}{25717}
\saveTG{鼄}{25717}
\saveTG{𡴿}{25717}
\saveTG{𡽍}{25718}
\saveTG{𣭯}{25718}
\saveTG{𣭴}{25719}
\saveTG{𣭐}{25719}
\saveTG{𪵛}{25719}
\saveTG{𪵞}{25719}
\saveTG{𡼣}{25724}
\saveTG{崝}{25727}
\saveTG{䶚}{25731}
\saveTG{嶩}{25732}
\saveTG{𧙍}{25732}
\saveTG{馕}{25732}
\saveTG{𡿝}{25732}
\saveTG{蚀}{25736}
\saveTG{𪩍}{25743}
\saveTG{嶁}{25744}
\saveTG{𡻉}{25747}
\saveTG{𤯟}{25757}
\saveTG{岫}{25760}
\saveTG{𪨰}{25765}
\saveTG{𪙡}{25766}
\saveTG{嶆}{25766}
\saveTG{𩠎}{25766}
\saveTG{峡}{25780}
\saveTG{岟}{25780}
\saveTG{崨}{25781}
\saveTG{峓}{25782}
\saveTG{馈}{25782}
\saveTG{䶦}{25786}
\saveTG{㠝}{25786}
\saveTG{𡶎}{25790}
\saveTG{崃}{25790}
\saveTG{𪨴}{25792}
\saveTG{𡻈}{25794}
\saveTG{嵥}{25794}
\saveTG{𫗧}{25796}
\saveTG{𡷽}{25796}
\saveTG{𪘜}{25796}
\saveTG{崠}{25796}
\saveTG{失}{25800}
\saveTG{𫙓}{25806}
\saveTG{肄}{25807}
\saveTG{𨗃}{25822}
\saveTG{𥏚}{25827}
\saveTG{𦤕}{25882}
\saveTG{𨽿}{25888}
\saveTG{𣝖}{25890}
\saveTG{𤒗}{25892}
\saveTG{𦠁}{25896}
\saveTG{𤐬}{25896}
\saveTG{𨽹}{25899}
\saveTG{𩸺}{25899}
\saveTG{𥝫}{25900}
\saveTG{𥾫}{25900}
\saveTG{𥾒}{25900}
\saveTG{朱}{25900}
\saveTG{䋅}{25902}
\saveTG{䵓}{25902}
\saveTG{桀}{25904}
\saveTG{𥝷}{25905}
\saveTG{𥣽}{25906}
\saveTG{种}{25906}
\saveTG{絏}{25906}
\saveTG{𫃞}{25906}
\saveTG{𥞁}{25906}
\saveTG{䄿}{25906}
\saveTG{𦀺}{25906}
\saveTG{紳}{25906}
\saveTG{𥞢}{25907}
\saveTG{𫆔}{25907}
\saveTG{𦂻}{25907}
\saveTG{𥞰}{25907}
\saveTG{䋖}{25907}
\saveTG{繣}{25916}
\saveTG{秇}{25917}
\saveTG{𥾴}{25917}
\saveTG{𦄌}{25917}
\saveTG{𥾑}{25917}
\saveTG{純}{25917}
\saveTG{𥡩}{25917}
\saveTG{䅖}{25917}
\saveTG{紈}{25917}
\saveTG{䌡}{25918}
\saveTG{𥟗}{25921}
\saveTG{𦂈}{25927}
\saveTG{䅢}{25927}
\saveTG{𦀔}{25927}
\saveTG{秭}{25927}
\saveTG{𦅎}{25927}
\saveTG{䄶}{25927}
\saveTG{𥾧}{25927}
\saveTG{紼}{25927}
\saveTG{𦀃}{25927}
\saveTG{繡}{25927}
\saveTG{繍}{25927}
\saveTG{綪}{25927}
\saveTG{絊}{25930}
\saveTG{縺}{25930}
\saveTG{𦁆}{25931}
\saveTG{穠}{25932}
\saveTG{繷}{25932}
\saveTG{䵜}{25932}
\saveTG{𦈃}{25932}
\saveTG{穗}{25933}
\saveTG{繐}{25933}
\saveTG{𫃪}{25933}
\saveTG{𦁨}{25933}
\saveTG{𪐍}{25936}
\saveTG{𥣴}{25937}
\saveTG{𦇀}{25937}
\saveTG{繾}{25937}
\saveTG{穂}{25937}
\saveTG{𪐔}{25938}
\saveTG{𦇞}{25938}
\saveTG{𦂾}{25938}
\saveTG{𦅋}{25938}
\saveTG{𥟀}{25942}
\saveTG{𥡵}{25943}
\saveTG{縳}{25943}
\saveTG{𦀳}{25943}
\saveTG{緀}{25944}
\saveTG{䅹}{25944}
\saveTG{縷}{25944}
\saveTG{𪏼}{25944}
\saveTG{𦁰}{25944}
\saveTG{𥿛}{25944}
\saveTG{𥠾}{25947}
\saveTG{𦃪}{25947}
\saveTG{𦂩}{25947}
\saveTG{𪐃}{25958}
\saveTG{紬}{25960}
\saveTG{釉}{25960}
\saveTG{秞}{25960}
\saveTG{𦃠}{25961}
\saveTG{𦄧}{25966}
\saveTG{𦀵}{25966}
\saveTG{𥠔}{25968}
\saveTG{𦅦}{25968}
\saveTG{𦄑}{25974}
\saveTG{𥡟}{25977}
\saveTG{紻}{25980}
\saveTG{䄮}{25980}
\saveTG{䊿}{25980}
\saveTG{𣗻}{25980}
\saveTG{秧}{25980}
\saveTG{秩}{25980}
\saveTG{紩}{25980}
\saveTG{緁}{25981}
\saveTG{𥞗}{25982}
\saveTG{䄺}{25982}
\saveTG{𥝭}{25982}
\saveTG{𦇰}{25982}
\saveTG{𪏿}{25982}
\saveTG{𦀊}{25982}
\saveTG{𥞦}{25982}
\saveTG{䊽}{25982}
\saveTG{𥠬}{25984}
\saveTG{𥢡}{25984}
\saveTG{績}{25986}
\saveTG{纉}{25986}
\saveTG{𥣪}{25986}
\saveTG{𥢼}{25986}
\saveTG{積}{25986}
\saveTG{繢}{25986}
\saveTG{𥢢}{25986}
\saveTG{𦇅}{25986}
\saveTG{䋘}{25990}
\saveTG{𫃟}{25990}
\saveTG{秼}{25990}
\saveTG{秣}{25990}
\saveTG{絑}{25990}
\saveTG{𥿉}{25990}
\saveTG{𥞊}{25990}
\saveTG{𦅱}{25992}
\saveTG{縤}{25993}
\saveTG{𥣆}{25994}
\saveTG{縥}{25994}
\saveTG{𥠹}{25994}
\saveTG{𥤕}{25995}
\saveTG{𪐓}{25995}
\saveTG{綀}{25996}
\saveTG{𦈌}{25996}
\saveTG{䵔}{25996}
\saveTG{𥟈}{25996}
\saveTG{練}{25996}
\saveTG{𨽾}{25999}
\saveTG{𥟤}{25999}
\saveTG{𫕙}{25999}
\saveTG{𨽻}{25999}
\saveTG{𨾀}{25999}
\saveTG{白}{26000}
\saveTG{囪}{26000}
\saveTG{囱}{26000}
\saveTG{甶}{26000}
\saveTG{旧}{26000}
\saveTG{囟}{26000}
\saveTG{自}{26000}
\saveTG{𢶀}{26011}
\saveTG{覑}{26012}
\saveTG{𤗧}{26017}
\saveTG{粵}{26027}
\saveTG{𫒀}{26027}
\saveTG{䅌}{26027}
\saveTG{𤗱}{26027}
\saveTG{粤}{26027}
\saveTG{牕}{26030}
\saveTG{𢙥}{26033}
\saveTG{𢝏}{26033}
\saveTG{牌}{26040}
\saveTG{𢰊}{26043}
\saveTG{𥊑}{26047}
\saveTG{𪮐}{26060}
\saveTG{𤗁}{26082}
\saveTG{𤽱}{26082}
\saveTG{𤗘}{26082}
\saveTG{𤾏}{26094}
\saveTG{}{26100}
\saveTG{𠷷}{26100}
\saveTG{𫚔}{26100}
\saveTG{𪛇}{26100}
\saveTG{𠲆}{26100}
\saveTG{鲴}{26100}
\saveTG{细}{26100}
\saveTG{缃}{26100}
\saveTG{𤽫}{26101}
\saveTG{𤽓}{26102}
\saveTG{𤽢}{26102}
\saveTG{𪵻}{26102}
\saveTG{鲌}{26102}
\saveTG{𤼿}{26102}
\saveTG{𥁬}{26102}
\saveTG{𤽋}{26103}
\saveTG{𦤃}{26104}
\saveTG{皨}{26104}
\saveTG{堡}{26104}
\saveTG{𤽧}{26104}
\saveTG{皇}{26104}
\saveTG{𤽔}{26109}
\saveTG{𦤐}{26109}
\saveTG{𨨑}{26109}
\saveTG{绲}{26112}
\saveTG{覬}{26112}
\saveTG{鲲}{26112}
\saveTG{皩}{26112}
\saveTG{缊}{26112}
\saveTG{鳁}{26112}
\saveTG{}{26114}
\saveTG{鳇}{26114}
\saveTG{鲤}{26115}
\saveTG{绳}{26116}
\saveTG{𧢫}{26117}
\saveTG{𧠒}{26117}
\saveTG{𧠛}{26117}
\saveTG{𧡁}{26117}
\saveTG{𤾿}{26117}
\saveTG{𤾐}{26117}
\saveTG{𧗗}{26121}
\saveTG{甥}{26127}
\saveTG{绵}{26127}
\saveTG{鳄}{26127}
\saveTG{绢}{26127}
\saveTG{𦈖}{26127}
\saveTG{鳎}{26127}
\saveTG{𩹄}{26127}
\saveTG{𧗂}{26130}
\saveTG{鳃}{26130}
\saveTG{缌}{26130}
\saveTG{𪾁}{26131}
\saveTG{𧑉}{26131}
\saveTG{𧕠}{26131}
\saveTG{𫄶}{26131}
\saveTG{缳}{26132}
\saveTG{鳂}{26132}
\saveTG{𧍭}{26132}
\saveTG{𨤼}{26132}
\saveTG{𦈓}{26132}
\saveTG{𫜅}{26132}
\saveTG{𧌠}{26135}
\saveTG{𧓎}{26135}
\saveTG{𧕞}{26136}
\saveTG{𢽗}{26140}
\saveTG{缉}{26141}
\saveTG{𫄯}{26142}
\saveTG{𦈛}{26147}
\saveTG{鳗}{26147}
\saveTG{缦}{26147}
\saveTG{𤾲}{26148}
\saveTG{𡏙}{26148}
\saveTG{𤾺}{26156}
\saveTG{鲳}{26160}
\saveTG{织}{26180}
\saveTG{鳀}{26181}
\saveTG{缇}{26181}
\saveTG{㱗}{26184}
\saveTG{缐}{26192}
\saveTG{鳈}{26192}
\saveTG{缧}{26193}
\saveTG{缲}{26194}
\saveTG{𫚝}{26195}
\saveTG{鳏}{26199}
\saveTG{俰}{26200}
\saveTG{𤼽}{26200}
\saveTG{𧣕}{26200}
\saveTG{𡖣}{26200}
\saveTG{𧤯}{26200}
\saveTG{𤱸}{26200}
\saveTG{𤳞}{26200}
\saveTG{𠇻}{26200}
\saveTG{𢒤}{26200}
\saveTG{倁}{26200}
\saveTG{伵}{26200}
\saveTG{侞}{26200}
\saveTG{佪}{26200}
\saveTG{徊}{26200}
\saveTG{個}{26200}
\saveTG{伽}{26200}
\saveTG{佃}{26200}
\saveTG{𪜶}{26200}
\saveTG{𪝋}{26200}
\saveTG{𠇂}{26200}
\saveTG{𧣛}{26200}
\saveTG{𠉍}{26200}
\saveTG{𪜮}{26200}
\saveTG{𠉢}{26200}
\saveTG{𠋼}{26200}
\saveTG{𣦦}{26200}
\saveTG{㒁}{26200}
\saveTG{𠎡}{26200}
\saveTG{𠌛}{26200}
\saveTG{㐰}{26200}
\saveTG{𪝩}{26200}
\saveTG{𪜭}{26200}
\saveTG{𢔊}{26200}
\saveTG{𢓨}{26200}
\saveTG{𦣺}{26200}
\saveTG{𦤓}{26201}
\saveTG{𠉬}{26201}
\saveTG{貃}{26202}
\saveTG{𤽖}{26202}
\saveTG{㣎}{26202}
\saveTG{㑑}{26202}
\saveTG{𧳄}{26202}
\saveTG{𢍑}{26202}
\saveTG{伯}{26202}
\saveTG{𠈈}{26202}
\saveTG{𢇆}{26203}
\saveTG{𠵓}{26206}
\saveTG{𦣿}{26208}
\saveTG{䖑}{26210}
\saveTG{𨈱}{26210}
\saveTG{觛}{26210}
\saveTG{但}{26210}
\saveTG{𣇙}{26210}
\saveTG{𩳓}{26211}
\saveTG{𪅔}{26211}
\saveTG{𠇝}{26211}
\saveTG{𧢉}{26212}
\saveTG{𧆽}{26212}
\saveTG{𧇚}{26212}
\saveTG{𢓼}{26212}
\saveTG{𧴞}{26212}
\saveTG{𩵅}{26212}
\saveTG{𧴢}{26212}
\saveTG{𫔹}{26212}
\saveTG{𪝑}{26212}
\saveTG{𠌔}{26212}
\saveTG{𩳻}{26212}
\saveTG{儭}{26212}
\saveTG{臰}{26212}
\saveTG{児}{26212}
\saveTG{倱}{26212}
\saveTG{儬}{26212}
\saveTG{侃}{26212}
\saveTG{覶}{26212}
\saveTG{貌}{26212}
\saveTG{皃}{26212}
\saveTG{貔}{26212}
\saveTG{俔}{26212}
\saveTG{覰}{26212}
\saveTG{覷}{26212}
\saveTG{覻}{26212}
\saveTG{覤}{26212}
\saveTG{𧠿}{26212}
\saveTG{𧳢}{26212}
\saveTG{𦣾}{26212}
\saveTG{𩳏}{26213}
\saveTG{傀}{26213}
\saveTG{𪞂}{26214}
\saveTG{𤽸}{26214}
\saveTG{䑟}{26214}
\saveTG{𩴮}{26214}
\saveTG{𠏄}{26214}
\saveTG{𩳪}{26214}
\saveTG{䰦}{26214}
\saveTG{𩵈}{26214}
\saveTG{𩲻}{26214}
\saveTG{侱}{26214}
\saveTG{徎}{26214}
\saveTG{徨}{26214}
\saveTG{㒦}{26214}
\saveTG{偟}{26214}
\saveTG{儸}{26215}
\saveTG{貍}{26215}
\saveTG{俚}{26215}
\saveTG{忂}{26215}
\saveTG{𩴫}{26215}
\saveTG{𫙎}{26215}
\saveTG{𤖃}{26215}
\saveTG{𩲳}{26215}
\saveTG{𠓀}{26216}
\saveTG{𩴼}{26216}
\saveTG{𠐌}{26217}
\saveTG{㑆}{26217}
\saveTG{𠋨}{26217}
\saveTG{𧡞}{26217}
\saveTG{𪝚}{26217}
\saveTG{𠏁}{26217}
\saveTG{𧢒}{26217}
\saveTG{𠒇}{26217}
\saveTG{𠒆}{26217}
\saveTG{𩳈}{26217}
\saveTG{𧠆}{26217}
\saveTG{𪖻}{26217}
\saveTG{𪕿}{26217}
\saveTG{臰}{26217}
\saveTG{𧠺}{26217}
\saveTG{𤽕}{26217}
\saveTG{𤾇}{26217}
\saveTG{𩲸}{26217}
\saveTG{𤽇}{26217}
\saveTG{𢅉}{26217}
\saveTG{㿡}{26217}
\saveTG{𤽨}{26217}
\saveTG{𨛸}{26217}
\saveTG{𧣤}{26217}
\saveTG{𨞦}{26217}
\saveTG{𧢘}{26217}
\saveTG{𪺎}{26217}
\saveTG{𧢧}{26217}
\saveTG{𫌠}{26217}
\saveTG{𣅏}{26217}
\saveTG{𩴛}{26217}
\saveTG{𩴥}{26217}
\saveTG{覰}{26217}
\saveTG{𧢏}{26217}
\saveTG{䚁}{26217}
\saveTG{俋}{26217}
\saveTG{𨝅}{26217}
\saveTG{𤾒}{26217}
\saveTG{𨟩}{26217}
\saveTG{𨛷}{26217}
\saveTG{𨝞}{26217}
\saveTG{𠨪}{26217}
\saveTG{𣆨}{26217}
\saveTG{𦛞}{26217}
\saveTG{𠊟}{26217}
\saveTG{𧡖}{26217}
\saveTG{𩲏}{26217}
\saveTG{𧠬}{26217}
\saveTG{𩴉}{26218}
\saveTG{𩳦}{26218}
\saveTG{𩳵}{26218}
\saveTG{𩴜}{26218}
\saveTG{𧇮}{26219}
\saveTG{𤽃}{26220}
\saveTG{𣇑}{26220}
\saveTG{𧴃}{26221}
\saveTG{𧤃}{26221}
\saveTG{𪖸}{26221}
\saveTG{𪖤}{26221}
\saveTG{𠏿}{26221}
\saveTG{㑭}{26221}
\saveTG{鼻}{26221}
\saveTG{𧳠}{26221}
\saveTG{𦤒}{26222}
\saveTG{𥚷}{26226}
\saveTG{偒}{26227}
\saveTG{觸}{26227}
\saveTG{𠎯}{26227}
\saveTG{帛}{26227}
\saveTG{𢇋}{26227}
\saveTG{𣆄}{26227}
\saveTG{𦙃}{26227}
\saveTG{𫌰}{26227}
\saveTG{𧤉}{26227}
\saveTG{𦤔}{26227}
\saveTG{𠉌}{26227}
\saveTG{㑥}{26227}
\saveTG{儩}{26227}
\saveTG{臱}{26227}
\saveTG{𠐈}{26227}
\saveTG{齃}{26227}
\saveTG{𢔥}{26227}
\saveTG{𠏑}{26227}
\saveTG{𪜯}{26227}
\saveTG{䚤}{26227}
\saveTG{偈}{26227}
\saveTG{𠜬}{26227}
\saveTG{𦛉}{26227}
\saveTG{㒙}{26227}
\saveTG{𠌳}{26227}
\saveTG{傝}{26227}
\saveTG{𢕨}{26227}
\saveTG{偶}{26227}
\saveTG{𧳪}{26227}
\saveTG{𤾜}{26227}
\saveTG{侽}{26227}
\saveTG{𧳼}{26227}
\saveTG{𦙪}{26227}
\saveTG{㡍}{26227}
\saveTG{𦤝}{26227}
\saveTG{偔}{26227}
\saveTG{㑩}{26227}
\saveTG{㒔}{26227}
\saveTG{𤽟}{26228}
\saveTG{𪽐}{26228}
\saveTG{𪺠}{26230}
\saveTG{傯}{26230}
\saveTG{偲}{26230}
\saveTG{𤽄}{26230}
\saveTG{𠎦}{26231}
\saveTG{𧴔}{26231}
\saveTG{𠎁}{26231}
\saveTG{𧟍}{26231}
\saveTG{𤆁}{26232}
\saveTG{𤲄}{26232}
\saveTG{𦤌}{26232}
\saveTG{𤰲}{26232}
\saveTG{𪝢}{26232}
\saveTG{𧱋}{26232}
\saveTG{𧤖}{26232}
\saveTG{𣶚}{26232}
\saveTG{𪫑}{26232}
\saveTG{䚪}{26232}
\saveTG{𠑟}{26232}
\saveTG{𣹻}{26232}
\saveTG{𢕼}{26232}
\saveTG{𠐛}{26232}
\saveTG{𠏋}{26232}
\saveTG{偎}{26232}
\saveTG{𣳻}{26232}
\saveTG{儇}{26232}
\saveTG{𣎕}{26232}
\saveTG{𤀁}{26232}
\saveTG{𤽡}{26233}
\saveTG{儑}{26233}
\saveTG{𠈉}{26237}
\saveTG{𢖚}{26238}
\saveTG{𧴣}{26238}
\saveTG{貏}{26240}
\saveTG{俾}{26240}
\saveTG{𠈷}{26240}
\saveTG{偮}{26241}
\saveTG{得}{26241}
\saveTG{貋}{26241}
\saveTG{䐕}{26241}
\saveTG{𢕚}{26242}
\saveTG{㣬}{26242}
\saveTG{𧇤}{26242}
\saveTG{𧤏}{26242}
\saveTG{𢓴}{26242}
\saveTG{𢔶}{26243}
\saveTG{𪖭}{26243}
\saveTG{𢔽}{26243}
\saveTG{𪝯}{26243}
\saveTG{𠊚}{26243}
\saveTG{䚟}{26243}
\saveTG{𧳴}{26244}
\saveTG{𧤨}{26244}
\saveTG{䙬}{26244}
\saveTG{𢖠}{26244}
\saveTG{𪝼}{26244}
\saveTG{𪜿}{26244}
\saveTG{𠉶}{26244}
\saveTG{𢔌}{26245}
\saveTG{䚜}{26245}
\saveTG{𢔠}{26247}
\saveTG{𠊗}{26247}
\saveTG{傻}{26247}
\saveTG{𠌟}{26247}
\saveTG{𢔕}{26247}
\saveTG{䝢}{26247}
\saveTG{𢩤}{26247}
\saveTG{𧤸}{26247}
\saveTG{𠋍}{26247}
\saveTG{𠋻}{26247}
\saveTG{貜}{26247}
\saveTG{僈}{26247}
\saveTG{𢖦}{26247}
\saveTG{𠑩}{26247}
\saveTG{儼}{26248}
\saveTG{𠉖}{26248}
\saveTG{䝥}{26249}
\saveTG{𠇺}{26250}
\saveTG{𠌫}{26254}
\saveTG{𢕏}{26254}
\saveTG{僤}{26256}
\saveTG{軃}{26256}
\saveTG{貚}{26256}
\saveTG{觶}{26256}
\saveTG{𤽳}{26260}
\saveTG{𢔒}{26260}
\saveTG{倡}{26260}
\saveTG{偘}{26260}
\saveTG{躳}{26260}
\saveTG{儡}{26260}
\saveTG{侣}{26260}
\saveTG{𦝕}{26260}
\saveTG{𠌜}{26261}
\saveTG{𧧹}{26261}
\saveTG{𠏘}{26261}
\saveTG{𠐠}{26261}
\saveTG{𠐨}{26261}
\saveTG{侶}{26262}
\saveTG{𠏲}{26264}
\saveTG{𡗈}{26266}
\saveTG{𠑪}{26268}
\saveTG{伿}{26280}
\saveTG{𧲻}{26280}
\saveTG{𧳒}{26280}
\saveTG{促}{26281}
\saveTG{徥}{26281}
\saveTG{偍}{26281}
\saveTG{倶}{26281}
\saveTG{䅠}{26281}
\saveTG{𧳖}{26282}
\saveTG{𠍸}{26282}
\saveTG{𣄍}{26282}
\saveTG{䚣}{26282}
\saveTG{𧿗}{26282}
\saveTG{𧣫}{26282}
\saveTG{𢓡}{26282}
\saveTG{㑨}{26284}
\saveTG{俣}{26284}
\saveTG{齅}{26284}
\saveTG{𠋬}{26284}
\saveTG{俁}{26284}
\saveTG{𦝳}{26284}
\saveTG{傊}{26286}
\saveTG{𧳷}{26286}
\saveTG{傫}{26293}
\saveTG{儽}{26293}
\saveTG{倮}{26294}
\saveTG{𠑔}{26294}
\saveTG{𧴜}{26294}
\saveTG{保}{26294}
\saveTG{𠋸}{26294}
\saveTG{僺}{26294}
\saveTG{躶}{26294}
\saveTG{𦤞}{26294}
\saveTG{㚌}{26295}
\saveTG{𤖇}{26295}
\saveTG{𢒬}{26296}
\saveTG{𠎠}{26296}
\saveTG{𪻌}{26299}
\saveTG{𢖔}{26299}
\saveTG{儤}{26299}
\saveTG{忁}{26299}
\saveTG{𧳕}{26299}
\saveTG{𧤩}{26299}
\saveTG{𫙶}{26300}
\saveTG{𩶖}{26300}
\saveTG{𩼯}{26300}
\saveTG{𩶛}{26300}
\saveTG{𩸴}{26300}
\saveTG{𪅦}{26300}
\saveTG{𪀁}{26300}
\saveTG{鯝}{26300}
\saveTG{鮂}{26300}
\saveTG{鮰}{26300}
\saveTG{𩶾}{26300}
\saveTG{鮊}{26302}
\saveTG{䱇}{26310}
\saveTG{鰮}{26312}
\saveTG{鯤}{26312}
\saveTG{鰛}{26312}
\saveTG{鰉}{26314}
\saveTG{鯹}{26315}
\saveTG{𩽩}{26315}
\saveTG{鯉}{26315}
\saveTG{𩽰}{26315}
\saveTG{䲅}{26317}
\saveTG{𪁗}{26317}
\saveTG{𪄔}{26317}
\saveTG{𪂳}{26317}
\saveTG{䱒}{26317}
\saveTG{𤑎}{26317}
\saveTG{𩹻}{26317}
\saveTG{𩹷}{26317}
\saveTG{𩺣}{26317}
\saveTG{𪇆}{26323}
\saveTG{𪃁}{26327}
\saveTG{鰐}{26327}
\saveTG{𩷫}{26327}
\saveTG{𩸠}{26327}
\saveTG{𪃼}{26327}
\saveTG{𩸊}{26327}
\saveTG{𩺑}{26327}
\saveTG{𩹂}{26327}
\saveTG{𩻠}{26327}
\saveTG{𩻋}{26327}
\saveTG{𩹡}{26327}
\saveTG{鯣}{26327}
\saveTG{鰅}{26327}
\saveTG{鰨}{26327}
\saveTG{鰑}{26327}
\saveTG{𩽞}{26327}
\saveTG{𩷪}{26327}
\saveTG{𪄛}{26330}
\saveTG{鰓}{26330}
\saveTG{恖}{26330}
\saveTG{息}{26330}
\saveTG{惒}{26330}
\saveTG{𢘣}{26330}
\saveTG{𢡔}{26330}
\saveTG{悤}{26330}
\saveTG{憩}{26330}
\saveTG{㦝}{26331}
\saveTG{𩻤}{26331}
\saveTG{𩼞}{26331}
\saveTG{𪹩}{26331}
\saveTG{㦟}{26331}
\saveTG{𩺩}{26331}
\saveTG{𩻄}{26332}
\saveTG{𢡕}{26332}
\saveTG{𢠙}{26332}
\saveTG{𢞫}{26332}
\saveTG{䴋}{26332}
\saveTG{鱞}{26332}
\saveTG{𩽲}{26332}
\saveTG{鰃}{26332}
\saveTG{𩹌}{26332}
\saveTG{𥄉}{26337}
\saveTG{𤿄}{26337}
\saveTG{𡿻}{26337}
\saveTG{㷛}{26339}
\saveTG{𢤁}{26339}
\saveTG{鵿}{26340}
\saveTG{𩼓}{26341}
\saveTG{𩹫}{26342}
\saveTG{𩷥}{26343}
\saveTG{𩷑}{26344}
\saveTG{𩹽}{26344}
\saveTG{𩽢}{26344}
\saveTG{𪈤}{26344}
\saveTG{䱝}{26345}
\saveTG{𪂃}{26345}
\saveTG{𩻙}{26347}
\saveTG{𩻐}{26347}
\saveTG{鰻}{26347}
\saveTG{𩽴}{26348}
\saveTG{魻}{26350}
\saveTG{𪀌}{26350}
\saveTG{𩽣}{26351}
\saveTG{𩺷}{26354}
\saveTG{鱓}{26356}
\saveTG{鯧}{26360}
\saveTG{鱰}{26364}
\saveTG{𩻂}{26364}
\saveTG{鱪}{26364}
\saveTG{䳅}{26380}
\saveTG{鯷}{26381}
\saveTG{𪂿}{26382}
\saveTG{𩶭}{26384}
\saveTG{鰁}{26392}
\saveTG{𪄈}{26394}
\saveTG{𪈫}{26394}
\saveTG{鱢}{26394}
\saveTG{𩸄}{26395}
\saveTG{𩻱}{26396}
\saveTG{鸔}{26399}
\saveTG{鰥}{26399}
\saveTG{𦨙}{26400}
\saveTG{𨉹}{26400}
\saveTG{卑}{26400}
\saveTG{皁}{26400}
\saveTG{𦨧}{26400}
\saveTG{睾}{26401}
\saveTG{𩲽}{26401}
\saveTG{𥄁}{26401}
\saveTG{辠}{26401}
\saveTG{舶}{26402}
\saveTG{𨈻}{26402}
\saveTG{𪥷}{26404}
\saveTG{𪥴}{26404}
\saveTG{𤾁}{26405}
\saveTG{𦤂}{26407}
\saveTG{𦤁}{26407}
\saveTG{𡕩}{26407}
\saveTG{皋}{26408}
\saveTG{臯}{26409}
\saveTG{皐}{26409}
\saveTG{𥇽}{26410}
\saveTG{𥇅}{26410}
\saveTG{𦨪}{26410}
\saveTG{𣆶}{26410}
\saveTG{𧠾}{26412}
\saveTG{𧡩}{26412}
\saveTG{𧡓}{26412}
\saveTG{覣}{26412}
\saveTG{魏}{26413}
\saveTG{艎}{26414}
\saveTG{𨉽}{26414}
\saveTG{𦪄}{26414}
\saveTG{𦪷}{26414}
\saveTG{𨉑}{26414}
\saveTG{艃}{26415}
\saveTG{𦩠}{26415}
\saveTG{𧡦}{26417}
\saveTG{䚀}{26417}
\saveTG{䚓}{26417}
\saveTG{𧢌}{26417}
\saveTG{䚌}{26417}
\saveTG{𨈺}{26417}
\saveTG{𨉵}{26417}
\saveTG{𨉮}{26417}
\saveTG{𤽮}{26418}
\saveTG{𨊀}{26421}
\saveTG{䑽}{26427}
\saveTG{𦩥}{26427}
\saveTG{艊}{26427}
\saveTG{𠢎}{26427}
\saveTG{𢅁}{26427}
\saveTG{𨊒}{26427}
\saveTG{𤽘}{26427}
\saveTG{𨉪}{26427}
\saveTG{𦩝}{26427}
\saveTG{𦩭}{26430}
\saveTG{𨊂}{26431}
\saveTG{𦪐}{26436}
\saveTG{𢍙}{26440}
\saveTG{𢍍}{26440}
\saveTG{𢍘}{26440}
\saveTG{𢍉}{26440}
\saveTG{𢍄}{26440}
\saveTG{𢍂}{26440}
\saveTG{𦩅}{26441}
\saveTG{𠧄}{26441}
\saveTG{𤽒}{26444}
\saveTG{𦩖}{26445}
\saveTG{𦫇}{26447}
\saveTG{𫇞}{26447}
\saveTG{舺}{26450}
\saveTG{𦪢}{26456}
\saveTG{艒}{26460}
\saveTG{𨉍}{26480}
\saveTG{𦤖}{26480}
\saveTG{𨈪}{26480}
\saveTG{𨉌}{26482}
\saveTG{𨉛}{26484}
\saveTG{𦨼}{26484}
\saveTG{𦨳}{26484}
\saveTG{𨊋}{26486}
\saveTG{𦩪}{26494}
\saveTG{牭}{26500}
\saveTG{㸶}{26500}
\saveTG{𤘘}{26500}
\saveTG{犩}{26501}
\saveTG{𢫗}{26502}
\saveTG{𦤑}{26508}
\saveTG{犤}{26512}
\saveTG{𤛁}{26512}
\saveTG{鬼}{26513}
\saveTG{𤚝}{26514}
\saveTG{魓}{26515}
\saveTG{𤙧}{26517}
\saveTG{𤛂}{26517}
\saveTG{𤚪}{26517}
\saveTG{𨛝}{26517}
\saveTG{𪺬}{26517}
\saveTG{𤽵}{26517}
\saveTG{𤙝}{26517}
\saveTG{𢬚}{26527}
\saveTG{𦤥}{26527}
\saveTG{㹇}{26527}
\saveTG{𤚺}{26527}
\saveTG{𤛯}{26527}
\saveTG{𤜑}{26531}
\saveTG{𤙡}{26541}
\saveTG{㹄}{26542}
\saveTG{𤙰}{26543}
\saveTG{𤜉}{26544}
\saveTG{𤛔}{26547}
\saveTG{𤙇}{26550}
\saveTG{𦤛}{26551}
\saveTG{𤽛}{26555}
\saveTG{𤜈}{26561}
\saveTG{𤜋}{26561}
\saveTG{𤜊}{26561}
\saveTG{𤚞}{26572}
\saveTG{㸽}{26580}
\saveTG{𤙬}{26584}
\saveTG{𤚯}{26584}
\saveTG{犑}{26584}
\saveTG{𤛬}{26586}
\saveTG{㹎}{26593}
\saveTG{𤜖}{26593}
\saveTG{𤜌}{26594}
\saveTG{犦}{26599}
\saveTG{𦤏}{26600}
\saveTG{𣉭}{26600}
\saveTG{畠}{26600}
\saveTG{諐}{26601}
\saveTG{𧨍}{26601}
\saveTG{𧫏}{26601}
\saveTG{㿟}{26602}
\saveTG{𥗶}{26602}
\saveTG{𤽚}{26603}
\saveTG{𤽙}{26605}
\saveTG{䭳}{26609}
\saveTG{𠿲}{26610}
\saveTG{𦤫}{26612}
\saveTG{𫜊}{26612}
\saveTG{𪉸}{26612}
\saveTG{覘}{26612}
\saveTG{馧}{26612}
\saveTG{𧡳}{26612}
\saveTG{𧠜}{26612}
\saveTG{魄}{26613}
\saveTG{𧠼}{26617}
\saveTG{𧢕}{26617}
\saveTG{䚃}{26617}
\saveTG{䴜}{26617}
\saveTG{舓}{26627}
\saveTG{馤}{26627}
\saveTG{𫜉}{26627}
\saveTG{㿣}{26627}
\saveTG{𤽲}{26627}
\saveTG{𪿹}{26627}
\saveTG{𤾉}{26627}
\saveTG{𡀻}{26627}
\saveTG{𦧯}{26630}
\saveTG{𪉭}{26632}
\saveTG{𫇋}{26632}
\saveTG{皔}{26641}
\saveTG{𤾍}{26643}
\saveTG{𤽹}{26645}
\saveTG{𪉽}{26647}
\saveTG{䴝}{26647}
\saveTG{皞}{26648}
\saveTG{皡}{26648}
\saveTG{皥}{26648}
\saveTG{𤾎}{26652}
\saveTG{𤾠}{26656}
\saveTG{𤽻}{26658}
\saveTG{𪉨}{26660}
\saveTG{皛}{26662}
\saveTG{𩡗}{26684}
\saveTG{𦧙}{26692}
\saveTG{𤾚}{26694}
\saveTG{𡹎}{26700}
\saveTG{𡶚}{26700}
\saveTG{𡻢}{26700}
\saveTG{𡻽}{26700}
\saveTG{𪘩}{26700}
\saveTG{𡶐}{26700}
\saveTG{}{26700}
\saveTG{崓}{26700}
\saveTG{𡼱}{26700}
\saveTG{𡸙}{26700}
\saveTG{齫}{26700}
\saveTG{岶}{26702}
\saveTG{𡶯}{26702}
\saveTG{𤼾}{26710}
\saveTG{𣬣}{26710}
\saveTG{𣬰}{26710}
\saveTG{𣭒}{26710}
\saveTG{𣮎}{26711}
\saveTG{𣰽}{26711}
\saveTG{𣱁}{26711}
\saveTG{𣱃}{26711}
\saveTG{𦓌}{26711}
\saveTG{𣮶}{26711}
\saveTG{嶵}{26711}
\saveTG{𧡾}{26712}
\saveTG{㲩}{26712}
\saveTG{𣮨}{26712}
\saveTG{皀}{26712}
\saveTG{𣮷}{26712}
\saveTG{𦫖}{26712}
\saveTG{峴}{26712}
\saveTG{覒}{26712}
\saveTG{崐}{26712}
\saveTG{岲}{26712}
\saveTG{馄}{26712}
\saveTG{𣭋}{26712}
\saveTG{𪙳}{26712}
\saveTG{𣱀}{26712}
\saveTG{毸}{26713}
\saveTG{𫗮}{26714}
\saveTG{𪘄}{26714}
\saveTG{𪘟}{26714}
\saveTG{皂}{26714}
\saveTG{𣭗}{26714}
\saveTG{崲}{26714}
\saveTG{㠥}{26714}
\saveTG{𣮐}{26714}
\saveTG{𫄺}{26714}
\saveTG{𡿇}{26715}
\saveTG{䭪}{26715}
\saveTG{𣯴}{26715}
\saveTG{𣮹}{26716}
\saveTG{𣮑}{26716}
\saveTG{𧢟}{26717}
\saveTG{𧡑}{26717}
\saveTG{𨜲}{26717}
\saveTG{𣰗}{26717}
\saveTG{𠃼}{26717}
\saveTG{𧡹}{26717}
\saveTG{𤱖}{26717}
\saveTG{𤭭}{26717}
\saveTG{𩱾}{26717}
\saveTG{𣯝}{26717}
\saveTG{𪓨}{26717}
\saveTG{𣬉}{26717}
\saveTG{𤽏}{26717}
\saveTG{㟴}{26717}
\saveTG{崐}{26717}
\saveTG{𩲖}{26717}
\saveTG{𣭓}{26718}
\saveTG{𣯒}{26718}
\saveTG{𣌣}{26718}
\saveTG{𨰒}{26719}
\saveTG{𣰕}{26719}
\saveTG{𣯂}{26719}
\saveTG{𣮃}{26719}
\saveTG{𡽶}{26721}
\saveTG{𪖵}{26721}
\saveTG{𪨹}{26727}
\saveTG{𡷡}{26727}
\saveTG{𪘹}{26727}
\saveTG{}{26727}
\saveTG{𪽕}{26727}
\saveTG{嵑}{26727}
\saveTG{㡫}{26727}
\saveTG{嵎}{26727}
\saveTG{皍}{26727}
\saveTG{崿}{26727}
\saveTG{崵}{26727}
\saveTG{𡼑}{26727}
\saveTG{齵}{26727}
\saveTG{𡹾}{26727}
\saveTG{齶}{26727}
\saveTG{𪩊}{26730}
\saveTG{𪙋}{26730}
\saveTG{㿝}{26731}
\saveTG{𢆿}{26731}
\saveTG{𦤆}{26731}
\saveTG{𫗳}{26731}
\saveTG{𪐽}{26731}
\saveTG{𤽎}{26731}
\saveTG{𧝛}{26732}
\saveTG{㟫}{26732}
\saveTG{𤱮}{26732}
\saveTG{𫗭}{26732}
\saveTG{𧛱}{26732}
\saveTG{𧚡}{26732}
\saveTG{𠃁}{26732}
\saveTG{𠂽}{26732}
\saveTG{㟪}{26732}
\saveTG{𡹯}{26740}
\saveTG{𡹋}{26740}
\saveTG{崥}{26740}
\saveTG{𡽁}{26741}
\saveTG{嶧}{26741}
\saveTG{𡼗}{26741}
\saveTG{𡷛}{26741}
\saveTG{䜥}{26741}
\saveTG{𡹮}{26744}
\saveTG{巊}{26744}
\saveTG{𠭛}{26747}
\saveTG{馒}{26747}
\saveTG{𡻩}{26747}
\saveTG{𡻆}{26748}
\saveTG{㟸}{26748}
\saveTG{巗}{26748}
\saveTG{䶫}{26748}
\saveTG{岬}{26750}
\saveTG{𡻇}{26752}
\saveTG{𡻞}{26754}
\saveTG{㠆}{26756}
\saveTG{𪙣}{26756}
\saveTG{𡾊}{26760}
\saveTG{𩲈}{26771}
\saveTG{𡿁}{26771}
\saveTG{𠂺}{26774}
\saveTG{𦤍}{26774}
\saveTG{齞}{26780}
\saveTG{龊}{26781}
\saveTG{齪}{26781}
\saveTG{崼}{26781}
\saveTG{㔭}{26782}
\saveTG{𡷿}{26782}
\saveTG{𡹊}{26784}
\saveTG{𡷤}{26784}
\saveTG{𨞑}{26784}
\saveTG{𡻖}{26786}
\saveTG{𡻱}{26793}
\saveTG{𡿔}{26793}
\saveTG{𪳦}{26794}
\saveTG{嵲}{26794}
\saveTG{馃}{26794}
\saveTG{𡸖}{26795}
\saveTG{呉}{26801}
\saveTG{臭}{26804}
\saveTG{吳}{26804}
\saveTG{𣅔}{26804}
\saveTG{𤾪}{26804}
\saveTG{㚖}{26804}
\saveTG{𣅳}{26804}
\saveTG{𧵯}{26806}
\saveTG{𪽾}{26806}
\saveTG{賲}{26806}
\saveTG{𤽬}{26808}
\saveTG{煲}{26809}
\saveTG{𤽈}{26809}
\saveTG{𦤨}{26812}
\saveTG{𧡠}{26812}
\saveTG{𨜁}{26817}
\saveTG{𩲫}{26817}
\saveTG{𦠺}{26817}
\saveTG{𦤡}{26817}
\saveTG{𦤦}{26827}
\saveTG{𥐂}{26827}
\saveTG{𪈪}{26827}
\saveTG{臮}{26832}
\saveTG{𠤨}{26841}
\saveTG{𦤀}{26842}
\saveTG{𦤈}{26882}
\saveTG{𫚫}{26894}
\saveTG{和}{26900}
\saveTG{絗}{26900}
\saveTG{稇}{26900}
\saveTG{稛}{26900}
\saveTG{𫃜}{26900}
\saveTG{𥝿}{26900}
\saveTG{𥞚}{26900}
\saveTG{䵒}{26900}
\saveTG{𦀌}{26900}
\saveTG{𥿀}{26900}
\saveTG{𦄰}{26900}
\saveTG{絪}{26900}
\saveTG{𦀶}{26900}
\saveTG{𦀄}{26900}
\saveTG{𦂮}{26900}
\saveTG{𥿖}{26900}
\saveTG{細}{26900}
\saveTG{緗}{26900}
\saveTG{秵}{26900}
\saveTG{綑}{26900}
\saveTG{𦀝}{26900}
\saveTG{稒}{26900}
\saveTG{県}{26901}
\saveTG{䄸}{26902}
\saveTG{泉}{26902}
\saveTG{𦀪}{26902}
\saveTG{𦀙}{26902}
\saveTG{𥿳}{26902}
\saveTG{絈}{26902}
\saveTG{𣑞}{26904}
\saveTG{臬}{26904}
\saveTG{䋎}{26910}
\saveTG{𦤗}{26911}
\saveTG{𦈅}{26912}
\saveTG{𥠺}{26912}
\saveTG{𦁻}{26912}
\saveTG{𠬓}{26912}
\saveTG{緄}{26912}
\saveTG{縕}{26912}
\saveTG{緼}{26912}
\saveTG{絸}{26912}
\saveTG{縨}{26912}
\saveTG{䌉}{26912}
\saveTG{覙}{26912}
\saveTG{繉}{26913}
\saveTG{程}{26914}
\saveTG{纆}{26914}
\saveTG{𫀴}{26914}
\saveTG{𦄇}{26914}
\saveTG{䅣}{26914}
\saveTG{𦀚}{26914}
\saveTG{𥤐}{26914}
\saveTG{𦄆}{26915}
\saveTG{𦆊}{26915}
\saveTG{𥠀}{26915}
\saveTG{𩵇}{26915}
\saveTG{纙}{26915}
\saveTG{䋥}{26915}
\saveTG{縄}{26916}
\saveTG{𦆘}{26917}
\saveTG{𦃋}{26917}
\saveTG{䋲}{26917}
\saveTG{𦃰}{26917}
\saveTG{𦀕}{26917}
\saveTG{絸}{26917}
\saveTG{縨}{26917}
\saveTG{𥞅}{26917}
\saveTG{𧠪}{26917}
\saveTG{䅐}{26917}
\saveTG{𥠎}{26917}
\saveTG{𨝠}{26917}
\saveTG{䅙}{26917}
\saveTG{䆉}{26917}
\saveTG{𥞷}{26917}
\saveTG{𦀸}{26917}
\saveTG{𦃻}{26917}
\saveTG{𥡂}{26917}
\saveTG{䌆}{26917}
\saveTG{𥞏}{26917}
\saveTG{𥫅}{26918}
\saveTG{𦂢}{26921}
\saveTG{綥}{26921}
\saveTG{𦇧}{26921}
\saveTG{穆}{26922}
\saveTG{𥠇}{26922}
\saveTG{𥢣}{26922}
\saveTG{𥡆}{26922}
\saveTG{䅥}{26924}
\saveTG{絹}{26927}
\saveTG{綿}{26927}
\saveTG{緭}{26927}
\saveTG{緆}{26927}
\saveTG{𥟘}{26927}
\saveTG{𥞡}{26927}
\saveTG{𥟄}{26927}
\saveTG{𥣋}{26927}
\saveTG{𥡐}{26927}
\saveTG{𥣰}{26927}
\saveTG{𦆅}{26927}
\saveTG{𦂕}{26927}
\saveTG{䌈}{26927}
\saveTG{䋵}{26927}
\saveTG{𦆂}{26927}
\saveTG{𥠜}{26927}
\saveTG{𥟮}{26927}
\saveTG{𥣄}{26927}
\saveTG{䵘}{26927}
\saveTG{𦄡}{26927}
\saveTG{𦂝}{26927}
\saveTG{𦆒}{26927}
\saveTG{稩}{26927}
\saveTG{𥡥}{26930}
\saveTG{緦}{26930}
\saveTG{總}{26930}
\saveTG{𦃞}{26930}
\saveTG{𦃿}{26931}
\saveTG{𪒚}{26931}
\saveTG{𥡊}{26931}
\saveTG{𥡋}{26931}
\saveTG{䆀}{26931}
\saveTG{𦄿}{26931}
\saveTG{𦇏}{26932}
\saveTG{䋿}{26932}
\saveTG{𦂸}{26932}
\saveTG{𥠘}{26932}
\saveTG{繯}{26932}
\saveTG{𡣉}{26932}
\saveTG{𥤓}{26933}
\saveTG{𦆴}{26933}
\saveTG{𥤍}{26936}
\saveTG{𦇭}{26936}
\saveTG{繦}{26936}
\saveTG{𥟑}{26940}
\saveTG{𦂆}{26940}
\saveTG{稗}{26940}
\saveTG{𣐩}{26940}
\saveTG{綼}{26940}
\saveTG{繹}{26941}
\saveTG{𦆎}{26941}
\saveTG{𨤟}{26941}
\saveTG{䆁}{26941}
\saveTG{釋}{26941}
\saveTG{緝}{26941}
\saveTG{稈}{26941}
\saveTG{𥠋}{26942}
\saveTG{𦃩}{26942}
\saveTG{䅞}{26943}
\saveTG{𦂖}{26944}
\saveTG{𦃍}{26944}
\saveTG{𦅳}{26944}
\saveTG{纓}{26944}
\saveTG{𪐄}{26945}
\saveTG{𥣊}{26946}
\saveTG{縵}{26947}
\saveTG{䅼}{26947}
\saveTG{𥠰}{26947}
\saveTG{𥢄}{26947}
\saveTG{稷}{26947}
\saveTG{繌}{26947}
\saveTG{繓}{26947}
\saveTG{𣠏}{26947}
\saveTG{穝}{26947}
\saveTG{𥤘}{26947}
\saveTG{𥡅}{26948}
\saveTG{𥢐}{26949}
\saveTG{䊬}{26949}
\saveTG{縪}{26954}
\saveTG{𥣐}{26956}
\saveTG{繟}{26956}
\saveTG{𦅿}{26956}
\saveTG{稆}{26960}
\saveTG{䅛}{26960}
\saveTG{𥡏}{26960}
\saveTG{絽}{26960}
\saveTG{𦇄}{26960}
\saveTG{𦃧}{26962}
\saveTG{𦄛}{26966}
\saveTG{积}{26980}
\saveTG{𥿗}{26980}
\saveTG{𦁀}{26980}
\saveTG{穓}{26981}
\saveTG{緹}{26981}
\saveTG{䅪}{26982}
\saveTG{𥞺}{26982}
\saveTG{𫃥}{26982}
\saveTG{𥠝}{26984}
\saveTG{𥟔}{26984}
\saveTG{縜}{26986}
\saveTG{灥}{26992}
\saveTG{線}{26992}
\saveTG{纝}{26993}
\saveTG{縲}{26993}
\saveTG{䆆}{26994}
\saveTG{緥}{26994}
\saveTG{稞}{26994}
\saveTG{綶}{26994}
\saveTG{繰}{26994}
\saveTG{𣕷}{26994}
\saveTG{𤿁}{26996}
\saveTG{𦅡}{26996}
\saveTG{𢑾}{26999}
\saveTG{𦆿}{26999}
\saveTG{𤗖}{27012}
\saveTG{𤖫}{27017}
\saveTG{㸥}{27017}
\saveTG{𠁢}{27017}
\saveTG{勹}{27020}
\saveTG{𠁡}{27020}
\saveTG{}{27020}
\saveTG{𠧓}{27021}
\saveTG{𠚬}{27021}
\saveTG{𤖮}{27023}
\saveTG{𤖾}{27026}
\saveTG{𤖽}{27026}
\saveTG{𤖵}{27026}
\saveTG{鳪}{27027}
\saveTG{帰}{27027}
\saveTG{𧣉}{27027}
\saveTG{𤖻}{27027}
\saveTG{𤗓}{27032}
\saveTG{牎}{27032}
\saveTG{𢶂}{27035}
\saveTG{𤗜}{27047}
\saveTG{𤖬}{27047}
\saveTG{𤖰}{27047}
\saveTG{𤗻}{27061}
\saveTG{牊}{27062}
\saveTG{𤗨}{27062}
\saveTG{𤗌}{27064}
\saveTG{归}{27070}
\saveTG{韰}{27101}
\saveTG{𥁽}{27102}
\saveTG{𥁤}{27102}
\saveTG{𥁙}{27102}
\saveTG{𥂃}{27102}
\saveTG{𪾐}{27102}
\saveTG{𪾋}{27102}
\saveTG{彑}{27102}
\saveTG{盠}{27102}
\saveTG{盘}{27102}
\saveTG{盤}{27102}
\saveTG{盌}{27102}
\saveTG{血}{27102}
\saveTG{𪾎}{27102}
\saveTG{玺}{27103}
\saveTG{𡒰}{27103}
\saveTG{𡐩}{27104}
\saveTG{𡺻}{27104}
\saveTG{墏}{27104}
\saveTG{墾}{27104}
\saveTG{坚}{27104}
\saveTG{壑}{27104}
\saveTG{𡎙}{27104}
\saveTG{𪤈}{27104}
\saveTG{𡑶}{27104}
\saveTG{𡓝}{27104}
\saveTG{𡉢}{27104}
\saveTG{𡋹}{27104}
\saveTG{㺱}{27104}
\saveTG{𡓬}{27104}
\saveTG{𪢶}{27104}
\saveTG{垼}{27104}
\saveTG{㺸}{27104}
\saveTG{𡋜}{27104}
\saveTG{𡉲}{27104}
\saveTG{𡔟}{27104}
\saveTG{㙰}{27104}
\saveTG{𡊂}{27104}
\saveTG{𤧧}{27104}
\saveTG{𤩤}{27104}
\saveTG{𤪰}{27104}
\saveTG{𩵘}{27104}
\saveTG{𨤣}{27105}
\saveTG{㚅}{27105}
\saveTG{𫒂}{27105}
\saveTG{鱼}{27106}
\saveTG{𫚞}{27106}
\saveTG{𤖛}{27106}
\saveTG{䖤}{27106}
\saveTG{竖}{27108}
\saveTG{𧰥}{27108}
\saveTG{𧯡}{27108}
\saveTG{豋}{27108}
\saveTG{鎥}{27109}
\saveTG{𨰢}{27109}
\saveTG{𨥒}{27109}
\saveTG{𨩂}{27109}
\saveTG{錖}{27109}
\saveTG{銞}{27109}
\saveTG{鎜}{27109}
\saveTG{𨫥}{27109}
\saveTG{錅}{27109}
\saveTG{𨬽}{27109}
\saveTG{鲺}{27110}
\saveTG{𣥇}{27110}
\saveTG{颽}{27110}
\saveTG{凱}{27110}
\saveTG{𪚺}{27111}
\saveTG{经}{27112}
\saveTG{鲍}{27112}
\saveTG{鲵}{27112}
\saveTG{纽}{27112}
\saveTG{衄}{27112}
\saveTG{组}{27112}
\saveTG{}{27112}
\saveTG{𩶜}{27114}
\saveTG{鲣}{27114}
\saveTG{𨤰}{27115}
\saveTG{𪚳}{27115}
\saveTG{𪤿}{27115}
\saveTG{𣥘}{27117}
\saveTG{𪚧}{27117}
\saveTG{𪛃}{27117}
\saveTG{𢀽}{27117}
\saveTG{𩾆}{27117}
\saveTG{𩾃}{27117}
\saveTG{𩾅}{27117}
\saveTG{绝}{27117}
\saveTG{𪚵}{27117}
\saveTG{鱾}{27117}
\saveTG{𧑴}{27117}
\saveTG{𠤣}{27117}
\saveTG{𧖵}{27117}
\saveTG{𫒃}{27117}
\saveTG{𫚜}{27117}
\saveTG{纪}{27117}
\saveTG{龜}{27117}
\saveTG{鲃}{27117}
\saveTG{艷}{27117}
\saveTG{𨰶}{27119}
\saveTG{约}{27120}
\saveTG{鲷}{27120}
\saveTG{纲}{27120}
\saveTG{鲫}{27120}
\saveTG{勻}{27120}
\saveTG{匀}{27120}
\saveTG{衂}{27120}
\saveTG{纫}{27120}
\saveTG{卹}{27120}
\saveTG{绹}{27120}
\saveTG{鲖}{27120}
\saveTG{绚}{27120}
\saveTG{㔩}{27120}
\saveTG{䲟}{27120}
\saveTG{𧖰}{27120}
\saveTG{𨈏}{27120}
\saveTG{绷}{27120}
\saveTG{绸}{27120}
\saveTG{鱽}{27120}
\saveTG{}{27120}
\saveTG{𦑝}{27120}
\saveTG{}{27120}
\saveTG{𪻎}{27121}
\saveTG{𨈑}{27121}
\saveTG{𦒀}{27121}
\saveTG{䌻}{27121}
\saveTG{𠝅}{27121}
\saveTG{𪴵}{27122}
\saveTG{𧇇}{27122}
\saveTG{𠞕}{27122}
\saveTG{缪}{27122}
\saveTG{纾}{27122}
\saveTG{𧍀}{27123}
\saveTG{𠤋}{27123}
\saveTG{𠣜}{27123}
\saveTG{𠣕}{27124}
\saveTG{䌹}{27126}
\saveTG{𠣙}{27126}
\saveTG{𨤯}{27126}
\saveTG{𫄡}{27126}
\saveTG{𠣡}{27126}
\saveTG{㓏}{27126}
\saveTG{𨚄}{27127}
\saveTG{𨞄}{27127}
\saveTG{䘏}{27127}
\saveTG{𨚥}{27127}
\saveTG{𨛽}{27127}
\saveTG{𨙼}{27127}
\saveTG{𨞷}{27127}
\saveTG{𨛘}{27127}
\saveTG{𪴷}{27127}
\saveTG{乌}{27127}
\saveTG{𩾊}{27127}
\saveTG{邬}{27127}
\saveTG{邹}{27127}
\saveTG{𨜔}{27127}
\saveTG{鸟}{27127}
\saveTG{𩿃}{27127}
\saveTG{歸}{27127}
\saveTG{𪞰}{27127}
\saveTG{𨤒}{27127}
\saveTG{䲼}{27127}
\saveTG{䳠}{27127}
\saveTG{䳨}{27127}
\saveTG{𩿿}{27127}
\saveTG{酆}{27127}
\saveTG{𫚽}{27127}
\saveTG{𫛃}{27127}
\saveTG{䳄}{27127}
\saveTG{𩾰}{27127}
\saveTG{𨝭}{27127}
\saveTG{𨞞}{27127}
\saveTG{𡖰}{27127}
\saveTG{𨚖}{27127}
\saveTG{𫚨}{27127}
\saveTG{𨜗}{27127}
\saveTG{郵}{27127}
\saveTG{绑}{27127}
\saveTG{𠂶}{27127}
\saveTG{𪅈}{27127}
\saveTG{𦈔}{27127}
\saveTG{𪈿}{27127}
\saveTG{𨓼}{27127}
\saveTG{㱕}{27127}
\saveTG{𫚡}{27127}
\saveTG{𪉓}{27127}
\saveTG{𢏻}{27127}
\saveTG{𫚪}{27127}
\saveTG{𤯜}{27127}
\saveTG{𫄝}{27127}
\saveTG{鲬}{27127}
\saveTG{鸳}{27127}
\saveTG{䘀}{27131}
\saveTG{𤄩}{27131}
\saveTG{𧒒}{27131}
\saveTG{𧔔}{27131}
\saveTG{𧍸}{27131}
\saveTG{𧒂}{27131}
\saveTG{𧒙}{27131}
\saveTG{𣥦}{27132}
\saveTG{缘}{27132}
\saveTG{𧖷}{27133}
\saveTG{𪚽}{27133}
\saveTG{终}{27133}
\saveTG{缝}{27135}
\saveTG{蟹}{27136}
\saveTG{蠁}{27136}
\saveTG{鳋}{27136}
\saveTG{螿}{27136}
\saveTG{蛗}{27136}
\saveTG{螌}{27136}
\saveTG{䗍}{27136}
\saveTG{𧊰}{27136}
\saveTG{𧌛}{27136}
\saveTG{𧋰}{27136}
\saveTG{𧊒}{27136}
\saveTG{䖿}{27136}
\saveTG{𧊃}{27136}
\saveTG{𧈸}{27136}
\saveTG{𧑸}{27136}
\saveTG{𧑬}{27136}
\saveTG{蠡}{27136}
\saveTG{螽}{27136}
\saveTG{𫊱}{27136}
\saveTG{𧕼}{27136}
\saveTG{𧖢}{27136}
\saveTG{𧉏}{27136}
\saveTG{𧋴}{27136}
\saveTG{𧓖}{27136}
\saveTG{缒}{27137}
\saveTG{𨈐}{27140}
\saveTG{鲰}{27140}
\saveTG{鳉}{27142}
\saveTG{缨}{27144}
\saveTG{𡐍}{27144}
\saveTG{𧖠}{27146}
\saveTG{𣪱}{27147}
\saveTG{级}{27147}
\saveTG{缎}{27147}
\saveTG{缀}{27147}
\saveTG{𫚥}{27147}
\saveTG{𣪜}{27147}
\saveTG{𣪰}{27147}
\saveTG{鲟}{27147}
\saveTG{𠮋}{27147}
\saveTG{𠭓}{27147}
\saveTG{𠭉}{27147}
\saveTG{𡒃}{27148}
\saveTG{𨮂}{27152}
\saveTG{绎}{27154}
\saveTG{绛}{27154}
\saveTG{𦈉}{27154}
\saveTG{𡒒}{27156}
\saveTG{𩽼}{27157}
\saveTG{𧩏}{27161}
\saveTG{绍}{27162}
\saveTG{𣵥}{27162}
\saveTG{鳛}{27162}
\saveTG{𠧡}{27162}
\saveTG{缗}{27164}
\saveTG{衉}{27164}
\saveTG{络}{27164}
\saveTG{𣦠}{27164}
\saveTG{鲪}{27167}
\saveTG{𧖱}{27172}
\saveTG{𧖾}{27172}
\saveTG{𣮋}{27175}
\saveTG{绉}{27177}
\saveTG{刍}{27177}
\saveTG{𡑈}{27177}
\saveTG{𦈝}{27181}
\saveTG{㰣}{27182}
\saveTG{𪽧}{27182}
\saveTG{𪉁}{27182}
\saveTG{𪴬}{27182}
\saveTG{欰}{27182}
\saveTG{𣢡}{27182}
\saveTG{欤}{27182}
\saveTG{𣤦}{27182}
\saveTG{𣢫}{27182}
\saveTG{𣢑}{27182}
\saveTG{𣣷}{27182}
\saveTG{𣢃}{27182}
\saveTG{缑}{27184}
\saveTG{𫛺}{27184}
\saveTG{𠗫}{27184}
\saveTG{}{27191}
\saveTG{𥂟}{27192}
\saveTG{𢀟}{27192}
\saveTG{𨧮}{27194}
\saveTG{鲦}{27194}
\saveTG{绦}{27194}
\saveTG{绿}{27199}
\saveTG{夕}{27200}
\saveTG{𠄐}{27200}
\saveTG{𫆣}{27200}
\saveTG{厃}{27201}
\saveTG{𠂊}{27201}
\saveTG{𨗭}{27201}
\saveTG{𨷿}{27201}
\saveTG{𨘅}{27202}
\saveTG{𢒢}{27202}
\saveTG{亇}{27202}
\saveTG{𢪸}{27205}
\saveTG{多}{27207}
\saveTG{伊}{27207}
\saveTG{𧖪}{27207}
\saveTG{亿}{27210}
\saveTG{偑}{27210}
\saveTG{𣄯}{27210}
\saveTG{𡖕}{27210}
\saveTG{𩘠}{27210}
\saveTG{𠘷}{27210}
\saveTG{仉}{27210}
\saveTG{𩙃}{27210}
\saveTG{𢒼}{27210}
\saveTG{𠏵}{27210}
\saveTG{㑉}{27210}
\saveTG{𠆰}{27210}
\saveTG{㐽}{27210}
\saveTG{𠑒}{27210}
\saveTG{𫗅}{27210}
\saveTG{佩}{27210}
\saveTG{𪽆}{27210}
\saveTG{𥸫}{27210}
\saveTG{䬌}{27210}
\saveTG{𠂸}{27210}
\saveTG{𤖚}{27211}
\saveTG{𣨖}{27211}
\saveTG{䖕}{27211}
\saveTG{𠍈}{27211}
\saveTG{𠖡}{27211}
\saveTG{𫙍}{27211}
\saveTG{𩲲}{27211}
\saveTG{𧴙}{27212}
\saveTG{𩴹}{27212}
\saveTG{𤖗}{27212}
\saveTG{𩲐}{27212}
\saveTG{𧳱}{27212}
\saveTG{觬}{27212}
\saveTG{貎}{27212}
\saveTG{伣}{27212}
\saveTG{觑}{27212}
\saveTG{伹}{27212}
\saveTG{夗}{27212}
\saveTG{危}{27212}
\saveTG{侐}{27212}
\saveTG{𠣑}{27212}
\saveTG{㒊}{27212}
\saveTG{𢄝}{27212}
\saveTG{䰬}{27212}
\saveTG{𪀷}{27212}
\saveTG{𢕙}{27212}
\saveTG{𠏫}{27212}
\saveTG{𠍄}{27212}
\saveTG{𠇹}{27212}
\saveTG{𩲁}{27212}
\saveTG{麁}{27212}
\saveTG{𧳺}{27212}
\saveTG{𢏩}{27212}
\saveTG{佨}{27212}
\saveTG{𢕬}{27212}
\saveTG{徂}{27212}
\saveTG{凢}{27212}
\saveTG{凣}{27212}
\saveTG{俛}{27212}
\saveTG{佹}{27212}
\saveTG{虝}{27212}
\saveTG{径}{27212}
\saveTG{倪}{27212}
\saveTG{伲}{27212}
\saveTG{𩳲}{27212}
\saveTG{𩲂}{27212}
\saveTG{𩲃}{27212}
\saveTG{䰣}{27212}
\saveTG{㒠}{27212}
\saveTG{𧇌}{27212}
\saveTG{𦜕}{27212}
\saveTG{𦝚}{27212}
\saveTG{𡲂}{27212}
\saveTG{𧣬}{27212}
\saveTG{𧴀}{27212}
\saveTG{𧇿}{27212}
\saveTG{𧇣}{27212}
\saveTG{觤}{27212}
\saveTG{䏣}{27212}
\saveTG{𧲼}{27212}
\saveTG{𩳩}{27212}
\saveTG{𢄄}{27212}
\saveTG{𤕲}{27212}
\saveTG{𧣞}{27212}
\saveTG{𠜞}{27212}
\saveTG{𩳱}{27212}
\saveTG{𢒿}{27212}
\saveTG{䶊}{27212}
\saveTG{𩱺}{27212}
\saveTG{䝚}{27212}
\saveTG{𨩙}{27212}
\saveTG{儳}{27213}
\saveTG{𩴃}{27213}
\saveTG{𧆼}{27213}
\saveTG{𢓞}{27213}
\saveTG{𧇰}{27213}
\saveTG{𧥒}{27213}
\saveTG{𦘮}{27214}
\saveTG{𤖍}{27214}
\saveTG{𩳇}{27214}
\saveTG{𧆰}{27214}
\saveTG{𠈬}{27214}
\saveTG{𪜫}{27214}
\saveTG{㑠}{27214}
\saveTG{𩴍}{27214}
\saveTG{𩴇}{27214}
\saveTG{𩳉}{27214}
\saveTG{𡌪}{27214}
\saveTG{𧤵}{27214}
\saveTG{𧲵}{27214}
\saveTG{𢔔}{27214}
\saveTG{𠊣}{27214}
\saveTG{偓}{27214}
\saveTG{㒛}{27215}
\saveTG{𧥋}{27215}
\saveTG{𨾉}{27215}
\saveTG{𨾞}{27215}
\saveTG{𢖈}{27215}
\saveTG{𨿯}{27215}
\saveTG{𩀒}{27215}
\saveTG{𠸜}{27216}
\saveTG{䰨}{27216}
\saveTG{𩲤}{27216}
\saveTG{𧇅}{27216}
\saveTG{𠃯}{27217}
\saveTG{䖘}{27217}
\saveTG{𧠇}{27217}
\saveTG{𧇝}{27217}
\saveTG{𡽏}{27217}
\saveTG{𡽑}{27217}
\saveTG{𧣶}{27217}
\saveTG{𧥓}{27217}
\saveTG{𣬚}{27217}
\saveTG{𠑾}{27217}
\saveTG{𪚪}{27217}
\saveTG{𡖴}{27217}
\saveTG{𤽅}{27217}
\saveTG{𪖡}{27217}
\saveTG{𪗂}{27217}
\saveTG{𡕚}{27217}
\saveTG{𢓚}{27217}
\saveTG{𢖞}{27217}
\saveTG{𧢦}{27217}
\saveTG{𠒎}{27217}
\saveTG{𩴷}{27217}
\saveTG{𠋞}{27217}
\saveTG{𠉜}{27217}
\saveTG{𦛑}{27217}
\saveTG{𦛾}{27217}
\saveTG{𢀸}{27217}
\saveTG{𠙝}{27217}
\saveTG{𠙀}{27217}
\saveTG{𧣃}{27217}
\saveTG{𧣍}{27217}
\saveTG{𧥎}{27217}
\saveTG{𦫱}{27217}
\saveTG{𪖱}{27217}
\saveTG{䶌}{27217}
\saveTG{𠈜}{27217}
\saveTG{䳃}{27217}
\saveTG{𪂧}{27217}
\saveTG{𪂭}{27217}
\saveTG{㒨}{27217}
\saveTG{𠑗}{27217}
\saveTG{𠑣}{27217}
\saveTG{𠙡}{27217}
\saveTG{伔}{27217}
\saveTG{凫}{27217}
\saveTG{鳧}{27217}
\saveTG{鳬}{27217}
\saveTG{僶}{27217}
\saveTG{𪕪}{27217}
\saveTG{俷}{27217}
\saveTG{𧲧}{27217}
\saveTG{𦘺}{27217}
\saveTG{𧇔}{27217}
\saveTG{𠙑}{27217}
\saveTG{𡖊}{27217}
\saveTG{𠙪}{27217}
\saveTG{𢃁}{27217}
\saveTG{𠇕}{27217}
\saveTG{𠆩}{27217}
\saveTG{㐶}{27217}
\saveTG{𠏙}{27217}
\saveTG{𠌋}{27217}
\saveTG{𠍠}{27217}
\saveTG{𠍃}{27217}
\saveTG{𩖿}{27217}
\saveTG{𩲞}{27217}
\saveTG{㒘}{27218}
\saveTG{𩳖}{27218}
\saveTG{𠐊}{27218}
\saveTG{𧇹}{27218}
\saveTG{𩲟}{27218}
\saveTG{䰯}{27218}
\saveTG{𩳗}{27219}
\saveTG{𩴙}{27219}
\saveTG{𩳫}{27219}
\saveTG{𩲪}{27219}
\saveTG{𨽝}{27220}
\saveTG{䠺}{27220}
\saveTG{𨽴}{27220}
\saveTG{𨽩}{27220}
\saveTG{𨽡}{27220}
\saveTG{𧲳}{27220}
\saveTG{𨽄}{27220}
\saveTG{𨼱}{27220}
\saveTG{𨺅}{27220}
\saveTG{𢔱}{27220}
\saveTG{𢔘}{27220}
\saveTG{𢓋}{27220}
\saveTG{𢓦}{27220}
\saveTG{𠣐}{27220}
\saveTG{𡖖}{27220}
\saveTG{𡖄}{27220}
\saveTG{𧢹}{27220}
\saveTG{𠚣}{27220}
\saveTG{𧳆}{27220}
\saveTG{𦙸}{27220}
\saveTG{𧲩}{27220}
\saveTG{𤕅}{27220}
\saveTG{卶}{27220}
\saveTG{伺}{27220}
\saveTG{倜}{27220}
\saveTG{侗}{27220}
\saveTG{匑}{27220}
\saveTG{佝}{27220}
\saveTG{夠}{27220}
\saveTG{豿}{27220}
\saveTG{齁}{27220}
\saveTG{翙}{27220}
\saveTG{翽}{27220}
\saveTG{躹}{27220}
\saveTG{𧳜}{27220}
\saveTG{翗}{27220}
\saveTG{甪}{27220}
\saveTG{夘}{27220}
\saveTG{彴}{27220}
\saveTG{们}{27220}
\saveTG{們}{27220}
\saveTG{勿}{27220}
\saveTG{倗}{27220}
\saveTG{匍}{27220}
\saveTG{仞}{27220}
\saveTG{仴}{27220}
\saveTG{伆}{27220}
\saveTG{御}{27220}
\saveTG{僩}{27220}
\saveTG{僴}{27220}
\saveTG{向}{27220}
\saveTG{徇}{27220}
\saveTG{伨}{27220}
\saveTG{侚}{27220}
\saveTG{匇}{27220}
\saveTG{佣}{27220}
\saveTG{徟}{27220}
\saveTG{仢}{27220}
\saveTG{豹}{27220}
\saveTG{仰}{27220}
\saveTG{彻}{27220}
\saveTG{𧖧}{27220}
\saveTG{𠊬}{27220}
\saveTG{㑡}{27220}
\saveTG{𠇩}{27220}
\saveTG{𠏣}{27220}
\saveTG{𠈟}{27220}
\saveTG{𨽵}{27220}
\saveTG{𠎞}{27221}
\saveTG{𠎫}{27221}
\saveTG{𪤸}{27221}
\saveTG{𪝪}{27221}
\saveTG{𠏦}{27221}
\saveTG{𪝱}{27221}
\saveTG{䎗}{27221}
\saveTG{𦑤}{27221}
\saveTG{𪜹}{27221}
\saveTG{𪝍}{27221}
\saveTG{𢓈}{27221}
\saveTG{𠤍}{27221}
\saveTG{㔨}{27221}
\saveTG{𥤢}{27221}
\saveTG{𪜪}{27221}
\saveTG{𠎒}{27221}
\saveTG{𠜰}{27221}
\saveTG{𠜧}{27221}
\saveTG{𧣏}{27221}
\saveTG{𧤽}{27221}
\saveTG{𡖒}{27221}
\saveTG{𤡥}{27221}
\saveTG{𠝎}{27221}
\saveTG{𠣒}{27221}
\saveTG{𣍝}{27221}
\saveTG{𦑙}{27221}
\saveTG{修}{27222}
\saveTG{俢}{27222}
\saveTG{㣊}{27222}
\saveTG{勿}{27222}
\saveTG{𠉨}{27222}
\saveTG{𧇂}{27222}
\saveTG{𠐋}{27222}
\saveTG{僇}{27222}
\saveTG{𠣠}{27222}
\saveTG{䚧}{27222}
\saveTG{𠤈}{27222}
\saveTG{𪖷}{27222}
\saveTG{伃}{27222}
\saveTG{𤖅}{27223}
\saveTG{𠑡}{27223}
\saveTG{𪖙}{27223}
\saveTG{𡖜}{27223}
\saveTG{𢕜}{27223}
\saveTG{𢓷}{27223}
\saveTG{𡖉}{27223}
\saveTG{㐴}{27223}
\saveTG{𠍖}{27223}
\saveTG{𡖇}{27223}
\saveTG{𡖑}{27223}
\saveTG{𪜩}{27223}
\saveTG{𢲾}{27224}
\saveTG{𡕓}{27224}
\saveTG{𪖔}{27224}
\saveTG{𠂥}{27224}
\saveTG{𠎓}{27224}
\saveTG{𠣾}{27224}
\saveTG{𠤇}{27224}
\saveTG{𠬦}{27224}
\saveTG{𠑨}{27224}
\saveTG{𢄗}{27226}
\saveTG{𠇶}{27226}
\saveTG{𧇟}{27226}
\saveTG{𧣷}{27226}
\saveTG{㚋}{27226}
\saveTG{𢕳}{27226}
\saveTG{㣚}{27226}
\saveTG{𠍒}{27226}
\saveTG{𣂝}{27226}
\saveTG{䶎}{27226}
\saveTG{㣘}{27226}
\saveTG{𫕑}{27226}
\saveTG{𠍻}{27227}
\saveTG{𪛐}{27227}
\saveTG{𠝷}{27227}
\saveTG{𩿉}{27227}
\saveTG{𣤃}{27227}
\saveTG{𢑞}{27227}
\saveTG{𦢇}{27227}
\saveTG{𦡼}{27227}
\saveTG{㣯}{27227}
\saveTG{𠋱}{27227}
\saveTG{侈}{27227}
\saveTG{鶨}{27227}
\saveTG{伄}{27227}
\saveTG{甮}{27227}
\saveTG{躬}{27227}
\saveTG{傦}{27227}
\saveTG{鴴}{27227}
\saveTG{鸻}{27227}
\saveTG{鄇}{27227}
\saveTG{鵤}{27227}
\saveTG{鷮}{27227}
\saveTG{侷}{27227}
\saveTG{僪}{27227}
\saveTG{剓}{27227}
\saveTG{鸬}{27227}
\saveTG{鸕}{27227}
\saveTG{幋}{27227}
\saveTG{仍}{27227}
\saveTG{肾}{27227}
\saveTG{鵢}{27227}
\saveTG{翛}{27227}
\saveTG{脩}{27227}
\saveTG{邜}{27227}
\saveTG{鄬}{27227}
\saveTG{酅}{27227}
\saveTG{嚮}{27227}
\saveTG{郷}{27227}
\saveTG{鄉}{27227}
\saveTG{鄊}{27227}
\saveTG{鄕}{27227}
\saveTG{胷}{27227}
\saveTG{鸺}{27227}
\saveTG{鵂}{27227}
\saveTG{偦}{27227}
\saveTG{倻}{27227}
\saveTG{觺}{27227}
\saveTG{俑}{27227}
\saveTG{鄅}{27227}
\saveTG{鸆}{27227}
\saveTG{𨜩}{27227}
\saveTG{𨝀}{27227}
\saveTG{𢓶}{27227}
\saveTG{𨝰}{27227}
\saveTG{䢾}{27227}
\saveTG{𢔴}{27227}
\saveTG{𨞳}{27227}
\saveTG{𨝔}{27227}
\saveTG{𨟃}{27227}
\saveTG{𦓐}{27227}
\saveTG{𪉍}{27227}
\saveTG{𢓇}{27227}
\saveTG{𢖁}{27227}
\saveTG{䣏}{27227}
\saveTG{𨝈}{27227}
\saveTG{𪝀}{27227}
\saveTG{𠊰}{27227}
\saveTG{𠈥}{27227}
\saveTG{𠈕}{27227}
\saveTG{𠋇}{27227}
\saveTG{𠐕}{27227}
\saveTG{𠊇}{27227}
\saveTG{𠆿}{27227}
\saveTG{㑚}{27227}
\saveTG{𠌇}{27227}
\saveTG{𨚛}{27227}
\saveTG{𨜕}{27227}
\saveTG{𨙵}{27227}
\saveTG{𪵕}{27227}
\saveTG{𪅩}{27227}
\saveTG{䳌}{27227}
\saveTG{𪁰}{27227}
\saveTG{𪅊}{27227}
\saveTG{𦝾}{27227}
\saveTG{䴄}{27227}
\saveTG{𪆺}{27227}
\saveTG{𪇸}{27227}
\saveTG{䖚}{27227}
\saveTG{𪂬}{27227}
\saveTG{𪆛}{27227}
\saveTG{𪇦}{27227}
\saveTG{𪂢}{27227}
\saveTG{䳪}{27227}
\saveTG{𩿭}{27227}
\saveTG{䳽}{27227}
\saveTG{𪁏}{27227}
\saveTG{𠄝}{27227}
\saveTG{𨝐}{27227}
\saveTG{𤔲}{27227}
\saveTG{𧳊}{27227}
\saveTG{𧳁}{27227}
\saveTG{𧳤}{27227}
\saveTG{𨟈}{27227}
\saveTG{𨟌}{27227}
\saveTG{𪤼}{27227}
\saveTG{䣜}{27227}
\saveTG{𨞙}{27227}
\saveTG{𨞘}{27227}
\saveTG{𨛵}{27227}
\saveTG{𨝹}{27227}
\saveTG{𨞹}{27227}
\saveTG{𨝘}{27227}
\saveTG{𨜻}{27227}
\saveTG{𨞣}{27227}
\saveTG{𨛒}{27227}
\saveTG{𨟬}{27227}
\saveTG{𨛖}{27227}
\saveTG{𣨷}{27227}
\saveTG{䣙}{27227}
\saveTG{𨟎}{27227}
\saveTG{𧢼}{27227}
\saveTG{䣀}{27227}
\saveTG{𨜫}{27227}
\saveTG{角}{27227}
\saveTG{𨛥}{27227}
\saveTG{𧤐}{27227}
\saveTG{𧣣}{27227}
\saveTG{𠣥}{27227}
\saveTG{𨝁}{27227}
\saveTG{𨜬}{27227}
\saveTG{𠕗}{27227}
\saveTG{𡖛}{27227}
\saveTG{䏑}{27227}
\saveTG{𧢲}{27227}
\saveTG{𨛅}{27227}
\saveTG{𧣡}{27227}
\saveTG{𠣨}{27227}
\saveTG{𡖥}{27227}
\saveTG{𠶰}{27227}
\saveTG{}{27227}
\saveTG{𪅬}{27227}
\saveTG{𪀗}{27227}
\saveTG{𤰅}{27227}
\saveTG{𦪤}{27227}
\saveTG{𢁙}{27227}
\saveTG{𧥁}{27227}
\saveTG{𢑭}{27227}
\saveTG{䏱}{27227}
\saveTG{䵼}{27227}
\saveTG{𧢺}{27227}
\saveTG{䚥}{27227}
\saveTG{𧣚}{27227}
\saveTG{𢑼}{27227}
\saveTG{𢂨}{27227}
\saveTG{𪝟}{27227}
\saveTG{𠑑}{27227}
\saveTG{𠐩}{27227}
\saveTG{𠑆}{27227}
\saveTG{𪈥}{27227}
\saveTG{𩱑}{27227}
\saveTG{𩿄}{27227}
\saveTG{𦙵}{27227}
\saveTG{㠾}{27227}
\saveTG{𪀈}{27227}
\saveTG{𪆻}{27227}
\saveTG{𧤬}{27227}
\saveTG{𩁼}{27227}
\saveTG{𫛔}{27227}
\saveTG{𫛖}{27227}
\saveTG{䴂}{27227}
\saveTG{𫛄}{27227}
\saveTG{𡖿}{27227}
\saveTG{𪀩}{27227}
\saveTG{𩾲}{27227}
\saveTG{𪈵}{27227}
\saveTG{𪇤}{27227}
\saveTG{𪁤}{27227}
\saveTG{𪁮}{27227}
\saveTG{𪃝}{27227}
\saveTG{䳯}{27227}
\saveTG{𪁛}{27227}
\saveTG{鵚}{27227}
\saveTG{𠆨}{27227}
\saveTG{𠉺}{27227}
\saveTG{㐷}{27227}
\saveTG{𪀢}{27227}
\saveTG{𩿧}{27227}
\saveTG{𪀆}{27227}
\saveTG{䳰}{27227}
\saveTG{𠈡}{27227}
\saveTG{𠇛}{27227}
\saveTG{𪃶}{27227}
\saveTG{𠋹}{27227}
\saveTG{𪂄}{27227}
\saveTG{𪁃}{27227}
\saveTG{𪀼}{27227}
\saveTG{𠌥}{27227}
\saveTG{𠌵}{27227}
\saveTG{䳡}{27227}
\saveTG{㣇}{27227}
\saveTG{㣈}{27227}
\saveTG{㔃}{27227}
\saveTG{𧳸}{27227}
\saveTG{𧳹}{27227}
\saveTG{𪢞}{27227}
\saveTG{𧤾}{27227}
\saveTG{𧤢}{27227}
\saveTG{𠜌}{27227}
\saveTG{𦙑}{27227}
\saveTG{𢔇}{27227}
\saveTG{𠋀}{27227}
\saveTG{𠏈}{27227}
\saveTG{㑳}{27227}
\saveTG{𠊐}{27227}
\saveTG{𩾖}{27227}
\saveTG{𦠂}{27227}
\saveTG{𠐽}{27227}
\saveTG{𢑷}{27227}
\saveTG{𢃝}{27227}
\saveTG{𢂺}{27227}
\saveTG{𢂭}{27227}
\saveTG{𦛫}{27227}
\saveTG{𢂟}{27227}
\saveTG{𦛬}{27227}
\saveTG{𡕙}{27227}
\saveTG{㐻}{27228}
\saveTG{𠈀}{27228}
\saveTG{𧱤}{27228}
\saveTG{𡗠}{27228}
\saveTG{㒩}{27231}
\saveTG{𪝔}{27231}
\saveTG{𧹾}{27231}
\saveTG{𢖛}{27231}
\saveTG{𢔳}{27231}
\saveTG{𧌜}{27231}
\saveTG{䖺}{27231}
\saveTG{虪}{27231}
\saveTG{𩾄}{27231}
\saveTG{𢄇}{27231}
\saveTG{儵}{27231}
\saveTG{𠋺}{27231}
\saveTG{𧰺}{27232}
\saveTG{𧱈}{27232}
\saveTG{𣼙}{27232}
\saveTG{漿}{27232}
\saveTG{偬}{27232}
\saveTG{𪺢}{27232}
\saveTG{𧣢}{27232}
\saveTG{𠏀}{27232}
\saveTG{𧰼}{27232}
\saveTG{彖}{27232}
\saveTG{很}{27232}
\saveTG{佷}{27232}
\saveTG{貇}{27232}
\saveTG{躵}{27232}
\saveTG{象}{27232}
\saveTG{像}{27232}
\saveTG{衆}{27232}
\saveTG{䝦}{27232}
\saveTG{𠉹}{27232}
\saveTG{𧳩}{27232}
\saveTG{㑰}{27232}
\saveTG{𠌄}{27232}
\saveTG{𢑡}{27232}
\saveTG{𠐃}{27232}
\saveTG{䝆}{27232}
\saveTG{𧰲}{27232}
\saveTG{佟}{27233}
\saveTG{侭}{27233}
\saveTG{𠋂}{27233}
\saveTG{𢓘}{27233}
\saveTG{𤬂}{27233}
\saveTG{𧈆}{27233}
\saveTG{𠑉}{27233}
\saveTG{𧲴}{27233}
\saveTG{㣠}{27233}
\saveTG{𠗆}{27233}
\saveTG{𠉘}{27234}
\saveTG{𢓿}{27234}
\saveTG{𢕝}{27235}
\saveTG{鯈}{27236}
\saveTG{𣩕}{27236}
\saveTG{𠋎}{27237}
\saveTG{𥠖}{27238}
\saveTG{𧥂}{27238}
\saveTG{𠐯}{27238}
\saveTG{伮}{27240}
\saveTG{仭}{27240}
\saveTG{侜}{27240}
\saveTG{仅}{27240}
\saveTG{俶}{27240}
\saveTG{貈}{27240}
\saveTG{𢕌}{27241}
\saveTG{伜}{27241}
\saveTG{偋}{27241}
\saveTG{𠌸}{27241}
\saveTG{𧇈}{27241}
\saveTG{㑗}{27241}
\saveTG{𩲇}{27241}
\saveTG{𢖓}{27242}
\saveTG{𧳳}{27242}
\saveTG{𠊷}{27242}
\saveTG{𢿳}{27242}
\saveTG{將}{27242}
\saveTG{𢔨}{27243}
\saveTG{𢒫}{27243}
\saveTG{𢔧}{27244}
\saveTG{𠈍}{27244}
\saveTG{𧴏}{27244}
\saveTG{𡖫}{27244}
\saveTG{𠎪}{27244}
\saveTG{𠎅}{27246}
\saveTG{𫏯}{27247}
\saveTG{𫜠}{27247}
\saveTG{𧈑}{27247}
\saveTG{𧈜}{27247}
\saveTG{㞨}{27247}
\saveTG{𠭝}{27247}
\saveTG{𧣇}{27247}
\saveTG{𣪈}{27247}
\saveTG{𤜭}{27247}
\saveTG{𣪧}{27247}
\saveTG{𣪽}{27247}
\saveTG{𢾨}{27247}
\saveTG{𢔀}{27247}
\saveTG{𢔾}{27247}
\saveTG{𠊩}{27247}
\saveTG{𪝒}{27247}
\saveTG{𪝛}{27247}
\saveTG{𠊫}{27247}
\saveTG{𠐮}{27247}
\saveTG{𡦁}{27247}
\saveTG{𣪴}{27247}
\saveTG{𣨏}{27247}
\saveTG{𪖧}{27247}
\saveTG{𢕛}{27247}
\saveTG{㑴}{27247}
\saveTG{𧣸}{27247}
\saveTG{𢔃}{27247}
\saveTG{𠌞}{27247}
\saveTG{𧳶}{27247}
\saveTG{𤕱}{27247}
\saveTG{𠭠}{27247}
\saveTG{𪖲}{27247}
\saveTG{𠈿}{27247}
\saveTG{𧲫}{27247}
\saveTG{僝}{27247}
\saveTG{伋}{27247}
\saveTG{觙}{27247}
\saveTG{假}{27247}
\saveTG{徦}{27247}
\saveTG{貑}{27247}
\saveTG{觼}{27247}
\saveTG{侵}{27247}
\saveTG{傁}{27247}
\saveTG{殷}{27247}
\saveTG{役}{27247}
\saveTG{伇}{27247}
\saveTG{仔}{27247}
\saveTG{彶}{27247}
\saveTG{䏜}{27247}
\saveTG{𦙹}{27247}
\saveTG{𫏬}{27247}
\saveTG{𠭯}{27247}
\saveTG{𧆤}{27247}
\saveTG{𠮏}{27247}
\saveTG{𠭗}{27247}
\saveTG{𠮉}{27247}
\saveTG{𧮸}{27247}
\saveTG{𣥹}{27247}
\saveTG{㕟}{27247}
\saveTG{𣦻}{27247}
\saveTG{𣦼}{27247}
\saveTG{𤕭}{27247}
\saveTG{𧢷}{27247}
\saveTG{𩽹}{27247}
\saveTG{𢃺}{27247}
\saveTG{𠈧}{27247}
\saveTG{䶋}{27247}
\saveTG{𠉧}{27247}
\saveTG{𪜳}{27247}
\saveTG{𠍭}{27247}
\saveTG{𠭣}{27247}
\saveTG{𣪶}{27247}
\saveTG{䝘}{27247}
\saveTG{䝟}{27248}
\saveTG{𧲾}{27250}
\saveTG{𢔦}{27251}
\saveTG{𠎿}{27252}
\saveTG{解}{27252}
\saveTG{𣦫}{27252}
\saveTG{𦡖}{27252}
\saveTG{𧴛}{27252}
\saveTG{𢖆}{27252}
\saveTG{𤖀}{27254}
\saveTG{佭}{27254}
\saveTG{㑄}{27254}
\saveTG{㟂}{27254}
\saveTG{𢓱}{27254}
\saveTG{𢓤}{27254}
\saveTG{𠉏}{27254}
\saveTG{𡖭}{27255}
\saveTG{𡖔}{27255}
\saveTG{𢓏}{27255}
\saveTG{𦚻}{27255}
\saveTG{𧳰}{27256}
\saveTG{𡺠}{27256}
\saveTG{㑮}{27256}
\saveTG{鞗}{27256}
\saveTG{𣎤}{27256}
\saveTG{𩎵}{27257}
\saveTG{𠌬}{27259}
\saveTG{徲}{27259}
\saveTG{𤖝}{27261}
\saveTG{𢕻}{27261}
\saveTG{儋}{27261}
\saveTG{詹}{27261}
\saveTG{㑾}{27261}
\saveTG{貂}{27262}
\saveTG{佲}{27262}
\saveTG{𧳽}{27262}
\saveTG{𪖠}{27262}
\saveTG{佋}{27262}
\saveTG{𩳊}{27262}
\saveTG{𠌃}{27262}
\saveTG{㸛}{27262}
\saveTG{䐲}{27262}
\saveTG{𢑦}{27262}
\saveTG{𢕍}{27262}
\saveTG{𤾌}{27262}
\saveTG{𧤝}{27262}
\saveTG{𢕠}{27263}
\saveTG{𠐔}{27263}
\saveTG{倨}{27264}
\saveTG{𦧕}{27264}
\saveTG{𧣻}{27264}
\saveTG{𢓜}{27264}
\saveTG{㫦}{27264}
\saveTG{偹}{27264}
\saveTG{㑼}{27264}
\saveTG{𤖕}{27264}
\saveTG{𫕓}{27264}
\saveTG{𠊽}{27264}
\saveTG{俻}{27264}
\saveTG{佫}{27264}
\saveTG{觡}{27264}
\saveTG{貉}{27264}
\saveTG{𠋥}{27267}
\saveTG{𧳬}{27267}
\saveTG{侰}{27267}
\saveTG{𢔰}{27267}
\saveTG{𧳈}{27270}
\saveTG{㑢}{27270}
\saveTG{䘓}{27271}
\saveTG{𦜆}{27272}
\saveTG{𢔈}{27272}
\saveTG{倔}{27272}
\saveTG{𠈆}{27277}
\saveTG{佀}{27277}
\saveTG{㑇}{27277}
\saveTG{}{27277}
\saveTG{𠈎}{27277}
\saveTG{㟰}{27280}
\saveTG{𠋶}{27280}
\saveTG{𧇻}{27280}
\saveTG{𤕶}{27281}
\saveTG{𠇉}{27281}
\saveTG{㒜}{27281}
\saveTG{𠊨}{27281}
\saveTG{儗}{27281}
\saveTG{僎}{27281}
\saveTG{俱}{27281}
\saveTG{𪖗}{27282}
\saveTG{𪗀}{27282}
\saveTG{𢕢}{27282}
\saveTG{𢓑}{27282}
\saveTG{𠈴}{27282}
\saveTG{𠏩}{27282}
\saveTG{𣣠}{27282}
\saveTG{𠍫}{27282}
\saveTG{㐸}{27282}
\saveTG{𠐀}{27282}
\saveTG{𠑁}{27282}
\saveTG{𣢉}{27282}
\saveTG{𣣢}{27282}
\saveTG{𠏭}{27282}
\saveTG{𢖂}{27282}
\saveTG{𦠑}{27282}
\saveTG{歂}{27282}
\saveTG{佽}{27282}
\saveTG{歑}{27282}
\saveTG{僛}{27282}
\saveTG{俽}{27282}
\saveTG{歔}{27282}
\saveTG{𪖓}{27282}
\saveTG{㱆}{27282}
\saveTG{𣤴}{27282}
\saveTG{𣣍}{27282}
\saveTG{𣣕}{27282}
\saveTG{𣤠}{27282}
\saveTG{𣢻}{27282}
\saveTG{欳}{27282}
\saveTG{𣤝}{27282}
\saveTG{𣢽}{27282}
\saveTG{𣢪}{27282}
\saveTG{𧥊}{27282}
\saveTG{𨥆}{27282}
\saveTG{𣢊}{27282}
\saveTG{𣣒}{27282}
\saveTG{𣣪}{27282}
\saveTG{㰾}{27282}
\saveTG{𣤙}{27282}
\saveTG{𢕫}{27282}
\saveTG{𢕤}{27282}
\saveTG{㰫}{27282}
\saveTG{侯}{27284}
\saveTG{𠉀}{27284}
\saveTG{𩴅}{27284}
\saveTG{𠎇}{27284}
\saveTG{𤟌}{27284}
\saveTG{䐅}{27284}
\saveTG{𡗁}{27284}
\saveTG{𠊱}{27284}
\saveTG{𡙝}{27284}
\saveTG{𠋫}{27284}
\saveTG{候}{27284}
\saveTG{矦}{27284}
\saveTG{倏}{27284}
\saveTG{偰}{27284}
\saveTG{𡘓}{27285}
\saveTG{𧴡}{27286}
\saveTG{偩}{27286}
\saveTG{𩔒}{27286}
\saveTG{𠍦}{27286}
\saveTG{𠏤}{27286}
\saveTG{𠍿}{27286}
\saveTG{伬}{27287}
\saveTG{𢖣}{27289}
\saveTG{𠊕}{27289}
\saveTG{𠑎}{27289}
\saveTG{𠈳}{27289}
\saveTG{𩵑}{27289}
\saveTG{倐}{27289}
\saveTG{𠊸}{27289}
\saveTG{㡜}{27291}
\saveTG{𠇡}{27291}
\saveTG{傺}{27291}
\saveTG{你}{27292}
\saveTG{𧤆}{27292}
\saveTG{𠎧}{27293}
\saveTG{絛}{27293}
\saveTG{𪜷}{27294}
\saveTG{𤖒}{27294}
\saveTG{𠍇}{27294}
\saveTG{躲}{27294}
\saveTG{躱}{27294}
\saveTG{條}{27294}
\saveTG{𠈃}{27294}
\saveTG{㑱}{27294}
\saveTG{𢔟}{27294}
\saveTG{𧳨}{27294}
\saveTG{𠌖}{27294}
\saveTG{𠋕}{27294}
\saveTG{儏}{27294}
\saveTG{觮}{27299}
\saveTG{𢅞}{27299}
\saveTG{𠉎}{27299}
\saveTG{𨒶}{27301}
\saveTG{𠚤}{27302}
\saveTG{𧻬}{27302}
\saveTG{冬}{27303}
\saveTG{𡗅}{27309}
\saveTG{𩹥}{27310}
\saveTG{鯴}{27310}
\saveTG{鳦}{27310}
\saveTG{𩻲}{27311}
\saveTG{𫙖}{27312}
\saveTG{𩶫}{27312}
\saveTG{𩽍}{27312}
\saveTG{䱉}{27312}
\saveTG{𦁧}{27312}
\saveTG{鮠}{27312}
\saveTG{鶃}{27312}
\saveTG{鯢}{27312}
\saveTG{鮑}{27312}
\saveTG{鯭}{27312}
\saveTG{鮸}{27312}
\saveTG{鰹}{27314}
\saveTG{𪬎}{27315}
\saveTG{𪃵}{27316}
\saveTG{𫙙}{27317}
\saveTG{𩷳}{27317}
\saveTG{𩻼}{27317}
\saveTG{𪔁}{27317}
\saveTG{𩵔}{27317}
\saveTG{𩶴}{27317}
\saveTG{𩾩}{27317}
\saveTG{鱦}{27317}
\saveTG{魢}{27317}
\saveTG{𩸦}{27317}
\saveTG{𩸃}{27317}
\saveTG{𩽝}{27317}
\saveTG{𩼍}{27317}
\saveTG{䳈}{27317}
\saveTG{䶱}{27317}
\saveTG{𩵨}{27317}
\saveTG{𩵗}{27317}
\saveTG{䰾}{27317}
\saveTG{鮣}{27320}
\saveTG{魩}{27320}
\saveTG{𫙑}{27320}
\saveTG{𤔺}{27320}
\saveTG{𡭒}{27320}
\saveTG{鯽}{27320}
\saveTG{鮦}{27320}
\saveTG{𪀃}{27320}
\saveTG{魛}{27320}
\saveTG{勺}{27320}
\saveTG{魡}{27320}
\saveTG{鯛}{27320}
\saveTG{鮙}{27320}
\saveTG{鮈}{27320}
\saveTG{鰗}{27320}
\saveTG{𩾱}{27320}
\saveTG{𩾓}{27320}
\saveTG{𩸀}{27321}
\saveTG{𩵺}{27321}
\saveTG{𩻹}{27321}
\saveTG{𪂙}{27321}
\saveTG{𩿖}{27321}
\saveTG{𩵜}{27321}
\saveTG{𩶒}{27321}
\saveTG{䱩}{27321}
\saveTG{𩶠}{27321}
\saveTG{𩸢}{27322}
\saveTG{𩵕}{27322}
\saveTG{𩻘}{27322}
\saveTG{𪅡}{27322}
\saveTG{魣}{27322}
\saveTG{𩵻}{27323}
\saveTG{𫚱}{27323}
\saveTG{𩾡}{27323}
\saveTG{𪀓}{27323}
\saveTG{𠣵}{27324}
\saveTG{𫙸}{27324}
\saveTG{𩻾}{27326}
\saveTG{𪀠}{27326}
\saveTG{𪀊}{27326}
\saveTG{𩷛}{27327}
\saveTG{𩶰}{27327}
\saveTG{𩻗}{27327}
\saveTG{𩺐}{27327}
\saveTG{𩷂}{27327}
\saveTG{䱶}{27327}
\saveTG{𤉱}{27327}
\saveTG{𨚟}{27327}
\saveTG{𪃀}{27327}
\saveTG{𨝱}{27327}
\saveTG{䳥}{27327}
\saveTG{䳿}{27327}
\saveTG{𩵌}{27327}
\saveTG{𫛋}{27327}
\saveTG{𪂒}{27327}
\saveTG{𪂺}{27327}
\saveTG{𪄀}{27327}
\saveTG{𪇑}{27327}
\saveTG{䳉}{27327}
\saveTG{𩾒}{27327}
\saveTG{𪄝}{27327}
\saveTG{𪅝}{27327}
\saveTG{𪈨}{27327}
\saveTG{𪈼}{27327}
\saveTG{𤊱}{27327}
\saveTG{𩽗}{27327}
\saveTG{䱻}{27327}
\saveTG{𩸒}{27327}
\saveTG{𩺥}{27327}
\saveTG{𫚹}{27327}
\saveTG{𪄥}{27327}
\saveTG{𪂋}{27327}
\saveTG{𪄞}{27327}
\saveTG{𩿋}{27327}
\saveTG{𫚇}{27327}
\saveTG{𤉢}{27327}
\saveTG{𩷐}{27327}
\saveTG{𦫯}{27327}
\saveTG{𩿴}{27327}
\saveTG{𩾫}{27327}
\saveTG{䱬}{27327}
\saveTG{鳥}{27327}
\saveTG{鴔}{27327}
\saveTG{鹪}{27327}
\saveTG{鵹}{27327}
\saveTG{鄥}{27327}
\saveTG{鰯}{27327}
\saveTG{烏}{27327}
\saveTG{鄔}{27327}
\saveTG{鰞}{27327}
\saveTG{鄎}{27327}
\saveTG{鱜}{27327}
\saveTG{鯒}{27327}
\saveTG{鱊}{27327}
\saveTG{鷠}{27327}
\saveTG{駌}{27327}
\saveTG{鴛}{27327}
\saveTG{鯞}{27327}
\saveTG{𩹢}{27327}
\saveTG{𫑨}{27327}
\saveTG{郻}{27327}
\saveTG{𫛡}{27327}
\saveTG{𩸾}{27327}
\saveTG{𩷾}{27327}
\saveTG{𩼁}{27327}
\saveTG{𩽂}{27327}
\saveTG{鷦}{27327}
\saveTG{𩹾}{27328}
\saveTG{𩽥}{27329}
\saveTG{䱡}{27329}
\saveTG{𤆒}{27330}
\saveTG{𤇗}{27331}
\saveTG{𤇘}{27331}
\saveTG{𤈖}{27331}
\saveTG{怨}{27331}
\saveTG{熈}{27331}
\saveTG{炰}{27331}
\saveTG{𩷰}{27331}
\saveTG{𩷃}{27331}
\saveTG{𩷱}{27331}
\saveTG{𪚰}{27331}
\saveTG{𤎹}{27331}
\saveTG{黧}{27331}
\saveTG{𪺋}{27331}
\saveTG{𤇐}{27331}
\saveTG{𨸋}{27331}
\saveTG{𪒀}{27331}
\saveTG{𤑪}{27331}
\saveTG{𢤰}{27331}
\saveTG{𢤕}{27331}
\saveTG{𩽵}{27331}
\saveTG{鱌}{27332}
\saveTG{忽}{27332}
\saveTG{怱}{27332}
\saveTG{𪇍}{27332}
\saveTG{𩽐}{27332}
\saveTG{𢚿}{27332}
\saveTG{㥎}{27332}
\saveTG{䱼}{27332}
\saveTG{䱲}{27332}
\saveTG{𤆫}{27332}
\saveTG{𪫫}{27332}
\saveTG{𢢍}{27332}
\saveTG{𩼦}{27332}
\saveTG{𢟅}{27332}
\saveTG{𢜷}{27332}
\saveTG{𢘵}{27332}
\saveTG{𡖋}{27332}
\saveTG{𢝰}{27332}
\saveTG{𤌉}{27332}
\saveTG{𩼣}{27332}
\saveTG{𤌝}{27332}
\saveTG{惣}{27332}
\saveTG{懇}{27333}
\saveTG{𩺒}{27333}
\saveTG{鮗}{27333}
\saveTG{鴤}{27333}
\saveTG{𢟁}{27334}
\saveTG{惄}{27334}
\saveTG{慇}{27334}
\saveTG{𢡈}{27334}
\saveTG{𢢣}{27335}
\saveTG{𤑊}{27335}
\saveTG{㶻}{27335}
\saveTG{𢜓}{27335}
\saveTG{䲆}{27336}
\saveTG{惫}{27336}
\saveTG{鯬}{27336}
\saveTG{鱻}{27336}
\saveTG{魚}{27336}
\saveTG{䲜}{27336}
\saveTG{鰠}{27336}
\saveTG{𩺰}{27336}
\saveTG{𩺓}{27336}
\saveTG{𩺈}{27336}
\saveTG{㤩}{27336}
\saveTG{𤌈}{27336}
\saveTG{𤑩}{27336}
\saveTG{𫙦}{27336}
\saveTG{䱗}{27336}
\saveTG{𩼠}{27336}
\saveTG{急}{27337}
\saveTG{𩺬}{27337}
\saveTG{𢙠}{27337}
\saveTG{𤈕}{27337}
\saveTG{𢝩}{27338}
\saveTG{𢛦}{27338}
\saveTG{𢞣}{27338}
\saveTG{𤆐}{27338}
\saveTG{𢣕}{27338}
\saveTG{燞}{27338}
\saveTG{𢘞}{27339}
\saveTG{𢤂}{27339}
\saveTG{您}{27339}
\saveTG{鯓}{27340}
\saveTG{鯫}{27340}
\saveTG{𪧶}{27342}
\saveTG{鱂}{27342}
\saveTG{𩺎}{27343}
\saveTG{𩶣}{27344}
\saveTG{鱘}{27346}
\saveTG{𩵎}{27347}
\saveTG{䱙}{27347}
\saveTG{𩷅}{27347}
\saveTG{𩾶}{27347}
\saveTG{𩹆}{27347}
\saveTG{䱆}{27347}
\saveTG{𩵤}{27347}
\saveTG{𩻣}{27347}
\saveTG{𩻈}{27347}
\saveTG{𩷍}{27347}
\saveTG{𩺪}{27347}
\saveTG{𩸤}{27347}
\saveTG{𫙞}{27347}
\saveTG{𩸯}{27347}
\saveTG{𫙿}{27347}
\saveTG{𫙨}{27347}
\saveTG{𩹨}{27347}
\saveTG{魥}{27347}
\saveTG{鮼}{27347}
\saveTG{鰕}{27347}
\saveTG{𫙗}{27347}
\saveTG{䲒}{27352}
\saveTG{鯶}{27352}
\saveTG{𩶋}{27354}
\saveTG{䱢}{27357}
\saveTG{𩷄}{27357}
\saveTG{𩺜}{27361}
\saveTG{鰡}{27362}
\saveTG{𩺠}{27362}
\saveTG{𩻎}{27362}
\saveTG{鮉}{27362}
\saveTG{鰼}{27362}
\saveTG{𩽈}{27363}
\saveTG{𩹕}{27364}
\saveTG{鴼}{27364}
\saveTG{䱟}{27364}
\saveTG{鮥}{27364}
\saveTG{𠤆}{27364}
\saveTG{𪆬}{27364}
\saveTG{𩷭}{27364}
\saveTG{𩹿}{27364}
\saveTG{鮶}{27367}
\saveTG{𩹪}{27367}
\saveTG{𫙩}{27368}
\saveTG{𩶧}{27370}
\saveTG{䱤}{27377}
\saveTG{鱮}{27381}
\saveTG{𩻝}{27381}
\saveTG{𣤿}{27382}
\saveTG{歍}{27382}
\saveTG{䲌}{27382}
\saveTG{𩵢}{27382}
\saveTG{𩺶}{27382}
\saveTG{𢥷}{27382}
\saveTG{𣤚}{27382}
\saveTG{𩼨}{27382}
\saveTG{𪇐}{27384}
\saveTG{𪂷}{27384}
\saveTG{鯸}{27384}
\saveTG{䳧}{27384}
\saveTG{𩺟}{27384}
\saveTG{𩼈}{27384}
\saveTG{𫙪}{27384}
\saveTG{𫚁}{27386}
\saveTG{䲚}{27386}
\saveTG{𪆔}{27387}
\saveTG{𩼾}{27389}
\saveTG{𩽯}{27389}
\saveTG{𩽬}{27389}
\saveTG{鰶}{27391}
\saveTG{𩶗}{27391}
\saveTG{𩸹}{27392}
\saveTG{𩼇}{27394}
\saveTG{𩺞}{27394}
\saveTG{鰷}{27394}
\saveTG{鰇}{27394}
\saveTG{𩻳}{27394}
\saveTG{𩶹}{27394}
\saveTG{鵦}{27399}
\saveTG{𩼽}{27399}
\saveTG{𩸙}{27399}
\saveTG{䱚}{27399}
\saveTG{夂}{27400}
\saveTG{身}{27400}
\saveTG{𠣌}{27401}
\saveTG{𡖏}{27401}
\saveTG{𦖥}{27401}
\saveTG{処}{27401}
\saveTG{𠦨}{27402}
\saveTG{𣁓}{27402}
\saveTG{𨓨}{27402}
\saveTG{妴}{27404}
\saveTG{𡟮}{27404}
\saveTG{媻}{27404}
\saveTG{𡠍}{27404}
\saveTG{𡛹}{27404}
\saveTG{𡕱}{27407}
\saveTG{𡕳}{27407}
\saveTG{𡕫}{27407}
\saveTG{阜}{27407}
\saveTG{夐}{27407}
\saveTG{𡕴}{27407}
\saveTG{𢿌}{27407}
\saveTG{𠭟}{27407}
\saveTG{𢻁}{27407}
\saveTG{𡕷}{27407}
\saveTG{敻}{27407}
\saveTG{𡕕}{27407}
\saveTG{𦥺}{27407}
\saveTG{𢑓}{27408}
\saveTG{𠥾}{27408}
\saveTG{䴻}{27408}
\saveTG{舤}{27410}
\saveTG{䑺}{27410}
\saveTG{𩗔}{27410}
\saveTG{𩗯}{27410}
\saveTG{免}{27412}
\saveTG{舰}{27412}
\saveTG{舧}{27412}
\saveTG{㕙}{27412}
\saveTG{𠬍}{27412}
\saveTG{艋}{27412}
\saveTG{毚}{27413}
\saveTG{艬}{27413}
\saveTG{𩙾}{27413}
\saveTG{兔}{27413}
\saveTG{𦩙}{27414}
\saveTG{艉}{27414}
\saveTG{𠄇}{27416}
\saveTG{𦨨}{27417}
\saveTG{𦨲}{27417}
\saveTG{𦨌}{27417}
\saveTG{𦫝}{27417}
\saveTG{𨉏}{27417}
\saveTG{𦩊}{27417}
\saveTG{𦫉}{27417}
\saveTG{𦨇}{27417}
\saveTG{𠃷}{27417}
\saveTG{𦫦}{27417}
\saveTG{舥}{27417}
\saveTG{𨉆}{27417}
\saveTG{𦨹}{27417}
\saveTG{𧡬}{27417}
\saveTG{𡤺}{27417}
\saveTG{𠓗}{27417}
\saveTG{𦑂}{27420}
\saveTG{翱}{27420}
\saveTG{翺}{27420}
\saveTG{匆}{27420}
\saveTG{舠}{27420}
\saveTG{翶}{27420}
\saveTG{匉}{27420}
\saveTG{勼}{27420}
\saveTG{匔}{27420}
\saveTG{𦑳}{27421}
\saveTG{𢻿}{27421}
\saveTG{𠝵}{27421}
\saveTG{𦨔}{27421}
\saveTG{𠚲}{27421}
\saveTG{𠣍}{27421}
\saveTG{𦒢}{27421}
\saveTG{𠤄}{27421}
\saveTG{𠤀}{27422}
\saveTG{𠤊}{27422}
\saveTG{𦨓}{27423}
\saveTG{𠣳}{27423}
\saveTG{𥠨}{27423}
\saveTG{𠣘}{27424}
\saveTG{𠣽}{27424}
\saveTG{𦨴}{27426}
\saveTG{𦩍}{27426}
\saveTG{䑦}{27426}
\saveTG{𨈳}{27426}
\saveTG{𠡹}{27427}
\saveTG{𠣴}{27427}
\saveTG{𨚕}{27427}
\saveTG{𨛐}{27427}
\saveTG{𨛶}{27427}
\saveTG{𪤾}{27427}
\saveTG{𨟜}{27427}
\saveTG{𨟣}{27427}
\saveTG{𫛱}{27427}
\saveTG{𨞭}{27427}
\saveTG{𫚶}{27427}
\saveTG{䣗}{27427}
\saveTG{䢴}{27427}
\saveTG{𦨰}{27427}
\saveTG{𦨋}{27427}
\saveTG{𨝲}{27427}
\saveTG{鵕}{27427}
\saveTG{鄒}{27427}
\saveTG{鷱}{27427}
\saveTG{鷎}{27427}
\saveTG{郍}{27427}
\saveTG{𠡴}{27427}
\saveTG{郛}{27427}
\saveTG{鸼}{27427}
\saveTG{鵃}{27427}
\saveTG{芻}{27427}
\saveTG{酁}{27427}
\saveTG{鴘}{27427}
\saveTG{鵯}{27427}
\saveTG{鹎}{27427}
\saveTG{鴇}{27427}
\saveTG{鸨}{27427}
\saveTG{𨉱}{27427}
\saveTG{𦩸}{27427}
\saveTG{𠣸}{27427}
\saveTG{𪇣}{27427}
\saveTG{𪈬}{27427}
\saveTG{𨉯}{27427}
\saveTG{𨻑}{27427}
\saveTG{𪅥}{27427}
\saveTG{㔢}{27427}
\saveTG{䑵}{27427}
\saveTG{䑼}{27427}
\saveTG{𨉦}{27427}
\saveTG{𠡵}{27427}
\saveTG{𨺔}{27427}
\saveTG{𩾜}{27427}
\saveTG{𠡡}{27427}
\saveTG{䑠}{27427}
\saveTG{𪃊}{27427}
\saveTG{鶵}{27427}
\saveTG{𪇈}{27427}
\saveTG{䳕}{27427}
\saveTG{𪉊}{27427}
\saveTG{𨛲}{27427}
\saveTG{𨈔}{27427}
\saveTG{𨉰}{27427}
\saveTG{𨈕}{27427}
\saveTG{𠢲}{27427}
\saveTG{鵫}{27427}
\saveTG{鶢}{27427}
\saveTG{务}{27427}
\saveTG{鵎}{27427}
\saveTG{鵵}{27427}
\saveTG{鶽}{27427}
\saveTG{郫}{27427}
\saveTG{䠾}{27428}
\saveTG{𦫈}{27431}
\saveTG{𦪶}{27431}
\saveTG{䎇}{27431}
\saveTG{𨊐}{27432}
\saveTG{𦪃}{27432}
\saveTG{𦪏}{27432}
\saveTG{𫙭}{27433}
\saveTG{𦪎}{27435}
\saveTG{舟}{27440}
\saveTG{𠔾}{27440}
\saveTG{匁}{27440}
\saveTG{𢍪}{27441}
\saveTG{𦨿}{27441}
\saveTG{彜}{27442}
\saveTG{𢌳}{27442}
\saveTG{𫙐}{27443}
\saveTG{𢍝}{27443}
\saveTG{䒂}{27443}
\saveTG{𢇉}{27443}
\saveTG{㛑}{27444}
\saveTG{𠂨}{27444}
\saveTG{㢡}{27444}
\saveTG{𨊖}{27444}
\saveTG{𨈝}{27444}
\saveTG{𨽪}{27447}
\saveTG{般}{27447}
\saveTG{艘}{27447}
\saveTG{䑡}{27447}
\saveTG{𨈞}{27447}
\saveTG{𣪞}{27447}
\saveTG{𣪙}{27447}
\saveTG{𪥸}{27447}
\saveTG{𡦆}{27447}
\saveTG{䑥}{27447}
\saveTG{𠬞}{27447}
\saveTG{𪥯}{27447}
\saveTG{𨈯}{27447}
\saveTG{𨉨}{27447}
\saveTG{𦨕}{27447}
\saveTG{𫇝}{27447}
\saveTG{𨉣}{27447}
\saveTG{𢍢}{27448}
\saveTG{𢍅}{27448}
\saveTG{𡣂}{27448}
\saveTG{彝}{27449}
\saveTG{𠦿}{27449}
\saveTG{𨉙}{27452}
\saveTG{艂}{27454}
\saveTG{舽}{27454}
\saveTG{𨈶}{27454}
\saveTG{𨉄}{27455}
\saveTG{𦨸}{27455}
\saveTG{𨈜}{27455}
\saveTG{𠹾}{27460}
\saveTG{𨊗}{27461}
\saveTG{𧭔}{27461}
\saveTG{𨊍}{27461}
\saveTG{船}{27461}
\saveTG{䒁}{27462}
\saveTG{𦨣}{27462}
\saveTG{艪}{27463}
\saveTG{嗠}{27464}
\saveTG{艍}{27464}
\saveTG{𦩯}{27465}
\saveTG{䒄}{27466}
\saveTG{𨉭}{27467}
\saveTG{𦨛}{27473}
\saveTG{𨈹}{27476}
\saveTG{𨉜}{27476}
\saveTG{𨉫}{27476}
\saveTG{𦩛}{27481}
\saveTG{𣤱}{27482}
\saveTG{㱀}{27482}
\saveTG{𪴭}{27482}
\saveTG{𦪬}{27482}
\saveTG{𣢳}{27482}
\saveTG{𨉻}{27482}
\saveTG{𪦂}{27484}
\saveTG{𦩣}{27484}
\saveTG{𦪫}{27494}
\saveTG{𦩁}{27494}
\saveTG{𩾍}{27494}
\saveTG{𦩺}{27494}
\saveTG{䑮}{27494}
\saveTG{𢵠}{27502}
\saveTG{𢩸}{27502}
\saveTG{𢴭}{27502}
\saveTG{𢮃}{27502}
\saveTG{犂}{27502}
\saveTG{搫}{27502}
\saveTG{𧢻}{27502}
\saveTG{夅}{27504}
\saveTG{𡕖}{27504}
\saveTG{𡕘}{27504}
\saveTG{𤚜}{27504}
\saveTG{夆}{27504}
\saveTG{𣬔}{27506}
\saveTG{鞶}{27506}
\saveTG{𨍂}{27506}
\saveTG{𣬒}{27506}
\saveTG{𣬐}{27506}
\saveTG{𨊬}{27506}
\saveTG{𡖌}{27506}
\saveTG{争}{27507}
\saveTG{𪺳}{27508}
\saveTG{𢭥}{27510}
\saveTG{𪮩}{27512}
\saveTG{𤙊}{27512}
\saveTG{魍}{27512}
\saveTG{𤚄}{27512}
\saveTG{𤛹}{27515}
\saveTG{𤘕}{27517}
\saveTG{𤘷}{27517}
\saveTG{𢭶}{27517}
\saveTG{㸭}{27517}
\saveTG{𢔫}{27517}
\saveTG{𤙙}{27517}
\saveTG{𤘗}{27517}
\saveTG{𤜇}{27517}
\saveTG{𤙌}{27517}
\saveTG{犅}{27520}
\saveTG{𠬏}{27520}
\saveTG{𥠈}{27520}
\saveTG{物}{27520}
\saveTG{𫔸}{27520}
\saveTG{𫅱}{27520}
\saveTG{牣}{27520}
\saveTG{𦙚}{27521}
\saveTG{𠣮}{27521}
\saveTG{𦒌}{27521}
\saveTG{𢪥}{27522}
\saveTG{𫅫}{27522}
\saveTG{㹋}{27522}
\saveTG{𤛞}{27524}
\saveTG{𤘞}{27524}
\saveTG{𠣞}{27526}
\saveTG{𤙓}{27526}
\saveTG{𤘽}{27526}
\saveTG{鴾}{27527}
\saveTG{鵝}{27527}
\saveTG{鹅}{27527}
\saveTG{犓}{27527}
\saveTG{𤙗}{27527}
\saveTG{𪀋}{27527}
\saveTG{𤛛}{27527}
\saveTG{𤛃}{27527}
\saveTG{𨛻}{27527}
\saveTG{𨚦}{27527}
\saveTG{𨛟}{27527}
\saveTG{𨛰}{27527}
\saveTG{𨞲}{27527}
\saveTG{觕}{27527}
\saveTG{𤚱}{27527}
\saveTG{𤛴}{27527}
\saveTG{𤛒}{27527}
\saveTG{𤘻}{27528}
\saveTG{㸾}{27532}
\saveTG{𪭳}{27532}
\saveTG{𤙹}{27532}
\saveTG{㹅}{27532}
\saveTG{𡕗}{27540}
\saveTG{𤛧}{27543}
\saveTG{𤘪}{27544}
\saveTG{𢪳}{27547}
\saveTG{犌}{27547}
\saveTG{牳}{27550}
\saveTG{𤛰}{27551}
\saveTG{𤛳}{27552}
\saveTG{䳗}{27553}
\saveTG{㸼}{27554}
\saveTG{𠭮}{27554}
\saveTG{𠭘}{27554}
\saveTG{𠭢}{27554}
\saveTG{𤙉}{27555}
\saveTG{𤘟}{27555}
\saveTG{𡖚}{27556}
\saveTG{㹆}{27556}
\saveTG{𩿼}{27562}
\saveTG{𤛊}{27562}
\saveTG{𤙳}{27564}
\saveTG{𤙑}{27564}
\saveTG{𤚤}{27567}
\saveTG{𤚭}{27572}
\saveTG{𤚂}{27579}
\saveTG{犋}{27581}
\saveTG{𢴞}{27581}
\saveTG{𢵬}{27581}
\saveTG{𤘯}{27582}
\saveTG{㰱}{27582}
\saveTG{𢳉}{27592}
\saveTG{𤛼}{27599}
\saveTG{𤛺}{27599}
\saveTG{𠮥}{27600}
\saveTG{𣬕}{27601}
\saveTG{𧖮}{27601}
\saveTG{𥖡}{27601}
\saveTG{𠤪}{27601}
\saveTG{𣆌}{27601}
\saveTG{㽜}{27601}
\saveTG{詧}{27601}
\saveTG{鲁}{27601}
\saveTG{響}{27601}
\saveTG{𧭐}{27601}
\saveTG{𧥧}{27601}
\saveTG{䚻}{27601}
\saveTG{𧥲}{27601}
\saveTG{𧧁}{27601}
\saveTG{𥅉}{27601}
\saveTG{眢}{27601}
\saveTG{𦐲}{27601}
\saveTG{𧬰}{27601}
\saveTG{名}{27602}
\saveTG{睝}{27602}
\saveTG{曶}{27602}
\saveTG{𠯳}{27602}
\saveTG{𤰧}{27602}
\saveTG{㫚}{27602}
\saveTG{䀜}{27602}
\saveTG{𥅕}{27602}
\saveTG{磐}{27602}
\saveTG{𥓍}{27602}
\saveTG{𥖞}{27602}
\saveTG{㗽}{27602}
\saveTG{𠯓}{27602}
\saveTG{魯}{27603}
\saveTG{𩶑}{27603}
\saveTG{醬}{27604}
\saveTG{各}{27604}
\saveTG{𥍁}{27604}
\saveTG{𨡓}{27604}
\saveTG{𣈉}{27604}
\saveTG{𥈼}{27604}
\saveTG{𥆛}{27604}
\saveTG{备}{27604}
\saveTG{督}{27604}
\saveTG{𨈖}{27604}
\saveTG{𩘅}{27610}
\saveTG{𫇔}{27610}
\saveTG{𦐛}{27611}
\saveTG{馜}{27612}
\saveTG{觇}{27612}
\saveTG{𪉛}{27612}
\saveTG{壡}{27614}
\saveTG{𦤊}{27614}
\saveTG{𤫀}{27614}
\saveTG{䴞}{27615}
\saveTG{𤾫}{27615}
\saveTG{皅}{27617}
\saveTG{𢁏}{27617}
\saveTG{𦦃}{27617}
\saveTG{𪕘}{27617}
\saveTG{𪝁}{27617}
\saveTG{𦧻}{27617}
\saveTG{𤾆}{27617}
\saveTG{䵶}{27617}
\saveTG{𦫙}{27617}
\saveTG{𪽿}{27617}
\saveTG{𤽶}{27617}
\saveTG{𩐢}{27618}
\saveTG{翻}{27620}
\saveTG{匐}{27620}
\saveTG{𥐻}{27620}
\saveTG{匌}{27620}
\saveTG{句}{27620}
\saveTG{訇}{27620}
\saveTG{皗}{27620}
\saveTG{匒}{27620}
\saveTG{的}{27620}
\saveTG{甸}{27620}
\saveTG{旬}{27620}
\saveTG{䎋}{27621}
\saveTG{䎅}{27621}
\saveTG{𠣱}{27621}
\saveTG{㔪}{27621}
\saveTG{𠣭}{27621}
\saveTG{䑚}{27621}
\saveTG{𪟂}{27621}
\saveTG{䀏}{27621}
\saveTG{𪱜}{27621}
\saveTG{𠵞}{27622}
\saveTG{𠣰}{27624}
\saveTG{𠣯}{27624}
\saveTG{𨡑}{27626}
\saveTG{𠣪}{27626}
\saveTG{𠣚}{27626}
\saveTG{𨚷}{27627}
\saveTG{𨛹}{27627}
\saveTG{𨛕}{27627}
\saveTG{𤽀}{27627}
\saveTG{鄱}{27627}
\saveTG{𪀄}{27627}
\saveTG{𨜡}{27627}
\saveTG{𪉆}{27627}
\saveTG{䳆}{27627}
\saveTG{𨟝}{27627}
\saveTG{𩡏}{27627}
\saveTG{𨚮}{27627}
\saveTG{𨟇}{27627}
\saveTG{𡖡}{27627}
\saveTG{𨝣}{27627}
\saveTG{邰}{27627}
\saveTG{𨜋}{27627}
\saveTG{𪆍}{27627}
\saveTG{𪊀}{27627}
\saveTG{𤾦}{27627}
\saveTG{鷭}{27627}
\saveTG{郜}{27627}
\saveTG{鵅}{27627}
\saveTG{够}{27627}
\saveTG{鴝}{27627}
\saveTG{鸲}{27627}
\saveTG{鹄}{27627}
\saveTG{鵠}{27627}
\saveTG{鸹}{27627}
\saveTG{鴰}{27627}
\saveTG{郇}{27627}
\saveTG{䣎}{27627}
\saveTG{邭}{27627}
\saveTG{郋}{27627}
\saveTG{鴲}{27627}
\saveTG{鶅}{27627}
\saveTG{鶛}{27627}
\saveTG{𣊸}{27627}
\saveTG{𪈂}{27627}
\saveTG{𪀽}{27627}
\saveTG{𫛘}{27627}
\saveTG{𦧽}{27631}
\saveTG{𦧫}{27632}
\saveTG{𦤜}{27633}
\saveTG{𣅞}{27640}
\saveTG{叡}{27640}
\saveTG{𣅈}{27641}
\saveTG{𪿃}{27643}
\saveTG{𤳫}{27644}
\saveTG{𣪦}{27647}
\saveTG{𪾄}{27647}
\saveTG{㲊}{27647}
\saveTG{𪉘}{27647}
\saveTG{𧧙}{27647}
\saveTG{㕡}{27647}
\saveTG{𧧞}{27647}
\saveTG{𩡇}{27654}
\saveTG{𤾈}{27656}
\saveTG{舚}{27661}
\saveTG{𠑕}{27661}
\saveTG{𤽥}{27664}
\saveTG{𪉦}{27677}
\saveTG{𣢤}{27682}
\saveTG{𫇘}{27682}
\saveTG{欨}{27682}
\saveTG{𤾰}{27682}
\saveTG{𣤣}{27682}
\saveTG{𣤍}{27682}
\saveTG{㰬}{27682}
\saveTG{𫘃}{27682}
\saveTG{𣣊}{27682}
\saveTG{𣤧}{27682}
\saveTG{𣤓}{27682}
\saveTG{𤴙}{27682}
\saveTG{㰧}{27682}
\saveTG{㰶}{27682}
\saveTG{𣢷}{27682}
\saveTG{𤽺}{27699}
\saveTG{饣}{27700}
\saveTG{飖}{27710}
\saveTG{颻}{27710}
\saveTG{凯}{27710}
\saveTG{𡶺}{27710}
\saveTG{饥}{27710}
\saveTG{𡴭}{27710}
\saveTG{𡺤}{27710}
\saveTG{𠃉}{27710}
\saveTG{𣰞}{27711}
\saveTG{𪗱}{27711}
\saveTG{𣭙}{27711}
\saveTG{𣭚}{27711}
\saveTG{𣬠}{27711}
\saveTG{𪵖}{27711}
\saveTG{𣬶}{27711}
\saveTG{𣬤}{27712}
\saveTG{㲌}{27712}
\saveTG{𣰇}{27712}
\saveTG{𣯑}{27712}
\saveTG{包}{27712}
\saveTG{饱}{27712}
\saveTG{龅}{27712}
\saveTG{齙}{27712}
\saveTG{觊}{27712}
\saveTG{龃}{27712}
\saveTG{岨}{27712}
\saveTG{岘}{27712}
\saveTG{峗}{27712}
\saveTG{齯}{27712}
\saveTG{齟}{27712}
\saveTG{𪩁}{27712}
\saveTG{𠚱}{27712}
\saveTG{𠤓}{27712}
\saveTG{𣭜}{27712}
\saveTG{彘}{27712}
\saveTG{毥}{27712}
\saveTG{𡺡}{27712}
\saveTG{𣮴}{27712}
\saveTG{𪵙}{27712}
\saveTG{𡗇}{27712}
\saveTG{𣭧}{27712}
\saveTG{𢇀}{27712}
\saveTG{𣬞}{27712}
\saveTG{毱}{27712}
\saveTG{𣬝}{27712}
\saveTG{𡶄}{27712}
\saveTG{㲧}{27713}
\saveTG{巉}{27713}
\saveTG{𪚫}{27714}
\saveTG{𣯲}{27714}
\saveTG{齷}{27714}
\saveTG{龌}{27714}
\saveTG{𣬬}{27714}
\saveTG{𪚾}{27714}
\saveTG{𪚿}{27714}
\saveTG{𪘴}{27714}
\saveTG{𡹥}{27714}
\saveTG{𡴓}{27714}
\saveTG{𣭇}{27715}
\saveTG{𪛉}{27715}
\saveTG{𡽢}{27715}
\saveTG{𪵟}{27716}
\saveTG{氌}{27716}
\saveTG{龟}{27716}
\saveTG{亀}{27716}
\saveTG{𣯥}{27716}
\saveTG{𣭨}{27716}
\saveTG{氇}{27716}
\saveTG{𤭜}{27717}
\saveTG{𨛫}{27717}
\saveTG{𪓝}{27717}
\saveTG{𪓑}{27717}
\saveTG{𪓻}{27717}
\saveTG{𪓟}{27717}
\saveTG{㼝}{27717}
\saveTG{𢀹}{27717}
\saveTG{𡶋}{27717}
\saveTG{色}{27717}
\saveTG{㞦}{27717}
\saveTG{𪚶}{27717}
\saveTG{䶕}{27717}
\saveTG{𪓠}{27717}
\saveTG{𪗼}{27717}
\saveTG{𡴱}{27717}
\saveTG{𡵟}{27717}
\saveTG{𡽡}{27717}
\saveTG{𡷭}{27717}
\saveTG{㞾}{27717}
\saveTG{𡵆}{27717}
\saveTG{𪚃}{27717}
\saveTG{𠙋}{27717}
\saveTG{𩠀}{27717}
\saveTG{𪗗}{27717}
\saveTG{㽈}{27717}
\saveTG{𪁨}{27717}
\saveTG{𣬋}{27717}
\saveTG{𣬌}{27717}
\saveTG{𪚭}{27717}
\saveTG{㐑}{27717}
\saveTG{𪘣}{27717}
\saveTG{𡷕}{27717}
\saveTG{𡶮}{27717}
\saveTG{㲋}{27717}
\saveTG{𡸣}{27717}
\saveTG{屺}{27717}
\saveTG{𡴾}{27717}
\saveTG{𪩒}{27718}
\saveTG{𣬨}{27718}
\saveTG{𣬴}{27718}
\saveTG{𣭬}{27719}
\saveTG{𣮓}{27719}
\saveTG{屻}{27720}
\saveTG{饲}{27720}
\saveTG{匋}{27720}
\saveTG{岉}{27720}
\saveTG{饷}{27720}
\saveTG{匈}{27720}
\saveTG{峋}{27720}
\saveTG{岄}{27720}
\saveTG{𣍧}{27720}
\saveTG{𫗫}{27720}
\saveTG{𡴹}{27720}
\saveTG{㟹}{27720}
\saveTG{𡵙}{27720}
\saveTG{𠨍}{27720}
\saveTG{𠨞}{27720}
\saveTG{𪨫}{27720}
\saveTG{𪩆}{27720}
\saveTG{𠨐}{27720}
\saveTG{匂}{27720}
\saveTG{齣}{27720}
\saveTG{峒}{27720}
\saveTG{匎}{27720}
\saveTG{匃}{27720}
\saveTG{匄}{27720}
\saveTG{勾}{27720}
\saveTG{岣}{27720}
\saveTG{幻}{27720}
\saveTG{卽}{27720}
\saveTG{匓}{27720}
\saveTG{𡴮}{27721}
\saveTG{𠣔}{27721}
\saveTG{㓜}{27721}
\saveTG{𡼼}{27721}
\saveTG{𡷄}{27721}
\saveTG{㠈}{27721}
\saveTG{𠣹}{27721}
\saveTG{𦒩}{27721}
\saveTG{𪯊}{27721}
\saveTG{𪙩}{27721}
\saveTG{𪘶}{27721}
\saveTG{𪙨}{27721}
\saveTG{𣍢}{27721}
\saveTG{𢆵}{27721}
\saveTG{𠣏}{27721}
\saveTG{𠤉}{27721}
\saveTG{𠚨}{27721}
\saveTG{𡵿}{27721}
\saveTG{嵺}{27722}
\saveTG{𠣖}{27723}
\saveTG{𪪊}{27723}
\saveTG{𠕫}{27723}
\saveTG{𠣗}{27723}
\saveTG{𠣎}{27723}
\saveTG{𡵺}{27723}
\saveTG{𡷃}{27723}
\saveTG{𠣓}{27723}
\saveTG{𠣿}{27724}
\saveTG{𡸀}{27726}
\saveTG{𪗪}{27726}
\saveTG{𪘍}{27726}
\saveTG{𡼏}{27726}
\saveTG{𣯀}{27726}
\saveTG{㟃}{27726}
\saveTG{㟘}{27726}
\saveTG{𧖨}{27727}
\saveTG{𩿂}{27727}
\saveTG{𩾮}{27727}
\saveTG{𡷊}{27727}
\saveTG{𫚰}{27727}
\saveTG{𪀀}{27727}
\saveTG{𪄢}{27727}
\saveTG{𠣬}{27727}
\saveTG{𡻅}{27727}
\saveTG{𪈖}{27727}
\saveTG{𡻋}{27727}
\saveTG{𪂓}{27727}
\saveTG{㟠}{27727}
\saveTG{𪀨}{27727}
\saveTG{𪃨}{27727}
\saveTG{𢆳}{27727}
\saveTG{𩿌}{27727}
\saveTG{鹐}{27727}
\saveTG{𪄜}{27727}
\saveTG{𪇹}{27727}
\saveTG{𪄉}{27727}
\saveTG{𢕊}{27727}
\saveTG{𪄆}{27727}
\saveTG{𢔐}{27727}
\saveTG{𢏽}{27727}
\saveTG{𩠋}{27727}
\saveTG{𪙑}{27727}
\saveTG{𪚌}{27727}
\saveTG{鴢}{27727}
\saveTG{岛}{27727}
\saveTG{島}{27727}
\saveTG{嶋}{27727}
\saveTG{鴭}{27727}
\saveTG{馉}{27727}
\saveTG{屷}{27727}
\saveTG{鵮}{27727}
\saveTG{𠣲}{27727}
\saveTG{崅}{27727}
\saveTG{嵶}{27727}
\saveTG{邖}{27727}
\saveTG{饧}{27727}
\saveTG{嵨}{27727}
\saveTG{峫}{27727}
\saveTG{鹞}{27727}
\saveTG{鷂}{27727}
\saveTG{齺}{27727}
\saveTG{𩾤}{27727}
\saveTG{𪁅}{27727}
\saveTG{䶤}{27727}
\saveTG{𪅎}{27727}
\saveTG{𫀫}{27727}
\saveTG{𩿩}{27727}
\saveTG{𩨳}{27727}
\saveTG{𨚍}{27727}
\saveTG{㔡}{27727}
\saveTG{𪉔}{27727}
\saveTG{𨝺}{27727}
\saveTG{𨜚}{27727}
\saveTG{𨛀}{27727}
\saveTG{𨚔}{27727}
\saveTG{𫑲}{27727}
\saveTG{𨛉}{27727}
\saveTG{𡻳}{27727}
\saveTG{𡷙}{27727}
\saveTG{𡹬}{27727}
\saveTG{𫑙}{27727}
\saveTG{𪗝}{27727}
\saveTG{𪙃}{27727}
\saveTG{䢺}{27727}
\saveTG{𪙆}{27727}
\saveTG{𪙀}{27727}
\saveTG{𢏒}{27727}
\saveTG{𢏔}{27727}
\saveTG{𡺱}{27727}
\saveTG{𠣢}{27727}
\saveTG{𢏃}{27727}
\saveTG{𡴰}{27727}
\saveTG{𡹲}{27727}
\saveTG{𡹙}{27727}
\saveTG{𠣤}{27728}
\saveTG{𡷞}{27731}
\saveTG{𩠇}{27731}
\saveTG{𫌏}{27732}
\saveTG{𧙘}{27732}
\saveTG{䙚}{27732}
\saveTG{𪙢}{27732}
\saveTG{𡻑}{27732}
\saveTG{𡺪}{27732}
\saveTG{𧛲}{27732}
\saveTG{𢇐}{27732}
\saveTG{饗}{27732}
\saveTG{𪞹}{27732}
\saveTG{嶑}{27732}
\saveTG{裊}{27732}
\saveTG{袅}{27732}
\saveTG{龈}{27732}
\saveTG{齦}{27732}
\saveTG{裻}{27732}
\saveTG{餐}{27732}
\saveTG{褩}{27732}
\saveTG{𩝶}{27732}
\saveTG{𩚏}{27732}
\saveTG{𩚴}{27732}
\saveTG{𩝫}{27732}
\saveTG{𡷐}{27732}
\saveTG{𩜨}{27732}
\saveTG{𧚩}{27732}
\saveTG{峂}{27733}
\saveTG{馋}{27733}
\saveTG{𡺎}{27733}
\saveTG{収}{27740}
\saveTG{齱}{27740}
\saveTG{𪙝}{27743}
\saveTG{𪗤}{27743}
\saveTG{𡼢}{27743}
\saveTG{𪨪}{27744}
\saveTG{𪩎}{27744}
\saveTG{𣬥}{27747}
\saveTG{𣪕}{27747}
\saveTG{𪩖}{27747}
\saveTG{㲃}{27747}
\saveTG{𪙦}{27747}
\saveTG{𠬳}{27747}
\saveTG{𡹧}{27747}
\saveTG{𪗵}{27747}
\saveTG{馊}{27747}
\saveTG{𡺁}{27747}
\saveTG{岷}{27747}
\saveTG{岋}{27747}
\saveTG{𡻁}{27747}
\saveTG{𣪘}{27747}
\saveTG{𪩀}{27749}
\saveTG{𡼧}{27752}
\saveTG{嶰}{27752}
\saveTG{𨱕}{27752}
\saveTG{齳}{27752}
\saveTG{峄}{27754}
\saveTG{峰}{27754}
\saveTG{𫗥}{27757}
\saveTG{峥}{27757}
\saveTG{𡶶}{27757}
\saveTG{嶦}{27761}
\saveTG{𪗸}{27762}
\saveTG{馏}{27762}
\saveTG{𨻿}{27762}
\saveTG{𦥵}{27762}
\saveTG{嵧}{27762}
\saveTG{𠸿}{27762}
\saveTG{𦦌}{27762}
\saveTG{齠}{27762}
\saveTG{龆}{27762}
\saveTG{岹}{27762}
\saveTG{嶍}{27762}
\saveTG{𪘊}{27764}
\saveTG{㟭}{27764}
\saveTG{崌}{27764}
\saveTG{饹}{27764}
\saveTG{峈}{27764}
\saveTG{𡾓}{27766}
\saveTG{𡄌}{27766}
\saveTG{嵋}{27767}
\saveTG{峮}{27767}
\saveTG{齨}{27770}
\saveTG{𪙗}{27771}
\saveTG{𪚆}{27771}
\saveTG{𡼥}{27771}
\saveTG{𡹔}{27771}
\saveTG{嶴}{27772}
\saveTG{崛}{27772}
\saveTG{峊}{27772}
\saveTG{崡}{27772}
\saveTG{𡴲}{27772}
\saveTG{𡵶}{27772}
\saveTG{𡵁}{27772}
\saveTG{𡿖}{27772}
\saveTG{𪩛}{27772}
\saveTG{𡸆}{27772}
\saveTG{𡵪}{27772}
\saveTG{𪤵}{27772}
\saveTG{𡺯}{27772}
\saveTG{𪘳}{27772}
\saveTG{𡽈}{27772}
\saveTG{㠀}{27772}
\saveTG{𦥜}{27777}
\saveTG{𦥢}{27777}
\saveTG{𡸬}{27777}
\saveTG{𡼻}{27777}
\saveTG{䶟}{27777}
\saveTG{𨸏}{27777}
\saveTG{𠂤}{27777}
\saveTG{臽}{27777}
\saveTG{馅}{27777}
\saveTG{𡸞}{27777}
\saveTG{嶼}{27781}
\saveTG{馔}{27781}
\saveTG{欿}{27782}
\saveTG{歃}{27782}
\saveTG{饮}{27782}
\saveTG{㠜}{27782}
\saveTG{𣣯}{27782}
\saveTG{𩰠}{27782}
\saveTG{𣢜}{27782}
\saveTG{𣢄}{27782}
\saveTG{𣢋}{27782}
\saveTG{𣢼}{27782}
\saveTG{𣣮}{27782}
\saveTG{欪}{27782}
\saveTG{𣣳}{27782}
\saveTG{𣣸}{27782}
\saveTG{𡼲}{27782}
\saveTG{㰞}{27782}
\saveTG{𡾴}{27782}
\saveTG{𡺌}{27784}
\saveTG{㠗}{27784}
\saveTG{𡺝}{27784}
\saveTG{𡹵}{27784}
\saveTG{𫗯}{27784}
\saveTG{𡵸}{27787}
\saveTG{𡽨}{27792}
\saveTG{𡶲}{27792}
\saveTG{𪘉}{27794}
\saveTG{𪙙}{27794}
\saveTG{𡸇}{27794}
\saveTG{𡷔}{27794}
\saveTG{𡸮}{27799}
\saveTG{㠟}{27799}
\saveTG{𩛼}{27799}
\saveTG{久}{27800}
\saveTG{𨄚}{27801}
\saveTG{𡖸}{27801}
\saveTG{𡖷}{27801}
\saveTG{𨙞}{27801}
\saveTG{𨑔}{27801}
\saveTG{𨗝}{27802}
\saveTG{负}{27802}
\saveTG{欠}{27802}
\saveTG{贤}{27802}
\saveTG{𣣝}{27802}
\saveTG{𣢢}{27802}
\saveTG{𨃾}{27802}
\saveTG{𧽩}{27802}
\saveTG{𤴠}{27802}
\saveTG{𨒼}{27803}
\saveTG{𨙧}{27803}
\saveTG{𡗛}{27804}
\saveTG{𪥌}{27804}
\saveTG{𢑣}{27804}
\saveTG{𡙟}{27804}
\saveTG{㚟}{27804}
\saveTG{𡘬}{27804}
\saveTG{奂}{27804}
\saveTG{奧}{27804}
\saveTG{奥}{27804}
\saveTG{𩵋}{27804}
\saveTG{𤟭}{27804}
\saveTG{𤟒}{27804}
\saveTG{𤟏}{27804}
\saveTG{𢑱}{27804}
\saveTG{𡘠}{27804}
\saveTG{奬}{27804}
\saveTG{𡘅}{27804}
\saveTG{獎}{27804}
\saveTG{夨}{27804}
\saveTG{奐}{27804}
\saveTG{𡙖}{27805}
\saveTG{𣬎}{27805}
\saveTG{𧵍}{27806}
\saveTG{䝨}{27806}
\saveTG{夤}{27806}
\saveTG{負}{27806}
\saveTG{𧵚}{27806}
\saveTG{䝳}{27806}
\saveTG{𧷴}{27806}
\saveTG{𨖤}{27808}
\saveTG{𨓶}{27808}
\saveTG{𨗬}{27809}
\saveTG{㷩}{27809}
\saveTG{𤌽}{27809}
\saveTG{烉}{27809}
\saveTG{炙}{27809}
\saveTG{灸}{27809}
\saveTG{𤍵}{27809}
\saveTG{𤒐}{27809}
\saveTG{𤉯}{27809}
\saveTG{𤓫}{27809}
\saveTG{𤐏}{27809}
\saveTG{𤋳}{27809}
\saveTG{𣣛}{27809}
\saveTG{𡕛}{27809}
\saveTG{𠔙}{27812}
\saveTG{𡑧}{27814}
\saveTG{𧷎}{27814}
\saveTG{𪠀}{27814}
\saveTG{𦫭}{27817}
\saveTG{𪓷}{27817}
\saveTG{𪛁}{27817}
\saveTG{𦫬}{27817}
\saveTG{灳}{27820}
\saveTG{勽}{27820}
\saveTG{䪥}{27822}
\saveTG{𠠋}{27824}
\saveTG{𨃞}{27826}
\saveTG{𨝤}{27827}
\saveTG{𫑞}{27827}
\saveTG{𨜓}{27827}
\saveTG{𨝳}{27827}
\saveTG{𨞆}{27827}
\saveTG{𨝉}{27827}
\saveTG{𨜭}{27827}
\saveTG{𪃰}{27827}
\saveTG{𪃤}{27827}
\saveTG{𪆈}{27827}
\saveTG{𪃓}{27827}
\saveTG{𪅞}{27827}
\saveTG{䴈}{27827}
\saveTG{𪄾}{27827}
\saveTG{𪆅}{27827}
\saveTG{𪁬}{27827}
\saveTG{𤕂}{27827}
\saveTG{𡚖}{27827}
\saveTG{𦠷}{27827}
\saveTG{鴁}{27827}
\saveTG{鴩}{27827}
\saveTG{鶏}{27827}
\saveTG{鷄}{27827}
\saveTG{酇}{27827}
\saveTG{酂}{27827}
\saveTG{鄓}{27827}
\saveTG{鷆}{27827}
\saveTG{}{27827}
\saveTG{𨞓}{27827}
\saveTG{𠣦}{27829}
\saveTG{𣪄}{27847}
\saveTG{㕢}{27847}
\saveTG{𠭸}{27847}
\saveTG{𦩉}{27847}
\saveTG{𦒣}{27862}
\saveTG{𥌄}{27864}
\saveTG{𧸆}{27867}
\saveTG{𥏘}{27872}
\saveTG{𤎡}{27877}
\saveTG{疑}{27881}
\saveTG{㰿}{27882}
\saveTG{𣣇}{27882}
\saveTG{欵}{27882}
\saveTG{欸}{27882}
\saveTG{𣤡}{27882}
\saveTG{𣢐}{27882}
\saveTG{𣢸}{27882}
\saveTG{𣣓}{27882}
\saveTG{𦠹}{27889}
\saveTG{𤋅}{27891}
\saveTG{㷗}{27891}
\saveTG{𤎇}{27892}
\saveTG{𤊘}{27892}
\saveTG{𦤱}{27893}
\saveTG{𥻊}{27894}
\saveTG{禦}{27901}
\saveTG{祭}{27901}
\saveTG{洜}{27902}
\saveTG{漿}{27902}
\saveTG{彔}{27902}
\saveTG{尔}{27902}
\saveTG{紧}{27903}
\saveTG{縏}{27903}
\saveTG{𡕝}{27904}
\saveTG{䊍}{27904}
\saveTG{𫂲}{27904}
\saveTG{粲}{27904}
\saveTG{𣘟}{27904}
\saveTG{棃}{27904}
\saveTG{槃}{27904}
\saveTG{条}{27904}
\saveTG{枭}{27904}
\saveTG{梟}{27904}
\saveTG{粂}{27904}
\saveTG{夈}{27904}
\saveTG{𣐗}{27904}
\saveTG{𣔵}{27904}
\saveTG{㭧}{27904}
\saveTG{𣕼}{27904}
\saveTG{䲷}{27904}
\saveTG{𣝟}{27904}
\saveTG{𣐦}{27904}
\saveTG{𧗅}{27904}
\saveTG{𣝆}{27904}
\saveTG{𥺤}{27904}
\saveTG{䊢}{27904}
\saveTG{梷}{27904}
\saveTG{槳}{27904}
\saveTG{彙}{27904}
\saveTG{𢑥}{27905}
\saveTG{彙}{27905}
\saveTG{𪂝}{27905}
\saveTG{𥟖}{27908}
\saveTG{𪐅}{27909}
\saveTG{黎}{27909}
\saveTG{𥣥}{27909}
\saveTG{𦂱}{27910}
\saveTG{𣔗}{27910}
\saveTG{䆇}{27910}
\saveTG{𤽰}{27910}
\saveTG{𥝡}{27910}
\saveTG{𥝎}{27910}
\saveTG{𫄍}{27910}
\saveTG{𥡑}{27911}
\saveTG{𦄪}{27911}
\saveTG{𥞓}{27911}
\saveTG{租}{27912}
\saveTG{縕}{27912}
\saveTG{䋼}{27912}
\saveTG{𪆐}{27912}
\saveTG{𦪹}{27912}
\saveTG{𦁁}{27912}
\saveTG{𫃩}{27912}
\saveTG{𫀨}{27912}
\saveTG{紐}{27912}
\saveTG{𪓷}{27912}
\saveTG{秜}{27912}
\saveTG{臲}{27912}
\saveTG{絻}{27912}
\saveTG{組}{27912}
\saveTG{纔}{27913}
\saveTG{𥟷}{27914}
\saveTG{𥝦}{27914}
\saveTG{䵛}{27914}
\saveTG{𪏲}{27914}
\saveTG{䌑}{27914}
\saveTG{経}{27914}
\saveTG{𥿲}{27914}
\saveTG{𥟽}{27914}
\saveTG{𥜺}{27915}
\saveTG{𨤠}{27915}
\saveTG{𥣞}{27915}
\saveTG{䌦}{27915}
\saveTG{糶}{27915}
\saveTG{穐}{27916}
\saveTG{𥾊}{27917}
\saveTG{𫃠}{27917}
\saveTG{𪏶}{27917}
\saveTG{𡗃}{27917}
\saveTG{𦅥}{27917}
\saveTG{𦀿}{27917}
\saveTG{𥡌}{27917}
\saveTG{𥡄}{27917}
\saveTG{穐}{27917}
\saveTG{䅏}{27917}
\saveTG{𫃧}{27917}
\saveTG{𥢓}{27917}
\saveTG{𥠢}{27917}
\saveTG{䋩}{27917}
\saveTG{𦃫}{27917}
\saveTG{𦇿}{27917}
\saveTG{𦆵}{27917}
\saveTG{𥿡}{27917}
\saveTG{𦂴}{27917}
\saveTG{𪏸}{27917}
\saveTG{䆋}{27917}
\saveTG{𥡾}{27917}
\saveTG{䅋}{27917}
\saveTG{𫌨}{27917}
\saveTG{𫃝}{27917}
\saveTG{𥣅}{27917}
\saveTG{𥝗}{27917}
\saveTG{䄫}{27917}
\saveTG{𥝧}{27917}
\saveTG{𥛁}{27917}
\saveTG{𥿒}{27917}
\saveTG{𠙎}{27917}
\saveTG{𥾚}{27917}
\saveTG{䄐}{27917}
\saveTG{𦄓}{27917}
\saveTG{紦}{27917}
\saveTG{紀}{27917}
\saveTG{絕}{27917}
\saveTG{絶}{27917}
\saveTG{繩}{27917}
\saveTG{龝}{27917}
\saveTG{𥿎}{27917}
\saveTG{𥣎}{27918}
\saveTG{𥡪}{27918}
\saveTG{綢}{27920}
\saveTG{約}{27920}
\saveTG{絧}{27920}
\saveTG{繝}{27920}
\saveTG{絅}{27920}
\saveTG{綗}{27920}
\saveTG{匊}{27920}
\saveTG{稝}{27920}
\saveTG{綯}{27920}
\saveTG{絇}{27920}
\saveTG{紉}{27920}
\saveTG{網}{27920}
\saveTG{糿}{27920}
\saveTG{絢}{27920}
\saveTG{秱}{27920}
\saveTG{𥿑}{27920}
\saveTG{𦃁}{27920}
\saveTG{綳}{27920}
\saveTG{稠}{27920}
\saveTG{綱}{27920}
\saveTG{䋚}{27921}
\saveTG{𦁺}{27921}
\saveTG{𥝢}{27921}
\saveTG{䵑}{27921}
\saveTG{𠟿}{27921}
\saveTG{𥾛}{27921}
\saveTG{𦁅}{27921}
\saveTG{𣎋}{27921}
\saveTG{䄴}{27921}
\saveTG{𪐉}{27921}
\saveTG{𥾶}{27921}
\saveTG{𦁠}{27921}
\saveTG{𥾡}{27921}
\saveTG{𦁮}{27921}
\saveTG{𦄺}{27921}
\saveTG{𥞭}{27921}
\saveTG{𥾉}{27921}
\saveTG{𦐨}{27921}
\saveTG{𥝱}{27921}
\saveTG{𦆢}{27922}
\saveTG{𥡓}{27922}
\saveTG{𥝤}{27922}
\saveTG{𠞰}{27922}
\saveTG{䋒}{27922}
\saveTG{𥾞}{27922}
\saveTG{𥾸}{27922}
\saveTG{𪏵}{27922}
\saveTG{繆}{27922}
\saveTG{𪏯}{27922}
\saveTG{𫃲}{27922}
\saveTG{穋}{27922}
\saveTG{紓}{27922}
\saveTG{㱁}{27922}
\saveTG{𥡚}{27923}
\saveTG{𪐀}{27923}
\saveTG{䄪}{27923}
\saveTG{𫃳}{27923}
\saveTG{𦆗}{27923}
\saveTG{𥞖}{27924}
\saveTG{𦅤}{27924}
\saveTG{䋞}{27924}
\saveTG{䋄}{27924}
\saveTG{𦇾}{27926}
\saveTG{𥢀}{27926}
\saveTG{𥿧}{27926}
\saveTG{𦅘}{27926}
\saveTG{𥿆}{27926}
\saveTG{𦄞}{27926}
\saveTG{緺}{27927}
\saveTG{𪃚}{27927}
\saveTG{𪀖}{27927}
\saveTG{𪆶}{27927}
\saveTG{𩧸}{27927}
\saveTG{縐}{27927}
\saveTG{𨝄}{27927}
\saveTG{鄛}{27927}
\saveTG{移}{27927}
\saveTG{縎}{27927}
\saveTG{鵴}{27927}
\saveTG{繘}{27927}
\saveTG{鄡}{27927}
\saveTG{鷍}{27927}
\saveTG{稰}{27927}
\saveTG{縃}{27927}
\saveTG{邾}{27927}
\saveTG{鴸}{27927}
\saveTG{𪅁}{27927}
\saveTG{𩿥}{27927}
\saveTG{𫚴}{27927}
\saveTG{䄦}{27927}
\saveTG{𦁯}{27927}
\saveTG{𥿷}{27927}
\saveTG{𥞼}{27927}
\saveTG{𥿸}{27927}
\saveTG{𦃬}{27927}
\saveTG{𥣟}{27927}
\saveTG{䌵}{27927}
\saveTG{𦀯}{27927}
\saveTG{𦅉}{27927}
\saveTG{䌪}{27927}
\saveTG{𥾯}{27927}
\saveTG{𦂷}{27927}
\saveTG{𥾏}{27927}
\saveTG{𪇱}{27927}
\saveTG{𨞶}{27927}
\saveTG{𪆓}{27927}
\saveTG{𨜿}{27927}
\saveTG{𨝡}{27927}
\saveTG{𪹌}{27927}
\saveTG{𫛧}{27927}
\saveTG{𨞒}{27927}
\saveTG{䄧}{27927}
\saveTG{𥠁}{27927}
\saveTG{䣋}{27927}
\saveTG{𨟀}{27927}
\saveTG{𦃭}{27927}
\saveTG{𥞮}{27927}
\saveTG{𨞃}{27927}
\saveTG{𨝋}{27927}
\saveTG{𦂫}{27927}
\saveTG{𥿫}{27927}
\saveTG{𦀨}{27927}
\saveTG{𦄄}{27927}
\saveTG{𦃛}{27927}
\saveTG{𦂧}{27927}
\saveTG{䳳}{27927}
\saveTG{綁}{27927}
\saveTG{𪀶}{27927}
\saveTG{𦄋}{27927}
\saveTG{𪈮}{27927}
\saveTG{𥠳}{27927}
\saveTG{𪈜}{27927}
\saveTG{䅳}{27927}
\saveTG{𪆋}{27927}
\saveTG{𥾇}{27927}
\saveTG{𪅕}{27927}
\saveTG{𩾑}{27927}
\saveTG{𥡼}{27927}
\saveTG{𪇺}{27927}
\saveTG{𪆜}{27927}
\saveTG{𪅨}{27927}
\saveTG{𪃩}{27927}
\saveTG{𥠫}{27927}
\saveTG{𥟺}{27927}
\saveTG{𠣧}{27927}
\saveTG{𪅌}{27927}
\saveTG{𥡕}{27928}
\saveTG{䆈}{27930}
\saveTG{𪒺}{27931}
\saveTG{𦄦}{27931}
\saveTG{𥤚}{27931}
\saveTG{𦃵}{27931}
\saveTG{䋟}{27931}
\saveTG{𥤛}{27931}
\saveTG{𦃐}{27931}
\saveTG{𥢘}{27931}
\saveTG{𦄷}{27932}
\saveTG{綛}{27932}
\saveTG{緣}{27932}
\saveTG{𥣙}{27932}
\saveTG{緫}{27932}
\saveTG{縁}{27932}
\saveTG{𦃕}{27932}
\saveTG{𥞾}{27932}
\saveTG{𦁕}{27932}
\saveTG{𪐁}{27932}
\saveTG{𫃱}{27932}
\saveTG{𫀿}{27932}
\saveTG{𥠡}{27932}
\saveTG{𦆚}{27932}
\saveTG{終}{27933}
\saveTG{𦄁}{27933}
\saveTG{𦂯}{27933}
\saveTG{𥣦}{27934}
\saveTG{𦆣}{27934}
\saveTG{𥣔}{27935}
\saveTG{縫}{27935}
\saveTG{𧑎}{27936}
\saveTG{𥢦}{27936}
\saveTG{縋}{27937}
\saveTG{稳}{27937}
\saveTG{𥡈}{27937}
\saveTG{𥢩}{27938}
\saveTG{𦇗}{27938}
\saveTG{緅}{27940}
\saveTG{𠭥}{27940}
\saveTG{叔}{27940}
\saveTG{𣗸}{27941}
\saveTG{稺}{27941}
\saveTG{紣}{27941}
\saveTG{𦃘}{27941}
\saveTG{䌂}{27941}
\saveTG{𣓞}{27942}
\saveTG{𥢛}{27943}
\saveTG{紁}{27943}
\saveTG{𦅀}{27943}
\saveTG{𦅛}{27943}
\saveTG{𥝳}{27944}
\saveTG{𥿦}{27944}
\saveTG{𦂤}{27944}
\saveTG{𫄄}{27944}
\saveTG{𦃏}{27947}
\saveTG{𦇍}{27947}
\saveTG{𣪑}{27947}
\saveTG{𥿱}{27947}
\saveTG{𥾌}{27947}
\saveTG{𥡽}{27947}
\saveTG{𣪖}{27947}
\saveTG{𣫋}{27947}
\saveTG{䋋}{27947}
\saveTG{䅓}{27947}
\saveTG{𥝻}{27947}
\saveTG{𥢨}{27947}
\saveTG{𥟧}{27947}
\saveTG{𥝥}{27947}
\saveTG{𥝔}{27947}
\saveTG{𦀩}{27947}
\saveTG{𦃌}{27947}
\saveTG{𥝐}{27947}
\saveTG{𫃦}{27947}
\saveTG{𫃬}{27947}
\saveTG{秄}{27947}
\saveTG{縀}{27947}
\saveTG{綅}{27947}
\saveTG{級}{27947}
\saveTG{緞}{27947}
\saveTG{綴}{27947}
\saveTG{𦃈}{27947}
\saveTG{𥞔}{27947}
\saveTG{𦀁}{27947}
\saveTG{𥟒}{27947}
\saveTG{𥡖}{27949}
\saveTG{𥠧}{27951}
\saveTG{𤛿}{27952}
\saveTG{𤚶}{27952}
\saveTG{緷}{27952}
\saveTG{繲}{27952}
\saveTG{释}{27954}
\saveTG{綘}{27954}
\saveTG{𥞜}{27954}
\saveTG{絳}{27954}
\saveTG{𥾢}{27955}
\saveTG{𥣡}{27957}
\saveTG{𫀭}{27957}
\saveTG{䋫}{27957}
\saveTG{𦇙}{27958}
\saveTG{穉}{27959}
\saveTG{𦃥}{27961}
\saveTG{𣡳}{27961}
\saveTG{穞}{27961}
\saveTG{𦂑}{27961}
\saveTG{𦅼}{27961}
\saveTG{𦇊}{27962}
\saveTG{𥠷}{27962}
\saveTG{䌌}{27962}
\saveTG{𥿨}{27962}
\saveTG{紹}{27962}
\saveTG{𦃓}{27962}
\saveTG{𦀆}{27962}
\saveTG{穭}{27963}
\saveTG{𫃶}{27964}
\saveTG{䋧}{27964}
\saveTG{𫄉}{27964}
\saveTG{䅂}{27964}
\saveTG{𦃴}{27964}
\saveTG{緡}{27964}
\saveTG{䵕}{27964}
\saveTG{䅕}{27964}
\saveTG{絡}{27964}
\saveTG{𣞴}{27964}
\saveTG{𥉆}{27964}
\saveTG{𡮠}{27967}
\saveTG{𫃷}{27967}
\saveTG{𦀲}{27967}
\saveTG{𦁴}{27970}
\saveTG{𦁐}{27972}
\saveTG{𥿓}{27977}
\saveTG{𥞐}{27977}
\saveTG{䋓}{27977}
\saveTG{𫀬}{27977}
\saveTG{𦁵}{27979}
\saveTG{𦃼}{27980}
\saveTG{𣤰}{27981}
\saveTG{𥟭}{27981}
\saveTG{𥤌}{27981}
\saveTG{穥}{27981}
\saveTG{繏}{27981}
\saveTG{絘}{27982}
\saveTG{𥣖}{27982}
\saveTG{𣣅}{27982}
\saveTG{䊻}{27982}
\saveTG{𣣜}{27982}
\saveTG{𣢰}{27982}
\saveTG{𦆃}{27982}
\saveTG{䅆}{27982}
\saveTG{𦆜}{27982}
\saveTG{𥿇}{27982}
\saveTG{㱍}{27982}
\saveTG{𦆦}{27982}
\saveTG{𣣗}{27982}
\saveTG{𫃴}{27984}
\saveTG{緱}{27984}
\saveTG{稧}{27984}
\saveTG{𥠅}{27984}
\saveTG{𤚙}{27984}
\saveTG{𦂐}{27984}
\saveTG{𦃺}{27984}
\saveTG{𪫇}{27986}
\saveTG{䆅}{27986}
\saveTG{𦇛}{27986}
\saveTG{𦄻}{27986}
\saveTG{𥛤}{27986}
\saveTG{䋇}{27987}
\saveTG{釈}{27987}
\saveTG{𦇳}{27989}
\saveTG{𦇐}{27989}
\saveTG{𥤎}{27989}
\saveTG{𦄜}{27991}
\saveTG{𢑗}{27991}
\saveTG{縩}{27991}
\saveTG{𣱹}{27991}
\saveTG{穄}{27991}
\saveTG{称}{27992}
\saveTG{𢑛}{27992}
\saveTG{綠}{27992}
\saveTG{𦃡}{27993}
\saveTG{𦂞}{27993}
\saveTG{𦣐}{27993}
\saveTG{𥢪}{27993}
\saveTG{𦂋}{27994}
\saveTG{縧}{27994}
\saveTG{𦃨}{27994}
\saveTG{𥿰}{27994}
\saveTG{𥠊}{27994}
\saveTG{䋴}{27994}
\saveTG{縔}{27994}
\saveTG{𥢽}{27994}
\saveTG{𥛬}{27994}
\saveTG{𥞛}{27994}
\saveTG{𦀉}{27994}
\saveTG{𥣵}{27999}
\saveTG{緑}{27999}
\saveTG{乆}{28000}
\saveTG{牏}{28021}
\saveTG{𤖭}{28027}
\saveTG{𪺥}{28030}
\saveTG{𤗉}{28033}
\saveTG{𤗆}{28054}
\saveTG{临}{28063}
\saveTG{㸝}{28082}
\saveTG{纵}{28100}
\saveTG{𪗔}{28100}
\saveTG{盭}{28102}
\saveTG{𥁮}{28102}
\saveTG{监}{28102}
\saveTG{𦥋}{28104}
\saveTG{𡉡}{28104}
\saveTG{𠏳}{28107}
\saveTG{𨪔}{28109}
\saveTG{鉴}{28109}
\saveTG{鋚}{28109}
\saveTG{𨨼}{28109}
\saveTG{鲊}{28111}
\saveTG{缢}{28112}
\saveTG{缆}{28112}
\saveTG{纶}{28112}
\saveTG{䍀}{28112}
\saveTG{纥}{28117}
\saveTG{䲝}{28117}
\saveTG{䌼}{28117}
\saveTG{𫚛}{28117}
\saveTG{𨯿}{28119}
\saveTG{}{28120}
\saveTG{𦈕}{28120}
\saveTG{𦑸}{28127}
\saveTG{𤯸}{28127}
\saveTG{𫄛}{28127}
\saveTG{𫚍}{28127}
\saveTG{𩽿}{28127}
\saveTG{𦐻}{28127}
\saveTG{}{28127}
\saveTG{豒}{28127}
\saveTG{绨}{28127}
\saveTG{纷}{28127}
\saveTG{𫚤}{28131}
\saveTG{鲙}{28132}
\saveTG{鲶}{28132}
\saveTG{绘}{28132}
\saveTG{㱓}{28132}
\saveTG{䍁}{28132}
\saveTG{𧐱}{28136}
\saveTG{𧌁}{28136}
\saveTG{缣}{28137}
\saveTG{鳒}{28137}
\saveTG{𣿗}{28140}
\saveTG{𢽠}{28140}
\saveTG{𩽻}{28140}
\saveTG{䲎}{28140}
\saveTG{𩷊}{28140}
\saveTG{𢼇}{28140}
\saveTG{𤯛}{28140}
\saveTG{敳}{28140}
\saveTG{缴}{28140}
\saveTG{𣦝}{28143}
\saveTG{𪶜}{28144}
\saveTG{𡐱}{28144}
\saveTG{𡓼}{28146}
\saveTG{鳟}{28146}
\saveTG{鳆}{28147}
\saveTG{𣦇}{28147}
\saveTG{鲜}{28151}
\saveTG{}{28153}
\saveTG{𦈎}{28156}
\saveTG{𪉕}{28156}
\saveTG{𫚗}{28161}
\saveTG{给}{28161}
\saveTG{缮}{28161}
\saveTG{鳝}{28161}
\saveTG{𧖼}{28164}
\saveTG{䲡}{28164}
\saveTG{}{28164}
\saveTG{缯}{28166}
\saveTG{䥘}{28168}
\saveTG{绤}{28168}
\saveTG{𣳬}{28172}
\saveTG{鳤}{28177}
\saveTG{㵪}{28182}
\saveTG{𣾇}{28184}
\saveTG{𪶅}{28191}
\saveTG{𢓅}{28200}
\saveTG{𠈽}{28200}
\saveTG{𣦸}{28200}
\saveTG{䖋}{28200}
\saveTG{𢓙}{28200}
\saveTG{似}{28200}
\saveTG{仈}{28200}
\saveTG{㐺}{28200}
\saveTG{𠆧}{28200}
\saveTG{𥲽}{28201}
\saveTG{𫕚}{28208}
\saveTG{𫙈}{28210}
\saveTG{𠈣}{28210}
\saveTG{𧲮}{28211}
\saveTG{𧣝}{28211}
\saveTG{𢓓}{28211}
\saveTG{𪫗}{28211}
\saveTG{𩴵}{28211}
\saveTG{作}{28211}
\saveTG{𩴌}{28211}
\saveTG{伦}{28212}
\saveTG{𧥈}{28212}
\saveTG{览}{28212}
\saveTG{儖}{28212}
\saveTG{𠈨}{28212}
\saveTG{㑅}{28212}
\saveTG{侻}{28212}
\saveTG{伧}{28212}
\saveTG{𪝞}{28212}
\saveTG{𨉋}{28212}
\saveTG{傞}{28212}
\saveTG{貖}{28212}
\saveTG{𠈝}{28212}
\saveTG{𠑈}{28212}
\saveTG{𪝕}{28212}
\saveTG{𩲝}{28212}
\saveTG{𠊧}{28212}
\saveTG{𧆷}{28213}
\saveTG{𩲩}{28213}
\saveTG{侳}{28214}
\saveTG{佺}{28214}
\saveTG{𩴰}{28214}
\saveTG{𪖜}{28214}
\saveTG{𩀰}{28215}
\saveTG{䰮}{28215}
\saveTG{𡖽}{28216}
\saveTG{𩳋}{28216}
\saveTG{𧇎}{28216}
\saveTG{𩦻}{28217}
\saveTG{𡢔}{28217}
\saveTG{𧤟}{28217}
\saveTG{𤕴}{28217}
\saveTG{仡}{28217}
\saveTG{𧣟}{28217}
\saveTG{𠏌}{28217}
\saveTG{㠩}{28217}
\saveTG{𠊡}{28217}
\saveTG{𧇐}{28217}
\saveTG{㣞}{28217}
\saveTG{𧢓}{28217}
\saveTG{𧤄}{28217}
\saveTG{𡲡}{28217}
\saveTG{𧇓}{28217}
\saveTG{𧇞}{28217}
\saveTG{㑶}{28217}
\saveTG{𤖌}{28217}
\saveTG{㐹}{28217}
\saveTG{𪖴}{28217}
\saveTG{𨝟}{28217}
\saveTG{𧆫}{28217}
\saveTG{𪞄}{28218}
\saveTG{𥏋}{28218}
\saveTG{俭}{28219}
\saveTG{𠊄}{28219}
\saveTG{价}{28220}
\saveTG{𧣋}{28220}
\saveTG{𢔢}{28220}
\saveTG{偂}{28221}
\saveTG{貐}{28221}
\saveTG{偷}{28221}
\saveTG{𢔜}{28222}
\saveTG{㒐}{28226}
\saveTG{𪝓}{28227}
\saveTG{𠆵}{28227}
\saveTG{䚫}{28227}
\saveTG{𦡒}{28227}
\saveTG{傷}{28227}
\saveTG{𧇾}{28227}
\saveTG{𧴐}{28227}
\saveTG{𠃰}{28227}
\saveTG{𧣵}{28227}
\saveTG{㣢}{28227}
\saveTG{𠊈}{28227}
\saveTG{觴}{28227}
\saveTG{傟}{28227}
\saveTG{佾}{28227}
\saveTG{𧳋}{28227}
\saveTG{𢔍}{28227}
\saveTG{份}{28227}
\saveTG{𪖦}{28227}
\saveTG{俤}{28227}
\saveTG{躮}{28227}
\saveTG{㒆}{28227}
\saveTG{𨿘}{28227}
\saveTG{仱}{28227}
\saveTG{倫}{28227}
\saveTG{𧴉}{28227}
\saveTG{㒢}{28227}
\saveTG{𠐉}{28227}
\saveTG{伤}{28227}
\saveTG{觞}{28227}
\saveTG{㑫}{28230}
\saveTG{𠀳}{28231}
\saveTG{㣳}{28231}
\saveTG{䚗}{28231}
\saveTG{𢔋}{28231}
\saveTG{𤕵}{28232}
\saveTG{𠋡}{28232}
\saveTG{飨}{28232}
\saveTG{𠏚}{28232}
\saveTG{𧥃}{28232}
\saveTG{𥢁}{28232}
\saveTG{𠍵}{28232}
\saveTG{𩛈}{28232}
\saveTG{𧥐}{28232}
\saveTG{𢕦}{28232}
\saveTG{𪝎}{28232}
\saveTG{𠐒}{28232}
\saveTG{侩}{28232}
\saveTG{伶}{28232}
\saveTG{彾}{28232}
\saveTG{倯}{28232}
\saveTG{飧}{28232}
\saveTG{偸}{28232}
\saveTG{彸}{28232}
\saveTG{伀}{28232}
\saveTG{𧫹}{28232}
\saveTG{倊}{28233}
\saveTG{𠐣}{28234}
\saveTG{𧒥}{28236}
\saveTG{𪖳}{28237}
\saveTG{傔}{28237}
\saveTG{𠌡}{28238}
\saveTG{徽}{28240}
\saveTG{幑}{28240}
\saveTG{徼}{28240}
\saveTG{敫}{28240}
\saveTG{儌}{28240}
\saveTG{敿}{28240}
\saveTG{儆}{28240}
\saveTG{徾}{28240}
\saveTG{黴}{28240}
\saveTG{微}{28240}
\saveTG{仵}{28240}
\saveTG{傚}{28240}
\saveTG{僌}{28240}
\saveTG{攸}{28240}
\saveTG{敒}{28240}
\saveTG{做}{28240}
\saveTG{𢽘}{28240}
\saveTG{𢾕}{28240}
\saveTG{𪫏}{28240}
\saveTG{𧈓}{28240}
\saveTG{䱷}{28240}
\saveTG{𢽋}{28240}
\saveTG{𢽸}{28240}
\saveTG{𢼸}{28240}
\saveTG{𣁋}{28240}
\saveTG{𢿗}{28240}
\saveTG{𢼮}{28240}
\saveTG{𡖙}{28240}
\saveTG{𢿬}{28240}
\saveTG{𢕧}{28240}
\saveTG{𢕿}{28240}
\saveTG{𢕴}{28240}
\saveTG{𢕡}{28240}
\saveTG{𢖉}{28240}
\saveTG{𢕟}{28240}
\saveTG{𪖖}{28240}
\saveTG{𢔼}{28240}
\saveTG{𢕣}{28240}
\saveTG{}{28240}
\saveTG{}{28240}
\saveTG{𢕄}{28240}
\saveTG{㣲}{28240}
\saveTG{𢓹}{28240}
\saveTG{𢖄}{28240}
\saveTG{𢽈}{28240}
\saveTG{𢕭}{28240}
\saveTG{𢖜}{28240}
\saveTG{𢕲}{28240}
\saveTG{𢕹}{28240}
\saveTG{𠎭}{28240}
\saveTG{𠊮}{28240}
\saveTG{𠍏}{28240}
\saveTG{𠊹}{28240}
\saveTG{𠈹}{28240}
\saveTG{𠌒}{28240}
\saveTG{𠉿}{28240}
\saveTG{𠐍}{28240}
\saveTG{𠌝}{28240}
\saveTG{㒈}{28240}
\saveTG{𠈅}{28240}
\saveTG{𠍯}{28240}
\saveTG{𠇑}{28240}
\saveTG{𢻺}{28240}
\saveTG{𠎄}{28240}
\saveTG{𪫕}{28240}
\saveTG{𢨵}{28240}
\saveTG{𤚓}{28240}
\saveTG{傲}{28240}
\saveTG{徶}{28240}
\saveTG{僘}{28240}
\saveTG{徹}{28240}
\saveTG{徴}{28240}
\saveTG{徵}{28240}
\saveTG{倣}{28240}
\saveTG{併}{28241}
\saveTG{𢕈}{28242}
\saveTG{𠎬}{28243}
\saveTG{𠑥}{28243}
\saveTG{𠐵}{28243}
\saveTG{𢕰}{28243}
\saveTG{𠐫}{28243}
\saveTG{𠌚}{28244}
\saveTG{𧳉}{28244}
\saveTG{𠋄}{28244}
\saveTG{僔}{28246}
\saveTG{𫏭}{28247}
\saveTG{𠈼}{28247}
\saveTG{復}{28247}
\saveTG{𢽆}{28248}
\saveTG{𠇪}{28250}
\saveTG{𠇫}{28250}
\saveTG{𦍹}{28251}
\saveTG{儛}{28251}
\saveTG{𫙊}{28251}
\saveTG{觧}{28251}
\saveTG{觲}{28251}
\saveTG{佯}{28251}
\saveTG{徉}{28251}
\saveTG{牂}{28251}
\saveTG{𦍲}{28251}
\saveTG{𠌧}{28252}
\saveTG{𠈠}{28252}
\saveTG{𣨦}{28253}
\saveTG{儀}{28253}
\saveTG{𪖫}{28254}
\saveTG{觯}{28256}
\saveTG{𠎕}{28257}
\saveTG{侮}{28257}
\saveTG{㒇}{28257}
\saveTG{𪖝}{28260}
\saveTG{𧳇}{28261}
\saveTG{僐}{28261}
\saveTG{㣛}{28261}
\saveTG{佮}{28261}
\saveTG{𠏜}{28261}
\saveTG{𠎨}{28261}
\saveTG{𠍹}{28261}
\saveTG{𣨄}{28261}
\saveTG{𣄅}{28262}
\saveTG{𧳘}{28262}
\saveTG{𠉐}{28262}
\saveTG{偤}{28264}
\saveTG{𧤕}{28264}
\saveTG{𧳫}{28264}
\saveTG{𠊆}{28264}
\saveTG{倽}{28264}
\saveTG{𠌪}{28264}
\saveTG{𥶯}{28265}
\saveTG{𣍐}{28266}
\saveTG{儈}{28266}
\saveTG{徻}{28266}
\saveTG{僧}{28266}
\saveTG{𫏱}{28266}
\saveTG{𠐼}{28266}
\saveTG{㬟}{28266}
\saveTG{𧴚}{28266}
\saveTG{䶐}{28266}
\saveTG{牄}{28267}
\saveTG{傖}{28267}
\saveTG{𧣳}{28268}
\saveTG{𢓾}{28268}
\saveTG{𠈻}{28268}
\saveTG{𪺡}{28268}
\saveTG{俗}{28268}
\saveTG{佡}{28272}
\saveTG{𦚥}{28272}
\saveTG{𠉚}{28272}
\saveTG{𠇒}{28280}
\saveTG{𢔩}{28281}
\saveTG{傱}{28281}
\saveTG{從}{28281}
\saveTG{従}{28281}
\saveTG{𩀨}{28282}
\saveTG{𢕩}{28282}
\saveTG{𢕇}{28282}
\saveTG{𨁀}{28282}
\saveTG{𢕐}{28282}
\saveTG{躾}{28284}
\saveTG{𪝝}{28284}
\saveTG{𠈪}{28284}
\saveTG{𤕼}{28284}
\saveTG{倹}{28286}
\saveTG{𥜋}{28286}
\saveTG{𧸞}{28286}
\saveTG{𥤒}{28286}
\saveTG{僋}{28286}
\saveTG{儉}{28286}
\saveTG{𠊅}{28289}
\saveTG{伱}{28290}
\saveTG{徐}{28294}
\saveTG{俆}{28294}
\saveTG{𪝉}{28294}
\saveTG{魜}{28300}
\saveTG{魞}{28300}
\saveTG{𩵒}{28300}
\saveTG{𨘻}{28303}
\saveTG{鮓}{28311}
\saveTG{𩺻}{28312}
\saveTG{𩼳}{28312}
\saveTG{䱹}{28312}
\saveTG{𩹁}{28312}
\saveTG{鱃}{28312}
\saveTG{鮵}{28312}
\saveTG{𫙱}{28317}
\saveTG{𩸑}{28317}
\saveTG{𩶡}{28317}
\saveTG{𩵸}{28317}
\saveTG{𪁸}{28317}
\saveTG{𩸟}{28317}
\saveTG{𩷿}{28317}
\saveTG{𩹦}{28317}
\saveTG{𩶸}{28317}
\saveTG{𫚿}{28317}
\saveTG{䰿}{28317}
\saveTG{䰴}{28317}
\saveTG{𩻒}{28318}
\saveTG{𩸱}{28319}
\saveTG{𩾴}{28320}
\saveTG{魪}{28320}
\saveTG{𪂀}{28321}
\saveTG{𩷲}{28322}
\saveTG{𫙯}{28322}
\saveTG{䰼}{28327}
\saveTG{𩻵}{28327}
\saveTG{㦘}{28327}
\saveTG{䳷}{28327}
\saveTG{𪅭}{28327}
\saveTG{𫙲}{28327}
\saveTG{䲙}{28327}
\saveTG{𩦨}{28327}
\saveTG{魵}{28327}
\saveTG{鸄}{28327}
\saveTG{鯩}{28327}
\saveTG{鮷}{28327}
\saveTG{𩽖}{28327}
\saveTG{𪁩}{28327}
\saveTG{𩷸}{28327}
\saveTG{𩸍}{28327}
\saveTG{𩸂}{28330}
\saveTG{𪫱}{28330}
\saveTG{𩸈}{28331}
\saveTG{𩻚}{28331}
\saveTG{㤀}{28331}
\saveTG{㤰}{28331}
\saveTG{𩷆}{28331}
\saveTG{𩹑}{28331}
\saveTG{䰸}{28331}
\saveTG{𢠚}{28332}
\saveTG{𢢱}{28332}
\saveTG{鰦}{28332}
\saveTG{鱶}{28332}
\saveTG{鯰}{28332}
\saveTG{魿}{28332}
\saveTG{鯲}{28333}
\saveTG{煞}{28334}
\saveTG{懲}{28334}
\saveTG{𪹉}{28334}
\saveTG{𤐳}{28334}
\saveTG{𢜕}{28334}
\saveTG{𢛮}{28334}
\saveTG{𤋈}{28334}
\saveTG{悠}{28334}
\saveTG{𤏺}{28334}
\saveTG{𪄧}{28334}
\saveTG{𢞀}{28334}
\saveTG{𩺝}{28334}
\saveTG{𢚐}{28334}
\saveTG{𢛃}{28334}
\saveTG{𤋥}{28334}
\saveTG{𢢡}{28334}
\saveTG{𢢑}{28336}
\saveTG{𩼡}{28336}
\saveTG{𡿬}{28337}
\saveTG{鰜}{28337}
\saveTG{慫}{28338}
\saveTG{𢟂}{28339}
\saveTG{𪯏}{28340}
\saveTG{𪈶}{28340}
\saveTG{鷻}{28340}
\saveTG{鰴}{28340}
\saveTG{𩹈}{28340}
\saveTG{𩽋}{28340}
\saveTG{𩺕}{28340}
\saveTG{𪑛}{28340}
\saveTG{𩸳}{28340}
\saveTG{𩵱}{28340}
\saveTG{䲄}{28340}
\saveTG{𩼌}{28340}
\saveTG{𩽚}{28340}
\saveTG{𩵹}{28340}
\saveTG{䰻}{28340}
\saveTG{䱔}{28340}
\saveTG{鮩}{28341}
\saveTG{𩼛}{28341}
\saveTG{𫙳}{28343}
\saveTG{𩹖}{28344}
\saveTG{𪄟}{28344}
\saveTG{鱒}{28346}
\saveTG{鰒}{28347}
\saveTG{𩽕}{28347}
\saveTG{𩹊}{28347}
\saveTG{鮮}{28351}
\saveTG{𩽡}{28352}
\saveTG{鸃}{28353}
\saveTG{鱵}{28353}
\saveTG{䱕}{28354}
\saveTG{䲑}{28355}
\saveTG{𪇳}{28356}
\saveTG{鮯}{28361}
\saveTG{䲕}{28361}
\saveTG{𨗡}{28361}
\saveTG{鱔}{28361}
\saveTG{鰌}{28364}
\saveTG{鱠}{28366}
\saveTG{鱛}{28366}
\saveTG{䱽}{28367}
\saveTG{𩷝}{28368}
\saveTG{𪁴}{28368}
\saveTG{𩻶}{28368}
\saveTG{䲘}{28377}
\saveTG{𪇥}{28381}
\saveTG{䲂}{28382}
\saveTG{𪅜}{28382}
\saveTG{𩻀}{28384}
\saveTG{𩶇}{28384}
\saveTG{䲓}{28386}
\saveTG{鵌}{28394}
\saveTG{鮽}{28394}
\saveTG{𦨈}{28400}
\saveTG{𦖴}{28401}
\saveTG{𫆉}{28401}
\saveTG{𫆃}{28401}
\saveTG{𦗷}{28401}
\saveTG{聳}{28401}
\saveTG{㛜}{28404}
\saveTG{𡚶}{28404}
\saveTG{㝁}{28407}
\saveTG{𪍾}{28407}
\saveTG{舴}{28411}
\saveTG{𦪮}{28412}
\saveTG{𥃢}{28412}
\saveTG{𦣸}{28412}
\saveTG{𨊔}{28412}
\saveTG{舱}{28412}
\saveTG{艖}{28412}
\saveTG{艗}{28412}
\saveTG{艦}{28412}
\saveTG{艞}{28413}
\saveTG{𦪋}{28414}
\saveTG{𦩃}{28417}
\saveTG{𥡉}{28417}
\saveTG{𦨏}{28417}
\saveTG{䑨}{28417}
\saveTG{𦩞}{28420}
\saveTG{䠼}{28420}
\saveTG{𨉅}{28421}
\saveTG{𨉶}{28421}
\saveTG{𦩔}{28422}
\saveTG{𠤂}{28422}
\saveTG{𦪙}{28427}
\saveTG{𨉟}{28427}
\saveTG{𨉓}{28427}
\saveTG{䑯}{28427}
\saveTG{勶}{28427}
\saveTG{𨉳}{28427}
\saveTG{䑳}{28427}
\saveTG{䑤}{28427}
\saveTG{舩}{28432}
\saveTG{艌}{28432}
\saveTG{䠲}{28432}
\saveTG{舲}{28432}
\saveTG{𦩵}{28437}
\saveTG{𢿐}{28440}
\saveTG{䒆}{28440}
\saveTG{𢽌}{28440}
\saveTG{𦪧}{28440}
\saveTG{𦪔}{28440}
\saveTG{𣀐}{28440}
\saveTG{𨈡}{28440}
\saveTG{𨈾}{28441}
\saveTG{𦩟}{28442}
\saveTG{䒀}{28443}
\saveTG{𦪚}{28443}
\saveTG{𢍸}{28444}
\saveTG{䑻}{28447}
\saveTG{𦪰}{28447}
\saveTG{𦎒}{28451}
\saveTG{𫇣}{28453}
\saveTG{艤}{28453}
\saveTG{𦪪}{28454}
\saveTG{𫇢}{28454}
\saveTG{䒉}{28457}
\saveTG{𨈴}{28460}
\saveTG{䑪}{28461}
\saveTG{艏}{28462}
\saveTG{𨉾}{28463}
\saveTG{𦩲}{28464}
\saveTG{艙}{28467}
\saveTG{谸}{28468}
\saveTG{𢾢}{28480}
\saveTG{𥐈}{28480}
\saveTG{𨈦}{28480}
\saveTG{𦩂}{28482}
\saveTG{𨈽}{28484}
\saveTG{𦪵}{28484}
\saveTG{𦨶}{28484}
\saveTG{䠶}{28484}
\saveTG{𥎻}{28484}
\saveTG{𧷽}{28486}
\saveTG{𨉺}{28486}
\saveTG{艅}{28494}
\saveTG{𦪕}{28496}
\saveTG{𢶗}{28502}
\saveTG{㧛}{28502}
\saveTG{𤙘}{28504}
\saveTG{䩦}{28506}
\saveTG{㹐}{28508}
\saveTG{𫄐}{28511}
\saveTG{㸲}{28511}
\saveTG{魀}{28512}
\saveTG{魐}{28513}
\saveTG{𫙉}{28514}
\saveTG{牷}{28514}
\saveTG{㸱}{28517}
\saveTG{𤙚}{28517}
\saveTG{犔}{28517}
\saveTG{𤘦}{28520}
\saveTG{𤚎}{28520}
\saveTG{𤙁}{28522}
\saveTG{𢶡}{28524}
\saveTG{㸮}{28527}
\saveTG{𤛣}{28527}
\saveTG{𤘡}{28527}
\saveTG{𤚋}{28530}
\saveTG{𢵩}{28532}
\saveTG{㸳}{28532}
\saveTG{牧}{28540}
\saveTG{犜}{28540}
\saveTG{𤚖}{28544}
\saveTG{犪}{28547}
\saveTG{𤚇}{28547}
\saveTG{䍧}{28551}
\saveTG{犠}{28553}
\saveTG{犧}{28553}
\saveTG{𤙩}{28554}
\saveTG{𥣿}{28557}
\saveTG{𤚥}{28562}
\saveTG{𤙱}{28564}
\saveTG{𤛢}{28566}
\saveTG{𤚬}{28567}
\saveTG{𤙛}{28594}
\saveTG{䛓}{28601}
\saveTG{譥}{28601}
\saveTG{𥋪}{28604}
\saveTG{𣈊}{28604}
\saveTG{䀺}{28604}
\saveTG{𥌟}{28609}
\saveTG{𪊇}{28612}
\saveTG{𥃠}{28612}
\saveTG{𩡚}{28612}
\saveTG{鹾}{28612}
\saveTG{鹺}{28612}
\saveTG{𫗽}{28614}
\saveTG{𣱡}{28617}
\saveTG{𪉡}{28617}
\saveTG{𤽍}{28617}
\saveTG{𫑷}{28619}
\saveTG{𪉰}{28620}
\saveTG{𤽉}{28627}
\saveTG{𦧣}{28627}
\saveTG{𩡓}{28627}
\saveTG{𩡡}{28627}
\saveTG{馚}{28627}
\saveTG{鹶}{28627}
\saveTG{𩠻}{28627}
\saveTG{𤽿}{28630}
\saveTG{𦤉}{28631}
\saveTG{𡅖}{28632}
\saveTG{𩡁}{28632}
\saveTG{皊}{28632}
\saveTG{鹷}{28632}
\saveTG{鹻}{28637}
\saveTG{馦}{28637}
\saveTG{𫇗}{28637}
\saveTG{𪉾}{28640}
\saveTG{𪉿}{28640}
\saveTG{𢿶}{28640}
\saveTG{𫘀}{28640}
\saveTG{𢾷}{28640}
\saveTG{𢽍}{28640}
\saveTG{敌}{28640}
\saveTG{敋}{28640}
\saveTG{皦}{28640}
\saveTG{敂}{28640}
\saveTG{敀}{28640}
\saveTG{𢼉}{28640}
\saveTG{皏}{28641}
\saveTG{𤾵}{28644}
\saveTG{馥}{28647}
\saveTG{𪉢}{28648}
\saveTG{𪾅}{28653}
\saveTG{𪉥}{28657}
\saveTG{𦧛}{28661}
\saveTG{𧭁}{28661}
\saveTG{䭫}{28662}
\saveTG{馠}{28662}
\saveTG{𩠥}{28662}
\saveTG{𤾥}{28666}
\saveTG{𤾙}{28667}
\saveTG{𧯄}{28668}
\saveTG{𤽦}{28672}
\saveTG{鹸}{28686}
\saveTG{鹼}{28686}
\saveTG{以}{28700}
\saveTG{𠓨}{28710}
\saveTG{亾}{28710}
\saveTG{兦}{28710}
\saveTG{𫗢}{28711}
\saveTG{𠆦}{28711}
\saveTG{𣬿}{28711}
\saveTG{𣰦}{28711}
\saveTG{齚}{28711}
\saveTG{岞}{28711}
\saveTG{𪙉}{28712}
\saveTG{𡽳}{28712}
\saveTG{𣭕}{28712}
\saveTG{𡻘}{28712}
\saveTG{𡺬}{28712}
\saveTG{㲐}{28712}
\saveTG{𣯾}{28712}
\saveTG{𪚬}{28712}
\saveTG{嵯}{28712}
\saveTG{毺}{28712}
\saveTG{馐}{28712}
\saveTG{齸}{28712}
\saveTG{𣬫}{28712}
\saveTG{𣰒}{28714}
\saveTG{𣰓}{28714}
\saveTG{𣰪}{28714}
\saveTG{𣮙}{28714}
\saveTG{𣬲}{28714}
\saveTG{𣰎}{28714}
\saveTG{𢼁}{28714}
\saveTG{𣯨}{28715}
\saveTG{㲮}{28716}
\saveTG{𣭝}{28716}
\saveTG{𣯙}{28716}
\saveTG{𣮞}{28716}
\saveTG{𣯿}{28716}
\saveTG{氆}{28716}
\saveTG{㟋}{28717}
\saveTG{𡷋}{28717}
\saveTG{𩠂}{28717}
\saveTG{𣰳}{28717}
\saveTG{𪓶}{28717}
\saveTG{𪓾}{28717}
\saveTG{𡹳}{28717}
\saveTG{𡺜}{28717}
\saveTG{㼻}{28717}
\saveTG{屹}{28717}
\saveTG{齕}{28717}
\saveTG{龁}{28717}
\saveTG{饩}{28717}
\saveTG{𪨦}{28717}
\saveTG{𡶊}{28718}
\saveTG{𣮺}{28718}
\saveTG{𪵜}{28718}
\saveTG{𣯘}{28719}
\saveTG{崄}{28719}
\saveTG{崯}{28719}
\saveTG{𡵚}{28720}
\saveTG{𢇄}{28720}
\saveTG{齘}{28720}
\saveTG{}{28720}
\saveTG{崳}{28721}
\saveTG{岎}{28727}
\saveTG{𡾬}{28727}
\saveTG{饰}{28727}
\saveTG{饬}{28727}
\saveTG{𪙊}{28727}
\saveTG{岒}{28727}
\saveTG{𡻐}{28727}
\saveTG{𪗣}{28727}
\saveTG{䶖}{28727}
\saveTG{崘}{28727}
\saveTG{嶖}{28727}
\saveTG{𡹓}{28730}
\saveTG{𩠈}{28730}
\saveTG{㟱}{28731}
\saveTG{㟣}{28731}
\saveTG{𡵴}{28731}
\saveTG{𫓪}{28731}
\saveTG{𨺵}{28732}
\saveTG{𡻨}{28732}
\saveTG{𠎜}{28732}
\saveTG{𩞐}{28732}
\saveTG{𩛢}{28732}
\saveTG{岭}{28732}
\saveTG{龄}{28732}
\saveTG{齢}{28732}
\saveTG{嵫}{28732}
\saveTG{齡}{28732}
\saveTG{𫗱}{28737}
\saveTG{𠔨}{28737}
\saveTG{嵰}{28737}
\saveTG{𩠌}{28738}
\saveTG{𡵛}{28740}
\saveTG{𢽶}{28740}
\saveTG{𪯌}{28740}
\saveTG{𢻹}{28740}
\saveTG{馓}{28740}
\saveTG{收}{28740}
\saveTG{㞰}{28740}
\saveTG{𢼃}{28740}
\saveTG{𡵲}{28740}
\saveTG{饼}{28741}
\saveTG{𪚈}{28743}
\saveTG{𪘀}{28744}
\saveTG{𪚏}{28744}
\saveTG{嶟}{28746}
\saveTG{巙}{28747}
\saveTG{𪙅}{28747}
\saveTG{𡹀}{28748}
\saveTG{㟄}{28751}
\saveTG{𡾮}{28751}
\saveTG{𢆭}{28752}
\saveTG{嶬}{28753}
\saveTG{𡾞}{28754}
\saveTG{𪙴}{28755}
\saveTG{饸}{28761}
\saveTG{𪘁}{28761}
\saveTG{峆}{28761}
\saveTG{㟏}{28762}
\saveTG{𪘒}{28762}
\saveTG{崷}{28764}
\saveTG{𡻡}{28764}
\saveTG{嶒}{28766}
\saveTG{𡼾}{28766}
\saveTG{𡻵}{28766}
\saveTG{嵢}{28767}
\saveTG{𪙎}{28767}
\saveTG{𪙭}{28768}
\saveTG{𪙮}{28768}
\saveTG{𡹒}{28768}
\saveTG{𡼪}{28768}
\saveTG{峪}{28768}
\saveTG{𪘗}{28772}
\saveTG{𡺧}{28772}
\saveTG{㠞}{28774}
\saveTG{嵷}{28781}
\saveTG{𩠍}{28782}
\saveTG{嵄}{28784}
\saveTG{𡼓}{28784}
\saveTG{𠤑}{28784}
\saveTG{𡸴}{28785}
\saveTG{䶨}{28786}
\saveTG{嶮}{28786}
\saveTG{𡼺}{28791}
\saveTG{𡷣}{28794}
\saveTG{馀}{28794}
\saveTG{跾}{28801}
\saveTG{𨇓}{28802}
\saveTG{𨄦}{28802}
\saveTG{𤢝}{28804}
\saveTG{𧸟}{28806}
\saveTG{𨕽}{28808}
\saveTG{𨑢}{28808}
\saveTG{熧}{28809}
\saveTG{㷺}{28809}
\saveTG{焂}{28809}
\saveTG{𤒮}{28809}
\saveTG{𣥐}{28812}
\saveTG{𣭥}{28813}
\saveTG{㡮}{28817}
\saveTG{𢇓}{28817}
\saveTG{𥟬}{28827}
\saveTG{𧎃}{28831}
\saveTG{𤑃}{28837}
\saveTG{𣀿}{28840}
\saveTG{𪵊}{28840}
\saveTG{𢿓}{28840}
\saveTG{𢾡}{28840}
\saveTG{谿}{28868}
\saveTG{𪜏}{28872}
\saveTG{𦤰}{28886}
\saveTG{㷏}{28894}
\saveTG{𥜄}{28901}
\saveTG{𣭔}{28903}
\saveTG{𣒼}{28904}
\saveTG{𥾺}{28908}
\saveTG{𡖗}{28911}
\saveTG{䋏}{28911}
\saveTG{秨}{28911}
\saveTG{𥤟}{28912}
\saveTG{𩁱}{28912}
\saveTG{縒}{28912}
\saveTG{綐}{28912}
\saveTG{繿}{28912}
\saveTG{纜}{28912}
\saveTG{絁}{28912}
\saveTG{税}{28912}
\saveTG{稅}{28912}
\saveTG{縊}{28912}
\saveTG{𦄙}{28912}
\saveTG{𦆧}{28912}
\saveTG{𦇲}{28914}
\saveTG{𥣺}{28914}
\saveTG{絟}{28914}
\saveTG{𥟨}{28915}
\saveTG{𥞀}{28917}
\saveTG{紇}{28917}
\saveTG{𦂛}{28917}
\saveTG{𥞙}{28917}
\saveTG{𦆳}{28917}
\saveTG{𥿪}{28917}
\saveTG{𥣍}{28917}
\saveTG{𥾨}{28917}
\saveTG{𥝖}{28917}
\saveTG{𫃨}{28917}
\saveTG{𥠥}{28917}
\saveTG{𥤝}{28917}
\saveTG{𥝬}{28917}
\saveTG{𡰇}{28917}
\saveTG{䌫}{28917}
\saveTG{䋮}{28919}
\saveTG{𥾓}{28920}
\saveTG{紒}{28920}
\saveTG{𥠕}{28920}
\saveTG{𥝵}{28920}
\saveTG{緰}{28921}
\saveTG{𦂒}{28921}
\saveTG{𥤇}{28922}
\saveTG{紾}{28922}
\saveTG{紟}{28927}
\saveTG{𦁸}{28927}
\saveTG{𦅖}{28927}
\saveTG{秎}{28927}
\saveTG{𦂚}{28927}
\saveTG{𥤉}{28927}
\saveTG{𥿐}{28927}
\saveTG{𪏱}{28927}
\saveTG{𦇬}{28927}
\saveTG{𦆤}{28927}
\saveTG{𥣮}{28927}
\saveTG{綸}{28927}
\saveTG{𥡺}{28927}
\saveTG{𣘈}{28927}
\saveTG{紛}{28927}
\saveTG{綈}{28927}
\saveTG{稊}{28927}
\saveTG{稐}{28927}
\saveTG{𦁤}{28930}
\saveTG{𦄸}{28931}
\saveTG{𦁇}{28931}
\saveTG{𦅾}{28931}
\saveTG{𧌢}{28931}
\saveTG{䅵}{28931}
\saveTG{𦁌}{28931}
\saveTG{䌗}{28931}
\saveTG{𥝶}{28931}
\saveTG{稔}{28932}
\saveTG{絵}{28932}
\saveTG{𥠂}{28932}
\saveTG{𦅭}{28932}
\saveTG{紷}{28932}
\saveTG{𦂁}{28932}
\saveTG{秢}{28932}
\saveTG{稵}{28932}
\saveTG{𥣒}{28932}
\saveTG{𥣓}{28933}
\saveTG{𥾪}{28933}
\saveTG{𥝨}{28933}
\saveTG{繸}{28933}
\saveTG{穟}{28933}
\saveTG{総}{28933}
\saveTG{縌}{28934}
\saveTG{繺}{28934}
\saveTG{縂}{28936}
\saveTG{縑}{28937}
\saveTG{𪐋}{28937}
\saveTG{稴}{28937}
\saveTG{𦂵}{28938}
\saveTG{𦇁}{28940}
\saveTG{𢼡}{28940}
\saveTG{𥠽}{28940}
\saveTG{𣞊}{28940}
\saveTG{𥾿}{28940}
\saveTG{繳}{28940}
\saveTG{敹}{28940}
\saveTG{繖}{28940}
\saveTG{緻}{28940}
\saveTG{𦁼}{28940}
\saveTG{𦇕}{28940}
\saveTG{𥢹}{28940}
\saveTG{𦆮}{28940}
\saveTG{𦅄}{28940}
\saveTG{𥝺}{28940}
\saveTG{𥢭}{28940}
\saveTG{𦅣}{28940}
\saveTG{𦥐}{28940}
\saveTG{𢾰}{28940}
\saveTG{𢿔}{28940}
\saveTG{𥿬}{28941}
\saveTG{絣}{28941}
\saveTG{𥢜}{28943}
\saveTG{𦅯}{28943}
\saveTG{𥢎}{28943}
\saveTG{𥞩}{28944}
\saveTG{𥟣}{28944}
\saveTG{繜}{28946}
\saveTG{𪲹}{28946}
\saveTG{稪}{28947}
\saveTG{𦆓}{28947}
\saveTG{緮}{28947}
\saveTG{𦅮}{28948}
\saveTG{𣜥}{28948}
\saveTG{𦀅}{28950}
\saveTG{絴}{28951}
\saveTG{𦇫}{28951}
\saveTG{䋦}{28954}
\saveTG{𫄏}{28956}
\saveTG{𦆞}{28957}
\saveTG{𫄂}{28957}
\saveTG{𦁃}{28960}
\saveTG{秴}{28961}
\saveTG{繕}{28961}
\saveTG{給}{28961}
\saveTG{𦅝}{28962}
\saveTG{緧}{28964}
\saveTG{𦇃}{28964}
\saveTG{𦅩}{28966}
\saveTG{𥢶}{28966}
\saveTG{𥢥}{28966}
\saveTG{繪}{28966}
\saveTG{繒}{28966}
\saveTG{䅮}{28967}
\saveTG{𦃹}{28967}
\saveTG{𦅊}{28968}
\saveTG{綌}{28968}
\saveTG{䌣}{28977}
\saveTG{縦}{28981}
\saveTG{縱}{28981}
\saveTG{緃}{28981}
\saveTG{縼}{28981}
\saveTG{𥣕}{28981}
\saveTG{𦆀}{28982}
\saveTG{𥡬}{28982}
\saveTG{𦅗}{28982}
\saveTG{𦀢}{28982}
\saveTG{𦅆}{28984}
\saveTG{𥠦}{28984}
\saveTG{䌞}{28986}
\saveTG{𦄚}{28988}
\saveTG{𥿜}{28990}
\saveTG{𥛓}{28991}
\saveTG{䋡}{28994}
\saveTG{稌}{28994}
\saveTG{𫖦}{29027}
\saveTG{牉}{29050}
\saveTG{𤗷}{29057}
\saveTG{𤗾}{29066}
\saveTG{鍫}{29109}
\saveTG{皝}{29112}
\saveTG{绻}{29112}
\saveTG{𫚠}{29117}
\saveTG{继}{29119}
\saveTG{缈}{29120}
\saveTG{𫚌}{29120}
\saveTG{𡮓}{29120}
\saveTG{纱}{29120}
\saveTG{𪽄}{29127}
\saveTG{绡}{29127}
\saveTG{绱}{29127}
\saveTG{𡮶}{29127}
\saveTG{𧖍}{29127}
\saveTG{鹙}{29127}
\saveTG{鲿}{29132}
\saveTG{蝵}{29136}
\saveTG{𧗐}{29140}
\saveTG{缕}{29144}
\saveTG{衅}{29150}
\saveTG{绊}{29150}
\saveTG{𩲯}{29150}
\saveTG{鳞}{29159}
\saveTG{鳅}{29180}
\saveTG{仦}{29200}
\saveTG{觥}{29212}
\saveTG{侊}{29212}
\saveTG{倦}{29212}
\saveTG{傥}{29212}
\saveTG{𩲎}{29212}
\saveTG{𩳑}{29212}
\saveTG{𩴎}{29212}
\saveTG{𩲿}{29212}
\saveTG{𠐹}{29212}
\saveTG{㑽}{29214}
\saveTG{𠐓}{29214}
\saveTG{𪝲}{29214}
\saveTG{𩴠}{29215}
\saveTG{䨂}{29215}
\saveTG{𢔑}{29217}
\saveTG{𧡣}{29217}
\saveTG{𠌻}{29217}
\saveTG{𧣪}{29217}
\saveTG{𠎽}{29217}
\saveTG{𧈒}{29217}
\saveTG{㒌}{29217}
\saveTG{𢓥}{29217}
\saveTG{𠇵}{29217}
\saveTG{𩴩}{29218}
\saveTG{𧇸}{29218}
\saveTG{𠈱}{29220}
\saveTG{觘}{29220}
\saveTG{𡭥}{29220}
\saveTG{𪝹}{29220}
\saveTG{𠋝}{29220}
\saveTG{仯}{29220}
\saveTG{𢆽}{29220}
\saveTG{𩵃}{29223}
\saveTG{倘}{29227}
\saveTG{儯}{29227}
\saveTG{俏}{29227}
\saveTG{𩺀}{29227}
\saveTG{躺}{29227}
\saveTG{𧳍}{29227}
\saveTG{𠉅}{29227}
\saveTG{軂}{29227}
\saveTG{僗}{29227}
\saveTG{𢓮}{29227}
\saveTG{𧤙}{29227}
\saveTG{𧣥}{29227}
\saveTG{𨾻}{29227}
\saveTG{㡑}{29227}
\saveTG{𠉮}{29227}
\saveTG{徜}{29227}
\saveTG{𪘞}{29227}
\saveTG{儻}{29231}
\saveTG{偿}{29232}
\saveTG{僽}{29238}
\saveTG{𠌎}{29239}
\saveTG{𠌍}{29240}
\saveTG{𢖕}{29244}
\saveTG{𫎌}{29244}
\saveTG{𠈤}{29244}
\saveTG{𠊶}{29244}
\saveTG{偻}{29244}
\saveTG{㑞}{29244}
\saveTG{𠑄}{29247}
\saveTG{𠑊}{29248}
\saveTG{伴}{29250}
\saveTG{𡖱}{29250}
\saveTG{𪧻}{29250}
\saveTG{䚬}{29257}
\saveTG{𦟂}{29257}
\saveTG{𢕸}{29257}
\saveTG{𠌏}{29257}
\saveTG{僯}{29259}
\saveTG{𠋜}{29261}
\saveTG{偗}{29262}
\saveTG{儅}{29266}
\saveTG{𠉇}{29268}
\saveTG{𠎥}{29268}
\saveTG{𤳿}{29269}
\saveTG{偢}{29280}
\saveTG{伙}{29280}
\saveTG{𤏜}{29282}
\saveTG{𠈖}{29285}
\saveTG{𠈫}{29285}
\saveTG{𠑇}{29286}
\saveTG{償}{29286}
\saveTG{𪝧}{29289}
\saveTG{倓}{29289}
\saveTG{𧣹}{29289}
\saveTG{𩸥}{29289}
\saveTG{𤎥}{29289}
\saveTG{𤏕}{29289}
\saveTG{㒉}{29294}
\saveTG{儝}{29294}
\saveTG{𥽲}{29294}
\saveTG{𪤹}{29294}
\saveTG{侎}{29294}
\saveTG{𡮫}{29300}
\saveTG{𩵖}{29300}
\saveTG{𨘥}{29306}
\saveTG{䱧}{29317}
\saveTG{魦}{29320}
\saveTG{鯋}{29320}
\saveTG{𪃐}{29320}
\saveTG{𫙥}{29327}
\saveTG{𩸁}{29327}
\saveTG{䱵}{29327}
\saveTG{𠒸}{29327}
\saveTG{𩷈}{29327}
\saveTG{鮹}{29327}
\saveTG{䲏}{29327}
\saveTG{𩷼}{29327}
\saveTG{鶖}{29327}
\saveTG{𩽳}{29331}
\saveTG{𩹤}{29336}
\saveTG{𤋦}{29338}
\saveTG{𩼗}{29338}
\saveTG{愁}{29338}
\saveTG{𢝱}{29339}
\saveTG{𩺍}{29339}
\saveTG{𩶐}{29350}
\saveTG{𩻰}{29352}
\saveTG{𩼩}{29357}
\saveTG{鱗}{29359}
\saveTG{𩸛}{29360}
\saveTG{鱨}{29361}
\saveTG{𩼉}{29366}
\saveTG{𩼝}{29374}
\saveTG{鰍}{29380}
\saveTG{𩵰}{29380}
\saveTG{𩶻}{29380}
\saveTG{𩷎}{29380}
\saveTG{𩹳}{29386}
\saveTG{䱊}{29394}
\saveTG{𩺭}{29399}
\saveTG{𨈓}{29400}
\saveTG{𪨁}{29400}
\saveTG{媝}{29404}
\saveTG{𡥻}{29407}
\saveTG{𡮒}{29410}
\saveTG{𡮈}{29410}
\saveTG{𦩱}{29414}
\saveTG{𦩏}{29415}
\saveTG{𦨻}{29417}
\saveTG{𨈨}{29417}
\saveTG{𨉁}{29417}
\saveTG{𡮿}{29420}
\saveTG{𦨖}{29420}
\saveTG{𨈘}{29420}
\saveTG{艄}{29427}
\saveTG{𨈲}{29427}
\saveTG{𦩫}{29427}
\saveTG{𦫀}{29427}
\saveTG{𪅻}{29427}
\saveTG{𪟜}{29427}
\saveTG{𠢧}{29427}
\saveTG{𧑥}{29431}
\saveTG{𧔶}{29431}
\saveTG{䲍}{29436}
\saveTG{𡡏}{29442}
\saveTG{𦩩}{29444}
\saveTG{𦩎}{29444}
\saveTG{𨉠}{29450}
\saveTG{𦪦}{29452}
\saveTG{𨊌}{29457}
\saveTG{𦫁}{29461}
\saveTG{𨊏}{29461}
\saveTG{𤳮}{29464}
\saveTG{艡}{29466}
\saveTG{䒅}{29468}
\saveTG{𤳔}{29468}
\saveTG{𦫂}{29486}
\saveTG{𨉇}{29488}
\saveTG{𦩗}{29489}
\saveTG{𦪝}{29493}
\saveTG{𣚗}{29494}
\saveTG{𡦠}{29497}
\saveTG{𣽨}{29499}
\saveTG{揫}{29502}
\saveTG{犈}{29512}
\saveTG{魈}{29512}
\saveTG{𤛋}{29514}
\saveTG{𢵥}{29522}
\saveTG{𤙜}{29527}
\saveTG{𤛮}{29527}
\saveTG{𤙽}{29527}
\saveTG{㹖}{29532}
\saveTG{㹙}{29562}
\saveTG{㹚}{29566}
\saveTG{𤘭}{29580}
\saveTG{𪺯}{29589}
\saveTG{𤔇}{29597}
\saveTG{醔}{29604}
\saveTG{㿠}{29617}
\saveTG{䴛}{29627}
\saveTG{𩡈}{29627}
\saveTG{䭱}{29627}
\saveTG{㿩}{29631}
\saveTG{𪉧}{29689}
\saveTG{𤾃}{29689}
\saveTG{𢀆}{29689}
\saveTG{舕}{29689}
\saveTG{𤑀}{29689}
\saveTG{𪊅}{29691}
\saveTG{𩡅}{29694}
\saveTG{𫜇}{29694}
\saveTG{𡭧}{29700}
\saveTG{毜}{29710}
\saveTG{𣮅}{29712}
\saveTG{𣭱}{29712}
\saveTG{𣮜}{29712}
\saveTG{𪵚}{29713}
\saveTG{𣯢}{29714}
\saveTG{甃}{29717}
\saveTG{𡸩}{29717}
\saveTG{𩠉}{29717}
\saveTG{䵸}{29717}
\saveTG{𪵘}{29718}
\saveTG{毯}{29718}
\saveTG{𥽠}{29719}
\saveTG{毩}{29719}
\saveTG{𡭱}{29720}
\saveTG{𢆷}{29720}
\saveTG{嶗}{29727}
\saveTG{峭}{29727}
\saveTG{𡿓}{29731}
\saveTG{㠙}{29732}
\saveTG{𣬗}{29744}
\saveTG{𣬘}{29744}
\saveTG{嵝}{29744}
\saveTG{𡶡}{29750}
\saveTG{𩠏}{29752}
\saveTG{𡺹}{29752}
\saveTG{𡽤}{29757}
\saveTG{嶙}{29759}
\saveTG{𡽊}{29766}
\saveTG{巆}{29766}
\saveTG{𪹗}{29780}
\saveTG{𡵼}{29780}
\saveTG{𡺘}{29780}
\saveTG{燄}{29789}
\saveTG{𦦨}{29789}
\saveTG{嶸}{29794}
\saveTG{𡙀}{29804}
\saveTG{𥸸}{29804}
\saveTG{𨕼}{29804}
\saveTG{𨕪}{29808}
\saveTG{𩁄}{29815}
\saveTG{𡚂}{29827}
\saveTG{𠦺}{29850}
\saveTG{𤎢}{29889}
\saveTG{𥾗}{29900}
\saveTG{湬}{29902}
\saveTG{𣜨}{29912}
\saveTG{綣}{29912}
\saveTG{絖}{29912}
\saveTG{𦂙}{29914}
\saveTG{䅚}{29917}
\saveTG{𪐂}{29917}
\saveTG{継}{29919}
\saveTG{紗}{29920}
\saveTG{秒}{29920}
\saveTG{緲}{29920}
\saveTG{𦀛}{29920}
\saveTG{𥣀}{29922}
\saveTG{𥟎}{29927}
\saveTG{𠢺}{29927}
\saveTG{𥢒}{29927}
\saveTG{𣟽}{29927}
\saveTG{𪐆}{29927}
\saveTG{綃}{29927}
\saveTG{稍}{29927}
\saveTG{緔}{29927}
\saveTG{𥞻}{29927}
\saveTG{𦄏}{29927}
\saveTG{𥡎}{29927}
\saveTG{𦂎}{29931}
\saveTG{𫄗}{29931}
\saveTG{䋝}{29944}
\saveTG{𥤅}{29947}
\saveTG{𦁂}{29950}
\saveTG{秚}{29950}
\saveTG{絆}{29950}
\saveTG{繗}{29959}
\saveTG{㮐}{29962}
\saveTG{𦀭}{29968}
\saveTG{𫀮}{29977}
\saveTG{秋}{29980}
\saveTG{䋺}{29980}
\saveTG{𦃸}{29980}
\saveTG{䆏}{29981}
\saveTG{𥤄}{29985}
\saveTG{緂}{29989}
\saveTG{𥟢}{29989}
\saveTG{𥢻}{29991}
\saveTG{𦅌}{29993}
\saveTG{𦇝}{29993}
\saveTG{𦆱}{29994}
\saveTG{䋛}{29994}
\saveTG{𥞪}{29994}
\saveTG{丶}{30000}
\saveTG{𢜐}{30015}
\saveTG{𥤶}{30017}
\saveTG{穹}{30027}
\saveTG{宆}{30027}
\saveTG{弿}{30027}
\saveTG{𣁉}{30040}
\saveTG{𫁘}{30071}
\saveTG{㲿}{30100}
\saveTG{氵}{30100}
\saveTG{冫}{30100}
\saveTG{浐}{30101}
\saveTG{𡩞}{30101}
\saveTG{𡩬}{30101}
\saveTG{𥥐}{30102}
\saveTG{𥨠}{30102}
\saveTG{𥦳}{30102}
\saveTG{𥥊}{30102}
\saveTG{𥧧}{30102}
\saveTG{𥁗}{30102}
\saveTG{䆝}{30102}
\saveTG{䀄}{30102}
\saveTG{㝞}{30102}
\saveTG{㝉}{30102}
\saveTG{𡨆}{30102}
\saveTG{䀂}{30102}
\saveTG{𡪖}{30102}
\saveTG{𡧧}{30102}
\saveTG{㿾}{30102}
\saveTG{寍}{30102}
\saveTG{𡨈}{30102}
\saveTG{𥧘}{30102}
\saveTG{𥦰}{30102}
\saveTG{䆙}{30102}
\saveTG{𥦐}{30102}
\saveTG{𥥻}{30102}
\saveTG{𥩔}{30102}
\saveTG{𡧡}{30102}
\saveTG{𡧇}{30102}
\saveTG{宐}{30102}
\saveTG{宜}{30102}
\saveTG{空}{30102}
\saveTG{氵}{30103}
\saveTG{瑬}{30103}
\saveTG{宝}{30103}
\saveTG{𡩳}{30104}
\saveTG{㝧}{30104}
\saveTG{室}{30104}
\saveTG{塞}{30104}
\saveTG{窐}{30104}
\saveTG{𥦊}{30104}
\saveTG{𡨠}{30104}
\saveTG{𡨾}{30104}
\saveTG{宔}{30104}
\saveTG{𡏬}{30104}
\saveTG{𡧩}{30104}
\saveTG{𡫑}{30104}
\saveTG{𥥲}{30104}
\saveTG{𤪉}{30104}
\saveTG{窒}{30104}
\saveTG{𨓀}{30104}
\saveTG{𡩿}{30104}
\saveTG{𡋅}{30104}
\saveTG{𥧬}{30104}
\saveTG{𡫓}{30104}
\saveTG{𡔂}{30104}
\saveTG{𡫼}{30104}
\saveTG{𡪈}{30104}
\saveTG{𥤧}{30104}
\saveTG{𥧚}{30104}
\saveTG{䆹}{30105}
\saveTG{𡪂}{30105}
\saveTG{𡨵}{30106}
\saveTG{𡩦}{30106}
\saveTG{𡧹}{30106}
\saveTG{𡪕}{30106}
\saveTG{𡩉}{30106}
\saveTG{𥥣}{30106}
\saveTG{𡪏}{30106}
\saveTG{宣}{30106}
\saveTG{𥥷}{30108}
\saveTG{䆸}{30108}
\saveTG{𡪺}{30108}
\saveTG{𧯳}{30108}
\saveTG{𡫋}{30108}
\saveTG{𧯜}{30108}
\saveTG{寷}{30108}
\saveTG{䥌}{30109}
\saveTG{𡬔}{30109}
\saveTG{𡫿}{30109}
\saveTG{𡫷}{30109}
\saveTG{𨭧}{30109}
\saveTG{𥦨}{30109}
\saveTG{鎏}{30109}
\saveTG{汒}{30110}
\saveTG{灖}{30111}
\saveTG{𤅈}{30111}
\saveTG{𡨰}{30111}
\saveTG{𥦯}{30111}
\saveTG{𡬐}{30112}
\saveTG{䆾}{30112}
\saveTG{漉}{30112}
\saveTG{流}{30112}
\saveTG{滰}{30112}
\saveTG{竀}{30112}
\saveTG{𡧾}{30112}
\saveTG{𤅣}{30112}
\saveTG{𥨮}{30112}
\saveTG{𥩌}{30112}
\saveTG{窕}{30113}
\saveTG{宨}{30113}
\saveTG{𡬏}{30114}
\saveTG{𤃟}{30114}
\saveTG{𡬀}{30114}
\saveTG{𥦠}{30114}
\saveTG{𪸇}{30114}
\saveTG{𪧧}{30114}
\saveTG{𡑫}{30114}
\saveTG{注}{30114}
\saveTG{𪶶}{30114}
\saveTG{𣶂}{30114}
\saveTG{㴤}{30114}
\saveTG{窪}{30114}
\saveTG{瀍}{30114}
\saveTG{𪷲}{30114}
\saveTG{𤃎}{30114}
\saveTG{𠗤}{30114}
\saveTG{𤄛}{30115}
\saveTG{𨾾}{30115}
\saveTG{𣻰}{30115}
\saveTG{𣼲}{30115}
\saveTG{淮}{30115}
\saveTG{潍}{30115}
\saveTG{澭}{30115}
\saveTG{𣼭}{30115}
\saveTG{濰}{30115}
\saveTG{灉}{30115}
\saveTG{灕}{30115}
\saveTG{准}{30115}
\saveTG{𪷐}{30115}
\saveTG{𤄬}{30115}
\saveTG{灘}{30115}
\saveTG{𤄖}{30115}
\saveTG{滩}{30115}
\saveTG{潼}{30115}
\saveTG{滻}{30115}
\saveTG{𩀈}{30115}
\saveTG{濉}{30115}
\saveTG{滝}{30116}
\saveTG{𠘐}{30116}
\saveTG{澶}{30116}
\saveTG{灗}{30116}
\saveTG{渷}{30116}
\saveTG{𧡧}{30117}
\saveTG{𪧢}{30117}
\saveTG{𪚨}{30117}
\saveTG{𥧕}{30117}
\saveTG{㓍}{30117}
\saveTG{𣳥}{30117}
\saveTG{𡧪}{30117}
\saveTG{𣻼}{30117}
\saveTG{𤄞}{30117}
\saveTG{𤅀}{30117}
\saveTG{𤅗}{30117}
\saveTG{𣼠}{30117}
\saveTG{沆}{30117}
\saveTG{湸}{30117}
\saveTG{瀛}{30117}
\saveTG{灜}{30117}
\saveTG{𣽾}{30117}
\saveTG{𣳌}{30117}
\saveTG{𣶑}{30117}
\saveTG{㳘}{30117}
\saveTG{𣻕}{30117}
\saveTG{𣴹}{30117}
\saveTG{𡩮}{30117}
\saveTG{𣺄}{30118}
\saveTG{泣}{30118}
\saveTG{涖}{30118}
\saveTG{𪷙}{30119}
\saveTG{渟}{30121}
\saveTG{𣽱}{30121}
\saveTG{𪞨}{30121}
\saveTG{𣸥}{30122}
\saveTG{𡧥}{30122}
\saveTG{濟}{30123}
\saveTG{济}{30124}
\saveTG{済}{30124}
\saveTG{𣹿}{30127}
\saveTG{骞}{30127}
\saveTG{漓}{30127}
\saveTG{滂}{30127}
\saveTG{窍}{30127}
\saveTG{湾}{30127}
\saveTG{滽}{30127}
\saveTG{淯}{30127}
\saveTG{㵝}{30127}
\saveTG{汸}{30127}
\saveTG{滈}{30127}
\saveTG{涥}{30127}
\saveTG{滴}{30127}
\saveTG{𥥙}{30127}
\saveTG{渧}{30127}
\saveTG{窎}{30127}
\saveTG{𥧲}{30127}
\saveTG{㝍}{30127}
\saveTG{𥨙}{30127}
\saveTG{𣶢}{30127}
\saveTG{滳}{30127}
\saveTG{𣳍}{30127}
\saveTG{𣿱}{30127}
\saveTG{𤅘}{30127}
\saveTG{𠗵}{30127}
\saveTG{𣿕}{30127}
\saveTG{𡧁}{30127}
\saveTG{𡩻}{30127}
\saveTG{𥦅}{30127}
\saveTG{湇}{30127}
\saveTG{}{30127}
\saveTG{汴}{30130}
\saveTG{洂}{30130}
\saveTG{𧔂}{30131}
\saveTG{㶝}{30131}
\saveTG{㶐}{30131}
\saveTG{𧒡}{30131}
\saveTG{𧒧}{30131}
\saveTG{𧕝}{30131}
\saveTG{𧓫}{30131}
\saveTG{𣵅}{30131}
\saveTG{𧒟}{30131}
\saveTG{𥦷}{30131}
\saveTG{瀌}{30131}
\saveTG{潐}{30131}
\saveTG{𧔃}{30131}
\saveTG{𡬛}{30131}
\saveTG{瀤}{30132}
\saveTG{𠗪}{30132}
\saveTG{滾}{30132}
\saveTG{滚}{30132}
\saveTG{𣼟}{30132}
\saveTG{灋}{30132}
\saveTG{𣺓}{30132}
\saveTG{𪞯}{30132}
\saveTG{濠}{30132}
\saveTG{𠘠}{30132}
\saveTG{𪶦}{30132}
\saveTG{𤅭}{30132}
\saveTG{𤁷}{30132}
\saveTG{㳖}{30132}
\saveTG{𣺒}{30132}
\saveTG{𤅑}{30132}
\saveTG{𣿞}{30132}
\saveTG{滖}{30132}
\saveTG{瀼}{30132}
\saveTG{泫}{30132}
\saveTG{㵂}{30134}
\saveTG{𥥢}{30134}
\saveTG{䗙}{30136}
\saveTG{澺}{30136}
\saveTG{蜜}{30136}
\saveTG{𡩓}{30136}
\saveTG{𧉴}{30136}
\saveTG{䗕}{30136}
\saveTG{𧉌}{30136}
\saveTG{濂}{30137}
\saveTG{濓}{30137}
\saveTG{𡫞}{30137}
\saveTG{𥦛}{30138}
\saveTG{𣳝}{30140}
\saveTG{𪶺}{30140}
\saveTG{汶}{30140}
\saveTG{𪞙}{30140}
\saveTG{窏}{30141}
\saveTG{澼}{30141}
\saveTG{𤀫}{30141}
\saveTG{𪶹}{30141}
\saveTG{𤂸}{30141}
\saveTG{𣴔}{30141}
\saveTG{𣼧}{30143}
\saveTG{𣷳}{30143}
\saveTG{淁}{30144}
\saveTG{㴒}{30144}
\saveTG{㳰}{30144}
\saveTG{𥨓}{30144}
\saveTG{𣹾}{30146}
\saveTG{漳}{30146}
\saveTG{𪞬}{30146}
\saveTG{𪶧}{30147}
\saveTG{𣺦}{30147}
\saveTG{𡪵}{30147}
\saveTG{䆮}{30147}
\saveTG{㴑}{30147}
\saveTG{𣷷}{30147}
\saveTG{淳}{30147}
\saveTG{渡}{30147}
\saveTG{液}{30147}
\saveTG{寖}{30147}
\saveTG{寝}{30147}
\saveTG{𡫏}{30147}
\saveTG{𤂭}{30148}
\saveTG{𠗚}{30148}
\saveTG{𥩄}{30148}
\saveTG{𥨿}{30148}
\saveTG{洨}{30148}
\saveTG{淬}{30148}
\saveTG{䇁}{30151}
\saveTG{𣼀}{30152}
\saveTG{𣴿}{30152}
\saveTG{𠘚}{30152}
\saveTG{𤁜}{30153}
\saveTG{窢}{30153}
\saveTG{𣹫}{30156}
\saveTG{𪷸}{30157}
\saveTG{湆}{30161}
\saveTG{𣼯}{30161}
\saveTG{𣾾}{30161}
\saveTG{𠗥}{30161}
\saveTG{涪}{30161}
\saveTG{𣵧}{30161}
\saveTG{𣻃}{30161}
\saveTG{𣷀}{30162}
\saveTG{𣾪}{30162}
\saveTG{𪧓}{30162}
\saveTG{滀}{30163}
\saveTG{𣵰}{30164}
\saveTG{𡪦}{30164}
\saveTG{𣾆}{30164}
\saveTG{溏}{30165}
\saveTG{𠗶}{30165}
\saveTG{𣽢}{30165}
\saveTG{湻}{30166}
\saveTG{𡩏}{30167}
\saveTG{𤄟}{30168}
\saveTG{𣷧}{30169}
\saveTG{𡩧}{30172}
\saveTG{𪧕}{30174}
\saveTG{𡩝}{30177}
\saveTG{𢈂}{30177}
\saveTG{𥥨}{30177}
\saveTG{𥦔}{30182}
\saveTG{𣴃}{30182}
\saveTG{𡫈}{30184}
\saveTG{𠘘}{30184}
\saveTG{𣵉}{30184}
\saveTG{湙}{30184}
\saveTG{𣷄}{30185}
\saveTG{瀇}{30186}
\saveTG{𠘛}{30186}
\saveTG{𤎟}{30189}
\saveTG{䆱}{30189}
\saveTG{𡫚}{30191}
\saveTG{𡫻}{30191}
\saveTG{凛}{30191}
\saveTG{𪷤}{30191}
\saveTG{𪶸}{30193}
\saveTG{𤃰}{30193}
\saveTG{𣼆}{30194}
\saveTG{𣽘}{30194}
\saveTG{𤀮}{30194}
\saveTG{㳿}{30194}
\saveTG{𤃢}{30194}
\saveTG{𣼅}{30194}
\saveTG{𣵄}{30194}
\saveTG{澟}{30194}
\saveTG{𤃱}{30194}
\saveTG{𣾩}{30194}
\saveTG{潗}{30194}
\saveTG{凜}{30194}
\saveTG{𡪜}{30194}
\saveTG{𠗛}{30194}
\saveTG{𠘡}{30194}
\saveTG{滦}{30194}
\saveTG{涼}{30196}
\saveTG{凉}{30196}
\saveTG{凉}{30196}
\saveTG{𪸉}{30196}
\saveTG{漮}{30199}
\saveTG{礻}{30200}
\saveTG{𡧃}{30201}
\saveTG{𡨬}{30201}
\saveTG{𡨴}{30201}
\saveTG{宁}{30201}
\saveTG{寕}{30201}
\saveTG{寜}{30201}
\saveTG{寧}{30201}
\saveTG{𪧉}{30202}
\saveTG{寥}{30202}
\saveTG{礻}{30203}
\saveTG{衤}{30203}
\saveTG{㝋}{30207}
\saveTG{𥤣}{30207}
\saveTG{户}{30207}
\saveTG{穸}{30207}
\saveTG{𥚥}{30210}
\saveTG{寵}{30211}
\saveTG{𢨧}{30211}
\saveTG{窄}{30211}
\saveTG{宱}{30211}
\saveTG{竉}{30211}
\saveTG{扉}{30211}
\saveTG{𥥠}{30212}
\saveTG{𡩄}{30212}
\saveTG{𡧭}{30212}
\saveTG{𡩊}{30212}
\saveTG{㝟}{30212}
\saveTG{宪}{30212}
\saveTG{宛}{30212}
\saveTG{裗}{30212}
\saveTG{寛}{30212}
\saveTG{完}{30212}
\saveTG{宽}{30212}
\saveTG{宺}{30212}
\saveTG{𥨻}{30212}
\saveTG{𡫯}{30212}
\saveTG{𢩔}{30212}
\saveTG{㦾}{30212}
\saveTG{寬}{30213}
\saveTG{𢨫}{30214}
\saveTG{𥘭}{30214}
\saveTG{𡨱}{30214}
\saveTG{𥧊}{30214}
\saveTG{𥛭}{30214}
\saveTG{𡨥}{30214}
\saveTG{𢩏}{30214}
\saveTG{寇}{30214}
\saveTG{宼}{30214}
\saveTG{窛}{30214}
\saveTG{𧚮}{30215}
\saveTG{寉}{30215}
\saveTG{𧝸}{30215}
\saveTG{𩁔}{30215}
\saveTG{㝫}{30215}
\saveTG{䙰}{30215}
\saveTG{𪧳}{30215}
\saveTG{𧝎}{30215}
\saveTG{雇}{30215}
\saveTG{窿}{30215}
\saveTG{䄠}{30216}
\saveTG{𥧍}{30216}
\saveTG{襢}{30216}
\saveTG{𡨓}{30217}
\saveTG{𡩇}{30217}
\saveTG{𡩽}{30217}
\saveTG{𡨚}{30217}
\saveTG{𡧉}{30217}
\saveTG{𫆥}{30217}
\saveTG{𡪨}{30217}
\saveTG{𡪩}{30217}
\saveTG{𡬃}{30217}
\saveTG{𡧬}{30217}
\saveTG{𥦀}{30217}
\saveTG{𡩖}{30217}
\saveTG{𠿕}{30217}
\saveTG{宂}{30217}
\saveTG{扈}{30217}
\saveTG{戹}{30217}
\saveTG{𪧫}{30217}
\saveTG{䘪}{30217}
\saveTG{𠙖}{30217}
\saveTG{𥨥}{30217}
\saveTG{𥤤}{30217}
\saveTG{𡧜}{30217}
\saveTG{𥥕}{30217}
\saveTG{𢩊}{30217}
\saveTG{𫁖}{30217}
\saveTG{𥛞}{30217}
\saveTG{𡧽}{30217}
\saveTG{𥥱}{30217}
\saveTG{䆓}{30217}
\saveTG{𧚢}{30217}
\saveTG{𡬎}{30217}
\saveTG{𡪳}{30217}
\saveTG{𥘸}{30218}
\saveTG{𡨷}{30218}
\saveTG{𥦥}{30218}
\saveTG{𥦻}{30218}
\saveTG{𢨶}{30218}
\saveTG{窬}{30221}
\saveTG{亣}{30221}
\saveTG{𥥒}{30221}
\saveTG{𪧇}{30221}
\saveTG{𡬙}{30222}
\saveTG{𡧘}{30222}
\saveTG{𧞓}{30223}
\saveTG{𪧟}{30223}
\saveTG{䄢}{30223}
\saveTG{𥤬}{30224}
\saveTG{𥦹}{30224}
\saveTG{𥜭}{30226}
\saveTG{𦚑}{30227}
\saveTG{𢨿}{30227}
\saveTG{𡦺}{30227}
\saveTG{𡪝}{30227}
\saveTG{𡪷}{30227}
\saveTG{𡧝}{30227}
\saveTG{𡨅}{30227}
\saveTG{㝢}{30227}
\saveTG{䆷}{30227}
\saveTG{𥨷}{30227}
\saveTG{䘻}{30227}
\saveTG{𧜽}{30227}
\saveTG{䆜}{30227}
\saveTG{𥧱}{30227}
\saveTG{𥦢}{30227}
\saveTG{𪭙}{30227}
\saveTG{𢩌}{30227}
\saveTG{𢩃}{30227}
\saveTG{𪩶}{30227}
\saveTG{𦞫}{30227}
\saveTG{𢄏}{30227}
\saveTG{𡫛}{30227}
\saveTG{𥥏}{30227}
\saveTG{𥦁}{30227}
\saveTG{𥦜}{30227}
\saveTG{𥥃}{30227}
\saveTG{𥥝}{30227}
\saveTG{𥧆}{30227}
\saveTG{𥥾}{30227}
\saveTG{𥦮}{30227}
\saveTG{𥥚}{30227}
\saveTG{褯}{30227}
\saveTG{𥧀}{30227}
\saveTG{䆚}{30227}
\saveTG{𥨲}{30227}
\saveTG{𥤽}{30227}
\saveTG{𥨣}{30227}
\saveTG{𥥄}{30227}
\saveTG{𪧵}{30227}
\saveTG{祊}{30227}
\saveTG{扁}{30227}
\saveTG{寎}{30227}
\saveTG{窉}{30227}
\saveTG{㝑}{30227}
\saveTG{𥧮}{30227}
\saveTG{䙗}{30227}
\saveTG{𫋳}{30227}
\saveTG{寓}{30227}
\saveTG{宥}{30227}
\saveTG{扅}{30227}
\saveTG{宵}{30227}
\saveTG{窩}{30227}
\saveTG{窝}{30227}
\saveTG{寪}{30227}
\saveTG{褅}{30227}
\saveTG{扇}{30227}
\saveTG{窮}{30227}
\saveTG{寈}{30227}
\saveTG{甯}{30227}
\saveTG{褵}{30227}
\saveTG{寯}{30227}
\saveTG{扄}{30227}
\saveTG{扃}{30227}
\saveTG{肩}{30227}
\saveTG{鳸}{30227}
\saveTG{帍}{30227}
\saveTG{寡}{30227}
\saveTG{禞}{30227}
\saveTG{房}{30227}
\saveTG{禘}{30227}
\saveTG{帘}{30227}
\saveTG{㝦}{30227}
\saveTG{𥧈}{30227}
\saveTG{𥧃}{30227}
\saveTG{𡨦}{30227}
\saveTG{㝰}{30227}
\saveTG{㝱}{30227}
\saveTG{㝲}{30227}
\saveTG{𡨪}{30227}
\saveTG{𡩅}{30227}
\saveTG{𡧤}{30227}
\saveTG{𡧂}{30227}
\saveTG{𡧓}{30227}
\saveTG{𡧊}{30227}
\saveTG{𢐭}{30227}
\saveTG{𥙅}{30227}
\saveTG{䄜}{30227}
\saveTG{䄘}{30227}
\saveTG{𥛚}{30227}
\saveTG{㝏}{30228}
\saveTG{𥦂}{30228}
\saveTG{㝯}{30228}
\saveTG{𠕉}{30228}
\saveTG{𠕜}{30228}
\saveTG{𥧯}{30228}
\saveTG{𥥩}{30228}
\saveTG{𧙡}{30230}
\saveTG{𥨀}{30231}
\saveTG{𥤲}{30231}
\saveTG{𡨁}{30231}
\saveTG{㧁}{30231}
\saveTG{𢨱}{30231}
\saveTG{𥛪}{30231}
\saveTG{𥥈}{30231}
\saveTG{𧞧}{30231}
\saveTG{𥛲}{30231}
\saveTG{𡩙}{30232}
\saveTG{𧞷}{30232}
\saveTG{宸}{30232}
\saveTG{宖}{30232}
\saveTG{家}{30232}
\saveTG{䆣}{30232}
\saveTG{𡩀}{30232}
\saveTG{䆽}{30232}
\saveTG{䆥}{30232}
\saveTG{䙑}{30232}
\saveTG{𡩚}{30232}
\saveTG{𧟄}{30232}
\saveTG{𧞱}{30232}
\saveTG{𡩵}{30232}
\saveTG{𫀊}{30232}
\saveTG{𧙿}{30232}
\saveTG{𧟞}{30232}
\saveTG{禳}{30232}
\saveTG{窊}{30232}
\saveTG{袨}{30232}
\saveTG{扆}{30232}
\saveTG{窳}{30232}
\saveTG{寙}{30232}
\saveTG{𡬄}{30232}
\saveTG{𥨡}{30232}
\saveTG{𢩞}{30232}
\saveTG{䙛}{30232}
\saveTG{𡦾}{30232}
\saveTG{𧛅}{30232}
\saveTG{𧝘}{30232}
\saveTG{𥤼}{30232}
\saveTG{𧚕}{30233}
\saveTG{𥧰}{30233}
\saveTG{𧜥}{30234}
\saveTG{𥜇}{30236}
\saveTG{𧞋}{30237}
\saveTG{𡫐}{30237}
\saveTG{𢥜}{30238}
\saveTG{𢞝}{30238}
\saveTG{穿}{30241}
\saveTG{𨐳}{30241}
\saveTG{𧞃}{30241}
\saveTG{𢨮}{30242}
\saveTG{䘸}{30242}
\saveTG{𡧛}{30243}
\saveTG{𧜠}{30243}
\saveTG{𫁚}{30243}
\saveTG{𪧋}{30244}
\saveTG{𥨁}{30244}
\saveTG{𧚪}{30244}
\saveTG{𡫽}{30244}
\saveTG{𧛥}{30244}
\saveTG{𢩀}{30244}
\saveTG{𥥑}{30245}
\saveTG{𥧻}{30245}
\saveTG{𡫒}{30247}
\saveTG{𥦦}{30247}
\saveTG{𡪌}{30247}
\saveTG{𡫺}{30247}
\saveTG{𡬓}{30247}
\saveTG{㧀}{30247}
\saveTG{𥨊}{30247}
\saveTG{𥨝}{30247}
\saveTG{𥚠}{30247}
\saveTG{戽}{30247}
\saveTG{𡪢}{30247}
\saveTG{𧜰}{30247}
\saveTG{寑}{30247}
\saveTG{寢}{30247}
\saveTG{𥨍}{30247}
\saveTG{𡫪}{30247}
\saveTG{𥨦}{30247}
\saveTG{𡫠}{30248}
\saveTG{竅}{30248}
\saveTG{𪧜}{30248}
\saveTG{𥨅}{30248}
\saveTG{𡫀}{30248}
\saveTG{𥨶}{30248}
\saveTG{𡫣}{30248}
\saveTG{𪓌}{30248}
\saveTG{䘹}{30248}
\saveTG{𢨰}{30248}
\saveTG{䘨}{30248}
\saveTG{祽}{30248}
\saveTG{𥨆}{30248}
\saveTG{𥨄}{30248}
\saveTG{褲}{30250}
\saveTG{𥧋}{30251}
\saveTG{𡩍}{30253}
\saveTG{𥥰}{30253}
\saveTG{窚}{30253}
\saveTG{宬}{30253}
\saveTG{𡧿}{30253}
\saveTG{裤}{30254}
\saveTG{𥥤}{30257}
\saveTG{肁}{30257}
\saveTG{𪭚}{30258}
\saveTG{寣}{30261}
\saveTG{寤}{30261}
\saveTG{𡩩}{30261}
\saveTG{𥨉}{30261}
\saveTG{𡬑}{30261}
\saveTG{𡧻}{30261}
\saveTG{㝛}{30261}
\saveTG{窹}{30261}
\saveTG{𡬋}{30261}
\saveTG{𡩤}{30261}
\saveTG{𡪗}{30261}
\saveTG{𥚱}{30261}
\saveTG{𫀎}{30261}
\saveTG{𪪞}{30261}
\saveTG{扂}{30261}
\saveTG{𥦇}{30262}
\saveTG{𡫧}{30262}
\saveTG{𡬌}{30262}
\saveTG{𡪁}{30262}
\saveTG{𧝐}{30262}
\saveTG{宿}{30262}
\saveTG{𪧐}{30262}
\saveTG{𡫔}{30262}
\saveTG{䙒}{30263}
\saveTG{𪧒}{30264}
\saveTG{𡨢}{30264}
\saveTG{禟}{30265}
\saveTG{𥜮}{30267}
\saveTG{𥜢}{30267}
\saveTG{启}{30267}
\saveTG{㧂}{30267}
\saveTG{𡧒}{30271}
\saveTG{㝖}{30272}
\saveTG{𢩐}{30272}
\saveTG{窟}{30272}
\saveTG{𪧥}{30272}
\saveTG{𢨳}{30277}
\saveTG{𥥹}{30277}
\saveTG{𥥺}{30277}
\saveTG{𡨝}{30278}
\saveTG{𡨨}{30279}
\saveTG{𢩉}{30281}
\saveTG{𥛩}{30284}
\saveTG{𡬊}{30284}
\saveTG{戾}{30284}
\saveTG{𥦃}{30284}
\saveTG{戻}{30284}
\saveTG{𥦵}{30284}
\saveTG{𢩜}{30284}
\saveTG{𥚏}{30285}
\saveTG{𥧨}{30286}
\saveTG{𢩛}{30286}
\saveTG{𢩡}{30286}
\saveTG{賔}{30286}
\saveTG{𢨨}{30287}
\saveTG{扊}{30289}
\saveTG{𡪶}{30289}
\saveTG{𥩒}{30289}
\saveTG{𡬇}{30291}
\saveTG{䋀}{30293}
\saveTG{𡬒}{30294}
\saveTG{𧚦}{30294}
\saveTG{㝥}{30294}
\saveTG{𡪘}{30294}
\saveTG{㦿}{30294}
\saveTG{寱}{30294}
\saveTG{𥧽}{30294}
\saveTG{䆿}{30294}
\saveTG{窱}{30294}
\saveTG{襍}{30294}
\saveTG{𡩢}{30294}
\saveTG{𡪼}{30294}
\saveTG{𡬍}{30294}
\saveTG{寐}{30295}
\saveTG{𢉫}{30295}
\saveTG{𥧌}{30295}
\saveTG{𡬖}{30298}
\saveTG{䆲}{30299}
\saveTG{㝩}{30299}
\saveTG{辶}{30300}
\saveTG{迒}{30301}
\saveTG{邅}{30301}
\saveTG{進}{30301}
\saveTG{𨓞}{30301}
\saveTG{𨘿}{30301}
\saveTG{迬}{30301}
\saveTG{𨗴}{30301}
\saveTG{𨑑}{30301}
\saveTG{逳}{30302}
\saveTG{𨘆}{30302}
\saveTG{䆼}{30302}
\saveTG{𨘛}{30302}
\saveTG{窆}{30302}
\saveTG{䢍}{30302}
\saveTG{遃}{30302}
\saveTG{適}{30302}
\saveTG{遆}{30302}
\saveTG{𨗑}{30302}
\saveTG{䆳}{30302}
\saveTG{𫐼}{30302}
\saveTG{之}{30302}
\saveTG{迹}{30303}
\saveTG{遮}{30303}
\saveTG{寒}{30303}
\saveTG{𨖥}{30303}
\saveTG{𨖵}{30303}
\saveTG{𨓋}{30304}
\saveTG{𠿡}{30304}
\saveTG{䢒}{30304}
\saveTG{𨔊}{30304}
\saveTG{𨒳}{30304}
\saveTG{这}{30304}
\saveTG{遧}{30304}
\saveTG{䢦}{30304}
\saveTG{避}{30304}
\saveTG{𫑆}{30304}
\saveTG{𨖱}{30305}
\saveTG{𨘙}{30305}
\saveTG{𨔉}{30305}
\saveTG{𨓍}{30305}
\saveTG{𫑒}{30306}
\saveTG{𨖄}{30306}
\saveTG{𨘔}{30306}
\saveTG{𨕢}{30306}
\saveTG{𨗍}{30306}
\saveTG{這}{30306}
\saveTG{𨘃}{30307}
\saveTG{㝎}{30307}
\saveTG{𨕕}{30307}
\saveTG{𨒨}{30308}
\saveTG{𨕾}{30308}
\saveTG{𫑃}{30309}
\saveTG{𨓫}{30309}
\saveTG{宀}{30320}
\saveTG{𨓓}{30327}
\saveTG{𪀱}{30327}
\saveTG{𪃂}{30327}
\saveTG{𥥋}{30327}
\saveTG{鴪}{30327}
\saveTG{寫}{30327}
\saveTG{鶱}{30327}
\saveTG{騫}{30327}
\saveTG{窵}{30327}
\saveTG{𫛙}{30327}
\saveTG{𥧓}{30327}
\saveTG{𡫕}{30327}
\saveTG{𥩏}{30327}
\saveTG{𪂦}{30327}
\saveTG{𠁼}{30330}
\saveTG{𢤑}{30331}
\saveTG{𥧉}{30331}
\saveTG{𡫇}{30331}
\saveTG{𥧟}{30331}
\saveTG{𤌞}{30331}
\saveTG{𡧫}{30331}
\saveTG{𥧛}{30331}
\saveTG{窯}{30331}
\saveTG{惌}{30331}
\saveTG{䆶}{30331}
\saveTG{𤌊}{30331}
\saveTG{䵫}{30331}
\saveTG{𢜇}{30332}
\saveTG{𢘯}{30332}
\saveTG{𡪾}{30332}
\saveTG{窓}{30333}
\saveTG{寭}{30333}
\saveTG{𢤊}{30333}
\saveTG{𡬜}{30333}
\saveTG{𢛬}{30333}
\saveTG{𥧳}{30334}
\saveTG{𢞩}{30334}
\saveTG{𥨭}{30334}
\saveTG{宓}{30334}
\saveTG{𥦾}{30334}
\saveTG{𡪍}{30336}
\saveTG{𩸪}{30336}
\saveTG{𪬅}{30336}
\saveTG{愙}{30336}
\saveTG{憲}{30336}
\saveTG{窻}{30336}
\saveTG{𥧁}{30336}
\saveTG{悹}{30337}
\saveTG{𥥁}{30338}
\saveTG{䆫}{30338}
\saveTG{𥥬}{30338}
\saveTG{𢥛}{30338}
\saveTG{𪹿}{30338}
\saveTG{𥨞}{30338}
\saveTG{㥶}{30338}
\saveTG{𪧑}{30338}
\saveTG{𡪹}{30338}
\saveTG{𡩥}{30338}
\saveTG{𥦗}{30338}
\saveTG{𪬛}{30338}
\saveTG{𢚠}{30339}
\saveTG{窸}{30339}
\saveTG{守}{30342}
\saveTG{𨙖}{30343}
\saveTG{𡬮}{30347}
\saveTG{𡨙}{30348}
\saveTG{𡨎}{30348}
\saveTG{𥩎}{30356}
\saveTG{𥩑}{30356}
\saveTG{𨗁}{30362}
\saveTG{𡪎}{30379}
\saveTG{𡫊}{30401}
\saveTG{𥤱}{30401}
\saveTG{𥤿}{30401}
\saveTG{𥥯}{30401}
\saveTG{𥦭}{30401}
\saveTG{凖}{30401}
\saveTG{宇}{30401}
\saveTG{穻}{30401}
\saveTG{宰}{30401}
\saveTG{準}{30401}
\saveTG{䆤}{30401}
\saveTG{㝚}{30401}
\saveTG{𥥫}{30401}
\saveTG{𥥶}{30401}
\saveTG{𣁍}{30402}
\saveTG{𡧵}{30403}
\saveTG{宴}{30404}
\saveTG{𥥛}{30404}
\saveTG{寠}{30404}
\saveTG{𥤹}{30404}
\saveTG{𡪧}{30404}
\saveTG{𫁙}{30404}
\saveTG{安}{30404}
\saveTG{窭}{30404}
\saveTG{窶}{30404}
\saveTG{𡩌}{30405}
\saveTG{窧}{30406}
\saveTG{𥨎}{30406}
\saveTG{𥧭}{30406}
\saveTG{𥥞}{30407}
\saveTG{㝊}{30407}
\saveTG{𡩠}{30407}
\saveTG{𡧕}{30407}
\saveTG{𡕹}{30407}
\saveTG{𥧜}{30407}
\saveTG{𡨞}{30407}
\saveTG{𡧌}{30407}
\saveTG{㕠}{30407}
\saveTG{𡥜}{30407}
\saveTG{𡧔}{30407}
\saveTG{𡨊}{30407}
\saveTG{㝕}{30407}
\saveTG{字}{30407}
\saveTG{窙}{30407}
\saveTG{𥦘}{30407}
\saveTG{𥨈}{30407}
\saveTG{𥥜}{30407}
\saveTG{𥤪}{30407}
\saveTG{𥦸}{30407}
\saveTG{𡨯}{30407}
\saveTG{𥤦}{30407}
\saveTG{叜}{30407}
\saveTG{宯}{30407}
\saveTG{窔}{30408}
\saveTG{窣}{30408}
\saveTG{𡨧}{30408}
\saveTG{㝔}{30408}
\saveTG{𡧞}{30412}
\saveTG{寃}{30413}
\saveTG{宠}{30414}
\saveTG{𨾶}{30415}
\saveTG{𥨪}{30417}
\saveTG{𡧋}{30417}
\saveTG{𥨬}{30417}
\saveTG{宄}{30417}
\saveTG{究}{30417}
\saveTG{㝪}{30417}
\saveTG{𥦙}{30417}
\saveTG{𥧢}{30417}
\saveTG{𥨼}{30417}
\saveTG{𡨻}{30417}
\saveTG{𥥅}{30417}
\saveTG{𥤷}{30417}
\saveTG{𥦱}{30417}
\saveTG{𡪐}{30417}
\saveTG{𥦧}{30418}
\saveTG{𡫭}{30427}
\saveTG{𪧔}{30427}
\saveTG{𥨵}{30427}
\saveTG{𡧴}{30427}
\saveTG{𡪬}{30427}
\saveTG{𡧚}{30427}
\saveTG{𥦡}{30427}
\saveTG{𥧗}{30427}
\saveTG{𥨚}{30427}
\saveTG{𥩅}{30427}
\saveTG{窷}{30427}
\saveTG{穷}{30427}
\saveTG{𡪅}{30427}
\saveTG{𡪉}{30427}
\saveTG{𡫌}{30428}
\saveTG{䆖}{30431}
\saveTG{𡨜}{30434}
\saveTG{𥤴}{30441}
\saveTG{𥦌}{30441}
\saveTG{𡫉}{30441}
\saveTG{𡫳}{30441}
\saveTG{𡦿}{30442}
\saveTG{𡧅}{30442}
\saveTG{䆯}{30444}
\saveTG{𡪋}{30444}
\saveTG{𡪚}{30444}
\saveTG{𪧙}{30444}
\saveTG{𪧯}{30446}
\saveTG{𡫤}{30446}
\saveTG{㝤}{30447}
\saveTG{㝡}{30447}
\saveTG{窡}{30447}
\saveTG{𪧨}{30447}
\saveTG{𫁌}{30447}
\saveTG{𥧒}{30447}
\saveTG{𥥳}{30447}
\saveTG{寗}{30447}
\saveTG{𠂂}{30447}
\saveTG{𪧲}{30447}
\saveTG{𡨤}{30447}
\saveTG{𡩋}{30447}
\saveTG{𡫾}{30447}
\saveTG{𡫙}{30447}
\saveTG{𥧖}{30447}
\saveTG{𥤨}{30448}
\saveTG{𡪯}{30448}
\saveTG{𥨂}{30448}
\saveTG{𥨧}{30448}
\saveTG{䆻}{30448}
\saveTG{𥤰}{30448}
\saveTG{𪧤}{30448}
\saveTG{䆧}{30449}
\saveTG{窭}{30449}
\saveTG{𡪃}{30449}
\saveTG{𪧘}{30449}
\saveTG{𪧞}{30454}
\saveTG{𥨳}{30456}
\saveTG{𥧺}{30463}
\saveTG{竆}{30466}
\saveTG{𥩁}{30483}
\saveTG{𡫬}{30483}
\saveTG{𥨴}{30489}
\saveTG{𥥵}{30501}
\saveTG{搴}{30502}
\saveTG{牢}{30502}
\saveTG{𥤺}{30502}
\saveTG{𢳔}{30502}
\saveTG{𢮘}{30502}
\saveTG{𡧷}{30503}
\saveTG{𪧮}{30503}
\saveTG{𡨛}{30504}
\saveTG{𡪤}{30504}
\saveTG{鞌}{30506}
\saveTG{窜}{30506}
\saveTG{审}{30506}
\saveTG{𤛝}{30506}
\saveTG{𥥥}{30506}
\saveTG{䆔}{30506}
\saveTG{䆘}{30506}
\saveTG{𡨔}{30508}
\saveTG{𥦑}{30508}
\saveTG{窂}{30508}
\saveTG{𡨘}{30517}
\saveTG{䆒}{30517}
\saveTG{𥨛}{30527}
\saveTG{𥨱}{30527}
\saveTG{䆭}{30541}
\saveTG{䇀}{30542}
\saveTG{𡨣}{30547}
\saveTG{𥧿}{30547}
\saveTG{𥦆}{30547}
\saveTG{𥨗}{30551}
\saveTG{宑}{30552}
\saveTG{𥥼}{30555}
\saveTG{穽}{30558}
\saveTG{𪧊}{30562}
\saveTG{𫁗}{30572}
\saveTG{䆢}{30585}
\saveTG{𥧷}{30594}
\saveTG{𥦤}{30595}
\saveTG{𥨘}{30595}
\saveTG{𥩃}{30596}
\saveTG{𡧱}{30601}
\saveTG{𡨂}{30601}
\saveTG{𪧪}{30601}
\saveTG{𡨟}{30601}
\saveTG{謇}{30601}
\saveTG{𡧟}{30601}
\saveTG{𪧏}{30601}
\saveTG{㝜}{30601}
\saveTG{㝓}{30601}
\saveTG{𡫆}{30601}
\saveTG{𡩜}{30601}
\saveTG{𡩘}{30601}
\saveTG{𡩐}{30601}
\saveTG{𥥭}{30601}
\saveTG{𧦱}{30601}
\saveTG{窨}{30601}
\saveTG{窖}{30601}
\saveTG{㝘}{30601}
\saveTG{𡫍}{30602}
\saveTG{𩈰}{30602}
\saveTG{𩈱}{30602}
\saveTG{𥥔}{30602}
\saveTG{𥨌}{30602}
\saveTG{𪠹}{30602}
\saveTG{𡪒}{30602}
\saveTG{𥧥}{30602}
\saveTG{窗}{30602}
\saveTG{宮}{30602}
\saveTG{宕}{30602}
\saveTG{𡨽}{30603}
\saveTG{䁇}{30603}
\saveTG{𨢛}{30604}
\saveTG{𠼆}{30604}
\saveTG{𥥖}{30604}
\saveTG{客}{30604}
\saveTG{㝒}{30604}
\saveTG{𥧏}{30604}
\saveTG{𡧣}{30604}
\saveTG{𡨩}{30604}
\saveTG{𡩒}{30604}
\saveTG{𡧳}{30604}
\saveTG{𥧠}{30605}
\saveTG{宙}{30605}
\saveTG{𥥉}{30605}
\saveTG{害}{30605}
\saveTG{畗}{30606}
\saveTG{富}{30606}
\saveTG{宫}{30606}
\saveTG{宭}{30607}
\saveTG{窘}{30607}
\saveTG{容}{30608}
\saveTG{窅}{30608}
\saveTG{𥈽}{30608}
\saveTG{𠹟}{30608}
\saveTG{𡧼}{30608}
\saveTG{𡧶}{30608}
\saveTG{𪧗}{30608}
\saveTG{䆟}{30608}
\saveTG{𥦬}{30608}
\saveTG{𡨐}{30608}
\saveTG{𥧎}{30609}
\saveTG{䆺}{30609}
\saveTG{審}{30609}
\saveTG{𡩨}{30609}
\saveTG{㗟}{30612}
\saveTG{䆛}{30617}
\saveTG{𪎵}{30617}
\saveTG{寄}{30621}
\saveTG{𡫱}{30622}
\saveTG{𡫰}{30627}
\saveTG{𡫘}{30627}
\saveTG{𡪙}{30627}
\saveTG{𡩫}{30627}
\saveTG{𥥎}{30628}
\saveTG{䆗}{30628}
\saveTG{𫌻}{30632}
\saveTG{𥦫}{30633}
\saveTG{𫍓}{30637}
\saveTG{𥦖}{30643}
\saveTG{𡬚}{30664}
\saveTG{䆵}{30668}
\saveTG{𡪠}{30668}
\saveTG{𪧬}{30672}
\saveTG{讠}{30700}
\saveTG{𥤩}{30710}
\saveTG{𡦽}{30710}
\saveTG{𡬕}{30711}
\saveTG{𥥧}{30711}
\saveTG{𥦪}{30711}
\saveTG{𥤫}{30711}
\saveTG{寋}{30712}
\saveTG{𥦒}{30712}
\saveTG{䆞}{30712}
\saveTG{𡨇}{30712}
\saveTG{𡧮}{30712}
\saveTG{𡧏}{30712}
\saveTG{䆠}{30712}
\saveTG{它}{30712}
\saveTG{窤}{30712}
\saveTG{窇}{30712}
\saveTG{竄}{30712}
\saveTG{宦}{30712}
\saveTG{𡦻}{30712}
\saveTG{宧}{30712}
\saveTG{𡧀}{30712}
\saveTG{𥤭}{30712}
\saveTG{𡪆}{30713}
\saveTG{宅}{30714}
\saveTG{竁}{30714}
\saveTG{𥤥}{30714}
\saveTG{𡨉}{30714}
\saveTG{宒}{30714}
\saveTG{𡨗}{30714}
\saveTG{谁}{30715}
\saveTG{窀}{30715}
\saveTG{𥦣}{30715}
\saveTG{䆴}{30715}
\saveTG{𥥗}{30715}
\saveTG{竃}{30715}
\saveTG{𥦩}{30715}
\saveTG{𡩯}{30715}
\saveTG{𡪫}{30715}
\saveTG{竃}{30716}
\saveTG{䆰}{30716}
\saveTG{𡩾}{30716}
\saveTG{𡬁}{30717}
\saveTG{𥨔}{30717}
\saveTG{竈}{30717}
\saveTG{穵}{30717}
\saveTG{窀}{30717}
\saveTG{𥨫}{30717}
\saveTG{乼}{30717}
\saveTG{𥥟}{30717}
\saveTG{𥩋}{30717}
\saveTG{𡧗}{30717}
\saveTG{𡪣}{30717}
\saveTG{㐎}{30717}
\saveTG{𥤻}{30717}
\saveTG{窤}{30717}
\saveTG{𥩊}{30717}
\saveTG{𡪰}{30717}
\saveTG{𥧄}{30718}
\saveTG{𡨳}{30718}
\saveTG{𡩭}{30721}
\saveTG{𡩱}{30721}
\saveTG{谚}{30722}
\saveTG{谛}{30727}
\saveTG{谤}{30727}
\saveTG{𥥆}{30727}
\saveTG{𥧫}{30727}
\saveTG{𡧈}{30727}
\saveTG{𥤵}{30727}
\saveTG{𡧍}{30727}
\saveTG{𡪑}{30727}
\saveTG{𡧙}{30727}
\saveTG{𡧎}{30727}
\saveTG{谪}{30727}
\saveTG{窈}{30727}
\saveTG{窃}{30727}
\saveTG{窌}{30727}
\saveTG{访}{30727}
\saveTG{谯}{30731}
\saveTG{𡩑}{30732}
\saveTG{𡪭}{30732}
\saveTG{𡪊}{30732}
\saveTG{𡩲}{30732}
\saveTG{𡨲}{30732}
\saveTG{𥥴}{30732}
\saveTG{𧙶}{30732}
\saveTG{㝨}{30732}
\saveTG{褰}{30732}
\saveTG{䆡}{30732}
\saveTG{良}{30732}
\saveTG{寰}{30732}
\saveTG{宏}{30732}
\saveTG{㝗}{30732}
\saveTG{𡫖}{30732}
\saveTG{㝐}{30738}
\saveTG{谆}{30747}
\saveTG{𥦞}{30747}
\saveTG{𥦄}{30748}
\saveTG{谇}{30748}
\saveTG{谙}{30761}
\saveTG{窋}{30772}
\saveTG{𡨑}{30772}
\saveTG{𡧨}{30772}
\saveTG{𪧅}{30772}
\saveTG{𡧸}{30772}
\saveTG{密}{30772}
\saveTG{窰}{30772}
\saveTG{宻}{30772}
\saveTG{窑}{30772}
\saveTG{崈}{30772}
\saveTG{寚}{30772}
\saveTG{𡧰}{30772}
\saveTG{𡩴}{30774}
\saveTG{官}{30777}
\saveTG{窞}{30777}
\saveTG{𡩹}{30777}
\saveTG{𡧺}{30777}
\saveTG{𡺶}{30778}
\saveTG{𥧹}{30779}
\saveTG{该}{30782}
\saveTG{谅}{30796}
\saveTG{𥥀}{30800}
\saveTG{宍}{30800}
\saveTG{㝠}{30800}
\saveTG{𡪴}{30800}
\saveTG{寁}{30801}
\saveTG{㝙}{30801}
\saveTG{𡧆}{30801}
\saveTG{定}{30801}
\saveTG{蹇}{30801}
\saveTG{寘}{30801}
\saveTG{寔}{30801}
\saveTG{䞿}{30801}
\saveTG{窴}{30801}
\saveTG{𡧢}{30801}
\saveTG{𡩂}{30801}
\saveTG{𡨋}{30801}
\saveTG{𡩼}{30801}
\saveTG{𡧐}{30801}
\saveTG{𥦴}{30801}
\saveTG{𥦟}{30801}
\saveTG{𥩆}{30801}
\saveTG{𪧴}{30801}
\saveTG{𥥡}{30801}
\saveTG{𡨄}{30801}
\saveTG{𡨕}{30801}
\saveTG{𥧑}{30801}
\saveTG{宾}{30801}
\saveTG{赛}{30802}
\saveTG{𡦼}{30802}
\saveTG{𥦽}{30802}
\saveTG{穴}{30802}
\saveTG{𡫟}{30804}
\saveTG{㝣}{30804}
\saveTG{𡨒}{30804}
\saveTG{𡫃}{30804}
\saveTG{穾}{30804}
\saveTG{宎}{30804}
\saveTG{窫}{30804}
\saveTG{宊}{30804}
\saveTG{突}{30804}
\saveTG{实}{30804}
\saveTG{寏}{30804}
\saveTG{窦}{30804}
\saveTG{𥧤}{30804}
\saveTG{𥤮}{30804}
\saveTG{𡪪}{30804}
\saveTG{𥥍}{30804}
\saveTG{𡧦}{30804}
\saveTG{𡫩}{30804}
\saveTG{𡩔}{30804}
\saveTG{𥦏}{30804}
\saveTG{𡨮}{30804}
\saveTG{寞}{30804}
\saveTG{𥨩}{30804}
\saveTG{𥦺}{30804}
\saveTG{𡪿}{30804}
\saveTG{𡨡}{30804}
\saveTG{𡧑}{30804}
\saveTG{𡧠}{30804}
\saveTG{䆨}{30804}
\saveTG{𡨏}{30804}
\saveTG{䆕}{30805}
\saveTG{実}{30805}
\saveTG{𥥌}{30805}
\saveTG{𥥂}{30805}
\saveTG{寳}{30806}
\saveTG{竇}{30806}
\saveTG{𡪓}{30806}
\saveTG{𧶡}{30806}
\saveTG{𥦕}{30806}
\saveTG{𡪲}{30806}
\saveTG{䆬}{30806}
\saveTG{𧶼}{30806}
\saveTG{𥦎}{30806}
\saveTG{賨}{30806}
\saveTG{𥧡}{30806}
\saveTG{寶}{30806}
\saveTG{𧶎}{30806}
\saveTG{𧵒}{30806}
\saveTG{𧸹}{30806}
\saveTG{𧵕}{30806}
\saveTG{𡪛}{30806}
\saveTG{𥩐}{30806}
\saveTG{寊}{30806}
\saveTG{寅}{30806}
\saveTG{賓}{30806}
\saveTG{實}{30806}
\saveTG{賽}{30806}
\saveTG{𡩟}{30807}
\saveTG{㝌}{30807}
\saveTG{𪧆}{30807}
\saveTG{𥤯}{30807}
\saveTG{𡫮}{30808}
\saveTG{𡫜}{30808}
\saveTG{𥤳}{30808}
\saveTG{𥥦}{30809}
\saveTG{𥨤}{30809}
\saveTG{𤇵}{30809}
\saveTG{𡨿}{30809}
\saveTG{𥥿}{30809}
\saveTG{𤑖}{30809}
\saveTG{䆦}{30809}
\saveTG{灾}{30809}
\saveTG{𪧖}{30809}
\saveTG{𡨶}{30809}
\saveTG{𤍘}{30809}
\saveTG{𡨼}{30809}
\saveTG{𪧝}{30809}
\saveTG{窥}{30812}
\saveTG{窺}{30812}
\saveTG{𩀆}{30814}
\saveTG{䨈}{30815}
\saveTG{𥨖}{30817}
\saveTG{𥤸}{30817}
\saveTG{𡪽}{30817}
\saveTG{𡬂}{30826}
\saveTG{𥧂}{30828}
\saveTG{𥨐}{30828}
\saveTG{䆩}{30846}
\saveTG{𡫄}{30848}
\saveTG{𥨇}{30848}
\saveTG{𥧅}{30861}
\saveTG{𥧸}{30861}
\saveTG{𧵿}{30864}
\saveTG{寲}{30881}
\saveTG{𥨯}{30882}
\saveTG{窽}{30882}
\saveTG{𡪱}{30884}
\saveTG{𥧵}{30885}
\saveTG{𡬅}{30886}
\saveTG{𡫅}{30896}
\saveTG{𣘏}{30901}
\saveTG{𡨫}{30901}
\saveTG{𥥓}{30901}
\saveTG{察}{30901}
\saveTG{宗}{30901}
\saveTG{㲾}{30902}
\saveTG{永}{30902}
\saveTG{𦀀}{30903}
\saveTG{𡩡}{30903}
\saveTG{𥥮}{30903}
\saveTG{宷}{30904}
\saveTG{𡩣}{30904}
\saveTG{𫁏}{30904}
\saveTG{𡧯}{30904}
\saveTG{𡩁}{30904}
\saveTG{𡩎}{30904}
\saveTG{梥}{30904}
\saveTG{宋}{30904}
\saveTG{𪧌}{30904}
\saveTG{宩}{30904}
\saveTG{穼}{30904}
\saveTG{寨}{30904}
\saveTG{寀}{30904}
\saveTG{𣑄}{30904}
\saveTG{𥥸}{30904}
\saveTG{𡫵}{30904}
\saveTG{䅁}{30904}
\saveTG{𥥇}{30904}
\saveTG{𡪔}{30904}
\saveTG{𥨑}{30904}
\saveTG{𡩈}{30904}
\saveTG{𥦼}{30904}
\saveTG{㮤}{30904}
\saveTG{𣘿}{30904}
\saveTG{𡧖}{30904}
\saveTG{𣒕}{30904}
\saveTG{𡫗}{30904}
\saveTG{𥧣}{30904}
\saveTG{案}{30904}
\saveTG{宲}{30904}
\saveTG{窠}{30904}
\saveTG{窼}{30904}
\saveTG{𫁍}{30905}
\saveTG{𥥘}{30905}
\saveTG{𡩪}{30905}
\saveTG{𥦈}{30905}
\saveTG{𥩀}{30906}
\saveTG{寮}{30906}
\saveTG{𪧛}{30906}
\saveTG{竂}{30906}
\saveTG{𥨹}{30909}
\saveTG{寴}{30912}
\saveTG{𡩃}{30912}
\saveTG{㝭}{30915}
\saveTG{𥨕}{30915}
\saveTG{𥨰}{30916}
\saveTG{𡫝}{30916}
\saveTG{𥧔}{30917}
\saveTG{𥨟}{30917}
\saveTG{𥨾}{30917}
\saveTG{𡫁}{30917}
\saveTG{𪧭}{30921}
\saveTG{𡨖}{30922}
\saveTG{𥦋}{30927}
\saveTG{𥨸}{30927}
\saveTG{𥨒}{30927}
\saveTG{𡫂}{30927}
\saveTG{𥩓}{30927}
\saveTG{𡩸}{30927}
\saveTG{𡫢}{30927}
\saveTG{𥧩}{30927}
\saveTG{𥩈}{30927}
\saveTG{𥧼}{30927}
\saveTG{竊}{30927}
\saveTG{𥦉}{30928}
\saveTG{𡨭}{30931}
\saveTG{𥩇}{30935}
\saveTG{𥨽}{30938}
\saveTG{𪳙}{30941}
\saveTG{𡨀}{30941}
\saveTG{𥧇}{30943}
\saveTG{窲}{30943}
\saveTG{𡩕}{30946}
\saveTG{𥧞}{30947}
\saveTG{㝮}{30947}
\saveTG{𥨢}{30947}
\saveTG{寂}{30947}
\saveTG{𡫎}{30948}
\saveTG{𥩂}{30948}
\saveTG{窲}{30948}
\saveTG{𥨺}{30948}
\saveTG{𥦓}{30949}
\saveTG{𥥪}{30949}
\saveTG{𡩺}{30961}
\saveTG{𡪟}{30961}
\saveTG{𧶉}{30961}
\saveTG{𥦿}{30961}
\saveTG{𡬆}{30961}
\saveTG{𥧶}{30961}
\saveTG{𥧦}{30961}
\saveTG{𥧝}{30961}
\saveTG{窾}{30982}
\saveTG{𥧾}{30982}
\saveTG{𡪡}{30982}
\saveTG{𥦝}{30994}
\saveTG{𥥽}{30994}
\saveTG{𡨃}{30994}
\saveTG{㝝}{30994}
\saveTG{𥧙}{30995}
\saveTG{𪧦}{30998}
\saveTG{𥧴}{30999}
\saveTG{𢘟}{31026}
\saveTG{𥁡}{31102}
\saveTG{𥂳}{31102}
\saveTG{盓}{31102}
\saveTG{𡔆}{31104}
\saveTG{𡒖}{31104}
\saveTG{𡊺}{31104}
\saveTG{泸}{31107}
\saveTG{汇}{31110}
\saveTG{沚}{31110}
\saveTG{𣲚}{31110}
\saveTG{滙}{31111}
\saveTG{洭}{31111}
\saveTG{瀧}{31111}
\saveTG{澁}{31111}
\saveTG{𣾀}{31111}
\saveTG{𤃍}{31111}
\saveTG{𣶕}{31111}
\saveTG{𣿬}{31111}
\saveTG{渄}{31111}
\saveTG{𪞩}{31111}
\saveTG{瀝}{31111}
\saveTG{泟}{31111}
\saveTG{瀘}{31112}
\saveTG{𣵾}{31112}
\saveTG{沅}{31112}
\saveTG{灑}{31112}
\saveTG{涇}{31112}
\saveTG{沍}{31112}
\saveTG{冱}{31112}
\saveTG{渱}{31112}
\saveTG{冮}{31112}
\saveTG{洍}{31112}
\saveTG{溉}{31112}
\saveTG{𠘟}{31112}
\saveTG{𤁋}{31112}
\saveTG{𣻐}{31112}
\saveTG{𣲅}{31112}
\saveTG{𣻝}{31112}
\saveTG{𣲘}{31112}
\saveTG{𠗊}{31112}
\saveTG{𣻽}{31112}
\saveTG{𣦍}{31112}
\saveTG{𣳋}{31112}
\saveTG{漑}{31112}
\saveTG{江}{31112}
\saveTG{𣴥}{31114}
\saveTG{𣵱}{31114}
\saveTG{㳹}{31114}
\saveTG{𣳊}{31114}
\saveTG{𪞢}{31114}
\saveTG{𣶖}{31114}
\saveTG{𤄊}{31114}
\saveTG{𤄽}{31114}
\saveTG{𣳐}{31114}
\saveTG{𣻶}{31114}
\saveTG{𣷉}{31114}
\saveTG{𤁄}{31114}
\saveTG{𤃁}{31114}
\saveTG{汪}{31114}
\saveTG{沤}{31114}
\saveTG{涯}{31114}
\saveTG{湮}{31114}
\saveTG{凐}{31114}
\saveTG{洷}{31114}
\saveTG{溼}{31114}
\saveTG{𩁧}{31115}
\saveTG{𤀶}{31115}
\saveTG{𪶫}{31115}
\saveTG{瀖}{31115}
\saveTG{湹}{31115}
\saveTG{洹}{31116}
\saveTG{漚}{31116}
\saveTG{𠘌}{31116}
\saveTG{𤀌}{31116}
\saveTG{𪶥}{31116}
\saveTG{𧇦}{31117}
\saveTG{𣴑}{31117}
\saveTG{𤂢}{31117}
\saveTG{𣲍}{31117}
\saveTG{漑}{31117}
\saveTG{𣾳}{31117}
\saveTG{𣸰}{31117}
\saveTG{𫁔}{31117}
\saveTG{𪷢}{31117}
\saveTG{㶁}{31117}
\saveTG{𪞣}{31117}
\saveTG{𤁦}{31117}
\saveTG{㵡}{31117}
\saveTG{淲}{31117}
\saveTG{潖}{31117}
\saveTG{𣲁}{31117}
\saveTG{洰}{31117}
\saveTG{𣸝}{31117}
\saveTG{𤃫}{31117}
\saveTG{𣲪}{31117}
\saveTG{𤭈}{31117}
\saveTG{㳧}{31117}
\saveTG{𤂆}{31117}
\saveTG{𤅷}{31118}
\saveTG{浢}{31118}
\saveTG{𣳎}{31119}
\saveTG{河}{31120}
\saveTG{汀}{31120}
\saveTG{㓅}{31120}
\saveTG{𣽄}{31120}
\saveTG{涉}{31121}
\saveTG{𣶹}{31121}
\saveTG{滒}{31121}
\saveTG{𣽣}{31121}
\saveTG{𤁰}{31121}
\saveTG{𣻚}{31121}
\saveTG{洐}{31121}
\saveTG{𣾅}{31121}
\saveTG{𣲇}{31121}
\saveTG{𤀍}{31122}
\saveTG{𣵣}{31126}
\saveTG{𣶰}{31126}
\saveTG{𤅦}{31126}
\saveTG{濡}{31127}
\saveTG{馮}{31127}
\saveTG{滆}{31127}
\saveTG{沥}{31127}
\saveTG{濿}{31127}
\saveTG{溤}{31127}
\saveTG{濔}{31127}
\saveTG{瀰}{31127}
\saveTG{沔}{31127}
\saveTG{汅}{31127}
\saveTG{灊}{31127}
\saveTG{𪷳}{31127}
\saveTG{浉}{31127}
\saveTG{汚}{31127}
\saveTG{污}{31127}
\saveTG{漹}{31127}
\saveTG{㶚}{31127}
\saveTG{𩣠}{31127}
\saveTG{㴐}{31127}
\saveTG{𣲠}{31127}
\saveTG{𣽐}{31127}
\saveTG{𣷌}{31127}
\saveTG{𩤡}{31127}
\saveTG{溮}{31127}
\saveTG{𠗦}{31127}
\saveTG{𣽈}{31127}
\saveTG{𤂮}{31127}
\saveTG{𣵋}{31127}
\saveTG{𣴯}{31127}
\saveTG{沞}{31127}
\saveTG{𤂥}{31127}
\saveTG{灞}{31127}
\saveTG{𣶭}{31127}
\saveTG{}{31127}
\saveTG{𤅉}{31127}
\saveTG{漘}{31127}
\saveTG{洏}{31127}
\saveTG{𪷓}{31127}
\saveTG{𣾔}{31127}
\saveTG{𤅤}{31127}
\saveTG{𣾥}{31127}
\saveTG{𣽀}{31128}
\saveTG{渉}{31129}
\saveTG{𪵩}{31130}
\saveTG{澐}{31131}
\saveTG{𣾲}{31131}
\saveTG{𪶮}{31131}
\saveTG{𣽏}{31131}
\saveTG{𤂪}{31131}
\saveTG{滤}{31131}
\saveTG{𣾤}{31131}
\saveTG{𤂡}{31131}
\saveTG{𠖺}{31131}
\saveTG{𣳓}{31131}
\saveTG{𣲑}{31131}
\saveTG{𧕎}{31131}
\saveTG{𣽗}{31132}
\saveTG{𣵠}{31132}
\saveTG{沄}{31132}
\saveTG{涿}{31132}
\saveTG{𤅫}{31132}
\saveTG{𪞮}{31132}
\saveTG{浱}{31132}
\saveTG{澪}{31132}
\saveTG{涱}{31132}
\saveTG{澽}{31132}
\saveTG{漲}{31132}
\saveTG{𣻴}{31132}
\saveTG{𣼰}{31134}
\saveTG{濏}{31134}
\saveTG{𤄦}{31136}
\saveTG{濾}{31136}
\saveTG{𣸼}{31136}
\saveTG{𣸾}{31136}
\saveTG{𤅒}{31138}
\saveTG{添}{31138}
\saveTG{𤃉}{31139}
\saveTG{𪷕}{31140}
\saveTG{洱}{31140}
\saveTG{𣲐}{31140}
\saveTG{汗}{31140}
\saveTG{涆}{31140}
\saveTG{冴}{31140}
\saveTG{汙}{31140}
\saveTG{汧}{31140}
\saveTG{渳}{31140}
\saveTG{𣵼}{31140}
\saveTG{𣺕}{31140}
\saveTG{𪵺}{31141}
\saveTG{灄}{31141}
\saveTG{𣼻}{31141}
\saveTG{㳥}{31141}
\saveTG{𣶿}{31141}
\saveTG{𣳙}{31141}
\saveTG{𣲨}{31142}
\saveTG{溽}{31143}
\saveTG{𤁊}{31143}
\saveTG{𪷏}{31143}
\saveTG{㴗}{31144}
\saveTG{𣷑}{31144}
\saveTG{𣷮}{31144}
\saveTG{𣿺}{31144}
\saveTG{𤁡}{31146}
\saveTG{浭}{31146}
\saveTG{潭}{31146}
\saveTG{淖}{31146}
\saveTG{𪸀}{31147}
\saveTG{潊}{31147}
\saveTG{洅}{31147}
\saveTG{㴟}{31147}
\saveTG{滠}{31147}
\saveTG{𪞧}{31147}
\saveTG{瀀}{31147}
\saveTG{𤄑}{31147}
\saveTG{𣶮}{31147}
\saveTG{𣽠}{31147}
\saveTG{𢽦}{31147}
\saveTG{𣲏}{31147}
\saveTG{𪷋}{31148}
\saveTG{𣿋}{31149}
\saveTG{滹}{31149}
\saveTG{泙}{31149}
\saveTG{𣳶}{31149}
\saveTG{𤁎}{31149}
\saveTG{湃}{31150}
\saveTG{𠖱}{31150}
\saveTG{𣴱}{31152}
\saveTG{𣼸}{31153}
\saveTG{濊}{31153}
\saveTG{𤁁}{31154}
\saveTG{㴁}{31155}
\saveTG{𣸇}{31156}
\saveTG{𤃸}{31156}
\saveTG{𪶍}{31158}
\saveTG{𣾒}{31158}
\saveTG{沾}{31160}
\saveTG{滷}{31160}
\saveTG{𤁳}{31161}
\saveTG{灀}{31161}
\saveTG{潛}{31161}
\saveTG{溍}{31161}
\saveTG{𦥉}{31161}
\saveTG{浯}{31161}
\saveTG{𣾝}{31161}
\saveTG{𤃩}{31161}
\saveTG{㵢}{31161}
\saveTG{𤀃}{31161}
\saveTG{𣴨}{31162}
\saveTG{澑}{31162}
\saveTG{洦}{31162}
\saveTG{湎}{31162}
\saveTG{沰}{31162}
\saveTG{𤄐}{31162}
\saveTG{𤃹}{31162}
\saveTG{𤂬}{31162}
\saveTG{𠗍}{31162}
\saveTG{㓈}{31162}
\saveTG{𪶯}{31162}
\saveTG{𣷶}{31162}
\saveTG{𣶓}{31162}
\saveTG{𣵔}{31162}
\saveTG{滣}{31163}
\saveTG{𤃔}{31164}
\saveTG{洒}{31164}
\saveTG{酒}{31164}
\saveTG{𤅟}{31164}
\saveTG{湢}{31166}
\saveTG{𣽊}{31168}
\saveTG{𣿼}{31168}
\saveTG{濬}{31168}
\saveTG{𣿰}{31168}
\saveTG{𤃒}{31168}
\saveTG{𣵞}{31168}
\saveTG{㳪}{31169}
\saveTG{𤃝}{31169}
\saveTG{𤁧}{31172}
\saveTG{𣶩}{31172}
\saveTG{𣵍}{31177}
\saveTG{㳁}{31180}
\saveTG{𣴸}{31181}
\saveTG{渋}{31181}
\saveTG{漇}{31181}
\saveTG{㵐}{31182}
\saveTG{𫖯}{31182}
\saveTG{𣵕}{31182}
\saveTG{𣶵}{31182}
\saveTG{𣲓}{31182}
\saveTG{滪}{31182}
\saveTG{灏}{31182}
\saveTG{浈}{31182}
\saveTG{濒}{31182}
\saveTG{𣾱}{31183}
\saveTG{渜}{31184}
\saveTG{𣴇}{31184}
\saveTG{澞}{31184}
\saveTG{𤂌}{31186}
\saveTG{𤁪}{31186}
\saveTG{𣽥}{31186}
\saveTG{𤃣}{31186}
\saveTG{𤅎}{31186}
\saveTG{㶊}{31186}
\saveTG{𠗸}{31186}
\saveTG{𥨨}{31186}
\saveTG{𤂯}{31186}
\saveTG{𣹟}{31186}
\saveTG{湞}{31186}
\saveTG{澦}{31186}
\saveTG{灦}{31186}
\saveTG{瀬}{31186}
\saveTG{湏}{31186}
\saveTG{澒}{31186}
\saveTG{灝}{31186}
\saveTG{頫}{31186}
\saveTG{瀩}{31186}
\saveTG{瀕}{31186}
\saveTG{𤂓}{31186}
\saveTG{𣿦}{31186}
\saveTG{𩓧}{31186}
\saveTG{𣻳}{31186}
\saveTG{𤁫}{31186}
\saveTG{𤄰}{31186}
\saveTG{𤅡}{31186}
\saveTG{㴿}{31186}
\saveTG{𤅆}{31186}
\saveTG{𪷿}{31186}
\saveTG{𤄹}{31186}
\saveTG{𤁾}{31186}
\saveTG{𤃑}{31186}
\saveTG{𤅓}{31186}
\saveTG{𤀪}{31186}
\saveTG{𤃌}{31186}
\saveTG{𤄭}{31188}
\saveTG{𠗇}{31189}
\saveTG{𣻔}{31189}
\saveTG{𣺗}{31189}
\saveTG{㳅}{31190}
\saveTG{漂}{31191}
\saveTG{沶}{31191}
\saveTG{潥}{31194}
\saveTG{𣵹}{31194}
\saveTG{𤂈}{31194}
\saveTG{𪷩}{31194}
\saveTG{𠘍}{31194}
\saveTG{溧}{31194}
\saveTG{凓}{31194}
\saveTG{瀮}{31194}
\saveTG{𣼝}{31194}
\saveTG{源}{31196}
\saveTG{𣸸}{31199}
\saveTG{祉}{31210}
\saveTG{裶}{31211}
\saveTG{襱}{31211}
\saveTG{䙣}{31211}
\saveTG{𣃱}{31211}
\saveTG{𧘍}{31212}
\saveTG{𥘡}{31212}
\saveTG{𧘲}{31212}
\saveTG{𧘕}{31212}
\saveTG{𧘛}{31212}
\saveTG{𧘿}{31212}
\saveTG{𠖈}{31212}
\saveTG{襹}{31212}
\saveTG{𧞿}{31212}
\saveTG{𧞁}{31212}
\saveTG{𥙠}{31212}
\saveTG{𥛜}{31212}
\saveTG{𧙼}{31212}
\saveTG{𥛳}{31212}
\saveTG{𪓐}{31212}
\saveTG{𧘢}{31212}
\saveTG{𧙸}{31212}
\saveTG{𧝔}{31212}
\saveTG{𧟙}{31212}
\saveTG{𥘺}{31212}
\saveTG{䄥}{31212}
\saveTG{褗}{31214}
\saveTG{𧛑}{31214}
\saveTG{𥙛}{31214}
\saveTG{𥚅}{31214}
\saveTG{䘭}{31214}
\saveTG{𡪮}{31214}
\saveTG{𥘛}{31214}
\saveTG{禋}{31214}
\saveTG{祬}{31214}
\saveTG{𥛡}{31214}
\saveTG{𫋲}{31214}
\saveTG{𧛘}{31214}
\saveTG{𥛉}{31215}
\saveTG{𧙅}{31216}
\saveTG{䙔}{31216}
\saveTG{𧝿}{31216}
\saveTG{𧟂}{31216}
\saveTG{𥘹}{31217}
\saveTG{裭}{31217}
\saveTG{褼}{31217}
\saveTG{凴}{31217}
\saveTG{𧜳}{31217}
\saveTG{𠙥}{31217}
\saveTG{𥜰}{31217}
\saveTG{𪭘}{31217}
\saveTG{𤬽}{31217}
\saveTG{𡨍}{31217}
\saveTG{𥚚}{31217}
\saveTG{𥘳}{31217}
\saveTG{甂}{31217}
\saveTG{裋}{31218}
\saveTG{䄈}{31218}
\saveTG{𥘻}{31219}
\saveTG{袔}{31220}
\saveTG{𥘕}{31221}
\saveTG{裄}{31221}
\saveTG{𥙺}{31222}
\saveTG{𥘫}{31226}
\saveTG{𫆬}{31227}
\saveTG{𢄘}{31227}
\saveTG{𧚊}{31227}
\saveTG{𥜗}{31227}
\saveTG{𧞝}{31227}
\saveTG{𧘚}{31227}
\saveTG{𥙷}{31227}
\saveTG{𦜹}{31227}
\saveTG{袻}{31227}
\saveTG{褃}{31227}
\saveTG{禲}{31227}
\saveTG{襧}{31227}
\saveTG{𧞵}{31227}
\saveTG{騗}{31227}
\saveTG{裲}{31227}
\saveTG{禰}{31227}
\saveTG{襦}{31227}
\saveTG{禡}{31227}
\saveTG{𫀌}{31227}
\saveTG{𧜗}{31227}
\saveTG{䙐}{31227}
\saveTG{𥜲}{31227}
\saveTG{𧜂}{31227}
\saveTG{𡬈}{31230}
\saveTG{𧘔}{31230}
\saveTG{𧟐}{31231}
\saveTG{𧟚}{31231}
\saveTG{𪝴}{31231}
\saveTG{裃}{31231}
\saveTG{𥘟}{31231}
\saveTG{𠖤}{31231}
\saveTG{𥘢}{31231}
\saveTG{裖}{31232}
\saveTG{𧛇}{31232}
\saveTG{祳}{31232}
\saveTG{𤂝}{31232}
\saveTG{𥙮}{31232}
\saveTG{𧘾}{31232}
\saveTG{䙥}{31232}
\saveTG{㵗}{31232}
\saveTG{𥜅}{31232}
\saveTG{𧝲}{31232}
\saveTG{𥜜}{31236}
\saveTG{𥘏}{31240}
\saveTG{衦}{31240}
\saveTG{衧}{31240}
\saveTG{𧙧}{31240}
\saveTG{襵}{31241}
\saveTG{𧘪}{31242}
\saveTG{𥙟}{31242}
\saveTG{𧙫}{31242}
\saveTG{䢇}{31243}
\saveTG{𥛑}{31243}
\saveTG{𡫦}{31243}
\saveTG{䢆}{31243}
\saveTG{褥}{31243}
\saveTG{𧙒}{31244}
\saveTG{𧚂}{31244}
\saveTG{𪧩}{31244}
\saveTG{䙅}{31244}
\saveTG{𥨋}{31244}
\saveTG{𥙴}{31245}
\saveTG{𫀑}{31246}
\saveTG{𦠰}{31246}
\saveTG{𧝓}{31246}
\saveTG{禫}{31246}
\saveTG{𢼐}{31247}
\saveTG{𢼀}{31247}
\saveTG{㪐}{31247}
\saveTG{𧛓}{31247}
\saveTG{𥜚}{31247}
\saveTG{𫌇}{31247}
\saveTG{𥚳}{31247}
\saveTG{𥘴}{31249}
\saveTG{袩}{31260}
\saveTG{袹}{31262}
\saveTG{祏}{31262}
\saveTG{袥}{31262}
\saveTG{𥛽}{31262}
\saveTG{𧣮}{31264}
\saveTG{𥩉}{31264}
\saveTG{𥙘}{31264}
\saveTG{𥙫}{31264}
\saveTG{福}{31266}
\saveTG{褔}{31266}
\saveTG{𧝆}{31267}
\saveTG{𫀛}{31272}
\saveTG{褷}{31281}
\saveTG{𥛺}{31281}
\saveTG{䙠}{31282}
\saveTG{祯}{31282}
\saveTG{𥛨}{31282}
\saveTG{𧚺}{31282}
\saveTG{𫀅}{31282}
\saveTG{𫖶}{31282}
\saveTG{𥜒}{31284}
\saveTG{𧞣}{31284}
\saveTG{䙇}{31284}
\saveTG{祆}{31284}
\saveTG{禷}{31286}
\saveTG{顧}{31286}
\saveTG{𥜝}{31286}
\saveTG{禎}{31286}
\saveTG{𥚴}{31286}
\saveTG{𩕥}{31286}
\saveTG{𩕄}{31286}
\saveTG{𩕳}{31286}
\saveTG{𩕋}{31286}
\saveTG{𩔧}{31286}
\saveTG{𥛌}{31286}
\saveTG{𧜙}{31286}
\saveTG{𩑜}{31286}
\saveTG{襭}{31286}
\saveTG{顅}{31286}
\saveTG{禵}{31288}
\saveTG{}{31288}
\saveTG{}{31288}
\saveTG{褾}{31291}
\saveTG{𥛦}{31291}
\saveTG{𠘔}{31292}
\saveTG{辷}{31300}
\saveTG{𨑗}{31301}
\saveTG{逗}{31301}
\saveTG{𨕿}{31301}
\saveTG{𨖢}{31301}
\saveTG{𫑄}{31301}
\saveTG{𨒑}{31301}
\saveTG{𨓷}{31301}
\saveTG{𨕸}{31301}
\saveTG{𨖹}{31301}
\saveTG{𨒬}{31301}
\saveTG{𨑭}{31301}
\saveTG{䢝}{31301}
\saveTG{逕}{31301}
\saveTG{迋}{31301}
\saveTG{逛}{31301}
\saveTG{𨓿}{31301}
\saveTG{𨓖}{31301}
\saveTG{𨑐}{31301}
\saveTG{𨑵}{31301}
\saveTG{𨔛}{31301}
\saveTG{𨘶}{31301}
\saveTG{𨖆}{31301}
\saveTG{𨒌}{31301}
\saveTG{邐}{31301}
\saveTG{远}{31301}
\saveTG{遷}{31301}
\saveTG{𨒎}{31301}
\saveTG{𨙕}{31301}
\saveTG{𨓊}{31301}
\saveTG{𨘸}{31301}
\saveTG{𨓃}{31302}
\saveTG{𨕆}{31302}
\saveTG{𨖜}{31302}
\saveTG{𨒩}{31302}
\saveTG{䢋}{31302}
\saveTG{𨑜}{31302}
\saveTG{𨔑}{31302}
\saveTG{迃}{31302}
\saveTG{遤}{31302}
\saveTG{迈}{31302}
\saveTG{逦}{31302}
\saveTG{迊}{31302}
\saveTG{𨙔}{31302}
\saveTG{𨑰}{31302}
\saveTG{𨓘}{31302}
\saveTG{𨔖}{31302}
\saveTG{𨕑}{31302}
\saveTG{𨙦}{31302}
\saveTG{𨖸}{31302}
\saveTG{邇}{31302}
\saveTG{𨖝}{31303}
\saveTG{𨑼}{31303}
\saveTG{运}{31303}
\saveTG{遽}{31303}
\saveTG{逐}{31303}
\saveTG{遯}{31303}
\saveTG{𨗠}{31303}
\saveTG{𨘈}{31304}
\saveTG{𨙓}{31304}
\saveTG{𨖛}{31304}
\saveTG{𨕯}{31304}
\saveTG{䢎}{31304}
\saveTG{𨕞}{31304}
\saveTG{𨘹}{31304}
\saveTG{迂}{31304}
\saveTG{迓}{31304}
\saveTG{迀}{31304}
\saveTG{逴}{31304}
\saveTG{𨙉}{31304}
\saveTG{𨙆}{31304}
\saveTG{𨙤}{31304}
\saveTG{䢚}{31305}
\saveTG{䢥}{31306}
\saveTG{𨔟}{31306}
\saveTG{逌}{31306}
\saveTG{逜}{31306}
\saveTG{逎}{31306}
\saveTG{𨒹}{31306}
\saveTG{迺}{31306}
\saveTG{迠}{31306}
\saveTG{逼}{31306}
\saveTG{𨕂}{31306}
\saveTG{𨒙}{31306}
\saveTG{𨖻}{31306}
\saveTG{𨓗}{31306}
\saveTG{𨔁}{31306}
\saveTG{𨕐}{31306}
\saveTG{遉}{31308}
\saveTG{迗}{31308}
\saveTG{𨘀}{31308}
\saveTG{𨘷}{31308}
\saveTG{𨔞}{31308}
\saveTG{𫑖}{31308}
\saveTG{𨗶}{31308}
\saveTG{𨔀}{31308}
\saveTG{还}{31309}
\saveTG{𫐾}{31309}
\saveTG{𣷻}{31311}
\saveTG{𪶾}{31312}
\saveTG{𣵨}{31312}
\saveTG{𪶬}{31312}
\saveTG{𣶪}{31314}
\saveTG{𨕅}{31314}
\saveTG{𣹐}{31314}
\saveTG{𣲳}{31316}
\saveTG{𨕹}{31316}
\saveTG{𪷝}{31317}
\saveTG{𢗈}{31320}
\saveTG{𪵭}{31327}
\saveTG{𪁘}{31327}
\saveTG{憑}{31332}
\saveTG{𨗢}{31332}
\saveTG{𨗇}{31332}
\saveTG{慿}{31332}
\saveTG{𤊁}{31336}
\saveTG{𤑏}{31336}
\saveTG{惉}{31336}
\saveTG{𢙁}{31337}
\saveTG{𣶾}{31338}
\saveTG{𢢵}{31339}
\saveTG{𡫶}{31344}
\saveTG{𨔰}{31344}
\saveTG{𨑈}{31347}
\saveTG{𨒿}{31364}
\saveTG{𣿤}{31367}
\saveTG{𤀹}{31368}
\saveTG{𨗆}{31372}
\saveTG{𩔦}{31386}
\saveTG{𩕜}{31386}
\saveTG{𩔢}{31386}
\saveTG{𨘍}{31393}
\saveTG{𣿚}{31394}
\saveTG{𣂇}{31411}
\saveTG{𦪾}{31412}
\saveTG{𡪸}{31415}
\saveTG{𣁿}{31417}
\saveTG{𣁯}{31417}
\saveTG{𣯌}{31417}
\saveTG{𩢌}{31427}
\saveTG{𡝝}{31441}
\saveTG{𡜂}{31442}
\saveTG{𡜡}{31444}
\saveTG{𡝫}{31446}
\saveTG{𢾐}{31447}
\saveTG{𡩛}{31462}
\saveTG{𣁹}{31464}
\saveTG{𩑯}{31486}
\saveTG{頞}{31486}
\saveTG{顐}{31586}
\saveTG{𧮝}{31601}
\saveTG{㼸}{31617}
\saveTG{𩤩}{31627}
\saveTG{㪡}{31647}
\saveTG{𢾒}{31647}
\saveTG{𢾮}{31647}
\saveTG{𢾏}{31647}
\saveTG{𪢓}{31664}
\saveTG{𩕤}{31668}
\saveTG{额}{31682}
\saveTG{額}{31686}
\saveTG{𩔜}{31686}
\saveTG{𫌶}{31690}
\saveTG{让}{31710}
\saveTG{诽}{31711}
\saveTG{诓}{31711}
\saveTG{证}{31711}
\saveTG{讧}{31712}
\saveTG{㐢}{31714}
\saveTG{诳}{31714}
\saveTG{谑}{31714}
\saveTG{讴}{31714}
\saveTG{}{31714}
\saveTG{𤭊}{31717}
\saveTG{㼠}{31717}
\saveTG{𤭒}{31717}
\saveTG{讵}{31717}
\saveTG{𫍹}{31717}
\saveTG{诬}{31718}
\saveTG{订}{31720}
\saveTG{诃}{31720}
\saveTG{𫍨}{31732}
\saveTG{诼}{31732}
\saveTG{讶}{31740}
\saveTG{讦}{31740}
\saveTG{}{31740}
\saveTG{讶}{31742}
\saveTG{𦫍}{31744}
\saveTG{谭}{31746}
\saveTG{评}{31749}
\saveTG{𧮪}{31760}
\saveTG{语}{31761}
\saveTG{谮}{31761}
\saveTG{𩒂}{31786}
\saveTG{𩓘}{31786}
\saveTG{𤑐}{31809}
\saveTG{𫁜}{31827}
\saveTG{䫤}{31886}
\saveTG{𩕽}{31886}
\saveTG{顮}{31886}
\saveTG{顁}{31886}
\saveTG{渠}{31904}
\saveTG{𦀞}{31912}
\saveTG{𣸺}{31946}
\saveTG{𩕟}{31986}
\saveTG{䫅}{31986}
\saveTG{𠛂}{32000}
\saveTG{𠛡}{32000}
\saveTG{州}{32000}
\saveTG{𢗮}{32070}
\saveTG{淛}{32100}
\saveTG{丬}{32100}
\saveTG{𪵰}{32100}
\saveTG{𪶃}{32100}
\saveTG{𠝳}{32100}
\saveTG{𣸠}{32100}
\saveTG{𣶄}{32100}
\saveTG{𣺤}{32100}
\saveTG{𣸟}{32100}
\saveTG{𣷨}{32100}
\saveTG{𣵃}{32100}
\saveTG{㴊}{32100}
\saveTG{𣿖}{32100}
\saveTG{𠛪}{32100}
\saveTG{㶜}{32100}
\saveTG{𣺳}{32100}
\saveTG{洲}{32100}
\saveTG{𣸛}{32100}
\saveTG{𤂄}{32100}
\saveTG{𤄉}{32100}
\saveTG{𣹨}{32100}
\saveTG{𪟅}{32100}
\saveTG{浰}{32100}
\saveTG{𤁴}{32100}
\saveTG{渕}{32100}
\saveTG{淵}{32100}
\saveTG{渊}{32100}
\saveTG{业}{32100}
\saveTG{涮}{32100}
\saveTG{瀏}{32100}
\saveTG{浏}{32100}
\saveTG{洌}{32100}
\saveTG{冽}{32100}
\saveTG{溂}{32100}
\saveTG{汌}{32100}
\saveTG{測}{32100}
\saveTG{测}{32100}
\saveTG{𠗢}{32100}
\saveTG{𠗧}{32100}
\saveTG{𠗗}{32100}
\saveTG{𠗜}{32100}
\saveTG{𣴈}{32100}
\saveTG{𥁼}{32102}
\saveTG{𥁩}{32102}
\saveTG{𥁷}{32102}
\saveTG{𠘦}{32102}
\saveTG{𤄎}{32103}
\saveTG{壍}{32104}
\saveTG{𪣡}{32104}
\saveTG{𡎒}{32104}
\saveTG{𡋈}{32104}
\saveTG{垽}{32104}
\saveTG{𨮜}{32109}
\saveTG{鋈}{32109}
\saveTG{沘}{32110}
\saveTG{𣶟}{32110}
\saveTG{泚}{32110}
\saveTG{湚}{32110}
\saveTG{滮}{32112}
\saveTG{㳋}{32112}
\saveTG{𤁺}{32112}
\saveTG{潂}{32112}
\saveTG{潉}{32112}
\saveTG{漞}{32112}
\saveTG{洮}{32113}
\saveTG{兆}{32113}
\saveTG{溌}{32114}
\saveTG{淫}{32114}
\saveTG{漄}{32114}
\saveTG{汑}{32114}
\saveTG{𠗝}{32114}
\saveTG{𥙲}{32114}
\saveTG{𪶁}{32114}
\saveTG{𣳴}{32114}
\saveTG{㳝}{32114}
\saveTG{𣷪}{32114}
\saveTG{}{32114}
\saveTG{凗}{32115}
\saveTG{𣺡}{32115}
\saveTG{涶}{32115}
\saveTG{湩}{32115}
\saveTG{漼}{32115}
\saveTG{𣵭}{32115}
\saveTG{𣳫}{32115}
\saveTG{𤁯}{32116}
\saveTG{𣴊}{32117}
\saveTG{𣲣}{32117}
\saveTG{𣺣}{32117}
\saveTG{𣴋}{32117}
\saveTG{㴩}{32117}
\saveTG{𣲞}{32117}
\saveTG{㳶}{32117}
\saveTG{㴲}{32117}
\saveTG{㳐}{32117}
\saveTG{𠖬}{32117}
\saveTG{㴰}{32117}
\saveTG{𣹵}{32117}
\saveTG{𣵘}{32117}
\saveTG{𤀖}{32117}
\saveTG{𣳸}{32117}
\saveTG{𡭰}{32117}
\saveTG{𣴜}{32117}
\saveTG{𣲭}{32117}
\saveTG{𣾽}{32117}
\saveTG{澄}{32118}
\saveTG{𤀆}{32118}
\saveTG{灃}{32118}
\saveTG{溰}{32118}
\saveTG{凒}{32118}
\saveTG{𤅐}{32118}
\saveTG{𪶌}{32120}
\saveTG{𤃮}{32121}
\saveTG{𤂊}{32121}
\saveTG{淅}{32121}
\saveTG{澵}{32121}
\saveTG{沂}{32121}
\saveTG{渐}{32121}
\saveTG{浙}{32121}
\saveTG{澌}{32121}
\saveTG{漸}{32121}
\saveTG{凘}{32121}
\saveTG{𣂩}{32121}
\saveTG{𣷲}{32121}
\saveTG{𪷀}{32121}
\saveTG{浵}{32122}
\saveTG{𤁀}{32122}
\saveTG{𣲀}{32122}
\saveTG{涁}{32122}
\saveTG{澎}{32122}
\saveTG{𣹊}{32122}
\saveTG{𣿿}{32123}
\saveTG{𪵯}{32123}
\saveTG{𣽵}{32124}
\saveTG{𣵙}{32127}
\saveTG{𣾡}{32127}
\saveTG{𣴖}{32127}
\saveTG{𣶁}{32127}
\saveTG{𠗬}{32127}
\saveTG{灣}{32127}
\saveTG{𣷔}{32127}
\saveTG{𣲸}{32127}
\saveTG{𣶇}{32127}
\saveTG{𣾷}{32127}
\saveTG{㳨}{32127}
\saveTG{𣺥}{32127}
\saveTG{𠘋}{32127}
\saveTG{湍}{32127}
\saveTG{漰}{32127}
\saveTG{沜}{32127}
\saveTG{潙}{32127}
\saveTG{渪}{32127}
\saveTG{濎}{32127}
\saveTG{涔}{32127}
\saveTG{𣵛}{32127}
\saveTG{𣹄}{32127}
\saveTG{𣼍}{32127}
\saveTG{㳢}{32128}
\saveTG{𪷌}{32130}
\saveTG{𣲖}{32130}
\saveTG{𪶭}{32130}
\saveTG{𣲙}{32130}
\saveTG{𣹭}{32130}
\saveTG{沠}{32130}
\saveTG{泓}{32130}
\saveTG{泒}{32130}
\saveTG{𠗹}{32130}
\saveTG{𣳯}{32131}
\saveTG{𣸢}{32131}
\saveTG{𣷹}{32132}
\saveTG{𩇨}{32132}
\saveTG{派}{32132}
\saveTG{𣾜}{32132}
\saveTG{泛}{32132}
\saveTG{𤂖}{32132}
\saveTG{𣳛}{32132}
\saveTG{𣷯}{32132}
\saveTG{𪷟}{32132}
\saveTG{𤂜}{32132}
\saveTG{𤃛}{32133}
\saveTG{涨}{32134}
\saveTG{滍}{32136}
\saveTG{𫊫}{32136}
\saveTG{濦}{32137}
\saveTG{㶏}{32137}
\saveTG{㴽}{32139}
\saveTG{𣿜}{32139}
\saveTG{汦}{32140}
\saveTG{汘}{32140}
\saveTG{泜}{32140}
\saveTG{渆}{32140}
\saveTG{𪶱}{32141}
\saveTG{𪶐}{32141}
\saveTG{𣸖}{32141}
\saveTG{涏}{32141}
\saveTG{泝}{32141}
\saveTG{涎}{32141}
\saveTG{𣶉}{32143}
\saveTG{𠘣}{32143}
\saveTG{𤅪}{32143}
\saveTG{𤃊}{32144}
\saveTG{𤃠}{32144}
\saveTG{浽}{32144}
\saveTG{涹}{32144}
\saveTG{𤀾}{32144}
\saveTG{𤃙}{32145}
\saveTG{灂}{32146}
\saveTG{灇}{32147}
\saveTG{浮}{32147}
\saveTG{鼗}{32147}
\saveTG{湲}{32147}
\saveTG{鼝}{32147}
\saveTG{𣳒}{32147}
\saveTG{㳵}{32147}
\saveTG{𣸍}{32147}
\saveTG{𣾧}{32147}
\saveTG{𤀣}{32147}
\saveTG{𣶧}{32147}
\saveTG{𤄪}{32147}
\saveTG{𤅥}{32147}
\saveTG{潑}{32147}
\saveTG{汳}{32147}
\saveTG{𣳭}{32147}
\saveTG{涭}{32147}
\saveTG{𤅶}{32148}
\saveTG{浖}{32149}
\saveTG{泘}{32149}
\saveTG{𣲬}{32150}
\saveTG{𪷘}{32152}
\saveTG{𣿄}{32152}
\saveTG{𤁑}{32153}
\saveTG{溄}{32154}
\saveTG{𣺿}{32154}
\saveTG{凈}{32157}
\saveTG{淨}{32157}
\saveTG{洉}{32161}
\saveTG{𣿉}{32161}
\saveTG{𣶫}{32161}
\saveTG{㴯}{32161}
\saveTG{湝}{32162}
\saveTG{𣷰}{32162}
\saveTG{𣾂}{32162}
\saveTG{𤀦}{32162}
\saveTG{𣻂}{32162}
\saveTG{湽}{32163}
\saveTG{淄}{32163}
\saveTG{𣴠}{32164}
\saveTG{𪶞}{32164}
\saveTG{𣽅}{32164}
\saveTG{𣸅}{32164}
\saveTG{涽}{32164}
\saveTG{活}{32164}
\saveTG{湉}{32164}
\saveTG{𣽰}{32164}
\saveTG{㵌}{32164}
\saveTG{㓉}{32164}
\saveTG{𣿛}{32164}
\saveTG{𣸩}{32164}
\saveTG{𣾨}{32167}
\saveTG{𣽪}{32167}
\saveTG{㴞}{32169}
\saveTG{𪶴}{32169}
\saveTG{𤃃}{32169}
\saveTG{𤄜}{32169}
\saveTG{潘}{32169}
\saveTG{涾}{32169}
\saveTG{㴡}{32169}
\saveTG{𣳈}{32170}
\saveTG{汕}{32170}
\saveTG{汹}{32170}
\saveTG{𪶕}{32172}
\saveTG{𣺧}{32172}
\saveTG{㶌}{32172}
\saveTG{𣳺}{32172}
\saveTG{𣽮}{32172}
\saveTG{泏}{32172}
\saveTG{滛}{32172}
\saveTG{𣵜}{32172}
\saveTG{㴈}{32172}
\saveTG{滔}{32177}
\saveTG{㴙}{32177}
\saveTG{𤄅}{32177}
\saveTG{浜}{32181}
\saveTG{㓇}{32184}
\saveTG{𣴌}{32184}
\saveTG{渓}{32184}
\saveTG{溪}{32184}
\saveTG{沃}{32184}
\saveTG{湀}{32184}
\saveTG{𣺍}{32184}
\saveTG{𤂛}{32185}
\saveTG{濮}{32185}
\saveTG{𣾴}{32185}
\saveTG{𠘖}{32186}
\saveTG{𣴽}{32189}
\saveTG{湠}{32189}
\saveTG{冰}{32190}
\saveTG{𣸔}{32191}
\saveTG{溗}{32191}
\saveTG{漴}{32191}
\saveTG{渁}{32192}
\saveTG{𣻆}{32193}
\saveTG{𤀇}{32193}
\saveTG{𤅇}{32193}
\saveTG{𣴍}{32193}
\saveTG{𪸅}{32193}
\saveTG{𪷜}{32193}
\saveTG{𤄏}{32193}
\saveTG{𣶸}{32194}
\saveTG{灤}{32194}
\saveTG{𣹌}{32194}
\saveTG{𣼖}{32194}
\saveTG{𣲲}{32194}
\saveTG{𤄿}{32194}
\saveTG{𠘙}{32194}
\saveTG{𤀻}{32194}
\saveTG{𣶶}{32194}
\saveTG{𣻬}{32194}
\saveTG{𠗡}{32194}
\saveTG{漅}{32194}
\saveTG{濼}{32194}
\saveTG{泺}{32194}
\saveTG{澲}{32195}
\saveTG{𣵚}{32198}
\saveTG{潻}{32199}
\saveTG{𡌋}{32199}
\saveTG{剜}{32200}
\saveTG{𠞛}{32200}
\saveTG{𥘊}{32200}
\saveTG{䄗}{32200}
\saveTG{𥚲}{32200}
\saveTG{𠟶}{32200}
\saveTG{𧙮}{32200}
\saveTG{𧝼}{32200}
\saveTG{𧙷}{32200}
\saveTG{𧚳}{32200}
\saveTG{裫}{32200}
\saveTG{礼}{32210}
\saveTG{𠜍}{32212}
\saveTG{𫆽}{32212}
\saveTG{𫀖}{32212}
\saveTG{𧜺}{32212}
\saveTG{𥙂}{32212}
\saveTG{𥘇}{32212}
\saveTG{褦}{32212}
\saveTG{𠖒}{32213}
\saveTG{䙍}{32213}
\saveTG{祧}{32213}
\saveTG{袵}{32214}
\saveTG{衽}{32214}
\saveTG{祍}{32214}
\saveTG{褈}{32215}
\saveTG{𧞪}{32216}
\saveTG{𥚇}{32216}
\saveTG{𫀢}{32216}
\saveTG{䃾}{32217}
\saveTG{𨓢}{32217}
\saveTG{𧘐}{32217}
\saveTG{禠}{32217}
\saveTG{褫}{32217}
\saveTG{𧜇}{32217}
\saveTG{㲢}{32217}
\saveTG{㲰}{32217}
\saveTG{𣭺}{32217}
\saveTG{𪎤}{32217}
\saveTG{𫋺}{32217}
\saveTG{䘣}{32217}
\saveTG{𧘱}{32217}
\saveTG{𥛟}{32217}
\saveTG{𥘍}{32217}
\saveTG{𥜨}{32218}
\saveTG{䙞}{32218}
\saveTG{祈}{32221}
\saveTG{𧝤}{32221}
\saveTG{𧘻}{32221}
\saveTG{𥛻}{32222}
\saveTG{𢒎}{32222}
\saveTG{𥘎}{32222}
\saveTG{𢒋}{32222}
\saveTG{衫}{32222}
\saveTG{䄔}{32227}
\saveTG{禙}{32227}
\saveTG{𥙾}{32227}
\saveTG{𧚘}{32227}
\saveTG{𥚹}{32227}
\saveTG{褙}{32227}
\saveTG{𧚸}{32227}
\saveTG{䙖}{32227}
\saveTG{𧜑}{32227}
\saveTG{𧚜}{32227}
\saveTG{𧚓}{32227}
\saveTG{𢄁}{32227}
\saveTG{𥚻}{32227}
\saveTG{𧞖}{32227}
\saveTG{𧝉}{32227}
\saveTG{𧟃}{32227}
\saveTG{𣾙}{32227}
\saveTG{𦢳}{32227}
\saveTG{𥛊}{32227}
\saveTG{𪓎}{32227}
\saveTG{黹}{32227}
\saveTG{脊}{32227}
\saveTG{褍}{32227}
\saveTG{𤔚}{32230}
\saveTG{𧘷}{32230}
\saveTG{𧞞}{32231}
\saveTG{䙧}{32231}
\saveTG{䝉}{32232}
\saveTG{𥙎}{32232}
\saveTG{𧛎}{32232}
\saveTG{𣲈}{32232}
\saveTG{𧞑}{32232}
\saveTG{㼐}{32233}
\saveTG{𧙆}{32233}
\saveTG{𧞎}{32237}
\saveTG{祗}{32240}
\saveTG{𧘜}{32240}
\saveTG{祇}{32240}
\saveTG{衹}{32240}
\saveTG{袛}{32240}
\saveTG{䘰}{32241}
\saveTG{𧚭}{32241}
\saveTG{𧙝}{32241}
\saveTG{𧜋}{32244}
\saveTG{𫌂}{32244}
\saveTG{𣱆}{32247}
\saveTG{𧞇}{32247}
\saveTG{𫌘}{32247}
\saveTG{褑}{32247}
\saveTG{𥚀}{32247}
\saveTG{禐}{32247}
\saveTG{襏}{32247}
\saveTG{𧚯}{32247}
\saveTG{𥚾}{32247}
\saveTG{𧛝}{32248}
\saveTG{𧟗}{32249}
\saveTG{禨}{32253}
\saveTG{𧝞}{32258}
\saveTG{𥙐}{32261}
\saveTG{𧙺}{32261}
\saveTG{𧚻}{32262}
\saveTG{𥚺}{32262}
\saveTG{𧛆}{32262}
\saveTG{䄕}{32262}
\saveTG{𥚉}{32263}
\saveTG{䙉}{32264}
\saveTG{𧝩}{32264}
\saveTG{䄑}{32264}
\saveTG{𥙱}{32264}
\saveTG{䄆}{32264}
\saveTG{𧚎}{32265}
\saveTG{襎}{32269}
\saveTG{𥛮}{32269}
\saveTG{䚦}{32272}
\saveTG{𥙋}{32272}
\saveTG{袦}{32272}
\saveTG{𥛃}{32277}
\saveTG{䙄}{32277}
\saveTG{𥛗}{32277}
\saveTG{𥜥}{32281}
\saveTG{𥛕}{32281}
\saveTG{䙫}{32281}
\saveTG{𫌃}{32282}
\saveTG{祅}{32284}
\saveTG{彂}{32284}
\saveTG{袄}{32284}
\saveTG{䙎}{32284}
\saveTG{䙆}{32284}
\saveTG{襥}{32285}
\saveTG{襆}{32285}
\saveTG{𧤫}{32287}
\saveTG{𥛢}{32291}
\saveTG{𫋼}{32293}
\saveTG{𧚃}{32293}
\saveTG{𥚖}{32294}
\saveTG{𫀔}{32294}
\saveTG{迾}{32300}
\saveTG{𨓹}{32301}
\saveTG{𨕗}{32301}
\saveTG{𨔠}{32301}
\saveTG{𨓒}{32301}
\saveTG{𨓮}{32301}
\saveTG{𨔝}{32301}
\saveTG{𨒤}{32301}
\saveTG{𨔅}{32301}
\saveTG{𨑺}{32301}
\saveTG{𫐿}{32301}
\saveTG{𨓱}{32301}
\saveTG{𨖭}{32301}
\saveTG{𨕰}{32301}
\saveTG{遞}{32301}
\saveTG{邋}{32301}
\saveTG{邆}{32301}
\saveTG{逃}{32301}
\saveTG{𨑳}{32301}
\saveTG{𨓏}{32301}
\saveTG{𨒯}{32301}
\saveTG{透}{32302}
\saveTG{逝}{32302}
\saveTG{近}{32302}
\saveTG{逓}{32302}
\saveTG{𨓛}{32302}
\saveTG{𨕘}{32302}
\saveTG{𨕋}{32302}
\saveTG{䢪}{32302}
\saveTG{𨖠}{32302}
\saveTG{𨓂}{32302}
\saveTG{𨔲}{32302}
\saveTG{遄}{32302}
\saveTG{𨘼}{32302}
\saveTG{𨖳}{32302}
\saveTG{𨒂}{32302}
\saveTG{𨑎}{32302}
\saveTG{𨖿}{32302}
\saveTG{𨖮}{32302}
\saveTG{𨑠}{32302}
\saveTG{𨖡}{32302}
\saveTG{𨔜}{32302}
\saveTG{𨔭}{32302}
\saveTG{𨔈}{32302}
\saveTG{𨓯}{32302}
\saveTG{𨑝}{32302}
\saveTG{𨑾}{32303}
\saveTG{𨔌}{32303}
\saveTG{巡}{32303}
\saveTG{𨖣}{32303}
\saveTG{𨔢}{32303}
\saveTG{𨖀}{32303}
\saveTG{𨖼}{32303}
\saveTG{𨒢}{32303}
\saveTG{𫑌}{32303}
\saveTG{𨘓}{32303}
\saveTG{𨙏}{32304}
\saveTG{䢑}{32304}
\saveTG{𨖦}{32304}
\saveTG{𨕧}{32304}
\saveTG{𨗙}{32304}
\saveTG{𨒰}{32304}
\saveTG{逬}{32304}
\saveTG{返}{32304}
\saveTG{逶}{32304}
\saveTG{迁}{32304}
\saveTG{𨔩}{32304}
\saveTG{𨒽}{32304}
\saveTG{𨓎}{32304}
\saveTG{𫐭}{32305}
\saveTG{𨑲}{32305}
\saveTG{䢬}{32305}
\saveTG{𨘱}{32305}
\saveTG{𨖐}{32305}
\saveTG{𨗂}{32305}
\saveTG{𨕱}{32305}
\saveTG{𨗞}{32306}
\saveTG{𨕥}{32306}
\saveTG{遁}{32306}
\saveTG{𨓬}{32306}
\saveTG{逅}{32306}
\saveTG{适}{32306}
\saveTG{𨒞}{32307}
\saveTG{𨒥}{32307}
\saveTG{𨔨}{32307}
\saveTG{遥}{32307}
\saveTG{辿}{32307}
\saveTG{𨓌}{32308}
\saveTG{𨙑}{32308}
\saveTG{遜}{32309}
\saveTG{𨘁}{32309}
\saveTG{𨖾}{32309}
\saveTG{邎}{32309}
\saveTG{𨗩}{32309}
\saveTG{𨙂}{32309}
\saveTG{𨘺}{32309}
\saveTG{𩙹}{32313}
\saveTG{𪁕}{32327}
\saveTG{𨙍}{32327}
\saveTG{𩷯}{32336}
\saveTG{𡭎}{32343}
\saveTG{𨗰}{32393}
\saveTG{𨙣}{32393}
\saveTG{𠝬}{32400}
\saveTG{𠝠}{32400}
\saveTG{聻}{32401}
\saveTG{𫆏}{32401}
\saveTG{丵}{32401}
\saveTG{嬱}{32404}
\saveTG{𣁶}{32417}
\saveTG{𣮯}{32417}
\saveTG{𣁰}{32421}
\saveTG{𣮸}{32427}
\saveTG{𡞠}{32446}
\saveTG{叢}{32447}
\saveTG{𦨫}{32449}
\saveTG{魙}{32513}
\saveTG{𩠫}{32562}
\saveTG{割}{32600}
\saveTG{𠯺}{32601}
\saveTG{𩉍}{32602}
\saveTG{𣋫}{32602}
\saveTG{𠻡}{32602}
\saveTG{𪡤}{32603}
\saveTG{䤔}{32604}
\saveTG{𣯃}{32617}
\saveTG{𣂦}{32621}
\saveTG{彮}{32622}
\saveTG{剆}{32700}
\saveTG{训}{32700}
\saveTG{𠞀}{32700}
\saveTG{𠧢}{32712}
\saveTG{讬}{32714}
\saveTG{𫍳}{32715}
\saveTG{}{32721}
\saveTG{䜣}{32721}
\saveTG{𣂞}{32721}
\saveTG{诱}{32727}
\saveTG{𫍤}{32728}
\saveTG{𩟗}{32732}
\saveTG{𩜸}{32732}
\saveTG{𫗚}{32732}
\saveTG{𩚮}{32737}
\saveTG{诋}{32740}
\saveTG{诞}{32741}
\saveTG{诉}{32741}
\saveTG{诿}{32744}
\saveTG{谖}{32747}
\saveTG{𫍞}{32749}
\saveTG{诟}{32761}
\saveTG{诣}{32761}
\saveTG{谐}{32762}
\saveTG{话}{32764}
\saveTG{讪}{32770}
\saveTG{讻}{32770}
\saveTG{诎}{32772}
\saveTG{谣}{32772}
\saveTG{峾}{32772}
\saveTG{凿}{32772}
\saveTG{𫍚}{32784}
\saveTG{}{32794}
\saveTG{𠠖}{32800}
\saveTG{𤄤}{32801}
\saveTG{奨}{32804}
\saveTG{菐}{32805}
\saveTG{𧵱}{32806}
\saveTG{𤎀}{32809}
\saveTG{㲫}{32817}
\saveTG{𣰨}{32817}
\saveTG{𠟪}{32900}
\saveTG{𠠙}{32900}
\saveTG{氷}{32900}
\saveTG{𥜙}{32901}
\saveTG{𣜅}{32904}
\saveTG{𣓮}{32904}
\saveTG{楽}{32904}
\saveTG{業}{32905}
\saveTG{䄳}{32911}
\saveTG{𠄅}{32917}
\saveTG{𣮥}{32917}
\saveTG{𣯸}{32917}
\saveTG{𥣳}{32969}
\saveTG{𥣇}{32986}
\saveTG{𪴨}{32995}
\saveTG{𢗰}{33000}
\saveTG{灬}{33000}
\saveTG{}{33000}
\saveTG{心}{33000}
\saveTG{必}{33004}
\saveTG{𢖩}{33017}
\saveTG{𧸈}{33086}
\saveTG{𪶚}{33100}
\saveTG{沁}{33100}
\saveTG{𠖶}{33100}
\saveTG{𣱶}{33100}
\saveTG{𣵝}{33100}
\saveTG{𣳢}{33100}
\saveTG{𥂈}{33102}
\saveTG{𧗉}{33102}
\saveTG{𥁑}{33102}
\saveTG{𥂆}{33102}
\saveTG{𪤊}{33104}
\saveTG{𡑠}{33104}
\saveTG{泌}{33104}
\saveTG{𡊭}{33104}
\saveTG{𡏋}{33104}
\saveTG{㙙}{33104}
\saveTG{𨤩}{33105}
\saveTG{沪}{33107}
\saveTG{㴉}{33111}
\saveTG{𣹧}{33111}
\saveTG{涳}{33112}
\saveTG{涴}{33112}
\saveTG{沋}{33112}
\saveTG{沱}{33112}
\saveTG{㴵}{33112}
\saveTG{浣}{33112}
\saveTG{沇}{33112}
\saveTG{𣶺}{33112}
\saveTG{㵥}{33112}
\saveTG{𣾈}{33112}
\saveTG{𡪀}{33112}
\saveTG{漥}{33114}
\saveTG{㳴}{33114}
\saveTG{𣾯}{33114}
\saveTG{𣷆}{33114}
\saveTG{滱}{33114}
\saveTG{泷}{33114}
\saveTG{浝}{33114}
\saveTG{潌}{33114}
\saveTG{𤀕}{33114}
\saveTG{𣻎}{33114}
\saveTG{𣽦}{33114}
\saveTG{𣿷}{33114}
\saveTG{𣷂}{33114}
\saveTG{𣸫}{33114}
\saveTG{㴏}{33114}
\saveTG{渲}{33116}
\saveTG{𣵻}{33117}
\saveTG{𠧞}{33117}
\saveTG{𪶷}{33117}
\saveTG{𣵆}{33117}
\saveTG{𣵇}{33117}
\saveTG{𣲼}{33117}
\saveTG{㴳}{33117}
\saveTG{𣸴}{33117}
\saveTG{𣾊}{33117}
\saveTG{滬}{33117}
\saveTG{𣲽}{33117}
\saveTG{㵃}{33117}
\saveTG{𣸓}{33118}
\saveTG{𣺴}{33121}
\saveTG{𣿾}{33121}
\saveTG{濘}{33121}
\saveTG{泞}{33121}
\saveTG{滲}{33122}
\saveTG{㵳}{33122}
\saveTG{渗}{33122}
\saveTG{浦}{33127}
\saveTG{㴜}{33127}
\saveTG{瀉}{33127}
\saveTG{𤅳}{33127}
\saveTG{浳}{33127}
\saveTG{𣴩}{33127}
\saveTG{𤆀}{33127}
\saveTG{㳙}{33127}
\saveTG{𣺃}{33127}
\saveTG{𣵈}{33127}
\saveTG{𣶆}{33127}
\saveTG{𣼡}{33127}
\saveTG{𤃼}{33127}
\saveTG{𣹷}{33127}
\saveTG{𤁭}{33127}
\saveTG{𪸋}{33127}
\saveTG{𪶉}{33127}
\saveTG{澝}{33127}
\saveTG{𤀨}{33128}
\saveTG{𣾫}{33130}
\saveTG{𪶩}{33130}
\saveTG{𣸘}{33131}
\saveTG{𣼄}{33131}
\saveTG{浤}{33132}
\saveTG{㴃}{33132}
\saveTG{𤃆}{33132}
\saveTG{𣿵}{33132}
\saveTG{𣺊}{33132}
\saveTG{𣾋}{33132}
\saveTG{浪}{33132}
\saveTG{溛}{33132}
\saveTG{𣽬}{33133}
\saveTG{淧}{33134}
\saveTG{澸}{33135}
\saveTG{𣿶}{33136}
\saveTG{𧍦}{33136}
\saveTG{瀗}{33136}
\saveTG{𣷁}{33137}
\saveTG{㵕}{33138}
\saveTG{𪷖}{33138}
\saveTG{㳯}{33141}
\saveTG{𣁦}{33141}
\saveTG{𣳿}{33141}
\saveTG{𤀏}{33141}
\saveTG{滓}{33141}
\saveTG{𠗕}{33142}
\saveTG{溥}{33142}
\saveTG{𣽡}{33143}
\saveTG{洝}{33144}
\saveTG{㳎}{33144}
\saveTG{𣹦}{33147}
\saveTG{漃}{33147}
\saveTG{濅}{33147}
\saveTG{浚}{33147}
\saveTG{泼}{33147}
\saveTG{𣸈}{33147}
\saveTG{𤄘}{33147}
\saveTG{沷}{33147}
\saveTG{𣿟}{33147}
\saveTG{冹}{33147}
\saveTG{𤀑}{33147}
\saveTG{𣽹}{33147}
\saveTG{𣽧}{33147}
\saveTG{𣿈}{33148}
\saveTG{渽}{33150}
\saveTG{𣿐}{33150}
\saveTG{瀻}{33150}
\saveTG{洠}{33150}
\saveTG{𣳡}{33150}
\saveTG{𤄔}{33150}
\saveTG{㳀}{33150}
\saveTG{溨}{33150}
\saveTG{𤁢}{33150}
\saveTG{𪭕}{33150}
\saveTG{𣿽}{33150}
\saveTG{𣸶}{33150}
\saveTG{涐}{33150}
\saveTG{濈}{33150}
\saveTG{减}{33150}
\saveTG{溅}{33150}
\saveTG{浅}{33150}
\saveTG{減}{33150}
\saveTG{瀐}{33150}
\saveTG{瀸}{33150}
\saveTG{滅}{33150}
\saveTG{浌}{33150}
\saveTG{泧}{33150}
\saveTG{淢}{33150}
\saveTG{𪵷}{33151}
\saveTG{𤂴}{33151}
\saveTG{𤃪}{33151}
\saveTG{㵴}{33151}
\saveTG{𣺁}{33152}
\saveTG{浶}{33152}
\saveTG{𣺭}{33152}
\saveTG{𪞞}{33152}
\saveTG{𣴤}{33153}
\saveTG{㳚}{33153}
\saveTG{𣽖}{33153}
\saveTG{淺}{33153}
\saveTG{濺}{33153}
\saveTG{𪷻}{33153}
\saveTG{𣸵}{33154}
\saveTG{𣶔}{33154}
\saveTG{𢦬}{33154}
\saveTG{𢧞}{33154}
\saveTG{𣴛}{33154}
\saveTG{㳦}{33154}
\saveTG{𤄣}{33155}
\saveTG{𣴮}{33155}
\saveTG{𣽚}{33156}
\saveTG{渖}{33156}
\saveTG{𪷚}{33156}
\saveTG{渽}{33156}
\saveTG{𪶠}{33157}
\saveTG{𠗱}{33157}
\saveTG{𣸂}{33158}
\saveTG{㵄}{33158}
\saveTG{㓕}{33158}
\saveTG{𠗼}{33159}
\saveTG{治}{33160}
\saveTG{冶}{33160}
\saveTG{𪶝}{33161}
\saveTG{𣽸}{33161}
\saveTG{㴼}{33162}
\saveTG{𣷅}{33164}
\saveTG{𪷛}{33166}
\saveTG{𣾉}{33167}
\saveTG{𣼂}{33168}
\saveTG{㴭}{33168}
\saveTG{𪸆}{33168}
\saveTG{溶}{33168}
\saveTG{瀋}{33169}
\saveTG{滵}{33172}
\saveTG{㵠}{33172}
\saveTG{𣺑}{33172}
\saveTG{涫}{33177}
\saveTG{𪷆}{33180}
\saveTG{瀽}{33181}
\saveTG{淀}{33181}
\saveTG{𤀋}{33181}
\saveTG{滨}{33181}
\saveTG{泬}{33182}
\saveTG{𪷱}{33182}
\saveTG{㴱}{33182}
\saveTG{𠗞}{33182}
\saveTG{㵓}{33182}
\saveTG{𣸋}{33184}
\saveTG{𤅊}{33184}
\saveTG{湥}{33184}
\saveTG{状}{33184}
\saveTG{涋}{33184}
\saveTG{涘}{33184}
\saveTG{淚}{33184}
\saveTG{涙}{33184}
\saveTG{洑}{33184}
\saveTG{汱}{33184}
\saveTG{𣽼}{33184}
\saveTG{𤀿}{33184}
\saveTG{𤠊}{33184}
\saveTG{𣶍}{33184}
\saveTG{𣾼}{33185}
\saveTG{濵}{33186}
\saveTG{濱}{33186}
\saveTG{𤁂}{33186}
\saveTG{𤅋}{33186}
\saveTG{演}{33186}
\saveTG{𪷅}{33189}
\saveTG{𪷯}{33191}
\saveTG{𪞥}{33191}
\saveTG{淙}{33191}
\saveTG{泳}{33192}
\saveTG{浨}{33194}
\saveTG{沭}{33194}
\saveTG{淭}{33194}
\saveTG{𪷷}{33196}
\saveTG{𠗈}{33199}
\saveTG{浗}{33199}
\saveTG{𪷫}{33199}
\saveTG{褂}{33200}
\saveTG{䃼}{33200}
\saveTG{𥘚}{33200}
\saveTG{补}{33200}
\saveTG{祕}{33204}
\saveTG{袐}{33204}
\saveTG{𥙤}{33211}
\saveTG{𧚬}{33212}
\saveTG{𧝱}{33212}
\saveTG{袉}{33212}
\saveTG{𧜐}{33213}
\saveTG{𥚐}{33214}
\saveTG{䙋}{33216}
\saveTG{𥙇}{33217}
\saveTG{䘼}{33217}
\saveTG{𥙜}{33217}
\saveTG{𧚁}{33217}
\saveTG{𣾃}{33217}
\saveTG{𧙇}{33217}
\saveTG{𧙀}{33218}
\saveTG{䘢}{33221}
\saveTG{襂}{33222}
\saveTG{𤰐}{33227}
\saveTG{𧟑}{33227}
\saveTG{黼}{33227}
\saveTG{𥜴}{33227}
\saveTG{褊}{33227}
\saveTG{補}{33227}
\saveTG{𪩾}{33227}
\saveTG{𧚴}{33231}
\saveTG{𧝈}{33231}
\saveTG{𣴏}{33232}
\saveTG{𧞼}{33232}
\saveTG{𧚅}{33232}
\saveTG{𧝯}{33232}
\saveTG{𧞰}{33235}
\saveTG{𧘵}{33240}
\saveTG{䘝}{33240}
\saveTG{𥘒}{33240}
\saveTG{𫀐}{33241}
\saveTG{𧙢}{33241}
\saveTG{𥙔}{33242}
\saveTG{禣}{33242}
\saveTG{䙏}{33243}
\saveTG{𡨺}{33243}
\saveTG{䘬}{33244}
\saveTG{袚}{33247}
\saveTG{黻}{33247}
\saveTG{祓}{33247}
\saveTG{𥚂}{33247}
\saveTG{𧚉}{33247}
\saveTG{袯}{33247}
\saveTG{𧜮}{33247}
\saveTG{𡨸}{33247}
\saveTG{𧞩}{33250}
\saveTG{𠶶}{33250}
\saveTG{䄀}{33250}
\saveTG{襶}{33250}
\saveTG{𧝧}{33250}
\saveTG{祴}{33250}
\saveTG{襳}{33250}
\saveTG{裓}{33250}
\saveTG{𧞛}{33251}
\saveTG{𧞩}{33251}
\saveTG{𧞬}{33251}
\saveTG{𧚶}{33252}
\saveTG{𧙠}{33253}
\saveTG{䙁}{33253}
\saveTG{𢣳}{33253}
\saveTG{𧟀}{33254}
\saveTG{䄉}{33255}
\saveTG{𧚄}{33255}
\saveTG{𧛷}{33256}
\saveTG{𫀍}{33256}
\saveTG{𧚑}{33256}
\saveTG{𧝊}{33256}
\saveTG{𧛡}{33256}
\saveTG{𧛠}{33258}
\saveTG{䙘}{33259}
\saveTG{𧛪}{33261}
\saveTG{𥙉}{33261}
\saveTG{𧜅}{33261}
\saveTG{𥚰}{33264}
\saveTG{褣}{33268}
\saveTG{䘾}{33277}
\saveTG{袕}{33282}
\saveTG{䙯}{33282}
\saveTG{䘺}{33282}
\saveTG{䙭}{33282}
\saveTG{袱}{33284}
\saveTG{𧞾}{33284}
\saveTG{𧛗}{33284}
\saveTG{𥛎}{33284}
\saveTG{𧞍}{33291}
\saveTG{𥚎}{33291}
\saveTG{𥛫}{33294}
\saveTG{䘤}{33294}
\saveTG{𥙹}{33299}
\saveTG{迯}{33300}
\saveTG{𨕳}{33301}
\saveTG{𨒜}{33301}
\saveTG{𫐸}{33301}
\saveTG{迱}{33301}
\saveTG{𨒆}{33301}
\saveTG{逋}{33302}
\saveTG{𫑈}{33302}
\saveTG{遪}{33302}
\saveTG{遍}{33302}
\saveTG{𨕠}{33302}
\saveTG{𨒓}{33302}
\saveTG{𨑫}{33302}
\saveTG{𨔎}{33302}
\saveTG{𨕝}{33302}
\saveTG{𨒴}{33303}
\saveTG{𨗘}{33303}
\saveTG{𨗉}{33303}
\saveTG{邃}{33303}
\saveTG{䢓}{33304}
\saveTG{逡}{33304}
\saveTG{𨕦}{33304}
\saveTG{䢘}{33304}
\saveTG{䢕}{33305}
\saveTG{𨒋}{33305}
\saveTG{𫑋}{33305}
\saveTG{迨}{33306}
\saveTG{逭}{33307}
\saveTG{逘}{33308}
\saveTG{𫑐}{33308}
\saveTG{𫐲}{33308}
\saveTG{𨙇}{33308}
\saveTG{𨘄}{33308}
\saveTG{𨖋}{33308}
\saveTG{𨓩}{33308}
\saveTG{迖}{33308}
\saveTG{𫐱}{33309}
\saveTG{述}{33309}
\saveTG{逑}{33309}
\saveTG{𨕁}{33309}
\saveTG{䢤}{33309}
\saveTG{𣳵}{33317}
\saveTG{𡨹}{33321}
\saveTG{惢}{33330}
\saveTG{𢠔}{33335}
\saveTG{𪬬}{33338}
\saveTG{𪭛}{33406}
\saveTG{𢖻}{33407}
\saveTG{𪦽}{33407}
\saveTG{𡺺}{33427}
\saveTG{𠂭}{33433}
\saveTG{㜑}{33442}
\saveTG{𪧚}{33443}
\saveTG{𡟖}{33444}
\saveTG{𡦂}{33447}
\saveTG{𡫹}{33456}
\saveTG{𤞮}{33484}
\saveTG{𪩓}{33504}
\saveTG{肈}{33507}
\saveTG{𦘥}{33574}
\saveTG{吢}{33600}
\saveTG{𠰣}{33601}
\saveTG{𤽣}{33602}
\saveTG{𡶾}{33602}
\saveTG{䁉}{33605}
\saveTG{啔}{33605}
\saveTG{𥒳}{33617}
\saveTG{𡫴}{33656}
\saveTG{𢙈}{33661}
\saveTG{𡫲}{33664}
\saveTG{𡫥}{33664}
\saveTG{𫍇}{33668}
\saveTG{讣}{33700}
\saveTG{𢗺}{33710}
\saveTG{谊}{33712}
\saveTG{谧}{33712}
\saveTG{}{33712}
\saveTG{诧}{33714}
\saveTG{𪩡}{33717}
\saveTG{𫍡}{33717}
\saveTG{谝}{33727}
\saveTG{㟧}{33727}
\saveTG{𧚚}{33732}
\saveTG{𩟸}{33732}
\saveTG{试}{33740}
\saveTG{𨐝}{33741}
\saveTG{诫}{33750}
\saveTG{诚}{33750}
\saveTG{}{33750}
\saveTG{}{33750}
\saveTG{谶}{33750}
\saveTG{谉}{33756}
\saveTG{𫍯}{33756}
\saveTG{诒}{33760}
\saveTG{𡶇}{33772}
\saveTG{谳}{33784}
\saveTG{诶}{33784}
\saveTG{𨕍}{33802}
\saveTG{𧹆}{33806}
\saveTG{𪸠}{33832}
\saveTG{戭}{33850}
\saveTG{𦁞}{33903}
\saveTG{䋯}{33903}
\saveTG{繠}{33903}
\saveTG{橤}{33904}
\saveTG{粱}{33904}
\saveTG{𥞹}{33904}
\saveTG{𣚘}{33904}
\saveTG{梁}{33904}
\saveTG{𣙤}{33945}
\saveTG{𣿀}{33998}
\saveTG{斗}{34000}
\saveTG{为}{34027}
\saveTG{㟿}{34044}
\saveTG{澍}{34100}
\saveTG{𣻁}{34100}
\saveTG{㲼}{34100}
\saveTG{汁}{34100}
\saveTG{湗}{34100}
\saveTG{濧}{34100}
\saveTG{對}{34100}
\saveTG{泭}{34100}
\saveTG{䀅}{34102}
\saveTG{𥁇}{34102}
\saveTG{𤀢}{34102}
\saveTG{𥁭}{34102}
\saveTG{𧖶}{34102}
\saveTG{𪶄}{34103}
\saveTG{𪶊}{34103}
\saveTG{㴬}{34103}
\saveTG{㵱}{34103}
\saveTG{𣻷}{34103}
\saveTG{㴻}{34103}
\saveTG{𣻠}{34103}
\saveTG{𣷊}{34103}
\saveTG{㳆}{34103}
\saveTG{𡒗}{34104}
\saveTG{𪣭}{34104}
\saveTG{𨮝}{34109}
\saveTG{鍌}{34109}
\saveTG{錃}{34109}
\saveTG{鍙}{34109}
\saveTG{𨧽}{34109}
\saveTG{𨧁}{34109}
\saveTG{𪶇}{34110}
\saveTG{汢}{34110}
\saveTG{𣴣}{34110}
\saveTG{壮}{34110}
\saveTG{沎}{34110}
\saveTG{淽}{34111}
\saveTG{灩}{34112}
\saveTG{灎}{34112}
\saveTG{洗}{34112}
\saveTG{冼}{34112}
\saveTG{灆}{34112}
\saveTG{溘}{34112}
\saveTG{涜}{34112}
\saveTG{淔}{34112}
\saveTG{沈}{34112}
\saveTG{池}{34112}
\saveTG{澆}{34112}
\saveTG{濭}{34112}
\saveTG{𪷞}{34112}
\saveTG{𣹆}{34112}
\saveTG{𤅿}{34112}
\saveTG{𤅢}{34112}
\saveTG{㳣}{34112}
\saveTG{𤃓}{34112}
\saveTG{𣳇}{34112}
\saveTG{滢}{34113}
\saveTG{𤁔}{34113}
\saveTG{𣺜}{34113}
\saveTG{漜}{34114}
\saveTG{溎}{34114}
\saveTG{洼}{34114}
\saveTG{𤁉}{34114}
\saveTG{𣻺}{34114}
\saveTG{𤁅}{34114}
\saveTG{𪶔}{34114}
\saveTG{淕}{34114}
\saveTG{𣳃}{34114}
\saveTG{𣼥}{34114}
\saveTG{𣾛}{34114}
\saveTG{𤄲}{34115}
\saveTG{㴶}{34115}
\saveTG{𣾑}{34115}
\saveTG{𣿅}{34115}
\saveTG{灌}{34115}
\saveTG{漌}{34115}
\saveTG{𠒮}{34116}
\saveTG{𤁮}{34116}
\saveTG{淹}{34116}
\saveTG{渣}{34116}
\saveTG{𣷾}{34117}
\saveTG{𣿑}{34117}
\saveTG{𣸱}{34117}
\saveTG{𣸚}{34117}
\saveTG{𣼜}{34117}
\saveTG{㳸}{34117}
\saveTG{𪞭}{34117}
\saveTG{𠗌}{34117}
\saveTG{𪸊}{34117}
\saveTG{㳈}{34117}
\saveTG{𣴧}{34117}
\saveTG{㳳}{34117}
\saveTG{㴷}{34117}
\saveTG{𣺬}{34117}
\saveTG{㵁}{34117}
\saveTG{𣸃}{34117}
\saveTG{𣻻}{34117}
\saveTG{𣸿}{34117}
\saveTG{𣸆}{34117}
\saveTG{㲺}{34117}
\saveTG{𣷝}{34117}
\saveTG{𤀘}{34117}
\saveTG{泄}{34117}
\saveTG{氿}{34117}
\saveTG{港}{34117}
\saveTG{滼}{34117}
\saveTG{潱}{34118}
\saveTG{𤃥}{34118}
\saveTG{𠗮}{34118}
\saveTG{𣲔}{34118}
\saveTG{𣺔}{34118}
\saveTG{湛}{34118}
\saveTG{𣺠}{34120}
\saveTG{𣼵}{34120}
\saveTG{𪷼}{34121}
\saveTG{𣺈}{34121}
\saveTG{𤀽}{34121}
\saveTG{渮}{34121}
\saveTG{渏}{34121}
\saveTG{漪}{34121}
\saveTG{𣾏}{34121}
\saveTG{𣾐}{34121}
\saveTG{灪}{34122}
\saveTG{𤁸}{34122}
\saveTG{𤂣}{34124}
\saveTG{𣷏}{34127}
\saveTG{𤅠}{34127}
\saveTG{𣾖}{34127}
\saveTG{𤂙}{34127}
\saveTG{𣹩}{34127}
\saveTG{𣾗}{34127}
\saveTG{𤀐}{34127}
\saveTG{𤅬}{34127}
\saveTG{𣵒}{34127}
\saveTG{𣷿}{34127}
\saveTG{𣿂}{34127}
\saveTG{𣻉}{34127}
\saveTG{𣺏}{34127}
\saveTG{𠘃}{34127}
\saveTG{㪲}{34127}
\saveTG{㓓}{34127}
\saveTG{𪞗}{34127}
\saveTG{𪵶}{34127}
\saveTG{𣲒}{34127}
\saveTG{𣷃}{34127}
\saveTG{𤂉}{34127}
\saveTG{𤃽}{34127}
\saveTG{𤃞}{34127}
\saveTG{𪷒}{34127}
\saveTG{𤁟}{34127}
\saveTG{𣵪}{34127}
\saveTG{㳍}{34127}
\saveTG{𤄆}{34127}
\saveTG{𤀚}{34127}
\saveTG{𤃦}{34127}
\saveTG{𣹀}{34127}
\saveTG{𣼛}{34127}
\saveTG{㵧}{34127}
\saveTG{𣼷}{34127}
\saveTG{𣷎}{34127}
\saveTG{𣿴}{34127}
\saveTG{瀟}{34127}
\saveTG{潇}{34127}
\saveTG{淆}{34127}
\saveTG{浠}{34127}
\saveTG{溈}{34127}
\saveTG{洧}{34127}
\saveTG{潸}{34127}
\saveTG{汭}{34127}
\saveTG{湳}{34127}
\saveTG{濗}{34127}
\saveTG{澫}{34127}
\saveTG{滿}{34127}
\saveTG{満}{34127}
\saveTG{满}{34127}
\saveTG{氻}{34127}
\saveTG{泐}{34127}
\saveTG{涝}{34127}
\saveTG{灡}{34127}
\saveTG{洘}{34127}
\saveTG{瀳}{34127}
\saveTG{洿}{34127}
\saveTG{沩}{34127}
\saveTG{淓}{34127}
\saveTG{滯}{34127}
\saveTG{滞}{34127}
\saveTG{渤}{34127}
\saveTG{泑}{34127}
\saveTG{𣽽}{34127}
\saveTG{𣶐}{34127}
\saveTG{𣶼}{34127}
\saveTG{𣽂}{34128}
\saveTG{㓔}{34130}
\saveTG{汰}{34130}
\saveTG{㳤}{34130}
\saveTG{浾}{34131}
\saveTG{㳒}{34131}
\saveTG{𤄝}{34131}
\saveTG{𤃤}{34131}
\saveTG{𣺮}{34131}
\saveTG{𪷵}{34131}
\saveTG{𤃇}{34131}
\saveTG{𣶽}{34131}
\saveTG{𤂠}{34131}
\saveTG{竑}{34131}
\saveTG{濍}{34132}
\saveTG{𣻡}{34132}
\saveTG{浓}{34132}
\saveTG{𣸳}{34132}
\saveTG{瀡}{34132}
\saveTG{濛}{34132}
\saveTG{汯}{34132}
\saveTG{法}{34132}
\saveTG{𣾕}{34132}
\saveTG{𣸷}{34132}
\saveTG{𣵟}{34132}
\saveTG{𤅵}{34132}
\saveTG{㵦}{34132}
\saveTG{𤄓}{34132}
\saveTG{𤂑}{34132}
\saveTG{𣾓}{34132}
\saveTG{𣺚}{34132}
\saveTG{溒}{34132}
\saveTG{㳡}{34133}
\saveTG{𪷺}{34133}
\saveTG{𣾣}{34134}
\saveTG{𤀱}{34134}
\saveTG{𣽑}{34134}
\saveTG{𤂧}{34135}
\saveTG{涟}{34135}
\saveTG{澾}{34135}
\saveTG{𤃧}{34135}
\saveTG{𣿔}{34135}
\saveTG{㵭}{34136}
\saveTG{𤃖}{34136}
\saveTG{𤁬}{34136}
\saveTG{𤀸}{34136}
\saveTG{𣷸}{34137}
\saveTG{㴀}{34137}
\saveTG{𣺖}{34138}
\saveTG{㳠}{34138}
\saveTG{𣶠}{34138}
\saveTG{𣽫}{34138}
\saveTG{}{34138}
\saveTG{𤃜}{34139}
\saveTG{妆}{34140}
\saveTG{汝}{34140}
\saveTG{泋}{34140}
\saveTG{㴛}{34140}
\saveTG{㴾}{34140}
\saveTG{溡}{34141}
\saveTG{濤}{34141}
\saveTG{㓑}{34141}
\saveTG{𣳹}{34141}
\saveTG{㳃}{34141}
\saveTG{𣻿}{34141}
\saveTG{洔}{34141}
\saveTG{涬}{34141}
\saveTG{𤃂}{34143}
\saveTG{𣹘}{34143}
\saveTG{𪸈}{34143}
\saveTG{𣿝}{34143}
\saveTG{渀}{34144}
\saveTG{𣾘}{34144}
\saveTG{𥙞}{34144}
\saveTG{泋}{34144}
\saveTG{漤}{34144}
\saveTG{𤂹}{34146}
\saveTG{𣺞}{34146}
\saveTG{凌}{34147}
\saveTG{淩}{34147}
\saveTG{涍}{34147}
\saveTG{𪸁}{34147}
\saveTG{𣽇}{34147}
\saveTG{𣵎}{34147}
\saveTG{𪵽}{34147}
\saveTG{𤅞}{34147}
\saveTG{𤄀}{34147}
\saveTG{𤃕}{34147}
\saveTG{𣿯}{34147}
\saveTG{𪷠}{34147}
\saveTG{㵻}{34147}
\saveTG{𣺙}{34147}
\saveTG{𣼊}{34147}
\saveTG{𪷥}{34147}
\saveTG{𤅨}{34147}
\saveTG{𤅯}{34147}
\saveTG{波}{34147}
\saveTG{浡}{34147}
\saveTG{洊}{34147}
\saveTG{濩}{34147}
\saveTG{汥}{34147}
\saveTG{漭}{34148}
\saveTG{𣸽}{34148}
\saveTG{㵏}{34148}
\saveTG{𣳰}{34149}
\saveTG{𤂘}{34151}
\saveTG{𣻟}{34151}
\saveTG{𣸉}{34151}
\saveTG{㶓}{34151}
\saveTG{𤃴}{34152}
\saveTG{𤂤}{34152}
\saveTG{𤂁}{34152}
\saveTG{𤂾}{34152}
\saveTG{瀎}{34153}
\saveTG{𤃿}{34153}
\saveTG{𣸻}{34153}
\saveTG{澕}{34154}
\saveTG{𣿎}{34156}
\saveTG{㵮}{34156}
\saveTG{湋}{34156}
\saveTG{㴖}{34156}
\saveTG{𣿢}{34156}
\saveTG{𣺛}{34157}
\saveTG{𣴰}{34157}
\saveTG{湋}{34157}
\saveTG{𣺾}{34160}
\saveTG{沽}{34160}
\saveTG{𪶲}{34160}
\saveTG{𣾟}{34160}
\saveTG{瀦}{34160}
\saveTG{渵}{34160}
\saveTG{潴}{34160}
\saveTG{渚}{34160}
\saveTG{㳓}{34160}
\saveTG{濇}{34161}
\saveTG{濳}{34161}
\saveTG{𣹡}{34161}
\saveTG{洁}{34161}
\saveTG{𣽋}{34161}
\saveTG{澔}{34161}
\saveTG{𪶼}{34161}
\saveTG{浩}{34161}
\saveTG{溚}{34161}
\saveTG{𤀺}{34161}
\saveTG{㵙}{34161}
\saveTG{𣿏}{34161}
\saveTG{𤁏}{34161}
\saveTG{㳻}{34161}
\saveTG{㵆}{34161}
\saveTG{𪧱}{34161}
\saveTG{瀒}{34161}
\saveTG{𤃐}{34162}
\saveTG{𣼤}{34162}
\saveTG{𣶥}{34162}
\saveTG{𣹓}{34163}
\saveTG{㵔}{34164}
\saveTG{渃}{34164}
\saveTG{𣺋}{34164}
\saveTG{𤀞}{34164}
\saveTG{𤂩}{34164}
\saveTG{㶆}{34164}
\saveTG{渵}{34164}
\saveTG{𤃺}{34166}
\saveTG{濸}{34167}
\saveTG{𤁇}{34168}
\saveTG{𪵾}{34169}
\saveTG{澘}{34169}
\saveTG{㵇}{34170}
\saveTG{𣲟}{34170}
\saveTG{𣳘}{34170}
\saveTG{泔}{34170}
\saveTG{𣼏}{34172}
\saveTG{𤁨}{34177}
\saveTG{𣴘}{34180}
\saveTG{𪥔}{34180}
\saveTG{汏}{34180}
\saveTG{𤄱}{34181}
\saveTG{洪}{34181}
\saveTG{𣼮}{34181}
\saveTG{𣾁}{34181}
\saveTG{濋}{34181}
\saveTG{滇}{34181}
\saveTG{淇}{34181}
\saveTG{𣺺}{34182}
\saveTG{𣸣}{34182}
\saveTG{𪷽}{34182}
\saveTG{㳲}{34183}
\saveTG{漠}{34184}
\saveTG{渎}{34184}
\saveTG{漺}{34184}
\saveTG{𤂨}{34184}
\saveTG{㵹}{34184}
\saveTG{𣿮}{34184}
\saveTG{𠗾}{34184}
\saveTG{𣵽}{34184}
\saveTG{漢}{34185}
\saveTG{渶}{34185}
\saveTG{灒}{34186}
\saveTG{凟}{34186}
\saveTG{瀆}{34186}
\saveTG{濆}{34186}
\saveTG{𤅖}{34186}
\saveTG{𤂲}{34186}
\saveTG{㶂}{34186}
\saveTG{𣽒}{34186}
\saveTG{㶇}{34186}
\saveTG{潢}{34186}
\saveTG{𤃘}{34186}
\saveTG{𤃋}{34186}
\saveTG{𤄠}{34186}
\saveTG{𤂫}{34186}
\saveTG{㳛}{34187}
\saveTG{浹}{34188}
\saveTG{㴺}{34188}
\saveTG{𣵩}{34189}
\saveTG{洃}{34189}
\saveTG{𣸨}{34189}
\saveTG{𣿳}{34189}
\saveTG{㵉}{34190}
\saveTG{㳜}{34190}
\saveTG{淋}{34190}
\saveTG{沐}{34190}
\saveTG{凚}{34191}
\saveTG{渿}{34191}
\saveTG{㴎}{34191}
\saveTG{澿}{34191}
\saveTG{𤁱}{34191}
\saveTG{𣿻}{34192}
\saveTG{濝}{34193}
\saveTG{㳔}{34193}
\saveTG{溹}{34193}
\saveTG{潆}{34193}
\saveTG{𣼒}{34193}
\saveTG{渫}{34194}
\saveTG{冻}{34194}
\saveTG{湈}{34194}
\saveTG{𣺢}{34194}
\saveTG{溁}{34194}
\saveTG{㵩}{34194}
\saveTG{𣸑}{34194}
\saveTG{𠗨}{34194}
\saveTG{㴕}{34194}
\saveTG{𤄶}{34194}
\saveTG{𤂋}{34194}
\saveTG{𤅅}{34194}
\saveTG{𣷞}{34194}
\saveTG{𦴶}{34194}
\saveTG{𣸡}{34194}
\saveTG{𠘆}{34194}
\saveTG{潹}{34194}
\saveTG{潦}{34196}
\saveTG{淶}{34198}
\saveTG{𤀛}{34198}
\saveTG{漆}{34199}
\saveTG{𣾰}{34199}
\saveTG{𧘬}{34200}
\saveTG{襨}{34200}
\saveTG{袝}{34200}
\saveTG{祔}{34200}
\saveTG{衬}{34200}
\saveTG{𧘓}{34200}
\saveTG{㐧}{34202}
\saveTG{𥘔}{34202}
\saveTG{𧝪}{34203}
\saveTG{𧘞}{34203}
\saveTG{𥘑}{34203}
\saveTG{𥘥}{34204}
\saveTG{𥚞}{34207}
\saveTG{社}{34210}
\saveTG{𥛐}{34212}
\saveTG{𧛾}{34212}
\saveTG{𧞔}{34212}
\saveTG{衴}{34212}
\saveTG{𧛍}{34212}
\saveTG{𥙀}{34212}
\saveTG{𧟋}{34212}
\saveTG{袏}{34212}
\saveTG{衪}{34212}
\saveTG{禃}{34212}
\saveTG{祂}{34212}
\saveTG{襓}{34212}
\saveTG{𥛄}{34212}
\saveTG{𥙒}{34213}
\saveTG{𥙭}{34214}
\saveTG{袿}{34214}
\saveTG{𫌊}{34214}
\saveTG{𥘠}{34214}
\saveTG{𥚊}{34214}
\saveTG{禥}{34214}
\saveTG{䙮}{34215}
\saveTG{𫌁}{34216}
\saveTG{裺}{34216}
\saveTG{𥚧}{34216}
\saveTG{㵜}{34217}
\saveTG{𧛍}{34217}
\saveTG{䄋}{34217}
\saveTG{袣}{34217}
\saveTG{𧳗}{34217}
\saveTG{𥙕}{34217}
\saveTG{䄁}{34217}
\saveTG{𥚮}{34218}
\saveTG{𡬗}{34218}
\saveTG{䄎}{34221}
\saveTG{裿}{34221}
\saveTG{𥘋}{34227}
\saveTG{𥛒}{34227}
\saveTG{𥛂}{34227}
\saveTG{𥜍}{34227}
\saveTG{䘜}{34227}
\saveTG{𧙛}{34227}
\saveTG{𧝶}{34227}
\saveTG{𧛩}{34227}
\saveTG{𫋸}{34227}
\saveTG{𫀃}{34227}
\saveTG{𧜿}{34227}
\saveTG{袎}{34227}
\saveTG{衲}{34227}
\saveTG{襔}{34227}
\saveTG{襽}{34227}
\saveTG{袴}{34227}
\saveTG{襺}{34227}
\saveTG{𧞳}{34227}
\saveTG{𦝎}{34227}
\saveTG{𥛣}{34227}
\saveTG{𧝫}{34227}
\saveTG{𧜵}{34227}
\saveTG{𥚓}{34227}
\saveTG{䙊}{34227}
\saveTG{𧙲}{34227}
\saveTG{䙃}{34227}
\saveTG{𧝍}{34227}
\saveTG{𧟆}{34227}
\saveTG{𫋾}{34227}
\saveTG{𣃜}{34227}
\saveTG{𥚨}{34227}
\saveTG{襼}{34231}
\saveTG{㼉}{34231}
\saveTG{䄊}{34231}
\saveTG{䙩}{34232}
\saveTG{𠖨}{34232}
\saveTG{袪}{34232}
\saveTG{褤}{34232}
\saveTG{𤁞}{34232}
\saveTG{祛}{34232}
\saveTG{𧞅}{34235}
\saveTG{裢}{34235}
\saveTG{𣀅}{34240}
\saveTG{䘠}{34240}
\saveTG{禱}{34241}
\saveTG{𥘧}{34241}
\saveTG{𥙶}{34241}
\saveTG{𧛼}{34243}
\saveTG{𧛶}{34243}
\saveTG{𧜒}{34244}
\saveTG{𧜆}{34244}
\saveTG{𥚙}{34244}
\saveTG{𢩣}{34245}
\saveTG{𫌆}{34247}
\saveTG{𧞤}{34247}
\saveTG{𧚆}{34247}
\saveTG{被}{34247}
\saveTG{袸}{34247}
\saveTG{祾}{34247}
\saveTG{衼}{34247}
\saveTG{裬}{34247}
\saveTG{䃽}{34247}
\saveTG{𥜳}{34251}
\saveTG{襻}{34252}
\saveTG{襪}{34253}
\saveTG{𥛵}{34254}
\saveTG{禕}{34256}
\saveTG{褘}{34256}
\saveTG{𪫚}{34257}
\saveTG{𧙖}{34260}
\saveTG{禇}{34260}
\saveTG{祜}{34260}
\saveTG{褚}{34260}
\saveTG{祐}{34260}
\saveTG{𧙗}{34260}
\saveTG{祰}{34261}
\saveTG{𥜎}{34261}
\saveTG{褡}{34261}
\saveTG{𥜑}{34261}
\saveTG{𧛊}{34261}
\saveTG{祮}{34261}
\saveTG{䄍}{34261}
\saveTG{袺}{34261}
\saveTG{𡫸}{34261}
\saveTG{禧}{34261}
\saveTG{𣂋}{34261}
\saveTG{𪞅}{34262}
\saveTG{𧛭}{34264}
\saveTG{𡪄}{34264}
\saveTG{𥛼}{34266}
\saveTG{𥙻}{34268}
\saveTG{𥙗}{34268}
\saveTG{𧜩}{34272}
\saveTG{䙦}{34272}
\saveTG{𧜖}{34281}
\saveTG{𣄃}{34281}
\saveTG{禛}{34281}
\saveTG{祺}{34281}
\saveTG{褀}{34281}
\saveTG{𥙖}{34281}
\saveTG{𧘹}{34283}
\saveTG{䄏}{34284}
\saveTG{𧟅}{34286}
\saveTG{𧟎}{34286}
\saveTG{𧹍}{34286}
\saveTG{䄣}{34286}
\saveTG{𧝭}{34286}
\saveTG{襸}{34286}
\saveTG{襩}{34286}
\saveTG{襫}{34286}
\saveTG{禶}{34286}
\saveTG{𧝒}{34286}
\saveTG{𫀋}{34288}
\saveTG{裌}{34288}
\saveTG{𧛮}{34291}
\saveTG{襟}{34291}
\saveTG{𧛻}{34293}
\saveTG{𧝵}{34294}
\saveTG{𥜈}{34294}
\saveTG{褋}{34294}
\saveTG{禖}{34294}
\saveTG{𥛰}{34296}
\saveTG{𧝜}{34296}
\saveTG{𥚒}{34298}
\saveTG{𦠘}{34298}
\saveTG{𧜝}{34299}
\saveTG{𨑮}{34300}
\saveTG{过}{34300}
\saveTG{辻}{34300}
\saveTG{𨑍}{34301}
\saveTG{𫐥}{34301}
\saveTG{𫐦}{34301}
\saveTG{𨑻}{34301}
\saveTG{𨘾}{34301}
\saveTG{𨖫}{34301}
\saveTG{𨘝}{34301}
\saveTG{𨘦}{34301}
\saveTG{𨖽}{34301}
\saveTG{𫐬}{34301}
\saveTG{𨖑}{34301}
\saveTG{𨑒}{34301}
\saveTG{𨑘}{34301}
\saveTG{遳}{34301}
\saveTG{迆}{34301}
\saveTG{选}{34301}
\saveTG{遶}{34301}
\saveTG{逵}{34301}
\saveTG{逇}{34301}
\saveTG{迣}{34301}
\saveTG{𨔦}{34301}
\saveTG{𨒒}{34302}
\saveTG{𨔤}{34302}
\saveTG{𨑧}{34302}
\saveTG{边}{34302}
\saveTG{遰}{34302}
\saveTG{邁}{34302}
\saveTG{遖}{34302}
\saveTG{遀}{34302}
\saveTG{迶}{34302}
\saveTG{𨕷}{34302}
\saveTG{𧅣}{34302}
\saveTG{𨔳}{34302}
\saveTG{𨗪}{34302}
\saveTG{𨓾}{34302}
\saveTG{𫑁}{34302}
\saveTG{𫐫}{34302}
\saveTG{𦳖}{34302}
\saveTG{𨗲}{34302}
\saveTG{𫈡}{34302}
\saveTG{𨙟}{34302}
\saveTG{遠}{34303}
\saveTG{𨘗}{34303}
\saveTG{𨒻}{34303}
\saveTG{𨘏}{34303}
\saveTG{𨔿}{34303}
\saveTG{𨗺}{34303}
\saveTG{䢏}{34303}
\saveTG{𨙚}{34303}
\saveTG{迏}{34303}
\saveTG{迲}{34303}
\saveTG{逺}{34303}
\saveTG{逩}{34304}
\saveTG{𨓇}{34304}
\saveTG{𨘬}{34304}
\saveTG{𨖟}{34304}
\saveTG{过}{34304}
\saveTG{𨓄}{34304}
\saveTG{𨘞}{34304}
\saveTG{𨕡}{34304}
\saveTG{𨘚}{34304}
\saveTG{𨓡}{34304}
\saveTG{𨒸}{34304}
\saveTG{𨑤}{34304}
\saveTG{𨕉}{34304}
\saveTG{逹}{34304}
\saveTG{𨒍}{34304}
\saveTG{𨔶}{34305}
\saveTG{𨗨}{34305}
\saveTG{𨕤}{34305}
\saveTG{𨔬}{34305}
\saveTG{连}{34305}
\saveTG{達}{34305}
\saveTG{𧅊}{34305}
\saveTG{𨔷}{34305}
\saveTG{違}{34305}
\saveTG{𨙗}{34306}
\saveTG{𨗟}{34306}
\saveTG{𨘫}{34306}
\saveTG{𨕺}{34306}
\saveTG{𨕎}{34306}
\saveTG{逪}{34306}
\saveTG{𨒐}{34306}
\saveTG{𨔽}{34306}
\saveTG{迼}{34306}
\saveTG{𨗣}{34306}
\saveTG{逽}{34306}
\saveTG{𨔾}{34306}
\saveTG{造}{34306}
\saveTG{𨕏}{34306}
\saveTG{𨔙}{34306}
\saveTG{𨔴}{34306}
\saveTG{𨑬}{34307}
\saveTG{𨕃}{34307}
\saveTG{𨒏}{34308}
\saveTG{𨒱}{34308}
\saveTG{𦽙}{34308}
\saveTG{𨙋}{34308}
\saveTG{𨙜}{34308}
\saveTG{𨘤}{34308}
\saveTG{𫐯}{34308}
\saveTG{𨒭}{34308}
\saveTG{达}{34308}
\saveTG{𦳯}{34309}
\saveTG{𨔘}{34309}
\saveTG{𨗸}{34309}
\saveTG{𨔋}{34309}
\saveTG{䢡}{34309}
\saveTG{䢞}{34309}
\saveTG{𨘎}{34309}
\saveTG{逨}{34309}
\saveTG{遼}{34309}
\saveTG{𣴭}{34318}
\saveTG{𪁶}{34327}
\saveTG{為}{34327}
\saveTG{𣴚}{34327}
\saveTG{𨓸}{34327}
\saveTG{懟}{34330}
\saveTG{𤉠}{34331}
\saveTG{懑}{34332}
\saveTG{𢡛}{34332}
\saveTG{懘}{34332}
\saveTG{懣}{34332}
\saveTG{𤒵}{34334}
\saveTG{𤂍}{34335}
\saveTG{𢢉}{34336}
\saveTG{𩸓}{34336}
\saveTG{𩼸}{34336}
\saveTG{𢜃}{34336}
\saveTG{㓒}{34338}
\saveTG{𢝳}{34338}
\saveTG{𤈀}{34339}
\saveTG{𨓣}{34343}
\saveTG{𣶃}{34346}
\saveTG{𨔱}{34358}
\saveTG{𨓳}{34369}
\saveTG{𨑿}{34372}
\saveTG{㓋}{34381}
\saveTG{𨖒}{34386}
\saveTG{𨕇}{34391}
\saveTG{䢫}{34392}
\saveTG{𨔻}{34392}
\saveTG{𨖚}{34396}
\saveTG{逨}{34398}
\saveTG{𣂎}{34402}
\saveTG{𫐨}{34402}
\saveTG{婆}{34404}
\saveTG{𨓈}{34406}
\saveTG{𪜛}{34412}
\saveTG{𠡓}{34427}
\saveTG{𠦑}{34430}
\saveTG{𡣩}{34442}
\saveTG{𣁾}{34443}
\saveTG{𡪻}{34460}
\saveTG{㨇}{34502}
\saveTG{皸}{34547}
\saveTG{皲}{34547}
\saveTG{碆}{34602}
\saveTG{𡁨}{34603}
\saveTG{𥆳}{34603}
\saveTG{𡭊}{34603}
\saveTG{𥇲}{34604}
\saveTG{𠴸}{34604}
\saveTG{㖳}{34612}
\saveTG{𫍉}{34617}
\saveTG{㖍}{34617}
\saveTG{𧺁}{34621}
\saveTG{㔤}{34627}
\saveTG{𪧰}{34627}
\saveTG{𠢆}{34627}
\saveTG{𢻜}{34647}
\saveTG{𩏓}{34657}
\saveTG{计}{34700}
\saveTG{谢}{34700}
\saveTG{讨}{34700}
\saveTG{议}{34703}
\saveTG{讹}{34710}
\saveTG{诜}{34712}
\saveTG{𫍙}{34712}
\saveTG{谎}{34712}
\saveTG{诖}{34714}
\saveTG{谨}{34715}
\saveTG{𫍫}{34717}
\saveTG{乧}{34717}
\saveTG{谌}{34718}
\saveTG{讷}{34727}
\saveTG{勆}{34727}
\saveTG{}{34731}
\saveTG{}{34731}
\saveTG{𫍜}{34731}
\saveTG{装}{34732}
\saveTG{𩜥}{34732}
\saveTG{𫍦}{34732}
\saveTG{𧚛}{34732}
\saveTG{}{34735}
\saveTG{诗}{34741}
\saveTG{𪧼}{34743}
\saveTG{𫍲}{34747}
\saveTG{诐}{34747}
\saveTG{诂}{34760}
\saveTG{诸}{34760}
\saveTG{诘}{34761}
\saveTG{诰}{34761}
\saveTG{诺}{34764}
\saveTG{𤁃}{34772}
\saveTG{𦈳}{34775}
\saveTG{㪳}{34777}
\saveTG{谟}{34784}
\saveTG{读}{34784}
\saveTG{诙}{34789}
\saveTG{𠣛}{34790}
\saveTG{谍}{34794}
\saveTG{谋}{34794}
\saveTG{头}{34800}
\saveTG{𨆻}{34802}
\saveTG{𤂀}{34804}
\saveTG{䢮}{34806}
\saveTG{䢱}{34808}
\saveTG{𢻟}{34847}
\saveTG{𤣘}{34864}
\saveTG{䅴}{34903}
\saveTG{㭍}{34904}
\saveTG{染}{34904}
\saveTG{柒}{34904}
\saveTG{𣼑}{34904}
\saveTG{𪳼}{34905}
\saveTG{𥡣}{34915}
\saveTG{𥠸}{34917}
\saveTG{𣚩}{34946}
\saveTG{𥠯}{34962}
\saveTG{𣒐}{34990}
\saveTG{䅷}{34994}
\saveTG{汼}{35100}
\saveTG{汫}{35100}
\saveTG{沣}{35100}
\saveTG{𥂵}{35102}
\saveTG{𣴓}{35102}
\saveTG{堻}{35104}
\saveTG{洩}{35106}
\saveTG{𣳑}{35106}
\saveTG{𣵐}{35106}
\saveTG{𣷱}{35106}
\saveTG{冲}{35106}
\saveTG{沖}{35106}
\saveTG{浺}{35106}
\saveTG{㴢}{35106}
\saveTG{𣳪}{35106}
\saveTG{𣶴}{35106}
\saveTG{㳞}{35106}
\saveTG{㳏}{35106}
\saveTG{津}{35107}
\saveTG{𣹕}{35107}
\saveTG{𣸁}{35107}
\saveTG{冿}{35107}
\saveTG{𪷎}{35108}
\saveTG{泩}{35110}
\saveTG{𣺻}{35112}
\saveTG{浇}{35112}
\saveTG{濜}{35112}
\saveTG{𣸀}{35113}
\saveTG{澅}{35116}
\saveTG{𤁘}{35116}
\saveTG{𣻱}{35116}
\saveTG{沌}{35117}
\saveTG{𣲃}{35117}
\saveTG{𣼳}{35117}
\saveTG{汍}{35117}
\saveTG{𤅏}{35118}
\saveTG{澧}{35118}
\saveTG{㴋}{35124}
\saveTG{沛}{35127}
\saveTG{涄}{35127}
\saveTG{𣹥}{35127}
\saveTG{清}{35127}
\saveTG{凊}{35127}
\saveTG{淸}{35127}
\saveTG{潚}{35127}
\saveTG{𣲗}{35127}
\saveTG{𣽉}{35127}
\saveTG{𤅱}{35127}
\saveTG{𤄵}{35127}
\saveTG{𤀜}{35127}
\saveTG{𣼌}{35127}
\saveTG{𣷈}{35127}
\saveTG{𤁿}{35127}
\saveTG{沸}{35127}
\saveTG{泲}{35127}
\saveTG{泍}{35130}
\saveTG{漣}{35130}
\saveTG{㶋}{35131}
\saveTG{𤅧}{35131}
\saveTG{𪷴}{35131}
\saveTG{𣷼}{35131}
\saveTG{𤄁}{35131}
\saveTG{㶙}{35131}
\saveTG{灢}{35132}
\saveTG{濃}{35132}
\saveTG{}{35132}
\saveTG{𤃈}{35132}
\saveTG{𣷴}{35132}
\saveTG{潓}{35133}
\saveTG{漶}{35136}
\saveTG{𣷡}{35136}
\saveTG{䗝}{35136}
\saveTG{瀜}{35136}
\saveTG{浊}{35136}
\saveTG{𣶣}{35137}
\saveTG{瀢}{35138}
\saveTG{湕}{35140}
\saveTG{涛}{35140}
\saveTG{漙}{35143}
\saveTG{𣲹}{35144}
\saveTG{漊}{35144}
\saveTG{凄}{35144}
\saveTG{淒}{35144}
\saveTG{𣶻}{35146}
\saveTG{㴂}{35147}
\saveTG{𣶞}{35147}
\saveTG{𣽯}{35147}
\saveTG{𣲺}{35147}
\saveTG{溝}{35147}
\saveTG{𣳏}{35147}
\saveTG{𣽲}{35153}
\saveTG{𤁥}{35156}
\saveTG{淎}{35158}
\saveTG{浀}{35160}
\saveTG{油}{35160}
\saveTG{𣻲}{35161}
\saveTG{𠗐}{35161}
\saveTG{𣹴}{35161}
\saveTG{湱}{35162}
\saveTG{漕}{35166}
\saveTG{𣶤}{35168}
\saveTG{䲭}{35168}
\saveTG{潜}{35168}
\saveTG{湷}{35168}
\saveTG{𤅍}{35169}
\saveTG{𣺇}{35174}
\saveTG{泱}{35180}
\saveTG{泆}{35180}
\saveTG{決}{35180}
\saveTG{决}{35180}
\saveTG{𩽺}{35180}
\saveTG{浃}{35180}
\saveTG{淟}{35181}
\saveTG{𣶏}{35181}
\saveTG{𠗘}{35181}
\saveTG{𡩶}{35182}
\saveTG{渍}{35182}
\saveTG{洟}{35182}
\saveTG{溃}{35182}
\saveTG{𣺐}{35182}
\saveTG{𪶵}{35182}
\saveTG{㴣}{35182}
\saveTG{湊}{35184}
\saveTG{凑}{35184}
\saveTG{㵒}{35186}
\saveTG{𣿙}{35186}
\saveTG{𤄴}{35186}
\saveTG{濽}{35186}
\saveTG{濻}{35186}
\saveTG{漬}{35186}
\saveTG{潰}{35186}
\saveTG{𤄸}{35189}
\saveTG{𤄼}{35189}
\saveTG{沬}{35190}
\saveTG{涞}{35190}
\saveTG{洡}{35190}
\saveTG{沫}{35190}
\saveTG{洙}{35190}
\saveTG{𠖾}{35190}
\saveTG{洓}{35192}
\saveTG{溸}{35193}
\saveTG{𤃲}{35193}
\saveTG{溱}{35194}
\saveTG{𣵏}{35194}
\saveTG{凁}{35196}
\saveTG{涷}{35196}
\saveTG{𤁕}{35196}
\saveTG{凍}{35196}
\saveTG{湅}{35196}
\saveTG{涑}{35196}
\saveTG{溙}{35199}
\saveTG{𥘬}{35202}
\saveTG{𡧲}{35206}
\saveTG{𫋵}{35206}
\saveTG{神}{35206}
\saveTG{祌}{35206}
\saveTG{𧙟}{35206}
\saveTG{𥔻}{35206}
\saveTG{衶}{35206}
\saveTG{𧙻}{35207}
\saveTG{𥙽}{35208}
\saveTG{𥚄}{35216}
\saveTG{褹}{35217}
\saveTG{𫋹}{35217}
\saveTG{𧘸}{35217}
\saveTG{禮}{35218}
\saveTG{𧚗}{35227}
\saveTG{𧜹}{35227}
\saveTG{𤀷}{35227}
\saveTG{𧘧}{35227}
\saveTG{𧝄}{35227}
\saveTG{𧙂}{35227}
\saveTG{𦛇}{35227}
\saveTG{袆}{35227}
\saveTG{祎}{35227}
\saveTG{𧚫}{35227}
\saveTG{𧘟}{35227}
\saveTG{𩰾}{35227}
\saveTG{𨒘}{35227}
\saveTG{褳}{35230}
\saveTG{𧙄}{35230}
\saveTG{𧙽}{35231}
\saveTG{𠁎}{35231}
\saveTG{𤯼}{35231}
\saveTG{㽔}{35231}
\saveTG{裱}{35232}
\saveTG{禯}{35232}
\saveTG{襛}{35232}
\saveTG{𧟘}{35232}
\saveTG{𧞸}{35238}
\saveTG{𧜦}{35239}
\saveTG{祷}{35240}
\saveTG{𥛥}{35243}
\saveTG{䄛}{35244}
\saveTG{褄}{35244}
\saveTG{褸}{35244}
\saveTG{褠}{35247}
\saveTG{袡}{35247}
\saveTG{𧛔}{35257}
\saveTG{𣻈}{35258}
\saveTG{䄂}{35260}
\saveTG{袖}{35260}
\saveTG{𪽠}{35260}
\saveTG{䄚}{35265}
\saveTG{褿}{35266}
\saveTG{𧟔}{35269}
\saveTG{𧜧}{35277}
\saveTG{䄝}{35277}
\saveTG{䃿}{35280}
\saveTG{袂}{35280}
\saveTG{祑}{35280}
\saveTG{袟}{35280}
\saveTG{衭}{35280}
\saveTG{𧚨}{35281}
\saveTG{䄃}{35282}
\saveTG{𦑗}{35282}
\saveTG{𫌀}{35282}
\saveTG{𧙣}{35282}
\saveTG{䙌}{35282}
\saveTG{䘧}{35282}
\saveTG{}{35282}
\saveTG{}{35286}
\saveTG{襀}{35286}
\saveTG{𧝇}{35286}
\saveTG{𧞲}{35286}
\saveTG{䙡}{35286}
\saveTG{𧙕}{35290}
\saveTG{祩}{35290}
\saveTG{袜}{35290}
\saveTG{𫀇}{35290}
\saveTG{袾}{35290}
\saveTG{𥘯}{35290}
\saveTG{祙}{35290}
\saveTG{襋}{35292}
\saveTG{𧙞}{35292}
\saveTG{𧟜}{35295}
\saveTG{𥜕}{35296}
\saveTG{𧚏}{35296}
\saveTG{𫀘}{35299}
\saveTG{𨕫}{35300}
\saveTG{迚}{35300}
\saveTG{进}{35300}
\saveTG{連}{35300}
\saveTG{𨑞}{35301}
\saveTG{迍}{35301}
\saveTG{𨒉}{35302}
\saveTG{违}{35302}
\saveTG{𨕨}{35302}
\saveTG{𨑴}{35302}
\saveTG{𨓽}{35302}
\saveTG{䢌}{35302}
\saveTG{𨗗}{35302}
\saveTG{𨒷}{35303}
\saveTG{𨙥}{35303}
\saveTG{𨕟}{35304}
\saveTG{遘}{35304}
\saveTG{遱}{35304}
\saveTG{𨖇}{35304}
\saveTG{𨘒}{35304}
\saveTG{𨘠}{35304}
\saveTG{𨔸}{35305}
\saveTG{𫑀}{35305}
\saveTG{𨘪}{35305}
\saveTG{𨔂}{35305}
\saveTG{𨘇}{35305}
\saveTG{𨔐}{35305}
\saveTG{𨗐}{35306}
\saveTG{迧}{35306}
\saveTG{𨕌}{35306}
\saveTG{遭}{35306}
\saveTG{迪}{35306}
\saveTG{䢗}{35306}
\saveTG{𨒧}{35306}
\saveTG{𡨌}{35306}
\saveTG{𨙠}{35306}
\saveTG{𨘜}{35306}
\saveTG{𨔥}{35307}
\saveTG{䢖}{35307}
\saveTG{𨕛}{35307}
\saveTG{遣}{35307}
\saveTG{遗}{35308}
\saveTG{𫐧}{35308}
\saveTG{𨑣}{35308}
\saveTG{迭}{35308}
\saveTG{𨘧}{35308}
\saveTG{遺}{35308}
\saveTG{𨓰}{35308}
\saveTG{逮}{35309}
\saveTG{𨒪}{35309}
\saveTG{𨒲}{35309}
\saveTG{速}{35309}
\saveTG{𢣣}{35336}
\saveTG{𩺨}{35358}
\saveTG{𨘨}{35366}
\saveTG{𪖽}{35368}
\saveTG{𩹀}{35384}
\saveTG{𣼼}{35396}
\saveTG{𠓍}{35411}
\saveTG{𡢎}{35414}
\saveTG{𠒾}{35415}
\saveTG{𪟛}{35427}
\saveTG{𧊹}{35431}
\saveTG{𤯞}{35447}
\saveTG{𪯦}{35481}
\saveTG{𦧋}{35620}
\saveTG{𫌿}{35643}
\saveTG{㝬}{35680}
\saveTG{𣠒}{35692}
\saveTG{讲}{35700}
\saveTG{𦕡}{35705}
\saveTG{讲}{35705}
\saveTG{𫍢}{35717}
\saveTG{讳}{35727}
\saveTG{请}{35727}
\saveTG{}{35733}
\saveTG{谴}{35737}
\saveTG{诪}{35740}
\saveTG{诀}{35780}
\saveTG{诔}{35790}
\saveTG{诛}{35790}
\saveTG{𫍧}{35792}
\saveTG{𫍝}{35794}
\saveTG{谏}{35796}
\saveTG{𪏨}{35818}
\saveTG{𣑱}{35904}
\saveTG{𣷵}{35906}
\saveTG{𫃯}{35927}
\saveTG{𠾾}{36012}
\saveTG{覕}{36012}
\saveTG{覕}{36017}
\saveTG{𣲶}{36100}
\saveTG{𪷉}{36100}
\saveTG{𪶑}{36100}
\saveTG{𣶱}{36100}
\saveTG{𣷓}{36100}
\saveTG{𠖻}{36100}
\saveTG{𠗃}{36100}
\saveTG{涠}{36100}
\saveTG{𪞛}{36100}
\saveTG{汩}{36100}
\saveTG{泇}{36100}
\saveTG{𣶡}{36100}
\saveTG{漍}{36100}
\saveTG{涸}{36100}
\saveTG{洄}{36100}
\saveTG{溷}{36100}
\saveTG{涃}{36100}
\saveTG{泪}{36100}
\saveTG{汨}{36100}
\saveTG{泅}{36100}
\saveTG{洳}{36100}
\saveTG{泗}{36100}
\saveTG{潿}{36100}
\saveTG{}{36100}
\saveTG{湘}{36100}
\saveTG{洇}{36100}
\saveTG{沺}{36100}
\saveTG{㳱}{36100}
\saveTG{𣳲}{36100}
\saveTG{𣲷}{36100}
\saveTG{𣶦}{36100}
\saveTG{𣺀}{36100}
\saveTG{𣴺}{36100}
\saveTG{凅}{36100}
\saveTG{𪞜}{36102}
\saveTG{湐}{36102}
\saveTG{洎}{36102}
\saveTG{𣶎}{36102}
\saveTG{𣳦}{36102}
\saveTG{𣵁}{36102}
\saveTG{泊}{36102}
\saveTG{盪}{36102}
\saveTG{璗}{36103}
\saveTG{𡎺}{36104}
\saveTG{𡑑}{36104}
\saveTG{塣}{36104}
\saveTG{㴄}{36110}
\saveTG{𣵤}{36110}
\saveTG{泹}{36110}
\saveTG{𣿓}{36111}
\saveTG{瀙}{36112}
\saveTG{𣴟}{36112}
\saveTG{況}{36112}
\saveTG{况}{36112}
\saveTG{𣳄}{36112}
\saveTG{𣵀}{36112}
\saveTG{𣴡}{36112}
\saveTG{𣽭}{36112}
\saveTG{滉}{36112}
\saveTG{混}{36112}
\saveTG{温}{36112}
\saveTG{覜}{36112}
\saveTG{湿}{36112}
\saveTG{涀}{36112}
\saveTG{溫}{36112}
\saveTG{溾}{36113}
\saveTG{𣹽}{36114}
\saveTG{𡐀}{36114}
\saveTG{𣽿}{36114}
\saveTG{浧}{36114}
\saveTG{涅}{36114}
\saveTG{湟}{36114}
\saveTG{灅}{36114}
\saveTG{濹}{36114}
\saveTG{𣳤}{36114}
\saveTG{湦}{36115}
\saveTG{浬}{36115}
\saveTG{灈}{36115}
\saveTG{𪷑}{36115}
\saveTG{𠗔}{36115}
\saveTG{𤄷}{36115}
\saveTG{渑}{36116}
\saveTG{浥}{36117}
\saveTG{𤅻}{36117}
\saveTG{𩇟}{36117}
\saveTG{𧡢}{36117}
\saveTG{𤁣}{36117}
\saveTG{𣹮}{36117}
\saveTG{𣵿}{36117}
\saveTG{㵾}{36117}
\saveTG{𣸭}{36118}
\saveTG{淠}{36121}
\saveTG{瀱}{36121}
\saveTG{濞}{36121}
\saveTG{𣼬}{36121}
\saveTG{𤀰}{36122}
\saveTG{𣿃}{36127}
\saveTG{𣷢}{36127}
\saveTG{澚}{36127}
\saveTG{𣿘}{36127}
\saveTG{𤀀}{36127}
\saveTG{淿}{36127}
\saveTG{湡}{36127}
\saveTG{渭}{36127}
\saveTG{溻}{36127}
\saveTG{瀃}{36127}
\saveTG{湯}{36127}
\saveTG{澷}{36127}
\saveTG{涓}{36127}
\saveTG{渴}{36127}
\saveTG{湂}{36127}
\saveTG{涡}{36127}
\saveTG{𣽴}{36127}
\saveTG{濁}{36127}
\saveTG{𣽷}{36127}
\saveTG{𪞦}{36127}
\saveTG{𪶀}{36127}
\saveTG{渇}{36127}
\saveTG{𣽺}{36127}
\saveTG{𣻵}{36127}
\saveTG{𠗓}{36127}
\saveTG{𣼱}{36127}
\saveTG{𪶒}{36127}
\saveTG{𤁻}{36127}
\saveTG{𣸤}{36127}
\saveTG{漗}{36130}
\saveTG{㴧}{36130}
\saveTG{㴓}{36130}
\saveTG{𣹑}{36131}
\saveTG{𣺝}{36131}
\saveTG{潶}{36131}
\saveTG{𣺽}{36131}
\saveTG{𤃭}{36131}
\saveTG{𣽙}{36131}
\saveTG{𤃅}{36131}
\saveTG{𣽍}{36132}
\saveTG{𣹺}{36132}
\saveTG{𤁆}{36132}
\saveTG{𣿌}{36132}
\saveTG{潀}{36132}
\saveTG{渨}{36132}
\saveTG{𣾞}{36132}
\saveTG{𣽻}{36132}
\saveTG{𠗯}{36132}
\saveTG{㶞}{36132}
\saveTG{𣾿}{36132}
\saveTG{澴}{36132}
\saveTG{𣽁}{36132}
\saveTG{濕}{36133}
\saveTG{㶎}{36133}
\saveTG{𧑘}{36136}
\saveTG{漒}{36136}
\saveTG{𣶜}{36137}
\saveTG{𣳧}{36137}
\saveTG{渂}{36140}
\saveTG{𣳀}{36140}
\saveTG{渒}{36140}
\saveTG{𣿒}{36141}
\saveTG{凙}{36141}
\saveTG{澤}{36141}
\saveTG{淂}{36141}
\saveTG{湒}{36141}
\saveTG{𤀎}{36141}
\saveTG{𣽎}{36141}
\saveTG{㶠}{36141}
\saveTG{𣴢}{36142}
\saveTG{瀴}{36144}
\saveTG{𣺂}{36144}
\saveTG{𪶣}{36144}
\saveTG{𪷗}{36145}
\saveTG{𣴉}{36147}
\saveTG{漫}{36147}
\saveTG{𣳚}{36147}
\saveTG{㵊}{36147}
\saveTG{𣴪}{36147}
\saveTG{𣼺}{36147}
\saveTG{溭}{36147}
\saveTG{𠘥}{36148}
\saveTG{滜}{36148}
\saveTG{𠖹}{36150}
\saveTG{㳌}{36150}
\saveTG{𣼦}{36152}
\saveTG{𣹤}{36152}
\saveTG{㓖}{36154}
\saveTG{滭}{36154}
\saveTG{潬}{36156}
\saveTG{淐}{36160}
\saveTG{𣳮}{36160}
\saveTG{𣼞}{36160}
\saveTG{潪}{36160}
\saveTG{𪣧}{36160}
\saveTG{𤀅}{36161}
\saveTG{𤃵}{36161}
\saveTG{㵿}{36162}
\saveTG{𤀗}{36162}
\saveTG{𣹈}{36162}
\saveTG{𣽞}{36164}
\saveTG{濖}{36164}
\saveTG{濐}{36164}
\saveTG{澏}{36172}
\saveTG{浿}{36180}
\saveTG{𣲵}{36180}
\saveTG{湜}{36181}
\saveTG{潩}{36181}
\saveTG{浞}{36181}
\saveTG{𣾄}{36182}
\saveTG{𣾸}{36182}
\saveTG{𣴼}{36182}
\saveTG{𣷐}{36182}
\saveTG{涢}{36182}
\saveTG{𤀈}{36184}
\saveTG{𣵖}{36184}
\saveTG{𤅙}{36184}
\saveTG{𣵗}{36184}
\saveTG{𤅹}{36184}
\saveTG{㶔}{36184}
\saveTG{溴}{36184}
\saveTG{淏}{36184}
\saveTG{湨}{36184}
\saveTG{洖}{36184}
\saveTG{㵑}{36186}
\saveTG{溳}{36186}
\saveTG{𤀭}{36186}
\saveTG{㵋}{36186}
\saveTG{𠘉}{36190}
\saveTG{湶}{36192}
\saveTG{㵖}{36193}
\saveTG{漯}{36193}
\saveTG{㶟}{36193}
\saveTG{澡}{36194}
\saveTG{㴪}{36194}
\saveTG{𠘏}{36194}
\saveTG{湺}{36194}
\saveTG{淉}{36194}
\saveTG{𤁖}{36194}
\saveTG{㳭}{36194}
\saveTG{𤄗}{36194}
\saveTG{澋}{36196}
\saveTG{瀑}{36199}
\saveTG{𧛂}{36200}
\saveTG{䙟}{36200}
\saveTG{𧙊}{36200}
\saveTG{𧛃}{36200}
\saveTG{𩼀}{36200}
\saveTG{𥘗}{36200}
\saveTG{䄄}{36200}
\saveTG{祵}{36200}
\saveTG{裍}{36200}
\saveTG{袽}{36200}
\saveTG{衵}{36200}
\saveTG{𠰦}{36200}
\saveTG{裀}{36200}
\saveTG{祻}{36200}
\saveTG{𥙃}{36202}
\saveTG{𥚽}{36202}
\saveTG{袙}{36202}
\saveTG{袒}{36210}
\saveTG{𥘵}{36210}
\saveTG{𧟕}{36212}
\saveTG{𠖞}{36212}
\saveTG{𫀚}{36212}
\saveTG{褞}{36212}
\saveTG{𥚯}{36212}
\saveTG{視}{36212}
\saveTG{裩}{36212}
\saveTG{襬}{36212}
\saveTG{襯}{36212}
\saveTG{祝}{36212}
\saveTG{裎}{36214}
\saveTG{䄇}{36214}
\saveTG{䄓}{36214}
\saveTG{𧛟}{36215}
\saveTG{裡}{36215}
\saveTG{𫀓}{36215}
\saveTG{𥚃}{36215}
\saveTG{𧟌}{36215}
\saveTG{𧜴}{36217}
\saveTG{𧢞}{36217}
\saveTG{𧚷}{36217}
\saveTG{䚠}{36217}
\saveTG{𧡌}{36217}
\saveTG{𥚜}{36217}
\saveTG{𥚛}{36217}
\saveTG{𧠧}{36217}
\saveTG{𫌓}{36218}
\saveTG{𥚈}{36221}
\saveTG{襣}{36221}
\saveTG{𥙙}{36226}
\saveTG{𥜊}{36227}
\saveTG{𣹁}{36227}
\saveTG{𦤢}{36227}
\saveTG{襡}{36227}
\saveTG{褐}{36227}
\saveTG{裐}{36227}
\saveTG{禓}{36227}
\saveTG{禢}{36227}
\saveTG{𪛎}{36227}
\saveTG{𧝁}{36227}
\saveTG{裼}{36227}
\saveTG{𫀒}{36227}
\saveTG{祸}{36227}
\saveTG{𫌅}{36227}
\saveTG{禤}{36227}
\saveTG{褟}{36227}
\saveTG{禑}{36227}
\saveTG{𥚝}{36227}
\saveTG{禗}{36230}
\saveTG{𧜢}{36230}
\saveTG{䙓}{36231}
\saveTG{𪶛}{36232}
\saveTG{𧛚}{36232}
\saveTG{𥚸}{36232}
\saveTG{𥜏}{36232}
\saveTG{䙨}{36232}
\saveTG{𥛸}{36233}
\saveTG{襁}{36236}
\saveTG{裨}{36240}
\saveTG{禆}{36240}
\saveTG{襗}{36241}
\saveTG{𥜃}{36241}
\saveTG{𥚶}{36242}
\saveTG{禝}{36247}
\saveTG{𫀡}{36247}
\saveTG{𧜞}{36247}
\saveTG{𥜵}{36247}
\saveTG{𧟝}{36247}
\saveTG{襊}{36247}
\saveTG{䘥}{36250}
\saveTG{襅}{36254}
\saveTG{𥛘}{36254}
\saveTG{襌}{36256}
\saveTG{禪}{36256}
\saveTG{裮}{36260}
\saveTG{𥚕}{36260}
\saveTG{𧞭}{36260}
\saveTG{𧙋}{36280}
\saveTG{禩}{36281}
\saveTG{𧝀}{36281}
\saveTG{禔}{36281}
\saveTG{褆}{36281}
\saveTG{𧚖}{36282}
\saveTG{祦}{36284}
\saveTG{𧜘}{36286}
\saveTG{𧝂}{36286}
\saveTG{𥛍}{36286}
\saveTG{𥛧}{36293}
\saveTG{𧟊}{36294}
\saveTG{祼}{36294}
\saveTG{襙}{36294}
\saveTG{裸}{36294}
\saveTG{褓}{36294}
\saveTG{祼}{36295}
\saveTG{襮}{36299}
\saveTG{𨒚}{36300}
\saveTG{}{36300}
\saveTG{逥}{36300}
\saveTG{迫}{36300}
\saveTG{迦}{36300}
\saveTG{迴}{36300}
\saveTG{𨒊}{36300}
\saveTG{𨕚}{36300}
\saveTG{𨑕}{36300}
\saveTG{𨑨}{36300}
\saveTG{𨘵}{36301}
\saveTG{𨓅}{36301}
\saveTG{𨓦}{36301}
\saveTG{𨕲}{36301}
\saveTG{𨕒}{36301}
\saveTG{逞}{36301}
\saveTG{遑}{36301}
\saveTG{邏}{36301}
\saveTG{邈}{36301}
\saveTG{𨓲}{36301}
\saveTG{暹}{36301}
\saveTG{𨓆}{36302}
\saveTG{遢}{36302}
\saveTG{遇}{36302}
\saveTG{逻}{36302}
\saveTG{𨔗}{36302}
\saveTG{逿}{36302}
\saveTG{邊}{36302}
\saveTG{𨓺}{36302}
\saveTG{𨔣}{36302}
\saveTG{𨕔}{36302}
\saveTG{𨔍}{36302}
\saveTG{𨖙}{36302}
\saveTG{逷}{36302}
\saveTG{𨘢}{36302}
\saveTG{遏}{36302}
\saveTG{𨗻}{36302}
\saveTG{𨕈}{36302}
\saveTG{𨔒}{36302}
\saveTG{𨙘}{36302}
\saveTG{遌}{36302}
\saveTG{𨘣}{36303}
\saveTG{𨗫}{36303}
\saveTG{𨔃}{36303}
\saveTG{還}{36303}
\saveTG{𨕻}{36303}
\saveTG{𨘖}{36303}
\saveTG{𨔺}{36304}
\saveTG{䢲}{36304}
\saveTG{𨕓}{36304}
\saveTG{𨘊}{36304}
\saveTG{遻}{36304}
\saveTG{𨕬}{36304}
\saveTG{𨔫}{36304}
\saveTG{𨓤}{36304}
\saveTG{𨕣}{36304}
\saveTG{𨓙}{36304}
\saveTG{𨒇}{36305}
\saveTG{𨗓}{36306}
\saveTG{𨖞}{36306}
\saveTG{𨗏}{36306}
\saveTG{𨘐}{36306}
\saveTG{𫑓}{36306}
\saveTG{邉}{36306}
\saveTG{𨔹}{36306}
\saveTG{𨔓}{36308}
\saveTG{䢙}{36308}
\saveTG{𦌻}{36308}
\saveTG{𨒅}{36308}
\saveTG{𫑊}{36308}
\saveTG{遈}{36308}
\saveTG{遝}{36309}
\saveTG{𨗈}{36309}
\saveTG{𨔯}{36309}
\saveTG{𣺯}{36312}
\saveTG{𩷣}{36314}
\saveTG{𩽤}{36317}
\saveTG{𩼟}{36327}
\saveTG{𢡂}{36332}
\saveTG{𤄺}{36332}
\saveTG{𤄢}{36335}
\saveTG{𢀎}{36337}
\saveTG{𡖦}{36337}
\saveTG{𤒕}{36339}
\saveTG{㵽}{36360}
\saveTG{𨓐}{36362}
\saveTG{𦤮}{36384}
\saveTG{𠡈}{36427}
\saveTG{𣂌}{36427}
\saveTG{𡢈}{36442}
\saveTG{𤂃}{36447}
\saveTG{𪳨}{36494}
\saveTG{𢴳}{36502}
\saveTG{𣋀}{36515}
\saveTG{𧡡}{36517}
\saveTG{𪮏}{36530}
\saveTG{𣉣}{36600}
\saveTG{覾}{36612}
\saveTG{𩳼}{36617}
\saveTG{𪾺}{36700}
\saveTG{𪽣}{36710}
\saveTG{郒}{36717}
\saveTG{䙾}{36717}
\saveTG{𧡃}{36717}
\saveTG{𡫨}{36717}
\saveTG{𫍩}{36727}
\saveTG{谓}{36727}
\saveTG{谔}{36727}
\saveTG{谒}{36727}
\saveTG{𫍰}{36730}
\saveTG{谡}{36747}
\saveTG{谩}{36747}
\saveTG{𫑯}{36748}
\saveTG{识}{36780}
\saveTG{}{36781}
\saveTG{误}{36784}
\saveTG{课}{36794}
\saveTG{㒵}{36800}
\saveTG{燙}{36809}
\saveTG{覭}{36812}
\saveTG{䚔}{36812}
\saveTG{𩴱}{36817}
\saveTG{𧠽}{36817}
\saveTG{𡘻}{36840}
\saveTG{昶}{36900}
\saveTG{𣓬}{36904}
\saveTG{𣕃}{36927}
\saveTG{𣝋}{36939}
\saveTG{𪳷}{36942}
\saveTG{门}{37001}
\saveTG{冖}{37001}
\saveTG{㓁}{37008}
\saveTG{𡀦}{37012}
\saveTG{冖}{37020}
\saveTG{𠨘}{37020}
\saveTG{邲}{37027}
\saveTG{鴓}{37027}
\saveTG{闩}{37101}
\saveTG{闫}{37101}
\saveTG{盗}{37102}
\saveTG{𪾛}{37102}
\saveTG{盜}{37102}
\saveTG{冝}{37102}
\saveTG{阖}{37102}
\saveTG{䀀}{37102}
\saveTG{䀊}{37102}
\saveTG{𥂖}{37102}
\saveTG{𪾒}{37102}
\saveTG{盕}{37102}
\saveTG{}{37103}
\saveTG{𨸅}{37104}
\saveTG{}{37104}
\saveTG{闰}{37104}
\saveTG{闺}{37104}
\saveTG{垐}{37104}
\saveTG{埿}{37104}
\saveTG{䦷}{37104}
\saveTG{}{37104}
\saveTG{塱}{37104}
\saveTG{𠖚}{37106}
\saveTG{洢}{37107}
\saveTG{𨸌}{37108}
\saveTG{𠖃}{37108}
\saveTG{𨯳}{37109}
\saveTG{𨦮}{37109}
\saveTG{𨧒}{37109}
\saveTG{鑿}{37109}
\saveTG{𣳽}{37110}
\saveTG{𣶘}{37110}
\saveTG{𤅜}{37110}
\saveTG{𩘒}{37110}
\saveTG{汎}{37110}
\saveTG{渢}{37110}
\saveTG{沨}{37110}
\saveTG{洬}{37110}
\saveTG{汛}{37110}
\saveTG{㵯}{37110}
\saveTG{涩}{37111}
\saveTG{㵎}{37111}
\saveTG{㲸}{37111}
\saveTG{瀣}{37111}
\saveTG{澀}{37111}
\saveTG{𣶝}{37112}
\saveTG{𣹖}{37112}
\saveTG{㳑}{37112}
\saveTG{𠗅}{37112}
\saveTG{𣴻}{37112}
\saveTG{𤁍}{37112}
\saveTG{𣹣}{37112}
\saveTG{𣶬}{37112}
\saveTG{𣿫}{37112}
\saveTG{㵬}{37112}
\saveTG{𤂚}{37112}
\saveTG{𣿣}{37112}
\saveTG{𤄮}{37112}
\saveTG{𣹃}{37112}
\saveTG{𣿍}{37112}
\saveTG{𣻭}{37112}
\saveTG{𤅚}{37112}
\saveTG{𤃗}{37112}
\saveTG{𠗠}{37112}
\saveTG{氾}{37112}
\saveTG{湼}{37112}
\saveTG{灚}{37112}
\saveTG{泾}{37112}
\saveTG{沮}{37112}
\saveTG{浼}{37112}
\saveTG{凂}{37112}
\saveTG{泥}{37112}
\saveTG{沑}{37112}
\saveTG{瀊}{37112}
\saveTG{泡}{37112}
\saveTG{洈}{37112}
\saveTG{洫}{37112}
\saveTG{溋}{37112}
\saveTG{淣}{37112}
\saveTG{洆}{37113}
\saveTG{瀺}{37113}
\saveTG{𣶳}{37114}
\saveTG{𣻹}{37114}
\saveTG{𣲾}{37114}
\saveTG{㳗}{37114}
\saveTG{𠗻}{37114}
\saveTG{渥}{37114}
\saveTG{浘}{37114}
\saveTG{湰}{37115}
\saveTG{漋}{37115}
\saveTG{濯}{37115}
\saveTG{𩶲}{37116}
\saveTG{渔}{37116}
\saveTG{𣲩}{37117}
\saveTG{𣾚}{37117}
\saveTG{㲹}{37117}
\saveTG{𣲆}{37117}
\saveTG{𣽳}{37117}
\saveTG{𨝗}{37117}
\saveTG{𠖰}{37117}
\saveTG{𪜚}{37117}
\saveTG{𪷍}{37117}
\saveTG{𪷃}{37117}
\saveTG{𣸦}{37117}
\saveTG{𣹠}{37117}
\saveTG{𤀧}{37117}
\saveTG{𠗟}{37117}
\saveTG{𣳞}{37117}
\saveTG{𠖵}{37117}
\saveTG{𣲻}{37117}
\saveTG{𪷾}{37117}
\saveTG{𪶪}{37117}
\saveTG{𤄻}{37117}
\saveTG{𤅌}{37117}
\saveTG{𣹋}{37117}
\saveTG{𣿆}{37117}
\saveTG{𪞖}{37117}
\saveTG{𩇝}{37117}
\saveTG{沉}{37117}
\saveTG{淝}{37117}
\saveTG{濪}{37117}
\saveTG{泦}{37117}
\saveTG{澠}{37117}
\saveTG{汜}{37117}
\saveTG{滟}{37117}
\saveTG{灔}{37117}
\saveTG{灧}{37117}
\saveTG{𪷊}{37118}
\saveTG{𣶅}{37118}
\saveTG{洆}{37119}
\saveTG{汮}{37120}
\saveTG{澜}{37120}
\saveTG{瀾}{37120}
\saveTG{泖}{37120}
\saveTG{潣}{37120}
\saveTG{淜}{37120}
\saveTG{沏}{37120}
\saveTG{汋}{37120}
\saveTG{润}{37120}
\saveTG{潤}{37120}
\saveTG{泀}{37120}
\saveTG{淘}{37120}
\saveTG{汐}{37120}
\saveTG{澖}{37120}
\saveTG{洶}{37120}
\saveTG{洵}{37120}
\saveTG{翧}{37120}
\saveTG{灁}{37120}
\saveTG{湖}{37120}
\saveTG{溯}{37120}
\saveTG{㵍}{37120}
\saveTG{涧}{37120}
\saveTG{澗}{37120}
\saveTG{泂}{37120}
\saveTG{浻}{37120}
\saveTG{淗}{37120}
\saveTG{灍}{37120}
\saveTG{渹}{37120}
\saveTG{𣼾}{37120}
\saveTG{𣴵}{37120}
\saveTG{𣹜}{37120}
\saveTG{𤂗}{37120}
\saveTG{灛}{37120}
\saveTG{潮}{37120}
\saveTG{凋}{37120}
\saveTG{淍}{37120}
\saveTG{汈}{37120}
\saveTG{洞}{37120}
\saveTG{沕}{37120}
\saveTG{沟}{37120}
\saveTG{泃}{37120}
\saveTG{𣷥}{37121}
\saveTG{㲽}{37121}
\saveTG{㳉}{37121}
\saveTG{𣷠}{37121}
\saveTG{𣼐}{37121}
\saveTG{𣶯}{37121}
\saveTG{𤀵}{37121}
\saveTG{𣼽}{37121}
\saveTG{𪸄}{37121}
\saveTG{𤄨}{37121}
\saveTG{𣻫}{37121}
\saveTG{𣼫}{37121}
\saveTG{𣳱}{37121}
\saveTG{𪷭}{37121}
\saveTG{𤂶}{37121}
\saveTG{𣾬}{37121}
\saveTG{𤂕}{37121}
\saveTG{𤃷}{37121}
\saveTG{𣶈}{37121}
\saveTG{𣱼}{37121}
\saveTG{𪶏}{37121}
\saveTG{汿}{37122}
\saveTG{漻}{37122}
\saveTG{𤄇}{37122}
\saveTG{㶒}{37122}
\saveTG{𤀄}{37122}
\saveTG{𠗽}{37122}
\saveTG{𣺫}{37122}
\saveTG{𣵓}{37122}
\saveTG{㶀}{37122}
\saveTG{𪷡}{37122}
\saveTG{𠖳}{37122}
\saveTG{𣿸}{37122}
\saveTG{𤂷}{37123}
\saveTG{𤁝}{37123}
\saveTG{汐}{37123}
\saveTG{𣾺}{37124}
\saveTG{𣳔}{37124}
\saveTG{𣲢}{37124}
\saveTG{𤀳}{37124}
\saveTG{𣲿}{37124}
\saveTG{𪸌}{37124}
\saveTG{𤅾}{37124}
\saveTG{𤁐}{37124}
\saveTG{𠖿}{37124}
\saveTG{𣼶}{37126}
\saveTG{㓊}{37126}
\saveTG{𣵦}{37126}
\saveTG{𤂂}{37126}
\saveTG{𠖷}{37126}
\saveTG{𤄃}{37126}
\saveTG{𤁵}{37126}
\saveTG{𣷣}{37127}
\saveTG{𣶀}{37127}
\saveTG{𪷔}{37127}
\saveTG{䳐}{37127}
\saveTG{写}{37127}
\saveTG{鴵}{37127}
\saveTG{鸂}{37127}
\saveTG{㵼}{37127}
\saveTG{澙}{37127}
\saveTG{潟}{37127}
\saveTG{溩}{37127}
\saveTG{阘}{37127}
\saveTG{汤}{37127}
\saveTG{邺}{37127}
\saveTG{溺}{37127}
\saveTG{漏}{37127}
\saveTG{漷}{37127}
\saveTG{鵼}{37127}
\saveTG{潏}{37127}
\saveTG{鴻}{37127}
\saveTG{鸿}{37127}
\saveTG{渦}{37127}
\saveTG{滑}{37127}
\saveTG{𠘁}{37127}
\saveTG{冯}{37127}
\saveTG{涌}{37127}
\saveTG{㴮}{37127}
\saveTG{𣹯}{37127}
\saveTG{𪶻}{37127}
\saveTG{𣺅}{37127}
\saveTG{𣳜}{37127}
\saveTG{𣴗}{37127}
\saveTG{𪆕}{37127}
\saveTG{𠗄}{37127}
\saveTG{䳦}{37127}
\saveTG{𣺘}{37127}
\saveTG{𪅯}{37127}
\saveTG{𪞠}{37127}
\saveTG{𪄄}{37127}
\saveTG{㴆}{37127}
\saveTG{𠖜}{37127}
\saveTG{𣵷}{37127}
\saveTG{𣹙}{37127}
\saveTG{𣱽}{37127}
\saveTG{㶉}{37127}
\saveTG{𤀝}{37127}
\saveTG{𣱴}{37127}
\saveTG{𣴙}{37127}
\saveTG{𣻩}{37127}
\saveTG{𪶘}{37127}
\saveTG{闯}{37127}
\saveTG{𣻗}{37127}
\saveTG{㴫}{37127}
\saveTG{𣴀}{37127}
\saveTG{𣾭}{37127}
\saveTG{𣽔}{37127}
\saveTG{𤅝}{37127}
\saveTG{𨛯}{37127}
\saveTG{𪵹}{37127}
\saveTG{𠗺}{37127}
\saveTG{𠗷}{37127}
\saveTG{𫑜}{37127}
\saveTG{𤅼}{37127}
\saveTG{𤅽}{37127}
\saveTG{𪈸}{37127}
\saveTG{𪃡}{37127}
\saveTG{𪃽}{37127}
\saveTG{𪂔}{37127}
\saveTG{𪈯}{37127}
\saveTG{𪄠}{37127}
\saveTG{𪆟}{37127}
\saveTG{𤀙}{37127}
\saveTG{𪆸}{37127}
\saveTG{𣼣}{37127}
\saveTG{𤂎}{37127}
\saveTG{𪁠}{37127}
\saveTG{𣻍}{37127}
\saveTG{𤄄}{37127}
\saveTG{𪅍}{37127}
\saveTG{𪆵}{37127}
\saveTG{𪄣}{37127}
\saveTG{𪃕}{37127}
\saveTG{𩾯}{37127}
\saveTG{𪄰}{37127}
\saveTG{𣺷}{37127}
\saveTG{𠖭}{37127}
\saveTG{灟}{37127}
\saveTG{湧}{37127}
\saveTG{瀥}{37127}
\saveTG{湑}{37127}
\saveTG{滫}{37127}
\saveTG{潃}{37127}
\saveTG{泻}{37127}
\saveTG{㴸}{37128}
\saveTG{𠖧}{37131}
\saveTG{𣴦}{37131}
\saveTG{𤂏}{37131}
\saveTG{𪷨}{37131}
\saveTG{𤀠}{37131}
\saveTG{𧔧}{37131}
\saveTG{𧔲}{37131}
\saveTG{𧖅}{37131}
\saveTG{𠖪}{37131}
\saveTG{𣼈}{37131}
\saveTG{㶖}{37131}
\saveTG{𣾎}{37131}
\saveTG{𣾹}{37131}
\saveTG{凞}{37131}
\saveTG{𠘕}{37131}
\saveTG{𣹔}{37132}
\saveTG{𣻢}{37132}
\saveTG{𤀊}{37132}
\saveTG{𣼹}{37132}
\saveTG{㳮}{37132}
\saveTG{𣵌}{37132}
\saveTG{潈}{37132}
\saveTG{潒}{37132}
\saveTG{淴}{37132}
\saveTG{溕}{37132}
\saveTG{涊}{37132}
\saveTG{湪}{37132}
\saveTG{泿}{37132}
\saveTG{𤅛}{37132}
\saveTG{𤀴}{37132}
\saveTG{𣺵}{37132}
\saveTG{濄}{37132}
\saveTG{𤅁}{37132}
\saveTG{𣺹}{37132}
\saveTG{㴍}{37132}
\saveTG{𠗋}{37132}
\saveTG{𣻨}{37132}
\saveTG{𤀓}{37132}
\saveTG{㵵}{37132}
\saveTG{𤂻}{37132}
\saveTG{𣻇}{37133}
\saveTG{泈}{37133}
\saveTG{浕}{37133}
\saveTG{𪶳}{37133}
\saveTG{𪶨}{37134}
\saveTG{漨}{37135}
\saveTG{𪷦}{37135}
\saveTG{溞}{37136}
\saveTG{漁}{37136}
\saveTG{䗬}{37136}
\saveTG{螀}{37136}
\saveTG{闽}{37136}
\saveTG{蠭}{37136}
\saveTG{㴔}{37137}
\saveTG{𧏴}{37137}
\saveTG{𤅩}{37137}
\saveTG{𤁠}{37137}
\saveTG{𤂿}{37138}
\saveTG{𤅮}{37138}
\saveTG{𣽓}{37138}
\saveTG{淑}{37140}
\saveTG{汷}{37140}
\saveTG{溆}{37140}
\saveTG{洀}{37140}
\saveTG{汉}{37140}
\saveTG{浫}{37141}
\saveTG{𣹲}{37141}
\saveTG{𡋋}{37141}
\saveTG{𣲌}{37141}
\saveTG{将}{37142}
\saveTG{𣷽}{37142}
\saveTG{𤁌}{37142}
\saveTG{𣴬}{37142}
\saveTG{汊}{37143}
\saveTG{𣴁}{37143}
\saveTG{𤀯}{37143}
\saveTG{㳩}{37144}
\saveTG{𣹚}{37144}
\saveTG{𠘢}{37144}
\saveTG{𣲥}{37144}
\saveTG{𤃯}{37144}
\saveTG{𠘇}{37144}
\saveTG{𣼨}{37146}
\saveTG{潯}{37146}
\saveTG{𣹂}{37147}
\saveTG{㶅}{37147}
\saveTG{𣻑}{37147}
\saveTG{𣴂}{37147}
\saveTG{𣺲}{37147}
\saveTG{𤁛}{37147}
\saveTG{𣸎}{37147}
\saveTG{冺}{37147}
\saveTG{𠬧}{37147}
\saveTG{𣷕}{37147}
\saveTG{溵}{37147}
\saveTG{浔}{37147}
\saveTG{溊}{37147}
\saveTG{溲}{37147}
\saveTG{汓}{37147}
\saveTG{泯}{37147}
\saveTG{沒}{37147}
\saveTG{没}{37147}
\saveTG{浸}{37147}
\saveTG{汲}{37147}
\saveTG{瀫}{37147}
\saveTG{𣻋}{37147}
\saveTG{瀔}{37147}
\saveTG{濲}{37147}
\saveTG{澱}{37147}
\saveTG{涰}{37147}
\saveTG{潺}{37147}
\saveTG{𠘒}{37147}
\saveTG{𣴫}{37147}
\saveTG{𣹰}{37147}
\saveTG{𣷭}{37147}
\saveTG{𣴄}{37147}
\saveTG{𣷺}{37147}
\saveTG{㓎}{37147}
\saveTG{𠘈}{37147}
\saveTG{㲁}{37147}
\saveTG{𣹏}{37147}
\saveTG{濢}{37148}
\saveTG{𫔵}{37148}
\saveTG{𣲋}{37150}
\saveTG{澥}{37152}
\saveTG{渾}{37152}
\saveTG{𤯦}{37152}
\saveTG{阈}{37153}
\saveTG{浑}{37154}
\saveTG{浲}{37154}
\saveTG{泽}{37154}
\saveTG{洚}{37154}
\saveTG{𪞝}{37154}
\saveTG{𣳷}{37155}
\saveTG{𣵢}{37155}
\saveTG{𣸙}{37155}
\saveTG{𣴅}{37157}
\saveTG{净}{37157}
\saveTG{瀞}{37157}
\saveTG{浄}{37157}
\saveTG{瀈}{37157}
\saveTG{漽}{37159}
\saveTG{𤄧}{37161}
\saveTG{滘}{37161}
\saveTG{澹}{37161}
\saveTG{濶}{37161}
\saveTG{澛}{37161}
\saveTG{沿}{37161}
\saveTG{𣽃}{37161}
\saveTG{𣿠}{37161}
\saveTG{𤄍}{37162}
\saveTG{溜}{37162}
\saveTG{洺}{37162}
\saveTG{㳷}{37162}
\saveTG{漝}{37162}
\saveTG{沼}{37162}
\saveTG{𣹒}{37162}
\saveTG{㴘}{37162}
\saveTG{𣸬}{37162}
\saveTG{𪷰}{37162}
\saveTG{𪶎}{37162}
\saveTG{𣷤}{37162}
\saveTG{𣹢}{37163}
\saveTG{瀂}{37163}
\saveTG{𠗴}{37163}
\saveTG{𣿧}{37164}
\saveTG{𠖠}{37164}
\saveTG{𠗂}{37164}
\saveTG{湣}{37164}
\saveTG{涺}{37164}
\saveTG{潞}{37164}
\saveTG{潳}{37164}
\saveTG{洛}{37164}
\saveTG{阔}{37164}
\saveTG{𣷍}{37167}
\saveTG{𣻥}{37167}
\saveTG{湄}{37167}
\saveTG{涒}{37167}
\saveTG{𪵲}{37170}
\saveTG{𪫆}{37170}
\saveTG{涵}{37172}
\saveTG{𣿲}{37172}
\saveTG{淈}{37172}
\saveTG{𣵊}{37172}
\saveTG{㴠}{37172}
\saveTG{𠗙}{37172}
\saveTG{𤁓}{37172}
\saveTG{𣼇}{37172}
\saveTG{𣲫}{37174}
\saveTG{𣳨}{37177}
\saveTG{淊}{37177}
\saveTG{溟}{37180}
\saveTG{汣}{37180}
\saveTG{凕}{37180}
\saveTG{𣼗}{37180}
\saveTG{𣹎}{37181}
\saveTG{㵰}{37181}
\saveTG{凝}{37181}
\saveTG{瀷}{37181}
\saveTG{潠}{37181}
\saveTG{𤃚}{37181}
\saveTG{𠖮}{37181}
\saveTG{𤄕}{37181}
\saveTG{𣾠}{37182}
\saveTG{𣳩}{37182}
\saveTG{㶑}{37182}
\saveTG{𪶿}{37182}
\saveTG{𤁤}{37182}
\saveTG{𣳟}{37182}
\saveTG{𣽟}{37182}
\saveTG{𣶛}{37182}
\saveTG{𣵶}{37182}
\saveTG{𣼴}{37182}
\saveTG{漱}{37182}
\saveTG{濑}{37182}
\saveTG{𣣑}{37182}
\saveTG{次}{37182}
\saveTG{𣸧}{37182}
\saveTG{𠘂}{37182}
\saveTG{𤀔}{37182}
\saveTG{𣹱}{37182}
\saveTG{}{37182}
\saveTG{𪶓}{37184}
\saveTG{𣵫}{37184}
\saveTG{澳}{37184}
\saveTG{涣}{37184}
\saveTG{渙}{37184}
\saveTG{𣸲}{37184}
\saveTG{澬}{37186}
\saveTG{𪷈}{37186}
\saveTG{𤅂}{37186}
\saveTG{灨}{37186}
\saveTG{𤂐}{37186}
\saveTG{𠘝}{37186}
\saveTG{濥}{37186}
\saveTG{瀨}{37186}
\saveTG{沢}{37187}
\saveTG{𪞚}{37187}
\saveTG{㵤}{37188}
\saveTG{𤄚}{37189}
\saveTG{𠖣}{37189}
\saveTG{𣶋}{37189}
\saveTG{𠖦}{37191}
\saveTG{𥛛}{37191}
\saveTG{漈}{37191}
\saveTG{沵}{37192}
\saveTG{淥}{37192}
\saveTG{潨}{37192}
\saveTG{潔}{37193}
\saveTG{㓗}{37193}
\saveTG{𪷬}{37193}
\saveTG{滌}{37194}
\saveTG{涤}{37194}
\saveTG{𣳼}{37194}
\saveTG{𣻏}{37194}
\saveTG{𪶽}{37194}
\saveTG{澯}{37194}
\saveTG{渘}{37194}
\saveTG{深}{37194}
\saveTG{渌}{37199}
\saveTG{𤀼}{37199}
\saveTG{𤂱}{37199}
\saveTG{𧘺}{37210}
\saveTG{𧙱}{37210}
\saveTG{𥘆}{37211}
\saveTG{𥛈}{37211}
\saveTG{𥙸}{37211}
\saveTG{袓}{37212}
\saveTG{阅}{37212}
\saveTG{阋}{37212}
\saveTG{视}{37212}
\saveTG{裇}{37212}
\saveTG{祪}{37212}
\saveTG{袍}{37212}
\saveTG{𫀉}{37212}
\saveTG{𫀗}{37212}
\saveTG{祖}{37212}
\saveTG{𥙓}{37212}
\saveTG{𥙥}{37212}
\saveTG{𧛐}{37214}
\saveTG{𧙔}{37214}
\saveTG{㓂}{37214}
\saveTG{𥚼}{37214}
\saveTG{𧜶}{37214}
\saveTG{𧘥}{37214}
\saveTG{𦨺}{37214}
\saveTG{冠}{37214}
\saveTG{𥜔}{37215}
\saveTG{𥜤}{37215}
\saveTG{𥜡}{37215}
\saveTG{𫔴}{37215}
\saveTG{𫌋}{37216}
\saveTG{闶}{37217}
\saveTG{戺}{37217}
\saveTG{祀}{37217}
\saveTG{䘦}{37217}
\saveTG{𠕶}{37217}
\saveTG{𠕻}{37217}
\saveTG{𩲛}{37217}
\saveTG{𪎲}{37217}
\saveTG{𤜾}{37217}
\saveTG{𥘌}{37217}
\saveTG{䘛}{37217}
\saveTG{𧚇}{37217}
\saveTG{𥜐}{37217}
\saveTG{𥙵}{37217}
\saveTG{𠕿}{37217}
\saveTG{𧚟}{37217}
\saveTG{𠖍}{37217}
\saveTG{冗}{37217}
\saveTG{䘽}{37217}
\saveTG{𠕵}{37217}
\saveTG{𠖥}{37217}
\saveTG{𦫪}{37217}
\saveTG{𪞏}{37217}
\saveTG{𪕗}{37217}
\saveTG{𧞫}{37218}
\saveTG{𧞀}{37218}
\saveTG{𥙧}{37219}
\saveTG{祤}{37220}
\saveTG{禂}{37220}
\saveTG{祠}{37220}
\saveTG{裯}{37220}
\saveTG{𠨙}{37220}
\saveTG{𫀈}{37220}
\saveTG{𪵵}{37220}
\saveTG{礿}{37220}
\saveTG{裪}{37220}
\saveTG{祹}{37220}
\saveTG{翩}{37220}
\saveTG{襴}{37220}
\saveTG{襕}{37220}
\saveTG{袀}{37220}
\saveTG{襉}{37220}
\saveTG{初}{37220}
\saveTG{袧}{37220}
\saveTG{裥}{37220}
\saveTG{襇}{37220}
\saveTG{𧛞}{37221}
\saveTG{䙀}{37221}
\saveTG{𥚬}{37221}
\saveTG{𥘉}{37221}
\saveTG{𠕍}{37221}
\saveTG{𥘩}{37221}
\saveTG{𥘨}{37222}
\saveTG{𡪇}{37222}
\saveTG{𧘤}{37223}
\saveTG{𧘑}{37223}
\saveTG{𫔲}{37224}
\saveTG{𫔳}{37224}
\saveTG{𥘝}{37224}
\saveTG{𥘮}{37226}
\saveTG{𧙥}{37226}
\saveTG{䘩}{37226}
\saveTG{𧙹}{37226}
\saveTG{𥙣}{37226}
\saveTG{𧙈}{37226}
\saveTG{𢩚}{37226}
\saveTG{𥚟}{37226}
\saveTG{𪅢}{37227}
\saveTG{𣍟}{37227}
\saveTG{𦙎}{37227}
\saveTG{𪆢}{37227}
\saveTG{𪇧}{37227}
\saveTG{闹}{37227}
\saveTG{𫛢}{37227}
\saveTG{𥙯}{37227}
\saveTG{𧜛}{37227}
\saveTG{𧘏}{37227}
\saveTG{䘟}{37227}
\saveTG{𧘌}{37227}
\saveTG{䘞}{37227}
\saveTG{𧚔}{37227}
\saveTG{鸋}{37227}
\saveTG{𥚍}{37227}
\saveTG{𥛇}{37227}
\saveTG{幂}{37227}
\saveTG{𥚩}{37227}
\saveTG{𧜔}{37227}
\saveTG{鶣}{37227}
\saveTG{袳}{37227}
\saveTG{鵍}{37227}
\saveTG{鶺}{37227}
\saveTG{鹡}{37227}
\saveTG{鵳}{37227}
\saveTG{肎}{37227}
\saveTG{祃}{37227}
\saveTG{冪}{37227}
\saveTG{祁}{37227}
\saveTG{礽}{37227}
\saveTG{鹇}{37227}
\saveTG{鴧}{37227}
\saveTG{鵷}{37227}
\saveTG{鹓}{37227}
\saveTG{𧞄}{37227}
\saveTG{𢁟}{37227}
\saveTG{𫛻}{37227}
\saveTG{𨸂}{37227}
\saveTG{𫀕}{37227}
\saveTG{𧞦}{37227}
\saveTG{𧝌}{37227}
\saveTG{𧜭}{37227}
\saveTG{䦸}{37227}
\saveTG{𫌈}{37227}
\saveTG{䙱}{37227}
\saveTG{𨸎}{37227}
\saveTG{𠖅}{37227}
\saveTG{𧛸}{37227}
\saveTG{𧝃}{37227}
\saveTG{𦘫}{37227}
\saveTG{𠖄}{37227}
\saveTG{鼏}{37227}
\saveTG{𥛯}{37227}
\saveTG{𥛔}{37227}
\saveTG{𧜣}{37227}
\saveTG{𧝰}{37227}
\saveTG{𧜾}{37227}
\saveTG{𧜓}{37227}
\saveTG{𪇶}{37227}
\saveTG{䴁}{37227}
\saveTG{禍}{37227}
\saveTG{𧘈}{37227}
\saveTG{𫋽}{37227}
\saveTG{𫑭}{37227}
\saveTG{𫔮}{37227}
\saveTG{𫔶}{37229}
\saveTG{}{37230}
\saveTG{𫔱}{37230}
\saveTG{𫔷}{37231}
\saveTG{𧚈}{37231}
\saveTG{𠘤}{37232}
\saveTG{𧝮}{37232}
\saveTG{裉}{37232}
\saveTG{褖}{37232}
\saveTG{禒}{37232}
\saveTG{襐}{37232}
\saveTG{冢}{37232}
\saveTG{㓀}{37232}
\saveTG{𠖔}{37232}
\saveTG{𥚆}{37232}
\saveTG{𣳆}{37232}
\saveTG{𠕽}{37232}
\saveTG{𧛤}{37232}
\saveTG{阛}{37232}
\saveTG{䙤}{37232}
\saveTG{𧟒}{37232}
\saveTG{阏}{37233}
\saveTG{褪}{37233}
\saveTG{𠖁}{37234}
\saveTG{𧞘}{37235}
\saveTG{𧞽}{37235}
\saveTG{䙜}{37235}
\saveTG{𧜨}{37235}
\saveTG{𥛝}{37235}
\saveTG{𥚔}{37240}
\saveTG{袇}{37240}
\saveTG{裑}{37240}
\saveTG{𧛺}{37241}
\saveTG{闭}{37241}
\saveTG{衩}{37243}
\saveTG{𠖌}{37243}
\saveTG{𧜱}{37244}
\saveTG{𧘒}{37244}
\saveTG{𦨤}{37244}
\saveTG{𥘘}{37244}
\saveTG{𨸇}{37245}
\saveTG{𧘖}{37245}
\saveTG{襑}{37246}
\saveTG{䄌}{37247}
\saveTG{𥛀}{37247}
\saveTG{祋}{37247}
\saveTG{祲}{37247}
\saveTG{𥙈}{37247}
\saveTG{𧜁}{37247}
\saveTG{𧘣}{37247}
\saveTG{𠖊}{37247}
\saveTG{衱}{37247}
\saveTG{𥘜}{37247}
\saveTG{𥘓}{37247}
\saveTG{𧚥}{37247}
\saveTG{䘲}{37247}
\saveTG{裰}{37247}
\saveTG{𥛆}{37247}
\saveTG{𣿡}{37247}
\saveTG{𠖢}{37248}
\saveTG{衻}{37250}
\saveTG{𫌔}{37251}
\saveTG{䙙}{37252}
\saveTG{𥛹}{37252}
\saveTG{襷}{37252}
\saveTG{褌}{37252}
\saveTG{禈}{37252}
\saveTG{𧞊}{37252}
\saveTG{阀}{37253}
\saveTG{𧚋}{37254}
\saveTG{𫋷}{37254}
\saveTG{袶}{37254}
\saveTG{裈}{37254}
\saveTG{褌}{37256}
\saveTG{𣄈}{37256}
\saveTG{裈}{37257}
\saveTG{襜}{37261}
\saveTG{𥚣}{37261}
\saveTG{𧞟}{37261}
\saveTG{䄡}{37261}
\saveTG{𧛕}{37262}
\saveTG{𧜄}{37262}
\saveTG{𧝨}{37262}
\saveTG{褶}{37262}
\saveTG{袑}{37262}
\saveTG{祒}{37262}
\saveTG{𥛅}{37262}
\saveTG{𥩍}{37263}
\saveTG{𧝴}{37264}
\saveTG{𥚁}{37264}
\saveTG{裾}{37264}
\saveTG{袼}{37264}
\saveTG{𥚑}{37264}
\saveTG{𧛰}{37267}
\saveTG{裙}{37267}
\saveTG{𠠭}{37267}
\saveTG{𥚵}{37267}
\saveTG{𥚭}{37268}
\saveTG{䘶}{37272}
\saveTG{䘿}{37272}
\saveTG{𧚧}{37277}
\saveTG{𥙢}{37277}
\saveTG{𥘰}{37277}
\saveTG{𧜀}{37280}
\saveTG{䄙}{37280}
\saveTG{𧞏}{37281}
\saveTG{襈}{37281}
\saveTG{𧛴}{37281}
\saveTG{𧝽}{37282}
\saveTG{𣽌}{37282}
\saveTG{㳄}{37282}
\saveTG{𧝝}{37282}
\saveTG{𣢖}{37282}
\saveTG{𧞨}{37282}
\saveTG{𥜌}{37284}
\saveTG{䙈}{37284}
\saveTG{襖}{37284}
\saveTG{禊}{37284}
\saveTG{褉}{37284}
\saveTG{𥚦}{37284}
\saveTG{䄤}{37286}
\saveTG{襰}{37286}
\saveTG{𧟈}{37289}
\saveTG{䄞}{37291}
\saveTG{祿}{37292}
\saveTG{祢}{37292}
\saveTG{袮}{37292}
\saveTG{𧙤}{37294}
\saveTG{褬}{37294}
\saveTG{𥛋}{37294}
\saveTG{𥙨}{37294}
\saveTG{禄}{37299}
\saveTG{𥛿}{37299}
\saveTG{䘵}{37299}
\saveTG{迉}{37300}
\saveTG{𨑖}{37301}
\saveTG{𫐶}{37301}
\saveTG{𨓜}{37301}
\saveTG{𨙙}{37301}
\saveTG{𨗱}{37301}
\saveTG{迡}{37301}
\saveTG{𨒁}{37301}
\saveTG{𨑓}{37301}
\saveTG{迅}{37301}
\saveTG{逸}{37301}
\saveTG{𨑽}{37301}
\saveTG{𨓟}{37301}
\saveTG{𨒈}{37301}
\saveTG{𫐰}{37301}
\saveTG{䢐}{37301}
\saveTG{䢰}{37301}
\saveTG{𨖽}{37301}
\saveTG{𨔚}{37301}
\saveTG{𨑙}{37301}
\saveTG{迳}{37301}
\saveTG{𨒖}{37302}
\saveTG{𨓝}{37302}
\saveTG{𨘌}{37302}
\saveTG{𨒝}{37302}
\saveTG{𨕙}{37302}
\saveTG{𨘮}{37302}
\saveTG{𨗊}{37302}
\saveTG{𨖖}{37302}
\saveTG{𨓔}{37302}
\saveTG{𨒡}{37302}
\saveTG{𨖨}{37302}
\saveTG{𨑚}{37302}
\saveTG{𨔪}{37302}
\saveTG{𨔡}{37302}
\saveTG{𨘯}{37302}
\saveTG{𨖶}{37302}
\saveTG{䢧}{37302}
\saveTG{迵}{37302}
\saveTG{𨑥}{37302}
\saveTG{𨓪}{37302}
\saveTG{𨗽}{37302}
\saveTG{𨓴}{37302}
\saveTG{𨒦}{37302}
\saveTG{𨒗}{37302}
\saveTG{𨑏}{37302}
\saveTG{𨖅}{37302}
\saveTG{辺}{37302}
\saveTG{過}{37302}
\saveTG{𨗔}{37302}
\saveTG{迥}{37302}
\saveTG{逈}{37302}
\saveTG{辽}{37302}
\saveTG{辸}{37302}
\saveTG{遡}{37302}
\saveTG{通}{37302}
\saveTG{迌}{37302}
\saveTG{迿}{37302}
\saveTG{迻}{37302}
\saveTG{𨒀}{37302}
\saveTG{𨘂}{37302}
\saveTG{迎}{37302}
\saveTG{𨖂}{37302}
\saveTG{𨑩}{37302}
\saveTG{遹}{37302}
\saveTG{週}{37302}
\saveTG{𨒵}{37302}
\saveTG{𨑦}{37302}
\saveTG{䢜}{37302}
\saveTG{𨙡}{37302}
\saveTG{𨙐}{37303}
\saveTG{𨗿}{37303}
\saveTG{𨙅}{37303}
\saveTG{𨔵}{37303}
\saveTG{𨒟}{37303}
\saveTG{𨔇}{37303}
\saveTG{𫐻}{37303}
\saveTG{退}{37303}
\saveTG{𫑂}{37304}
\saveTG{𨖔}{37304}
\saveTG{𨔧}{37304}
\saveTG{䢊}{37304}
\saveTG{𨙀}{37304}
\saveTG{遟}{37304}
\saveTG{𨗋}{37304}
\saveTG{𨖏}{37304}
\saveTG{𫐺}{37304}
\saveTG{𨕴}{37304}
\saveTG{𨓉}{37304}
\saveTG{𨕵}{37304}
\saveTG{𨗮}{37304}
\saveTG{遚}{37304}
\saveTG{逫}{37304}
\saveTG{遐}{37304}
\saveTG{𨕶}{37304}
\saveTG{𨕮}{37304}
\saveTG{﨤}{37304}
\saveTG{𨓭}{37304}
\saveTG{𨑱}{37304}
\saveTG{𨓠}{37305}
\saveTG{遲}{37305}
\saveTG{逢}{37305}
\saveTG{逄}{37305}
\saveTG{邂}{37305}
\saveTG{運}{37305}
\saveTG{遅}{37305}
\saveTG{𨘡}{37305}
\saveTG{𨙈}{37305}
\saveTG{遙}{37306}
\saveTG{迢}{37306}
\saveTG{𫐹}{37306}
\saveTG{𨔔}{37306}
\saveTG{𨖈}{37306}
\saveTG{䢣}{37306}
\saveTG{𨗧}{37306}
\saveTG{遛}{37306}
\saveTG{𨙃}{37307}
\saveTG{追}{37307}
\saveTG{𫑕}{37307}
\saveTG{䢛}{37307}
\saveTG{𨘽}{37307}
\saveTG{𨙒}{37308}
\saveTG{𨕖}{37308}
\saveTG{𨘕}{37308}
\saveTG{選}{37308}
\saveTG{遬}{37308}
\saveTG{闼}{37308}
\saveTG{遦}{37308}
\saveTG{迟}{37308}
\saveTG{𨒮}{37308}
\saveTG{𣤎}{37308}
\saveTG{𨕊}{37308}
\saveTG{𨖯}{37308}
\saveTG{𨕄}{37309}
\saveTG{𨓻}{37309}
\saveTG{逯}{37309}
\saveTG{邌}{37309}
\saveTG{𫐪}{37309}
\saveTG{邍}{37309}
\saveTG{迩}{37309}
\saveTG{鶐}{37327}
\saveTG{冩}{37327}
\saveTG{𨗷}{37327}
\saveTG{𪆹}{37327}
\saveTG{𪁲}{37327}
\saveTG{𪂁}{37327}
\saveTG{𪒍}{37331}
\saveTG{𤏁}{37331}
\saveTG{𪬂}{37331}
\saveTG{闷}{37331}
\saveTG{𪑪}{37331}
\saveTG{𢛜}{37331}
\saveTG{𢚞}{37332}
\saveTG{𢚑}{37332}
\saveTG{慂}{37332}
\saveTG{𠖓}{37332}
\saveTG{𢛣}{37334}
\saveTG{𩸧}{37336}
\saveTG{𩷞}{37336}
\saveTG{恣}{37338}
\saveTG{𣺆}{37344}
\saveTG{𣳠}{37357}
\saveTG{㵫}{37364}
\saveTG{𣢠}{37382}
\saveTG{㵣}{37382}
\saveTG{𨘭}{37399}
\saveTG{闵}{37400}
\saveTG{㕚}{37400}
\saveTG{𦔰}{37401}
\saveTG{闻}{37401}
\saveTG{𦕕}{37401}
\saveTG{闬}{37401}
\saveTG{罕}{37401}
\saveTG{𠖗}{37402}
\saveTG{𫔯}{37403}
\saveTG{姿}{37404}
\saveTG{𦋐}{37406}
\saveTG{𣺎}{37407}
\saveTG{𠦥}{37407}
\saveTG{𨸀}{37407}
\saveTG{𠦭}{37407}
\saveTG{𪎇}{37407}
\saveTG{阌}{37407}
\saveTG{𠕸}{37407}
\saveTG{𦉲}{37408}
\saveTG{冤}{37413}
\saveTG{𠖂}{37413}
\saveTG{𠕴}{37417}
\saveTG{𡤳}{37417}
\saveTG{𨝢}{37427}
\saveTG{𪂸}{37427}
\saveTG{𫛩}{37427}
\saveTG{䢿}{37427}
\saveTG{𠣟}{37427}
\saveTG{鴳}{37427}
\saveTG{𠖛}{37427}
\saveTG{𨜛}{37427}
\saveTG{𪄑}{37427}
\saveTG{𠡑}{37427}
\saveTG{𣹞}{37432}
\saveTG{𣁲}{37433}
\saveTG{𫔭}{37441}
\saveTG{𠕷}{37442}
\saveTG{𫔰}{37443}
\saveTG{𣁻}{37443}
\saveTG{𢍔}{37446}
\saveTG{𥦍}{37446}
\saveTG{𣪮}{37447}
\saveTG{𣪲}{37447}
\saveTG{𪞔}{37447}
\saveTG{冣}{37447}
\saveTG{阚}{37448}
\saveTG{阙}{37482}
\saveTG{𠖫}{37486}
\saveTG{𤙟}{37502}
\saveTG{军}{37504}
\saveTG{阐}{37506}
\saveTG{軍}{37506}
\saveTG{闸}{37506}
\saveTG{𪞑}{37506}
\saveTG{}{37507}
\saveTG{䦶}{37507}
\saveTG{𤘰}{37508}
\saveTG{𡫡}{37526}
\saveTG{𩧷}{37527}
\saveTG{郓}{37527}
\saveTG{𨟨}{37527}
\saveTG{𪁔}{37527}
\saveTG{鶤}{37527}
\saveTG{闱}{37527}
\saveTG{鄆}{37527}
\saveTG{𣲕}{37557}
\saveTG{𣣞}{37582}
\saveTG{𠕾}{37600}
\saveTG{訚}{37601}
\saveTG{问}{37601}
\saveTG{间}{37601}
\saveTG{𠸆}{37601}
\saveTG{𧮙}{37601}
\saveTG{阃}{37601}
\saveTG{}{37601}
\saveTG{𨸉}{37601}
\saveTG{𥒟}{37602}
\saveTG{阇}{37604}
\saveTG{𥋩}{37604}
\saveTG{𣇹}{37604}
\saveTG{𠸗}{37604}
\saveTG{醤}{37604}
\saveTG{阍}{37604}
\saveTG{𦊙}{37604}
\saveTG{酱}{37604}
\saveTG{阁}{37604}
\saveTG{阊}{37606}
\saveTG{闾}{37606}
\saveTG{冨}{37606}
\saveTG{𨸆}{37606}
\saveTG{咨}{37608}
\saveTG{𠖋}{37608}
\saveTG{𪡌}{37608}
\saveTG{𩘨}{37610}
\saveTG{𧭅}{37614}
\saveTG{㓃}{37617}
\saveTG{𠖆}{37617}
\saveTG{𠖏}{37621}
\saveTG{𫍋}{37621}
\saveTG{𫍏}{37627}
\saveTG{𫛕}{37627}
\saveTG{鶷}{37627}
\saveTG{𨜱}{37627}
\saveTG{𨝃}{37627}
\saveTG{𪃜}{37627}
\saveTG{𪆠}{37627}
\saveTG{𪃭}{37627}
\saveTG{𪃾}{37627}
\saveTG{𫍃}{37627}
\saveTG{𨜳}{37627}
\saveTG{𫍅}{37642}
\saveTG{𣪯}{37647}
\saveTG{𪠮}{37647}
\saveTG{㲅}{37647}
\saveTG{𠖐}{37661}
\saveTG{𡪞}{37664}
\saveTG{𥇠}{37664}
\saveTG{𦌛}{37668}
\saveTG{𣤄}{37682}
\saveTG{𣣟}{37682}
\saveTG{𣣶}{37682}
\saveTG{𨸁}{37710}
\saveTG{讽}{37710}
\saveTG{讯}{37710}
\saveTG{讥}{37710}
\saveTG{诅}{37712}
\saveTG{诡}{37712}
\saveTG{冟}{37712}
\saveTG{𨸃}{37715}
\saveTG{阉}{37716}
\saveTG{阄}{37716}
\saveTG{𤬦}{37717}
\saveTG{乲}{37717}
\saveTG{𫍠}{37717}
\saveTG{瓷}{37717}
\saveTG{𫜨}{37717}
\saveTG{记}{37717}
\saveTG{闿}{37717}
\saveTG{词}{37720}
\saveTG{翝}{37720}
\saveTG{谰}{37720}
\saveTG{朗}{37720}
\saveTG{诇}{37720}
\saveTG{调}{37720}
\saveTG{诩}{37720}
\saveTG{讱}{37720}
\saveTG{询}{37720}
\saveTG{𫆴}{37720}
\saveTG{㓪}{37721}
\saveTG{𫍬}{37721}
\saveTG{谬}{37722}
\saveTG{𫍣}{37726}
\saveTG{郎}{37727}
\saveTG{𨟧}{37727}
\saveTG{𪂨}{37727}
\saveTG{𪃖}{37727}
\saveTG{𨜌}{37727}
\saveTG{郞}{37727}
\saveTG{𪁜}{37727}
\saveTG{𪄋}{37727}
\saveTG{𪅮}{37727}
\saveTG{𪀥}{37727}
\saveTG{谞}{37727}
\saveTG{谲}{37727}
\saveTG{郎}{37727}
\saveTG{诵}{37727}
\saveTG{𫍪}{37731}
\saveTG{}{37732}
\saveTG{阛}{37732}
\saveTG{阆}{37732}
\saveTG{闳}{37732}
\saveTG{餈}{37732}
\saveTG{𨸄}{37732}
\saveTG{𪞓}{37732}
\saveTG{谗}{37733}
\saveTG{}{37737}
\saveTG{𠕺}{37738}
\saveTG{诹}{37740}
\saveTG{𣫞}{37747}
\saveTG{设}{37747}
\saveTG{诨}{37754}
\saveTG{译}{37754}
\saveTG{诤}{37757}
\saveTG{谵}{37761}
\saveTG{诏}{37762}
\saveTG{谘}{37768}
\saveTG{冚}{37772}
\saveTG{阎}{37777}
\saveTG{𦦹}{37777}
\saveTG{诌}{37777}
\saveTG{谄}{37777}
\saveTG{𣣨}{37782}
\saveTG{欴}{37782}
\saveTG{𨸊}{37785}
\saveTG{谀}{37787}
\saveTG{𥽿}{37794}
\saveTG{𦊇}{37800}
\saveTG{𠕼}{37800}
\saveTG{冥}{37800}
\saveTG{闪}{37801}
\saveTG{𠖕}{37801}
\saveTG{𠔸}{37801}
\saveTG{阗}{37801}
\saveTG{阓}{37802}
\saveTG{资}{37802}
\saveTG{阂}{37802}
\saveTG{𠕳}{37802}
\saveTG{𧻦}{37803}
\saveTG{𠖇}{37804}
\saveTG{𦊅}{37804}
\saveTG{阒}{37804}
\saveTG{阕}{37804}
\saveTG{𨸈}{37804}
\saveTG{𠕹}{37804}
\saveTG{𡗺}{37804}
\saveTG{奖}{37804}
\saveTG{資}{37806}
\saveTG{烫}{37809}
\saveTG{焈}{37809}
\saveTG{𤏷}{37809}
\saveTG{鼆}{37817}
\saveTG{䒌}{37817}
\saveTG{䎙}{37820}
\saveTG{𫅭}{37821}
\saveTG{𦐝}{37821}
\saveTG{䴐}{37827}
\saveTG{𫑫}{37827}
\saveTG{䴆}{37827}
\saveTG{𪇕}{37827}
\saveTG{鶟}{37827}
\saveTG{鄍}{37827}
\saveTG{𨞊}{37827}
\saveTG{𣪻}{37847}
\saveTG{𪹀}{37854}
\saveTG{𠖩}{37886}
\saveTG{㷭}{37893}
\saveTG{𤈾}{37898}
\saveTG{浆}{37902}
\saveTG{𠖀}{37902}
\saveTG{䋜}{37903}
\saveTG{𥹦}{37904}
\saveTG{糳}{37904}
\saveTG{秶}{37904}
\saveTG{栥}{37904}
\saveTG{闲}{37904}
\saveTG{罙}{37904}
\saveTG{𣑃}{37904}
\saveTG{桨}{37904}
\saveTG{𣓥}{37904}
\saveTG{𣝨}{37904}
\saveTG{𣎾}{37904}
\saveTG{冞}{37904}
\saveTG{㮾}{37904}
\saveTG{𣒿}{37904}
\saveTG{𪲁}{37904}
\saveTG{𥺜}{37904}
\saveTG{𥹭}{37904}
\saveTG{𪀳}{37904}
\saveTG{𥽦}{37904}
\saveTG{粢}{37904}
\saveTG{楶}{37904}
\saveTG{阑}{37906}
\saveTG{𠣺}{37917}
\saveTG{𣏝}{37917}
\saveTG{𡩰}{37920}
\saveTG{𠖟}{37920}
\saveTG{𪴞}{37921}
\saveTG{𨛱}{37927}
\saveTG{𪆂}{37927}
\saveTG{鶎}{37927}
\saveTG{鸈}{37927}
\saveTG{鄴}{37927}
\saveTG{𠖘}{37927}
\saveTG{𣫩}{37947}
\saveTG{㱉}{37982}
\saveTG{冧}{37994}
\saveTG{𠔂}{38000}
\saveTG{𣱿}{38100}
\saveTG{汃}{38100}
\saveTG{泤}{38100}
\saveTG{汄}{38100}
\saveTG{𣲄}{38101}
\saveTG{𥂋}{38102}
\saveTG{𥁳}{38102}
\saveTG{𥁿}{38102}
\saveTG{𥁣}{38102}
\saveTG{𡌂}{38104}
\saveTG{𣴞}{38104}
\saveTG{𡋇}{38104}
\saveTG{𡍼}{38104}
\saveTG{塗}{38104}
\saveTG{塰}{38104}
\saveTG{𣳁}{38110}
\saveTG{𠖽}{38111}
\saveTG{㳕}{38111}
\saveTG{𠖸}{38111}
\saveTG{泎}{38111}
\saveTG{湓}{38112}
\saveTG{濫}{38112}
\saveTG{滥}{38112}
\saveTG{沦}{38112}
\saveTG{沲}{38112}
\saveTG{灠}{38112}
\saveTG{湴}{38112}
\saveTG{𣴾}{38112}
\saveTG{𤃾}{38112}
\saveTG{𤅸}{38112}
\saveTG{溠}{38112}
\saveTG{㓆}{38112}
\saveTG{沧}{38112}
\saveTG{溢}{38112}
\saveTG{涚}{38112}
\saveTG{涗}{38112}
\saveTG{湤}{38112}
\saveTG{溬}{38113}
\saveTG{洤}{38114}
\saveTG{滗}{38114}
\saveTG{㓌}{38114}
\saveTG{𠗎}{38114}
\saveTG{𣷒}{38114}
\saveTG{𤀟}{38114}
\saveTG{𣻓}{38115}
\saveTG{潅}{38115}
\saveTG{𠘓}{38115}
\saveTG{𤂦}{38115}
\saveTG{𪷮}{38116}
\saveTG{滊}{38117}
\saveTG{𤅴}{38117}
\saveTG{𪸃}{38117}
\saveTG{𤂺}{38117}
\saveTG{㳾}{38117}
\saveTG{𪷄}{38117}
\saveTG{汔}{38117}
\saveTG{漧}{38117}
\saveTG{汽}{38117}
\saveTG{𣵸}{38117}
\saveTG{𣽆}{38117}
\saveTG{𣽩}{38117}
\saveTG{𣹪}{38117}
\saveTG{𣵺}{38117}
\saveTG{𨰲}{38117}
\saveTG{澨}{38118}
\saveTG{凎}{38119}
\saveTG{淦}{38119}
\saveTG{滏}{38119}
\saveTG{𪞪}{38119}
\saveTG{㵚}{38119}
\saveTG{𤁈}{38119}
\saveTG{𣺨}{38119}
\saveTG{𤅺}{38119}
\saveTG{𣲤}{38120}
\saveTG{渝}{38121}
\saveTG{瀭}{38121}
\saveTG{湔}{38121}
\saveTG{𤀥}{38121}
\saveTG{沴}{38122}
\saveTG{𣹗}{38122}
\saveTG{𠗰}{38122}
\saveTG{𤄙}{38124}
\saveTG{𪞘}{38127}
\saveTG{𪷇}{38127}
\saveTG{𣵴}{38127}
\saveTG{𤅲}{38127}
\saveTG{𠗣}{38127}
\saveTG{𤀉}{38127}
\saveTG{㵸}{38127}
\saveTG{汵}{38127}
\saveTG{漡}{38127}
\saveTG{𠘅}{38127}
\saveTG{𤁶}{38127}
\saveTG{𠖲}{38127}
\saveTG{𪵴}{38127}
\saveTG{𪶡}{38127}
\saveTG{𣸜}{38127}
\saveTG{𤄒}{38127}
\saveTG{㶕}{38127}
\saveTG{𣼕}{38127}
\saveTG{瀄}{38127}
\saveTG{洕}{38127}
\saveTG{瀹}{38127}
\saveTG{潝}{38127}
\saveTG{滃}{38127}
\saveTG{涕}{38127}
\saveTG{溣}{38127}
\saveTG{瀚}{38127}
\saveTG{淪}{38127}
\saveTG{汾}{38127}
\saveTG{𪞤}{38130}
\saveTG{𤀒}{38130}
\saveTG{𣿇}{38130}
\saveTG{𣻧}{38130}
\saveTG{溔}{38131}
\saveTG{潕}{38131}
\saveTG{𤂳}{38131}
\saveTG{𣼩}{38131}
\saveTG{㳂}{38131}
\saveTG{𣸮}{38131}
\saveTG{𣸊}{38131}
\saveTG{冷}{38132}
\saveTG{泠}{38132}
\saveTG{淞}{38132}
\saveTG{凇}{38132}
\saveTG{瀁}{38132}
\saveTG{滋}{38132}
\saveTG{浍}{38132}
\saveTG{飡}{38132}
\saveTG{湌}{38132}
\saveTG{𩝩}{38132}
\saveTG{𣻒}{38132}
\saveTG{𣶗}{38132}
\saveTG{𪷶}{38132}
\saveTG{𣻌}{38132}
\saveTG{㴚}{38132}
\saveTG{𣾶}{38132}
\saveTG{淰}{38132}
\saveTG{澻}{38133}
\saveTG{濨}{38133}
\saveTG{淤}{38133}
\saveTG{滺}{38134}
\saveTG{𣿪}{38136}
\saveTG{𠗳}{38137}
\saveTG{溓}{38137}
\saveTG{𤅄}{38137}
\saveTG{𤅃}{38137}
\saveTG{𤁚}{38140}
\saveTG{㵟}{38140}
\saveTG{激}{38140}
\saveTG{𣶌}{38140}
\saveTG{𣿊}{38140}
\saveTG{𪷁}{38140}
\saveTG{𣻪}{38140}
\saveTG{𣵑}{38140}
\saveTG{瀓}{38140}
\saveTG{𤀂}{38140}
\saveTG{溦}{38140}
\saveTG{𪞫}{38140}
\saveTG{漵}{38140}
\saveTG{滧}{38140}
\saveTG{潄}{38140}
\saveTG{潵}{38140}
\saveTG{潎}{38140}
\saveTG{瀲}{38140}
\saveTG{潋}{38140}
\saveTG{潡}{38140}
\saveTG{漖}{38140}
\saveTG{滸}{38140}
\saveTG{汻}{38140}
\saveTG{浒}{38140}
\saveTG{澉}{38140}
\saveTG{浟}{38140}
\saveTG{濣}{38140}
\saveTG{澂}{38140}
\saveTG{澈}{38140}
\saveTG{滶}{38140}
\saveTG{𠘜}{38140}
\saveTG{𠖴}{38140}
\saveTG{𣷫}{38140}
\saveTG{㳇}{38140}
\saveTG{𤁲}{38140}
\saveTG{㳊}{38140}
\saveTG{澣}{38141}
\saveTG{洴}{38141}
\saveTG{𣷟}{38141}
\saveTG{𤅕}{38143}
\saveTG{𤀤}{38144}
\saveTG{㵺}{38145}
\saveTG{㶘}{38146}
\saveTG{澊}{38146}
\saveTG{渰}{38146}
\saveTG{游}{38147}
\saveTG{澓}{38147}
\saveTG{𣸪}{38147}
\saveTG{𤀡}{38147}
\saveTG{𣸯}{38147}
\saveTG{𤂇}{38147}
\saveTG{㳺}{38147}
\saveTG{𣹶}{38147}
\saveTG{𠘄}{38148}
\saveTG{潷}{38150}
\saveTG{𪷧}{38151}
\saveTG{𣿨}{38151}
\saveTG{㶍}{38151}
\saveTG{洋}{38151}
\saveTG{𣿭}{38155}
\saveTG{㵲}{38157}
\saveTG{海}{38157}
\saveTG{冾}{38161}
\saveTG{潽}{38161}
\saveTG{洽}{38161}
\saveTG{湁}{38161}
\saveTG{㵛}{38161}
\saveTG{𤅔}{38161}
\saveTG{渞}{38162}
\saveTG{𤃏}{38162}
\saveTG{浛}{38162}
\saveTG{𣸄}{38164}
\saveTG{𣻦}{38164}
\saveTG{涻}{38164}
\saveTG{湭}{38164}
\saveTG{𣿥}{38164}
\saveTG{瀶}{38166}
\saveTG{潧}{38166}
\saveTG{𣻙}{38166}
\saveTG{𠘎}{38166}
\saveTG{澮}{38166}
\saveTG{滄}{38167}
\saveTG{凔}{38167}
\saveTG{𠗖}{38168}
\saveTG{浴}{38168}
\saveTG{𣽛}{38168}
\saveTG{𤄫}{38169}
\saveTG{𤃳}{38169}
\saveTG{𣺱}{38174}
\saveTG{漩}{38181}
\saveTG{漎}{38181}
\saveTG{㶛}{38181}
\saveTG{㳬}{38182}
\saveTG{浂}{38184}
\saveTG{渼}{38184}
\saveTG{㵀}{38184}
\saveTG{𤄾}{38184}
\saveTG{𣽜}{38184}
\saveTG{𣾍}{38184}
\saveTG{𤄂}{38185}
\saveTG{𪶤}{38185}
\saveTG{㵅}{38186}
\saveTG{𪷪}{38186}
\saveTG{澰}{38186}
\saveTG{湵}{38187}
\saveTG{𣳅}{38190}
\saveTG{漾}{38192}
\saveTG{瀿}{38193}
\saveTG{𣻄}{38194}
\saveTG{涂}{38194}
\saveTG{凃}{38194}
\saveTG{滁}{38194}
\saveTG{𣷦}{38194}
\saveTG{𪶢}{38194}
\saveTG{𣽤}{38195}
\saveTG{𧘋}{38200}
\saveTG{𨖴}{38202}
\saveTG{祚}{38211}
\saveTG{𧚙}{38211}
\saveTG{𧙓}{38211}
\saveTG{𥘶}{38212}
\saveTG{褴}{38212}
\saveTG{裞}{38212}
\saveTG{袘}{38212}
\saveTG{襤}{38212}
\saveTG{𥜓}{38212}
\saveTG{𧜡}{38212}
\saveTG{褨}{38212}
\saveTG{𪰄}{38212}
\saveTG{祱}{38212}
\saveTG{𧛖}{38217}
\saveTG{𥙁}{38217}
\saveTG{𧜃}{38217}
\saveTG{𥙰}{38217}
\saveTG{𥙩}{38217}
\saveTG{𥚤}{38219}
\saveTG{裣}{38219}
\saveTG{䘳}{38219}
\saveTG{祄}{38220}
\saveTG{衸}{38220}
\saveTG{䄖}{38220}
\saveTG{𥙚}{38221}
\saveTG{𧟉}{38221}
\saveTG{褕}{38221}
\saveTG{𥘼}{38222}
\saveTG{袗}{38222}
\saveTG{𧟇}{38227}
\saveTG{𧜲}{38227}
\saveTG{祶}{38227}
\saveTG{衯}{38227}
\saveTG{𥛙}{38227}
\saveTG{黺}{38227}
\saveTG{衿}{38227}
\saveTG{禴}{38227}
\saveTG{𦢽}{38227}
\saveTG{𧛈}{38227}
\saveTG{𧛹}{38227}
\saveTG{𧞌}{38227}
\saveTG{䏿}{38227}
\saveTG{𫌙}{38227}
\saveTG{𥘞}{38227}
\saveTG{𫁛}{38227}
\saveTG{𧝅}{38227}
\saveTG{𥚗}{38227}
\saveTG{𧞶}{38227}
\saveTG{䄒}{38230}
\saveTG{𧛋}{38230}
\saveTG{𫌉}{38231}
\saveTG{𧛏}{38231}
\saveTG{䘴}{38231}
\saveTG{𫋻}{38231}
\saveTG{𧛛}{38231}
\saveTG{禚}{38231}
\saveTG{祣}{38232}
\saveTG{禌}{38232}
\saveTG{衳}{38232}
\saveTG{袊}{38232}
\saveTG{𠖝}{38232}
\saveTG{𣼁}{38232}
\saveTG{𣵂}{38232}
\saveTG{𧛄}{38232}
\saveTG{禭}{38233}
\saveTG{襚}{38233}
\saveTG{䙂}{38233}
\saveTG{𨗳}{38233}
\saveTG{𢼚}{38240}
\saveTG{𢼄}{38240}
\saveTG{襒}{38240}
\saveTG{𧝳}{38240}
\saveTG{啟}{38240}
\saveTG{𧝠}{38240}
\saveTG{𧝋}{38240}
\saveTG{𧘶}{38240}
\saveTG{𧛢}{38240}
\saveTG{𢽉}{38240}
\saveTG{𥘪}{38240}
\saveTG{𥙡}{38244}
\saveTG{𥚫}{38244}
\saveTG{𧟁}{38244}
\saveTG{𥜶}{38247}
\saveTG{複}{38247}
\saveTG{𧟓}{38248}
\saveTG{祥}{38251}
\saveTG{𧟖}{38251}
\saveTG{𧞆}{38251}
\saveTG{𥜉}{38251}
\saveTG{𧚀}{38254}
\saveTG{禅}{38256}
\saveTG{𫌑}{38261}
\saveTG{𧝡}{38261}
\saveTG{𥛶}{38261}
\saveTG{袷}{38261}
\saveTG{祫}{38261}
\saveTG{𧚼}{38263}
\saveTG{禉}{38264}
\saveTG{𧝙}{38265}
\saveTG{襘}{38266}
\saveTG{䙢}{38266}
\saveTG{𥜫}{38266}
\saveTG{禬}{38266}
\saveTG{𫀞}{38267}
\saveTG{𥛾}{38268}
\saveTG{𥙿}{38268}
\saveTG{裕}{38268}
\saveTG{𧛳}{38272}
\saveTG{𥛏}{38281}
\saveTG{𫀣}{38281}
\saveTG{䙕}{38282}
\saveTG{𧛒}{38284}
\saveTG{𥜁}{38284}
\saveTG{𧜕}{38284}
\saveTG{襝}{38286}
\saveTG{𧜬}{38289}
\saveTG{𥙄}{38290}
\saveTG{込}{38300}
\saveTG{𨕀}{38301}
\saveTG{𨓚}{38301}
\saveTG{遾}{38301}
\saveTG{𨘲}{38301}
\saveTG{迮}{38301}
\saveTG{迤}{38301}
\saveTG{迄}{38301}
\saveTG{𨖬}{38301}
\saveTG{𨑟}{38301}
\saveTG{𫐩}{38301}
\saveTG{递}{38302}
\saveTG{𨓕}{38302}
\saveTG{𨑸}{38302}
\saveTG{𨗼}{38302}
\saveTG{逾}{38302}
\saveTG{𨓧}{38302}
\saveTG{𫐮}{38302}
\saveTG{𨗹}{38302}
\saveTG{𨗌}{38302}
\saveTG{𨗤}{38302}
\saveTG{𨙢}{38302}
\saveTG{𨖃}{38302}
\saveTG{遂}{38303}
\saveTG{迕}{38303}
\saveTG{𨑪}{38303}
\saveTG{𫑅}{38303}
\saveTG{𨖌}{38303}
\saveTG{𨗎}{38303}
\saveTG{𨗅}{38303}
\saveTG{迸}{38304}
\saveTG{𨔼}{38304}
\saveTG{𨘋}{38304}
\saveTG{逰}{38304}
\saveTG{𨗕}{38304}
\saveTG{𥳪}{38304}
\saveTG{𨒾}{38304}
\saveTG{𫑎}{38304}
\saveTG{𨑷}{38304}
\saveTG{遨}{38304}
\saveTG{遵}{38304}
\saveTG{遊}{38304}
\saveTG{迕}{38304}
\saveTG{邀}{38304}
\saveTG{逆}{38304}
\saveTG{𨗯}{38304}
\saveTG{遫}{38304}
\saveTG{𫑍}{38304}
\saveTG{𫑑}{38305}
\saveTG{𨖷}{38305}
\saveTG{𨗵}{38305}
\saveTG{𨒫}{38305}
\saveTG{𨘉}{38305}
\saveTG{𨗾}{38305}
\saveTG{𨔏}{38306}
\saveTG{𨒔}{38306}
\saveTG{𨗥}{38306}
\saveTG{䢢}{38306}
\saveTG{逧}{38306}
\saveTG{𨗀}{38306}
\saveTG{道}{38306}
\saveTG{𨖩}{38306}
\saveTG{𨒄}{38306}
\saveTG{遒}{38306}
\saveTG{䢔}{38306}
\saveTG{𨗚}{38306}
\saveTG{𨖎}{38306}
\saveTG{送}{38308}
\saveTG{𨑹}{38308}
\saveTG{䢨}{38308}
\saveTG{𨗖}{38308}
\saveTG{𨓵}{38308}
\saveTG{𨘟}{38308}
\saveTG{𨗦}{38308}
\saveTG{䢭}{38308}
\saveTG{𨒛}{38309}
\saveTG{途}{38309}
\saveTG{𪆎}{38327}
\saveTG{𪄬}{38327}
\saveTG{𪄫}{38327}
\saveTG{𤏄}{38332}
\saveTG{𢞾}{38334}
\saveTG{𢜅}{38337}
\saveTG{𪫎}{38340}
\saveTG{𠗭}{38341}
\saveTG{導}{38343}
\saveTG{𪧸}{38346}
\saveTG{𨒣}{38347}
\saveTG{𫑉}{38392}
\saveTG{𤃬}{38431}
\saveTG{𢍡}{38446}
\saveTG{肇}{38507}
\saveTG{𦘦}{38574}
\saveTG{噵}{38603}
\saveTG{晵}{38604}
\saveTG{𨢯}{38604}
\saveTG{啓}{38604}
\saveTG{𠳛}{38604}
\saveTG{𣉯}{38609}
\saveTG{㘏}{38617}
\saveTG{𫍀}{38619}
\saveTG{𫍆}{38637}
\saveTG{𤾟}{38640}
\saveTG{𧯉}{38668}
\saveTG{豁}{38668}
\saveTG{认}{38700}
\saveTG{诈}{38711}
\saveTG{𫍟}{38712}
\saveTG{谥}{38712}
\saveTG{说}{38712}
\saveTG{论}{38712}
\saveTG{诠}{38714}
\saveTG{讫}{38717}
\saveTG{谕}{38721}
\saveTG{诊}{38722}
\saveTG{𫍛}{38727}
\saveTG{谫}{38727}
\saveTG{讼}{38732}
\saveTG{谂}{38732}
\saveTG{𩝓}{38732}
\saveTG{谦}{38737}
\saveTG{𢼊}{38740}
\saveTG{许}{38740}
\saveTG{𫍵}{38740}
\saveTG{𦫏}{38744}
\saveTG{详}{38751}
\saveTG{诲}{38757}
\saveTG{谱}{38761}
\saveTG{𡹘}{38772}
\saveTG{㽏}{38774}
\saveTG{㸂}{38809}
\saveTG{𤉰}{38809}
\saveTG{𨖺}{38830}
\saveTG{㪦}{38840}
\saveTG{綮}{38903}
\saveTG{𣓫}{38904}
\saveTG{䆃}{38904}
\saveTG{棨}{38904}
\saveTG{𪲜}{38904}
\saveTG{𣜦}{38904}
\saveTG{𥘦}{38940}
\saveTG{𣘼}{38944}
\saveTG{𥁲}{39102}
\saveTG{𡋷}{39104}
\saveTG{鲨}{39106}
\saveTG{洸}{39112}
\saveTG{淃}{39112}
\saveTG{瀅}{39113}
\saveTG{𣿩}{39114}
\saveTG{漟}{39114}
\saveTG{𣼎}{39115}
\saveTG{𣻖}{39117}
\saveTG{𣴕}{39117}
\saveTG{𣺼}{39117}
\saveTG{㴴}{39118}
\saveTG{𪶈}{39119}
\saveTG{灐}{39119}
\saveTG{𪶋}{39120}
\saveTG{𣺌}{39120}
\saveTG{𣻾}{39120}
\saveTG{𣴷}{39120}
\saveTG{𣶲}{39120}
\saveTG{渺}{39120}
\saveTG{沙}{39120}
\saveTG{𣸌}{39121}
\saveTG{𤁹}{39123}
\saveTG{𣷩}{39127}
\saveTG{𣻸}{39127}
\saveTG{淌}{39127}
\saveTG{消}{39127}
\saveTG{潲}{39127}
\saveTG{𣻘}{39127}
\saveTG{澇}{39127}
\saveTG{㴥}{39127}
\saveTG{𣺰}{39127}
\saveTG{𣺟}{39127}
\saveTG{𣲡}{39130}
\saveTG{灙}{39131}
\saveTG{𣽕}{39131}
\saveTG{㶈}{39131}
\saveTG{𧋊}{39136}
\saveTG{㵞}{39138}
\saveTG{𣷇}{39138}
\saveTG{㴹}{39139}
\saveTG{溇}{39144}
\saveTG{𪸂}{39147}
\saveTG{泮}{39150}
\saveTG{冸}{39150}
\saveTG{溿}{39150}
\saveTG{𣵲}{39152}
\saveTG{𣾦}{39152}
\saveTG{潾}{39159}
\saveTG{𤁽}{39161}
\saveTG{㸺}{39162}
\saveTG{𤃨}{39162}
\saveTG{渻}{39162}
\saveTG{瀯}{39166}
\saveTG{澢}{39166}
\saveTG{湫}{39180}
\saveTG{𣺸}{39180}
\saveTG{𪶆}{39180}
\saveTG{瀵}{39181}
\saveTG{𤀬}{39184}
\saveTG{𣴒}{39184}
\saveTG{𣹸}{39185}
\saveTG{溑}{39186}
\saveTG{濙}{39189}
\saveTG{淡}{39189}
\saveTG{濴}{39192}
\saveTG{𪧠}{39192}
\saveTG{瀠}{39193}
\saveTG{𤂵}{39193}
\saveTG{潫}{39193}
\saveTG{洣}{39194}
\saveTG{𣼃}{39194}
\saveTG{濚}{39194}
\saveTG{漛}{39199}
\saveTG{𡭜}{39200}
\saveTG{𥘷}{39212}
\saveTG{裷}{39212}
\saveTG{𥦲}{39214}
\saveTG{𥙪}{39215}
\saveTG{𥙑}{39217}
\saveTG{𢩂}{39217}
\saveTG{𧘡}{39220}
\saveTG{𥘤}{39220}
\saveTG{䘯}{39227}
\saveTG{𥙬}{39227}
\saveTG{䘷}{39227}
\saveTG{𡩗}{39228}
\saveTG{𥚿}{39244}
\saveTG{褛}{39244}
\saveTG{褝}{39250}
\saveTG{袢}{39250}
\saveTG{𧜎}{39252}
\saveTG{𥛷}{39257}
\saveTG{襠}{39266}
\saveTG{裆}{39277}
\saveTG{𥘙}{39280}
\saveTG{𤇷}{39280}
\saveTG{裧}{39289}
\saveTG{䆑}{39291}
\saveTG{𪓋}{39294}
\saveTG{逊}{39300}
\saveTG{辶}{39300}
\saveTG{𡮾}{39300}
\saveTG{𨒺}{39301}
\saveTG{𨙄}{39302}
\saveTG{逍}{39302}
\saveTG{逤}{39302}
\saveTG{𫐽}{39302}
\saveTG{𨖉}{39304}
\saveTG{𫐷}{39304}
\saveTG{𨔮}{39304}
\saveTG{𨕩}{39304}
\saveTG{䢠}{39304}
\saveTG{𨒃}{39305}
\saveTG{䢯}{39305}
\saveTG{遴}{39305}
\saveTG{𫑔}{39306}
\saveTG{𫐵}{39308}
\saveTG{𨗄}{39308}
\saveTG{𨑯}{39308}
\saveTG{逖}{39308}
\saveTG{迷}{39309}
\saveTG{𡮗}{39309}
\saveTG{𩣟}{39327}
\saveTG{鯊}{39336}
\saveTG{𣲧}{39380}
\saveTG{娑}{39404}
\saveTG{𡩷}{39442}
\saveTG{挲}{39502}
\saveTG{𥆝}{39602}
\saveTG{𩉀}{39602}
\saveTG{硰}{39602}
\saveTG{𡮞}{39620}
\saveTG{𣹇}{39621}
\saveTG{𫍂}{39662}
\saveTG{𡪥}{39712}
\saveTG{谠}{39712}
\saveTG{㲚}{39715}
\saveTG{䣉}{39717}
\saveTG{乷}{39717}
\saveTG{诮}{39727}
\saveTG{裟}{39732}
\saveTG{谜}{39739}
\saveTG{谈}{39789}
\saveTG{𨕜}{39803}
\saveTG{𦀟}{39903}
\saveTG{桬}{39904}
\saveTG{𥘽}{39950}
\saveTG{}{40000}
\saveTG{乂}{40000}
\saveTG{十}{40000}
\saveTG{𠂇}{40000}
\saveTG{㐅}{40000}
\saveTG{义}{40003}
\saveTG{𦥙}{40007}
\saveTG{𡰒}{40011}
\saveTG{𡰠}{40012}
\saveTG{尢}{40012}
\saveTG{冘}{40012}
\saveTG{㔫}{40012}
\saveTG{𡯃}{40014}
\saveTG{𨿙}{40015}
\saveTG{䧱}{40015}
\saveTG{䧵}{40015}
\saveTG{隿}{40015}
\saveTG{訄}{40016}
\saveTG{𡯳}{40016}
\saveTG{𡯁}{40017}
\saveTG{𠒍}{40017}
\saveTG{𢺧}{40017}
\saveTG{九}{40017}
\saveTG{力}{40027}
\saveTG{夸}{40027}
\saveTG{𦥕}{40027}
\saveTG{𠀁}{40027}
\saveTG{太}{40030}
\saveTG{𧥠}{40061}
\saveTG{𡭡}{40092}
\saveTG{圹}{40100}
\saveTG{士}{40100}
\saveTG{土}{40100}
\saveTG{夳}{40101}
\saveTG{𪩣}{40102}
\saveTG{𤜚}{40102}
\saveTG{𥁊}{40102}
\saveTG{䀁}{40102}
\saveTG{𡔳}{40102}
\saveTG{𥁈}{40102}
\saveTG{𥁋}{40102}
\saveTG{𡗨}{40102}
\saveTG{𡘃}{40102}
\saveTG{𡙞}{40102}
\saveTG{左}{40102}
\saveTG{㭗}{40102}
\saveTG{𥁓}{40102}
\saveTG{𡔲}{40102}
\saveTG{𡕂}{40102}
\saveTG{𡕈}{40102}
\saveTG{𡕍}{40102}
\saveTG{盍}{40102}
\saveTG{𡔾}{40102}
\saveTG{杢}{40102}
\saveTG{𥂙}{40102}
\saveTG{𥁅}{40102}
\saveTG{盇}{40102}
\saveTG{壶}{40102}
\saveTG{壸}{40102}
\saveTG{壼}{40102}
\saveTG{𡘟}{40102}
\saveTG{𢀡}{40102}
\saveTG{盔}{40102}
\saveTG{直}{40102}
\saveTG{㚗}{40102}
\saveTG{𤫒}{40102}
\saveTG{𡘄}{40102}
\saveTG{圡}{40103}
\saveTG{冭}{40103}
\saveTG{𪥓}{40104}
\saveTG{𡒙}{40104}
\saveTG{𡉂}{40104}
\saveTG{𦥂}{40104}
\saveTG{𦤼}{40104}
\saveTG{𤦃}{40104}
\saveTG{𡚛}{40104}
\saveTG{𡍊}{40104}
\saveTG{𡔦}{40104}
\saveTG{𡔪}{40104}
\saveTG{𡔼}{40104}
\saveTG{𡘫}{40104}
\saveTG{堏}{40104}
\saveTG{圭}{40104}
\saveTG{奎}{40104}
\saveTG{坴}{40104}
\saveTG{圶}{40104}
\saveTG{臺}{40104}
\saveTG{𡉄}{40104}
\saveTG{𦒼}{40104}
\saveTG{𡙔}{40104}
\saveTG{桽}{40104}
\saveTG{𡙃}{40104}
\saveTG{𡘇}{40104}
\saveTG{𡙙}{40104}
\saveTG{㙓}{40104}
\saveTG{𡌬}{40104}
\saveTG{𡔸}{40104}
\saveTG{𤨻}{40104}
\saveTG{𨤭}{40105}
\saveTG{壷}{40105}
\saveTG{𤯗}{40105}
\saveTG{𡔫}{40106}
\saveTG{𡋑}{40106}
\saveTG{𥄂}{40106}
\saveTG{𧉓}{40106}
\saveTG{查}{40106}
\saveTG{𡘰}{40106}
\saveTG{𡘍}{40106}
\saveTG{壺}{40107}
\saveTG{𡕄}{40108}
\saveTG{𣩉}{40108}
\saveTG{㘴}{40108}
\saveTG{𥩴}{40108}
\saveTG{𡗓}{40108}
\saveTG{壴}{40108}
\saveTG{壹}{40108}
\saveTG{𡘵}{40108}
\saveTG{𡔹}{40108}
\saveTG{𧯣}{40108}
\saveTG{𪲓}{40108}
\saveTG{𡕃}{40108}
\saveTG{𤫍}{40109}
\saveTG{䥅}{40109}
\saveTG{𨫹}{40109}
\saveTG{𣎶}{40109}
\saveTG{𨩓}{40109}
\saveTG{𡔭}{40111}
\saveTG{𣔉}{40111}
\saveTG{𩇩}{40111}
\saveTG{塶}{40112}
\saveTG{矗}{40112}
\saveTG{境}{40112}
\saveTG{𡗼}{40112}
\saveTG{𡐁}{40112}
\saveTG{𪣐}{40114}
\saveTG{𡊲}{40114}
\saveTG{㙵}{40114}
\saveTG{𡐎}{40114}
\saveTG{𡎻}{40114}
\saveTG{壵}{40114}
\saveTG{垚}{40114}
\saveTG{㙻}{40114}
\saveTG{𩁣}{40115}
\saveTG{𩁚}{40115}
\saveTG{𩀤}{40115}
\saveTG{𩁢}{40115}
\saveTG{𩁬}{40115}
\saveTG{堆}{40115}
\saveTG{墥}{40115}
\saveTG{𡏂}{40115}
\saveTG{𢆪}{40115}
\saveTG{𪤕}{40115}
\saveTG{㙲}{40115}
\saveTG{𨿲}{40115}
\saveTG{𩁪}{40115}
\saveTG{𩁘}{40115}
\saveTG{𩁤}{40115}
\saveTG{𡍠}{40116}
\saveTG{壇}{40116}
\saveTG{𪢷}{40117}
\saveTG{𡔬}{40117}
\saveTG{𡒲}{40117}
\saveTG{𣑯}{40117}
\saveTG{𡊾}{40117}
\saveTG{𡐒}{40117}
\saveTG{坑}{40117}
\saveTG{𡊿}{40117}
\saveTG{𡎭}{40117}
\saveTG{垃}{40118}
\saveTG{𡗸}{40118}
\saveTG{𡙨}{40121}
\saveTG{𪣹}{40121}
\saveTG{垿}{40122}
\saveTG{𡎑}{40122}
\saveTG{𨿾}{40125}
\saveTG{𡚍}{40126}
\saveTG{塝}{40127}
\saveTG{墑}{40127}
\saveTG{坊}{40127}
\saveTG{墒}{40127}
\saveTG{𦏷}{40127}
\saveTG{堉}{40127}
\saveTG{墉}{40127}
\saveTG{塙}{40127}
\saveTG{𪤋}{40127}
\saveTG{塆}{40127}
\saveTG{𡊔}{40127}
\saveTG{䎝}{40127}
\saveTG{𪢳}{40127}
\saveTG{䎞}{40127}
\saveTG{𡐴}{40127}
\saveTG{𪤪}{40127}
\saveTG{𪥁}{40127}
\saveTG{奫}{40128}
\saveTG{𡊨}{40131}
\saveTG{𧎆}{40131}
\saveTG{䘄}{40131}
\saveTG{壞}{40132}
\saveTG{壌}{40132}
\saveTG{壤}{40132}
\saveTG{㙥}{40132}
\saveTG{𣑏}{40132}
\saveTG{壕}{40132}
\saveTG{螙}{40136}
\saveTG{䖯}{40136}
\saveTG{𧍾}{40136}
\saveTG{𧈠}{40136}
\saveTG{𪤥}{40136}
\saveTG{𫋒}{40136}
\saveTG{𧕴}{40136}
\saveTG{蠧}{40136}
\saveTG{𧈹}{40136}
\saveTG{墌}{40137}
\saveTG{𪣢}{40140}
\saveTG{坟}{40140}
\saveTG{𡍔}{40141}
\saveTG{壀}{40141}
\saveTG{𡓑}{40141}
\saveTG{垶}{40141}
\saveTG{𡍣}{40143}
\saveTG{𡍓}{40143}
\saveTG{𡓣}{40146}
\saveTG{墇}{40146}
\saveTG{𡍨}{40147}
\saveTG{𪔌}{40147}
\saveTG{𡍩}{40147}
\saveTG{𡒡}{40147}
\saveTG{𡓊}{40147}
\saveTG{鼖}{40147}
\saveTG{埻}{40147}
\saveTG{埣}{40148}
\saveTG{𡉣}{40149}
\saveTG{𪤯}{40151}
\saveTG{㚜}{40153}
\saveTG{𡙮}{40153}
\saveTG{𡒅}{40161}
\saveTG{𡏧}{40161}
\saveTG{𡓜}{40161}
\saveTG{培}{40161}
\saveTG{堷}{40161}
\saveTG{𡌽}{40162}
\saveTG{𪣠}{40164}
\saveTG{塘}{40165}
\saveTG{𡌮}{40169}
\saveTG{𡒷}{40169}
\saveTG{垴}{40172}
\saveTG{垓}{40182}
\saveTG{𪣼}{40184}
\saveTG{𡌻}{40185}
\saveTG{壙}{40186}
\saveTG{𡐝}{40189}
\saveTG{𡒄}{40191}
\saveTG{𡓔}{40194}
\saveTG{𡍕}{40194}
\saveTG{壈}{40194}
\saveTG{㙫}{40194}
\saveTG{𡌿}{40196}
\saveTG{𡐓}{40199}
\saveTG{才}{40200}
\saveTG{犭}{40200}
\saveTG{犷}{40200}
\saveTG{乄}{40200}
\saveTG{𡔜}{40207}
\saveTG{𡖘}{40207}
\saveTG{声}{40207}
\saveTG{奓}{40207}
\saveTG{𡗔}{40208}
\saveTG{𡗟}{40208}
\saveTG{𤜪}{40210}
\saveTG{䙽}{40212}
\saveTG{𫌦}{40212}
\saveTG{獍}{40212}
\saveTG{圥}{40212}
\saveTG{売}{40212}
\saveTG{尭}{40212}
\saveTG{堯}{40212}
\saveTG{克}{40212}
\saveTG{𪱰}{40214}
\saveTG{㚝}{40214}
\saveTG{㹥}{40214}
\saveTG{𦒱}{40214}
\saveTG{在}{40214}
\saveTG{幢}{40215}
\saveTG{𨾮}{40215}
\saveTG{𤣕}{40215}
\saveTG{𨿕}{40215}
\saveTG{𪻉}{40215}
\saveTG{㮅}{40215}
\saveTG{𤢐}{40215}
\saveTG{𡚝}{40215}
\saveTG{𨾺}{40215}
\saveTG{𨾭}{40215}
\saveTG{𡚜}{40215}
\saveTG{𩁩}{40215}
\saveTG{𡚊}{40215}
\saveTG{奞}{40215}
\saveTG{猚}{40215}
\saveTG{帷}{40215}
\saveTG{獞}{40215}
\saveTG{隺}{40215}
\saveTG{𤢏}{40216}
\saveTG{𢅒}{40216}
\saveTG{𢀃}{40217}
\saveTG{𠧋}{40217}
\saveTG{𩖘}{40217}
\saveTG{𡕀}{40217}
\saveTG{𡙚}{40217}
\saveTG{𠘮}{40217}
\saveTG{𡋞}{40217}
\saveTG{𠦐}{40217}
\saveTG{𤡇}{40217}
\saveTG{𧡛}{40217}
\saveTG{𤡊}{40217}
\saveTG{𤞀}{40217}
\saveTG{𠒯}{40217}
\saveTG{㡆}{40217}
\saveTG{𢁣}{40217}
\saveTG{𡕁}{40217}
\saveTG{𡊪}{40217}
\saveTG{𪞱}{40217}
\saveTG{𠒀}{40217}
\saveTG{𦒲}{40217}
\saveTG{𩲘}{40217}
\saveTG{𣎳}{40217}
\saveTG{𠒶}{40217}
\saveTG{犺}{40217}
\saveTG{壳}{40217}
\saveTG{𣨑}{40217}
\saveTG{𠒈}{40217}
\saveTG{𡔻}{40217}
\saveTG{𧹦}{40217}
\saveTG{𡔧}{40217}
\saveTG{𠒖}{40217}
\saveTG{𡔙}{40217}
\saveTG{𤱿}{40221}
\saveTG{𡗲}{40221}
\saveTG{𦣚}{40221}
\saveTG{𡘧}{40222}
\saveTG{𢒊}{40222}
\saveTG{𡙓}{40222}
\saveTG{䶓}{40223}
\saveTG{㦞}{40224}
\saveTG{𥀪}{40225}
\saveTG{㕯}{40226}
\saveTG{𠙻}{40227}
\saveTG{脅}{40227}
\saveTG{肴}{40227}
\saveTG{𠳮}{40227}
\saveTG{𠕆}{40227}
\saveTG{𠕇}{40227}
\saveTG{𢁛}{40227}
\saveTG{𡉉}{40227}
\saveTG{𦝸}{40227}
\saveTG{𣃤}{40227}
\saveTG{𢄎}{40227}
\saveTG{𢁸}{40227}
\saveTG{𠕒}{40227}
\saveTG{𤠻}{40227}
\saveTG{𤡢}{40227}
\saveTG{𤠖}{40227}
\saveTG{𤢒}{40227}
\saveTG{𦥤}{40227}
\saveTG{𦦔}{40227}
\saveTG{𦣊}{40227}
\saveTG{𡉅}{40227}
\saveTG{鼒}{40227}
\saveTG{𦒸}{40227}
\saveTG{𤟂}{40227}
\saveTG{𣝻}{40227}
\saveTG{𡜚}{40227}
\saveTG{𢁝}{40227}
\saveTG{𤟾}{40227}
\saveTG{𢺶}{40227}
\saveTG{𠬖}{40227}
\saveTG{㠻}{40227}
\saveTG{𦙩}{40227}
\saveTG{𦒶}{40227}
\saveTG{𦓁}{40227}
\saveTG{𢂞}{40227}
\saveTG{𢁫}{40227}
\saveTG{𠿧}{40227}
\saveTG{𡘣}{40227}
\saveTG{𪔄}{40227}
\saveTG{𪗄}{40227}
\saveTG{㺎}{40227}
\saveTG{𪔇}{40227}
\saveTG{𪥜}{40227}
\saveTG{𦚉}{40227}
\saveTG{𡘴}{40227}
\saveTG{𠛞}{40227}
\saveTG{𡔴}{40227}
\saveTG{𢂯}{40227}
\saveTG{𣝧}{40227}
\saveTG{𣙒}{40227}
\saveTG{奟}{40227}
\saveTG{布}{40227}
\saveTG{奝}{40227}
\saveTG{巾}{40227}
\saveTG{臡}{40227}
\saveTG{冇}{40227}
\saveTG{南}{40227}
\saveTG{內}{40227}
\saveTG{内}{40227}
\saveTG{肉}{40227}
\saveTG{禸}{40227}
\saveTG{壭}{40227}
\saveTG{有}{40227}
\saveTG{希}{40227}
\saveTG{𡗦}{40228}
\saveTG{𡙆}{40228}
\saveTG{𡗵}{40228}
\saveTG{夰}{40228}
\saveTG{夼}{40228}
\saveTG{𡥇}{40230}
\saveTG{犿}{40230}
\saveTG{㝳}{40230}
\saveTG{𠀊}{40230}
\saveTG{𤣄}{40231}
\saveTG{𪱱}{40231}
\saveTG{赤}{40231}
\saveTG{𢂄}{40231}
\saveTG{㹡}{40231}
\saveTG{𪐡}{40231}
\saveTG{𢄺}{40231}
\saveTG{𪒁}{40231}
\saveTG{㺘}{40231}
\saveTG{㡉}{40231}
\saveTG{𡙤}{40232}
\saveTG{𤢢}{40232}
\saveTG{𪵪}{40232}
\saveTG{𡗷}{40232}
\saveTG{𡘎}{40232}
\saveTG{𣏶}{40232}
\saveTG{獽}{40232}
\saveTG{𣴐}{40232}
\saveTG{𡉺}{40232}
\saveTG{𤠠}{40232}
\saveTG{𤠆}{40232}
\saveTG{𣳂}{40232}
\saveTG{𡙥}{40232}
\saveTG{𢄴}{40232}
\saveTG{𤢭}{40232}
\saveTG{𤢼}{40232}
\saveTG{䖁}{40236}
\saveTG{𤢛}{40236}
\saveTG{𤡖}{40236}
\saveTG{𢅏}{40237}
\saveTG{𢅖}{40237}
\saveTG{𢅳}{40237}
\saveTG{𢂩}{40237}
\saveTG{𤝋}{40240}
\saveTG{幮}{40240}
\saveTG{𤢣}{40241}
\saveTG{𪣄}{40241}
\saveTG{𨐘}{40241}
\saveTG{𤠜}{40241}
\saveTG{𢋴}{40242}
\saveTG{𡑉}{40242}
\saveTG{𣝣}{40243}
\saveTG{帹}{40244}
\saveTG{幛}{40246}
\saveTG{獐}{40246}
\saveTG{𢅻}{40247}
\saveTG{皮}{40247}
\saveTG{存}{40247}
\saveTG{𢃒}{40248}
\saveTG{猝}{40248}
\saveTG{狡}{40248}
\saveTG{奯}{40253}
\saveTG{𡚓}{40253}
\saveTG{𦨅}{40259}
\saveTG{狺}{40261}
\saveTG{𤟟}{40261}
\saveTG{𢄈}{40261}
\saveTG{𧹴}{40263}
\saveTG{𤠕}{40263}
\saveTG{𤞼}{40264}
\saveTG{𤠯}{40265}
\saveTG{赯}{40265}
\saveTG{𪥕}{40272}
\saveTG{𡱠}{40277}
\saveTG{𢃨}{40284}
\saveTG{㹹}{40285}
\saveTG{獷}{40286}
\saveTG{𤢤}{40294}
\saveTG{𢃅}{40294}
\saveTG{𡕊}{40296}
\saveTG{猄}{40296}
\saveTG{寸}{40300}
\saveTG{𠦹}{40303}
\saveTG{赱}{40308}
\saveTG{𫆭}{40309}
\saveTG{𡙯}{40319}
\saveTG{𩿟}{40323}
\saveTG{𩿜}{40323}
\saveTG{𡗪}{40327}
\saveTG{𩤺}{40327}
\saveTG{𩾪}{40327}
\saveTG{𫛏}{40327}
\saveTG{𩡱}{40327}
\saveTG{𩢙}{40327}
\saveTG{𪀧}{40327}
\saveTG{㣻}{40330}
\saveTG{办}{40330}
\saveTG{𢛳}{40331}
\saveTG{𢡫}{40331}
\saveTG{戁}{40331}
\saveTG{悫}{40331}
\saveTG{志}{40331}
\saveTG{𢤖}{40331}
\saveTG{𪑵}{40331}
\saveTG{𢗐}{40331}
\saveTG{悳}{40331}
\saveTG{惪}{40331}
\saveTG{𢝫}{40331}
\saveTG{𢖽}{40331}
\saveTG{㥣}{40331}
\saveTG{𢙮}{40331}
\saveTG{㚠}{40331}
\saveTG{㸐}{40331}
\saveTG{𤒿}{40331}
\saveTG{恚}{40331}
\saveTG{𤑌}{40332}
\saveTG{𢖰}{40332}
\saveTG{㤫}{40332}
\saveTG{𢡘}{40333}
\saveTG{态}{40333}
\saveTG{燾}{40334}
\saveTG{𢚦}{40334}
\saveTG{𤎿}{40334}
\saveTG{㣽}{40334}
\saveTG{𢝬}{40334}
\saveTG{怘}{40336}
\saveTG{𡚔}{40336}
\saveTG{憙}{40336}
\saveTG{熹}{40336}
\saveTG{𩵴}{40336}
\saveTG{𪞟}{40338}
\saveTG{㥲}{40338}
\saveTG{𡚉}{40338}
\saveTG{㤲}{40338}
\saveTG{𣑰}{40338}
\saveTG{𤋯}{40338}
\saveTG{杰}{40339}
\saveTG{𢠃}{40339}
\saveTG{𪤳}{40341}
\saveTG{寺}{40341}
\saveTG{奪}{40341}
\saveTG{𡔺}{40346}
\saveTG{夺}{40348}
\saveTG{𡚢}{40394}
\saveTG{爻}{40400}
\saveTG{龺}{40400}
\saveTG{㚧}{40400}
\saveTG{女}{40400}
\saveTG{𡘪}{40401}
\saveTG{𠦡}{40401}
\saveTG{𢆎}{40401}
\saveTG{辜}{40401}
\saveTG{卆}{40401}
\saveTG{耷}{40401}
\saveTG{幸}{40401}
\saveTG{𤕜}{40401}
\saveTG{㚔}{40401}
\saveTG{𡗴}{40403}
\saveTG{𡚦}{40403}
\saveTG{姭}{40404}
\saveTG{㚣}{40404}
\saveTG{𡜦}{40404}
\saveTG{𪥺}{40404}
\saveTG{𠧉}{40405}
\saveTG{𠧈}{40405}
\saveTG{𠦝}{40406}
\saveTG{}{40406}
\saveTG{𥆗}{40407}
\saveTG{𡙜}{40407}
\saveTG{𡕟}{40407}
\saveTG{𡕲}{40407}
\saveTG{𠭩}{40407}
\saveTG{𡥉}{40407}
\saveTG{㕝}{40407}
\saveTG{𡙣}{40407}
\saveTG{𡘨}{40407}
\saveTG{𡙋}{40407}
\saveTG{𡥂}{40407}
\saveTG{夌}{40407}
\saveTG{𪥆}{40407}
\saveTG{麥}{40407}
\saveTG{支}{40407}
\saveTG{友}{40407}
\saveTG{李}{40407}
\saveTG{孛}{40407}
\saveTG{𪠣}{40407}
\saveTG{𠬲}{40407}
\saveTG{𡚠}{40407}
\saveTG{𤕐}{40408}
\saveTG{𠦪}{40408}
\saveTG{夲}{40408}
\saveTG{𢾶}{40408}
\saveTG{𡗧}{40408}
\saveTG{𡙐}{40408}
\saveTG{𠦵}{40408}
\saveTG{𪥈}{40409}
\saveTG{孊}{40411}
\saveTG{䬡}{40413}
\saveTG{㛇}{40414}
\saveTG{妵}{40414}
\saveTG{𪌘}{40414}
\saveTG{𨿆}{40415}
\saveTG{𡚟}{40415}
\saveTG{𪍇}{40415}
\saveTG{𡡳}{40415}
\saveTG{𨾙}{40415}
\saveTG{婎}{40415}
\saveTG{𨿨}{40415}
\saveTG{䧴}{40415}
\saveTG{𨾩}{40415}
\saveTG{嬗}{40416}
\saveTG{㜔}{40417}
\saveTG{妔}{40417}
\saveTG{㜲}{40417}
\saveTG{𡢧}{40417}
\saveTG{㜙}{40417}
\saveTG{𡜋}{40417}
\saveTG{𪌁}{40417}
\saveTG{𡛩}{40418}
\saveTG{𡝰}{40418}
\saveTG{婷}{40421}
\saveTG{𡜾}{40421}
\saveTG{𪦎}{40422}
\saveTG{𡣙}{40423}
\saveTG{𠝊}{40424}
\saveTG{㔛}{40427}
\saveTG{𡠀}{40427}
\saveTG{䵂}{40427}
\saveTG{𡦴}{40427}
\saveTG{㛩}{40427}
\saveTG{嫞}{40427}
\saveTG{劦}{40427}
\saveTG{妨}{40427}
\saveTG{媂}{40427}
\saveTG{嫡}{40427}
\saveTG{夯}{40427}
\saveTG{麶}{40427}
\saveTG{嫎}{40427}
\saveTG{𠡊}{40427}
\saveTG{𡦭}{40427}
\saveTG{𪾉}{40427}
\saveTG{𡦳}{40427}
\saveTG{𡣡}{40427}
\saveTG{𡢹}{40427}
\saveTG{𡔵}{40427}
\saveTG{𡦲}{40427}
\saveTG{嫶}{40431}
\saveTG{嬷}{40432}
\saveTG{𡟓}{40432}
\saveTG{𡣪}{40432}
\saveTG{㛄}{40432}
\saveTG{娹}{40432}
\saveTG{妶}{40432}
\saveTG{㜵}{40432}
\saveTG{孃}{40432}
\saveTG{嬢}{40432}
\saveTG{嬤}{40432}
\saveTG{嬑}{40436}
\saveTG{𡡫}{40437}
\saveTG{嫬}{40437}
\saveTG{嬚}{40437}
\saveTG{𡣇}{40438}
\saveTG{㜳}{40439}
\saveTG{妏}{40440}
\saveTG{𠦃}{40440}
\saveTG{𠦄}{40440}
\saveTG{卉}{40440}
\saveTG{𡜌}{40441}
\saveTG{㛙}{40441}
\saveTG{𡟾}{40441}
\saveTG{𡉓}{40441}
\saveTG{𢌷}{40442}
\saveTG{𪍅}{40442}
\saveTG{弆}{40443}
\saveTG{𡗹}{40443}
\saveTG{𡞖}{40443}
\saveTG{𡙘}{40443}
\saveTG{𡘛}{40444}
\saveTG{𡣥}{40444}
\saveTG{𡛬}{40444}
\saveTG{奔}{40444}
\saveTG{姦}{40444}
\saveTG{嫜}{40446}
\saveTG{𢍚}{40446}
\saveTG{𨍔}{40447}
\saveTG{𡘏}{40447}
\saveTG{𡔤}{40447}
\saveTG{𣏪}{40447}
\saveTG{轰}{40447}
\saveTG{孇}{40447}
\saveTG{𡝵}{40448}
\saveTG{𢌻}{40448}
\saveTG{姣}{40448}
\saveTG{㚏}{40448}
\saveTG{𦺏}{40448}
\saveTG{𪜅}{40449}
\saveTG{𢌶}{40449}
\saveTG{𡚄}{40453}
\saveTG{𡜿}{40460}
\saveTG{婄}{40461}
\saveTG{㜃}{40461}
\saveTG{𧧿}{40461}
\saveTG{䴺}{40461}
\saveTG{㛺}{40461}
\saveTG{𡞙}{40461}
\saveTG{娮}{40461}
\saveTG{嘉}{40461}
\saveTG{𪦃}{40462}
\saveTG{𪍼}{40462}
\saveTG{𡡿}{40462}
\saveTG{㜅}{40463}
\saveTG{㜍}{40465}
\saveTG{𡗡}{40480}
\saveTG{𢌾}{40482}
\saveTG{姟}{40482}
\saveTG{嫉}{40484}
\saveTG{𪍿}{40486}
\saveTG{媇}{40494}
\saveTG{嫲}{40494}
\saveTG{𡝥}{40494}
\saveTG{𪦖}{40494}
\saveTG{婛}{40496}
\saveTG{嫝}{40499}
\saveTG{𪟋}{40500}
\saveTG{车}{40500}
\saveTG{𦍬}{40501}
\saveTG{𦍒}{40501}
\saveTG{羍}{40501}
\saveTG{𢪿}{40502}
\saveTG{𦥹}{40502}
\saveTG{牵}{40502}
\saveTG{𢺋}{40502}
\saveTG{𢫀}{40502}
\saveTG{舝}{40503}
\saveTG{𤙏}{40503}
\saveTG{𪺝}{40504}
\saveTG{𡍦}{40506}
\saveTG{𨊦}{40506}
\saveTG{䡤}{40506}
\saveTG{𩎲}{40507}
\saveTG{𡗢}{40508}
\saveTG{辘}{40512}
\saveTG{𩌫}{40512}
\saveTG{䪒}{40514}
\saveTG{𩏨}{40515}
\saveTG{𩍓}{40515}
\saveTG{𩋘}{40515}
\saveTG{𩍅}{40515}
\saveTG{𩀵}{40515}
\saveTG{𩍕}{40516}
\saveTG{𩏈}{40517}
\saveTG{𩎾}{40517}
\saveTG{鞡}{40518}
\saveTG{𩊌}{40518}
\saveTG{䢂}{40518}
\saveTG{𩋣}{40527}
\saveTG{䩷}{40527}
\saveTG{韀}{40527}
\saveTG{㚕}{40527}
\saveTG{𩌡}{40527}
\saveTG{𩎭}{40530}
\saveTG{𩍶}{40531}
\saveTG{𩏢}{40531}
\saveTG{䩙}{40532}
\saveTG{䩾}{40534}
\saveTG{𩍖}{40536}
\saveTG{𩌬}{40546}
\saveTG{𩏎}{40547}
\saveTG{𩋛}{40547}
\saveTG{䩲}{40547}
\saveTG{鞟}{40547}
\saveTG{韕}{40547}
\saveTG{𩎦}{40548}
\saveTG{𩊔}{40548}
\saveTG{较}{40548}
\saveTG{毐}{40557}
\saveTG{鞛}{40561}
\saveTG{𩋻}{40566}
\saveTG{𩋑}{40594}
\saveTG{𣟦}{40594}
\saveTG{辌}{40596}
\saveTG{𥃭}{40600}
\saveTG{𠮣}{40600}
\saveTG{𡘐}{40600}
\saveTG{甴}{40600}
\saveTG{古}{40600}
\saveTG{右}{40600}
\saveTG{𥄽}{40600}
\saveTG{𡔯}{40601}
\saveTG{𡙬}{40601}
\saveTG{𤱒}{40601}
\saveTG{𡐑}{40601}
\saveTG{𠹫}{40601}
\saveTG{𡍐}{40601}
\saveTG{𡍬}{40601}
\saveTG{喜}{40601}
\saveTG{奮}{40601}
\saveTG{旮}{40601}
\saveTG{吉}{40601}
\saveTG{叴}{40601}
\saveTG{啬}{40601}
\saveTG{嗇}{40601}
\saveTG{𣅎}{40601}
\saveTG{𠮷}{40601}
\saveTG{𡔶}{40601}
\saveTG{𡙄}{40601}
\saveTG{𧥢}{40601}
\saveTG{㚛}{40601}
\saveTG{𦒿}{40601}
\saveTG{𦓀}{40601}
\saveTG{𣆕}{40602}
\saveTG{𥒐}{40602}
\saveTG{奤}{40602}
\saveTG{𤰛}{40602}
\saveTG{㚚}{40602}
\saveTG{𡙂}{40603}
\saveTG{𣓩}{40603}
\saveTG{𠶮}{40604}
\saveTG{𠯌}{40604}
\saveTG{奢}{40604}
\saveTG{𠯛}{40604}
\saveTG{㫩}{40604}
\saveTG{𡙍}{40605}
\saveTG{夁}{40606}
\saveTG{𡘊}{40606}
\saveTG{𡙲}{40606}
\saveTG{䎛}{40607}
\saveTG{㖈}{40607}
\saveTG{𠺇}{40607}
\saveTG{昚}{40608}
\saveTG{夻}{40608}
\saveTG{𤲷}{40608}
\saveTG{奋}{40608}
\saveTG{眘}{40608}
\saveTG{𣇸}{40608}
\saveTG{𠻮}{40609}
\saveTG{𤳛}{40609}
\saveTG{𣐭}{40609}
\saveTG{𤲝}{40609}
\saveTG{杏}{40609}
\saveTG{杳}{40609}
\saveTG{𠾤}{40609}
\saveTG{唟}{40612}
\saveTG{㖛}{40612}
\saveTG{嗭}{40612}
\saveTG{䧺}{40615}
\saveTG{𨿫}{40615}
\saveTG{䧸}{40615}
\saveTG{𨿟}{40615}
\saveTG{𪏌}{40615}
\saveTG{𧺃}{40617}
\saveTG{㗯}{40617}
\saveTG{𦒾}{40621}
\saveTG{奇}{40621}
\saveTG{𦓆}{40627}
\saveTG{𧹼}{40627}
\saveTG{奛}{40627}
\saveTG{𦒽}{40627}
\saveTG{𡊦}{40627}
\saveTG{耇}{40627}
\saveTG{𦓂}{40627}
\saveTG{𦓃}{40627}
\saveTG{𡙵}{40632}
\saveTG{𪜙}{40641}
\saveTG{壽}{40641}
\saveTG{𨐒}{40641}
\saveTG{𡔽}{40643}
\saveTG{𣋄}{40647}
\saveTG{𧁷}{40656}
\saveTG{嚞}{40661}
\saveTG{𡚗}{40666}
\saveTG{𣖎}{40666}
\saveTG{𡙗}{40666}
\saveTG{楍}{40669}
\saveTG{𤱑}{40687}
\saveTG{畞}{40687}
\saveTG{𠾂}{40691}
\saveTG{䧿}{40695}
\saveTG{𡚏}{40710}
\saveTG{㐊}{40710}
\saveTG{𠤎}{40710}
\saveTG{七}{40710}
\saveTG{㐋}{40711}
\saveTG{㐗}{40711}
\saveTG{壱}{40712}
\saveTG{𡘁}{40712}
\saveTG{𠃾}{40712}
\saveTG{𪓤}{40712}
\saveTG{乜}{40712}
\saveTG{㔺}{40714}
\saveTG{㐂}{40714}
\saveTG{奁}{40714}
\saveTG{𨲨}{40715}
\saveTG{雄}{40715}
\saveTG{𪥍}{40715}
\saveTG{𡔩}{40715}
\saveTG{𡘤}{40715}
\saveTG{𡗥}{40715}
\saveTG{𩀔}{40715}
\saveTG{𦒷}{40715}
\saveTG{奩}{40716}
\saveTG{奄}{40716}
\saveTG{𦓈}{40717}
\saveTG{𨚻}{40717}
\saveTG{𪓜}{40717}
\saveTG{𠠾}{40717}
\saveTG{夿}{40717}
\saveTG{鼀}{40717}
\saveTG{奆}{40717}
\saveTG{鼁}{40717}
\saveTG{朰}{40717}
\saveTG{𡘹}{40717}
\saveTG{𪓡}{40717}
\saveTG{鼃}{40717}
\saveTG{𪚷}{40717}
\saveTG{𧹺}{40717}
\saveTG{𦒹}{40718}
\saveTG{𠷔}{40718}
\saveTG{𣓚}{40727}
\saveTG{𡗝}{40727}
\saveTG{奅}{40727}
\saveTG{𡘆}{40727}
\saveTG{𡙿}{40731}
\saveTG{𠫓}{40731}
\saveTG{𠫤}{40731}
\saveTG{夽}{40731}
\saveTG{𡘚}{40732}
\saveTG{𡋡}{40732}
\saveTG{䘮}{40732}
\saveTG{𡄜}{40732}
\saveTG{𡕅}{40732}
\saveTG{𡊮}{40732}
\saveTG{𡗰}{40732}
\saveTG{农}{40732}
\saveTG{夽}{40732}
\saveTG{袁}{40732}
\saveTG{套}{40732}
\saveTG{枩}{40732}
\saveTG{喪}{40732}
\saveTG{厺}{40732}
\saveTG{𧙐}{40732}
\saveTG{去}{40732}
\saveTG{𡘳}{40732}
\saveTG{厷}{40732}
\saveTG{𡘷}{40732}
\saveTG{丧}{40732}
\saveTG{𡃟}{40732}
\saveTG{𠷫}{40732}
\saveTG{厹}{40732}
\saveTG{𤯋}{40734}
\saveTG{𣡫}{40740}
\saveTG{奃}{40742}
\saveTG{𩰦}{40747}
\saveTG{𣏭}{40757}
\saveTG{𣫵}{40757}
\saveTG{𣫴}{40757}
\saveTG{𤯆}{40761}
\saveTG{𦒴}{40771}
\saveTG{㞭}{40772}
\saveTG{𡿊}{40772}
\saveTG{𡴷}{40772}
\saveTG{𪘘}{40772}
\saveTG{𡉆}{40772}
\saveTG{㚎}{40772}
\saveTG{𡕉}{40773}
\saveTG{𠦔}{40774}
\saveTG{𤯃}{40774}
\saveTG{卋}{40774}
\saveTG{𡇛}{40777}
\saveTG{𦥳}{40777}
\saveTG{𡚒}{40777}
\saveTG{𪠢}{40786}
\saveTG{𤯑}{40794}
\saveTG{大}{40800}
\saveTG{𧻃}{40800}
\saveTG{㻎}{40801}
\saveTG{𠡥}{40801}
\saveTG{𡘼}{40801}
\saveTG{𡘂}{40801}
\saveTG{𡘈}{40801}
\saveTG{𪞉}{40801}
\saveTG{𧾡}{40801}
\saveTG{𧻄}{40801}
\saveTG{𤕤}{40801}
\saveTG{真}{40801}
\saveTG{走}{40801}
\saveTG{疐}{40801}
\saveTG{趇}{40801}
\saveTG{趡}{40801}
\saveTG{𧾍}{40801}
\saveTG{𧽥}{40801}
\saveTG{𧽿}{40801}
\saveTG{𡐢}{40801}
\saveTG{𠔧}{40801}
\saveTG{奒}{40802}
\saveTG{𡗜}{40802}
\saveTG{𤴞}{40802}
\saveTG{𧽦}{40802}
\saveTG{𡔰}{40802}
\saveTG{㚄}{40802}
\saveTG{𡗱}{40802}
\saveTG{贲}{40802}
\saveTG{赍}{40802}
\saveTG{𤴡}{40802}
\saveTG{𧾕}{40802}
\saveTG{𧾙}{40802}
\saveTG{𨂬}{40802}
\saveTG{𨄟}{40802}
\saveTG{趭}{40803}
\saveTG{𧾄}{40803}
\saveTG{𧺻}{40803}
\saveTG{𧻨}{40804}
\saveTG{𡘖}{40804}
\saveTG{𡟗}{40804}
\saveTG{𧼳}{40804}
\saveTG{𡔱}{40804}
\saveTG{𡔢}{40804}
\saveTG{𧽣}{40804}
\saveTG{𠁗}{40804}
\saveTG{𡘘}{40804}
\saveTG{𡘥}{40804}
\saveTG{𧾑}{40804}
\saveTG{𪥄}{40804}
\saveTG{爽}{40804}
\saveTG{卖}{40804}
\saveTG{𡘲}{40804}
\saveTG{𤕡}{40804}
\saveTG{㚐}{40804}
\saveTG{䞳}{40806}
\saveTG{𧷏}{40806}
\saveTG{𧷗}{40806}
\saveTG{𧶠}{40806}
\saveTG{𧵎}{40806}
\saveTG{𧴥}{40806}
\saveTG{䝱}{40806}
\saveTG{𡚐}{40806}
\saveTG{𧷈}{40806}
\saveTG{𧽴}{40806}
\saveTG{𧶚}{40806}
\saveTG{𧴦}{40806}
\saveTG{𡘜}{40806}
\saveTG{𧸇}{40806}
\saveTG{奭}{40806}
\saveTG{賣}{40806}
\saveTG{賚}{40806}
\saveTG{賷}{40806}
\saveTG{賫}{40806}
\saveTG{賁}{40806}
\saveTG{䝿}{40806}
\saveTG{𧶌}{40806}
\saveTG{𥇛}{40806}
\saveTG{𡘩}{40807}
\saveTG{㚒}{40808}
\saveTG{𡔡}{40808}
\saveTG{𧽑}{40808}
\saveTG{𡙁}{40808}
\saveTG{夾}{40808}
\saveTG{灰}{40809}
\saveTG{灻}{40809}
\saveTG{𤓉}{40809}
\saveTG{𤉲}{40809}
\saveTG{𤏴}{40809}
\saveTG{𤊽}{40809}
\saveTG{𤆯}{40809}
\saveTG{𤋔}{40809}
\saveTG{𡗕}{40809}
\saveTG{黈}{40814}
\saveTG{𩀴}{40815}
\saveTG{𨿣}{40815}
\saveTG{𩀑}{40815}
\saveTG{難}{40815}
\saveTG{𤻓}{40827}
\saveTG{𪏁}{40848}
\saveTG{𧶐}{40861}
\saveTG{𧷨}{40862}
\saveTG{𧶘}{40868}
\saveTG{𧴿}{40869}
\saveTG{𡙒}{40883}
\saveTG{𡘙}{40884}
\saveTG{𤆍}{40898}
\saveTG{㶫}{40898}
\saveTG{𪏢}{40899}
\saveTG{𤆰}{40899}
\saveTG{木}{40900}
\saveTG{朩}{40900}
\saveTG{𣏄}{40901}
\saveTG{奈}{40901}
\saveTG{𥘾}{40901}
\saveTG{柰}{40901}
\saveTG{𫀂}{40901}
\saveTG{𠦙}{40901}
\saveTG{𥾥}{40903}
\saveTG{𦃦}{40903}
\saveTG{索}{40903}
\saveTG{𣙇}{40904}
\saveTG{𣏅}{40904}
\saveTG{𣕫}{40904}
\saveTG{𡙸}{40904}
\saveTG{东}{40904}
\saveTG{𣏋}{40904}
\saveTG{枽}{40904}
\saveTG{槖}{40904}
\saveTG{杀}{40904}
\saveTG{桒}{40904}
\saveTG{杂}{40904}
\saveTG{𫂿}{40904}
\saveTG{𣟘}{40904}
\saveTG{𡒩}{40904}
\saveTG{𣓟}{40904}
\saveTG{𣙲}{40904}
\saveTG{𣝷}{40904}
\saveTG{𣏂}{40904}
\saveTG{𠡷}{40904}
\saveTG{𥞤}{40904}
\saveTG{𣓪}{40904}
\saveTG{𪥋}{40905}
\saveTG{㭐}{40905}
\saveTG{𪳚}{40905}
\saveTG{尞}{40906}
\saveTG{𡘝}{40906}
\saveTG{𡮃}{40906}
\saveTG{𥈸}{40906}
\saveTG{𠬇}{40906}
\saveTG{𠐇}{40908}
\saveTG{來}{40908}
\saveTG{夵}{40908}
\saveTG{𤱀}{40908}
\saveTG{𣷚}{40909}
\saveTG{桼}{40909}
\saveTG{杧}{40910}
\saveTG{𣡂}{40911}
\saveTG{樚}{40912}
\saveTG{樈}{40912}
\saveTG{梳}{40912}
\saveTG{橀}{40912}
\saveTG{𣚪}{40914}
\saveTG{桩}{40914}
\saveTG{柱}{40914}
\saveTG{𣖵}{40914}
\saveTG{𧹥}{40914}
\saveTG{㰙}{40915}
\saveTG{𣡻}{40915}
\saveTG{𣗙}{40915}
\saveTG{𣙯}{40915}
\saveTG{𪴓}{40915}
\saveTG{㰚}{40915}
\saveTG{𣚳}{40915}
\saveTG{橦}{40915}
\saveTG{椎}{40915}
\saveTG{雑}{40915}
\saveTG{𣠥}{40915}
\saveTG{䨅}{40915}
\saveTG{𩁞}{40915}
\saveTG{𣟐}{40915}
\saveTG{檀}{40916}
\saveTG{槞}{40916}
\saveTG{𣓄}{40917}
\saveTG{𣝂}{40917}
\saveTG{𣟂}{40917}
\saveTG{𣖲}{40917}
\saveTG{𣠾}{40917}
\saveTG{𪳄}{40917}
\saveTG{𣟅}{40917}
\saveTG{杭}{40917}
\saveTG{𣑁}{40917}
\saveTG{柆}{40918}
\saveTG{楟}{40921}
\saveTG{楌}{40922}
\saveTG{櫅}{40923}
\saveTG{𪲎}{40924}
\saveTG{𣙮}{40924}
\saveTG{㯆}{40925}
\saveTG{𣠐}{40926}
\saveTG{枋}{40927}
\saveTG{檇}{40927}
\saveTG{棛}{40927}
\saveTG{樆}{40927}
\saveTG{𣘋}{40927}
\saveTG{楴}{40927}
\saveTG{槁}{40927}
\saveTG{槜}{40927}
\saveTG{樀}{40927}
\saveTG{𪴇}{40927}
\saveTG{𣓲}{40927}
\saveTG{榜}{40927}
\saveTG{槦}{40927}
\saveTG{柿}{40927}
\saveTG{梈}{40927}
\saveTG{𣙛}{40927}
\saveTG{𣏣}{40930}
\saveTG{𣡹}{40931}
\saveTG{樵}{40931}
\saveTG{㭹}{40931}
\saveTG{𣑇}{40931}
\saveTG{𣐙}{40931}
\saveTG{橠}{40932}
\saveTG{櫰}{40932}
\saveTG{榱}{40932}
\saveTG{㰅}{40932}
\saveTG{𣡬}{40932}
\saveTG{𣟎}{40932}
\saveTG{檺}{40932}
\saveTG{𣘨}{40932}
\saveTG{𣠯}{40932}
\saveTG{𣐿}{40932}
\saveTG{𣛎}{40932}
\saveTG{𣟊}{40932}
\saveTG{欀}{40932}
\saveTG{𣟥}{40936}
\saveTG{𧅬}{40936}
\saveTG{𣟻}{40936}
\saveTG{檍}{40936}
\saveTG{𣟚}{40937}
\saveTG{𣜰}{40937}
\saveTG{樜}{40937}
\saveTG{椨}{40940}
\saveTG{𣎽}{40940}
\saveTG{櫥}{40940}
\saveTG{𣐀}{40940}
\saveTG{𣖸}{40941}
\saveTG{𣔳}{40941}
\saveTG{梓}{40941}
\saveTG{榳}{40941}
\saveTG{榫}{40941}
\saveTG{𣐼}{40941}
\saveTG{𨐖}{40941}
\saveTG{𣘲}{40941}
\saveTG{檘}{40941}
\saveTG{𣕛}{40942}
\saveTG{𣐁}{40942}
\saveTG{𣘚}{40943}
\saveTG{𦆽}{40943}
\saveTG{㭽}{40943}
\saveTG{𣙦}{40943}
\saveTG{椄}{40944}
\saveTG{𣓑}{40945}
\saveTG{㯅}{40945}
\saveTG{𣕣}{40946}
\saveTG{樟}{40946}
\saveTG{𣓷}{40946}
\saveTG{𣑑}{40947}
\saveTG{𣕿}{40947}
\saveTG{櫠}{40947}
\saveTG{棭}{40947}
\saveTG{欆}{40947}
\saveTG{櫦}{40947}
\saveTG{椁}{40947}
\saveTG{𪲮}{40947}
\saveTG{𪳁}{40947}
\saveTG{校}{40948}
\saveTG{椊}{40948}
\saveTG{㚓}{40948}
\saveTG{𠦩}{40948}
\saveTG{𢆧}{40948}
\saveTG{𥸮}{40949}
\saveTG{𣖞}{40952}
\saveTG{㯠}{40953}
\saveTG{𣘍}{40956}
\saveTG{棓}{40961}
\saveTG{㯁}{40961}
\saveTG{𪳀}{40961}
\saveTG{𣚌}{40962}
\saveTG{𣟖}{40962}
\saveTG{槒}{40963}
\saveTG{榶}{40965}
\saveTG{𣚍}{40965}
\saveTG{核}{40982}
\saveTG{𣚲}{40984}
\saveTG{𪲱}{40984}
\saveTG{槉}{40984}
\saveTG{櫎}{40986}
\saveTG{𡚚}{40986}
\saveTG{椩}{40987}
\saveTG{𣛑}{40989}
\saveTG{𣙃}{40989}
\saveTG{𣛜}{40991}
\saveTG{檩}{40991}
\saveTG{𪲖}{40994}
\saveTG{𪲝}{40994}
\saveTG{𣙪}{40994}
\saveTG{㚞}{40994}
\saveTG{𣞤}{40994}
\saveTG{𣝩}{40994}
\saveTG{𣓼}{40994}
\saveTG{森}{40994}
\saveTG{檁}{40994}
\saveTG{榇}{40994}
\saveTG{𣟸}{40994}
\saveTG{𣟒}{40994}
\saveTG{𡙻}{40994}
\saveTG{𥢮}{40994}
\saveTG{𣠮}{40994}
\saveTG{椋}{40996}
\saveTG{𪴡}{40996}
\saveTG{槺}{40999}
\saveTG{𡰝}{41011}
\saveTG{尪}{41011}
\saveTG{𠦾}{41012}
\saveTG{𦓏}{41012}
\saveTG{𡯞}{41012}
\saveTG{𡯘}{41014}
\saveTG{𡯴}{41014}
\saveTG{𦔱}{41014}
\saveTG{𡯋}{41014}
\saveTG{𡯍}{41014}
\saveTG{䖊}{41017}
\saveTG{虓}{41017}
\saveTG{瓭}{41017}
\saveTG{𡯺}{41018}
\saveTG{𩡪}{41027}
\saveTG{㝼}{41040}
\saveTG{𢻼}{41047}
\saveTG{𩒿}{41086}
\saveTG{䪴}{41086}
\saveTG{𩑐}{41086}
\saveTG{頄}{41086}
\saveTG{𩑣}{41086}
\saveTG{𫖫}{41087}
\saveTG{𣗲}{41094}
\saveTG{𡉃}{41100}
\saveTG{𧗑}{41102}
\saveTG{𪉩}{41102}
\saveTG{𥂱}{41102}
\saveTG{𤫏}{41104}
\saveTG{𡋖}{41104}
\saveTG{𡌎}{41104}
\saveTG{垆}{41107}
\saveTG{𧯧}{41108}
\saveTG{圵}{41110}
\saveTG{址}{41110}
\saveTG{壠}{41111}
\saveTG{壢}{41111}
\saveTG{𡉎}{41112}
\saveTG{墟}{41112}
\saveTG{埡}{41112}
\saveTG{坘}{41112}
\saveTG{𡏲}{41112}
\saveTG{壚}{41112}
\saveTG{𡍍}{41112}
\saveTG{𧰡}{41112}
\saveTG{坃}{41112}
\saveTG{垭}{41112}
\saveTG{𡎪}{41114}
\saveTG{堰}{41114}
\saveTG{堙}{41114}
\saveTG{𡏀}{41114}
\saveTG{𡊩}{41114}
\saveTG{壥}{41114}
\saveTG{垤}{41114}
\saveTG{}{41114}
\saveTG{堐}{41114}
\saveTG{𧕁}{41115}
\saveTG{𡓘}{41115}
\saveTG{𡓐}{41116}
\saveTG{垣}{41116}
\saveTG{壃}{41116}
\saveTG{堩}{41116}
\saveTG{塸}{41116}
\saveTG{𡍷}{41116}
\saveTG{𤬪}{41117}
\saveTG{㙈}{41117}
\saveTG{𤭝}{41117}
\saveTG{𪚝}{41117}
\saveTG{𥂇}{41117}
\saveTG{𡓺}{41117}
\saveTG{𡐸}{41117}
\saveTG{𤬿}{41117}
\saveTG{𡔉}{41117}
\saveTG{𡋮}{41117}
\saveTG{壾}{41117}
\saveTG{㙺}{41118}
\saveTG{坯}{41119}
\saveTG{坷}{41120}
\saveTG{圢}{41120}
\saveTG{埗}{41121}
\saveTG{垳}{41121}
\saveTG{𡓎}{41121}
\saveTG{𡉨}{41121}
\saveTG{𡎍}{41127}
\saveTG{𡒚}{41127}
\saveTG{𡓰}{41127}
\saveTG{塥}{41127}
\saveTG{坜}{41127}
\saveTG{圬}{41127}
\saveTG{𪣩}{41127}
\saveTG{墕}{41127}
\saveTG{𡑂}{41127}
\saveTG{壩}{41127}
\saveTG{𡏢}{41127}
\saveTG{𢎏}{41127}
\saveTG{𫛭}{41127}
\saveTG{𡌱}{41127}
\saveTG{壖}{41127}
\saveTG{圷}{41130}
\saveTG{垰}{41131}
\saveTG{墵}{41131}
\saveTG{㙊}{41132}
\saveTG{㙇}{41132}
\saveTG{𡉬}{41132}
\saveTG{㙣}{41132}
\saveTG{坛}{41132}
\saveTG{壉}{41132}
\saveTG{𨲗}{41132}
\saveTG{𡐌}{41133}
\saveTG{𡍞}{41138}
\saveTG{𡉪}{41140}
\saveTG{圩}{41140}
\saveTG{𤞢}{41141}
\saveTG{𪣌}{41141}
\saveTG{𡓳}{41141}
\saveTG{𡏌}{41143}
\saveTG{𪣳}{41143}
\saveTG{𣗃}{41143}
\saveTG{𡍿}{41144}
\saveTG{𡍟}{41144}
\saveTG{𡋱}{41144}
\saveTG{㙘}{41144}
\saveTG{𧰘}{41146}
\saveTG{墰}{41146}
\saveTG{埂}{41146}
\saveTG{𡐯}{41147}
\saveTG{𡏘}{41147}
\saveTG{鼔}{41147}
\saveTG{𪢺}{41147}
\saveTG{𡎋}{41147}
\saveTG{𡎩}{41147}
\saveTG{𡔋}{41147}
\saveTG{𢾩}{41147}
\saveTG{㙤}{41149}
\saveTG{坪}{41149}
\saveTG{𡑬}{41157}
\saveTG{坫}{41160}
\saveTG{塷}{41160}
\saveTG{𪣔}{41161}
\saveTG{𡋦}{41162}
\saveTG{㙧}{41162}
\saveTG{𡓃}{41162}
\saveTG{坧}{41162}
\saveTG{𡐧}{41166}
\saveTG{堛}{41166}
\saveTG{𧯻}{41169}
\saveTG{𡎕}{41172}
\saveTG{𪤭}{41172}
\saveTG{𡐅}{41174}
\saveTG{𡎴}{41181}
\saveTG{𡊕}{41182}
\saveTG{㙭}{41182}
\saveTG{𠏏}{41183}
\saveTG{𡊘}{41184}
\saveTG{堧}{41184}
\saveTG{𡑾}{41184}
\saveTG{𡓡}{41186}
\saveTG{𩕭}{41186}
\saveTG{𡎞}{41186}
\saveTG{𪤙}{41189}
\saveTG{坏}{41190}
\saveTG{墂}{41191}
\saveTG{𡊆}{41191}
\saveTG{塛}{41194}
\saveTG{塬}{41196}
\saveTG{𤞤}{41201}
\saveTG{兣}{41211}
\saveTG{𫜱}{41211}
\saveTG{𪚓}{41211}
\saveTG{䶭}{41211}
\saveTG{龓}{41211}
\saveTG{猅}{41211}
\saveTG{𤝿}{41211}
\saveTG{𢁚}{41211}
\saveTG{㹵}{41212}
\saveTG{㺡}{41212}
\saveTG{𦻮}{41212}
\saveTG{}{41212}
\saveTG{䞓}{41212}
\saveTG{𢃀}{41212}
\saveTG{𤣋}{41212}
\saveTG{𤠅}{41212}
\saveTG{𤞈}{41212}
\saveTG{𤟷}{41212}
\saveTG{𤡣}{41212}
\saveTG{𠒐}{41212}
\saveTG{𠄼}{41212}
\saveTG{𤞸}{41212}
\saveTG{獹}{41212}
\saveTG{𢁿}{41212}
\saveTG{䞑}{41212}
\saveTG{𧹬}{41214}
\saveTG{𤣔}{41214}
\saveTG{𤡝}{41214}
\saveTG{狂}{41214}
\saveTG{𤡰}{41214}
\saveTG{𤣐}{41214}
\saveTG{𤞂}{41214}
\saveTG{𤣗}{41214}
\saveTG{㺢}{41215}
\saveTG{𨿗}{41215}
\saveTG{兡}{41216}
\saveTG{𢂡}{41216}
\saveTG{𣎥}{41216}
\saveTG{𢄠}{41216}
\saveTG{𤠾}{41216}
\saveTG{狟}{41216}
\saveTG{𤭨}{41217}
\saveTG{瓻}{41217}
\saveTG{猇}{41217}
\saveTG{㼥}{41217}
\saveTG{𤮴}{41217}
\saveTG{㼹}{41217}
\saveTG{𤭩}{41217}
\saveTG{𪢸}{41217}
\saveTG{𪲭}{41217}
\saveTG{𤝌}{41217}
\saveTG{𤡚}{41217}
\saveTG{𡗐}{41217}
\saveTG{㼙}{41217}
\saveTG{𤜸}{41217}
\saveTG{𤜷}{41217}
\saveTG{𤞟}{41218}
\saveTG{𢃫}{41218}
\saveTG{𤣤}{41218}
\saveTG{狉}{41219}
\saveTG{𣍳}{41220}
\saveTG{𤿆}{41220}
\saveTG{帄}{41220}
\saveTG{𧹙}{41220}
\saveTG{𤠙}{41221}
\saveTG{𪻇}{41226}
\saveTG{𤢵}{41227}
\saveTG{獼}{41227}
\saveTG{獳}{41227}
\saveTG{狮}{41227}
\saveTG{獅}{41227}
\saveTG{獮}{41227}
\saveTG{𧄪}{41227}
\saveTG{䞕}{41227}
\saveTG{獁}{41227}
\saveTG{𥀭}{41227}
\saveTG{𤡤}{41227}
\saveTG{𤟨}{41227}
\saveTG{𤟝}{41227}
\saveTG{𧹸}{41227}
\saveTG{𦟖}{41227}
\saveTG{𢁢}{41227}
\saveTG{𠁑}{41230}
\saveTG{𤝃}{41231}
\saveTG{𪺼}{41232}
\saveTG{𤟔}{41232}
\saveTG{𤢓}{41232}
\saveTG{帳}{41232}
\saveTG{帪}{41232}
\saveTG{𤞱}{41232}
\saveTG{皯}{41240}
\saveTG{犽}{41240}
\saveTG{犴}{41240}
\saveTG{𢁗}{41240}
\saveTG{𤿍}{41240}
\saveTG{𤣒}{41242}
\saveTG{𤡹}{41242}
\saveTG{𢅥}{41243}
\saveTG{㡡}{41243}
\saveTG{𢃈}{41244}
\saveTG{𤜵}{41244}
\saveTG{𤿬}{41244}
\saveTG{𪾈}{41244}
\saveTG{𤢫}{41244}
\saveTG{𢃽}{41244}
\saveTG{㺛}{41244}
\saveTG{𤞬}{41244}
\saveTG{㹴}{41245}
\saveTG{𢅀}{41246}
\saveTG{㹿}{41246}
\saveTG{𢻦}{41247}
\saveTG{𣀙}{41247}
\saveTG{𣀧}{41247}
\saveTG{𢿣}{41247}
\saveTG{𢁵}{41247}
\saveTG{𢅼}{41247}
\saveTG{獶}{41247}
\saveTG{獿}{41247}
\saveTG{𢿨}{41247}
\saveTG{𣀍}{41247}
\saveTG{𤞕}{41247}
\saveTG{獩}{41253}
\saveTG{帖}{41260}
\saveTG{𥀸}{41261}
\saveTG{𤣍}{41261}
\saveTG{𧹿}{41261}
\saveTG{㹳}{41261}
\saveTG{𧃭}{41262}
\saveTG{𧄋}{41262}
\saveTG{𧹡}{41262}
\saveTG{𢃮}{41262}
\saveTG{𤟯}{41262}
\saveTG{㹮}{41262}
\saveTG{𤝓}{41262}
\saveTG{𤢿}{41262}
\saveTG{𤡼}{41262}
\saveTG{帞}{41262}
\saveTG{𤞏}{41264}
\saveTG{𢅔}{41264}
\saveTG{𤢗}{41264}
\saveTG{𧹭}{41266}
\saveTG{幅}{41266}
\saveTG{𤞜}{41269}
\saveTG{𢅅}{41282}
\saveTG{𤟦}{41282}
\saveTG{赪}{41282}
\saveTG{獗}{41282}
\saveTG{颟}{41282}
\saveTG{颇}{41282}
\saveTG{颧}{41282}
\saveTG{帧}{41282}
\saveTG{㿵}{41284}
\saveTG{𢃾}{41284}
\saveTG{𪩽}{41284}
\saveTG{顢}{41286}
\saveTG{幊}{41286}
\saveTG{頳}{41286}
\saveTG{赬}{41286}
\saveTG{𤢺}{41286}
\saveTG{𩒽}{41286}
\saveTG{𩓊}{41286}
\saveTG{𫖡}{41286}
\saveTG{𩕑}{41286}
\saveTG{𩕹}{41286}
\saveTG{𩕱}{41286}
\saveTG{𩖉}{41286}
\saveTG{𧄴}{41286}
\saveTG{顤}{41286}
\saveTG{𩕕}{41286}
\saveTG{幀}{41286}
\saveTG{幁}{41286}
\saveTG{顴}{41286}
\saveTG{頗}{41286}
\saveTG{顭}{41286}
\saveTG{𤡑}{41291}
\saveTG{幖}{41291}
\saveTG{狋}{41291}
\saveTG{𤡷}{41294}
\saveTG{𤠫}{41294}
\saveTG{𤢂}{41294}
\saveTG{獂}{41296}
\saveTG{𪦆}{41307}
\saveTG{𩍼}{41312}
\saveTG{𩽟}{41316}
\saveTG{鵟}{41327}
\saveTG{𢟇}{41331}
\saveTG{㤮}{41331}
\saveTG{𢢷}{41332}
\saveTG{㥈}{41336}
\saveTG{𩷬}{41336}
\saveTG{𪳝}{41336}
\saveTG{𢦁}{41338}
\saveTG{𪭅}{41338}
\saveTG{𠧀}{41402}
\saveTG{𡝕}{41411}
\saveTG{𠭅}{41411}
\saveTG{𡛘}{41411}
\saveTG{婔}{41411}
\saveTG{姃}{41411}
\saveTG{姫}{41412}
\saveTG{婭}{41412}
\saveTG{娙}{41412}
\saveTG{妩}{41412}
\saveTG{妧}{41412}
\saveTG{孋}{41412}
\saveTG{姬}{41412}
\saveTG{妅}{41412}
\saveTG{𡤌}{41412}
\saveTG{㜘}{41412}
\saveTG{𦷖}{41412}
\saveTG{𪥧}{41412}
\saveTG{娅}{41412}
\saveTG{𪥦}{41412}
\saveTG{𡛼}{41414}
\saveTG{妪}{41414}
\saveTG{𡉠}{41414}
\saveTG{𪦰}{41414}
\saveTG{𡟽}{41414}
\saveTG{娾}{41414}
\saveTG{姪}{41414}
\saveTG{𪦈}{41414}
\saveTG{𪌞}{41414}
\saveTG{𡞻}{41414}
\saveTG{姮}{41416}
\saveTG{嫗}{41416}
\saveTG{嫟}{41416}
\saveTG{𫜐}{41417}
\saveTG{𡚲}{41417}
\saveTG{𡜯}{41417}
\saveTG{婋}{41417}
\saveTG{𤝙}{41417}
\saveTG{𪌖}{41417}
\saveTG{姖}{41417}
\saveTG{𡛖}{41417}
\saveTG{𡛣}{41417}
\saveTG{㼬}{41417}
\saveTG{𡠣}{41417}
\saveTG{㛒}{41418}
\saveTG{𡣓}{41418}
\saveTG{𡞡}{41418}
\saveTG{㚰}{41419}
\saveTG{奵}{41420}
\saveTG{妸}{41420}
\saveTG{㚦}{41420}
\saveTG{婀}{41420}
\saveTG{𡡸}{41421}
\saveTG{𡠡}{41421}
\saveTG{𡠔}{41421}
\saveTG{𡝃}{41422}
\saveTG{𡤴}{41423}
\saveTG{𡤲}{41426}
\saveTG{𡛔}{41427}
\saveTG{𡟣}{41427}
\saveTG{𡞚}{41427}
\saveTG{𡞂}{41427}
\saveTG{𡢵}{41427}
\saveTG{𡢓}{41427}
\saveTG{嬭}{41427}
\saveTG{嫮}{41427}
\saveTG{媽}{41427}
\saveTG{嬬}{41427}
\saveTG{嫣}{41427}
\saveTG{麪}{41427}
\saveTG{𡡱}{41427}
\saveTG{麫}{41427}
\saveTG{𪋽}{41427}
\saveTG{䎟}{41427}
\saveTG{𡛦}{41427}
\saveTG{㛤}{41427}
\saveTG{𡛃}{41427}
\saveTG{𡜕}{41427}
\saveTG{𩤦}{41427}
\saveTG{𡟪}{41427}
\saveTG{𡢉}{41427}
\saveTG{𡚯}{41427}
\saveTG{𡢅}{41431}
\saveTG{𡢇}{41431}
\saveTG{嬺}{41431}
\saveTG{嫕}{41431}
\saveTG{𡛨}{41431}
\saveTG{𡝍}{41432}
\saveTG{𪍸}{41432}
\saveTG{𪥽}{41432}
\saveTG{𡠘}{41432}
\saveTG{𡤘}{41432}
\saveTG{娠}{41432}
\saveTG{妘}{41432}
\saveTG{𡣭}{41436}
\saveTG{婖}{41438}
\saveTG{㚥}{41440}
\saveTG{姸}{41440}
\saveTG{妍}{41440}
\saveTG{姧}{41440}
\saveTG{奸}{41440}
\saveTG{𪦒}{41440}
\saveTG{𪪂}{41440}
\saveTG{𪌃}{41440}
\saveTG{𡟡}{41441}
\saveTG{㛞}{41441}
\saveTG{𡤙}{41442}
\saveTG{㛅}{41442}
\saveTG{媷}{41443}
\saveTG{𡢙}{41443}
\saveTG{𡞾}{41444}
\saveTG{𪦗}{41444}
\saveTG{𡤆}{41444}
\saveTG{婹}{41444}
\saveTG{𪦦}{41445}
\saveTG{㛐}{41445}
\saveTG{𪍈}{41446}
\saveTG{𪍵}{41446}
\saveTG{㜤}{41446}
\saveTG{𡟵}{41446}
\saveTG{婥}{41446}
\saveTG{𡤕}{41447}
\saveTG{𡟺}{41447}
\saveTG{𪌏}{41447}
\saveTG{𢼬}{41447}
\saveTG{𤕝}{41447}
\saveTG{𪥩}{41447}
\saveTG{𢽞}{41447}
\saveTG{𢾴}{41447}
\saveTG{𡣜}{41448}
\saveTG{𡢩}{41448}
\saveTG{嫭}{41449}
\saveTG{㛁}{41449}
\saveTG{𢆠}{41449}
\saveTG{㛹}{41456}
\saveTG{𡣌}{41461}
\saveTG{娪}{41461}
\saveTG{孀}{41461}
\saveTG{𡜖}{41461}
\saveTG{𡡖}{41461}
\saveTG{𡠂}{41461}
\saveTG{𡤮}{41461}
\saveTG{㚲}{41462}
\saveTG{麵}{41462}
\saveTG{媔}{41462}
\saveTG{妬}{41462}
\saveTG{㐡}{41462}
\saveTG{䴴}{41462}
\saveTG{𡞎}{41462}
\saveTG{㛉}{41464}
\saveTG{𡜳}{41464}
\saveTG{𡢽}{41464}
\saveTG{娝}{41469}
\saveTG{婳}{41472}
\saveTG{𡠭}{41474}
\saveTG{媫}{41481}
\saveTG{㜧}{41482}
\saveTG{媆}{41484}
\saveTG{𡢢}{41484}
\saveTG{頍}{41486}
\saveTG{𡣧}{41486}
\saveTG{𩑱}{41486}
\saveTG{𩑛}{41486}
\saveTG{𩓐}{41486}
\saveTG{𫖟}{41486}
\saveTG{媜}{41486}
\saveTG{𪍕}{41486}
\saveTG{𩔵}{41486}
\saveTG{㛲}{41486}
\saveTG{𡤉}{41486}
\saveTG{𡟫}{41486}
\saveTG{妚}{41490}
\saveTG{嫖}{41491}
\saveTG{𡛭}{41491}
\saveTG{𡜔}{41492}
\saveTG{𡞮}{41494}
\saveTG{𡡥}{41494}
\saveTG{嫄}{41496}
\saveTG{𡞼}{41496}
\saveTG{轳}{41507}
\saveTG{𩋂}{41511}
\saveTG{𩎻}{41511}
\saveTG{䪊}{41511}
\saveTG{轭}{41512}
\saveTG{𩌕}{41512}
\saveTG{𩉮}{41512}
\saveTG{𩎉}{41512}
\saveTG{靰}{41512}
\saveTG{𩉱}{41512}
\saveTG{𩉯}{41512}
\saveTG{辄}{41512}
\saveTG{𩊢}{41512}
\saveTG{𩊇}{41513}
\saveTG{𩏊}{41514}
\saveTG{𩊞}{41514}
\saveTG{轾}{41514}
\saveTG{䩽}{41516}
\saveTG{韁}{41516}
\saveTG{𩎨}{41516}
\saveTG{𩌶}{41517}
\saveTG{𩌯}{41517}
\saveTG{𩍊}{41517}
\saveTG{𩍋}{41517}
\saveTG{𩉸}{41517}
\saveTG{𤭫}{41517}
\saveTG{𩉟}{41517}
\saveTG{𩍨}{41518}
\saveTG{𩊪}{41518}
\saveTG{𩏱}{41518}
\saveTG{𩎜}{41519}
\saveTG{𩊆}{41520}
\saveTG{靪}{41520}
\saveTG{轲}{41520}
\saveTG{𩊶}{41522}
\saveTG{𩌷}{41522}
\saveTG{𩍦}{41527}
\saveTG{𩉞}{41527}
\saveTG{𩍥}{41527}
\saveTG{䩻}{41527}
\saveTG{𩉣}{41527}
\saveTG{𩍄}{41527}
\saveTG{䩹}{41527}
\saveTG{𩉰}{41527}
\saveTG{䩫}{41527}
\saveTG{辆}{41527}
\saveTG{鞆}{41527}
\saveTG{𩌸}{41527}
\saveTG{𦨂}{41529}
\saveTG{鞐}{41531}
\saveTG{韆}{41531}
\saveTG{䩘}{41532}
\saveTG{䩨}{41532}
\saveTG{韔}{41532}
\saveTG{䩒}{41540}
\saveTG{𩊐}{41540}
\saveTG{𢆐}{41540}
\saveTG{靬}{41540}
\saveTG{𩎒}{41540}
\saveTG{轩}{41540}
\saveTG{𩌲}{41541}
\saveTG{𨐆}{41544}
\saveTG{鞕}{41546}
\saveTG{鞭}{41546}
\saveTG{𫖊}{41546}
\saveTG{𩌛}{41546}
\saveTG{𩋸}{41547}
\saveTG{𢾁}{41547}
\saveTG{𩌻}{41547}
\saveTG{𩉲}{41547}
\saveTG{𩋠}{41562}
\saveTG{䩞}{41562}
\saveTG{𩊘}{41562}
\saveTG{䪓}{41562}
\saveTG{𩍢}{41564}
\saveTG{𩋨}{41566}
\saveTG{𤜕}{41566}
\saveTG{辐}{41566}
\saveTG{𩌳}{41581}
\saveTG{𩋀}{41582}
\saveTG{𩎌}{41586}
\saveTG{𩌌}{41586}
\saveTG{𩔈}{41586}
\saveTG{𩋹}{41589}
\saveTG{𩏁}{41594}
\saveTG{𥌕}{41604}
\saveTG{𨡶}{41604}
\saveTG{𣬍}{41612}
\saveTG{𪴾}{41612}
\saveTG{𡄂}{41614}
\saveTG{𪴋}{41616}
\saveTG{𤭱}{41617}
\saveTG{瓳}{41617}
\saveTG{噽}{41619}
\saveTG{𠁂}{41620}
\saveTG{𩥞}{41627}
\saveTG{𠸙}{41641}
\saveTG{敼}{41647}
\saveTG{𠫅}{41647}
\saveTG{𢿿}{41647}
\saveTG{𢾀}{41647}
\saveTG{𣀓}{41647}
\saveTG{敧}{41647}
\saveTG{𢼣}{41647}
\saveTG{嚭}{41669}
\saveTG{颉}{41682}
\saveTG{頡}{41686}
\saveTG{𩕡}{41686}
\saveTG{䫑}{41686}
\saveTG{𩑲}{41686}
\saveTG{𩑶}{41686}
\saveTG{𩔩}{41686}
\saveTG{𩕯}{41686}
\saveTG{𡀆}{41690}
\saveTG{𤮿}{41712}
\saveTG{𦓇}{41716}
\saveTG{㽍}{41717}
\saveTG{𤮀}{41717}
\saveTG{𠳌}{41726}
\saveTG{𤮽}{41740}
\saveTG{㽑}{41746}
\saveTG{㪛}{41747}
\saveTG{𢻫}{41747}
\saveTG{㪑}{41747}
\saveTG{𢻴}{41747}
\saveTG{𦓜}{41784}
\saveTG{𩓜}{41786}
\saveTG{䫖}{41786}
\saveTG{𩑪}{41786}
\saveTG{䪺}{41786}
\saveTG{𩔭}{41786}
\saveTG{𩓹}{41786}
\saveTG{䞝}{41801}
\saveTG{𧼉}{41801}
\saveTG{𧺳}{41801}
\saveTG{𧻔}{41801}
\saveTG{䟐}{41801}
\saveTG{𧻚}{41801}
\saveTG{𧺹}{41801}
\saveTG{𫎾}{41801}
\saveTG{𧾪}{41801}
\saveTG{䞪}{41801}
\saveTG{𧻺}{41801}
\saveTG{𧺠}{41801}
\saveTG{𧽆}{41801}
\saveTG{𧻟}{41802}
\saveTG{𧽙}{41802}
\saveTG{䞙}{41802}
\saveTG{䟊}{41802}
\saveTG{𧼙}{41802}
\saveTG{𧼝}{41802}
\saveTG{𧾦}{41802}
\saveTG{𧺗}{41802}
\saveTG{𧼈}{41802}
\saveTG{趰}{41802}
\saveTG{𧺒}{41802}
\saveTG{𧻿}{41802}
\saveTG{𧺰}{41802}
\saveTG{𧽞}{41802}
\saveTG{𧺪}{41802}
\saveTG{䞛}{41802}
\saveTG{𧾇}{41803}
\saveTG{𧾧}{41803}
\saveTG{𧺦}{41803}
\saveTG{𧽐}{41804}
\saveTG{赶}{41804}
\saveTG{趠}{41804}
\saveTG{𧻈}{41804}
\saveTG{𧻀}{41804}
\saveTG{𧾬}{41804}
\saveTG{𧾤}{41804}
\saveTG{𧾛}{41804}
\saveTG{𧽘}{41804}
\saveTG{𧽼}{41804}
\saveTG{𧼟}{41806}
\saveTG{䞠}{41806}
\saveTG{䞸}{41806}
\saveTG{𧻳}{41806}
\saveTG{趈}{41806}
\saveTG{𧼸}{41806}
\saveTG{𧻙}{41806}
\saveTG{𧾮}{41806}
\saveTG{𧼞}{41807}
\saveTG{𧼰}{41807}
\saveTG{𧽸}{41808}
\saveTG{𧼥}{41808}
\saveTG{𧽤}{41809}
\saveTG{𤎪}{41809}
\saveTG{𧽺}{41809}
\saveTG{䞜}{41809}
\saveTG{𪏛}{41811}
\saveTG{𪎹}{41812}
\saveTG{𪏅}{41812}
\saveTG{𪏀}{41814}
\saveTG{𤭣}{41817}
\saveTG{𤮏}{41817}
\saveTG{㼽}{41817}
\saveTG{㼪}{41817}
\saveTG{𤮓}{41817}
\saveTG{𪏐}{41817}
\saveTG{𪏄}{41818}
\saveTG{𪏏}{41821}
\saveTG{𪎿}{41827}
\saveTG{𪏣}{41827}
\saveTG{𩣻}{41827}
\saveTG{𢿠}{41847}
\saveTG{𢽀}{41847}
\saveTG{㪎}{41847}
\saveTG{𢼦}{41847}
\saveTG{黇}{41860}
\saveTG{𩈧}{41862}
\saveTG{𪎾}{41864}
\saveTG{𫖺}{41882}
\saveTG{颠}{41882}
\saveTG{𩒓}{41886}
\saveTG{𩖃}{41886}
\saveTG{𩕦}{41886}
\saveTG{𩔄}{41886}
\saveTG{顛}{41886}
\saveTG{䫪}{41886}
\saveTG{䫶}{41886}
\saveTG{頰}{41886}
\saveTG{䫏}{41886}
\saveTG{𩔷}{41886}
\saveTG{𪏔}{41886}
\saveTG{𪏂}{41889}
\saveTG{𪏃}{41889}
\saveTG{𤉀}{41894}
\saveTG{𤍕}{41896}
\saveTG{𣏖}{41900}
\saveTG{栌}{41907}
\saveTG{杫}{41910}
\saveTG{櫳}{41911}
\saveTG{櫪}{41911}
\saveTG{柾}{41911}
\saveTG{榧}{41911}
\saveTG{棑}{41911}
\saveTG{𣒸}{41911}
\saveTG{㮜}{41911}
\saveTG{㭅}{41911}
\saveTG{框}{41911}
\saveTG{𠖑}{41911}
\saveTG{𣖘}{41912}
\saveTG{𣏴}{41912}
\saveTG{𪱲}{41912}
\saveTG{栕}{41912}
\saveTG{樝}{41912}
\saveTG{𣖍}{41912}
\saveTG{𣚛}{41912}
\saveTG{𣐵}{41912}
\saveTG{𣑙}{41912}
\saveTG{𣐝}{41912}
\saveTG{𣒘}{41912}
\saveTG{㮎}{41912}
\saveTG{𣒯}{41912}
\saveTG{杌}{41912}
\saveTG{枑}{41912}
\saveTG{㮓}{41912}
\saveTG{𣐛}{41912}
\saveTG{𣚒}{41912}
\saveTG{枙}{41912}
\saveTG{椏}{41912}
\saveTG{𣜂}{41912}
\saveTG{概}{41912}
\saveTG{槪}{41912}
\saveTG{杠}{41912}
\saveTG{桱}{41912}
\saveTG{櫨}{41912}
\saveTG{欐}{41912}
\saveTG{桠}{41912}
\saveTG{杬}{41912}
\saveTG{𣗝}{41914}
\saveTG{㯇}{41914}
\saveTG{𣔦}{41914}
\saveTG{𣘰}{41914}
\saveTG{𥒯}{41914}
\saveTG{㮒}{41914}
\saveTG{𣝽}{41914}
\saveTG{枉}{41914}
\saveTG{枢}{41914}
\saveTG{桎}{41914}
\saveTG{椻}{41914}
\saveTG{𣚼}{41914}
\saveTG{𣖭}{41914}
\saveTG{㰌}{41915}
\saveTG{㭱}{41915}
\saveTG{櫭}{41916}
\saveTG{樞}{41916}
\saveTG{橿}{41916}
\saveTG{桓}{41916}
\saveTG{櫮}{41916}
\saveTG{𣕲}{41916}
\saveTG{𣙁}{41916}
\saveTG{𣘗}{41916}
\saveTG{𣐎}{41917}
\saveTG{椃}{41917}
\saveTG{𣜵}{41917}
\saveTG{𣒽}{41917}
\saveTG{㭯}{41917}
\saveTG{𤭴}{41917}
\saveTG{柜}{41917}
\saveTG{𧈏}{41917}
\saveTG{𪲲}{41917}
\saveTG{𣛏}{41917}
\saveTG{𣡛}{41917}
\saveTG{梪}{41918}
\saveTG{櫃}{41918}
\saveTG{柩}{41918}
\saveTG{欞}{41918}
\saveTG{𣞘}{41918}
\saveTG{𣛼}{41918}
\saveTG{柸}{41919}
\saveTG{𣗫}{41919}
\saveTG{朾}{41920}
\saveTG{𣐘}{41920}
\saveTG{柯}{41920}
\saveTG{椼}{41921}
\saveTG{𪴟}{41921}
\saveTG{𣑴}{41921}
\saveTG{𪴺}{41921}
\saveTG{桁}{41921}
\saveTG{𣚀}{41922}
\saveTG{𣚷}{41922}
\saveTG{𣡣}{41926}
\saveTG{𣘁}{41926}
\saveTG{𣒍}{41926}
\saveTG{榪}{41927}
\saveTG{檷}{41927}
\saveTG{檽}{41927}
\saveTG{㯭}{41927}
\saveTG{杤}{41927}
\saveTG{杇}{41927}
\saveTG{櫔}{41927}
\saveTG{𣏜}{41927}
\saveTG{𣓈}{41927}
\saveTG{𪳣}{41927}
\saveTG{𥡗}{41927}
\saveTG{𣠟}{41927}
\saveTG{𣚊}{41927}
\saveTG{㮄}{41927}
\saveTG{𣠌}{41927}
\saveTG{𣠷}{41927}
\saveTG{𣏓}{41927}
\saveTG{𩣗}{41927}
\saveTG{欛}{41927}
\saveTG{柄}{41927}
\saveTG{樗}{41927}
\saveTG{槅}{41927}
\saveTG{枥}{41927}
\saveTG{㯊}{41927}
\saveTG{𣚶}{41927}
\saveTG{𣘣}{41927}
\saveTG{栭}{41927}
\saveTG{𪲔}{41927}
\saveTG{𣠽}{41927}
\saveTG{朽}{41927}
\saveTG{𣔯}{41927}
\saveTG{𣐣}{41929}
\saveTG{桛}{41931}
\saveTG{橒}{41931}
\saveTG{𣛥}{41931}
\saveTG{𣏇}{41931}
\saveTG{𣏳}{41931}
\saveTG{𣜡}{41931}
\saveTG{櫏}{41931}
\saveTG{𪴠}{41932}
\saveTG{㯌}{41932}
\saveTG{㯪}{41932}
\saveTG{𣚝}{41932}
\saveTG{桭}{41932}
\saveTG{𣛓}{41932}
\saveTG{㯫}{41932}
\saveTG{㯑}{41932}
\saveTG{椓}{41932}
\saveTG{棖}{41932}
\saveTG{枟}{41932}
\saveTG{櫖}{41936}
\saveTG{槱}{41936}
\saveTG{𣗇}{41936}
\saveTG{𪳉}{41936}
\saveTG{𪴐}{41937}
\saveTG{㮇}{41938}
\saveTG{橱}{41940}
\saveTG{栮}{41940}
\saveTG{杆}{41940}
\saveTG{枅}{41940}
\saveTG{榩}{41940}
\saveTG{枒}{41940}
\saveTG{杅}{41940}
\saveTG{櫉}{41940}
\saveTG{𣓣}{41941}
\saveTG{梇}{41941}
\saveTG{欇}{41941}
\saveTG{𣘅}{41942}
\saveTG{槈}{41943}
\saveTG{𣕩}{41944}
\saveTG{𣔃}{41944}
\saveTG{楆}{41944}
\saveTG{棹}{41946}
\saveTG{楩}{41946}
\saveTG{梗}{41946}
\saveTG{橝}{41946}
\saveTG{𦅕}{41946}
\saveTG{𣒉}{41947}
\saveTG{𣗀}{41947}
\saveTG{𣖀}{41947}
\saveTG{𪳍}{41947}
\saveTG{榎}{41947}
\saveTG{𢿷}{41947}
\saveTG{𢼢}{41947}
\saveTG{𣏽}{41947}
\saveTG{㪖}{41947}
\saveTG{櫌}{41947}
\saveTG{𣖔}{41947}
\saveTG{㪔}{41947}
\saveTG{𪳧}{41947}
\saveTG{𣖳}{41948}
\saveTG{㯉}{41949}
\saveTG{枰}{41949}
\saveTG{𣛲}{41949}
\saveTG{𪲻}{41950}
\saveTG{𣝱}{41953}
\saveTG{檅}{41953}
\saveTG{𣟝}{41955}
\saveTG{𣘭}{41956}
\saveTG{𣏛}{41959}
\saveTG{枮}{41960}
\saveTG{樐}{41960}
\saveTG{𪴜}{41961}
\saveTG{𪲼}{41961}
\saveTG{榗}{41961}
\saveTG{梧}{41961}
\saveTG{橬}{41961}
\saveTG{櫺}{41961}
\saveTG{檑}{41961}
\saveTG{𣔧}{41962}
\saveTG{㰁}{41962}
\saveTG{𪳠}{41962}
\saveTG{𣠚}{41962}
\saveTG{栢}{41962}
\saveTG{橊}{41962}
\saveTG{檆}{41962}
\saveTG{柘}{41962}
\saveTG{梄}{41964}
\saveTG{𣐸}{41964}
\saveTG{栖}{41964}
\saveTG{𣠄}{41966}
\saveTG{楅}{41966}
\saveTG{桮}{41969}
\saveTG{𣕨}{41969}
\saveTG{𣘝}{41971}
\saveTG{𣔨}{41977}
\saveTG{樰}{41977}
\saveTG{桺}{41977}
\saveTG{𣘩}{41981}
\saveTG{𣝛}{41981}
\saveTG{𪴗}{41981}
\saveTG{槚}{41982}
\saveTG{𣏡}{41982}
\saveTG{𫖷}{41982}
\saveTG{桢}{41982}
\saveTG{橛}{41982}
\saveTG{𣕹}{41982}
\saveTG{𩖗}{41982}
\saveTG{𣏿}{41984}
\saveTG{𣝓}{41984}
\saveTG{𣒟}{41985}
\saveTG{𩓠}{41986}
\saveTG{䫐}{41986}
\saveTG{䫙}{41986}
\saveTG{𩕐}{41986}
\saveTG{𩔑}{41986}
\saveTG{𩕒}{41986}
\saveTG{𪴣}{41986}
\saveTG{𣛪}{41986}
\saveTG{㰜}{41986}
\saveTG{𣛿}{41986}
\saveTG{𣡲}{41986}
\saveTG{𣚁}{41986}
\saveTG{𣟩}{41986}
\saveTG{楨}{41986}
\saveTG{櫇}{41986}
\saveTG{顂}{41986}
\saveTG{檟}{41986}
\saveTG{槓}{41986}
\saveTG{𣟤}{41986}
\saveTG{𣟓}{41986}
\saveTG{䫴}{41986}
\saveTG{㰋}{41986}
\saveTG{𩓡}{41986}
\saveTG{𥢙}{41986}
\saveTG{𩒮}{41986}
\saveTG{𣠢}{41988}
\saveTG{樮}{41989}
\saveTG{杯}{41990}
\saveTG{標}{41991}
\saveTG{标}{41991}
\saveTG{𣐹}{41991}
\saveTG{柡}{41992}
\saveTG{𣚡}{41992}
\saveTG{㰃}{41993}
\saveTG{𣖪}{41994}
\saveTG{𣙽}{41994}
\saveTG{𪴖}{41994}
\saveTG{𣗖}{41994}
\saveTG{㯨}{41994}
\saveTG{槕}{41994}
\saveTG{𥤔}{41995}
\saveTG{榞}{41996}
\saveTG{𣙏}{41996}
\saveTG{𩕌}{41996}
\saveTG{刈}{42000}
\saveTG{劜}{42010}
\saveTG{𡯇}{42010}
\saveTG{𡯭}{42011}
\saveTG{𡯪}{42011}
\saveTG{尰}{42011}
\saveTG{𡰢}{42012}
\saveTG{𡰑}{42012}
\saveTG{𢒂}{42012}
\saveTG{𡯆}{42013}
\saveTG{㝽}{42013}
\saveTG{𤫯}{42013}
\saveTG{㞂}{42014}
\saveTG{𡯵}{42014}
\saveTG{𡯢}{42016}
\saveTG{𡰖}{42018}
\saveTG{沊}{42019}
\saveTG{𠛌}{42020}
\saveTG{𣂒}{42021}
\saveTG{𠦉}{42021}
\saveTG{𢁜}{42030}
\saveTG{𤞲}{42047}
\saveTG{𩠒}{42062}
\saveTG{𪟌}{42100}
\saveTG{𠝥}{42100}
\saveTG{𠟜}{42100}
\saveTG{𪣫}{42100}
\saveTG{𡊻}{42100}
\saveTG{𡌀}{42100}
\saveTG{𡎗}{42100}
\saveTG{𦳲}{42100}
\saveTG{𪣛}{42100}
\saveTG{㔈}{42100}
\saveTG{𢅚}{42100}
\saveTG{𠞱}{42100}
\saveTG{圳}{42100}
\saveTG{刲}{42100}
\saveTG{𡍘}{42100}
\saveTG{𡍫}{42100}
\saveTG{𪣏}{42100}
\saveTG{𨧂}{42100}
\saveTG{𥂌}{42102}
\saveTG{𥃁}{42102}
\saveTG{𡌁}{42104}
\saveTG{堑}{42104}
\saveTG{𡉴}{42107}
\saveTG{錾}{42109}
\saveTG{圠}{42110}
\saveTG{亄}{42110}
\saveTG{𡐥}{42111}
\saveTG{𡋒}{42112}
\saveTG{𡏓}{42112}
\saveTG{坵}{42112}
\saveTG{𡒁}{42112}
\saveTG{垗}{42113}
\saveTG{㙄}{42114}
\saveTG{𡍛}{42114}
\saveTG{圫}{42114}
\saveTG{𡋨}{42114}
\saveTG{堹}{42115}
\saveTG{埵}{42115}
\saveTG{𡓯}{42115}
\saveTG{𪤛}{42115}
\saveTG{𣰛}{42115}
\saveTG{墔}{42115}
\saveTG{𡒏}{42116}
\saveTG{𡓍}{42116}
\saveTG{墱}{42116}
\saveTG{𧰠}{42116}
\saveTG{𠃸}{42117}
\saveTG{𪣾}{42117}
\saveTG{𡋸}{42117}
\saveTG{𡏚}{42117}
\saveTG{垲}{42117}
\saveTG{㘩}{42117}
\saveTG{𪣂}{42117}
\saveTG{𡋀}{42117}
\saveTG{塏}{42118}
\saveTG{墱}{42118}
\saveTG{斳}{42121}
\saveTG{𡐛}{42121}
\saveTG{𡒧}{42121}
\saveTG{𣂷}{42121}
\saveTG{圻}{42121}
\saveTG{𡐶}{42122}
\saveTG{彭}{42122}
\saveTG{𪤆}{42127}
\saveTG{𡐫}{42127}
\saveTG{壪}{42127}
\saveTG{墧}{42127}
\saveTG{塴}{42127}
\saveTG{𪣜}{42127}
\saveTG{𡐇}{42127}
\saveTG{䵺}{42127}
\saveTG{𪉅}{42127}
\saveTG{塉}{42127}
\saveTG{𪤔}{42127}
\saveTG{埁}{42127}
\saveTG{𡐮}{42127}
\saveTG{㙖}{42127}
\saveTG{𡎚}{42127}
\saveTG{𡍀}{42127}
\saveTG{𪀦}{42127}
\saveTG{𪣵}{42127}
\saveTG{𡐟}{42127}
\saveTG{𫛠}{42127}
\saveTG{坬}{42130}
\saveTG{𪢿}{42130}
\saveTG{壎}{42131}
\saveTG{𡑜}{42132}
\saveTG{𡍢}{42133}
\saveTG{𤬗}{42133}
\saveTG{蜤}{42136}
\saveTG{蚻}{42136}
\saveTG{蟴}{42136}
\saveTG{𧎀}{42136}
\saveTG{蟚}{42136}
\saveTG{𪣇}{42137}
\saveTG{𡑐}{42138}
\saveTG{坻}{42140}
\saveTG{坁}{42140}
\saveTG{圲}{42140}
\saveTG{𡋺}{42141}
\saveTG{㙐}{42141}
\saveTG{堓}{42141}
\saveTG{坼}{42141}
\saveTG{埏}{42141}
\saveTG{𡒇}{42142}
\saveTG{埓}{42142}
\saveTG{𡌠}{42143}
\saveTG{𡊯}{42144}
\saveTG{𡐕}{42145}
\saveTG{𡍰}{42147}
\saveTG{𡓙}{42147}
\saveTG{𪤫}{42147}
\saveTG{堫}{42147}
\saveTG{垺}{42147}
\saveTG{坂}{42147}
\saveTG{墢}{42147}
\saveTG{𡔷}{42147}
\saveTG{𡔖}{42148}
\saveTG{𡓌}{42148}
\saveTG{埒}{42149}
\saveTG{垀}{42149}
\saveTG{𡍹}{42150}
\saveTG{㙨}{42153}
\saveTG{垢}{42161}
\saveTG{𡎵}{42162}
\saveTG{𡏨}{42162}
\saveTG{𡏥}{42162}
\saveTG{堦}{42162}
\saveTG{堖}{42162}
\saveTG{𡌐}{42162}
\saveTG{𡍗}{42163}
\saveTG{𡌩}{42163}
\saveTG{𧰀}{42164}
\saveTG{𡎆}{42164}
\saveTG{𡤊}{42164}
\saveTG{𡿤}{42167}
\saveTG{墦}{42169}
\saveTG{𢑵}{42169}
\saveTG{𡎟}{42169}
\saveTG{圸}{42170}
\saveTG{𨪺}{42172}
\saveTG{壣}{42172}
\saveTG{𡎫}{42172}
\saveTG{㘪}{42172}
\saveTG{𡏟}{42174}
\saveTG{𡍪}{42177}
\saveTG{塪}{42177}
\saveTG{㙃}{42181}
\saveTG{塡}{42181}
\saveTG{𡏛}{42184}
\saveTG{𡎝}{42184}
\saveTG{㙸}{42185}
\saveTG{墣}{42185}
\saveTG{𡒻}{42186}
\saveTG{𡎅}{42189}
\saveTG{塖}{42191}
\saveTG{𡋔}{42193}
\saveTG{𡍥}{42194}
\saveTG{𡏮}{42194}
\saveTG{𡏪}{42194}
\saveTG{埰}{42194}
\saveTG{𡎽}{42194}
\saveTG{𤠌}{42194}
\saveTG{𡑿}{42195}
\saveTG{𠠈}{42200}
\saveTG{𠠉}{42200}
\saveTG{𢃉}{42200}
\saveTG{𢆂}{42200}
\saveTG{𦙢}{42200}
\saveTG{𠜗}{42200}
\saveTG{𤠧}{42200}
\saveTG{𤞊}{42200}
\saveTG{𠟋}{42200}
\saveTG{𠛘}{42200}
\saveTG{𠝐}{42200}
\saveTG{𠞝}{42200}
\saveTG{猘}{42200}
\saveTG{蒯}{42200}
\saveTG{𤟆}{42200}
\saveTG{𤜟}{42200}
\saveTG{𤞹}{42200}
\saveTG{𤟰}{42200}
\saveTG{刳}{42200}
\saveTG{𪟆}{42200}
\saveTG{㓟}{42200}
\saveTG{𪟉}{42200}
\saveTG{𠞘}{42200}
\saveTG{猁}{42200}
\saveTG{𤢾}{42200}
\saveTG{剋}{42210}
\saveTG{犼}{42210}
\saveTG{兞}{42211}
\saveTG{𠒙}{42212}
\saveTG{獵}{42212}
\saveTG{㠼}{42212}
\saveTG{𤢪}{42212}
\saveTG{𤜺}{42212}
\saveTG{𤞘}{42213}
\saveTG{狣}{42213}
\saveTG{狅}{42214}
\saveTG{狴}{42214}
\saveTG{毻}{42214}
\saveTG{𢂧}{42214}
\saveTG{兛}{42214}
\saveTG{氋}{42214}
\saveTG{𣮠}{42215}
\saveTG{㡖}{42215}
\saveTG{氋}{42215}
\saveTG{獕}{42215}
\saveTG{𤜻}{42217}
\saveTG{𤜤}{42217}
\saveTG{𤝄}{42217}
\saveTG{𠘿}{42217}
\saveTG{㡗}{42217}
\saveTG{𤝤}{42217}
\saveTG{𤝭}{42217}
\saveTG{𤠗}{42217}
\saveTG{𢁦}{42217}
\saveTG{㠲}{42217}
\saveTG{𢂮}{42217}
\saveTG{𠃴}{42217}
\saveTG{𠒟}{42217}
\saveTG{𢅃}{42217}
\saveTG{㡠}{42218}
\saveTG{𤠲}{42218}
\saveTG{𠒭}{42219}
\saveTG{𣃄}{42221}
\saveTG{𤠥}{42221}
\saveTG{𤠄}{42221}
\saveTG{獑}{42221}
\saveTG{狾}{42221}
\saveTG{㹞}{42221}
\saveTG{玂}{42221}
\saveTG{𤡭}{42222}
\saveTG{𢁘}{42222}
\saveTG{𤞝}{42222}
\saveTG{𤞃}{42223}
\saveTG{獢}{42227}
\saveTG{𤢠}{42227}
\saveTG{𤣑}{42227}
\saveTG{𢄹}{42227}
\saveTG{𪩼}{42227}
\saveTG{㺔}{42227}
\saveTG{猯}{42227}
\saveTG{𤝣}{42230}
\saveTG{瓠}{42230}
\saveTG{狐}{42230}
\saveTG{𤝎}{42230}
\saveTG{𤔻}{42230}
\saveTG{𤔾}{42230}
\saveTG{𤬄}{42230}
\saveTG{𢁬}{42230}
\saveTG{𤜶}{42230}
\saveTG{㺈}{42231}
\saveTG{獯}{42231}
\saveTG{𢄃}{42232}
\saveTG{𤞅}{42232}
\saveTG{𪼵}{42233}
\saveTG{𤢦}{42234}
\saveTG{帐}{42234}
\saveTG{𢁤}{42237}
\saveTG{㡥}{42237}
\saveTG{𤝑}{42237}
\saveTG{狿}{42241}
\saveTG{帲}{42241}
\saveTG{㹶}{42241}
\saveTG{𤟉}{42241}
\saveTG{𤞙}{42243}
\saveTG{𦛷}{42243}
\saveTG{㹻}{42244}
\saveTG{𦡝}{42247}
\saveTG{𫆫}{42247}
\saveTG{㹝}{42247}
\saveTG{猨}{42247}
\saveTG{猣}{42247}
\saveTG{㡪}{42247}
\saveTG{𤝬}{42247}
\saveTG{𤟗}{42247}
\saveTG{𤝘}{42249}
\saveTG{𪩹}{42250}
\saveTG{𤟤}{42256}
\saveTG{𦽰}{42257}
\saveTG{猙}{42257}
\saveTG{𥀑}{42261}
\saveTG{𢄰}{42261}
\saveTG{𥀋}{42261}
\saveTG{㹺}{42262}
\saveTG{㺁}{42262}
\saveTG{𦞉}{42262}
\saveTG{𢃕}{42263}
\saveTG{㿳}{42263}
\saveTG{𪻂}{42264}
\saveTG{𦧘}{42264}
\saveTG{𤞳}{42264}
\saveTG{狧}{42264}
\saveTG{㡒}{42264}
\saveTG{㺕}{42269}
\saveTG{𧹰}{42269}
\saveTG{幡}{42269}
\saveTG{𤜬}{42270}
\saveTG{𣨗}{42271}
\saveTG{猺}{42272}
\saveTG{𤝾}{42272}
\saveTG{㺦}{42272}
\saveTG{幍}{42277}
\saveTG{𣣐}{42282}
\saveTG{𤣈}{42283}
\saveTG{𤟑}{42284}
\saveTG{𢃯}{42284}
\saveTG{𤠓}{42284}
\saveTG{𢁱}{42284}
\saveTG{猤}{42284}
\saveTG{幞}{42285}
\saveTG{獛}{42285}
\saveTG{𤢽}{42286}
\saveTG{綔}{42293}
\saveTG{𤡨}{42293}
\saveTG{猻}{42293}
\saveTG{𤢴}{42294}
\saveTG{𤟖}{42294}
\saveTG{㺐}{42295}
\saveTG{㡤}{42295}
\saveTG{刌}{42300}
\saveTG{𪁻}{42327}
\saveTG{𪆗}{42327}
\saveTG{𢞪}{42331}
\saveTG{𪬟}{42332}
\saveTG{𢜣}{42332}
\saveTG{惁}{42332}
\saveTG{𪫾}{42335}
\saveTG{䰊}{42342}
\saveTG{𠛟}{42400}
\saveTG{婣}{42400}
\saveTG{𡜇}{42400}
\saveTG{𪌱}{42400}
\saveTG{𠜽}{42400}
\saveTG{𠛄}{42400}
\saveTG{𪟁}{42400}
\saveTG{𠟝}{42400}
\saveTG{嬼}{42400}
\saveTG{娳}{42400}
\saveTG{荆}{42400}
\saveTG{劐}{42400}
\saveTG{𡛅}{42400}
\saveTG{𠚰}{42400}
\saveTG{𡢱}{42400}
\saveTG{𡞸}{42400}
\saveTG{㓫}{42400}
\saveTG{𦗿}{42401}
\saveTG{𡛢}{42401}
\saveTG{𪍖}{42401}
\saveTG{㲍}{42401}
\saveTG{𦗭}{42401}
\saveTG{𠭐}{42401}
\saveTG{乹}{42410}
\saveTG{妣}{42410}
\saveTG{𡛆}{42412}
\saveTG{𪥹}{42412}
\saveTG{𡛪}{42412}
\saveTG{𡢭}{42412}
\saveTG{㚱}{42412}
\saveTG{𡢋}{42412}
\saveTG{𡜣}{42412}
\saveTG{𪎀}{42412}
\saveTG{𡞢}{42412}
\saveTG{𡠒}{42413}
\saveTG{𡣦}{42413}
\saveTG{𫗌}{42413}
\saveTG{姚}{42413}
\saveTG{姙}{42414}
\saveTG{婬}{42414}
\saveTG{𡝇}{42414}
\saveTG{奼}{42414}
\saveTG{妊}{42414}
\saveTG{娷}{42415}
\saveTG{媑}{42415}
\saveTG{㲨}{42415}
\saveTG{𣯬}{42415}
\saveTG{㜠}{42415}
\saveTG{𣮡}{42415}
\saveTG{𡚧}{42417}
\saveTG{㚪}{42417}
\saveTG{㛂}{42417}
\saveTG{𡜮}{42417}
\saveTG{㜉}{42417}
\saveTG{𪋼}{42417}
\saveTG{𡜞}{42417}
\saveTG{𡠰}{42417}
\saveTG{𪌈}{42417}
\saveTG{𣭷}{42417}
\saveTG{𪌪}{42417}
\saveTG{𤕥}{42417}
\saveTG{𡝦}{42417}
\saveTG{𠃱}{42417}
\saveTG{𪌂}{42417}
\saveTG{𡰄}{42418}
\saveTG{㜐}{42418}
\saveTG{嬁}{42418}
\saveTG{麷}{42418}
\saveTG{𪦢}{42419}
\saveTG{𠚽}{42420}
\saveTG{𪌍}{42421}
\saveTG{𡡒}{42421}
\saveTG{𡝊}{42421}
\saveTG{𪥭}{42421}
\saveTG{妡}{42421}
\saveTG{𪌷}{42422}
\saveTG{㣏}{42422}
\saveTG{嬌}{42427}
\saveTG{𡤶}{42427}
\saveTG{嬀}{42427}
\saveTG{媏}{42427}
\saveTG{㛢}{42427}
\saveTG{𡠱}{42427}
\saveTG{𪍷}{42427}
\saveTG{𧄳}{42427}
\saveTG{𡡈}{42427}
\saveTG{𡤻}{42427}
\saveTG{𡜏}{42427}
\saveTG{𡣸}{42427}
\saveTG{𡟥}{42427}
\saveTG{𡡗}{42427}
\saveTG{孈}{42427}
\saveTG{娇}{42428}
\saveTG{瓡}{42430}
\saveTG{𡛴}{42431}
\saveTG{𡡺}{42432}
\saveTG{姂}{42432}
\saveTG{𡣘}{42432}
\saveTG{𪍔}{42432}
\saveTG{𡜁}{42433}
\saveTG{媸}{42436}
\saveTG{𡢆}{42436}
\saveTG{𡚸}{42437}
\saveTG{𪦧}{42437}
\saveTG{𪥻}{42437}
\saveTG{𡡁}{42439}
\saveTG{㜻}{42439}
\saveTG{奷}{42440}
\saveTG{婩}{42441}
\saveTG{娫}{42441}
\saveTG{娗}{42441}
\saveTG{𡝟}{42442}
\saveTG{𪌳}{42443}
\saveTG{𡛜}{42443}
\saveTG{娞}{42444}
\saveTG{婑}{42444}
\saveTG{𡞬}{42444}
\saveTG{嬡}{42447}
\saveTG{𪦣}{42447}
\saveTG{娐}{42447}
\saveTG{媛}{42447}
\saveTG{𪌆}{42447}
\saveTG{𡚼}{42447}
\saveTG{𪥲}{42447}
\saveTG{𡞧}{42447}
\saveTG{㛵}{42447}
\saveTG{嫒}{42447}
\saveTG{䴸}{42447}
\saveTG{𡤇}{42448}
\saveTG{𡛚}{42449}
\saveTG{𡛊}{42450}
\saveTG{𪌙}{42450}
\saveTG{𪌐}{42450}
\saveTG{𡡤}{42453}
\saveTG{𡟶}{42455}
\saveTG{𡡞}{42457}
\saveTG{姤}{42461}
\saveTG{媘}{42462}
\saveTG{𩠖}{42462}
\saveTG{㛧}{42462}
\saveTG{㛴}{42462}
\saveTG{㛥}{42463}
\saveTG{𪦅}{42463}
\saveTG{𡝜}{42464}
\saveTG{𡟈}{42464}
\saveTG{𪌩}{42464}
\saveTG{𡜶}{42464}
\saveTG{姡}{42464}
\saveTG{婚}{42464}
\saveTG{𡡾}{42467}
\saveTG{𡡻}{42467}
\saveTG{嬏}{42469}
\saveTG{奾}{42470}
\saveTG{䴮}{42470}
\saveTG{𡛛}{42472}
\saveTG{𡜪}{42472}
\saveTG{媱}{42472}
\saveTG{𡟢}{42472}
\saveTG{嫍}{42477}
\saveTG{㛼}{42477}
\saveTG{娦}{42481}
\saveTG{㜎}{42484}
\saveTG{𡝢}{42484}
\saveTG{妖}{42484}
\saveTG{嬽}{42484}
\saveTG{𡞳}{42484}
\saveTG{𡡐}{42485}
\saveTG{㜱}{42486}
\saveTG{𡢍}{42486}
\saveTG{𪍰}{42488}
\saveTG{𡞊}{42489}
\saveTG{㛶}{42489}
\saveTG{𪦤}{42493}
\saveTG{𦁝}{42493}
\saveTG{𡤣}{42493}
\saveTG{姀}{42494}
\saveTG{𪌗}{42494}
\saveTG{㜰}{42494}
\saveTG{婇}{42494}
\saveTG{𪍨}{42495}
\saveTG{𡡊}{42495}
\saveTG{靷}{42500}
\saveTG{鞩}{42500}
\saveTG{𩋷}{42500}
\saveTG{𩊡}{42500}
\saveTG{𪮃}{42502}
\saveTG{轧}{42510}
\saveTG{𩉫}{42512}
\saveTG{𩍽}{42512}
\saveTG{𩋍}{42512}
\saveTG{𩉴}{42512}
\saveTG{鞉}{42513}
\saveTG{𩏘}{42514}
\saveTG{𩊰}{42514}
\saveTG{𩎄}{42515}
\saveTG{𩉪}{42515}
\saveTG{𩌩}{42515}
\saveTG{䪉}{42516}
\saveTG{𩏠}{42516}
\saveTG{𩌋}{42517}
\saveTG{𩏔}{42517}
\saveTG{𩍐}{42518}
\saveTG{斩}{42521}
\saveTG{䩢}{42521}
\saveTG{靳}{42521}
\saveTG{鞽}{42527}
\saveTG{𩏬}{42527}
\saveTG{𩌮}{42527}
\saveTG{𩎇}{42527}
\saveTG{䪎}{42527}
\saveTG{𩍺}{42527}
\saveTG{𩍇}{42527}
\saveTG{鞒}{42528}
\saveTG{轿}{42528}
\saveTG{鞃}{42530}
\saveTG{䩝}{42530}
\saveTG{𩋅}{42538}
\saveTG{䩚}{42540}
\saveTG{}{42540}
\saveTG{𩉬}{42540}
\saveTG{䩠}{42541}
\saveTG{䩥}{42541}
\saveTG{𩎚}{42541}
\saveTG{鞖}{42544}
\saveTG{𩋫}{42544}
\saveTG{𩎋}{42547}
\saveTG{𩏅}{42547}
\saveTG{𩋯}{42547}
\saveTG{𩊤}{42547}
\saveTG{𩍱}{42547}
\saveTG{𩍫}{42547}
\saveTG{𩎢}{42547}
\saveTG{轷}{42549}
\saveTG{鞿}{42553}
\saveTG{𩊝}{42561}
\saveTG{𩋈}{42562}
\saveTG{𩋧}{42562}
\saveTG{𩏀}{42563}
\saveTG{𩋝}{42563}
\saveTG{𩎽}{42563}
\saveTG{辎}{42563}
\saveTG{𩏋}{42564}
\saveTG{鞜}{42569}
\saveTG{䪛}{42569}
\saveTG{𩊼}{42572}
\saveTG{鞱}{42577}
\saveTG{韜}{42577}
\saveTG{𩌦}{42582}
\saveTG{鞵}{42584}
\saveTG{䪁}{42585}
\saveTG{𩍩}{42585}
\saveTG{𩍵}{42586}
\saveTG{鞂}{42594}
\saveTG{轹}{42594}
\saveTG{𩏙}{42595}
\saveTG{𩍀}{42595}
\saveTG{𠟩}{42600}
\saveTG{𡔠}{42600}
\saveTG{㓤}{42600}
\saveTG{𠝛}{42600}
\saveTG{𠟌}{42600}
\saveTG{𠟽}{42600}
\saveTG{剒}{42600}
\saveTG{剞}{42600}
\saveTG{剳}{42600}
\saveTG{𪟈}{42600}
\saveTG{𧬊}{42601}
\saveTG{𥓊}{42602}
\saveTG{皙}{42602}
\saveTG{𣇮}{42602}
\saveTG{𠵍}{42602}
\saveTG{𥕽}{42602}
\saveTG{晳}{42602}
\saveTG{暂}{42602}
\saveTG{𨢹}{42604}
\saveTG{𣯈}{42615}
\saveTG{𧃵}{42616}
\saveTG{𣯆}{42617}
\saveTG{𪜕}{42617}
\saveTG{㐖}{42617}
\saveTG{斱}{42621}
\saveTG{斮}{42621}
\saveTG{𧪷}{42621}
\saveTG{𣃈}{42621}
\saveTG{𤗡}{42627}
\saveTG{𫊔}{42630}
\saveTG{㼋}{42630}
\saveTG{𤔎}{42632}
\saveTG{𤫶}{42633}
\saveTG{𣤺}{42682}
\saveTG{𦁬}{42693}
\saveTG{剦}{42700}
\saveTG{刦}{42700}
\saveTG{𠝻}{42700}
\saveTG{𠛗}{42700}
\saveTG{𠞒}{42700}
\saveTG{𠞗}{42700}
\saveTG{𠜩}{42700}
\saveTG{𤯉}{42712}
\saveTG{䖔}{42712}
\saveTG{𣰘}{42715}
\saveTG{𣰌}{42715}
\saveTG{甏}{42717}
\saveTG{㼭}{42717}
\saveTG{𣭢}{42717}
\saveTG{𣰐}{42717}
\saveTG{㽄}{42717}
\saveTG{长}{42730}
\saveTG{𤯄}{42740}
\saveTG{𠬉}{42757}
\saveTG{甛}{42764}
\saveTG{𤳗}{42769}
\saveTG{赳}{42800}
\saveTG{𠝟}{42800}
\saveTG{剘}{42800}
\saveTG{𠞦}{42800}
\saveTG{𠟰}{42800}
\saveTG{𠟒}{42800}
\saveTG{𠠄}{42800}
\saveTG{𧼤}{42800}
\saveTG{𧼕}{42800}
\saveTG{趔}{42800}
\saveTG{㓨}{42800}
\saveTG{𠠘}{42800}
\saveTG{𧽊}{42801}
\saveTG{䞾}{42801}
\saveTG{𧻁}{42801}
\saveTG{𧾓}{42801}
\saveTG{𧼩}{42801}
\saveTG{𧽨}{42801}
\saveTG{𧺼}{42801}
\saveTG{𧺇}{42801}
\saveTG{𧾳}{42801}
\saveTG{𧾊}{42801}
\saveTG{𧻆}{42801}
\saveTG{𧺲}{42801}
\saveTG{𧺊}{42801}
\saveTG{趒}{42801}
\saveTG{𧼋}{42802}
\saveTG{𫏐}{42802}
\saveTG{𨅘}{42802}
\saveTG{𧽠}{42802}
\saveTG{𧽯}{42802}
\saveTG{𧺈}{42802}
\saveTG{𧻸}{42802}
\saveTG{𧾁}{42802}
\saveTG{趫}{42802}
\saveTG{䞬}{42802}
\saveTG{𧽄}{42802}
\saveTG{赾}{42802}
\saveTG{𧼴}{42802}
\saveTG{𧾖}{42802}
\saveTG{𧽶}{42802}
\saveTG{𧻽}{42803}
\saveTG{𧾶}{42803}
\saveTG{𧺵}{42803}
\saveTG{趆}{42804}
\saveTG{赿}{42804}
\saveTG{𧺛}{42804}
\saveTG{䞣}{42804}
\saveTG{𫎳}{42804}
\saveTG{䞯}{42804}
\saveTG{𧼬}{42804}
\saveTG{𥏷}{42804}
\saveTG{𧽃}{42804}
\saveTG{𪥟}{42804}
\saveTG{䟇}{42805}
\saveTG{𧼗}{42805}
\saveTG{䞧}{42806}
\saveTG{𧽂}{42806}
\saveTG{趏}{42806}
\saveTG{䞺}{42806}
\saveTG{赸}{42807}
\saveTG{𧼶}{42807}
\saveTG{赳}{42807}
\saveTG{趉}{42807}
\saveTG{𧼆}{42809}
\saveTG{䟁}{42809}
\saveTG{䟏}{42809}
\saveTG{𧽧}{42809}
\saveTG{𤏍}{42809}
\saveTG{㔮}{42812}
\saveTG{𣰰}{42815}
\saveTG{㲟}{42815}
\saveTG{𣰋}{42815}
\saveTG{𣭶}{42815}
\saveTG{𪎷}{42815}
\saveTG{𪏘}{42815}
\saveTG{𣮵}{42817}
\saveTG{𠞮}{42820}
\saveTG{斯}{42821}
\saveTG{𣯻}{42825}
\saveTG{䵎}{42827}
\saveTG{𪎺}{42831}
\saveTG{𧌝}{42836}
\saveTG{𩠣}{42862}
\saveTG{𩔯}{42886}
\saveTG{𪏩}{42889}
\saveTG{𪏫}{42891}
\saveTG{𪏦}{42899}
\saveTG{剎}{42900}
\saveTG{𠝚}{42900}
\saveTG{𠜀}{42900}
\saveTG{𠠕}{42900}
\saveTG{𠝝}{42900}
\saveTG{𠝺}{42900}
\saveTG{𠟴}{42900}
\saveTG{栵}{42900}
\saveTG{梸}{42900}
\saveTG{杊}{42900}
\saveTG{棩}{42900}
\saveTG{𪴝}{42900}
\saveTG{朻}{42900}
\saveTG{椡}{42900}
\saveTG{栦}{42900}
\saveTG{刹}{42900}
\saveTG{檦}{42900}
\saveTG{㓼}{42900}
\saveTG{𣔜}{42900}
\saveTG{𣖊}{42900}
\saveTG{𣒜}{42900}
\saveTG{𣜙}{42900}
\saveTG{𣞗}{42900}
\saveTG{𣖡}{42900}
\saveTG{𣐲}{42900}
\saveTG{㭭}{42900}
\saveTG{𣏀}{42900}
\saveTG{楋}{42900}
\saveTG{㭢}{42900}
\saveTG{𣭳}{42901}
\saveTG{𥛱}{42901}
\saveTG{紮}{42903}
\saveTG{𣚄}{42904}
\saveTG{𣓇}{42904}
\saveTG{𣒡}{42904}
\saveTG{椠}{42904}
\saveTG{椞}{42904}
\saveTG{札}{42910}
\saveTG{朼}{42910}
\saveTG{枇}{42910}
\saveTG{橷}{42911}
\saveTG{𣒄}{42911}
\saveTG{𪲀}{42912}
\saveTG{𣔥}{42912}
\saveTG{栀}{42912}
\saveTG{𣛽}{42912}
\saveTG{𣖏}{42912}
\saveTG{𣕅}{42912}
\saveTG{𣛯}{42912}
\saveTG{𣞚}{42912}
\saveTG{桃}{42913}
\saveTG{榌}{42913}
\saveTG{檵}{42913}
\saveTG{梐}{42914}
\saveTG{杔}{42914}
\saveTG{栣}{42914}
\saveTG{枆}{42914}
\saveTG{𣕭}{42914}
\saveTG{𣓆}{42914}
\saveTG{𣐅}{42914}
\saveTG{橇}{42914}
\saveTG{𣮦}{42915}
\saveTG{槯}{42915}
\saveTG{㮔}{42915}
\saveTG{𣒈}{42915}
\saveTG{𣜁}{42915}
\saveTG{棰}{42915}
\saveTG{㯿}{42916}
\saveTG{梔}{42917}
\saveTG{𣙍}{42917}
\saveTG{𪴁}{42917}
\saveTG{𣐑}{42917}
\saveTG{㯒}{42917}
\saveTG{𣒖}{42917}
\saveTG{𣑶}{42917}
\saveTG{𣏺}{42917}
\saveTG{𣘄}{42917}
\saveTG{𣮾}{42917}
\saveTG{𣓽}{42917}
\saveTG{𣒇}{42917}
\saveTG{𣔪}{42917}
\saveTG{㲡}{42917}
\saveTG{𣮉}{42917}
\saveTG{𪲤}{42917}
\saveTG{𣑿}{42917}
\saveTG{𣛄}{42917}
\saveTG{櫈}{42917}
\saveTG{桤}{42917}
\saveTG{榹}{42917}
\saveTG{橙}{42918}
\saveTG{榿}{42918}
\saveTG{𣙌}{42919}
\saveTG{𪳢}{42919}
\saveTG{𠞁}{42920}
\saveTG{析}{42921}
\saveTG{㯕}{42921}
\saveTG{𪲽}{42921}
\saveTG{彬}{42922}
\saveTG{杉}{42922}
\saveTG{㭮}{42923}
\saveTG{𣐰}{42927}
\saveTG{㭊}{42927}
\saveTG{㰎}{42927}
\saveTG{𣑂}{42927}
\saveTG{𫊍}{42927}
\saveTG{楀}{42927}
\saveTG{椯}{42927}
\saveTG{欈}{42927}
\saveTG{栃}{42927}
\saveTG{桚}{42927}
\saveTG{梤}{42927}
\saveTG{橋}{42927}
\saveTG{𣡩}{42927}
\saveTG{𣑹}{42927}
\saveTG{𣙋}{42927}
\saveTG{𣚠}{42927}
\saveTG{𪳵}{42927}
\saveTG{𣒴}{42927}
\saveTG{梣}{42927}
\saveTG{𣕄}{42927}
\saveTG{檙}{42927}
\saveTG{桥}{42928}
\saveTG{槬}{42930}
\saveTG{柧}{42930}
\saveTG{𣐚}{42930}
\saveTG{枛}{42930}
\saveTG{㭃}{42931}
\saveTG{𣏈}{42931}
\saveTG{𣐜}{42931}
\saveTG{櫄}{42931}
\saveTG{𣗮}{42931}
\saveTG{𣝤}{42931}
\saveTG{𣠍}{42931}
\saveTG{𣝙}{42931}
\saveTG{𪱹}{42932}
\saveTG{㭬}{42932}
\saveTG{㭛}{42932}
\saveTG{𣖩}{42932}
\saveTG{𣟴}{42932}
\saveTG{𣔅}{42932}
\saveTG{柉}{42932}
\saveTG{𣞧}{42932}
\saveTG{𣛹}{42932}
\saveTG{𣔞}{42933}
\saveTG{枨}{42934}
\saveTG{𣜲}{42936}
\saveTG{檼}{42937}
\saveTG{𣓦}{42937}
\saveTG{柢}{42940}
\saveTG{杄}{42940}
\saveTG{𣠴}{42940}
\saveTG{𣏸}{42940}
\saveTG{梃}{42941}
\saveTG{梴}{42941}
\saveTG{𣖅}{42941}
\saveTG{柝}{42941}
\saveTG{𣞡}{42942}
\saveTG{𣞛}{42942}
\saveTG{㭩}{42943}
\saveTG{㮃}{42944}
\saveTG{桜}{42944}
\saveTG{桵}{42944}
\saveTG{𣐧}{42945}
\saveTG{𣜬}{42947}
\saveTG{𪲏}{42947}
\saveTG{𣖱}{42947}
\saveTG{𪳗}{42947}
\saveTG{𣞞}{42947}
\saveTG{𣐪}{42947}
\saveTG{𣏚}{42947}
\saveTG{𪴉}{42947}
\saveTG{橃}{42947}
\saveTG{桴}{42947}
\saveTG{㯶}{42947}
\saveTG{楥}{42947}
\saveTG{椶}{42947}
\saveTG{板}{42947}
\saveTG{欉}{42947}
\saveTG{𣖿}{42948}
\saveTG{㭔}{42949}
\saveTG{杽}{42950}
\saveTG{橓}{42952}
\saveTG{𣒬}{42953}
\saveTG{機}{42953}
\saveTG{𣗏}{42954}
\saveTG{栺}{42961}
\saveTG{𪲉}{42961}
\saveTG{㭼}{42962}
\saveTG{𣒩}{42962}
\saveTG{楷}{42962}
\saveTG{椔}{42963}
\saveTG{栝}{42964}
\saveTG{桰}{42964}
\saveTG{棔}{42964}
\saveTG{楯}{42964}
\saveTG{𣕻}{42965}
\saveTG{橎}{42969}
\saveTG{楿}{42969}
\saveTG{杣}{42970}
\saveTG{榣}{42972}
\saveTG{柮}{42972}
\saveTG{𣠭}{42972}
\saveTG{㮑}{42977}
\saveTG{槄}{42977}
\saveTG{𣓠}{42977}
\saveTG{槇}{42981}
\saveTG{梹}{42981}
\saveTG{}{42982}
\saveTG{𣝢}{42984}
\saveTG{楑}{42984}
\saveTG{枖}{42984}
\saveTG{榽}{42984}
\saveTG{𣔹}{42984}
\saveTG{樸}{42985}
\saveTG{㯷}{42985}
\saveTG{櫍}{42986}
\saveTG{𣕴}{42989}
\saveTG{栤}{42990}
\saveTG{𣙩}{42991}
\saveTG{𣠡}{42993}
\saveTG{櫾}{42993}
\saveTG{槂}{42993}
\saveTG{橴}{42993}
\saveTG{𣛐}{42993}
\saveTG{𣞿}{42993}
\saveTG{𣟠}{42993}
\saveTG{𪳕}{42994}
\saveTG{𣙔}{42994}
\saveTG{𣗈}{42994}
\saveTG{櫯}{42994}
\saveTG{𣓅}{42994}
\saveTG{𣡵}{42994}
\saveTG{棌}{42994}
\saveTG{樔}{42994}
\saveTG{柇}{42994}
\saveTG{栎}{42994}
\saveTG{檪}{42994}
\saveTG{櫟}{42994}
\saveTG{𣖧}{42994}
\saveTG{檏}{42995}
\saveTG{棅}{42997}
\saveTG{𣕊}{42999}
\saveTG{𠡙}{43000}
\saveTG{弋}{43000}
\saveTG{𡰋}{43011}
\saveTG{㞈}{43012}
\saveTG{尤}{43012}
\saveTG{龙}{43014}
\saveTG{尨}{43014}
\saveTG{𡯰}{43015}
\saveTG{𡯽}{43015}
\saveTG{𪜗}{43015}
\saveTG{𡯜}{43018}
\saveTG{𤢈}{43018}
\saveTG{𢍽}{43040}
\saveTG{博}{43042}
\saveTG{}{43050}
\saveTG{𢦎}{43050}
\saveTG{𢦔}{43050}
\saveTG{𪭣}{43052}
\saveTG{𢦫}{43054}
\saveTG{𧈱}{43100}
\saveTG{𡉾}{43100}
\saveTG{𡈽}{43100}
\saveTG{弍}{43100}
\saveTG{卦}{43100}
\saveTG{圤}{43100}
\saveTG{式}{43100}
\saveTG{弌}{43100}
\saveTG{弎}{43100}
\saveTG{𢎆}{43100}
\saveTG{𪾓}{43102}
\saveTG{盋}{43102}
\saveTG{盚}{43102}
\saveTG{盐}{43102}
\saveTG{𥁦}{43102}
\saveTG{𪾊}{43102}
\saveTG{𪤟}{43104}
\saveTG{𡌊}{43104}
\saveTG{𤥲}{43104}
\saveTG{𪣊}{43104}
\saveTG{垄}{43104}
\saveTG{㙬}{43104}
\saveTG{𤯯}{43105}
\saveTG{䥭}{43109}
\saveTG{𡌾}{43112}
\saveTG{垸}{43112}
\saveTG{墭}{43112}
\saveTG{埦}{43112}
\saveTG{𪣯}{43112}
\saveTG{坨}{43112}
\saveTG{埪}{43112}
\saveTG{垅}{43114}
\saveTG{𡑮}{43114}
\saveTG{垞}{43114}
\saveTG{𡔃}{43115}
\saveTG{𨾬}{43115}
\saveTG{塇}{43116}
\saveTG{㙀}{43117}
\saveTG{𪣞}{43117}
\saveTG{𡒵}{43117}
\saveTG{𡊫}{43117}
\saveTG{𡋉}{43117}
\saveTG{𡏾}{43121}
\saveTG{坾}{43121}
\saveTG{墋}{43122}
\saveTG{}{43127}
\saveTG{𫛟}{43127}
\saveTG{}{43127}
\saveTG{埔}{43127}
\saveTG{鸢}{43127}
\saveTG{㙆}{43131}
\saveTG{埌}{43132}
\saveTG{𪣉}{43132}
\saveTG{㙳}{43135}
\saveTG{𧈩}{43136}
\saveTG{𧒔}{43136}
\saveTG{𧋛}{43136}
\saveTG{𡌺}{43136}
\saveTG{𧕾}{43136}
\saveTG{䘁}{43136}
\saveTG{𧓤}{43136}
\saveTG{𡑋}{43138}
\saveTG{𡌨}{43140}
\saveTG{𪣥}{43141}
\saveTG{㘾}{43141}
\saveTG{垨}{43142}
\saveTG{㙛}{43143}
\saveTG{𡏉}{43144}
\saveTG{垵}{43144}
\saveTG{坺}{43147}
\saveTG{埈}{43147}
\saveTG{㘺}{43150}
\saveTG{𡓥}{43150}
\saveTG{𢧤}{43150}
\saveTG{域}{43150}
\saveTG{韯}{43150}
\saveTG{堿}{43150}
\saveTG{臷}{43150}
\saveTG{蛓}{43150}
\saveTG{城}{43150}
\saveTG{墄}{43150}
\saveTG{𡑌}{43150}
\saveTG{𡒾}{43151}
\saveTG{𡒤}{43151}
\saveTG{㚀}{43151}
\saveTG{𡊸}{43154}
\saveTG{㘽}{43154}
\saveTG{𢧑}{43154}
\saveTG{㦳}{43154}
\saveTG{𢧜}{43154}
\saveTG{㙎}{43154}
\saveTG{坮}{43160}
\saveTG{𧯥}{43160}
\saveTG{𡑛}{43161}
\saveTG{𡓈}{43161}
\saveTG{塎}{43168}
\saveTG{𡑣}{43172}
\saveTG{𪣬}{43177}
\saveTG{𡒆}{43181}
\saveTG{埞}{43181}
\saveTG{坹}{43182}
\saveTG{𤞷}{43184}
\saveTG{垘}{43184}
\saveTG{埃}{43184}
\saveTG{堗}{43184}
\saveTG{𡉩}{43184}
\saveTG{𡌤}{43184}
\saveTG{𡔎}{43184}
\saveTG{𡎖}{43186}
\saveTG{𡒨}{43186}
\saveTG{𡐔}{43186}
\saveTG{𡔍}{43186}
\saveTG{𡊍}{43194}
\saveTG{墚}{43194}
\saveTG{𡋠}{43194}
\saveTG{𪤮}{43196}
\saveTG{𪤦}{43199}
\saveTG{𢎉}{43200}
\saveTG{𦣰}{43200}
\saveTG{𢁽}{43204}
\saveTG{𤡙}{43211}
\saveTG{𠒩}{43212}
\saveTG{朮}{43212}
\saveTG{帵}{43212}
\saveTG{狁}{43212}
\saveTG{𤟄}{43212}
\saveTG{𢃐}{43212}
\saveTG{犹}{43212}
\saveTG{𡯎}{43212}
\saveTG{𩀸}{43212}
\saveTG{𤢱}{43213}
\saveTG{狵}{43214}
\saveTG{尨}{43214}
\saveTG{𩁓}{43215}
\saveTG{𨾍}{43215}
\saveTG{𠒲}{43215}
\saveTG{𤟿}{43216}
\saveTG{𤞌}{43217}
\saveTG{𫜲}{43217}
\saveTG{𤟊}{43217}
\saveTG{𤞵}{43217}
\saveTG{𧹟}{43217}
\saveTG{𤝨}{43217}
\saveTG{𢁲}{43217}
\saveTG{𤠴}{43217}
\saveTG{𤝛}{43217}
\saveTG{㡧}{43218}
\saveTG{𠒨}{43219}
\saveTG{狞}{43221}
\saveTG{獰}{43221}
\saveTG{𢁼}{43221}
\saveTG{𤡛}{43221}
\saveTG{𩰧}{43222}
\saveTG{幓}{43222}
\saveTG{㡎}{43222}
\saveTG{㺑}{43222}
\saveTG{䳣}{43223}
\saveTG{猏}{43227}
\saveTG{𤢶}{43227}
\saveTG{𢎀}{43227}
\saveTG{𤢆}{43227}
\saveTG{𤿭}{43227}
\saveTG{𧁡}{43227}
\saveTG{𤞨}{43227}
\saveTG{𤰍}{43227}
\saveTG{𠝔}{43227}
\saveTG{𢄒}{43227}
\saveTG{猵}{43227}
\saveTG{𤡧}{43230}
\saveTG{狼}{43232}
\saveTG{𦫐}{43232}
\saveTG{幏}{43232}
\saveTG{㺠}{43233}
\saveTG{𤣛}{43233}
\saveTG{𤡍}{43234}
\saveTG{幰}{43236}
\saveTG{𤡮}{43238}
\saveTG{㢤}{43240}
\saveTG{𢂑}{43241}
\saveTG{𤞔}{43241}
\saveTG{𤝥}{43242}
\saveTG{猼}{43242}
\saveTG{狩}{43242}
\saveTG{𤝏}{43244}
\saveTG{𤝜}{43244}
\saveTG{𦛅}{43244}
\saveTG{狻}{43247}
\saveTG{𤟫}{43247}
\saveTG{𠕯}{43247}
\saveTG{𤡺}{43247}
\saveTG{帗}{43247}
\saveTG{𢄽}{43248}
\saveTG{𢧴}{43250}
\saveTG{𢦷}{43250}
\saveTG{𣎉}{43250}
\saveTG{𢒰}{43250}
\saveTG{㦲}{43250}
\saveTG{狨}{43250}
\saveTG{㦽}{43250}
\saveTG{胾}{43250}
\saveTG{截}{43250}
\saveTG{幟}{43250}
\saveTG{戫}{43250}
\saveTG{狘}{43250}
\saveTG{𢦖}{43250}
\saveTG{𤢷}{43251}
\saveTG{㡨}{43251}
\saveTG{㺣}{43251}
\saveTG{𢃎}{43251}
\saveTG{㺤}{43251}
\saveTG{𡓖}{43252}
\saveTG{𤞉}{43253}
\saveTG{㹽}{43253}
\saveTG{帴}{43253}
\saveTG{𢦼}{43253}
\saveTG{𢅪}{43253}
\saveTG{𢂵}{43254}
\saveTG{𤞫}{43254}
\saveTG{𪱤}{43255}
\saveTG{𧹹}{43256}
\saveTG{㺂}{43256}
\saveTG{𢄅}{43258}
\saveTG{𪻀}{43258}
\saveTG{𤠽}{43259}
\saveTG{𥀝}{43262}
\saveTG{㹾}{43263}
\saveTG{𣎬}{43266}
\saveTG{𧺀}{43269}
\saveTG{𢃙}{43277}
\saveTG{}{43277}
\saveTG{㺍}{43281}
\saveTG{狖}{43282}
\saveTG{㺙}{43282}
\saveTG{𤠒}{43284}
\saveTG{𤜧}{43284}
\saveTG{獄}{43284}
\saveTG{𤣊}{43284}
\saveTG{㹷}{43284}
\saveTG{㺇}{43284}
\saveTG{𤠱}{43284}
\saveTG{𤞿}{43284}
\saveTG{𤝯}{43284}
\saveTG{𤟓}{43284}
\saveTG{𤟪}{43284}
\saveTG{献}{43284}
\saveTG{犾}{43284}
\saveTG{狱}{43284}
\saveTG{𤠨}{43284}
\saveTG{𤡜}{43284}
\saveTG{𢄨}{43286}
\saveTG{㡦}{43286}
\saveTG{獱}{43286}
\saveTG{𪻅}{43286}
\saveTG{𤡗}{43289}
\saveTG{𤢅}{43289}
\saveTG{𢃏}{43291}
\saveTG{猔}{43291}
\saveTG{𤝞}{43294}
\saveTG{𤢸}{43296}
\saveTG{𤡅}{43298}
\saveTG{𤞰}{43299}
\saveTG{忒}{43300}
\saveTG{𢂥}{43300}
\saveTG{𪃘}{43327}
\saveTG{𩡷}{43327}
\saveTG{鳶}{43327}
\saveTG{鸑}{43327}
\saveTG{𪅫}{43327}
\saveTG{𢗥}{43330}
\saveTG{𢖼}{43330}
\saveTG{怷}{43331}
\saveTG{𤇍}{43331}
\saveTG{𤈿}{43333}
\saveTG{𢚡}{43333}
\saveTG{𩻜}{43336}
\saveTG{憖}{43338}
\saveTG{𢤍}{43338}
\saveTG{𤌔}{43338}
\saveTG{𤏇}{43339}
\saveTG{怸}{43339}
\saveTG{𢨗}{43350}
\saveTG{䳒}{43352}
\saveTG{䵧}{43353}
\saveTG{𡚿}{43400}
\saveTG{𢂌}{43400}
\saveTG{聋}{43401}
\saveTG{𪋿}{43402}
\saveTG{妼}{43404}
\saveTG{𡏜}{43405}
\saveTG{𪌝}{43405}
\saveTG{犮}{43407}
\saveTG{妒}{43407}
\saveTG{婉}{43412}
\saveTG{𡡛}{43412}
\saveTG{𡝮}{43412}
\saveTG{𪍂}{43412}
\saveTG{𪥬}{43412}
\saveTG{娏}{43414}
\saveTG{𡝬}{43414}
\saveTG{姹}{43414}
\saveTG{媗}{43416}
\saveTG{䴷}{43417}
\saveTG{𡛥}{43417}
\saveTG{𡡃}{43417}
\saveTG{㛡}{43417}
\saveTG{𡣚}{43417}
\saveTG{㚭}{43417}
\saveTG{𡛏}{43417}
\saveTG{𪌡}{43417}
\saveTG{䴱}{43417}
\saveTG{𤕞}{43421}
\saveTG{𪥰}{43421}
\saveTG{嬣}{43421}
\saveTG{㜗}{43422}
\saveTG{𡞋}{43422}
\saveTG{𡜵}{43427}
\saveTG{𡝏}{43427}
\saveTG{𡜷}{43427}
\saveTG{麱}{43427}
\saveTG{𡢠}{43427}
\saveTG{𡞶}{43427}
\saveTG{媥}{43427}
\saveTG{姢}{43427}
\saveTG{娘}{43432}
\saveTG{嫁}{43432}
\saveTG{𡡪}{43432}
\saveTG{𡛻}{43432}
\saveTG{𡢳}{43435}
\saveTG{𡤁}{43436}
\saveTG{㜣}{43438}
\saveTG{𡛲}{43440}
\saveTG{㚤}{43440}
\saveTG{䴬}{43440}
\saveTG{娬}{43440}
\saveTG{𡞛}{43441}
\saveTG{𪍡}{43443}
\saveTG{𤕠}{43443}
\saveTG{𡛞}{43444}
\saveTG{𪌚}{43444}
\saveTG{姲}{43444}
\saveTG{㛮}{43447}
\saveTG{㛖}{43447}
\saveTG{𡜫}{43447}
\saveTG{𡢷}{43447}
\saveTG{妭}{43447}
\saveTG{𪪵}{43449}
\saveTG{𪭋}{43450}
\saveTG{嬂}{43450}
\saveTG{媙}{43450}
\saveTG{娀}{43450}
\saveTG{孅}{43450}
\saveTG{娍}{43450}
\saveTG{麰}{43450}
\saveTG{戟}{43450}
\saveTG{娥}{43450}
\saveTG{𡣯}{43450}
\saveTG{䴰}{43450}
\saveTG{𪭒}{43450}
\saveTG{𡤪}{43451}
\saveTG{𡣳}{43451}
\saveTG{𡚞}{43451}
\saveTG{㛌}{43451}
\saveTG{𪭊}{43452}
\saveTG{𡜐}{43453}
\saveTG{𢦛}{43454}
\saveTG{婶}{43456}
\saveTG{𡞨}{43456}
\saveTG{㛾}{43456}
\saveTG{𡠐}{43456}
\saveTG{𡛟}{43457}
\saveTG{𢧇}{43458}
\saveTG{𡟬}{43458}
\saveTG{𡠽}{43459}
\saveTG{始}{43460}
\saveTG{𡜗}{43461}
\saveTG{㜚}{43462}
\saveTG{𡟲}{43465}
\saveTG{𡡩}{43466}
\saveTG{𡡮}{43467}
\saveTG{嫆}{43468}
\saveTG{嬸}{43469}
\saveTG{婠}{43477}
\saveTG{嫔}{43481}
\saveTG{婝}{43481}
\saveTG{𡡨}{43481}
\saveTG{𡢌}{43481}
\saveTG{䴳}{43482}
\saveTG{𥨜}{43484}
\saveTG{娭}{43484}
\saveTG{𪦛}{43484}
\saveTG{嬪}{43486}
\saveTG{𡤧}{43486}
\saveTG{𡣕}{43486}
\saveTG{𡤐}{43486}
\saveTG{婃}{43491}
\saveTG{𡜻}{43494}
\saveTG{㛽}{43494}
\saveTG{𡣑}{43496}
\saveTG{𡣲}{43496}
\saveTG{㛏}{43499}
\saveTG{𪌵}{43499}
\saveTG{𩊃}{43500}
\saveTG{䩛}{43504}
\saveTG{鞚}{43512}
\saveTG{䩩}{43512}
\saveTG{𩉺}{43512}
\saveTG{𩏟}{43514}
\saveTG{𩋡}{43514}
\saveTG{𫖅}{43514}
\saveTG{𩋢}{43516}
\saveTG{𩏆}{43516}
\saveTG{𨐇}{43517}
\saveTG{𩎺}{43517}
\saveTG{𩎼}{43517}
\saveTG{𩌰}{43522}
\saveTG{䪔}{43527}
\saveTG{𩊬}{43527}
\saveTG{辅}{43527}
\saveTG{𩊍}{43532}
\saveTG{𩍎}{43533}
\saveTG{䪐}{43534}
\saveTG{𨐊}{43536}
\saveTG{𩍹}{43536}
\saveTG{䪀}{43536}
\saveTG{𩌾}{43536}
\saveTG{𩍑}{43538}
\saveTG{𢎇}{43540}
\saveTG{轼}{43540}
\saveTG{𩌏}{43542}
\saveTG{𩊦}{43542}
\saveTG{䪙}{43543}
\saveTG{鞍}{43544}
\saveTG{𩊻}{43547}
\saveTG{𩊊}{43547}
\saveTG{韍}{43547}
\saveTG{䩳}{43547}
\saveTG{載}{43550}
\saveTG{载}{43550}
\saveTG{𩋉}{43550}
\saveTG{𢨏}{43550}
\saveTG{𩎹}{43551}
\saveTG{𩋋}{43553}
\saveTG{𩎙}{43557}
\saveTG{𩋽}{43564}
\saveTG{辖}{43565}
\saveTG{𩍏}{43566}
\saveTG{䩪}{43577}
\saveTG{𩎀}{43582}
\saveTG{𩎅}{43582}
\saveTG{𩊙}{43584}
\saveTG{𨟲}{43600}
\saveTG{𢎃}{43600}
\saveTG{𢎈}{43600}
\saveTG{詟}{43601}
\saveTG{砻}{43602}
\saveTG{㗠}{43617}
\saveTG{𢎅}{43627}
\saveTG{𢎋}{43640}
\saveTG{𢧉}{43650}
\saveTG{酨}{43650}
\saveTG{哉}{43650}
\saveTG{𢧭}{43650}
\saveTG{𧧬}{43650}
\saveTG{𥅤}{43650}
\saveTG{𤱱}{43650}
\saveTG{𢦮}{43650}
\saveTG{𥅰}{43650}
\saveTG{𧟭}{43651}
\saveTG{𢨣}{43653}
\saveTG{𢧨}{43656}
\saveTG{甙}{43700}
\saveTG{𢍻}{43700}
\saveTG{𢍾}{43700}
\saveTG{𢍺}{43710}
\saveTG{𤬩}{43717}
\saveTG{𪓫}{43717}
\saveTG{𡯮}{43717}
\saveTG{𠬕}{43727}
\saveTG{𠬃}{43727}
\saveTG{袭}{43732}
\saveTG{裘}{43732}
\saveTG{𩛰}{43732}
\saveTG{䬥}{43732}
\saveTG{𢎌}{43740}
\saveTG{戡}{43750}
\saveTG{裁}{43750}
\saveTG{𨚵}{43751}
\saveTG{𢨎}{43751}
\saveTG{𢨆}{43751}
\saveTG{𡽺}{43772}
\saveTG{𫜰}{43772}
\saveTG{𪙓}{43772}
\saveTG{𪙤}{43772}
\saveTG{𤮼}{43774}
\saveTG{𫇒}{43777}
\saveTG{𤯂}{43784}
\saveTG{𠧭}{43800}
\saveTG{贰}{43800}
\saveTG{赴}{43800}
\saveTG{犬}{43800}
\saveTG{𢎑}{43800}
\saveTG{𧴮}{43800}
\saveTG{𢎂}{43800}
\saveTG{𧺨}{43801}
\saveTG{𡯯}{43801}
\saveTG{龚}{43801}
\saveTG{𧻏}{43802}
\saveTG{赴}{43802}
\saveTG{䟃}{43802}
\saveTG{𧻷}{43802}
\saveTG{𨆕}{43802}
\saveTG{𨁛}{43802}
\saveTG{𫎺}{43802}
\saveTG{𧻴}{43803}
\saveTG{𧼏}{43803}
\saveTG{𧾋}{43803}
\saveTG{𧾨}{43803}
\saveTG{䞭}{43804}
\saveTG{𫎼}{43804}
\saveTG{𧺺}{43804}
\saveTG{𧻧}{43804}
\saveTG{䶮}{43804}
\saveTG{䞖}{43804}
\saveTG{}{43804}
\saveTG{𧽔}{43804}
\saveTG{𧼭}{43804}
\saveTG{𧾝}{43804}
\saveTG{𧾠}{43804}
\saveTG{𧺱}{43805}
\saveTG{䟈}{43805}
\saveTG{䟌}{43805}
\saveTG{𧾔}{43805}
\saveTG{䞲}{43805}
\saveTG{𧽫}{43805}
\saveTG{𧼑}{43805}
\saveTG{𧾢}{43805}
\saveTG{𧽮}{43805}
\saveTG{𧾞}{43805}
\saveTG{越}{43805}
\saveTG{𧾂}{43805}
\saveTG{𧻗}{43805}
\saveTG{貣}{43806}
\saveTG{趤}{43806}
\saveTG{𡤓}{43806}
\saveTG{𧼁}{43806}
\saveTG{𧻱}{43809}
\saveTG{𧺶}{43809}
\saveTG{貳}{43810}
\saveTG{𣯳}{43815}
\saveTG{𨇌}{43826}
\saveTG{𢩟}{43827}
\saveTG{𪏗}{43827}
\saveTG{𪏤}{43835}
\saveTG{䵍}{43847}
\saveTG{𢨃}{43850}
\saveTG{𪏇}{43850}
\saveTG{𪏞}{43850}
\saveTG{𪭍}{43850}
\saveTG{𪥅}{43850}
\saveTG{戴}{43850}
\saveTG{烖}{43850}
\saveTG{𪏊}{43853}
\saveTG{𧻪}{43854}
\saveTG{𪏉}{43882}
\saveTG{猋}{43884}
\saveTG{㹜}{43884}
\saveTG{朴}{43900}
\saveTG{术}{43900}
\saveTG{杺}{43900}
\saveTG{𣘂}{43904}
\saveTG{柲}{43904}
\saveTG{枦}{43907}
\saveTG{求}{43909}
\saveTG{榨}{43911}
\saveTG{榓}{43912}
\saveTG{𣛮}{43912}
\saveTG{𣛷}{43912}
\saveTG{椌}{43912}
\saveTG{𪲆}{43912}
\saveTG{椀}{43912}
\saveTG{梡}{43912}
\saveTG{柁}{43912}
\saveTG{椬}{43912}
\saveTG{榁}{43914}
\saveTG{栊}{43914}
\saveTG{𣙜}{43915}
\saveTG{楦}{43916}
\saveTG{𪲩}{43916}
\saveTG{𪝗}{43917}
\saveTG{㭦}{43917}
\saveTG{𣐖}{43917}
\saveTG{𣔛}{43917}
\saveTG{𣜐}{43917}
\saveTG{𣒔}{43917}
\saveTG{𣑒}{43917}
\saveTG{槴}{43917}
\saveTG{𣏞}{43917}
\saveTG{㭇}{43917}
\saveTG{𪳃}{43917}
\saveTG{槣}{43921}
\saveTG{檸}{43921}
\saveTG{𣔔}{43921}
\saveTG{柠}{43921}
\saveTG{𣙓}{43922}
\saveTG{椮}{43922}
\saveTG{槮}{43922}
\saveTG{𣠣}{43926}
\saveTG{椖}{43927}
\saveTG{橣}{43927}
\saveTG{㮼}{43927}
\saveTG{𣗉}{43927}
\saveTG{𣓖}{43927}
\saveTG{㭪}{43927}
\saveTG{𣡐}{43927}
\saveTG{𣞐}{43927}
\saveTG{𣘕}{43927}
\saveTG{𣛡}{43927}
\saveTG{𣚓}{43927}
\saveTG{楄}{43927}
\saveTG{𣛚}{43930}
\saveTG{𣟋}{43932}
\saveTG{榢}{43932}
\saveTG{桹}{43932}
\saveTG{𪳶}{43933}
\saveTG{𣛴}{43935}
\saveTG{櫁}{43936}
\saveTG{櫶}{43936}
\saveTG{橪}{43938}
\saveTG{𣚈}{43938}
\saveTG{杙}{43940}
\saveTG{弒}{43940}
\saveTG{栻}{43940}
\saveTG{㭖}{43940}
\saveTG{樲}{43940}
\saveTG{弑}{43940}
\saveTG{榟}{43941}
\saveTG{𣓸}{43941}
\saveTG{𪲃}{43941}
\saveTG{榑}{43942}
\saveTG{㭓}{43944}
\saveTG{桉}{43944}
\saveTG{𣜏}{43947}
\saveTG{𣔱}{43947}
\saveTG{柭}{43947}
\saveTG{梭}{43947}
\saveTG{桳}{43948}
\saveTG{械}{43950}
\saveTG{楲}{43950}
\saveTG{槭}{43950}
\saveTG{桙}{43950}
\saveTG{櫼}{43950}
\saveTG{橶}{43950}
\saveTG{椷}{43950}
\saveTG{檝}{43950}
\saveTG{𣛔}{43950}
\saveTG{𣟵}{43950}
\saveTG{𣏾}{43950}
\saveTG{𢧣}{43950}
\saveTG{𢧺}{43950}
\saveTG{𦀂}{43950}
\saveTG{𢦏}{43950}
\saveTG{𢨝}{43950}
\saveTG{栰}{43950}
\saveTG{栈}{43950}
\saveTG{棫}{43950}
\saveTG{桟}{43950}
\saveTG{樴}{43950}
\saveTG{栽}{43950}
\saveTG{𣝫}{43951}
\saveTG{㰇}{43951}
\saveTG{𣔮}{43952}
\saveTG{𣒲}{43952}
\saveTG{棧}{43953}
\saveTG{𣚙}{43953}
\saveTG{㰄}{43953}
\saveTG{𣕀}{43953}
\saveTG{𣚸}{43953}
\saveTG{𦀂}{43953}
\saveTG{𣬀}{43954}
\saveTG{㭜}{43954}
\saveTG{𣛸}{43954}
\saveTG{𪳾}{43955}
\saveTG{𪲘}{43955}
\saveTG{𣖋}{43956}
\saveTG{𣐋}{43957}
\saveTG{枱}{43960}
\saveTG{㮫}{43960}
\saveTG{𣖤}{43961}
\saveTG{㮷}{43961}
\saveTG{樎}{43962}
\saveTG{楁}{43964}
\saveTG{榕}{43968}
\saveTG{𣙴}{43968}
\saveTG{㰂}{43968}
\saveTG{㮥}{43968}
\saveTG{𣔊}{43968}
\saveTG{樒}{43972}
\saveTG{𣒎}{43972}
\saveTG{棺}{43977}
\saveTG{𣛱}{43977}
\saveTG{𣙣}{43980}
\saveTG{椗}{43981}
\saveTG{槟}{43981}
\saveTG{柼}{43982}
\saveTG{𪴢}{43982}
\saveTG{𣟯}{43982}
\saveTG{㰗}{43982}
\saveTG{𣚕}{43984}
\saveTG{𣘔}{43984}
\saveTG{𥏳}{43984}
\saveTG{栿}{43984}
\saveTG{𣡌}{43984}
\saveTG{𣔻}{43984}
\saveTG{猌}{43984}
\saveTG{棙}{43984}
\saveTG{枤}{43984}
\saveTG{樾}{43985}
\saveTG{𣜀}{43985}
\saveTG{𣘹}{43986}
\saveTG{𪴎}{43986}
\saveTG{檳}{43986}
\saveTG{𪴥}{43986}
\saveTG{𣠦}{43986}
\saveTG{㯽}{43986}
\saveTG{棕}{43991}
\saveTG{檫}{43991}
\saveTG{栐}{43992}
\saveTG{㰑}{43994}
\saveTG{樑}{43994}
\saveTG{𣖉}{43994}
\saveTG{𣕘}{43994}
\saveTG{𪴔}{43996}
\saveTG{梂}{43999}
\saveTG{卅}{44000}
\saveTG{卌}{44000}
\saveTG{廾}{44000}
\saveTG{卄}{44000}
\saveTG{耂}{44000}
\saveTG{𠔀}{44000}
\saveTG{𠥼}{44000}
\saveTG{𠦌}{44000}
\saveTG{艹}{44000}
\saveTG{龷}{44001}
\saveTG{𢌬}{44010}
\saveTG{㞄}{44011}
\saveTG{㝾}{44011}
\saveTG{㞆}{44012}
\saveTG{𡯄}{44012}
\saveTG{𠠵}{44012}
\saveTG{𠡃}{44012}
\saveTG{㝿}{44014}
\saveTG{𡚪}{44014}
\saveTG{𡯏}{44014}
\saveTG{𦬈}{44017}
\saveTG{芞}{44017}
\saveTG{𦫴}{44017}
\saveTG{㐤}{44018}
\saveTG{㞀}{44018}
\saveTG{𡯐}{44019}
\saveTG{芌}{44027}
\saveTG{荂}{44027}
\saveTG{芎}{44027}
\saveTG{考}{44027}
\saveTG{萼}{44027}
\saveTG{協}{44027}
\saveTG{𠠴}{44027}
\saveTG{㘦}{44027}
\saveTG{㔹}{44027}
\saveTG{𦰉}{44027}
\saveTG{䒓}{44027}
\saveTG{𪜀}{44027}
\saveTG{蕚}{44027}
\saveTG{𠠹}{44027}
\saveTG{协}{44030}
\saveTG{}{44100}
\saveTG{𢍤}{44100}
\saveTG{封}{44100}
\saveTG{坿}{44100}
\saveTG{埘}{44100}
\saveTG{𡈿}{44100}
\saveTG{墛}{44100}
\saveTG{塮}{44100}
\saveTG{尌}{44100}
\saveTG{𠥻}{44100}
\saveTG{芷}{44101}
\saveTG{𦫽}{44101}
\saveTG{虀}{44101}
\saveTG{𦲕}{44101}
\saveTG{𧂊}{44101}
\saveTG{𦼩}{44101}
\saveTG{𦲞}{44101}
\saveTG{𩐍}{44101}
\saveTG{韮}{44101}
\saveTG{𫇩}{44102}
\saveTG{𡸽}{44102}
\saveTG{𪔮}{44102}
\saveTG{𦶩}{44102}
\saveTG{𦷐}{44102}
\saveTG{𧁔}{44102}
\saveTG{𦶻}{44102}
\saveTG{𥂊}{44102}
\saveTG{𦱣}{44102}
\saveTG{𦿘}{44102}
\saveTG{𥂅}{44102}
\saveTG{𧄸}{44102}
\saveTG{𦾞}{44102}
\saveTG{𦹧}{44102}
\saveTG{𦾗}{44102}
\saveTG{𧀽}{44102}
\saveTG{𧃗}{44102}
\saveTG{𦳏}{44102}
\saveTG{𫊜}{44102}
\saveTG{𧗘}{44102}
\saveTG{𦺝}{44102}
\saveTG{𦮊}{44102}
\saveTG{𦭍}{44102}
\saveTG{𫊀}{44102}
\saveTG{𦽘}{44102}
\saveTG{𧗎}{44102}
\saveTG{𦼬}{44102}
\saveTG{𧆗}{44102}
\saveTG{𧅽}{44102}
\saveTG{𦬚}{44102}
\saveTG{𫊗}{44102}
\saveTG{𦱃}{44102}
\saveTG{𦻌}{44102}
\saveTG{𦺟}{44102}
\saveTG{𥂯}{44102}
\saveTG{𥂪}{44102}
\saveTG{𦰆}{44102}
\saveTG{𦼦}{44102}
\saveTG{䓝}{44102}
\saveTG{𧗆}{44102}
\saveTG{䕄}{44102}
\saveTG{𦾐}{44102}
\saveTG{𦾟}{44102}
\saveTG{𣦀}{44102}
\saveTG{䒱}{44102}
\saveTG{𦒳}{44102}
\saveTG{𪔣}{44102}
\saveTG{𧅴}{44102}
\saveTG{𦹒}{44102}
\saveTG{𦮨}{44102}
\saveTG{𦱇}{44102}
\saveTG{𦭢}{44102}
\saveTG{𦭒}{44102}
\saveTG{䖅}{44102}
\saveTG{𤯅}{44102}
\saveTG{䒙}{44102}
\saveTG{蒫}{44102}
\saveTG{𥂿}{44102}
\saveTG{茊}{44102}
\saveTG{苎}{44102}
\saveTG{蒕}{44102}
\saveTG{蒀}{44102}
\saveTG{萾}{44102}
\saveTG{萓}{44102}
\saveTG{藍}{44102}
\saveTG{蓝}{44102}
\saveTG{莖}{44102}
\saveTG{茎}{44102}
\saveTG{藎}{44102}
\saveTG{蓋}{44102}
\saveTG{葢}{44102}
\saveTG{葐}{44102}
\saveTG{蘯}{44102}
\saveTG{苴}{44102}
\saveTG{𫇳}{44102}
\saveTG{𦮳}{44102}
\saveTG{𥁹}{44102}
\saveTG{𪤁}{44103}
\saveTG{㘰}{44103}
\saveTG{𡬾}{44103}
\saveTG{𡎈}{44103}
\saveTG{莹}{44103}
\saveTG{𡐗}{44103}
\saveTG{𦬰}{44104}
\saveTG{䓰}{44104}
\saveTG{𦯀}{44104}
\saveTG{𡐨}{44104}
\saveTG{𡒎}{44104}
\saveTG{𡋍}{44104}
\saveTG{𦹆}{44104}
\saveTG{𦹨}{44104}
\saveTG{𡐳}{44104}
\saveTG{𡏳}{44104}
\saveTG{𡎸}{44104}
\saveTG{𤦢}{44104}
\saveTG{𨬛}{44104}
\saveTG{𦸧}{44104}
\saveTG{𫉲}{44104}
\saveTG{𦸀}{44104}
\saveTG{𦵐}{44104}
\saveTG{𦱩}{44104}
\saveTG{𦱪}{44104}
\saveTG{䓧}{44104}
\saveTG{𫈠}{44104}
\saveTG{𦸃}{44104}
\saveTG{𡉧}{44104}
\saveTG{莝}{44104}
\saveTG{𧁅}{44104}
\saveTG{𦯖}{44104}
\saveTG{𫉷}{44104}
\saveTG{𪔋}{44104}
\saveTG{𡒢}{44104}
\saveTG{𧂸}{44104}
\saveTG{𫊎}{44104}
\saveTG{𦲻}{44104}
\saveTG{𦱬}{44104}
\saveTG{䑓}{44104}
\saveTG{𦿳}{44104}
\saveTG{𦥄}{44104}
\saveTG{𡎊}{44104}
\saveTG{𧂖}{44104}
\saveTG{𤨙}{44104}
\saveTG{𦬬}{44104}
\saveTG{𦹑}{44104}
\saveTG{𦭦}{44104}
\saveTG{𧁚}{44104}
\saveTG{荎}{44104}
\saveTG{荃}{44104}
\saveTG{耋}{44104}
\saveTG{耊}{44104}
\saveTG{芏}{44104}
\saveTG{𦷅}{44104}
\saveTG{堼}{44104}
\saveTG{葟}{44104}
\saveTG{基}{44104}
\saveTG{蘲}{44104}
\saveTG{墓}{44104}
\saveTG{薹}{44104}
\saveTG{鼞}{44104}
\saveTG{茥}{44104}
\saveTG{壄}{44104}
\saveTG{茔}{44104}
\saveTG{塟}{44104}
\saveTG{𦻒}{44104}
\saveTG{𧅀}{44104}
\saveTG{𦹦}{44104}
\saveTG{𨬪}{44104}
\saveTG{埜}{44104}
\saveTG{𦻍}{44105}
\saveTG{𨤬}{44105}
\saveTG{𦱦}{44105}
\saveTG{𦷗}{44105}
\saveTG{𦰌}{44105}
\saveTG{𦸨}{44105}
\saveTG{𦱥}{44105}
\saveTG{菫}{44105}
\saveTG{苼}{44105}
\saveTG{荲}{44105}
\saveTG{蓳}{44105}
\saveTG{堇}{44105}
\saveTG{蕫}{44105}
\saveTG{董}{44105}
\saveTG{菙}{44105}
\saveTG{𦬹}{44106}
\saveTG{𦹓}{44106}
\saveTG{䕊}{44106}
\saveTG{𦳘}{44106}
\saveTG{蘁}{44106}
\saveTG{荁}{44106}
\saveTG{薑}{44106}
\saveTG{萱}{44106}
\saveTG{葏}{44107}
\saveTG{𡉜}{44107}
\saveTG{𦭩}{44107}
\saveTG{𠀤}{44108}
\saveTG{𦷳}{44108}
\saveTG{䔇}{44108}
\saveTG{鼟}{44108}
\saveTG{𧀆}{44108}
\saveTG{𧯷}{44108}
\saveTG{莁}{44108}
\saveTG{苙}{44108}
\saveTG{蘴}{44108}
\saveTG{荳}{44108}
\saveTG{䔲}{44108}
\saveTG{𨨗}{44109}
\saveTG{𧅒}{44109}
\saveTG{𫓖}{44109}
\saveTG{𨬮}{44109}
\saveTG{𨪧}{44109}
\saveTG{菳}{44109}
\saveTG{䥢}{44109}
\saveTG{莶}{44109}
\saveTG{苤}{44109}
\saveTG{鐢}{44109}
\saveTG{蓥}{44109}
\saveTG{𪢴}{44110}
\saveTG{茫}{44110}
\saveTG{𦸪}{44111}
\saveTG{𡊌}{44111}
\saveTG{菲}{44111}
\saveTG{䓙}{44111}
\saveTG{蕋}{44111}
\saveTG{𦻅}{44111}
\saveTG{蕴}{44112}
\saveTG{蔬}{44112}
\saveTG{墝}{44112}
\saveTG{蓅}{44112}
\saveTG{菹}{44112}
\saveTG{莐}{44112}
\saveTG{薀}{44112}
\saveTG{茳}{44112}
\saveTG{塃}{44112}
\saveTG{荭}{44112}
\saveTG{蘫}{44112}
\saveTG{范}{44112}
\saveTG{地}{44112}
\saveTG{茈}{44112}
\saveTG{苝}{44112}
\saveTG{壒}{44112}
\saveTG{𧄻}{44112}
\saveTG{𢂉}{44112}
\saveTG{𥃕}{44112}
\saveTG{𧄭}{44112}
\saveTG{𦴸}{44112}
\saveTG{𡎌}{44112}
\saveTG{𡑷}{44112}
\saveTG{𦴍}{44112}
\saveTG{𦰫}{44112}
\saveTG{蕰}{44112}
\saveTG{𫇻}{44112}
\saveTG{𦿖}{44112}
\saveTG{𦹲}{44112}
\saveTG{𦶜}{44112}
\saveTG{𦱈}{44112}
\saveTG{𫉭}{44112}
\saveTG{䔼}{44112}
\saveTG{𧀏}{44112}
\saveTG{𦶉}{44112}
\saveTG{𦮃}{44112}
\saveTG{𦾺}{44112}
\saveTG{𦴢}{44112}
\saveTG{埴}{44112}
\saveTG{萢}{44112}
\saveTG{藱}{44113}
\saveTG{𧀐}{44113}
\saveTG{𪔛}{44113}
\saveTG{𫈲}{44114}
\saveTG{𦶄}{44114}
\saveTG{𧒨}{44114}
\saveTG{𡍙}{44114}
\saveTG{𪤏}{44114}
\saveTG{𡒣}{44114}
\saveTG{𦼂}{44114}
\saveTG{荘}{44114}
\saveTG{𦹹}{44114}
\saveTG{𦹻}{44114}
\saveTG{𦶾}{44114}
\saveTG{埖}{44114}
\saveTG{𦴂}{44114}
\saveTG{𡓨}{44114}
\saveTG{龳}{44114}
\saveTG{𡋣}{44114}
\saveTG{𦴷}{44114}
\saveTG{𦿻}{44114}
\saveTG{𦹏}{44115}
\saveTG{墐}{44115}
\saveTG{壦}{44115}
\saveTG{虄}{44115}
\saveTG{𧄓}{44115}
\saveTG{𦷯}{44115}
\saveTG{𡑗}{44115}
\saveTG{𧅁}{44116}
\saveTG{㙪}{44116}
\saveTG{埯}{44116}
\saveTG{𫇼}{44117}
\saveTG{莼}{44117}
\saveTG{塂}{44117}
\saveTG{𦻻}{44117}
\saveTG{𫈵}{44117}
\saveTG{𡋎}{44117}
\saveTG{𪣅}{44117}
\saveTG{𪣕}{44117}
\saveTG{𪢼}{44117}
\saveTG{𡑵}{44117}
\saveTG{䔵}{44117}
\saveTG{𦯋}{44117}
\saveTG{𦲉}{44117}
\saveTG{茿}{44117}
\saveTG{薽}{44117}
\saveTG{蓺}{44117}
\saveTG{菃}{44117}
\saveTG{蒞}{44118}
\saveTG{堪}{44118}
\saveTG{𦲷}{44118}
\saveTG{𫉃}{44120}
\saveTG{藰}{44120}
\saveTG{蓟}{44120}
\saveTG{菿}{44120}
\saveTG{𫉋}{44120}
\saveTG{𧁲}{44121}
\saveTG{𦭑}{44121}
\saveTG{𦶒}{44121}
\saveTG{𦾶}{44121}
\saveTG{䓅}{44121}
\saveTG{𦺍}{44121}
\saveTG{菏}{44121}
\saveTG{䓷}{44121}
\saveTG{埼}{44121}
\saveTG{蕍}{44121}
\saveTG{𦶼}{44122}
\saveTG{𦾷}{44122}
\saveTG{𪔱}{44124}
\saveTG{蔳}{44127}
\saveTG{勤}{44127}
\saveTG{蕅}{44127}
\saveTG{茑}{44127}
\saveTG{蓦}{44127}
\saveTG{药}{44127}
\saveTG{勎}{44127}
\saveTG{垮}{44127}
\saveTG{墈}{44127}
\saveTG{蒟}{44127}
\saveTG{葝}{44127}
\saveTG{𦿭}{44127}
\saveTG{蒻}{44127}
\saveTG{}{44127}
\saveTG{荺}{44127}
\saveTG{翥}{44127}
\saveTG{莇}{44127}
\saveTG{𦭎}{44127}
\saveTG{𦮢}{44127}
\saveTG{𫉙}{44127}
\saveTG{䔽}{44127}
\saveTG{𦴥}{44127}
\saveTG{𧂙}{44127}
\saveTG{𧃐}{44127}
\saveTG{𠢀}{44127}
\saveTG{𫉟}{44127}
\saveTG{𦭳}{44127}
\saveTG{䒒}{44127}
\saveTG{𧆖}{44127}
\saveTG{𦮱}{44127}
\saveTG{𦽋}{44127}
\saveTG{䔙}{44127}
\saveTG{䒸}{44127}
\saveTG{𦶌}{44127}
\saveTG{𦵑}{44127}
\saveTG{𦲂}{44127}
\saveTG{𧆇}{44127}
\saveTG{𫈩}{44127}
\saveTG{𫉣}{44127}
\saveTG{𡑪}{44127}
\saveTG{𡍝}{44127}
\saveTG{㙢}{44127}
\saveTG{𡐺}{44127}
\saveTG{𡏗}{44127}
\saveTG{㘵}{44127}
\saveTG{𦻵}{44127}
\saveTG{𡔔}{44127}
\saveTG{𦵈}{44127}
\saveTG{劸}{44127}
\saveTG{𡍧}{44127}
\saveTG{𦿼}{44127}
\saveTG{𧃃}{44127}
\saveTG{𦹝}{44127}
\saveTG{𦿯}{44127}
\saveTG{㘨}{44127}
\saveTG{𡓕}{44127}
\saveTG{𫈖}{44127}
\saveTG{𫈎}{44127}
\saveTG{𫈊}{44127}
\saveTG{𧅻}{44127}
\saveTG{𧂧}{44127}
\saveTG{𦶊}{44127}
\saveTG{𫊓}{44127}
\saveTG{𧂚}{44127}
\saveTG{䓺}{44127}
\saveTG{𦽴}{44127}
\saveTG{𦾸}{44127}
\saveTG{𫊁}{44127}
\saveTG{𡐦}{44127}
\saveTG{𡎜}{44127}
\saveTG{㘼}{44127}
\saveTG{𪳌}{44127}
\saveTG{㙹}{44127}
\saveTG{𡋼}{44127}
\saveTG{𡍡}{44127}
\saveTG{㙝}{44127}
\saveTG{𫉗}{44127}
\saveTG{𪔦}{44127}
\saveTG{𦿰}{44127}
\saveTG{𦶑}{44127}
\saveTG{䕽}{44127}
\saveTG{𦭭}{44127}
\saveTG{𫊛}{44127}
\saveTG{𧀰}{44127}
\saveTG{𦸙}{44127}
\saveTG{𡓀}{44127}
\saveTG{𦰛}{44127}
\saveTG{𡍆}{44127}
\saveTG{𦲖}{44127}
\saveTG{𦷟}{44127}
\saveTG{薃}{44127}
\saveTG{蘬}{44127}
\saveTG{坳}{44127}
\saveTG{墆}{44127}
\saveTG{蕩}{44127}
\saveTG{荡}{44127}
\saveTG{蒲}{44127}
\saveTG{莺}{44127}
\saveTG{蘙}{44127}
\saveTG{蓊}{44127}
\saveTG{堶}{44127}
\saveTG{薥}{44127}
\saveTG{莎}{44129}
\saveTG{𠀖}{44130}
\saveTG{𠀪}{44130}
\saveTG{𦴵}{44131}
\saveTG{𡓅}{44131}
\saveTG{𡋽}{44131}
\saveTG{𡐬}{44131}
\saveTG{𧁛}{44131}
\saveTG{𧔐}{44131}
\saveTG{𡊛}{44131}
\saveTG{𡉞}{44131}
\saveTG{𦘏}{44131}
\saveTG{𧰣}{44131}
\saveTG{𧔏}{44131}
\saveTG{𦻩}{44131}
\saveTG{𡓽}{44131}
\saveTG{𧀡}{44131}
\saveTG{𦲾}{44131}
\saveTG{蘾}{44132}
\saveTG{蒗}{44132}
\saveTG{𦯫}{44132}
\saveTG{壊}{44132}
\saveTG{𡏏}{44132}
\saveTG{𦴿}{44132}
\saveTG{𦻂}{44132}
\saveTG{蒎}{44132}
\saveTG{𡒯}{44132}
\saveTG{𦶍}{44132}
\saveTG{𫈞}{44132}
\saveTG{𡐿}{44134}
\saveTG{墶}{44135}
\saveTG{𡓄}{44135}
\saveTG{𦿸}{44136}
\saveTG{𦸞}{44136}
\saveTG{𧉺}{44136}
\saveTG{𧍣}{44136}
\saveTG{𦿞}{44136}
\saveTG{𦾖}{44136}
\saveTG{𧖡}{44136}
\saveTG{𧕷}{44136}
\saveTG{蛬}{44136}
\saveTG{𧊝}{44136}
\saveTG{𧓿}{44136}
\saveTG{𧒚}{44136}
\saveTG{𧍟}{44136}
\saveTG{𧒣}{44136}
\saveTG{萤}{44136}
\saveTG{蜝}{44136}
\saveTG{藌}{44136}
\saveTG{蟇}{44136}
\saveTG{蠚}{44136}
\saveTG{蠜}{44136}
\saveTG{𧊬}{44136}
\saveTG{茧}{44136}
\saveTG{蠆}{44136}
\saveTG{𧍷}{44136}
\saveTG{鼜}{44136}
\saveTG{𧌔}{44136}
\saveTG{𧓵}{44136}
\saveTG{䘍}{44136}
\saveTG{𪔖}{44136}
\saveTG{𧏊}{44136}
\saveTG{𧅪}{44136}
\saveTG{𧒤}{44136}
\saveTG{䗣}{44136}
\saveTG{𧑭}{44136}
\saveTG{𧍙}{44136}
\saveTG{𦹌}{44136}
\saveTG{𡒹}{44136}
\saveTG{𦳶}{44136}
\saveTG{𧔩}{44136}
\saveTG{𧅮}{44136}
\saveTG{𧕏}{44136}
\saveTG{𧊮}{44136}
\saveTG{𧏤}{44136}
\saveTG{垯}{44138}
\saveTG{塨}{44138}
\saveTG{𧁑}{44138}
\saveTG{𦯻}{44140}
\saveTG{䕑}{44140}
\saveTG{𫈑}{44140}
\saveTG{薱}{44140}
\saveTG{葑}{44140}
\saveTG{荮}{44140}
\saveTG{𫊙}{44141}
\saveTG{䓑}{44141}
\saveTG{𦱾}{44141}
\saveTG{𡉻}{44141}
\saveTG{𦻙}{44141}
\saveTG{𡋌}{44141}
\saveTG{塒}{44141}
\saveTG{蓱}{44141}
\saveTG{壔}{44141}
\saveTG{𪳯}{44141}
\saveTG{𡍲}{44141}
\saveTG{䕪}{44141}
\saveTG{䔊}{44141}
\saveTG{蒋}{44142}
\saveTG{䓋}{44142}
\saveTG{薄}{44142}
\saveTG{𡎼}{44142}
\saveTG{𦷘}{44142}
\saveTG{䓜}{44142}
\saveTG{𫈓}{44143}
\saveTG{𡎃}{44143}
\saveTG{𫉰}{44144}
\saveTG{𦭰}{44144}
\saveTG{𦴴}{44144}
\saveTG{𡍁}{44144}
\saveTG{𦷪}{44144}
\saveTG{𡍋}{44144}
\saveTG{𧃑}{44145}
\saveTG{𫉊}{44145}
\saveTG{𧔴}{44146}
\saveTG{𡍎}{44146}
\saveTG{藫}{44146}
\saveTG{𧃀}{44147}
\saveTG{𦮆}{44147}
\saveTG{㪈}{44147}
\saveTG{皼}{44147}
\saveTG{皶}{44147}
\saveTG{𦳙}{44147}
\saveTG{𡏞}{44147}
\saveTG{皷}{44147}
\saveTG{鼓}{44147}
\saveTG{蔋}{44147}
\saveTG{蓤}{44147}
\saveTG{䥑}{44147}
\saveTG{𦺌}{44147}
\saveTG{𦷰}{44147}
\saveTG{䕕}{44147}
\saveTG{菠}{44147}
\saveTG{堎}{44147}
\saveTG{蔆}{44147}
\saveTG{莈}{44147}
\saveTG{坡}{44147}
\saveTG{蓡}{44147}
\saveTG{薓}{44147}
\saveTG{𡋯}{44147}
\saveTG{𪣋}{44147}
\saveTG{薣}{44147}
\saveTG{𦽐}{44147}
\saveTG{𡌉}{44147}
\saveTG{𦻈}{44147}
\saveTG{𧁎}{44148}
\saveTG{𦼻}{44148}
\saveTG{蔹}{44148}
\saveTG{𫉦}{44148}
\saveTG{𡐵}{44149}
\saveTG{萍}{44149}
\saveTG{𡑀}{44149}
\saveTG{𫈭}{44151}
\saveTG{藓}{44151}
\saveTG{𦯊}{44152}
\saveTG{䔐}{44153}
\saveTG{𧃖}{44153}
\saveTG{𧂪}{44153}
\saveTG{𦸮}{44153}
\saveTG{𦺘}{44153}
\saveTG{𦼺}{44153}
\saveTG{𦸴}{44153}
\saveTG{𦱂}{44153}
\saveTG{虃}{44153}
\saveTG{𧂂}{44153}
\saveTG{𦾌}{44154}
\saveTG{𪔢}{44154}
\saveTG{墷}{44154}
\saveTG{𦴝}{44156}
\saveTG{𦶶}{44156}
\saveTG{𫈬}{44156}
\saveTG{𫈸}{44156}
\saveTG{𦷫}{44157}
\saveTG{𦺇}{44157}
\saveTG{㙔}{44157}
\saveTG{𦻄}{44157}
\saveTG{𦰪}{44160}
\saveTG{𦵙}{44160}
\saveTG{堵}{44160}
\saveTG{𡊜}{44160}
\saveTG{塔}{44161}
\saveTG{墙}{44161}
\saveTG{墻}{44161}
\saveTG{𡆒}{44161}
\saveTG{𫈰}{44161}
\saveTG{𦹃}{44161}
\saveTG{𡆐}{44161}
\saveTG{𦹷}{44161}
\saveTG{𡓠}{44161}
\saveTG{𪣤}{44161}
\saveTG{𡋥}{44161}
\saveTG{𦳸}{44162}
\saveTG{𦽾}{44162}
\saveTG{萡}{44162}
\saveTG{菬}{44162}
\saveTG{菭}{44163}
\saveTG{𧃌}{44163}
\saveTG{墸}{44164}
\saveTG{𦼥}{44164}
\saveTG{萿}{44164}
\saveTG{𦵜}{44164}
\saveTG{𦵩}{44164}
\saveTG{蕗}{44164}
\saveTG{𪣲}{44164}
\saveTG{落}{44164}
\saveTG{虂}{44164}
\saveTG{𦶡}{44165}
\saveTG{䕋}{44165}
\saveTG{𫈜}{44165}
\saveTG{𡒝}{44167}
\saveTG{𡒜}{44168}
\saveTG{藩}{44169}
\saveTG{坩}{44170}
\saveTG{𦻭}{44172}
\saveTG{𦶷}{44172}
\saveTG{𤓨}{44174}
\saveTG{𦹙}{44174}
\saveTG{𫇴}{44177}
\saveTG{蔧}{44177}
\saveTG{𦮟}{44180}
\saveTG{𡉑}{44180}
\saveTG{填}{44181}
\saveTG{𦲍}{44181}
\saveTG{㙋}{44181}
\saveTG{葓}{44181}
\saveTG{垬}{44181}
\saveTG{𧃏}{44182}
\saveTG{𧀌}{44182}
\saveTG{𦳂}{44182}
\saveTG{莰}{44182}
\saveTG{茨}{44182}
\saveTG{𧰊}{44182}
\saveTG{𡊀}{44183}
\saveTG{蒺}{44184}
\saveTG{𦰚}{44184}
\saveTG{𦮮}{44184}
\saveTG{𡑽}{44184}
\saveTG{𧄢}{44184}
\saveTG{塻}{44184}
\saveTG{塽}{44184}
\saveTG{𦺾}{44184}
\saveTG{𡎘}{44185}
\saveTG{𦶱}{44185}
\saveTG{𧯴}{44186}
\saveTG{𧰔}{44186}
\saveTG{墴}{44186}
\saveTG{墳}{44186}
\saveTG{𡓤}{44186}
\saveTG{䕱}{44186}
\saveTG{㙽}{44186}
\saveTG{𦺣}{44186}
\saveTG{𦮛}{44187}
\saveTG{𧰇}{44187}
\saveTG{埉}{44188}
\saveTG{𧃉}{44189}
\saveTG{𦼓}{44189}
\saveTG{𦸁}{44189}
\saveTG{𡍚}{44190}
\saveTG{𡉿}{44190}
\saveTG{𡑲}{44191}
\saveTG{𦶵}{44191}
\saveTG{薸}{44191}
\saveTG{𧆙}{44193}
\saveTG{䔫}{44194}
\saveTG{𦸂}{44194}
\saveTG{𧀢}{44194}
\saveTG{𡑢}{44194}
\saveTG{𦻜}{44194}
\saveTG{𡐽}{44194}
\saveTG{𦺁}{44194}
\saveTG{𧀝}{44194}
\saveTG{𦽼}{44194}
\saveTG{𧃡}{44194}
\saveTG{𡎡}{44194}
\saveTG{𡑓}{44194}
\saveTG{莯}{44194}
\saveTG{蒤}{44194}
\saveTG{堞}{44194}
\saveTG{薻}{44194}
\saveTG{藻}{44194}
\saveTG{𫊋}{44194}
\saveTG{𧀪}{44196}
\saveTG{㙩}{44196}
\saveTG{𡒭}{44196}
\saveTG{𧀉}{44198}
\saveTG{𦸼}{44198}
\saveTG{𧃆}{44199}
\saveTG{𦫳}{44200}
\saveTG{犲}{44200}
\saveTG{𠀗}{44201}
\saveTG{𦬪}{44201}
\saveTG{葶}{44201}
\saveTG{𦷣}{44201}
\saveTG{䕜}{44201}
\saveTG{荹}{44201}
\saveTG{艼}{44201}
\saveTG{苧}{44201}
\saveTG{薴}{44201}
\saveTG{萨}{44201}
\saveTG{𦫺}{44201}
\saveTG{𦫻}{44202}
\saveTG{𤿋}{44202}
\saveTG{𤝦}{44202}
\saveTG{𦭏}{44202}
\saveTG{𦷇}{44202}
\saveTG{葪}{44202}
\saveTG{蓼}{44202}
\saveTG{蔘}{44202}
\saveTG{芗}{44202}
\saveTG{芧}{44202}
\saveTG{𣂍}{44203}
\saveTG{𢂆}{44203}
\saveTG{𤜮}{44203}
\saveTG{𤝔}{44203}
\saveTG{𤜥}{44203}
\saveTG{蔛}{44204}
\saveTG{茽}{44206}
\saveTG{芩}{44207}
\saveTG{棽}{44207}
\saveTG{𦴌}{44207}
\saveTG{萝}{44207}
\saveTG{梦}{44207}
\saveTG{夢}{44207}
\saveTG{𦻸}{44207}
\saveTG{𦱏}{44207}
\saveTG{㱳}{44207}
\saveTG{𡖶}{44207}
\saveTG{𦬉}{44207}
\saveTG{𫇧}{44207}
\saveTG{𣙄}{44207}
\saveTG{芕}{44207}
\saveTG{茤}{44207}
\saveTG{𧅸}{44207}
\saveTG{𦿏}{44207}
\saveTG{芦}{44207}
\saveTG{䒚}{44209}
\saveTG{𢂾}{44210}
\saveTG{𤞽}{44210}
\saveTG{尅}{44210}
\saveTG{兙}{44210}
\saveTG{𠀫}{44210}
\saveTG{䒡}{44210}
\saveTG{𠒚}{44210}
\saveTG{莋}{44211}
\saveTG{藶}{44211}
\saveTG{蘢}{44211}
\saveTG{蘼}{44211}
\saveTG{芢}{44211}
\saveTG{薤}{44211}
\saveTG{苲}{44211}
\saveTG{𦰼}{44211}
\saveTG{葄}{44211}
\saveTG{𦱯}{44211}
\saveTG{𦻰}{44211}
\saveTG{茺}{44212}
\saveTG{𧂎}{44212}
\saveTG{𦴪}{44212}
\saveTG{𧆓}{44212}
\saveTG{𧀔}{44212}
\saveTG{𦸡}{44212}
\saveTG{𦸍}{44212}
\saveTG{𦬺}{44212}
\saveTG{𧄵}{44212}
\saveTG{𦯜}{44212}
\saveTG{䔔}{44212}
\saveTG{𦳆}{44212}
\saveTG{𧅫}{44212}
\saveTG{𧠩}{44212}
\saveTG{䓲}{44212}
\saveTG{𦯇}{44212}
\saveTG{𤣡}{44212}
\saveTG{𤠡}{44212}
\saveTG{𢄍}{44212}
\saveTG{𢅤}{44212}
\saveTG{𠢩}{44212}
\saveTG{𦯣}{44212}
\saveTG{𦭟}{44212}
\saveTG{𦬳}{44212}
\saveTG{𦯵}{44212}
\saveTG{𦿅}{44212}
\saveTG{䓚}{44212}
\saveTG{䔃}{44212}
\saveTG{䒞}{44212}
\saveTG{𫉌}{44212}
\saveTG{𧅳}{44212}
\saveTG{䕻}{44212}
\saveTG{𧄍}{44212}
\saveTG{𦳎}{44212}
\saveTG{𦴓}{44212}
\saveTG{𦲮}{44212}
\saveTG{𦰜}{44212}
\saveTG{𦱆}{44212}
\saveTG{𦯓}{44212}
\saveTG{萖}{44212}
\saveTG{𦯲}{44212}
\saveTG{𣩾}{44212}
\saveTG{𦯱}{44212}
\saveTG{𦲵}{44212}
\saveTG{𦺷}{44212}
\saveTG{𦭶}{44212}
\saveTG{𦳽}{44212}
\saveTG{𦬮}{44212}
\saveTG{𦵽}{44212}
\saveTG{𦬂}{44212}
\saveTG{𠒌}{44212}
\saveTG{𢃜}{44212}
\saveTG{𦸌}{44212}
\saveTG{𦶤}{44212}
\saveTG{𦰟}{44212}
\saveTG{𦮲}{44212}
\saveTG{𦱔}{44212}
\saveTG{苑}{44212}
\saveTG{莸}{44212}
\saveTG{芫}{44212}
\saveTG{獟}{44212}
\saveTG{莧}{44212}
\saveTG{苋}{44212}
\saveTG{菀}{44212}
\saveTG{莌}{44212}
\saveTG{葹}{44212}
\saveTG{蕘}{44212}
\saveTG{荛}{44212}
\saveTG{莥}{44212}
\saveTG{苨}{44212}
\saveTG{藐}{44212}
\saveTG{薍}{44212}
\saveTG{蘆}{44212}
\saveTG{麓}{44212}
\saveTG{勊}{44212}
\saveTG{萒}{44212}
\saveTG{兢}{44212}
\saveTG{荒}{44212}
\saveTG{薨}{44212}
\saveTG{薧}{44212}
\saveTG{茪}{44212}
\saveTG{莞}{44212}
\saveTG{苊}{44212}
\saveTG{蔸}{44212}
\saveTG{狫}{44212}
\saveTG{帎}{44212}
\saveTG{蔖}{44212}
\saveTG{蔍}{44212}
\saveTG{藣}{44212}
\saveTG{萈}{44213}
\saveTG{𦳾}{44213}
\saveTG{𫈯}{44213}
\saveTG{䕇}{44213}
\saveTG{𡭄}{44213}
\saveTG{𫇺}{44214}
\saveTG{𦱍}{44214}
\saveTG{𦯿}{44214}
\saveTG{𠡘}{44214}
\saveTG{𦹠}{44214}
\saveTG{𧁆}{44214}
\saveTG{𦵤}{44214}
\saveTG{𫈙}{44214}
\saveTG{𦲒}{44214}
\saveTG{𫈃}{44214}
\saveTG{𦵞}{44214}
\saveTG{𦼄}{44214}
\saveTG{𢅣}{44214}
\saveTG{䓻}{44214}
\saveTG{𦸅}{44214}
\saveTG{蓙}{44214}
\saveTG{𦱙}{44214}
\saveTG{𦸐}{44214}
\saveTG{茬}{44214}
\saveTG{荱}{44214}
\saveTG{薼}{44214}
\saveTG{茌}{44214}
\saveTG{蒄}{44214}
\saveTG{花}{44214}
\saveTG{蔻}{44214}
\saveTG{蔲}{44214}
\saveTG{荏}{44214}
\saveTG{𧁶}{44214}
\saveTG{莊}{44214}
\saveTG{𦶲}{44214}
\saveTG{𪔴}{44215}
\saveTG{𥀼}{44215}
\saveTG{𦿮}{44215}
\saveTG{𦷃}{44215}
\saveTG{𦸰}{44215}
\saveTG{䔨}{44215}
\saveTG{𦻃}{44215}
\saveTG{𩁂}{44215}
\saveTG{𦸏}{44215}
\saveTG{𧆑}{44215}
\saveTG{𧃦}{44215}
\saveTG{𦶏}{44215}
\saveTG{𦼉}{44215}
\saveTG{𨤳}{44215}
\saveTG{蒮}{44215}
\saveTG{蕥}{44215}
\saveTG{蕹}{44215}
\saveTG{䔆}{44215}
\saveTG{𧄚}{44215}
\saveTG{薩}{44215}
\saveTG{虇}{44215}
\saveTG{薶}{44215}
\saveTG{蕯}{44215}
\saveTG{蘺}{44215}
\saveTG{萑}{44215}
\saveTG{獾}{44215}
\saveTG{藿}{44215}
\saveTG{雚}{44215}
\saveTG{藋}{44215}
\saveTG{𧄒}{44215}
\saveTG{𦿷}{44215}
\saveTG{𧅕}{44216}
\saveTG{䕬}{44216}
\saveTG{𠓘}{44216}
\saveTG{猹}{44216}
\saveTG{蓭}{44216}
\saveTG{𠙯}{44217}
\saveTG{𦜽}{44217}
\saveTG{𣮳}{44217}
\saveTG{㡋}{44217}
\saveTG{𦷔}{44217}
\saveTG{𦽏}{44217}
\saveTG{𫌥}{44217}
\saveTG{𫈋}{44217}
\saveTG{𣝹}{44217}
\saveTG{𪋤}{44217}
\saveTG{𤞓}{44217}
\saveTG{𤞠}{44217}
\saveTG{𤜴}{44217}
\saveTG{𤝂}{44217}
\saveTG{𤠤}{44217}
\saveTG{𤠛}{44217}
\saveTG{𢂢}{44217}
\saveTG{㡛}{44217}
\saveTG{𡔚}{44217}
\saveTG{𡋰}{44217}
\saveTG{𦶴}{44217}
\saveTG{㚁}{44217}
\saveTG{㯄}{44217}
\saveTG{𦬖}{44217}
\saveTG{𧆃}{44217}
\saveTG{𣏟}{44217}
\saveTG{𣛾}{44217}
\saveTG{𣝥}{44217}
\saveTG{梵}{44217}
\saveTG{萉}{44217}
\saveTG{檒}{44217}
\saveTG{苀}{44217}
\saveTG{萀}{44217}
\saveTG{蔰}{44217}
\saveTG{蘎}{44217}
\saveTG{芁}{44217}
\saveTG{葻}{44217}
\saveTG{芃}{44217}
\saveTG{棾}{44217}
\saveTG{𧁉}{44217}
\saveTG{𧅋}{44217}
\saveTG{𧄿}{44217}
\saveTG{䕦}{44217}
\saveTG{𦭀}{44217}
\saveTG{𦺺}{44217}
\saveTG{䒮}{44217}
\saveTG{𧁪}{44217}
\saveTG{𧁳}{44217}
\saveTG{𦲳}{44217}
\saveTG{𠙨}{44217}
\saveTG{𦳓}{44217}
\saveTG{𦶁}{44217}
\saveTG{𦶋}{44217}
\saveTG{𣜺}{44217}
\saveTG{𦴱}{44217}
\saveTG{𦾀}{44217}
\saveTG{𦹟}{44217}
\saveTG{𦻯}{44217}
\saveTG{𧆍}{44217}
\saveTG{𪔷}{44217}
\saveTG{𢅘}{44217}
\saveTG{𣗡}{44217}
\saveTG{𣑽}{44217}
\saveTG{𣗗}{44217}
\saveTG{犰}{44217}
\saveTG{𦱼}{44217}
\saveTG{莅}{44218}
\saveTG{𤡬}{44218}
\saveTG{𪔶}{44218}
\saveTG{𦮀}{44218}
\saveTG{𦼘}{44219}
\saveTG{𦱝}{44220}
\saveTG{萷}{44220}
\saveTG{茢}{44220}
\saveTG{𦼃}{44220}
\saveTG{𦴟}{44220}
\saveTG{𦮯}{44220}
\saveTG{𦲟}{44220}
\saveTG{𫇨}{44220}
\saveTG{芹}{44221}
\saveTG{萮}{44221}
\saveTG{猗}{44221}
\saveTG{𦸒}{44221}
\saveTG{𣟈}{44221}
\saveTG{荷}{44221}
\saveTG{蘅}{44221}
\saveTG{䕔}{44221}
\saveTG{𦺶}{44221}
\saveTG{𧁠}{44221}
\saveTG{𧁬}{44221}
\saveTG{𧁮}{44221}
\saveTG{𧄇}{44221}
\saveTG{䓄}{44221}
\saveTG{荇}{44221}
\saveTG{𦯅}{44221}
\saveTG{䕗}{44221}
\saveTG{𦮬}{44221}
\saveTG{𦯆}{44221}
\saveTG{𦶘}{44221}
\saveTG{葕}{44221}
\saveTG{𦸇}{44221}
\saveTG{蘮}{44221}
\saveTG{葥}{44221}
\saveTG{𦸔}{44222}
\saveTG{𢢐}{44222}
\saveTG{蓚}{44222}
\saveTG{茅}{44222}
\saveTG{𦳍}{44222}
\saveTG{𫈱}{44222}
\saveTG{𧅱}{44223}
\saveTG{𦻛}{44223}
\saveTG{薺}{44223}
\saveTG{𦺅}{44223}
\saveTG{𧆌}{44223}
\saveTG{鼘}{44224}
\saveTG{𧁋}{44224}
\saveTG{𦿝}{44224}
\saveTG{荠}{44224}
\saveTG{萕}{44224}
\saveTG{䔥}{44224}
\saveTG{𦳢}{44226}
\saveTG{𦺃}{44227}
\saveTG{𦬿}{44227}
\saveTG{𫊖}{44227}
\saveTG{𥀾}{44227}
\saveTG{𢄬}{44227}
\saveTG{𢄩}{44227}
\saveTG{𤠹}{44227}
\saveTG{𦮌}{44227}
\saveTG{䔚}{44227}
\saveTG{䔭}{44227}
\saveTG{𦱀}{44227}
\saveTG{䕍}{44227}
\saveTG{𦳉}{44227}
\saveTG{䒴}{44227}
\saveTG{𦯞}{44227}
\saveTG{𢃛}{44227}
\saveTG{𦯟}{44227}
\saveTG{䔕}{44227}
\saveTG{𦬝}{44227}
\saveTG{䒥}{44227}
\saveTG{𦲸}{44227}
\saveTG{𦬷}{44227}
\saveTG{㔑}{44227}
\saveTG{𦼡}{44227}
\saveTG{𦲐}{44227}
\saveTG{𧅑}{44227}
\saveTG{𦻾}{44227}
\saveTG{𪔜}{44227}
\saveTG{𪔝}{44227}
\saveTG{𪔚}{44227}
\saveTG{𦭄}{44227}
\saveTG{𧀴}{44227}
\saveTG{𦴗}{44227}
\saveTG{𧁿}{44227}
\saveTG{𦿈}{44227}
\saveTG{𧀎}{44227}
\saveTG{䓉}{44227}
\saveTG{𦰢}{44227}
\saveTG{𦯢}{44227}
\saveTG{𦽍}{44227}
\saveTG{𫉘}{44227}
\saveTG{𦵱}{44227}
\saveTG{𦽃}{44227}
\saveTG{𦰿}{44227}
\saveTG{𦭲}{44227}
\saveTG{𦵁}{44227}
\saveTG{𦼅}{44227}
\saveTG{𠢂}{44227}
\saveTG{𩱀}{44227}
\saveTG{𧀾}{44227}
\saveTG{𦵾}{44227}
\saveTG{𦚴}{44227}
\saveTG{𧅇}{44227}
\saveTG{䕥}{44227}
\saveTG{𦮪}{44227}
\saveTG{𦷉}{44227}
\saveTG{𦰱}{44227}
\saveTG{䕳}{44227}
\saveTG{蒡}{44227}
\saveTG{幇}{44227}
\saveTG{幫}{44227}
\saveTG{葡}{44227}
\saveTG{萹}{44227}
\saveTG{苪}{44227}
\saveTG{蔕}{44227}
\saveTG{带}{44227}
\saveTG{帯}{44227}
\saveTG{帶}{44227}
\saveTG{蒂}{44227}
\saveTG{苐}{44227}
\saveTG{蔐}{44227}
\saveTG{薡}{44227}
\saveTG{荋}{44227}
\saveTG{薾}{44227}
\saveTG{芳}{44227}
\saveTG{芾}{44227}
\saveTG{芬}{44227}
\saveTG{棼}{44227}
\saveTG{莆}{44227}
\saveTG{蒿}{44227}
\saveTG{獦}{44227}
\saveTG{蓇}{44227}
\saveTG{芴}{44227}
\saveTG{繭}{44227}
\saveTG{菺}{44227}
\saveTG{蕑}{44227}
\saveTG{蕳}{44227}
\saveTG{薦}{44227}
\saveTG{蕎}{44227}
\saveTG{荕}{44227}
\saveTG{菁}{44227}
\saveTG{萭}{44227}
\saveTG{勀}{44227}
\saveTG{蘭}{44227}
\saveTG{﨟}{44227}
\saveTG{苈}{44227}
\saveTG{蓠}{44227}
\saveTG{茘}{44227}
\saveTG{蒚}{44227}
\saveTG{蔺}{44227}
\saveTG{藺}{44227}
\saveTG{菕}{44227}
\saveTG{勱}{44227}
\saveTG{幕}{44227}
\saveTG{蔄}{44227}
\saveTG{蓩}{44227}
\saveTG{菛}{44227}
\saveTG{虋}{44227}
\saveTG{萠}{44227}
\saveTG{蕄}{44227}
\saveTG{芇}{44227}
\saveTG{艿}{44227}
\saveTG{萳}{44227}
\saveTG{朞}{44227}
\saveTG{蒨}{44227}
\saveTG{苘}{44227}
\saveTG{葋}{44227}
\saveTG{勸}{44227}
\saveTG{芿}{44227}
\saveTG{薷}{44227}
\saveTG{芮}{44227}
\saveTG{蔏}{44227}
\saveTG{莦}{44227}
\saveTG{蕂}{44227}
\saveTG{狶}{44227}
\saveTG{薚}{44227}
\saveTG{虅}{44227}
\saveTG{芀}{44227}
\saveTG{蓨}{44227}
\saveTG{茼}{44227}
\saveTG{茒}{44227}
\saveTG{萬}{44227}
\saveTG{菵}{44227}
\saveTG{蔿}{44227}
\saveTG{莴}{44227}
\saveTG{萵}{44227}
\saveTG{莃}{44227}
\saveTG{蓆}{44227}
\saveTG{薌}{44227}
\saveTG{萧}{44227}
\saveTG{蕭}{44227}
\saveTG{莠}{44227}
\saveTG{}{44227}
\saveTG{}{44227}
\saveTG{}{44227}
\saveTG{狕}{44227}
\saveTG{虉}{44227}
\saveTG{荫}{44227}
\saveTG{苚}{44227}
\saveTG{蓹}{44227}
\saveTG{蘥}{44227}
\saveTG{菷}{44227}
\saveTG{𧅆}{44227}
\saveTG{𦱘}{44227}
\saveTG{𦽦}{44227}
\saveTG{䓪}{44227}
\saveTG{𦸫}{44227}
\saveTG{𦮻}{44227}
\saveTG{𦽊}{44227}
\saveTG{𧃣}{44227}
\saveTG{𦬣}{44227}
\saveTG{䓟}{44227}
\saveTG{㒼}{44227}
\saveTG{𦮏}{44227}
\saveTG{䒽}{44227}
\saveTG{𦱌}{44227}
\saveTG{𦯶}{44227}
\saveTG{𦬌}{44227}
\saveTG{𦯃}{44227}
\saveTG{䕣}{44227}
\saveTG{𦭘}{44227}
\saveTG{𦰬}{44227}
\saveTG{𫉑}{44227}
\saveTG{𦾭}{44227}
\saveTG{𦼠}{44227}
\saveTG{䕡}{44227}
\saveTG{𧂄}{44227}
\saveTG{𦽅}{44227}
\saveTG{𧀲}{44227}
\saveTG{𦯔}{44227}
\saveTG{䕞}{44227}
\saveTG{𦬄}{44227}
\saveTG{𫈧}{44227}
\saveTG{𦷛}{44227}
\saveTG{𦸲}{44227}
\saveTG{𦯒}{44227}
\saveTG{𫇪}{44227}
\saveTG{𦷒}{44227}
\saveTG{𦾊}{44227}
\saveTG{𦻷}{44227}
\saveTG{𦹚}{44227}
\saveTG{𦲏}{44227}
\saveTG{𦻶}{44227}
\saveTG{𦻺}{44227}
\saveTG{𧅶}{44227}
\saveTG{𧄱}{44227}
\saveTG{𧃁}{44227}
\saveTG{𪢹}{44227}
\saveTG{𠠸}{44227}
\saveTG{𫈉}{44227}
\saveTG{𧁧}{44227}
\saveTG{𧁥}{44227}
\saveTG{𠡄}{44227}
\saveTG{𢅆}{44227}
\saveTG{𢆄}{44227}
\saveTG{𢁻}{44227}
\saveTG{𢂐}{44227}
\saveTG{𢂊}{44227}
\saveTG{𤣢}{44227}
\saveTG{㺃}{44227}
\saveTG{𤢨}{44227}
\saveTG{𤡁}{44227}
\saveTG{𤢥}{44227}
\saveTG{𤠬}{44227}
\saveTG{𤢃}{44227}
\saveTG{𤞩}{44227}
\saveTG{𤠃}{44227}
\saveTG{𤜜}{44227}
\saveTG{𤟋}{44227}
\saveTG{𫈏}{44227}
\saveTG{𫈂}{44227}
\saveTG{䔺}{44227}
\saveTG{𧄙}{44227}
\saveTG{𦼳}{44227}
\saveTG{𦻋}{44227}
\saveTG{䓣}{44227}
\saveTG{𧀫}{44227}
\saveTG{𧃷}{44227}
\saveTG{𦳝}{44227}
\saveTG{𦝙}{44227}
\saveTG{𧅿}{44227}
\saveTG{𦲭}{44227}
\saveTG{𦱫}{44227}
\saveTG{𦵓}{44227}
\saveTG{𧅢}{44227}
\saveTG{𦸢}{44227}
\saveTG{𢃄}{44227}
\saveTG{䒿}{44227}
\saveTG{𦭬}{44227}
\saveTG{节}{44227}
\saveTG{𢂸}{44227}
\saveTG{𠓶}{44227}
\saveTG{𦹤}{44227}
\saveTG{𫉤}{44227}
\saveTG{𤰇}{44227}
\saveTG{𦲅}{44227}
\saveTG{𧄷}{44227}
\saveTG{𦼱}{44227}
\saveTG{𦿶}{44227}
\saveTG{𧅯}{44227}
\saveTG{𫉒}{44227}
\saveTG{𦿬}{44227}
\saveTG{𧂶}{44227}
\saveTG{𪔤}{44227}
\saveTG{㡁}{44227}
\saveTG{𤡪}{44227}
\saveTG{𣛅}{44227}
\saveTG{𦺖}{44227}
\saveTG{𫉆}{44227}
\saveTG{芥}{44228}
\saveTG{荞}{44228}
\saveTG{𦸄}{44228}
\saveTG{𣓴}{44228}
\saveTG{𠍳}{44228}
\saveTG{𫈨}{44228}
\saveTG{𢟽}{44228}
\saveTG{𠄫}{44228}
\saveTG{𫇸}{44228}
\saveTG{𡗾}{44228}
\saveTG{𦰵}{44229}
\saveTG{𦰸}{44230}
\saveTG{苄}{44230}
\saveTG{𡭌}{44230}
\saveTG{𦬙}{44230}
\saveTG{𦭌}{44231}
\saveTG{𦳰}{44231}
\saveTG{㹤}{44231}
\saveTG{𦭐}{44231}
\saveTG{䕃}{44231}
\saveTG{蔭}{44231}
\saveTG{𢂴}{44231}
\saveTG{梺}{44231}
\saveTG{蕤}{44231}
\saveTG{芐}{44231}
\saveTG{赫}{44231}
\saveTG{藨}{44231}
\saveTG{𦚒}{44231}
\saveTG{𧃽}{44231}
\saveTG{𧹫}{44231}
\saveTG{𦹭}{44232}
\saveTG{𦴼}{44232}
\saveTG{𦲑}{44232}
\saveTG{𪷂}{44232}
\saveTG{莀}{44232}
\saveTG{藂}{44232}
\saveTG{苽}{44232}
\saveTG{苰}{44232}
\saveTG{蓏}{44232}
\saveTG{蒙}{44232}
\saveTG{獴}{44232}
\saveTG{蕽}{44232}
\saveTG{辳}{44232}
\saveTG{𦬭}{44232}
\saveTG{蕵}{44232}
\saveTG{薞}{44232}
\saveTG{蔭}{44232}
\saveTG{猿}{44232}
\saveTG{蒃}{44232}
\saveTG{𦴉}{44232}
\saveTG{𦺠}{44232}
\saveTG{𦱄}{44232}
\saveTG{𦬔}{44232}
\saveTG{𫈻}{44232}
\saveTG{𦲤}{44232}
\saveTG{𦭔}{44232}
\saveTG{𦹥}{44232}
\saveTG{䔹}{44232}
\saveTG{𦮈}{44232}
\saveTG{𦼫}{44232}
\saveTG{𦺛}{44232}
\saveTG{𧃶}{44232}
\saveTG{𧁓}{44232}
\saveTG{𦸾}{44232}
\saveTG{𧁼}{44232}
\saveTG{𦭕}{44232}
\saveTG{𤎗}{44232}
\saveTG{𧃚}{44232}
\saveTG{𦾜}{44232}
\saveTG{𦵧}{44232}
\saveTG{𦰐}{44232}
\saveTG{𪺻}{44232}
\saveTG{𦲶}{44232}
\saveTG{𦿹}{44232}
\saveTG{𦺨}{44232}
\saveTG{𦵺}{44232}
\saveTG{𦲙}{44232}
\saveTG{䓳}{44232}
\saveTG{𦱁}{44232}
\saveTG{𧁂}{44232}
\saveTG{𨑋}{44232}
\saveTG{幪}{44232}
\saveTG{𪔟}{44232}
\saveTG{𦻼}{44232}
\saveTG{𧲛}{44232}
\saveTG{𧄟}{44232}
\saveTG{𦴭}{44232}
\saveTG{𤢻}{44232}
\saveTG{𤣇}{44232}
\saveTG{𤢩}{44232}
\saveTG{𤢍}{44232}
\saveTG{𦮭}{44232}
\saveTG{菸}{44233}
\saveTG{𧂿}{44233}
\saveTG{苯}{44234}
\saveTG{㺚}{44234}
\saveTG{𧀕}{44234}
\saveTG{㿹}{44235}
\saveTG{蔃}{44236}
\saveTG{藘}{44236}
\saveTG{𦷸}{44236}
\saveTG{蔗}{44237}
\saveTG{𦻕}{44237}
\saveTG{𦮩}{44237}
\saveTG{蒹}{44237}
\saveTG{薕}{44237}
\saveTG{蘟}{44237}
\saveTG{𧀋}{44239}
\saveTG{𦭚}{44240}
\saveTG{𦲴}{44240}
\saveTG{苻}{44240}
\saveTG{蔚}{44240}
\saveTG{𦱖}{44240}
\saveTG{𦬁}{44240}
\saveTG{𦬴}{44240}
\saveTG{𦬯}{44240}
\saveTG{𦘽}{44240}
\saveTG{𦵮}{44241}
\saveTG{𤝊}{44241}
\saveTG{𦱎}{44241}
\saveTG{𦭿}{44241}
\saveTG{葞}{44241}
\saveTG{芽}{44241}
\saveTG{幬}{44241}
\saveTG{薜}{44241}
\saveTG{𫊏}{44241}
\saveTG{𪶗}{44241}
\saveTG{𦰣}{44242}
\saveTG{𦭛}{44242}
\saveTG{𦰘}{44242}
\saveTG{𣃕}{44242}
\saveTG{𦺉}{44242}
\saveTG{菧}{44242}
\saveTG{蔣}{44242}
\saveTG{蓐}{44243}
\saveTG{𣡇}{44243}
\saveTG{𣡸}{44243}
\saveTG{𣟜}{44243}
\saveTG{𤠭}{44243}
\saveTG{䒫}{44243}
\saveTG{𠁀}{44243}
\saveTG{𧃱}{44244}
\saveTG{䔀}{44244}
\saveTG{𫉛}{44244}
\saveTG{𧀓}{44244}
\saveTG{𦽵}{44244}
\saveTG{㘷}{44244}
\saveTG{𫈘}{44244}
\saveTG{𦽇}{44245}
\saveTG{𦳅}{44245}
\saveTG{𦹫}{44246}
\saveTG{𦹽}{44246}
\saveTG{𦸣}{44246}
\saveTG{𦲺}{44246}
\saveTG{𧀷}{44246}
\saveTG{𦳄}{44246}
\saveTG{𦲹}{44247}
\saveTG{䓈}{44247}
\saveTG{蒑}{44247}
\saveTG{藙}{44247}
\saveTG{葠}{44247}
\saveTG{狓}{44247}
\saveTG{帔}{44247}
\saveTG{蘉}{44247}
\saveTG{葰}{44247}
\saveTG{荐}{44247}
\saveTG{菔}{44247}
\saveTG{葭}{44247}
\saveTG{蕧}{44247}
\saveTG{蕟}{44247}
\saveTG{𦶬}{44247}
\saveTG{𦹿}{44247}
\saveTG{䔖}{44247}
\saveTG{𦹜}{44247}
\saveTG{𧄁}{44247}
\saveTG{𦲗}{44247}
\saveTG{𦳧}{44247}
\saveTG{𫈢}{44247}
\saveTG{𫉖}{44247}
\saveTG{𦹡}{44247}
\saveTG{𧄏}{44247}
\saveTG{𥀙}{44247}
\saveTG{𦴯}{44247}
\saveTG{𦳔}{44247}
\saveTG{𦡨}{44247}
\saveTG{𦚆}{44247}
\saveTG{𧹞}{44247}
\saveTG{獲}{44247}
\saveTG{𥀇}{44247}
\saveTG{𦵦}{44247}
\saveTG{𦵥}{44247}
\saveTG{𦳋}{44247}
\saveTG{𢻎}{44247}
\saveTG{𦬐}{44247}
\saveTG{𧄘}{44247}
\saveTG{𢂣}{44247}
\saveTG{𡔮}{44247}
\saveTG{𢆆}{44247}
\saveTG{𤞐}{44247}
\saveTG{㹲}{44247}
\saveTG{𥀹}{44247}
\saveTG{𦺭}{44247}
\saveTG{𥀂}{44247}
\saveTG{芨}{44247}
\saveTG{𫇱}{44247}
\saveTG{𠮂}{44247}
\saveTG{䕠}{44247}
\saveTG{𦰞}{44247}
\saveTG{𦺔}{44247}
\saveTG{𦽛}{44247}
\saveTG{𦽄}{44247}
\saveTG{𦼷}{44247}
\saveTG{𫉬}{44247}
\saveTG{𨤺}{44247}
\saveTG{𧹛}{44247}
\saveTG{𧄬}{44247}
\saveTG{䕾}{44248}
\saveTG{𦽺}{44248}
\saveTG{蔜}{44248}
\saveTG{蔽}{44248}
\saveTG{𦽗}{44248}
\saveTG{薇}{44248}
\saveTG{薂}{44248}
\saveTG{𦺻}{44248}
\saveTG{𦽠}{44248}
\saveTG{䓮}{44248}
\saveTG{𧀮}{44248}
\saveTG{𫉪}{44248}
\saveTG{𫉡}{44248}
\saveTG{䕧}{44248}
\saveTG{𦵨}{44248}
\saveTG{𦱿}{44248}
\saveTG{𧃸}{44248}
\saveTG{𧁱}{44248}
\saveTG{𦵌}{44248}
\saveTG{𫈚}{44248}
\saveTG{莜}{44248}
\saveTG{藢}{44248}
\saveTG{荈}{44250}
\saveTG{𧂌}{44251}
\saveTG{𦿱}{44251}
\saveTG{䔗}{44251}
\saveTG{𦳟}{44251}
\saveTG{藦}{44252}
\saveTG{蕣}{44252}
\saveTG{薢}{44252}
\saveTG{𦺐}{44253}
\saveTG{𫈼}{44253}
\saveTG{𥀯}{44253}
\saveTG{𣜕}{44253}
\saveTG{茷}{44253}
\saveTG{藏}{44253}
\saveTG{蔵}{44253}
\saveTG{蒇}{44253}
\saveTG{蕆}{44253}
\saveTG{荿}{44253}
\saveTG{蒧}{44253}
\saveTG{薉}{44253}
\saveTG{茂}{44253}
\saveTG{蔑}{44253}
\saveTG{幭}{44253}
\saveTG{薎}{44253}
\saveTG{荗}{44253}
\saveTG{葳}{44253}
\saveTG{𥀽}{44253}
\saveTG{葴}{44253}
\saveTG{𥀻}{44253}
\saveTG{𦿧}{44253}
\saveTG{𫈦}{44253}
\saveTG{𦿐}{44253}
\saveTG{䕙}{44253}
\saveTG{𦸗}{44253}
\saveTG{𦮠}{44253}
\saveTG{𪔯}{44253}
\saveTG{𪔩}{44253}
\saveTG{𦴹}{44253}
\saveTG{𦵋}{44255}
\saveTG{𫉔}{44256}
\saveTG{𤠇}{44256}
\saveTG{𢃲}{44256}
\saveTG{𪳽}{44256}
\saveTG{幃}{44256}
\saveTG{𦸵}{44257}
\saveTG{𥀊}{44257}
\saveTG{𦲋}{44257}
\saveTG{𧂮}{44257}
\saveTG{𦱲}{44257}
\saveTG{𦳳}{44257}
\saveTG{葎}{44257}
\saveTG{𦾋}{44257}
\saveTG{𦼗}{44259}
\saveTG{𦺸}{44259}
\saveTG{𫇰}{44260}
\saveTG{𦻿}{44260}
\saveTG{𦲡}{44260}
\saveTG{𤿛}{44260}
\saveTG{猪}{44260}
\saveTG{赭}{44260}
\saveTG{猫}{44260}
\saveTG{狜}{44260}
\saveTG{帾}{44260}
\saveTG{𪡻}{44260}
\saveTG{𫈆}{44260}
\saveTG{猎}{44261}
\saveTG{𢃟}{44261}
\saveTG{𢅯}{44261}
\saveTG{𦾔}{44261}
\saveTG{𤢁}{44261}
\saveTG{𦸻}{44261}
\saveTG{𤿸}{44261}
\saveTG{蘠}{44261}
\saveTG{𦶹}{44261}
\saveTG{𤞺}{44261}
\saveTG{𧄯}{44261}
\saveTG{𦴩}{44261}
\saveTG{薝}{44261}
\saveTG{萜}{44261}
\saveTG{𧀻}{44261}
\saveTG{𦯰}{44261}
\saveTG{狤}{44261}
\saveTG{茩}{44261}
\saveTG{蓓}{44261}
\saveTG{𤢀}{44261}
\saveTG{𤠟}{44261}
\saveTG{𧃇}{44261}
\saveTG{𦺎}{44261}
\saveTG{䔒}{44261}
\saveTG{𦺼}{44261}
\saveTG{𢅴}{44261}
\saveTG{𦯉}{44262}
\saveTG{𦯐}{44262}
\saveTG{𦰒}{44262}
\saveTG{𦼑}{44262}
\saveTG{蓿}{44262}
\saveTG{𦿛}{44262}
\saveTG{蘑}{44262}
\saveTG{𥀌}{44263}
\saveTG{𧂤}{44263}
\saveTG{𦿊}{44263}
\saveTG{𤠂}{44263}
\saveTG{藸}{44264}
\saveTG{蕏}{44264}
\saveTG{𦱅}{44264}
\saveTG{𢄳}{44264}
\saveTG{𧄔}{44264}
\saveTG{𫈈}{44264}
\saveTG{𧹻}{44264}
\saveTG{蕕}{44264}
\saveTG{𦲥}{44265}
\saveTG{蓎}{44265}
\saveTG{𦾕}{44266}
\saveTG{䕐}{44266}
\saveTG{𦽪}{44266}
\saveTG{𦽌}{44266}
\saveTG{葿}{44267}
\saveTG{𦴨}{44267}
\saveTG{蒼}{44267}
\saveTG{䓛}{44272}
\saveTG{苮}{44272}
\saveTG{𦭽}{44274}
\saveTG{𡔘}{44275}
\saveTG{𦫼}{44277}
\saveTG{𦾩}{44277}
\saveTG{𦯄}{44277}
\saveTG{䔷}{44278}
\saveTG{𣜓}{44278}
\saveTG{𢁖}{44280}
\saveTG{𤠶}{44281}
\saveTG{𧃢}{44281}
\saveTG{𦻬}{44281}
\saveTG{𢂔}{44281}
\saveTG{𤞒}{44281}
\saveTG{蓯}{44281}
\saveTG{帺}{44281}
\saveTG{猉}{44281}
\saveTG{蓗}{44281}
\saveTG{蔙}{44281}
\saveTG{蓰}{44281}
\saveTG{}{44282}
\saveTG{𫈾}{44282}
\saveTG{𦻀}{44282}
\saveTG{𪩸}{44282}
\saveTG{𦲽}{44282}
\saveTG{㡕}{44282}
\saveTG{㺆}{44282}
\saveTG{藃}{44282}
\saveTG{蕨}{44282}
\saveTG{蓣}{44282}
\saveTG{𧃩}{44282}
\saveTG{𤝁}{44283}
\saveTG{𫈄}{44283}
\saveTG{𤢑}{44284}
\saveTG{𦽕}{44284}
\saveTG{䓞}{44284}
\saveTG{𦲚}{44284}
\saveTG{𧀀}{44284}
\saveTG{𦾚}{44284}
\saveTG{𥀳}{44284}
\saveTG{𦟦}{44284}
\saveTG{蔟}{44284}
\saveTG{茯}{44284}
\saveTG{葔}{44284}
\saveTG{獏}{44284}
\saveTG{幙}{44284}
\saveTG{𦺮}{44284}
\saveTG{获}{44284}
\saveTG{𦿍}{44285}
\saveTG{𤡞}{44285}
\saveTG{𤠉}{44285}
\saveTG{𦴊}{44285}
\saveTG{𤡉}{44285}
\saveTG{𦿜}{44286}
\saveTG{𧂹}{44286}
\saveTG{幩}{44286}
\saveTG{獚}{44286}
\saveTG{蘋}{44286}
\saveTG{藬}{44286}
\saveTG{獖}{44286}
\saveTG{蘈}{44286}
\saveTG{蕷}{44286}
\saveTG{𧅙}{44286}
\saveTG{𧂀}{44286}
\saveTG{𦹗}{44286}
\saveTG{蕦}{44286}
\saveTG{𥀲}{44286}
\saveTG{𦡛}{44286}
\saveTG{菮}{44287}
\saveTG{𢂿}{44288}
\saveTG{𦷈}{44288}
\saveTG{狹}{44288}
\saveTG{𤢙}{44289}
\saveTG{荻}{44289}
\saveTG{㹯}{44290}
\saveTG{狇}{44290}
\saveTG{䕲}{44291}
\saveTG{𧂷}{44291}
\saveTG{𦺯}{44291}
\saveTG{𦰴}{44292}
\saveTG{𦵖}{44292}
\saveTG{䕷}{44293}
\saveTG{𧃧}{44293}
\saveTG{𦵠}{44293}
\saveTG{𦺰}{44293}
\saveTG{䕨}{44293}
\saveTG{𦽝}{44293}
\saveTG{𦶽}{44293}
\saveTG{𧃲}{44294}
\saveTG{葆}{44294}
\saveTG{蒢}{44294}
\saveTG{蓧}{44294}
\saveTG{幉}{44294}
\saveTG{𧂾}{44294}
\saveTG{茠}{44294}
\saveTG{蔴}{44294}
\saveTG{蘪}{44294}
\saveTG{蒣}{44294}
\saveTG{𢅿}{44294}
\saveTG{𦾰}{44294}
\saveTG{𫈟}{44294}
\saveTG{𧁫}{44294}
\saveTG{𤟬}{44294}
\saveTG{𢃱}{44294}
\saveTG{𦼶}{44294}
\saveTG{䕈}{44294}
\saveTG{}{44294}
\saveTG{𫊅}{44295}
\saveTG{𧀸}{44295}
\saveTG{蒝}{44296}
\saveTG{𢄷}{44296}
\saveTG{蔯}{44296}
\saveTG{獠}{44296}
\saveTG{𦰀}{44297}
\saveTG{𦲎}{44297}
\saveTG{猍}{44298}
\saveTG{𦱨}{44298}
\saveTG{𦽎}{44299}
\saveTG{藤}{44299}
\saveTG{𧀬}{44299}
\saveTG{𡬝}{44300}
\saveTG{𦻗}{44301}
\saveTG{䓕}{44301}
\saveTG{𦹇}{44301}
\saveTG{𦰭}{44301}
\saveTG{苓}{44302}
\saveTG{蕶}{44302}
\saveTG{菦}{44302}
\saveTG{蓪}{44302}
\saveTG{芝}{44302}
\saveTG{𨙊}{44302}
\saveTG{𦳭}{44302}
\saveTG{𦼛}{44302}
\saveTG{𦼯}{44302}
\saveTG{𦽟}{44302}
\saveTG{薖}{44302}
\saveTG{藡}{44302}
\saveTG{薊}{44302}
\saveTG{𣕯}{44302}
\saveTG{椘}{44302}
\saveTG{鼕}{44303}
\saveTG{𦺦}{44303}
\saveTG{蓫}{44303}
\saveTG{蘧}{44303}
\saveTG{薳}{44303}
\saveTG{荩}{44303}
\saveTG{𧀹}{44303}
\saveTG{𦰈}{44303}
\saveTG{苳}{44303}
\saveTG{𫉽}{44304}
\saveTG{蕸}{44304}
\saveTG{𦳺}{44304}
\saveTG{𦹢}{44304}
\saveTG{𪔲}{44305}
\saveTG{𦾼}{44305}
\saveTG{𧁽}{44305}
\saveTG{薘}{44305}
\saveTG{莲}{44305}
\saveTG{蓬}{44305}
\saveTG{蓮}{44305}
\saveTG{𦶅}{44306}
\saveTG{𦴚}{44306}
\saveTG{䔏}{44306}
\saveTG{𦷿}{44306}
\saveTG{𨖘}{44306}
\saveTG{𨖗}{44306}
\saveTG{䒦}{44307}
\saveTG{䕂}{44307}
\saveTG{𦾾}{44307}
\saveTG{𦷷}{44308}
\saveTG{𧂍}{44308}
\saveTG{𦷴}{44308}
\saveTG{𧂠}{44308}
\saveTG{𦿚}{44308}
\saveTG{荙}{44308}
\saveTG{藗}{44308}
\saveTG{䕖}{44309}
\saveTG{䔎}{44309}
\saveTG{蒾}{44309}
\saveTG{蒁}{44309}
\saveTG{𧂏}{44309}
\saveTG{𦾽}{44309}
\saveTG{𪧷}{44310}
\saveTG{𫊉}{44312}
\saveTG{𧂫}{44312}
\saveTG{𫊒}{44314}
\saveTG{𩼻}{44316}
\saveTG{𦭴}{44320}
\saveTG{𦱰}{44320}
\saveTG{薊}{44320}
\saveTG{𪁍}{44323}
\saveTG{𧁦}{44327}
\saveTG{蕮}{44327}
\saveTG{𦿇}{44327}
\saveTG{𪄪}{44327}
\saveTG{蔫}{44327}
\saveTG{芍}{44327}
\saveTG{𪃳}{44327}
\saveTG{𩤆}{44327}
\saveTG{𦵟}{44327}
\saveTG{𩤯}{44327}
\saveTG{䔍}{44327}
\saveTG{𪂅}{44327}
\saveTG{𪈞}{44327}
\saveTG{𪇞}{44327}
\saveTG{𪇀}{44327}
\saveTG{𦾉}{44327}
\saveTG{𧅧}{44327}
\saveTG{𪅇}{44327}
\saveTG{蒍}{44327}
\saveTG{𪇔}{44327}
\saveTG{蔦}{44327}
\saveTG{𪈏}{44327}
\saveTG{𦿲}{44327}
\saveTG{𦿂}{44327}
\saveTG{鷰}{44327}
\saveTG{蘍}{44327}
\saveTG{藛}{44327}
\saveTG{𦶀}{44327}
\saveTG{驀}{44327}
\saveTG{𢡗}{44330}
\saveTG{𫈝}{44330}
\saveTG{𢚳}{44330}
\saveTG{芯}{44330}
\saveTG{𢝨}{44330}
\saveTG{𦯹}{44330}
\saveTG{苏}{44330}
\saveTG{𦮼}{44331}
\saveTG{𦶟}{44331}
\saveTG{𪸦}{44331}
\saveTG{𢝖}{44331}
\saveTG{𢦊}{44331}
\saveTG{蕉}{44331}
\saveTG{爇}{44331}
\saveTG{莣}{44331}
\saveTG{𦯌}{44331}
\saveTG{蕪}{44331}
\saveTG{薰}{44331}
\saveTG{薫}{44331}
\saveTG{燕}{44331}
\saveTG{蓔}{44331}
\saveTG{葾}{44331}
\saveTG{蒸}{44331}
\saveTG{蕜}{44331}
\saveTG{䓇}{44331}
\saveTG{𫉧}{44331}
\saveTG{𦻹}{44331}
\saveTG{𧀛}{44331}
\saveTG{𦼇}{44331}
\saveTG{𧄾}{44331}
\saveTG{𦳤}{44331}
\saveTG{㥨}{44331}
\saveTG{𢛖}{44331}
\saveTG{𦮤}{44331}
\saveTG{䓌}{44331}
\saveTG{𫉝}{44331}
\saveTG{𧅌}{44331}
\saveTG{𦸽}{44331}
\saveTG{䖄}{44331}
\saveTG{𧆒}{44331}
\saveTG{𧅗}{44331}
\saveTG{𧄄}{44331}
\saveTG{𪒅}{44331}
\saveTG{𦶐}{44331}
\saveTG{愸}{44332}
\saveTG{䓤}{44332}
\saveTG{𧂡}{44332}
\saveTG{𢤘}{44332}
\saveTG{𤏤}{44332}
\saveTG{𦾤}{44332}
\saveTG{愂}{44332}
\saveTG{懃}{44332}
\saveTG{菍}{44332}
\saveTG{𦽳}{44332}
\saveTG{𦱻}{44332}
\saveTG{𧀱}{44332}
\saveTG{𦾂}{44332}
\saveTG{𦶣}{44332}
\saveTG{𦯡}{44332}
\saveTG{𧂋}{44332}
\saveTG{荵}{44332}
\saveTG{慸}{44332}
\saveTG{葱}{44332}
\saveTG{𢣋}{44332}
\saveTG{𢟮}{44332}
\saveTG{𤓘}{44332}
\saveTG{𤐂}{44332}
\saveTG{𤏊}{44332}
\saveTG{𧁊}{44333}
\saveTG{蕙}{44333}
\saveTG{𧄅}{44333}
\saveTG{𦿌}{44333}
\saveTG{蕊}{44333}
\saveTG{𢠱}{44334}
\saveTG{𧅐}{44334}
\saveTG{苾}{44334}
\saveTG{𦼈}{44334}
\saveTG{𣕾}{44334}
\saveTG{𦰍}{44334}
\saveTG{𦵚}{44334}
\saveTG{藯}{44334}
\saveTG{𦽫}{44335}
\saveTG{䔡}{44336}
\saveTG{蔥}{44336}
\saveTG{𧆊}{44336}
\saveTG{𦴜}{44336}
\saveTG{𢞨}{44336}
\saveTG{𦷋}{44336}
\saveTG{蒽}{44336}
\saveTG{蔒}{44336}
\saveTG{惹}{44336}
\saveTG{葸}{44336}
\saveTG{蒠}{44336}
\saveTG{藼}{44336}
\saveTG{薏}{44336}
\saveTG{𫊌}{44336}
\saveTG{𦷓}{44336}
\saveTG{𦵇}{44336}
\saveTG{𦵅}{44336}
\saveTG{𫉅}{44336}
\saveTG{煮}{44336}
\saveTG{𦷢}{44336}
\saveTG{𢡳}{44336}
\saveTG{𢙄}{44336}
\saveTG{𦷚}{44337}
\saveTG{𦮡}{44337}
\saveTG{𧁜}{44337}
\saveTG{𦳌}{44337}
\saveTG{𦮋}{44337}
\saveTG{𦾻}{44337}
\saveTG{惎}{44338}
\saveTG{恭}{44338}
\saveTG{𧀙}{44338}
\saveTG{䔳}{44338}
\saveTG{𢜯}{44338}
\saveTG{𢚈}{44338}
\saveTG{䓗}{44338}
\saveTG{𦼒}{44338}
\saveTG{𦹎}{44338}
\saveTG{𦷧}{44338}
\saveTG{𤈉}{44338}
\saveTG{𢤵}{44338}
\saveTG{𦵃}{44338}
\saveTG{𦮕}{44338}
\saveTG{菾}{44338}
\saveTG{慕}{44338}
\saveTG{𦸝}{44339}
\saveTG{㷊}{44339}
\saveTG{𢜨}{44339}
\saveTG{𦲓}{44339}
\saveTG{懋}{44339}
\saveTG{𧃅}{44339}
\saveTG{𢠖}{44339}
\saveTG{𢠘}{44339}
\saveTG{𢛓}{44339}
\saveTG{𢣬}{44339}
\saveTG{𢗦}{44339}
\saveTG{𦱷}{44339}
\saveTG{𧆎}{44339}
\saveTG{𢡟}{44339}
\saveTG{䒭}{44341}
\saveTG{䓁}{44341}
\saveTG{𡭆}{44341}
\saveTG{蒪}{44342}
\saveTG{蓴}{44343}
\saveTG{䓓}{44344}
\saveTG{䔿}{44346}
\saveTG{蕁}{44346}
\saveTG{荨}{44347}
\saveTG{𦼼}{44347}
\saveTG{蘚}{44351}
\saveTG{㰈}{44356}
\saveTG{䕸}{44361}
\saveTG{𫇦}{44370}
\saveTG{𡚡}{44380}
\saveTG{䕘}{44386}
\saveTG{𧆈}{44386}
\saveTG{蘇}{44394}
\saveTG{𢌰}{44400}
\saveTG{𦶎}{44400}
\saveTG{𠂀}{44400}
\saveTG{𣁙}{44400}
\saveTG{𦻔}{44400}
\saveTG{𦻱}{44400}
\saveTG{艾}{44400}
\saveTG{芠}{44400}
\saveTG{𠦧}{44400}
\saveTG{𦺆}{44401}
\saveTG{𦭨}{44401}
\saveTG{葺}{44401}
\saveTG{𥮌}{44401}
\saveTG{莚}{44401}
\saveTG{芋}{44401}
\saveTG{莛}{44401}
\saveTG{莘}{44401}
\saveTG{茸}{44401}
\saveTG{荢}{44401}
\saveTG{𦗺}{44401}
\saveTG{𦯼}{44401}
\saveTG{𦬡}{44401}
\saveTG{𤯀}{44401}
\saveTG{𦰗}{44401}
\saveTG{𪔪}{44401}
\saveTG{𫉞}{44401}
\saveTG{𦿵}{44401}
\saveTG{䓍}{44401}
\saveTG{𦴁}{44401}
\saveTG{𫉯}{44401}
\saveTG{𦼆}{44401}
\saveTG{芉}{44401}
\saveTG{𪯆}{44401}
\saveTG{荜}{44401}
\saveTG{𫉻}{44401}
\saveTG{𨐞}{44401}
\saveTG{䔂}{44401}
\saveTG{䕉}{44401}
\saveTG{𫉠}{44401}
\saveTG{𦭱}{44401}
\saveTG{𦮜}{44402}
\saveTG{芊}{44402}
\saveTG{荊}{44402}
\saveTG{䴭}{44402}
\saveTG{攰}{44402}
\saveTG{莂}{44402}
\saveTG{芆}{44403}
\saveTG{䔞}{44403}
\saveTG{䕫}{44403}
\saveTG{芅}{44403}
\saveTG{𦾨}{44404}
\saveTG{𦽚}{44404}
\saveTG{㜂}{44404}
\saveTG{𡤈}{44404}
\saveTG{𦵘}{44404}
\saveTG{嬊}{44404}
\saveTG{𦯽}{44404}
\saveTG{䓾}{44404}
\saveTG{𦼧}{44404}
\saveTG{𦬑}{44404}
\saveTG{䒝}{44404}
\saveTG{𦾦}{44404}
\saveTG{蔢}{44404}
\saveTG{蘡}{44404}
\saveTG{葽}{44404}
\saveTG{蔅}{44404}
\saveTG{蒆}{44404}
\saveTG{萎}{44404}
\saveTG{荽}{44404}
\saveTG{菨}{44404}
\saveTG{萋}{44404}
\saveTG{婪}{44404}
\saveTG{蔞}{44404}
\saveTG{蒌}{44404}
\saveTG{葁}{44404}
\saveTG{荌}{44404}
\saveTG{𪦵}{44404}
\saveTG{𦯺}{44404}
\saveTG{𡠜}{44404}
\saveTG{𧂜}{44404}
\saveTG{𦰇}{44404}
\saveTG{㜸}{44404}
\saveTG{𦽮}{44404}
\saveTG{𡣐}{44404}
\saveTG{𧂳}{44405}
\saveTG{𥀷}{44405}
\saveTG{䓬}{44406}
\saveTG{𧅵}{44406}
\saveTG{𦯑}{44406}
\saveTG{𫉹}{44406}
\saveTG{萆}{44406}
\saveTG{草}{44406}
\saveTG{鼙}{44406}
\saveTG{蕈}{44406}
\saveTG{蔁}{44406}
\saveTG{蒦}{44407}
\saveTG{孝}{44407}
\saveTG{萲}{44407}
\saveTG{苃}{44407}
\saveTG{芓}{44407}
\saveTG{茡}{44407}
\saveTG{葼}{44407}
\saveTG{𧃰}{44407}
\saveTG{𧃍}{44407}
\saveTG{𧅚}{44407}
\saveTG{𦳡}{44407}
\saveTG{䒘}{44407}
\saveTG{𦫿}{44407}
\saveTG{𦯈}{44407}
\saveTG{𫉍}{44407}
\saveTG{𫉁}{44407}
\saveTG{𦰹}{44407}
\saveTG{𧅄}{44407}
\saveTG{茇}{44407}
\saveTG{薆}{44407}
\saveTG{𦭼}{44407}
\saveTG{𦶨}{44407}
\saveTG{𪎃}{44407}
\saveTG{虁}{44407}
\saveTG{𧃜}{44407}
\saveTG{𦲯}{44407}
\saveTG{𦯘}{44407}
\saveTG{𦴅}{44407}
\saveTG{𦲪}{44407}
\saveTG{𦾪}{44407}
\saveTG{𦰺}{44407}
\saveTG{𦴞}{44407}
\saveTG{𠦶}{44407}
\saveTG{𦰊}{44407}
\saveTG{𦴋}{44407}
\saveTG{䓔}{44407}
\saveTG{𧃯}{44407}
\saveTG{𦴲}{44407}
\saveTG{𦱟}{44407}
\saveTG{𦱛}{44407}
\saveTG{𦭞}{44407}
\saveTG{𤼱}{44407}
\saveTG{蓌}{44407}
\saveTG{荾}{44407}
\saveTG{蓃}{44407}
\saveTG{芟}{44407}
\saveTG{藑}{44407}
\saveTG{孽}{44407}
\saveTG{蔓}{44407}
\saveTG{菱}{44407}
\saveTG{芰}{44407}
\saveTG{莩}{44407}
\saveTG{荸}{44407}
\saveTG{𧀥}{44407}
\saveTG{𦯩}{44407}
\saveTG{𦵲}{44407}
\saveTG{𦰑}{44407}
\saveTG{𦬏}{44407}
\saveTG{𡥈}{44407}
\saveTG{𧄐}{44407}
\saveTG{𦳠}{44407}
\saveTG{𧁰}{44407}
\saveTG{𧀧}{44407}
\saveTG{萃}{44408}
\saveTG{𦬶}{44408}
\saveTG{茭}{44408}
\saveTG{𣗚}{44408}
\saveTG{𫇫}{44408}
\saveTG{𦮫}{44408}
\saveTG{𧄀}{44409}
\saveTG{苸}{44409}
\saveTG{苹}{44409}
\saveTG{𡚳}{44410}
\saveTG{𡝂}{44410}
\saveTG{葂}{44412}
\saveTG{嬈}{44412}
\saveTG{姺}{44412}
\saveTG{芜}{44412}
\saveTG{𦬓}{44412}
\saveTG{𦯍}{44412}
\saveTG{𡡰}{44412}
\saveTG{她}{44412}
\saveTG{妉}{44412}
\saveTG{芤}{44412}
\saveTG{娔}{44412}
\saveTG{姥}{44412}
\saveTG{莬}{44412}
\saveTG{𦲁}{44412}
\saveTG{𦬍}{44412}
\saveTG{𡛿}{44412}
\saveTG{𦳨}{44412}
\saveTG{𡢪}{44412}
\saveTG{𦭋}{44412}
\saveTG{𦵔}{44412}
\saveTG{𦯏}{44412}
\saveTG{𦳹}{44412}
\saveTG{𦹅}{44412}
\saveTG{𦰾}{44412}
\saveTG{𡖻}{44412}
\saveTG{𡣨}{44412}
\saveTG{㛯}{44412}
\saveTG{𦴧}{44413}
\saveTG{蒬}{44413}
\saveTG{菟}{44413}
\saveTG{蘶}{44413}
\saveTG{㛬}{44414}
\saveTG{𦰋}{44414}
\saveTG{䒲}{44414}
\saveTG{㛻}{44414}
\saveTG{䓼}{44414}
\saveTG{娃}{44414}
\saveTG{嬯}{44414}
\saveTG{茏}{44414}
\saveTG{婲}{44414}
\saveTG{嫤}{44415}
\saveTG{嬞}{44415}
\saveTG{𡟻}{44415}
\saveTG{𪎅}{44415}
\saveTG{孉}{44415}
\saveTG{𦶇}{44415}
\saveTG{㜁}{44416}
\saveTG{㭝}{44417}
\saveTG{𡝴}{44417}
\saveTG{𤕩}{44417}
\saveTG{𡡄}{44417}
\saveTG{𦲩}{44417}
\saveTG{𡚨}{44417}
\saveTG{𪟵}{44417}
\saveTG{𦶢}{44417}
\saveTG{䒖}{44417}
\saveTG{𦽓}{44417}
\saveTG{茕}{44417}
\saveTG{蓻}{44417}
\saveTG{艽}{44417}
\saveTG{𡟃}{44417}
\saveTG{𡡶}{44417}
\saveTG{𦲛}{44417}
\saveTG{𧅉}{44417}
\saveTG{㛪}{44417}
\saveTG{𡛶}{44417}
\saveTG{𡜤}{44417}
\saveTG{嬄}{44418}
\saveTG{𫈐}{44418}
\saveTG{㛸}{44418}
\saveTG{媅}{44418}
\saveTG{𦲨}{44418}
\saveTG{𦮓}{44420}
\saveTG{苅}{44420}
\saveTG{婍}{44421}
\saveTG{𡣾}{44424}
\saveTG{𫊘}{44424}
\saveTG{𡤂}{44427}
\saveTG{𡛙}{44427}
\saveTG{𦺜}{44427}
\saveTG{𦸱}{44427}
\saveTG{𫉥}{44427}
\saveTG{𪍳}{44427}
\saveTG{𡝻}{44427}
\saveTG{𡞲}{44427}
\saveTG{𧃨}{44427}
\saveTG{𡠹}{44427}
\saveTG{㛿}{44427}
\saveTG{𦷾}{44427}
\saveTG{𠣅}{44427}
\saveTG{𦲿}{44427}
\saveTG{𪌹}{44427}
\saveTG{𦳃}{44427}
\saveTG{𦶦}{44427}
\saveTG{𫇵}{44427}
\saveTG{𦻚}{44427}
\saveTG{𦹐}{44427}
\saveTG{䓖}{44427}
\saveTG{孏}{44427}
\saveTG{劳}{44427}
\saveTG{荔}{44427}
\saveTG{妠}{44427}
\saveTG{婻}{44427}
\saveTG{莮}{44427}
\saveTG{藭}{44427}
\saveTG{𧅹}{44427}
\saveTG{蒴}{44427}
\saveTG{姷}{44427}
\saveTG{}{44427}
\saveTG{𩼥}{44427}
\saveTG{𧂲}{44427}
\saveTG{𡤄}{44427}
\saveTG{𡤃}{44427}
\saveTG{㛓}{44427}
\saveTG{㚴}{44427}
\saveTG{𡢼}{44427}
\saveTG{𡝅}{44427}
\saveTG{𡡣}{44427}
\saveTG{𡟱}{44427}
\saveTG{𡠪}{44427}
\saveTG{𪌅}{44427}
\saveTG{𪋾}{44427}
\saveTG{𫇭}{44427}
\saveTG{𠡭}{44427}
\saveTG{𫇬}{44427}
\saveTG{𦰰}{44427}
\saveTG{𦷵}{44427}
\saveTG{𧃈}{44427}
\saveTG{𦵳}{44427}
\saveTG{𦬋}{44427}
\saveTG{䕮}{44427}
\saveTG{𦺓}{44427}
\saveTG{𧃓}{44427}
\saveTG{𦺹}{44427}
\saveTG{𪦀}{44427}
\saveTG{艻}{44427}
\saveTG{姱}{44427}
\saveTG{媯}{44427}
\saveTG{妫}{44427}
\saveTG{嫷}{44427}
\saveTG{媠}{44427}
\saveTG{茐}{44427}
\saveTG{蒭}{44427}
\saveTG{葧}{44427}
\saveTG{募}{44427}
\saveTG{勃}{44427}
\saveTG{𠢵}{44427}
\saveTG{𡝱}{44427}
\saveTG{𦾘}{44427}
\saveTG{𪍬}{44429}
\saveTG{𪍋}{44429}
\saveTG{𦯷}{44429}
\saveTG{𪦇}{44430}
\saveTG{𡣋}{44430}
\saveTG{娡}{44431}
\saveTG{嬿}{44431}
\saveTG{𡢫}{44431}
\saveTG{㜇}{44431}
\saveTG{𡛠}{44431}
\saveTG{𡤦}{44431}
\saveTG{𧂺}{44431}
\saveTG{𦵢}{44432}
\saveTG{𧅭}{44432}
\saveTG{𦴣}{44432}
\saveTG{𧁾}{44432}
\saveTG{菰}{44432}
\saveTG{麮}{44432}
\saveTG{媴}{44432}
\saveTG{𡟀}{44432}
\saveTG{𦷄}{44432}
\saveTG{䵆}{44432}
\saveTG{𪍎}{44433}
\saveTG{𪌉}{44433}
\saveTG{𦯂}{44434}
\saveTG{𦭊}{44434}
\saveTG{𡣻}{44435}
\saveTG{𡠻}{44436}
\saveTG{𧃿}{44436}
\saveTG{𡝳}{44437}
\saveTG{𡡌}{44438}
\saveTG{𡜘}{44440}
\saveTG{𡟌}{44440}
\saveTG{𡟘}{44440}
\saveTG{𦬇}{44440}
\saveTG{𦬠}{44440}
\saveTG{𨌸}{44440}
\saveTG{㸚}{44440}
\saveTG{𦬢}{44440}
\saveTG{𦬱}{44440}
\saveTG{𪱶}{44440}
\saveTG{奻}{44440}
\saveTG{嫐}{44440}
\saveTG{𡝋}{44440}
\saveTG{𡝈}{44440}
\saveTG{𦻥}{44441}
\saveTG{𦸟}{44441}
\saveTG{𡝄}{44441}
\saveTG{䓸}{44441}
\saveTG{𦶯}{44441}
\saveTG{𦴏}{44441}
\saveTG{葬}{44441}
\saveTG{婞}{44441}
\saveTG{媶}{44441}
\saveTG{茾}{44441}
\saveTG{荓}{44441}
\saveTG{嫴}{44441}
\saveTG{嬦}{44441}
\saveTG{𡤏}{44441}
\saveTG{𨑀}{44441}
\saveTG{𦼮}{44441}
\saveTG{𦯸}{44441}
\saveTG{𦰃}{44441}
\saveTG{𦻧}{44441}
\saveTG{𦿠}{44441}
\saveTG{𧂴}{44441}
\saveTG{}{44441}
\saveTG{𦱭}{44441}
\saveTG{𫈴}{44441}
\saveTG{𦽬}{44441}
\saveTG{𦻠}{44441}
\saveTG{𦮁}{44441}
\saveTG{𦴮}{44441}
\saveTG{𦿥}{44441}
\saveTG{𦰙}{44441}
\saveTG{𧂑}{44442}
\saveTG{𦳴}{44442}
\saveTG{𦲇}{44442}
\saveTG{𦹈}{44442}
\saveTG{𡡷}{44442}
\saveTG{𦻦}{44442}
\saveTG{𦿣}{44442}
\saveTG{𤿓}{44442}
\saveTG{𦻢}{44442}
\saveTG{𦯛}{44442}
\saveTG{𦭂}{44442}
\saveTG{𦯳}{44442}
\saveTG{𡝔}{44442}
\saveTG{𧆚}{44442}
\saveTG{𦮘}{44442}
\saveTG{𦴳}{44442}
\saveTG{𡛈}{44442}
\saveTG{𦿢}{44442}
\saveTG{𡝿}{44442}
\saveTG{𦹉}{44442}
\saveTG{𦶭}{44442}
\saveTG{𦿪}{44443}
\saveTG{𪎄}{44443}
\saveTG{𦻨}{44443}
\saveTG{𦸳}{44443}
\saveTG{𧁘}{44443}
\saveTG{䒪}{44443}
\saveTG{薅}{44443}
\saveTG{𦰠}{44443}
\saveTG{𧁗}{44443}
\saveTG{𦲱}{44444}
\saveTG{𧁻}{44444}
\saveTG{莾}{44444}
\saveTG{𦿔}{44444}
\saveTG{𦺒}{44444}
\saveTG{𧆁}{44444}
\saveTG{𦳕}{44444}
\saveTG{𢌽}{44444}
\saveTG{𦻡}{44444}
\saveTG{𦭺}{44444}
\saveTG{𡢯}{44444}
\saveTG{葌}{44444}
\saveTG{𦯨}{44444}
\saveTG{𧁕}{44444}
\saveTG{𦶪}{44444}
\saveTG{𡚹}{44445}
\saveTG{𦻪}{44445}
\saveTG{𦽷}{44445}
\saveTG{𧃻}{44446}
\saveTG{𦻟}{44446}
\saveTG{萛}{44446}
\saveTG{葊}{44446}
\saveTG{𦿤}{44446}
\saveTG{𧂇}{44446}
\saveTG{𦷼}{44446}
\saveTG{𧂵}{44446}
\saveTG{𦿩}{44446}
\saveTG{𦱤}{44446}
\saveTG{𦽹}{44446}
\saveTG{𦴰}{44446}
\saveTG{𦹊}{44446}
\saveTG{𦱓}{44446}
\saveTG{𦶳}{44446}
\saveTG{𧅞}{44447}
\saveTG{𦷱}{44447}
\saveTG{䕺}{44447}
\saveTG{𦴎}{44447}
\saveTG{𫇶}{44447}
\saveTG{𪌰}{44447}
\saveTG{嬳}{44447}
\saveTG{𡜒}{44447}
\saveTG{㛘}{44447}
\saveTG{𦸹}{44447}
\saveTG{𦸤}{44447}
\saveTG{𦭸}{44447}
\saveTG{𫉨}{44447}
\saveTG{㿴}{44447}
\saveTG{𦴬}{44447}
\saveTG{𦧄}{44447}
\saveTG{𤿹}{44447}
\saveTG{𡛡}{44447}
\saveTG{䕅}{44447}
\saveTG{𦲔}{44447}
\saveTG{𦱱}{44447}
\saveTG{𢺽}{44447}
\saveTG{𦻉}{44447}
\saveTG{䒵}{44447}
\saveTG{𦶞}{44447}
\saveTG{𦺋}{44447}
\saveTG{蕔}{44447}
\saveTG{菆}{44447}
\saveTG{樷}{44447}
\saveTG{葮}{44447}
\saveTG{䒟}{44447}
\saveTG{妓}{44447}
\saveTG{蕞}{44447}
\saveTG{婈}{44447}
\saveTG{蒰}{44447}
\saveTG{苒}{44447}
\saveTG{𢍌}{44447}
\saveTG{麬}{44447}
\saveTG{𦼎}{44447}
\saveTG{𦬻}{44447}
\saveTG{𢺸}{44447}
\saveTG{𢻏}{44447}
\saveTG{𡛀}{44447}
\saveTG{𦵷}{44447}
\saveTG{𦵝}{44448}
\saveTG{𧀒}{44448}
\saveTG{莽}{44448}
\saveTG{䔻}{44448}
\saveTG{𦲈}{44448}
\saveTG{𦾡}{44448}
\saveTG{𦽸}{44448}
\saveTG{𦵍}{44448}
\saveTG{𧂱}{44448}
\saveTG{𧅎}{44448}
\saveTG{𦸦}{44448}
\saveTG{薮}{44448}
\saveTG{藪}{44448}
\saveTG{𧃺}{44449}
\saveTG{𦻣}{44449}
\saveTG{𡞴}{44449}
\saveTG{𧅏}{44449}
\saveTG{𦽻}{44449}
\saveTG{𧄫}{44449}
\saveTG{𡡴}{44449}
\saveTG{𧄩}{44449}
\saveTG{𦹍}{44449}
\saveTG{𦻝}{44453}
\saveTG{𦶥}{44453}
\saveTG{蕺}{44453}
\saveTG{𪦙}{44453}
\saveTG{𦺩}{44453}
\saveTG{蘵}{44453}
\saveTG{茙}{44453}
\saveTG{嬅}{44454}
\saveTG{𧂽}{44454}
\saveTG{𦲝}{44454}
\saveTG{𡟍}{44456}
\saveTG{𧃙}{44456}
\saveTG{韓}{44456}
\saveTG{媁}{44456}
\saveTG{䔜}{44457}
\saveTG{㯬}{44457}
\saveTG{𦰓}{44457}
\saveTG{𦶈}{44457}
\saveTG{媎}{44460}
\saveTG{𦹵}{44460}
\saveTG{媌}{44460}
\saveTG{𡛮}{44460}
\saveTG{姑}{44460}
\saveTG{茄}{44460}
\saveTG{茹}{44460}
\saveTG{𦲣}{44460}
\saveTG{姞}{44461}
\saveTG{嫱}{44461}
\saveTG{㜓}{44461}
\saveTG{嬙}{44461}
\saveTG{嬉}{44461}
\saveTG{𪌧}{44461}
\saveTG{𡤑}{44461}
\saveTG{㛭}{44461}
\saveTG{𡜲}{44461}
\saveTG{㜴}{44461}
\saveTG{䕒}{44461}
\saveTG{𡣗}{44461}
\saveTG{𡣰}{44461}
\saveTG{𧅓}{44462}
\saveTG{𦵎}{44462}
\saveTG{𦲼}{44462}
\saveTG{𦰯}{44463}
\saveTG{㛈}{44464}
\saveTG{𡞯}{44464}
\saveTG{婼}{44464}
\saveTG{菇}{44464}
\saveTG{𪦜}{44464}
\saveTG{𡜺}{44469}
\saveTG{𦕐}{44470}
\saveTG{姏}{44470}
\saveTG{𦷜}{44476}
\saveTG{𦰁}{44477}
\saveTG{𦹰}{44478}
\saveTG{𡚻}{44480}
\saveTG{𦶗}{44480}
\saveTG{嫃}{44481}
\saveTG{娸}{44481}
\saveTG{𪍀}{44481}
\saveTG{娂}{44481}
\saveTG{𡢟}{44482}
\saveTG{𦼙}{44482}
\saveTG{𡞁}{44482}
\saveTG{𡛕}{44483}
\saveTG{䔌}{44484}
\saveTG{𪍤}{44484}
\saveTG{嫫}{44484}
\saveTG{𡝩}{44484}
\saveTG{𪥿}{44484}
\saveTG{𧄊}{44485}
\saveTG{嫨}{44485}
\saveTG{媖}{44485}
\saveTG{㜺}{44486}
\saveTG{嬻}{44486}
\saveTG{𧆐}{44486}
\saveTG{嫹}{44486}
\saveTG{𦽀}{44486}
\saveTG{䵃}{44486}
\saveTG{㛍}{44488}
\saveTG{𡢴}{44489}
\saveTG{𧂞}{44489}
\saveTG{𦵸}{44489}
\saveTG{𤒒}{44489}
\saveTG{𡜨}{44490}
\saveTG{㛦}{44490}
\saveTG{𪌎}{44490}
\saveTG{荪}{44490}
\saveTG{𡞫}{44491}
\saveTG{𡣮}{44491}
\saveTG{𡢾}{44491}
\saveTG{𡞏}{44491}
\saveTG{𪍟}{44493}
\saveTG{蓀}{44493}
\saveTG{媒}{44494}
\saveTG{𧁀}{44494}
\saveTG{𦾮}{44494}
\saveTG{媣}{44494}
\saveTG{𡤤}{44494}
\saveTG{𡜉}{44494}
\saveTG{𡢬}{44494}
\saveTG{𪍐}{44494}
\saveTG{𧃹}{44494}
\saveTG{𦰕}{44494}
\saveTG{媟}{44494}
\saveTG{𦾇}{44495}
\saveTG{𦳀}{44495}
\saveTG{嫽}{44496}
\saveTG{𦽆}{44497}
\saveTG{麳}{44498}
\saveTG{婡}{44498}
\saveTG{𪎂}{44498}
\saveTG{𩋰}{44500}
\saveTG{䒜}{44500}
\saveTG{䒠}{44500}
\saveTG{𦬒}{44500}
\saveTG{犎}{44500}
\saveTG{𦬤}{44500}
\saveTG{𦭵}{44501}
\saveTG{𪔙}{44501}
\saveTG{䔣}{44502}
\saveTG{𦰤}{44502}
\saveTG{𦺙}{44502}
\saveTG{𤙼}{44502}
\saveTG{𢮜}{44502}
\saveTG{𧅩}{44502}
\saveTG{𦶸}{44502}
\saveTG{䖂}{44502}
\saveTG{葷}{44502}
\saveTG{藆}{44502}
\saveTG{荦}{44502}
\saveTG{摹}{44502}
\saveTG{蒘}{44502}
\saveTG{攀}{44502}
\saveTG{㨍}{44502}
\saveTG{拲}{44502}
\saveTG{𥗣}{44502}
\saveTG{𦺡}{44502}
\saveTG{𩉠}{44503}
\saveTG{𦭷}{44503}
\saveTG{𦸘}{44503}
\saveTG{𩎠}{44503}
\saveTG{菚}{44503}
\saveTG{華}{44504}
\saveTG{𦮔}{44504}
\saveTG{𪺶}{44504}
\saveTG{蓽}{44504}
\saveTG{莑}{44504}
\saveTG{荤}{44504}
\saveTG{𦮧}{44504}
\saveTG{𪔍}{44504}
\saveTG{𪔞}{44504}
\saveTG{𩊈}{44506}
\saveTG{莗}{44506}
\saveTG{蕇}{44506}
\saveTG{革}{44506}
\saveTG{莄}{44506}
\saveTG{輂}{44506}
\saveTG{葦}{44506}
\saveTG{䒶}{44506}
\saveTG{𦬕}{44506}
\saveTG{𪔨}{44506}
\saveTG{𫈹}{44506}
\saveTG{𦭖}{44506}
\saveTG{鞤}{44506}
\saveTG{茰}{44506}
\saveTG{𦾥}{44506}
\saveTG{𦳱}{44506}
\saveTG{𦱊}{44507}
\saveTG{𪔔}{44507}
\saveTG{𦸩}{44507}
\saveTG{茟}{44507}
\saveTG{芛}{44507}
\saveTG{𦮴}{44508}
\saveTG{菶}{44508}
\saveTG{蕐}{44508}
\saveTG{靯}{44510}
\saveTG{靴}{44510}
\saveTG{𩋭}{44512}
\saveTG{𩌍}{44512}
\saveTG{菢}{44512}
\saveTG{𫈣}{44512}
\saveTG{𩌿}{44512}
\saveTG{𩋪}{44512}
\saveTG{𦽞}{44512}
\saveTG{𩊲}{44512}
\saveTG{𦾃}{44512}
\saveTG{𦻲}{44512}
\saveTG{𦰖}{44512}
\saveTG{𧄧}{44512}
\saveTG{𩍰}{44512}
\saveTG{蒐}{44513}
\saveTG{莵}{44513}
\saveTG{鞋}{44514}
\saveTG{𦴽}{44514}
\saveTG{𩋔}{44514}
\saveTG{𩋖}{44514}
\saveTG{𧃔}{44515}
\saveTG{𦾣}{44515}
\saveTG{𩋊}{44515}
\saveTG{蓷}{44515}
\saveTG{𦶂}{44517}
\saveTG{轨}{44517}
\saveTG{芄}{44517}
\saveTG{靾}{44517}
\saveTG{𦳛}{44517}
\saveTG{䒯}{44517}
\saveTG{𩋕}{44517}
\saveTG{菈}{44518}
\saveTG{䓆}{44521}
\saveTG{龩}{44521}
\saveTG{蔪}{44521}
\saveTG{蘄}{44521}
\saveTG{蕲}{44521}
\saveTG{䩭}{44521}
\saveTG{勒}{44527}
\saveTG{韊}{44527}
\saveTG{蘜}{44527}
\saveTG{韉}{44527}
\saveTG{茀}{44527}
\saveTG{韛}{44527}
\saveTG{鞴}{44527}
\saveTG{𦰡}{44527}
\saveTG{𦰥}{44527}
\saveTG{𦲫}{44527}
\saveTG{𩍘}{44527}
\saveTG{蒱}{44527}
\saveTG{𫖉}{44527}
\saveTG{𩌴}{44527}
\saveTG{𩊟}{44527}
\saveTG{𩊽}{44527}
\saveTG{𧄛}{44527}
\saveTG{𩍛}{44527}
\saveTG{𩎐}{44527}
\saveTG{𦳻}{44527}
\saveTG{䕤}{44527}
\saveTG{靿}{44527}
\saveTG{苇}{44527}
\saveTG{靹}{44527}
\saveTG{𩌚}{44527}
\saveTG{𠢽}{44527}
\saveTG{𦹩}{44527}
\saveTG{䔾}{44527}
\saveTG{𦮾}{44527}
\saveTG{𦼸}{44527}
\saveTG{𪟥}{44527}
\saveTG{𩏕}{44527}
\saveTG{䪏}{44527}
\saveTG{𩍁}{44527}
\saveTG{𩎁}{44527}
\saveTG{𩊓}{44527}
\saveTG{𦿄}{44527}
\saveTG{𦰄}{44527}
\saveTG{𧅘}{44527}
\saveTG{莏}{44529}
\saveTG{𦬗}{44530}
\saveTG{𩊴}{44531}
\saveTG{𧁵}{44531}
\saveTG{𩉦}{44532}
\saveTG{辕}{44532}
\saveTG{𩍬}{44532}
\saveTG{韃}{44535}
\saveTG{䩶}{44536}
\saveTG{鞑}{44538}
\saveTG{𩎔}{44540}
\saveTG{𩉤}{44540}
\saveTG{蘀}{44541}
\saveTG{蓒}{44541}
\saveTG{䩸}{44541}
\saveTG{䔱}{44541}
\saveTG{𦯙}{44541}
\saveTG{𫖇}{44541}
\saveTG{䪇}{44542}
\saveTG{𩍿}{44542}
\saveTG{𩏵}{44543}
\saveTG{𩏯}{44543}
\saveTG{𦰽}{44544}
\saveTG{𦵭}{44544}
\saveTG{𧁟}{44544}
\saveTG{𩋐}{44547}
\saveTG{𩏃}{44547}
\saveTG{𥀀}{44547}
\saveTG{鞁}{44547}
\saveTG{𦼽}{44547}
\saveTG{韄}{44547}
\saveTG{鞯}{44547}
\saveTG{𩉨}{44547}
\saveTG{𩌅}{44547}
\saveTG{𧁍}{44547}
\saveTG{𨐉}{44547}
\saveTG{菝}{44547}
\saveTG{𦱒}{44548}
\saveTG{𦾙}{44548}
\saveTG{𦽨}{44548}
\saveTG{𦳞}{44551}
\saveTG{𧂁}{44553}
\saveTG{䕏}{44553}
\saveTG{莪}{44553}
\saveTG{韈}{44553}
\saveTG{韤}{44553}
\saveTG{𦾒}{44553}
\saveTG{鞾}{44554}
\saveTG{韡}{44554}
\saveTG{萚}{44554}
\saveTG{𩏧}{44556}
\saveTG{𦮉}{44556}
\saveTG{𩋾}{44556}
\saveTG{韚}{44556}
\saveTG{䒣}{44557}
\saveTG{𦮙}{44557}
\saveTG{𩊉}{44560}
\saveTG{轱}{44560}
\saveTG{𩊿}{44561}
\saveTG{鞳}{44561}
\saveTG{鞊}{44561}
\saveTG{𩍙}{44561}
\saveTG{𩏴}{44561}
\saveTG{𩏫}{44561}
\saveTG{𩏒}{44561}
\saveTG{𦾬}{44562}
\saveTG{𦼾}{44562}
\saveTG{𩌓}{44562}
\saveTG{萔}{44562}
\saveTG{𩌂}{44563}
\saveTG{𩌤}{44564}
\saveTG{𩌀}{44564}
\saveTG{葀}{44564}
\saveTG{菗}{44565}
\saveTG{𩋵}{44566}
\saveTG{𩊵}{44568}
\saveTG{𦵛}{44572}
\saveTG{轪}{44580}
\saveTG{蓵}{44581}
\saveTG{䔶}{44582}
\saveTG{𦻴}{44582}
\saveTG{𩌙}{44582}
\saveTG{𦺕}{44584}
\saveTG{𩌧}{44584}
\saveTG{𣝴}{44584}
\saveTG{荴}{44585}
\saveTG{𩍾}{44586}
\saveTG{韇}{44586}
\saveTG{韥}{44586}
\saveTG{䪄}{44586}
\saveTG{䩿}{44586}
\saveTG{𩎈}{44586}
\saveTG{𪏥}{44586}
\saveTG{𩍟}{44586}
\saveTG{𩎱}{44588}
\saveTG{䩡}{44588}
\saveTG{𦰂}{44589}
\saveTG{䕭}{44589}
\saveTG{䪂}{44592}
\saveTG{𩎎}{44592}
\saveTG{𩌈}{44593}
\saveTG{韘}{44594}
\saveTG{鞢}{44594}
\saveTG{𦽥}{44594}
\saveTG{𩍣}{44594}
\saveTG{𦺤}{44594}
\saveTG{𦾱}{44594}
\saveTG{𧂈}{44594}
\saveTG{𩋞}{44594}
\saveTG{𦷀}{44594}
\saveTG{𩏖}{44595}
\saveTG{𩍠}{44595}
\saveTG{𫈽}{44600}
\saveTG{𫈁}{44600}
\saveTG{蔮}{44600}
\saveTG{茴}{44600}
\saveTG{蔨}{44600}
\saveTG{菌}{44600}
\saveTG{莔}{44600}
\saveTG{苗}{44600}
\saveTG{苜}{44600}
\saveTG{苬}{44600}
\saveTG{茵}{44600}
\saveTG{薗}{44600}
\saveTG{者}{44600}
\saveTG{𦯁}{44600}
\saveTG{䒤}{44600}
\saveTG{𦭁}{44600}
\saveTG{𦳩}{44600}
\saveTG{䓢}{44600}
\saveTG{𦵣}{44600}
\saveTG{𦭝}{44600}
\saveTG{𦮄}{44600}
\saveTG{𦸠}{44600}
\saveTG{𦳊}{44600}
\saveTG{𦭤}{44600}
\saveTG{𧅲}{44600}
\saveTG{𧆘}{44600}
\saveTG{𦱵}{44600}
\saveTG{𧆕}{44600}
\saveTG{𦭉}{44600}
\saveTG{𦮗}{44600}
\saveTG{𦳬}{44600}
\saveTG{𦯯}{44600}
\saveTG{𧀇}{44600}
\saveTG{𥓙}{44601}
\saveTG{䶀}{44601}
\saveTG{䓂}{44601}
\saveTG{𦴀}{44601}
\saveTG{𤰻}{44601}
\saveTG{𧨅}{44601}
\saveTG{蓍}{44601}
\saveTG{𣋅}{44601}
\saveTG{𧫃}{44601}
\saveTG{𦱋}{44601}
\saveTG{𦶕}{44601}
\saveTG{𦵰}{44601}
\saveTG{𥈄}{44601}
\saveTG{𠹝}{44601}
\saveTG{𪔧}{44601}
\saveTG{萻}{44601}
\saveTG{菩}{44601}
\saveTG{苫}{44601}
\saveTG{萅}{44601}
\saveTG{荅}{44601}
\saveTG{諅}{44601}
\saveTG{蕾}{44601}
\saveTG{蓾}{44601}
\saveTG{謩}{44601}
\saveTG{耆}{44601}
\saveTG{蔷}{44601}
\saveTG{薔}{44601}
\saveTG{昔}{44601}
\saveTG{𫉫}{44601}
\saveTG{𦸜}{44601}
\saveTG{𦮽}{44601}
\saveTG{䓀}{44601}
\saveTG{𦱗}{44601}
\saveTG{𦵻}{44601}
\saveTG{𦯾}{44601}
\saveTG{𦮂}{44601}
\saveTG{𦮑}{44601}
\saveTG{䓊}{44601}
\saveTG{𥖯}{44601}
\saveTG{𥖎}{44601}
\saveTG{𧁝}{44601}
\saveTG{𦸚}{44602}
\saveTG{𦵀}{44602}
\saveTG{𪔑}{44602}
\saveTG{䂲}{44602}
\saveTG{𦷦}{44602}
\saveTG{𦒻}{44602}
\saveTG{𦯎}{44602}
\saveTG{𦬲}{44602}
\saveTG{𦤅}{44602}
\saveTG{蒥}{44602}
\saveTG{菪}{44602}
\saveTG{𦳗}{44602}
\saveTG{𣜞}{44602}
\saveTG{𪔓}{44602}
\saveTG{𦻳}{44602}
\saveTG{𦷕}{44602}
\saveTG{苩}{44602}
\saveTG{碁}{44602}
\saveTG{茗}{44602}
\saveTG{瞢}{44602}
\saveTG{蒈}{44602}
\saveTG{莟}{44602}
\saveTG{苕}{44602}
\saveTG{礬}{44602}
\saveTG{𦭾}{44602}
\saveTG{𥖫}{44602}
\saveTG{𤽠}{44602}
\saveTG{䵿}{44602}
\saveTG{䓠}{44603}
\saveTG{䓿}{44603}
\saveTG{𧁖}{44603}
\saveTG{葘}{44603}
\saveTG{菑}{44603}
\saveTG{㬫}{44603}
\saveTG{蓄}{44603}
\saveTG{苔}{44603}
\saveTG{𣂃}{44603}
\saveTG{䶁}{44603}
\saveTG{茜}{44604}
\saveTG{𦴦}{44604}
\saveTG{茖}{44604}
\saveTG{𪔗}{44604}
\saveTG{𥌒}{44604}
\saveTG{著}{44604}
\saveTG{瞽}{44604}
\saveTG{鼛}{44604}
\saveTG{苦}{44604}
\saveTG{若}{44604}
\saveTG{薯}{44604}
\saveTG{莤}{44604}
\saveTG{蒏}{44604}
\saveTG{𦴈}{44604}
\saveTG{𧂢}{44604}
\saveTG{𦭮}{44604}
\saveTG{𦴖}{44604}
\saveTG{䓘}{44604}
\saveTG{𦳷}{44604}
\saveTG{𦯚}{44604}
\saveTG{𦾧}{44604}
\saveTG{𪔐}{44604}
\saveTG{𦳜}{44604}
\saveTG{𫈛}{44605}
\saveTG{𦺳}{44605}
\saveTG{𦵯}{44605}
\saveTG{䒼}{44605}
\saveTG{苖}{44605}
\saveTG{营}{44606}
\saveTG{莒}{44606}
\saveTG{𦸥}{44606}
\saveTG{𦼲}{44606}
\saveTG{薈}{44606}
\saveTG{𦴄}{44606}
\saveTG{𦵆}{44606}
\saveTG{䔰}{44606}
\saveTG{𦼏}{44606}
\saveTG{蓸}{44606}
\saveTG{菖}{44606}
\saveTG{葍}{44606}
\saveTG{萺}{44606}
\saveTG{莙}{44607}
\saveTG{𫉚}{44607}
\saveTG{𦻤}{44607}
\saveTG{𧀈}{44607}
\saveTG{𦷌}{44608}
\saveTG{𠻚}{44608}
\saveTG{𦻘}{44608}
\saveTG{𫈅}{44608}
\saveTG{萶}{44608}
\saveTG{暮}{44608}
\saveTG{蓉}{44608}
\saveTG{𤱺}{44608}
\saveTG{𥆇}{44608}
\saveTG{㫷}{44608}
\saveTG{榃}{44609}
\saveTG{萫}{44609}
\saveTG{莕}{44609}
\saveTG{𧆏}{44609}
\saveTG{𠵂}{44609}
\saveTG{𣞖}{44609}
\saveTG{𧀯}{44609}
\saveTG{𧅥}{44609}
\saveTG{䓏}{44609}
\saveTG{𦼞}{44609}
\saveTG{𫈗}{44609}
\saveTG{𧅝}{44609}
\saveTG{蕃}{44609}
\saveTG{𡎉}{44610}
\saveTG{𧩋}{44611}
\saveTG{葃}{44611}
\saveTG{𦯠}{44612}
\saveTG{𧅔}{44612}
\saveTG{𦸿}{44612}
\saveTG{𦳈}{44612}
\saveTG{𦼍}{44612}
\saveTG{𦺊}{44612}
\saveTG{䕢}{44612}
\saveTG{𫊐}{44612}
\saveTG{㗡}{44612}
\saveTG{𦸭}{44612}
\saveTG{𧂯}{44612}
\saveTG{莻}{44612}
\saveTG{蒊}{44612}
\saveTG{𧃝}{44613}
\saveTG{𡅸}{44615}
\saveTG{𡅤}{44615}
\saveTG{𦺑}{44615}
\saveTG{𦺽}{44615}
\saveTG{𧀣}{44615}
\saveTG{蓶}{44615}
\saveTG{𪔘}{44617}
\saveTG{䓽}{44617}
\saveTG{𦲠}{44617}
\saveTG{𫈕}{44617}
\saveTG{葩}{44617}
\saveTG{蓜}{44617}
\saveTG{𫅳}{44617}
\saveTG{𦶰}{44617}
\saveTG{𤲽}{44617}
\saveTG{𦺧}{44620}
\saveTG{𦸕}{44620}
\saveTG{苛}{44621}
\saveTG{𦺗}{44621}
\saveTG{䔅}{44621}
\saveTG{𦳵}{44621}
\saveTG{䓫}{44621}
\saveTG{𦺞}{44621}
\saveTG{𦱕}{44621}
\saveTG{㚡}{44621}
\saveTG{𢒻}{44622}
\saveTG{𦵡}{44626}
\saveTG{藅}{44626}
\saveTG{𦯗}{44627}
\saveTG{𦴆}{44627}
\saveTG{𦲰}{44627}
\saveTG{𦲆}{44627}
\saveTG{𦺥}{44627}
\saveTG{𦷽}{44627}
\saveTG{𦵼}{44627}
\saveTG{𣕽}{44627}
\saveTG{藹}{44627}
\saveTG{蔔}{44627}
\saveTG{耇}{44627}
\saveTG{菂}{44627}
\saveTG{蔀}{44627}
\saveTG{苟}{44627}
\saveTG{𦰶}{44627}
\saveTG{𦺚}{44627}
\saveTG{䓒}{44627}
\saveTG{𦮿}{44627}
\saveTG{𦉚}{44627}
\saveTG{𧁢}{44627}
\saveTG{𠡉}{44627}
\saveTG{𧀄}{44627}
\saveTG{𦰅}{44627}
\saveTG{䕵}{44627}
\saveTG{𫈔}{44627}
\saveTG{𠢷}{44627}
\saveTG{𠢳}{44627}
\saveTG{荶}{44627}
\saveTG{𧃄}{44627}
\saveTG{𦴛}{44627}
\saveTG{𦹣}{44627}
\saveTG{𦿆}{44627}
\saveTG{𦉠}{44627}
\saveTG{劼}{44627}
\saveTG{耉}{44627}
\saveTG{𦶔}{44627}
\saveTG{𦮶}{44627}
\saveTG{𦷹}{44627}
\saveTG{𦭡}{44627}
\saveTG{荀}{44627}
\saveTG{萌}{44627}
\saveTG{蘤}{44627}
\saveTG{葫}{44627}
\saveTG{耈}{44627}
\saveTG{𦵿}{44629}
\saveTG{𦳥}{44629}
\saveTG{蘸}{44631}
\saveTG{𨣂}{44631}
\saveTG{𧹨}{44631}
\saveTG{𦸉}{44632}
\saveTG{𦸈}{44632}
\saveTG{𦺫}{44632}
\saveTG{𦽤}{44632}
\saveTG{䖆}{44632}
\saveTG{𦻑}{44632}
\saveTG{𧀦}{44636}
\saveTG{𦷩}{44637}
\saveTG{莳}{44640}
\saveTG{𦵗}{44640}
\saveTG{𣀘}{44640}
\saveTG{𡜩}{44640}
\saveTG{蒔}{44641}
\saveTG{薵}{44641}
\saveTG{𦮣}{44641}
\saveTG{𦸋}{44641}
\saveTG{𦸎}{44641}
\saveTG{𦰲}{44641}
\saveTG{䒷}{44642}
\saveTG{𣠈}{44643}
\saveTG{𦹯}{44646}
\saveTG{𧅰}{44647}
\saveTG{攲}{44647}
\saveTG{𧄦}{44647}
\saveTG{𥀁}{44647}
\saveTG{𤿠}{44647}
\saveTG{𤿞}{44647}
\saveTG{𧅷}{44647}
\saveTG{䕶}{44647}
\saveTG{𦾹}{44647}
\saveTG{𦿓}{44647}
\saveTG{𦰨}{44647}
\saveTG{皵}{44647}
\saveTG{蘐}{44647}
\saveTG{蔎}{44647}
\saveTG{𦲧}{44648}
\saveTG{䔓}{44648}
\saveTG{𦳦}{44653}
\saveTG{𦺿}{44653}
\saveTG{𧆂}{44653}
\saveTG{𧄹}{44653}
\saveTG{𦾛}{44656}
\saveTG{𦽖}{44660}
\saveTG{𦴒}{44660}
\saveTG{𠳬}{44660}
\saveTG{𡕇}{44661}
\saveTG{囍}{44661}
\saveTG{𣊣}{44661}
\saveTG{𧄤}{44661}
\saveTG{蘦}{44661}
\saveTG{𫊂}{44661}
\saveTG{喆}{44661}
\saveTG{藠}{44662}
\saveTG{藞}{44662}
\saveTG{䔤}{44662}
\saveTG{𧅠}{44663}
\saveTG{䖃}{44664}
\saveTG{𡄝}{44664}
\saveTG{𡅕}{44664}
\saveTG{藷}{44664}
\saveTG{䔯}{44664}
\saveTG{𫊑}{44665}
\saveTG{𫉢}{44665}
\saveTG{䓵}{44666}
\saveTG{蕌}{44666}
\saveTG{藟}{44666}
\saveTG{虈}{44666}
\saveTG{䕎}{44666}
\saveTG{𦿫}{44668}
\saveTG{䖀}{44668}
\saveTG{𧄮}{44668}
\saveTG{𧂉}{44669}
\saveTG{𫈀}{44682}
\saveTG{𫉮}{44682}
\saveTG{𫉸}{44682}
\saveTG{𦿡}{44684}
\saveTG{𦽈}{44684}
\saveTG{𧄲}{44684}
\saveTG{藈}{44684}
\saveTG{䕛}{44684}
\saveTG{𦴤}{44685}
\saveTG{𧀺}{44686}
\saveTG{𧂃}{44686}
\saveTG{𦽉}{44689}
\saveTG{𣈅}{44690}
\saveTG{𧄎}{44693}
\saveTG{蔝}{44694}
\saveTG{菋}{44695}
\saveTG{𧄈}{44696}
\saveTG{𨡸}{44698}
\saveTG{斟}{44700}
\saveTG{卙}{44700}
\saveTG{𦰎}{44707}
\saveTG{芒}{44710}
\saveTG{𦭆}{44710}
\saveTG{𠃒}{44710}
\saveTG{苉}{44711}
\saveTG{𣛍}{44711}
\saveTG{𫈇}{44711}
\saveTG{𧂕}{44711}
\saveTG{𦭠}{44711}
\saveTG{䒰}{44711}
\saveTG{𦷁}{44711}
\saveTG{𦬀}{44711}
\saveTG{苞}{44712}
\saveTG{𢁀}{44712}
\saveTG{𦭥}{44712}
\saveTG{𦸊}{44712}
\saveTG{𧃘}{44712}
\saveTG{䔘}{44712}
\saveTG{𦭯}{44712}
\saveTG{𦬎}{44712}
\saveTG{𦫾}{44712}
\saveTG{𧀨}{44712}
\saveTG{𦶃}{44712}
\saveTG{㽎}{44712}
\saveTG{𧂛}{44712}
\saveTG{𡉵}{44712}
\saveTG{𦱉}{44712}
\saveTG{𦫷}{44712}
\saveTG{蓖}{44712}
\saveTG{𦰮}{44712}
\saveTG{苍}{44712}
\saveTG{荖}{44712}
\saveTG{茝}{44712}
\saveTG{茞}{44712}
\saveTG{芲}{44712}
\saveTG{蔇}{44712}
\saveTG{菤}{44712}
\saveTG{菎}{44712}
\saveTG{老}{44712}
\saveTG{也}{44712}
\saveTG{芘}{44712}
\saveTG{萞}{44712}
\saveTG{𦬃}{44714}
\saveTG{𦽡}{44714}
\saveTG{耄}{44714}
\saveTG{芼}{44714}
\saveTG{𠀍}{44714}
\saveTG{𦭧}{44714}
\saveTG{𦯮}{44715}
\saveTG{𦿗}{44715}
\saveTG{𦹾}{44715}
\saveTG{𣯕}{44715}
\saveTG{𣰺}{44715}
\saveTG{𪵝}{44715}
\saveTG{𤲅}{44715}
\saveTG{𦹕}{44715}
\saveTG{䓐}{44715}
\saveTG{𦼵}{44715}
\saveTG{𦵵}{44716}
\saveTG{蓲}{44716}
\saveTG{𦼤}{44716}
\saveTG{菴}{44716}
\saveTG{𦭓}{44717}
\saveTG{𡯤}{44717}
\saveTG{𦓋}{44717}
\saveTG{䎜}{44717}
\saveTG{𦭈}{44717}
\saveTG{𤭦}{44717}
\saveTG{𠃟}{44717}
\saveTG{𢀼}{44717}
\saveTG{艺}{44717}
\saveTG{世}{44717}
\saveTG{芑}{44717}
\saveTG{甍}{44717}
\saveTG{苣}{44717}
\saveTG{巷}{44717}
\saveTG{芚}{44717}
\saveTG{乽}{44717}
\saveTG{虌}{44717}
\saveTG{芭}{44717}
\saveTG{䒗}{44717}
\saveTG{㐞}{44717}
\saveTG{䢽}{44717}
\saveTG{𦯭}{44717}
\saveTG{𦬊}{44717}
\saveTG{䒻}{44717}
\saveTG{䓃}{44717}
\saveTG{𢁂}{44717}
\saveTG{𦮹}{44717}
\saveTG{葚}{44718}
\saveTG{䕚}{44718}
\saveTG{𨟄}{44718}
\saveTG{甚}{44718}
\saveTG{𦬆}{44718}
\saveTG{𦭣}{44720}
\saveTG{𦫶}{44720}
\saveTG{鬰}{44722}
\saveTG{鬱}{44722}
\saveTG{䖇}{44722}
\saveTG{蘛}{44727}
\saveTG{茚}{44727}
\saveTG{苭}{44727}
\saveTG{萄}{44727}
\saveTG{劫}{44727}
\saveTG{𦭪}{44727}
\saveTG{𦮍}{44727}
\saveTG{䒢}{44727}
\saveTG{𦿋}{44727}
\saveTG{蒒}{44727}
\saveTG{勘}{44727}
\saveTG{葛}{44727}
\saveTG{蔼}{44727}
\saveTG{𧆅}{44727}
\saveTG{𦿃}{44727}
\saveTG{𣠫}{44727}
\saveTG{𣡨}{44727}
\saveTG{𦮦}{44727}
\saveTG{𢐱}{44727}
\saveTG{𫇯}{44727}
\saveTG{𦴾}{44727}
\saveTG{𦷆}{44727}
\saveTG{𦲄}{44727}
\saveTG{𫇷}{44727}
\saveTG{芶}{44727}
\saveTG{𦴫}{44727}
\saveTG{𦹛}{44727}
\saveTG{𦾓}{44727}
\saveTG{𦻏}{44727}
\saveTG{䔢}{44727}
\saveTG{莭}{44727}
\saveTG{𦭃}{44727}
\saveTG{𦬛}{44727}
\saveTG{𦏡}{44727}
\saveTG{𧃠}{44727}
\saveTG{䒛}{44727}
\saveTG{𦰝}{44727}
\saveTG{𦺄}{44727}
\saveTG{蓈}{44727}
\saveTG{蓢}{44727}
\saveTG{茆}{44727}
\saveTG{苆}{44727}
\saveTG{𦭙}{44727}
\saveTG{𠬑}{44731}
\saveTG{𦾆}{44731}
\saveTG{𪲡}{44731}
\saveTG{䵽}{44731}
\saveTG{𪔎}{44731}
\saveTG{䒧}{44731}
\saveTG{𣓕}{44731}
\saveTG{𦮚}{44731}
\saveTG{蕓}{44731}
\saveTG{𦱚}{44731}
\saveTG{𦽲}{44732}
\saveTG{𧃳}{44732}
\saveTG{𧃥}{44732}
\saveTG{𦶧}{44732}
\saveTG{䒾}{44732}
\saveTG{𦬾}{44732}
\saveTG{𧛜}{44732}
\saveTG{𦻫}{44732}
\saveTG{𧞉}{44732}
\saveTG{䙪}{44732}
\saveTG{𧛀}{44732}
\saveTG{𧆔}{44732}
\saveTG{𧅺}{44732}
\saveTG{𧁁}{44732}
\saveTG{𧄂}{44732}
\saveTG{𫈒}{44732}
\saveTG{𦬜}{44732}
\saveTG{𫇽}{44732}
\saveTG{𫈶}{44732}
\saveTG{𫈮}{44732}
\saveTG{𫗍}{44732}
\saveTG{𦿉}{44732}
\saveTG{𧃊}{44732}
\saveTG{䓹}{44732}
\saveTG{𧁙}{44732}
\saveTG{𩟚}{44732}
\saveTG{蓑}{44732}
\saveTG{蘘}{44732}
\saveTG{莨}{44732}
\saveTG{荟}{44732}
\saveTG{蔉}{44732}
\saveTG{蓘}{44732}
\saveTG{茛}{44732}
\saveTG{茲}{44732}
\saveTG{鼚}{44732}
\saveTG{萇}{44732}
\saveTG{藵}{44732}
\saveTG{𣁩}{44732}
\saveTG{藝}{44732}
\saveTG{芸}{44732}
\saveTG{葨}{44732}
\saveTG{𫈍}{44733}
\saveTG{苌}{44734}
\saveTG{𦬘}{44738}
\saveTG{𦬥}{44740}
\saveTG{𪴦}{44740}
\saveTG{𧀼}{44741}
\saveTG{𧃎}{44741}
\saveTG{薛}{44741}
\saveTG{芪}{44742}
\saveTG{茋}{44742}
\saveTG{𦉤}{44743}
\saveTG{𣠵}{44743}
\saveTG{𣡡}{44743}
\saveTG{𣝪}{44743}
\saveTG{𩰩}{44743}
\saveTG{𣡱}{44743}
\saveTG{𦷠}{44744}
\saveTG{𧂘}{44744}
\saveTG{𧅍}{44744}
\saveTG{欎}{44746}
\saveTG{欝}{44746}
\saveTG{𦷊}{44747}
\saveTG{𥀥}{44747}
\saveTG{𤿜}{44747}
\saveTG{𫈿}{44747}
\saveTG{𡕮}{44747}
\saveTG{菣}{44747}
\saveTG{苠}{44747}
\saveTG{𢺿}{44747}
\saveTG{荍}{44748}
\saveTG{𦰦}{44748}
\saveTG{𫉏}{44753}
\saveTG{𫇮}{44753}
\saveTG{𣡜}{44755}
\saveTG{𫖐}{44756}
\saveTG{䓯}{44757}
\saveTG{𦱞}{44757}
\saveTG{莓}{44757}
\saveTG{苺}{44757}
\saveTG{蕼}{44757}
\saveTG{𤯈}{44763}
\saveTG{𫅶}{44764}
\saveTG{𫉄}{44764}
\saveTG{廿}{44770}
\saveTG{甘}{44770}
\saveTG{𦴻}{44770}
\saveTG{𠀠}{44770}
\saveTG{𠦜}{44770}
\saveTG{丗}{44770}
\saveTG{𡿉}{44772}
\saveTG{𦷤}{44772}
\saveTG{𦾝}{44772}
\saveTG{𡷺}{44772}
\saveTG{菡}{44772}
\saveTG{蔤}{44772}
\saveTG{茁}{44772}
\saveTG{𦱳}{44772}
\saveTG{𪙐}{44772}
\saveTG{𧀤}{44772}
\saveTG{𦮒}{44772}
\saveTG{𦰷}{44772}
\saveTG{𪔕}{44772}
\saveTG{㟚}{44772}
\saveTG{𡽌}{44772}
\saveTG{䔄}{44774}
\saveTG{䓨}{44774}
\saveTG{苷}{44774}
\saveTG{𦬵}{44774}
\saveTG{苢}{44777}
\saveTG{蓞}{44777}
\saveTG{舊}{44777}
\saveTG{菅}{44777}
\saveTG{萏}{44777}
\saveTG{𦽜}{44777}
\saveTG{𦽭}{44777}
\saveTG{𠁤}{44777}
\saveTG{𦱸}{44777}
\saveTG{𦭻}{44777}
\saveTG{𪲵}{44779}
\saveTG{苡}{44780}
\saveTG{𦭗}{44780}
\saveTG{𦱽}{44780}
\saveTG{蕻}{44781}
\saveTG{𦻁}{44782}
\saveTG{蒛}{44785}
\saveTG{𦼿}{44786}
\saveTG{䔛}{44786}
\saveTG{𦾴}{44793}
\saveTG{蘨}{44793}
\saveTG{𧃋}{44794}
\saveTG{𦓅}{44794}
\saveTG{𧀳}{44799}
\saveTG{㪸}{44800}
\saveTG{𦬅}{44800}
\saveTG{𦬩}{44800}
\saveTG{䒔}{44800}
\saveTG{𦫸}{44800}
\saveTG{𣂏}{44800}
\saveTG{㪺}{44800}
\saveTG{𦴠}{44800}
\saveTG{赵}{44800}
\saveTG{蓂}{44800}
\saveTG{斢}{44800}
\saveTG{𪏡}{44800}
\saveTG{﨣}{44800}
\saveTG{𧺑}{44801}
\saveTG{𧺔}{44801}
\saveTG{𧼎}{44801}
\saveTG{𧻩}{44801}
\saveTG{𧺟}{44801}
\saveTG{𧺎}{44801}
\saveTG{𪤱}{44801}
\saveTG{楚}{44801}
\saveTG{萣}{44801}
\saveTG{躉}{44801}
\saveTG{共}{44801}
\saveTG{其}{44801}
\saveTG{萁}{44801}
\saveTG{趬}{44801}
\saveTG{躠}{44801}
\saveTG{萐}{44801}
\saveTG{藇}{44801}
\saveTG{蒖}{44801}
\saveTG{𠔝}{44801}
\saveTG{𠔏}{44801}
\saveTG{𦸬}{44801}
\saveTG{䟒}{44801}
\saveTG{䞨}{44801}
\saveTG{𧃞}{44801}
\saveTG{莡}{44801}
\saveTG{𧺏}{44801}
\saveTG{䓦}{44801}
\saveTG{䕟}{44801}
\saveTG{𦳪}{44801}
\saveTG{𦾲}{44801}
\saveTG{𧾲}{44801}
\saveTG{𦮎}{44801}
\saveTG{䔬}{44801}
\saveTG{𧺿}{44801}
\saveTG{𦸆}{44801}
\saveTG{𧃟}{44801}
\saveTG{𦺈}{44801}
\saveTG{𧀶}{44802}
\saveTG{𧽭}{44802}
\saveTG{𧼘}{44802}
\saveTG{𦵉}{44802}
\saveTG{𧁏}{44802}
\saveTG{𧄌}{44802}
\saveTG{𦳚}{44802}
\saveTG{𦽢}{44802}
\saveTG{䞗}{44802}
\saveTG{𧾯}{44802}
\saveTG{茓}{44802}
\saveTG{荑}{44802}
\saveTG{贳}{44802}
\saveTG{芡}{44802}
\saveTG{赲}{44802}
\saveTG{蒉}{44802}
\saveTG{荄}{44802}
\saveTG{𪔭}{44802}
\saveTG{𧼦}{44802}
\saveTG{𧺬}{44802}
\saveTG{𧻣}{44802}
\saveTG{𧻒}{44802}
\saveTG{𧾗}{44802}
\saveTG{𧽬}{44802}
\saveTG{䞥}{44802}
\saveTG{𧾆}{44802}
\saveTG{𧽹}{44802}
\saveTG{𨆊}{44802}
\saveTG{𧼡}{44802}
\saveTG{𧂨}{44802}
\saveTG{䠢}{44802}
\saveTG{𧃕}{44802}
\saveTG{䠠}{44802}
\saveTG{𧺷}{44803}
\saveTG{𦲃}{44803}
\saveTG{𧻐}{44803}
\saveTG{𧺘}{44803}
\saveTG{𧺯}{44803}
\saveTG{𧽚}{44803}
\saveTG{䞰}{44803}
\saveTG{䔪}{44804}
\saveTG{𧻶}{44804}
\saveTG{𦮥}{44804}
\saveTG{𡚙}{44804}
\saveTG{𦽽}{44804}
\saveTG{𦳣}{44804}
\saveTG{𧀵}{44804}
\saveTG{𦼣}{44804}
\saveTG{䓴}{44804}
\saveTG{𡗿}{44804}
\saveTG{𦻞}{44804}
\saveTG{𦵏}{44804}
\saveTG{𦴐}{44804}
\saveTG{蒵}{44804}
\saveTG{𦬞}{44804}
\saveTG{𧺜}{44804}
\saveTG{𦹬}{44804}
\saveTG{䞚}{44804}
\saveTG{𧼔}{44804}
\saveTG{𦮸}{44804}
\saveTG{𦷡}{44804}
\saveTG{𦴺}{44804}
\saveTG{𧀭}{44804}
\saveTG{芺}{44804}
\saveTG{樊}{44804}
\saveTG{葵}{44804}
\saveTG{荬}{44804}
\saveTG{莫}{44804}
\saveTG{葜}{44804}
\saveTG{葖}{44804}
\saveTG{茣}{44804}
\saveTG{薁}{44804}
\saveTG{芖}{44804}
\saveTG{𦬫}{44804}
\saveTG{𦮐}{44804}
\saveTG{𦶺}{44804}
\saveTG{𡙠}{44804}
\saveTG{苵}{44805}
\saveTG{芙}{44805}
\saveTG{荚}{44805}
\saveTG{芵}{44805}
\saveTG{𫈌}{44805}
\saveTG{𤴟}{44805}
\saveTG{𦿙}{44805}
\saveTG{英}{44805}
\saveTG{𦰩}{44805}
\saveTG{蔶}{44806}
\saveTG{蒷}{44806}
\saveTG{蔩}{44806}
\saveTG{藚}{44806}
\saveTG{貰}{44806}
\saveTG{藖}{44806}
\saveTG{薲}{44806}
\saveTG{蕒}{44806}
\saveTG{蕢}{44806}
\saveTG{趌}{44806}
\saveTG{黃}{44806}
\saveTG{黄}{44806}
\saveTG{𧄽}{44806}
\saveTG{薋}{44806}
\saveTG{萯}{44806}
\saveTG{𧂟}{44806}
\saveTG{𦽿}{44806}
\saveTG{䔈}{44806}
\saveTG{𪔵}{44806}
\saveTG{𧼼}{44806}
\saveTG{𦹘}{44806}
\saveTG{𧂰}{44806}
\saveTG{𧷖}{44806}
\saveTG{𦵄}{44806}
\saveTG{𩔎}{44806}
\saveTG{𧻰}{44806}
\saveTG{𦵊}{44806}
\saveTG{趞}{44806}
\saveTG{𦺱}{44806}
\saveTG{𫉉}{44806}
\saveTG{𧴹}{44806}
\saveTG{𦮷}{44806}
\saveTG{𧅤}{44806}
\saveTG{𧁐}{44806}
\saveTG{𦭜}{44806}
\saveTG{𤊾}{44806}
\saveTG{蕡}{44806}
\saveTG{𦮖}{44807}
\saveTG{𦬨}{44807}
\saveTG{𧺧}{44807}
\saveTG{萸}{44807}
\saveTG{趱}{44808}
\saveTG{莢}{44808}
\saveTG{𧾥}{44808}
\saveTG{𤑒}{44808}
\saveTG{𧻵}{44808}
\saveTG{趪}{44808}
\saveTG{趲}{44808}
\saveTG{𧽍}{44808}
\saveTG{𧼛}{44808}
\saveTG{燓}{44809}
\saveTG{焚}{44809}
\saveTG{𪺂}{44809}
\saveTG{𤈈}{44809}
\saveTG{𦹪}{44809}
\saveTG{𧽽}{44809}
\saveTG{煑}{44809}
\saveTG{𧽅}{44809}
\saveTG{䞽}{44809}
\saveTG{𤓮}{44809}
\saveTG{𧆋}{44809}
\saveTG{𧄖}{44809}
\saveTG{𦳒}{44809}
\saveTG{藀}{44809}
\saveTG{荧}{44809}
\saveTG{菼}{44809}
\saveTG{𤑔}{44809}
\saveTG{苂}{44809}
\saveTG{炗}{44809}
\saveTG{𦭹}{44809}
\saveTG{葅}{44812}
\saveTG{黆}{44812}
\saveTG{𪎽}{44812}
\saveTG{蘣}{44814}
\saveTG{蘳}{44814}
\saveTG{黊}{44814}
\saveTG{薙}{44815}
\saveTG{𧆀}{44815}
\saveTG{𧄰}{44815}
\saveTG{䕼}{44815}
\saveTG{𧃂}{44817}
\saveTG{𦾳}{44817}
\saveTG{𧁴}{44820}
\saveTG{𦵹}{44820}
\saveTG{萴}{44820}
\saveTG{荝}{44820}
\saveTG{𠔵}{44821}
\saveTG{䔮}{44821}
\saveTG{𦹱}{44821}
\saveTG{𣔠}{44821}
\saveTG{𧼖}{44821}
\saveTG{䠂}{44826}
\saveTG{𨁥}{44826}
\saveTG{𠢓}{44827}
\saveTG{𧀩}{44827}
\saveTG{䵋}{44827}
\saveTG{勚}{44827}
\saveTG{勩}{44827}
\saveTG{䓎}{44827}
\saveTG{𧂆}{44827}
\saveTG{䕿}{44827}
\saveTG{𧅖}{44827}
\saveTG{𧄞}{44827}
\saveTG{𦳼}{44827}
\saveTG{𦴔}{44827}
\saveTG{𧅅}{44827}
\saveTG{𧀿}{44827}
\saveTG{𫉩}{44827}
\saveTG{𧅃}{44831}
\saveTG{𧂒}{44831}
\saveTG{蔊}{44841}
\saveTG{𦻐}{44846}
\saveTG{𧂗}{44846}
\saveTG{𫊆}{44847}
\saveTG{𧃛}{44847}
\saveTG{𥀢}{44847}
\saveTG{皾}{44847}
\saveTG{蕿}{44847}
\saveTG{𤿺}{44847}
\saveTG{䒨}{44848}
\saveTG{蘞}{44848}
\saveTG{𡘽}{44849}
\saveTG{𦽒}{44853}
\saveTG{𦺬}{44853}
\saveTG{𧃮}{44859}
\saveTG{䓡}{44860}
\saveTG{𣞱}{44861}
\saveTG{囏}{44861}
\saveTG{𦺲}{44861}
\saveTG{𧅡}{44861}
\saveTG{𪏈}{44861}
\saveTG{𦹳}{44862}
\saveTG{𧃤}{44863}
\saveTG{𦹺}{44864}
\saveTG{䕰}{44869}
\saveTG{𧀜}{44877}
\saveTG{𦲊}{44877}
\saveTG{苁}{44880}
\saveTG{𡘋}{44880}
\saveTG{夶}{44880}
\saveTG{𠔶}{44881}
\saveTG{㒹}{44881}
\saveTG{薿}{44881}
\saveTG{𫈫}{44882}
\saveTG{𧼂}{44882}
\saveTG{𧾜}{44882}
\saveTG{𦹔}{44882}
\saveTG{蘝}{44882}
\saveTG{𪏟}{44884}
\saveTG{䔸}{44884}
\saveTG{薟}{44886}
\saveTG{薠}{44886}
\saveTG{𧁤}{44886}
\saveTG{𧄺}{44886}
\saveTG{蘱}{44886}
\saveTG{䵌}{44888}
\saveTG{𦲌}{44889}
\saveTG{𦼐}{44889}
\saveTG{㷼}{44891}
\saveTG{𦵒}{44894}
\saveTG{𫉜}{44896}
\saveTG{𤍾}{44898}
\saveTG{𦼜}{44899}
\saveTG{䒕}{44899}
\saveTG{槲}{44900}
\saveTG{枡}{44900}
\saveTG{樹}{44900}
\saveTG{榭}{44900}
\saveTG{树}{44900}
\saveTG{𣗒}{44900}
\saveTG{𣖟}{44900}
\saveTG{材}{44900}
\saveTG{村}{44900}
\saveTG{枓}{44900}
\saveTG{柎}{44900}
\saveTG{䕓}{44901}
\saveTG{𦽔}{44901}
\saveTG{𦯬}{44901}
\saveTG{蘌}{44901}
\saveTG{禁}{44901}
\saveTG{茮}{44901}
\saveTG{芣}{44901}
\saveTG{𦯝}{44901}
\saveTG{萗}{44901}
\saveTG{蔡}{44901}
\saveTG{蔈}{44901}
\saveTG{𦹁}{44901}
\saveTG{萘}{44901}
\saveTG{䒬}{44901}
\saveTG{葲}{44902}
\saveTG{荥}{44902}
\saveTG{茦}{44902}
\saveTG{𦵫}{44903}
\saveTG{虆}{44903}
\saveTG{䔝}{44903}
\saveTG{𦷲}{44903}
\saveTG{𦂌}{44903}
\saveTG{䒺}{44903}
\saveTG{𫉺}{44903}
\saveTG{𧄜}{44903}
\saveTG{𦹖}{44903}
\saveTG{𣙞}{44903}
\saveTG{蘩}{44903}
\saveTG{蘻}{44903}
\saveTG{蔂}{44903}
\saveTG{綦}{44903}
\saveTG{藄}{44903}
\saveTG{蕠}{44903}
\saveTG{萦}{44903}
\saveTG{繤}{44903}
\saveTG{𣑓}{44903}
\saveTG{𣗳}{44903}
\saveTG{𣓙}{44903}
\saveTG{𦴃}{44903}
\saveTG{䓱}{44904}
\saveTG{䒳}{44904}
\saveTG{𦯕}{44904}
\saveTG{𣛳}{44904}
\saveTG{𦺴}{44904}
\saveTG{𧃾}{44904}
\saveTG{𦲜}{44904}
\saveTG{𣕁}{44904}
\saveTG{𦴇}{44904}
\saveTG{橥}{44904}
\saveTG{𥡡}{44904}
\saveTG{𦾵}{44904}
\saveTG{𦰔}{44904}
\saveTG{䔧}{44904}
\saveTG{𦶝}{44904}
\saveTG{𦼹}{44904}
\saveTG{𦶫}{44904}
\saveTG{𦶖}{44904}
\saveTG{𦰧}{44904}
\saveTG{𦯧}{44904}
\saveTG{𦺵}{44904}
\saveTG{𦸯}{44904}
\saveTG{𦶿}{44904}
\saveTG{𦲢}{44904}
\saveTG{𧁸}{44904}
\saveTG{𧅾}{44904}
\saveTG{𫉂}{44904}
\saveTG{𧁣}{44904}
\saveTG{𦳮}{44904}
\saveTG{𦼕}{44904}
\saveTG{䔁}{44904}
\saveTG{𦾈}{44904}
\saveTG{𦷎}{44904}
\saveTG{𦳑}{44904}
\saveTG{𦬽}{44904}
\saveTG{𦻆}{44904}
\saveTG{㭟}{44904}
\saveTG{𦷞}{44904}
\saveTG{𥽚}{44904}
\saveTG{𦴕}{44904}
\saveTG{𦿺}{44904}
\saveTG{𣓏}{44904}
\saveTG{蘗}{44904}
\saveTG{蘖}{44904}
\saveTG{菜}{44904}
\saveTG{薒}{44904}
\saveTG{茶}{44904}
\saveTG{荼}{44904}
\saveTG{棻}{44904}
\saveTG{藁}{44904}
\saveTG{菒}{44904}
\saveTG{藳}{44904}
\saveTG{菓}{44904}
\saveTG{蘽}{44904}
\saveTG{蔾}{44904}
\saveTG{虊}{44904}
\saveTG{藥}{44904}
\saveTG{某}{44904}
\saveTG{糵}{44904}
\saveTG{棊}{44904}
\saveTG{𦵶}{44904}
\saveTG{蓁}{44904}
\saveTG{蕖}{44904}
\saveTG{蒅}{44904}
\saveTG{荣}{44904}
\saveTG{葇}{44904}
\saveTG{蘂}{44904}
\saveTG{蘃}{44904}
\saveTG{葉}{44904}
\saveTG{葈}{44904}
\saveTG{薬}{44904}
\saveTG{枼}{44904}
\saveTG{𦮞}{44904}
\saveTG{䕁}{44904}
\saveTG{𦮇}{44904}
\saveTG{𧁈}{44904}
\saveTG{𦸺}{44904}
\saveTG{𪔬}{44905}
\saveTG{茉}{44905}
\saveTG{茱}{44905}
\saveTG{䒹}{44905}
\saveTG{𦵴}{44905}
\saveTG{苿}{44905}
\saveTG{莱}{44905}
\saveTG{𦮵}{44905}
\saveTG{𦱮}{44905}
\saveTG{𦸛}{44905}
\saveTG{藔}{44906}
\saveTG{𦾁}{44906}
\saveTG{𦼔}{44906}
\saveTG{𦿴}{44906}
\saveTG{𦾍}{44906}
\saveTG{菄}{44906}
\saveTG{萂}{44906}
\saveTG{萰}{44906}
\saveTG{𦬼}{44908}
\saveTG{苶}{44908}
\saveTG{萊}{44908}
\saveTG{𣗅}{44908}
\saveTG{𫈤}{44909}
\saveTG{莍}{44909}
\saveTG{菉}{44909}
\saveTG{藜}{44909}
\saveTG{𦿾}{44909}
\saveTG{𦾅}{44909}
\saveTG{𦭅}{44909}
\saveTG{㳟}{44909}
\saveTG{櫆}{44910}
\saveTG{杹}{44910}
\saveTG{杜}{44910}
\saveTG{𣒣}{44910}
\saveTG{梉}{44910}
\saveTG{𦵬}{44911}
\saveTG{𦳐}{44911}
\saveTG{𣞳}{44911}
\saveTG{𣔶}{44912}
\saveTG{杝}{44912}
\saveTG{藴}{44912}
\saveTG{蘊}{44912}
\saveTG{萙}{44912}
\saveTG{𧅜}{44912}
\saveTG{𪱺}{44912}
\saveTG{𣖁}{44912}
\saveTG{𦼟}{44912}
\saveTG{㯼}{44912}
\saveTG{𦽁}{44912}
\saveTG{𣐴}{44912}
\saveTG{𦮝}{44912}
\saveTG{𫉇}{44912}
\saveTG{𦵕}{44912}
\saveTG{植}{44912}
\saveTG{𧄗}{44912}
\saveTG{𣠩}{44912}
\saveTG{𣕈}{44912}
\saveTG{葒}{44912}
\saveTG{枕}{44912}
\saveTG{𣞎}{44912}
\saveTG{㯛}{44912}
\saveTG{𣖻}{44912}
\saveTG{𣠅}{44912}
\saveTG{橈}{44912}
\saveTG{栳}{44912}
\saveTG{榼}{44912}
\saveTG{蒩}{44912}
\saveTG{藽}{44912}
\saveTG{𦱐}{44912}
\saveTG{𣏫}{44912}
\saveTG{𧁩}{44913}
\saveTG{𣔭}{44914}
\saveTG{檯}{44914}
\saveTG{楏}{44914}
\saveTG{𣑊}{44914}
\saveTG{𣜉}{44914}
\saveTG{𣗎}{44914}
\saveTG{𫇾}{44914}
\saveTG{𦻓}{44914}
\saveTG{𣟁}{44914}
\saveTG{𧄥}{44914}
\saveTG{𣖝}{44914}
\saveTG{𪲙}{44914}
\saveTG{𦽯}{44914}
\saveTG{椛}{44914}
\saveTG{𦶆}{44914}
\saveTG{𦽂}{44914}
\saveTG{𣑖}{44914}
\saveTG{荰}{44914}
\saveTG{蓕}{44914}
\saveTG{𦷺}{44914}
\saveTG{樭}{44914}
\saveTG{桂}{44914}
\saveTG{𧅛}{44915}
\saveTG{𧅈}{44915}
\saveTG{𧀑}{44915}
\saveTG{𦳁}{44915}
\saveTG{䕹}{44915}
\saveTG{蘿}{44915}
\saveTG{榷}{44915}
\saveTG{槿}{44915}
\saveTG{權}{44915}
\saveTG{㯵}{44915}
\saveTG{䕌}{44915}
\saveTG{楂}{44916}
\saveTG{𣜯}{44916}
\saveTG{𦶙}{44916}
\saveTG{𣘇}{44916}
\saveTG{櫙}{44916}
\saveTG{藲}{44916}
\saveTG{檶}{44916}
\saveTG{蘒}{44916}
\saveTG{𦹞}{44917}
\saveTG{𫉀}{44917}
\saveTG{𣎹}{44917}
\saveTG{𣒾}{44917}
\saveTG{𣜳}{44917}
\saveTG{𧁯}{44917}
\saveTG{萟}{44917}
\saveTG{枻}{44917}
\saveTG{蕝}{44917}
\saveTG{朹}{44917}
\saveTG{蒓}{44917}
\saveTG{𣞑}{44917}
\saveTG{𣞕}{44917}
\saveTG{㐜}{44917}
\saveTG{𦼀}{44917}
\saveTG{㭺}{44917}
\saveTG{𣚖}{44917}
\saveTG{𣕢}{44917}
\saveTG{蘒}{44917}
\saveTG{㭠}{44917}
\saveTG{𣓘}{44917}
\saveTG{𣒆}{44917}
\saveTG{𣞓}{44917}
\saveTG{𣏕}{44917}
\saveTG{𣓋}{44917}
\saveTG{㮱}{44917}
\saveTG{𣕺}{44917}
\saveTG{𣖨}{44917}
\saveTG{𣗄}{44917}
\saveTG{𣗩}{44917}
\saveTG{𫅷}{44917}
\saveTG{𦷏}{44918}
\saveTG{𦾢}{44918}
\saveTG{𪳮}{44918}
\saveTG{椹}{44918}
\saveTG{䓶}{44920}
\saveTG{䓭}{44920}
\saveTG{𦾑}{44920}
\saveTG{𦿦}{44920}
\saveTG{𣜎}{44920}
\saveTG{莉}{44920}
\saveTG{莿}{44920}
\saveTG{𣑥}{44921}
\saveTG{𫈥}{44921}
\saveTG{𫉈}{44921}
\saveTG{𦼢}{44921}
\saveTG{𧂦}{44921}
\saveTG{𦿟}{44921}
\saveTG{薪}{44921}
\saveTG{𣚂}{44921}
\saveTG{菥}{44921}
\saveTG{椅}{44921}
\saveTG{𧂅}{44922}
\saveTG{𦹮}{44922}
\saveTG{𣡒}{44927}
\saveTG{栛}{44927}
\saveTG{𣜜}{44927}
\saveTG{𫊊}{44927}
\saveTG{𠡟}{44927}
\saveTG{𧀁}{44927}
\saveTG{𦻽}{44927}
\saveTG{𦵂}{44927}
\saveTG{𦻖}{44927}
\saveTG{𦱜}{44927}
\saveTG{𦴙}{44927}
\saveTG{䔟}{44927}
\saveTG{𦯪}{44927}
\saveTG{𦼰}{44927}
\saveTG{𫈷}{44927}
\saveTG{𫊇}{44927}
\saveTG{𣡋}{44927}
\saveTG{栉}{44927}
\saveTG{栯}{44927}
\saveTG{葯}{44927}
\saveTG{桸}{44927}
\saveTG{楕}{44927}
\saveTG{椭}{44927}
\saveTG{蕛}{44927}
\saveTG{櫹}{44927}
\saveTG{藒}{44927}
\saveTG{槗}{44927}
\saveTG{藕}{44927}
\saveTG{枘}{44927}
\saveTG{楠}{44927}
\saveTG{蒳}{44927}
\saveTG{橗}{44927}
\saveTG{菞}{44927}
\saveTG{朸}{44927}
\saveTG{樠}{44927}
\saveTG{柨}{44927}
\saveTG{𣝀}{44927}
\saveTG{𣡑}{44927}
\saveTG{㯂}{44927}
\saveTG{𣓿}{44927}
\saveTG{𣖾}{44927}
\saveTG{𣟃}{44927}
\saveTG{𧄨}{44927}
\saveTG{𦾿}{44927}
\saveTG{𣘆}{44927}
\saveTG{勑}{44927}
\saveTG{𠢱}{44927}
\saveTG{𣏑}{44927}
\saveTG{𪲶}{44927}
\saveTG{𣑛}{44927}
\saveTG{𣔎}{44927}
\saveTG{㭶}{44927}
\saveTG{蕱}{44927}
\saveTG{𣝏}{44927}
\saveTG{𣟫}{44927}
\saveTG{𧆆}{44927}
\saveTG{𦽧}{44927}
\saveTG{𦽶}{44927}
\saveTG{𣚋}{44927}
\saveTG{𦶚}{44927}
\saveTG{𧂼}{44927}
\saveTG{𣔁}{44927}
\saveTG{𣔄}{44927}
\saveTG{䔠}{44927}
\saveTG{𣠝}{44927}
\saveTG{柪}{44927}
\saveTG{藊}{44927}
\saveTG{槆}{44927}
\saveTG{橢}{44927}
\saveTG{蒶}{44927}
\saveTG{𦷝}{44927}
\saveTG{菊}{44927}
\saveTG{栲}{44927}
\saveTG{欗}{44927}
\saveTG{𣚭}{44927}
\saveTG{𣕉}{44927}
\saveTG{𣗦}{44927}
\saveTG{𣓃}{44927}
\saveTG{𦼴}{44927}
\saveTG{𣚥}{44927}
\saveTG{𦻎}{44927}
\saveTG{𣝸}{44927}
\saveTG{𣗊}{44927}
\saveTG{㮁}{44927}
\saveTG{桍}{44927}
\saveTG{𣔷}{44928}
\saveTG{䕆}{44928}
\saveTG{䔋}{44929}
\saveTG{𣘡}{44929}
\saveTG{𣝉}{44930}
\saveTG{𣔀}{44930}
\saveTG{𣟑}{44930}
\saveTG{𣑗}{44930}
\saveTG{㯹}{44930}
\saveTG{𣒏}{44930}
\saveTG{𣑣}{44930}
\saveTG{𣚅}{44931}
\saveTG{㯖}{44931}
\saveTG{𣟛}{44931}
\saveTG{𣔝}{44931}
\saveTG{𣑆}{44931}
\saveTG{櫵}{44931}
\saveTG{梽}{44931}
\saveTG{藮}{44931}
\saveTG{𪲒}{44931}
\saveTG{𦶮}{44931}
\saveTG{𦷭}{44931}
\saveTG{𣡊}{44931}
\saveTG{㭕}{44931}
\saveTG{𧄝}{44931}
\saveTG{䕴}{44931}
\saveTG{𣕵}{44932}
\saveTG{𧁒}{44932}
\saveTG{𫉱}{44932}
\saveTG{𧅼}{44932}
\saveTG{𦮺}{44932}
\saveTG{䕯}{44932}
\saveTG{蘹}{44932}
\saveTG{榬}{44932}
\saveTG{檧}{44932}
\saveTG{菘}{44932}
\saveTG{檬}{44932}
\saveTG{𦸷}{44932}
\saveTG{𦶠}{44932}
\saveTG{𣙉}{44932}
\saveTG{𪴚}{44932}
\saveTG{㰐}{44932}
\saveTG{𣗠}{44933}
\saveTG{蔠}{44933}
\saveTG{𦷬}{44934}
\saveTG{𣞢}{44934}
\saveTG{蘕}{44935}
\saveTG{橽}{44935}
\saveTG{𣝘}{44935}
\saveTG{𣟀}{44935}
\saveTG{櫣}{44935}
\saveTG{梿}{44935}
\saveTG{𪳤}{44936}
\saveTG{𧂩}{44936}
\saveTG{𣞷}{44936}
\saveTG{蘓}{44936}
\saveTG{𧄉}{44936}
\saveTG{𦹴}{44937}
\saveTG{𣝈}{44937}
\saveTG{𣚜}{44938}
\saveTG{𣟆}{44939}
\saveTG{枿}{44940}
\saveTG{葤}{44940}
\saveTG{﨓}{44940}
\saveTG{㭉}{44940}
\saveTG{𣏼}{44940}
\saveTG{䔑}{44940}
\saveTG{𣝒}{44940}
\saveTG{𣏠}{44940}
\saveTG{萪}{44940}
\saveTG{𣟶}{44941}
\saveTG{桦}{44941}
\saveTG{𧄡}{44941}
\saveTG{檮}{44941}
\saveTG{𦿕}{44941}
\saveTG{𦰳}{44941}
\saveTG{𧁭}{44941}
\saveTG{𪲟}{44941}
\saveTG{橭}{44941}
\saveTG{榯}{44941}
\saveTG{榵}{44941}
\saveTG{𣘘}{44941}
\saveTG{𣔏}{44941}
\saveTG{𥢞}{44942}
\saveTG{䒩}{44942}
\saveTG{欂}{44942}
\saveTG{𦼭}{44942}
\saveTG{𧂭}{44943}
\saveTG{欂}{44943}
\saveTG{𣖖}{44943}
\saveTG{𣞨}{44943}
\saveTG{𣝍}{44943}
\saveTG{𦬸}{44943}
\saveTG{㭙}{44943}
\saveTG{𧃒}{44944}
\saveTG{𦿿}{44944}
\saveTG{𦾎}{44944}
\saveTG{𦰏}{44944}
\saveTG{𣙷}{44944}
\saveTG{𣞾}{44944}
\saveTG{𣒢}{44944}
\saveTG{𦾄}{44944}
\saveTG{𪳪}{44944}
\saveTG{薭}{44946}
\saveTG{枝}{44947}
\saveTG{皳}{44947}
\saveTG{𦷨}{44947}
\saveTG{檴}{44947}
\saveTG{䔴}{44947}
\saveTG{𣙻}{44947}
\saveTG{𢻢}{44947}
\saveTG{𢿞}{44947}
\saveTG{𣔰}{44947}
\saveTG{𦾫}{44947}
\saveTG{𧂔}{44947}
\saveTG{𧂣}{44947}
\saveTG{蔱}{44947}
\saveTG{䓩}{44947}
\saveTG{𦼊}{44947}
\saveTG{䕝}{44947}
\saveTG{𦵪}{44947}
\saveTG{𣡀}{44947}
\saveTG{㭳}{44947}
\saveTG{𥖥}{44947}
\saveTG{𣒶}{44947}
\saveTG{𣛵}{44947}
\saveTG{𣔡}{44947}
\saveTG{𣔓}{44947}
\saveTG{𦴡}{44947}
\saveTG{𦽑}{44947}
\saveTG{𣓒}{44947}
\saveTG{柀}{44947}
\saveTG{桲}{44947}
\saveTG{藧}{44947}
\saveTG{栫}{44947}
\saveTG{菽}{44947}
\saveTG{棱}{44947}
\saveTG{薐}{44947}
\saveTG{蘰}{44947}
\saveTG{𧄆}{44948}
\saveTG{㯜}{44948}
\saveTG{𦿎}{44948}
\saveTG{蓛}{44948}
\saveTG{𦷥}{44948}
\saveTG{𦲲}{44948}
\saveTG{䔩}{44948}
\saveTG{𪴈}{44949}
\saveTG{𣖂}{44951}
\saveTG{𣡄}{44952}
\saveTG{𦷑}{44952}
\saveTG{𣠧}{44952}
\saveTG{𣖥}{44952}
\saveTG{欌}{44953}
\saveTG{櫗}{44953}
\saveTG{𫊕}{44953}
\saveTG{𧂻}{44953}
\saveTG{𧁺}{44953}
\saveTG{𧄕}{44953}
\saveTG{𣕚}{44953}
\saveTG{𧀅}{44953}
\saveTG{樺}{44954}
\saveTG{樥}{44954}
\saveTG{𦷂}{44955}
\saveTG{𫇿}{44956}
\saveTG{㮖}{44956}
\saveTG{椲}{44956}
\saveTG{䔦}{44957}
\saveTG{檋}{44958}
\saveTG{枯}{44960}
\saveTG{葙}{44960}
\saveTG{櫧}{44960}
\saveTG{楮}{44960}
\saveTG{𦸖}{44960}
\saveTG{𣖚}{44960}
\saveTG{𣙊}{44960}
\saveTG{槠}{44960}
\saveTG{𣛃}{44960}
\saveTG{𣐞}{44960}
\saveTG{榰}{44961}
\saveTG{𣚬}{44961}
\saveTG{棤}{44961}
\saveTG{𦷮}{44961}
\saveTG{𫉎}{44961}
\saveTG{𧃫}{44961}
\saveTG{𣞸}{44961}
\saveTG{𦺢}{44961}
\saveTG{梏}{44961}
\saveTG{藉}{44961}
\saveTG{桔}{44961}
\saveTG{樯}{44961}
\saveTG{檣}{44961}
\saveTG{榙}{44961}
\saveTG{橲}{44961}
\saveTG{𦹄}{44962}
\saveTG{𦼝}{44962}
\saveTG{𦹀}{44962}
\saveTG{𧃴}{44962}
\saveTG{𦷙}{44962}
\saveTG{𦼁}{44962}
\saveTG{𦲀}{44963}
\saveTG{𧁃}{44963}
\saveTG{𧀗}{44963}
\saveTG{𦿨}{44963}
\saveTG{𫊄}{44963}
\saveTG{楛}{44964}
\saveTG{𣞦}{44964}
\saveTG{𣛰}{44964}
\saveTG{𦿀}{44964}
\saveTG{𣒅}{44964}
\saveTG{𣚫}{44964}
\saveTG{楉}{44964}
\saveTG{𦳇}{44965}
\saveTG{𣟞}{44967}
\saveTG{㯴}{44968}
\saveTG{𣞩}{44968}
\saveTG{𣟟}{44968}
\saveTG{𣞋}{44969}
\saveTG{𣒙}{44969}
\saveTG{柑}{44970}
\saveTG{𤯊}{44970}
\saveTG{𣖠}{44972}
\saveTG{𦱑}{44972}
\saveTG{𦴑}{44974}
\saveTG{欍}{44977}
\saveTG{杕}{44980}
\saveTG{𦻊}{44981}
\saveTG{𧅦}{44981}
\saveTG{𣜪}{44981}
\saveTG{𦶓}{44981}
\saveTG{槙}{44981}
\saveTG{檚}{44981}
\saveTG{棋}{44981}
\saveTG{栱}{44981}
\saveTG{𧀖}{44982}
\saveTG{𣘊}{44982}
\saveTG{䕀}{44982}
\saveTG{𪴙}{44982}
\saveTG{𫈺}{44982}
\saveTG{蔌}{44982}
\saveTG{𦳫}{44982}
\saveTG{𦰻}{44982}
\saveTG{𣘃}{44982}
\saveTG{𧀠}{44982}
\saveTG{㮠}{44984}
\saveTG{𧁨}{44984}
\saveTG{𣓎}{44984}
\saveTG{模}{44984}
\saveTG{樉}{44984}
\saveTG{𣙑}{44984}
\saveTG{𣙘}{44984}
\saveTG{𦳿}{44984}
\saveTG{𧅨}{44984}
\saveTG{椟}{44984}
\saveTG{𦺀}{44985}
\saveTG{楧}{44985}
\saveTG{𦷍}{44985}
\saveTG{𧄑}{44986}
\saveTG{𪴑}{44986}
\saveTG{藾}{44986}
\saveTG{蘔}{44986}
\saveTG{蘏}{44986}
\saveTG{橫}{44986}
\saveTG{𦹶}{44986}
\saveTG{𧀘}{44986}
\saveTG{𧂐}{44986}
\saveTG{横}{44986}
\saveTG{𦿽}{44986}
\saveTG{𣖌}{44986}
\saveTG{𧂬}{44986}
\saveTG{𣠇}{44986}
\saveTG{𣟨}{44986}
\saveTG{橨}{44986}
\saveTG{櫝}{44986}
\saveTG{欑}{44986}
\saveTG{𦲦}{44987}
\saveTG{楰}{44987}
\saveTG{𦿁}{44988}
\saveTG{梜}{44988}
\saveTG{𧀚}{44989}
\saveTG{萩}{44989}
\saveTG{𣚵}{44989}
\saveTG{𪲄}{44989}
\saveTG{𧃬}{44989}
\saveTG{楙}{44990}
\saveTG{欁}{44990}
\saveTG{𣗂}{44990}
\saveTG{𣓜}{44990}
\saveTG{𣑔}{44990}
\saveTG{𣝺}{44990}
\saveTG{㮟}{44990}
\saveTG{𪲴}{44990}
\saveTG{𣕪}{44990}
\saveTG{𣒑}{44990}
\saveTG{𣙳}{44990}
\saveTG{𣕎}{44990}
\saveTG{𣕑}{44990}
\saveTG{㮊}{44990}
\saveTG{𣗢}{44990}
\saveTG{𤴜}{44990}
\saveTG{𪳻}{44990}
\saveTG{棥}{44990}
\saveTG{𣐾}{44990}
\saveTG{㯎}{44990}
\saveTG{椕}{44990}
\saveTG{林}{44990}
\saveTG{㮈}{44991}
\saveTG{𣛂}{44991}
\saveTG{𧁞}{44991}
\saveTG{𧀂}{44991}
\saveTG{𡎓}{44991}
\saveTG{蒜}{44991}
\saveTG{櫒}{44991}
\saveTG{㯟}{44991}
\saveTG{𧄠}{44991}
\saveTG{𣔟}{44991}
\saveTG{㯲}{44991}
\saveTG{㮏}{44991}
\saveTG{𣗟}{44991}
\saveTG{蕀}{44992}
\saveTG{櫀}{44993}
\saveTG{𧄶}{44993}
\saveTG{㮦}{44993}
\saveTG{蕬}{44993}
\saveTG{𣡕}{44994}
\saveTG{𣡽}{44994}
\saveTG{楪}{44994}
\saveTG{栋}{44994}
\saveTG{菻}{44994}
\saveTG{楳}{44994}
\saveTG{𣛧}{44994}
\saveTG{𦼪}{44994}
\saveTG{𦺪}{44994}
\saveTG{𣜿}{44994}
\saveTG{𣛻}{44994}
\saveTG{𣘻}{44994}
\saveTG{𦼖}{44994}
\saveTG{𦽩}{44994}
\saveTG{𦿒}{44994}
\saveTG{䔉}{44994}
\saveTG{𣝾}{44994}
\saveTG{㯣}{44994}
\saveTG{𣘑}{44994}
\saveTG{𧅟}{44994}
\saveTG{𣟿}{44994}
\saveTG{㰛}{44994}
\saveTG{𧅂}{44994}
\saveTG{𤯏}{44994}
\saveTG{𣚨}{44994}
\saveTG{𦸸}{44994}
\saveTG{㯦}{44994}
\saveTG{𣗪}{44994}
\saveTG{𧀃}{44994}
\saveTG{𦼚}{44994}
\saveTG{𧀊}{44994}
\saveTG{}{44994}
\saveTG{𣙆}{44995}
\saveTG{𪳰}{44995}
\saveTG{𣝔}{44996}
\saveTG{䕩}{44996}
\saveTG{橑}{44996}
\saveTG{𦹂}{44996}
\saveTG{㯤}{44998}
\saveTG{𦸓}{44998}
\saveTG{棶}{44998}
\saveTG{㯃}{44999}
\saveTG{𣠬}{44999}
\saveTG{𦼋}{44999}
\saveTG{𫈳}{44999}
\saveTG{𦾯}{44999}
\saveTG{𧄃}{44999}
\saveTG{𪮺}{45012}
\saveTG{𡯥}{45013}
\saveTG{𧈤}{45013}
\saveTG{𡰌}{45014}
\saveTG{𪜔}{45016}
\saveTG{尵}{45018}
\saveTG{𠦽}{45033}
\saveTG{𪢽}{45100}
\saveTG{𡉋}{45100}
\saveTG{𡉯}{45102}
\saveTG{𥂕}{45102}
\saveTG{𦥎}{45104}
\saveTG{墊}{45104}
\saveTG{㘫}{45105}
\saveTG{𡉥}{45106}
\saveTG{坤}{45106}
\saveTG{垏}{45107}
\saveTG{𨫔}{45109}
\saveTG{䥍}{45109}
\saveTG{𡊳}{45110}
\saveTG{壗}{45112}
\saveTG{埶}{45117}
\saveTG{坉}{45117}
\saveTG{𡉕}{45117}
\saveTG{𡓾}{45118}
\saveTG{𪤡}{45118}
\saveTG{𧍐}{45120}
\saveTG{埥}{45127}
\saveTG{坲}{45127}
\saveTG{𡊖}{45130}
\saveTG{𪣑}{45132}
\saveTG{蟄}{45136}
\saveTG{𪤎}{45136}
\saveTG{𧐨}{45136}
\saveTG{𡒌}{45137}
\saveTG{𡋶}{45137}
\saveTG{壝}{45138}
\saveTG{𡐡}{45139}
\saveTG{塼}{45143}
\saveTG{塿}{45144}
\saveTG{𡒫}{45147}
\saveTG{埲}{45158}
\saveTG{𡊡}{45160}
\saveTG{𡌄}{45160}
\saveTG{𡌃}{45162}
\saveTG{𥀛}{45166}
\saveTG{堾}{45168}
\saveTG{坱}{45180}
\saveTG{块}{45180}
\saveTG{㙉}{45181}
\saveTG{墤}{45186}
\saveTG{𡋃}{45190}
\saveTG{𡊉}{45190}
\saveTG{𪣎}{45192}
\saveTG{塐}{45193}
\saveTG{𡏝}{45194}
\saveTG{埬}{45196}
\saveTG{𧯾}{45196}
\saveTG{堜}{45196}
\saveTG{埭}{45199}
\saveTG{𨥅}{45200}
\saveTG{𢂀}{45202}
\saveTG{𤞗}{45202}
\saveTG{𤚁}{45205}
\saveTG{㹬}{45206}
\saveTG{狆}{45206}
\saveTG{㹭}{45206}
\saveTG{𢂝}{45206}
\saveTG{𤝚}{45206}
\saveTG{𪻃}{45207}
\saveTG{𤯾}{45210}
\saveTG{狌}{45210}
\saveTG{𩙐}{45211}
\saveTG{𧖂}{45212}
\saveTG{𧑣}{45213}
\saveTG{𢄶}{45216}
\saveTG{𦘾}{45217}
\saveTG{㿪}{45217}
\saveTG{犱}{45217}
\saveTG{𤜨}{45217}
\saveTG{㹠}{45217}
\saveTG{𧰖}{45218}
\saveTG{𤣁}{45218}
\saveTG{𢄢}{45227}
\saveTG{𢃢}{45227}
\saveTG{猜}{45227}
\saveTG{狒}{45227}
\saveTG{帏}{45227}
\saveTG{犻}{45227}
\saveTG{𢂍}{45227}
\saveTG{𤢰}{45227}
\saveTG{㺜}{45232}
\saveTG{赨}{45236}
\saveTG{独}{45236}
\saveTG{𤡟}{45236}
\saveTG{𢅫}{45237}
\saveTG{帱}{45240}
\saveTG{㡞}{45244}
\saveTG{㺏}{45244}
\saveTG{𤝫}{45247}
\saveTG{𦞘}{45247}
\saveTG{𤟺}{45247}
\saveTG{㡚}{45247}
\saveTG{𢃶}{45257}
\saveTG{㹨}{45260}
\saveTG{𢂎}{45260}
\saveTG{𥀖}{45261}
\saveTG{𥀒}{45261}
\saveTG{㡟}{45266}
\saveTG{𤡐}{45266}
\saveTG{𤞍}{45272}
\saveTG{𢄣}{45274}
\saveTG{𤝽}{45274}
\saveTG{帙}{45280}
\saveTG{㠸}{45280}
\saveTG{狭}{45280}
\saveTG{猠}{45281}
\saveTG{𢂒}{45282}
\saveTG{㹟}{45282}
\saveTG{帻}{45282}
\saveTG{㹧}{45282}
\saveTG{㹫}{45282}
\saveTG{𢁪}{45282}
\saveTG{𡔿}{45284}
\saveTG{𢅙}{45286}
\saveTG{𤡱}{45286}
\saveTG{𢄱}{45286}
\saveTG{𥀠}{45286}
\saveTG{㺓}{45286}
\saveTG{幘}{45286}
\saveTG{帓}{45290}
\saveTG{𤞖}{45290}
\saveTG{𫈪}{45290}
\saveTG{𤿗}{45290}
\saveTG{𤝹}{45292}
\saveTG{獉}{45294}
\saveTG{𧹯}{45296}
\saveTG{𧹩}{45296}
\saveTG{𢃿}{45296}
\saveTG{𧥆}{45296}
\saveTG{𢣞}{45317}
\saveTG{鷙}{45327}
\saveTG{驇}{45327}
\saveTG{騺}{45327}
\saveTG{慹}{45331}
\saveTG{熱}{45331}
\saveTG{𢟯}{45331}
\saveTG{𤍠}{45331}
\saveTG{𪍦}{45332}
\saveTG{䲀}{45336}
\saveTG{𩻉}{45336}
\saveTG{𢢫}{45338}
\saveTG{滐}{45394}
\saveTG{妦}{45400}
\saveTG{妌}{45400}
\saveTG{𪌜}{45401}
\saveTG{𣁞}{45401}
\saveTG{𦗙}{45401}
\saveTG{妽}{45406}
\saveTG{𡜄}{45406}
\saveTG{妕}{45406}
\saveTG{𡝀}{45406}
\saveTG{𢻝}{45407}
\saveTG{姓}{45410}
\saveTG{嬔}{45411}
\saveTG{嬎}{45411}
\saveTG{嬧}{45412}
\saveTG{娆}{45412}
\saveTG{嫿}{45416}
\saveTG{執}{45417}
\saveTG{𡚺}{45417}
\saveTG{𡛋}{45417}
\saveTG{𡘺}{45417}
\saveTG{𪌋}{45417}
\saveTG{𡤠}{45418}
\saveTG{䵄}{45418}
\saveTG{𡞑}{45421}
\saveTG{𡛷}{45427}
\saveTG{韩}{45427}
\saveTG{𡛯}{45427}
\saveTG{婧}{45427}
\saveTG{娉}{45427}
\saveTG{勢}{45427}
\saveTG{姊}{45427}
\saveTG{姉}{45427}
\saveTG{}{45427}
\saveTG{𡞍}{45431}
\saveTG{𡞤}{45431}
\saveTG{𪍊}{45432}
\saveTG{𡢿}{45432}
\saveTG{婊}{45432}
\saveTG{𡠛}{45436}
\saveTG{㜕}{45436}
\saveTG{𡣺}{45437}
\saveTG{𡠗}{45441}
\saveTG{𡠦}{45441}
\saveTG{𡤛}{45443}
\saveTG{嫥}{45443}
\saveTG{𪥼}{45444}
\saveTG{㜢}{45444}
\saveTG{𪍣}{45444}
\saveTG{媾}{45447}
\saveTG{姌}{45447}
\saveTG{𡞹}{45447}
\saveTG{𡞗}{45458}
\saveTG{妯}{45460}
\saveTG{麯}{45460}
\saveTG{𡞐}{45466}
\saveTG{媋}{45468}
\saveTG{𪍲}{45468}
\saveTG{麩}{45480}
\saveTG{妋}{45480}
\saveTG{妷}{45480}
\saveTG{姎}{45480}
\saveTG{妜}{45480}
\saveTG{婕}{45481}
\saveTG{婰}{45481}
\saveTG{}{45482}
\saveTG{𪌊}{45482}
\saveTG{姨}{45482}
\saveTG{𡣶}{45486}
\saveTG{𡢲}{45486}
\saveTG{𡢻}{45486}
\saveTG{嫧}{45486}
\saveTG{嬇}{45486}
\saveTG{䴲}{45490}
\saveTG{妹}{45490}
\saveTG{姝}{45490}
\saveTG{妺}{45490}
\saveTG{𪥱}{45492}
\saveTG{嫊}{45493}
\saveTG{嫀}{45494}
\saveTG{𡣒}{45495}
\saveTG{𪦡}{45495}
\saveTG{𣙈}{45496}
\saveTG{媡}{45496}
\saveTG{娻}{45496}
\saveTG{娕}{45496}
\saveTG{𪌶}{45496}
\saveTG{𡠑}{45497}
\saveTG{𡝯}{45499}
\saveTG{𩉧}{45500}
\saveTG{𦎷}{45501}
\saveTG{摯}{45502}
\saveTG{𫖍}{45502}
\saveTG{𩎛}{45502}
\saveTG{摰}{45502}
\saveTG{𩏭}{45503}
\saveTG{𩊒}{45506}
\saveTG{𩎥}{45506}
\saveTG{𨎌}{45506}
\saveTG{𨎐}{45506}
\saveTG{𩉼}{45506}
\saveTG{𩉝}{45517}
\saveTG{𦎃}{45517}
\saveTG{䪆}{45518}
\saveTG{𫐏}{45520}
\saveTG{𩉽}{45527}
\saveTG{转}{45532}
\saveTG{韢}{45533}
\saveTG{𩏲}{45534}
\saveTG{䪈}{45537}
\saveTG{䪋}{45538}
\saveTG{鞬}{45540}
\saveTG{𩍌}{45542}
\saveTG{鞻}{45544}
\saveTG{𩏝}{45544}
\saveTG{韝}{45547}
\saveTG{鞲}{45547}
\saveTG{䩬}{45558}
\saveTG{䩜}{45560}
\saveTG{轴}{45560}
\saveTG{𩏚}{45574}
\saveTG{鞅}{45580}
\saveTG{轶}{45580}
\saveTG{辏}{45584}
\saveTG{𩏡}{45586}
\saveTG{𩍴}{45586}
\saveTG{鞼}{45586}
\saveTG{𩍭}{45586}
\saveTG{𩌪}{45586}
\saveTG{䩟}{45587}
\saveTG{韎}{45590}
\saveTG{𩎟}{45590}
\saveTG{靺}{45590}
\saveTG{𩎴}{45592}
\saveTG{𩎩}{45592}
\saveTG{𩊣}{45592}
\saveTG{𩌘}{45594}
\saveTG{𩏏}{45596}
\saveTG{𩎯}{45596}
\saveTG{𩊯}{45596}
\saveTG{暬}{45601}
\saveTG{謺}{45601}
\saveTG{𥊍}{45607}
\saveTG{𣊓}{45607}
\saveTG{𣛋}{45608}
\saveTG{𤯱}{45612}
\saveTG{𠫷}{45717}
\saveTG{𤮅}{45717}
\saveTG{𠅀}{45731}
\saveTG{𧜼}{45732}
\saveTG{褺}{45732}
\saveTG{㙯}{45732}
\saveTG{𩇘}{45747}
\saveTG{𦦘}{45781}
\saveTG{𤮡}{45782}
\saveTG{𧻊}{45801}
\saveTG{𨄧}{45801}
\saveTG{𧺡}{45802}
\saveTG{䠟}{45802}
\saveTG{𤴢}{45802}
\saveTG{趀}{45802}
\saveTG{䞞}{45802}
\saveTG{𡘗}{45803}
\saveTG{𧽝}{45804}
\saveTG{𧽢}{45804}
\saveTG{贄}{45806}
\saveTG{𧻭}{45806}
\saveTG{𧻉}{45806}
\saveTG{䞇}{45806}
\saveTG{赽}{45808}
\saveTG{䟎}{45808}
\saveTG{𧻯}{45808}
\saveTG{䟄}{45808}
\saveTG{𧻑}{45808}
\saveTG{趃}{45808}
\saveTG{𤍽}{45809}
\saveTG{趎}{45809}
\saveTG{趚}{45809}
\saveTG{𧻇}{45809}
\saveTG{𧻕}{45809}
\saveTG{𧽕}{45809}
\saveTG{𧼓}{45809}
\saveTG{𪎴}{45810}
\saveTG{𪎶}{45810}
\saveTG{𨃢}{45826}
\saveTG{𧽈}{45850}
\saveTG{㭋}{45900}
\saveTG{桝}{45900}
\saveTG{杖}{45900}
\saveTG{漐}{45902}
\saveTG{㭌}{45902}
\saveTG{縶}{45903}
\saveTG{𪱻}{45903}
\saveTG{𪲳}{45904}
\saveTG{槷}{45904}
\saveTG{𣏨}{45905}
\saveTG{榊}{45906}
\saveTG{柛}{45906}
\saveTG{𧀞}{45906}
\saveTG{𣗑}{45906}
\saveTG{𣒞}{45906}
\saveTG{梙}{45906}
\saveTG{栧}{45906}
\saveTG{𣑵}{45907}
\saveTG{𣓊}{45907}
\saveTG{𣕖}{45907}
\saveTG{𫆕}{45907}
\saveTG{栍}{45910}
\saveTG{𣖮}{45912}
\saveTG{𣝄}{45912}
\saveTG{㯸}{45912}
\saveTG{桡}{45912}
\saveTG{𣙶}{45915}
\saveTG{𣛛}{45916}
\saveTG{𣏱}{45917}
\saveTG{𣚆}{45917}
\saveTG{𣙀}{45917}
\saveTG{𪳡}{45917}
\saveTG{杶}{45917}
\saveTG{槸}{45917}
\saveTG{𣏒}{45917}
\saveTG{𠂃}{45917}
\saveTG{𪴊}{45918}
\saveTG{𪴀}{45918}
\saveTG{柹}{45927}
\saveTG{杮}{45927}
\saveTG{梻}{45927}
\saveTG{柫}{45927}
\saveTG{橚}{45927}
\saveTG{梬}{45927}
\saveTG{㭏}{45927}
\saveTG{𣚐}{45927}
\saveTG{𣐈}{45927}
\saveTG{𪲯}{45927}
\saveTG{棈}{45927}
\saveTG{枾}{45930}
\saveTG{槤}{45930}
\saveTG{𣑺}{45931}
\saveTG{𣡢}{45931}
\saveTG{𨘑}{45932}
\saveTG{𣓻}{45932}
\saveTG{欜}{45932}
\saveTG{檂}{45932}
\saveTG{𣏢}{45932}
\saveTG{𣚢}{45933}
\saveTG{橞}{45933}
\saveTG{槵}{45936}
\saveTG{㯾}{45936}
\saveTG{𣔴}{45936}
\saveTG{𣞝}{45937}
\saveTG{櫘}{45937}
\saveTG{𣟧}{45938}
\saveTG{𣜚}{45938}
\saveTG{㯈}{45939}
\saveTG{楗}{45940}
\saveTG{梼}{45940}
\saveTG{槫}{45943}
\saveTG{樓}{45944}
\saveTG{棲}{45944}
\saveTG{𣔤}{45945}
\saveTG{𣐳}{45945}
\saveTG{柟}{45947}
\saveTG{構}{45947}
\saveTG{𣞜}{45948}
\saveTG{𣞶}{45956}
\saveTG{𪳱}{45956}
\saveTG{𣕬}{45957}
\saveTG{棒}{45958}
\saveTG{柚}{45960}
\saveTG{𣓐}{45960}
\saveTG{𣛇}{45961}
\saveTG{𪳛}{45961}
\saveTG{𪲇}{45965}
\saveTG{槽}{45966}
\saveTG{𣚽}{45968}
\saveTG{椿}{45968}
\saveTG{𣡘}{45969}
\saveTG{𩏮}{45969}
\saveTG{樁}{45977}
\saveTG{槥}{45977}
\saveTG{}{45980}
\saveTG{柣}{45980}
\saveTG{枎}{45980}
\saveTG{柍}{45980}
\saveTG{椣}{45981}
\saveTG{𨠗}{45982}
\saveTG{桋}{45982}
\saveTG{㭈}{45982}
\saveTG{㮉}{45982}
\saveTG{楱}{45984}
\saveTG{𣙿}{45986}
\saveTG{樻}{45986}
\saveTG{樍}{45986}
\saveTG{櫕}{45986}
\saveTG{𣛫}{45989}
\saveTG{𣑳}{45990}
\saveTG{梾}{45990}
\saveTG{㭑}{45990}
\saveTG{枺}{45990}
\saveTG{株}{45990}
\saveTG{𫊚}{45992}
\saveTG{栜}{45992}
\saveTG{𥢉}{45992}
\saveTG{榡}{45993}
\saveTG{榛}{45994}
\saveTG{㰉}{45994}
\saveTG{𣡎}{45994}
\saveTG{𣟄}{45994}
\saveTG{榤}{45994}
\saveTG{𣠻}{45994}
\saveTG{𣝬}{45995}
\saveTG{梀}{45996}
\saveTG{楝}{45996}
\saveTG{棟}{45996}
\saveTG{樄}{45996}
\saveTG{隸}{45999}
\saveTG{隷}{45999}
\saveTG{棣}{45999}
\saveTG{𣗘}{45999}
\saveTG{𣟌}{45999}
\saveTG{𪴛}{45999}
\saveTG{加}{46000}
\saveTG{𡯟}{46010}
\saveTG{𤰙}{46010}
\saveTG{旭}{46010}
\saveTG{𡯙}{46010}
\saveTG{𩲄}{46011}
\saveTG{㞁}{46011}
\saveTG{𠦠}{46012}
\saveTG{𡯹}{46013}
\saveTG{㞇}{46013}
\saveTG{𪨈}{46013}
\saveTG{𠃳}{46016}
\saveTG{𡯬}{46018}
\saveTG{𡯻}{46018}
\saveTG{𡰆}{46019}
\saveTG{㞅}{46019}
\saveTG{㔖}{46027}
\saveTG{𥇹}{46043}
\saveTG{㽡}{46045}
\saveTG{𠢋}{46086}
\saveTG{𡊰}{46100}
\saveTG{𡊗}{46100}
\saveTG{𡋙}{46100}
\saveTG{堌}{46100}
\saveTG{𡊟}{46100}
\saveTG{𪤑}{46100}
\saveTG{𡑰}{46100}
\saveTG{𡉭}{46100}
\saveTG{𧰒}{46100}
\saveTG{㘻}{46100}
\saveTG{𡊚}{46102}
\saveTG{𥂸}{46102}
\saveTG{垍}{46102}
\saveTG{𤥏}{46104}
\saveTG{𤯥}{46105}
\saveTG{𨦔}{46109}
\saveTG{坦}{46110}
\saveTG{堽}{46111}
\saveTG{𧢰}{46112}
\saveTG{𧢚}{46112}
\saveTG{𡐘}{46112}
\saveTG{𡋳}{46112}
\saveTG{垷}{46112}
\saveTG{𥌵}{46112}
\saveTG{堒}{46112}
\saveTG{覲}{46112}
\saveTG{𪤂}{46112}
\saveTG{塭}{46112}
\saveTG{𫙌}{46113}
\saveTG{塊}{46113}
\saveTG{𡔁}{46114}
\saveTG{𪣝}{46114}
\saveTG{㘿}{46114}
\saveTG{埕}{46114}
\saveTG{堭}{46114}
\saveTG{埋}{46115}
\saveTG{𡑆}{46115}
\saveTG{𫓉}{46115}
\saveTG{𧠹}{46117}
\saveTG{𧡚}{46117}
\saveTG{𡓁}{46117}
\saveTG{𡌗}{46117}
\saveTG{郌}{46117}
\saveTG{𡌣}{46117}
\saveTG{𡉒}{46117}
\saveTG{𪣙}{46118}
\saveTG{𡌸}{46121}
\saveTG{𪤨}{46121}
\saveTG{㙌}{46124}
\saveTG{堮}{46127}
\saveTG{塄}{46127}
\saveTG{場}{46127}
\saveTG{𫛤}{46127}
\saveTG{𫛪}{46127}
\saveTG{塌}{46127}
\saveTG{堨}{46127}
\saveTG{𡒳}{46127}
\saveTG{𡍈}{46127}
\saveTG{㙕}{46127}
\saveTG{𪣍}{46127}
\saveTG{𡑦}{46127}
\saveTG{埚}{46127}
\saveTG{驾}{46127}
\saveTG{埍}{46127}
\saveTG{埸}{46127}
\saveTG{堣}{46127}
\saveTG{堺}{46128}
\saveTG{𡏁}{46130}
\saveTG{𪤄}{46131}
\saveTG{壜}{46131}
\saveTG{𡑡}{46132}
\saveTG{㙗}{46132}
\saveTG{𡔒}{46132}
\saveTG{𡎏}{46132}
\saveTG{㙷}{46133}
\saveTG{𧊟}{46136}
\saveTG{𧉪}{46136}
\saveTG{埤}{46140}
\saveTG{𡎎}{46141}
\saveTG{垾}{46141}
\saveTG{𪣨}{46141}
\saveTG{墿}{46141}
\saveTG{𡍇}{46142}
\saveTG{𡐙}{46143}
\saveTG{𪣰}{46144}
\saveTG{𡌹}{46144}
\saveTG{𪣚}{46144}
\saveTG{𪤃}{46144}
\saveTG{𧯿}{46145}
\saveTG{𡑇}{46147}
\saveTG{墁}{46147}
\saveTG{壧}{46148}
\saveTG{𡋟}{46148}
\saveTG{𡏻}{46148}
\saveTG{𡊠}{46150}
\saveTG{墠}{46156}
\saveTG{㙼}{46160}
\saveTG{𡋿}{46162}
\saveTG{𡑭}{46166}
\saveTG{𡐻}{46168}
\saveTG{𡌕}{46172}
\saveTG{𡉽}{46180}
\saveTG{垻}{46180}
\saveTG{堤}{46181}
\saveTG{埙}{46182}
\saveTG{𡐾}{46182}
\saveTG{𡏣}{46184}
\saveTG{𪣘}{46184}
\saveTG{塤}{46186}
\saveTG{堁}{46194}
\saveTG{堢}{46194}
\saveTG{𪤢}{46194}
\saveTG{㙞}{46194}
\saveTG{㙅}{46194}
\saveTG{𡐹}{46196}
\saveTG{𤞑}{46200}
\saveTG{𤝱}{46200}
\saveTG{𤝗}{46200}
\saveTG{𤡓}{46200}
\saveTG{㹢}{46200}
\saveTG{𤞧}{46200}
\saveTG{帼}{46200}
\saveTG{幗}{46200}
\saveTG{𪺺}{46200}
\saveTG{𧹢}{46200}
\saveTG{𢁯}{46200}
\saveTG{𢃠}{46200}
\saveTG{𤝍}{46200}
\saveTG{𪟗}{46201}
\saveTG{𤝼}{46202}
\saveTG{帕}{46202}
\saveTG{狛}{46202}
\saveTG{𥋈}{46210}
\saveTG{狚}{46210}
\saveTG{𦓍}{46211}
\saveTG{𧢖}{46212}
\saveTG{𧠶}{46212}
\saveTG{𩲾}{46212}
\saveTG{𪥚}{46212}
\saveTG{𤞾}{46212}
\saveTG{𠒠}{46212}
\saveTG{𧢬}{46212}
\saveTG{觀}{46212}
\saveTG{幌}{46212}
\saveTG{猑}{46212}
\saveTG{𤟡}{46214}
\saveTG{𢆀}{46214}
\saveTG{𨾵}{46215}
\saveTG{猩}{46215}
\saveTG{狸}{46215}
\saveTG{𩁒}{46215}
\saveTG{𢃇}{46215}
\saveTG{𢅾}{46215}
\saveTG{𤣓}{46215}
\saveTG{玀}{46215}
\saveTG{𤠞}{46217}
\saveTG{𧡮}{46217}
\saveTG{㡙}{46217}
\saveTG{𠙓}{46217}
\saveTG{𦞱}{46217}
\saveTG{𪻆}{46217}
\saveTG{𩳙}{46217}
\saveTG{𤞭}{46217}
\saveTG{㹸}{46217}
\saveTG{𢄊}{46217}
\saveTG{𢃚}{46217}
\saveTG{𩴿}{46217}
\saveTG{幌}{46217}
\saveTG{䚂}{46217}
\saveTG{𨚺}{46217}
\saveTG{𢂹}{46217}
\saveTG{䑄}{46221}
\saveTG{𪖪}{46221}
\saveTG{𤢳}{46221}
\saveTG{𢅩}{46227}
\saveTG{猡}{46227}
\saveTG{幆}{46227}
\saveTG{狷}{46227}
\saveTG{猲}{46227}
\saveTG{獨}{46227}
\saveTG{帤}{46227}
\saveTG{𤝐}{46227}
\saveTG{𤟕}{46227}
\saveTG{㡢}{46227}
\saveTG{𤢞}{46227}
\saveTG{𤟹}{46227}
\saveTG{𤟍}{46227}
\saveTG{𤠐}{46227}
\saveTG{𢂱}{46227}
\saveTG{𢅧}{46227}
\saveTG{𢃡}{46227}
\saveTG{𦙲}{46227}
\saveTG{𦙺}{46227}
\saveTG{𥀤}{46227}
\saveTG{嗧}{46227}
\saveTG{猬}{46227}
\saveTG{𤟧}{46230}
\saveTG{幒}{46230}
\saveTG{𪩿}{46231}
\saveTG{𧺂}{46231}
\saveTG{獧}{46232}
\saveTG{猥}{46232}
\saveTG{𤡠}{46232}
\saveTG{𧹽}{46233}
\saveTG{𤞋}{46240}
\saveTG{猈}{46240}
\saveTG{𤢟}{46241}
\saveTG{𦡇}{46241}
\saveTG{猂}{46241}
\saveTG{𤣎}{46244}
\saveTG{𢃍}{46245}
\saveTG{𤠎}{46247}
\saveTG{玃}{46247}
\saveTG{幔}{46247}
\saveTG{獌}{46247}
\saveTG{𢄸}{46247}
\saveTG{㿸}{46247}
\saveTG{獋}{46248}
\saveTG{獔}{46248}
\saveTG{獆}{46248}
\saveTG{玁}{46248}
\saveTG{狎}{46250}
\saveTG{𤠺}{46254}
\saveTG{㺗}{46256}
\saveTG{幝}{46256}
\saveTG{帽}{46260}
\saveTG{𤣠}{46260}
\saveTG{𢃑}{46260}
\saveTG{猖}{46260}
\saveTG{𤢹}{46260}
\saveTG{𤞪}{46262}
\saveTG{𥀜}{46264}
\saveTG{㺧}{46268}
\saveTG{𤣣}{46268}
\saveTG{𩏍}{46269}
\saveTG{𦼨}{46280}
\saveTG{𤝖}{46280}
\saveTG{狽}{46280}
\saveTG{帜}{46280}
\saveTG{㹱}{46282}
\saveTG{𧹤}{46282}
\saveTG{𤟥}{46282}
\saveTG{𢃰}{46282}
\saveTG{𤝲}{46284}
\saveTG{𢄙}{46286}
\saveTG{𤠔}{46286}
\saveTG{𤡂}{46293}
\saveTG{𧹶}{46293}
\saveTG{𤞥}{46294}
\saveTG{猓}{46294}
\saveTG{幧}{46294}
\saveTG{𤢖}{46294}
\saveTG{𢃦}{46296}
\saveTG{幜}{46296}
\saveTG{𪰛}{46300}
\saveTG{覟}{46312}
\saveTG{鸉}{46327}
\saveTG{𩣉}{46327}
\saveTG{𪂼}{46327}
\saveTG{𪈌}{46327}
\saveTG{鴐}{46327}
\saveTG{駕}{46327}
\saveTG{鴽}{46327}
\saveTG{㤎}{46330}
\saveTG{𢚤}{46330}
\saveTG{恕}{46330}
\saveTG{想}{46330}
\saveTG{𢟆}{46331}
\saveTG{𢤶}{46334}
\saveTG{𩶯}{46336}
\saveTG{𡟏}{46400}
\saveTG{㚳}{46400}
\saveTG{𡛸}{46400}
\saveTG{㜀}{46400}
\saveTG{𪥮}{46400}
\saveTG{𡞈}{46400}
\saveTG{婟}{46400}
\saveTG{如}{46400}
\saveTG{姻}{46400}
\saveTG{𦹸}{46400}
\saveTG{𪥨}{46400}
\saveTG{𪌦}{46400}
\saveTG{𪌔}{46400}
\saveTG{𪍁}{46400}
\saveTG{㚼}{46400}
\saveTG{𡜼}{46400}
\saveTG{𡜧}{46402}
\saveTG{𡜍}{46402}
\saveTG{𡛳}{46402}
\saveTG{妿}{46404}
\saveTG{𢻋}{46406}
\saveTG{妲}{46410}
\saveTG{媼}{46412}
\saveTG{麲}{46412}
\saveTG{婫}{46412}
\saveTG{娊}{46412}
\saveTG{媲}{46412}
\saveTG{媪}{46412}
\saveTG{㜼}{46412}
\saveTG{𪍝}{46412}
\saveTG{𧡵}{46412}
\saveTG{媿}{46413}
\saveTG{𩳔}{46413}
\saveTG{媓}{46414}
\saveTG{𡝚}{46414}
\saveTG{𡢨}{46414}
\saveTG{𡣫}{46414}
\saveTG{𪎆}{46415}
\saveTG{𡟙}{46415}
\saveTG{𡤢}{46415}
\saveTG{娌}{46415}
\saveTG{㜹}{46415}
\saveTG{𨙾}{46417}
\saveTG{𧠔}{46417}
\saveTG{𧢣}{46417}
\saveTG{𪌽}{46417}
\saveTG{𡟉}{46417}
\saveTG{㛕}{46417}
\saveTG{媲}{46417}
\saveTG{𪍜}{46417}
\saveTG{𡠌}{46417}
\saveTG{㚾}{46417}
\saveTG{𡤅}{46417}
\saveTG{嬶}{46421}
\saveTG{𡝭}{46427}
\saveTG{𪍹}{46427}
\saveTG{𪌭}{46427}
\saveTG{𪦊}{46427}
\saveTG{媀}{46427}
\saveTG{媦}{46427}
\saveTG{𪦨}{46427}
\saveTG{娲}{46427}
\saveTG{嬵}{46427}
\saveTG{婂}{46427}
\saveTG{娟}{46427}
\saveTG{婸}{46427}
\saveTG{𪦳}{46427}
\saveTG{𡤀}{46427}
\saveTG{𪦁}{46427}
\saveTG{娚}{46427}
\saveTG{𪥳}{46427}
\saveTG{𡣃}{46427}
\saveTG{𡛱}{46427}
\saveTG{㛫}{46427}
\saveTG{𡢚}{46427}
\saveTG{𡠴}{46430}
\saveTG{𡟯}{46430}
\saveTG{媳}{46430}
\saveTG{媤}{46430}
\saveTG{勰}{46430}
\saveTG{𡠤}{46431}
\saveTG{嫼}{46431}
\saveTG{嬛}{46432}
\saveTG{𪍭}{46432}
\saveTG{㛱}{46432}
\saveTG{𪍺}{46432}
\saveTG{𡣱}{46432}
\saveTG{𡣏}{46433}
\saveTG{𡟎}{46436}
\saveTG{𡟠}{46437}
\saveTG{婢}{46440}
\saveTG{𡣅}{46441}
\saveTG{娨}{46441}
\saveTG{嬕}{46441}
\saveTG{𡡎}{46444}
\saveTG{孆}{46444}
\saveTG{𡞌}{46444}
\saveTG{𡢞}{46444}
\saveTG{䴽}{46445}
\saveTG{𪍩}{46446}
\saveTG{𡤬}{46447}
\saveTG{𡡔}{46447}
\saveTG{𪴏}{46447}
\saveTG{嫚}{46447}
\saveTG{𦨦}{46447}
\saveTG{𡟷}{46448}
\saveTG{孍}{46448}
\saveTG{𡠖}{46449}
\saveTG{𪌸}{46452}
\saveTG{𪍪}{46454}
\saveTG{𡠚}{46454}
\saveTG{嬋}{46456}
\saveTG{㛎}{46460}
\saveTG{媢}{46460}
\saveTG{娼}{46460}
\saveTG{𪦮}{46460}
\saveTG{𡟤}{46461}
\saveTG{𡣈}{46464}
\saveTG{𡡧}{46468}
\saveTG{㛝}{46480}
\saveTG{𡛰}{46480}
\saveTG{𫖆}{46480}
\saveTG{娯}{46481}
\saveTG{媞}{46481}
\saveTG{娖}{46481}
\saveTG{𡠲}{46481}
\saveTG{㛣}{46482}
\saveTG{𡠕}{46482}
\saveTG{𡡍}{46482}
\saveTG{𡏱}{46483}
\saveTG{𡣬}{46484}
\saveTG{𡢊}{46484}
\saveTG{娛}{46484}
\saveTG{娱}{46484}
\saveTG{𡞃}{46484}
\saveTG{㜏}{46486}
\saveTG{𪦲}{46486}
\saveTG{㜥}{46486}
\saveTG{𡤯}{46493}
\saveTG{嫘}{46493}
\saveTG{𪍯}{46493}
\saveTG{媬}{46494}
\saveTG{婐}{46494}
\saveTG{𪍽}{46494}
\saveTG{嬠}{46494}
\saveTG{䴹}{46495}
\saveTG{𡡡}{46496}
\saveTG{𤙄}{46500}
\saveTG{𩎪}{46500}
\saveTG{鞇}{46500}
\saveTG{𩊏}{46500}
\saveTG{挐}{46502}
\saveTG{㧝}{46502}
\saveTG{靼}{46510}
\saveTG{𩊹}{46510}
\saveTG{辊}{46512}
\saveTG{𧡔}{46512}
\saveTG{𩋲}{46512}
\saveTG{䩤}{46512}
\saveTG{𩊷}{46512}
\saveTG{鞰}{46512}
\saveTG{韞}{46512}
\saveTG{辒}{46512}
\saveTG{𩌃}{46513}
\saveTG{鞓}{46514}
\saveTG{𩎊}{46515}
\saveTG{𩏳}{46517}
\saveTG{𩎏}{46517}
\saveTG{𨜢}{46517}
\saveTG{𩏐}{46517}
\saveTG{𩍗}{46520}
\saveTG{𩌇}{46527}
\saveTG{𩋌}{46527}
\saveTG{𩋤}{46527}
\saveTG{䪅}{46527}
\saveTG{韣}{46527}
\saveTG{𩋬}{46527}
\saveTG{䪚}{46527}
\saveTG{𩏌}{46527}
\saveTG{鞨}{46527}
\saveTG{鞙}{46527}
\saveTG{𦤎}{46530}
\saveTG{}{46532}
\saveTG{𩍡}{46532}
\saveTG{𩏰}{46533}
\saveTG{韅}{46533}
\saveTG{鞞}{46540}
\saveTG{𩍜}{46541}
\saveTG{𩏪}{46541}
\saveTG{䩰}{46541}
\saveTG{辑}{46541}
\saveTG{𩏤}{46541}
\saveTG{𩏂}{46545}
\saveTG{𫖏}{46547}
\saveTG{韟}{46548}
\saveTG{𩉾}{46550}
\saveTG{𦾏}{46550}
\saveTG{韠}{46554}
\saveTG{鞸}{46554}
\saveTG{𩏥}{46556}
\saveTG{𩍍}{46556}
\saveTG{𩎿}{46560}
\saveTG{𩏺}{46571}
\saveTG{𩊸}{46577}
\saveTG{轵}{46580}
\saveTG{鞮}{46581}
\saveTG{䪘}{46581}
\saveTG{𩍃}{46586}
\saveTG{𩌔}{46589}
\saveTG{𩌹}{46593}
\saveTG{𩏞}{46593}
\saveTG{𩋗}{46594}
\saveTG{𩌐}{46599}
\saveTG{𥆅}{46600}
\saveTG{𪡐}{46600}
\saveTG{𡄲}{46601}
\saveTG{𧧏}{46601}
\saveTG{𧦲}{46601}
\saveTG{䪪}{46601}
\saveTG{𡅦}{46608}
\saveTG{覩}{46612}
\saveTG{𩴟}{46613}
\saveTG{魗}{46613}
\saveTG{𧠯}{46617}
\saveTG{𧡺}{46617}
\saveTG{㖙}{46617}
\saveTG{㖲}{46617}
\saveTG{𨜞}{46617}
\saveTG{䰩}{46617}
\saveTG{哿}{46621}
\saveTG{䂟}{46621}
\saveTG{𪃙}{46627}
\saveTG{奲}{46656}
\saveTG{𥖸}{46691}
\saveTG{𦒵}{46700}
\saveTG{𧡥}{46712}
\saveTG{𧡪}{46712}
\saveTG{𧠉}{46712}
\saveTG{毠}{46714}
\saveTG{𣭠}{46715}
\saveTG{𨚠}{46717}
\saveTG{乫}{46717}
\saveTG{𤭪}{46717}
\saveTG{㠰}{46717}
\saveTG{𨜀}{46717}
\saveTG{朅}{46727}
\saveTG{𠡐}{46727}
\saveTG{𦡾}{46727}
\saveTG{袈}{46732}
\saveTG{䘫}{46732}
\saveTG{𠬈}{46745}
\saveTG{𡾎}{46772}
\saveTG{𪗬}{46772}
\saveTG{㔔}{46777}
\saveTG{𧺝}{46800}
\saveTG{𧺚}{46800}
\saveTG{𧻢}{46800}
\saveTG{𤲱}{46800}
\saveTG{𧼐}{46800}
\saveTG{𧼊}{46801}
\saveTG{𫎻}{46801}
\saveTG{𧻲}{46801}
\saveTG{䞹}{46801}
\saveTG{𧻼}{46801}
\saveTG{䞡}{46801}
\saveTG{𧼻}{46801}
\saveTG{𧾫}{46802}
\saveTG{贺}{46802}
\saveTG{𧽰}{46802}
\saveTG{𧾈}{46802}
\saveTG{𧽱}{46802}
\saveTG{䞶}{46802}
\saveTG{𧼮}{46802}
\saveTG{𧼺}{46802}
\saveTG{𧼫}{46802}
\saveTG{䟉}{46802}
\saveTG{𨇜}{46802}
\saveTG{𧽷}{46802}
\saveTG{𧾱}{46802}
\saveTG{𧼨}{46802}
\saveTG{𧻅}{46802}
\saveTG{䞟}{46802}
\saveTG{𧾎}{46803}
\saveTG{𧼷}{46803}
\saveTG{㚙}{46804}
\saveTG{𧾵}{46804}
\saveTG{𧾉}{46804}
\saveTG{䟂}{46804}
\saveTG{𧼠}{46804}
\saveTG{𧽉}{46804}
\saveTG{𧼣}{46804}
\saveTG{趕}{46804}
\saveTG{䟆}{46805}
\saveTG{賀}{46806}
\saveTG{趗}{46808}
\saveTG{𧽛}{46808}
\saveTG{𧽒}{46808}
\saveTG{𧽀}{46808}
\saveTG{趩}{46808}
\saveTG{𧼀}{46808}
\saveTG{𧻍}{46808}
\saveTG{趧}{46808}
\saveTG{趮}{46809}
\saveTG{𤇞}{46809}
\saveTG{𧽲}{46809}
\saveTG{𡤗}{46809}
\saveTG{𤈟}{46809}
\saveTG{覿}{46812}
\saveTG{𧡙}{46812}
\saveTG{䚆}{46812}
\saveTG{𧡅}{46812}
\saveTG{𧡂}{46817}
\saveTG{𧢜}{46817}
\saveTG{𩴺}{46817}
\saveTG{䵐}{46856}
\saveTG{𪏚}{46886}
\saveTG{𫉕}{46900}
\saveTG{𥈷}{46900}
\saveTG{𥡮}{46900}
\saveTG{𣐤}{46900}
\saveTG{𣓭}{46900}
\saveTG{𪲗}{46900}
\saveTG{𣏬}{46900}
\saveTG{𣐏}{46900}
\saveTG{㮯}{46900}
\saveTG{𣑝}{46900}
\saveTG{𣐬}{46900}
\saveTG{𣑩}{46900}
\saveTG{梱}{46900}
\saveTG{枷}{46900}
\saveTG{柶}{46900}
\saveTG{檲}{46900}
\saveTG{𣕳}{46900}
\saveTG{相}{46900}
\saveTG{栶}{46900}
\saveTG{椥}{46900}
\saveTG{𣒪}{46900}
\saveTG{𪱳}{46900}
\saveTG{𣑟}{46900}
\saveTG{槶}{46900}
\saveTG{棞}{46900}
\saveTG{椢}{46900}
\saveTG{棝}{46900}
\saveTG{𣒦}{46900}
\saveTG{𣕦}{46900}
\saveTG{𣙢}{46900}
\saveTG{𥙦}{46901}
\saveTG{㭡}{46902}
\saveTG{𣑉}{46902}
\saveTG{柏}{46902}
\saveTG{絮}{46903}
\saveTG{𥿃}{46903}
\saveTG{桇}{46904}
\saveTG{架}{46904}
\saveTG{𥹡}{46904}
\saveTG{𥹌}{46904}
\saveTG{𣐄}{46906}
\saveTG{柦}{46910}
\saveTG{檌}{46911}
\saveTG{𣒨}{46911}
\saveTG{㯰}{46912}
\saveTG{欟}{46912}
\saveTG{𣡭}{46912}
\saveTG{梎}{46912}
\saveTG{棍}{46912}
\saveTG{槻}{46912}
\saveTG{𣗵}{46912}
\saveTG{榥}{46912}
\saveTG{梘}{46912}
\saveTG{榅}{46912}
\saveTG{榲}{46912}
\saveTG{柷}{46912}
\saveTG{櫬}{46912}
\saveTG{𣓛}{46912}
\saveTG{𣙐}{46912}
\saveTG{𣡟}{46912}
\saveTG{𩳶}{46913}
\saveTG{槐}{46913}
\saveTG{𣠠}{46914}
\saveTG{檉}{46914}
\saveTG{楻}{46914}
\saveTG{桯}{46914}
\saveTG{梍}{46914}
\saveTG{𣞪}{46914}
\saveTG{欏}{46915}
\saveTG{𣛀}{46915}
\saveTG{榸}{46915}
\saveTG{欋}{46915}
\saveTG{梩}{46915}
\saveTG{𧡘}{46917}
\saveTG{𩳃}{46917}
\saveTG{𣞻}{46917}
\saveTG{𣘌}{46917}
\saveTG{𧢎}{46917}
\saveTG{𣠤}{46917}
\saveTG{𪳩}{46917}
\saveTG{㮰}{46917}
\saveTG{䚅}{46917}
\saveTG{𣑢}{46917}
\saveTG{𪳒}{46917}
\saveTG{𣠲}{46918}
\saveTG{𣟭}{46921}
\saveTG{𣒋}{46921}
\saveTG{椤}{46927}
\saveTG{楊}{46927}
\saveTG{櫋}{46927}
\saveTG{𣟡}{46927}
\saveTG{𪲺}{46927}
\saveTG{𣞌}{46927}
\saveTG{㭿}{46927}
\saveTG{𣓾}{46927}
\saveTG{𣗨}{46927}
\saveTG{檰}{46927}
\saveTG{棉}{46927}
\saveTG{𣑐}{46927}
\saveTG{楞}{46927}
\saveTG{梋}{46927}
\saveTG{柺}{46927}
\saveTG{枴}{46927}
\saveTG{𣝼}{46927}
\saveTG{𣙾}{46927}
\saveTG{㯞}{46927}
\saveTG{𤳇}{46927}
\saveTG{𪳏}{46927}
\saveTG{𣕶}{46927}
\saveTG{㯮}{46927}
\saveTG{㭷}{46927}
\saveTG{𣖜}{46927}
\saveTG{𣒌}{46927}
\saveTG{楬}{46927}
\saveTG{榻}{46927}
\saveTG{枵}{46927}
\saveTG{楐}{46928}
\saveTG{𣜮}{46929}
\saveTG{㮩}{46930}
\saveTG{樬}{46930}
\saveTG{楒}{46930}
\saveTG{𪳴}{46931}
\saveTG{𪴘}{46931}
\saveTG{𣘓}{46931}
\saveTG{𣘸}{46931}
\saveTG{𣚦}{46931}
\saveTG{𣡤}{46932}
\saveTG{𣖐}{46932}
\saveTG{𪳇}{46932}
\saveTG{檈}{46932}
\saveTG{椳}{46932}
\saveTG{𣜶}{46932}
\saveTG{𣜛}{46932}
\saveTG{𣞲}{46932}
\saveTG{𣟳}{46932}
\saveTG{𣑬}{46940}
\saveTG{椑}{46940}
\saveTG{棏}{46941}
\saveTG{桿}{46941}
\saveTG{楫}{46941}
\saveTG{檡}{46941}
\saveTG{𣙵}{46943}
\saveTG{𣘱}{46943}
\saveTG{𣗤}{46944}
\saveTG{𣔬}{46944}
\saveTG{櫻}{46944}
\saveTG{𣡠}{46947}
\saveTG{㮨}{46947}
\saveTG{𣙺}{46947}
\saveTG{槾}{46947}
\saveTG{樶}{46947}
\saveTG{欔}{46947}
\saveTG{𣠨}{46948}
\saveTG{𣘶}{46948}
\saveTG{欕}{46948}
\saveTG{槔}{46948}
\saveTG{槹}{46948}
\saveTG{橰}{46948}
\saveTG{柙}{46950}
\saveTG{𣖹}{46952}
\saveTG{𪳔}{46953}
\saveTG{㮿}{46954}
\saveTG{𣜸}{46956}
\saveTG{𨢿}{46956}
\saveTG{𣞹}{46956}
\saveTG{樿}{46956}
\saveTG{梠}{46960}
\saveTG{椙}{46960}
\saveTG{橸}{46960}
\saveTG{櫑}{46960}
\saveTG{榀}{46960}
\saveTG{𣓡}{46962}
\saveTG{𣞍}{46964}
\saveTG{𪳲}{46968}
\saveTG{㮙}{46971}
\saveTG{𣡧}{46974}
\saveTG{𣏥}{46980}
\saveTG{梖}{46980}
\saveTG{枳}{46980}
\saveTG{𣚣}{46981}
\saveTG{㮛}{46982}
\saveTG{𣔑}{46984}
\saveTG{𣗬}{46984}
\saveTG{𣑀}{46984}
\saveTG{𣞥}{46986}
\saveTG{𣗼}{46986}
\saveTG{𣛠}{46986}
\saveTG{𪳺}{46986}
\saveTG{𣙠}{46989}
\saveTG{楾}{46992}
\saveTG{樏}{46993}
\saveTG{欙}{46993}
\saveTG{𣕱}{46994}
\saveTG{𪲷}{46994}
\saveTG{𣕧}{46994}
\saveTG{𣝇}{46994}
\saveTG{𪳑}{46994}
\saveTG{棵}{46994}
\saveTG{椺}{46994}
\saveTG{橾}{46994}
\saveTG{𣗶}{46999}
\saveTG{𣞺}{46999}
\saveTG{𧬓}{47001}
\saveTG{䫺}{47010}
\saveTG{𡯡}{47011}
\saveTG{尯}{47011}
\saveTG{𡯖}{47012}
\saveTG{𡯌}{47012}
\saveTG{𡰎}{47012}
\saveTG{𩨔}{47012}
\saveTG{𩾛}{47012}
\saveTG{尳}{47012}
\saveTG{尥}{47012}
\saveTG{𡯣}{47013}
\saveTG{𩵛}{47013}
\saveTG{𡯉}{47014}
\saveTG{𡰀}{47015}
\saveTG{𡰃}{47017}
\saveTG{𡯿}{47018}
\saveTG{𡯦}{47019}
\saveTG{尮}{47019}
\saveTG{𩾵}{47022}
\saveTG{𣚾}{47024}
\saveTG{𨙿}{47027}
\saveTG{𨙩}{47027}
\saveTG{𠦢}{47027}
\saveTG{𨚧}{47027}
\saveTG{𩾢}{47027}
\saveTG{𪁪}{47027}
\saveTG{𩾘}{47027}
\saveTG{𩾞}{47027}
\saveTG{𪤺}{47027}
\saveTG{鸠}{47027}
\saveTG{鳩}{47027}
\saveTG{弩}{47027}
\saveTG{邥}{47027}
\saveTG{鸩}{47027}
\saveTG{鴆}{47027}
\saveTG{𩵫}{47036}
\saveTG{㱽}{47047}
\saveTG{𣖆}{47047}
\saveTG{𠦗}{47077}
\saveTG{𠦰}{47082}
\saveTG{𣢌}{47082}
\saveTG{𢹦}{47082}
\saveTG{𣫒}{47102}
\saveTG{𣫝}{47102}
\saveTG{𣫆}{47102}
\saveTG{𥂩}{47102}
\saveTG{𣫨}{47102}
\saveTG{𥃟}{47102}
\saveTG{𥂎}{47102}
\saveTG{𥃃}{47102}
\saveTG{𥁨}{47102}
\saveTG{𨰌}{47104}
\saveTG{𨥓}{47104}
\saveTG{𪣖}{47104}
\saveTG{𡌰}{47104}
\saveTG{𡎷}{47104}
\saveTG{𡍭}{47104}
\saveTG{𤣬}{47104}
\saveTG{𡒼}{47104}
\saveTG{𣫤}{47108}
\saveTG{鋆}{47109}
\saveTG{鏧}{47109}
\saveTG{𨫦}{47109}
\saveTG{䤰}{47109}
\saveTG{𨪄}{47109}
\saveTG{𨬝}{47109}
\saveTG{𨥬}{47109}
\saveTG{𩙏}{47110}
\saveTG{堸}{47110}
\saveTG{𡎐}{47110}
\saveTG{𩙣}{47112}
\saveTG{𡎠}{47112}
\saveTG{𧰕}{47112}
\saveTG{𡉶}{47112}
\saveTG{垝}{47112}
\saveTG{觐}{47112}
\saveTG{坥}{47112}
\saveTG{坭}{47112}
\saveTG{堄}{47112}
\saveTG{垉}{47112}
\saveTG{堍}{47113}
\saveTG{𡐖}{47114}
\saveTG{𡍤}{47114}
\saveTG{𡒘}{47114}
\saveTG{𡎔}{47114}
\saveTG{𡒔}{47115}
\saveTG{𪣮}{47117}
\saveTG{䵷}{47117}
\saveTG{𪣀}{47117}
\saveTG{𡉏}{47117}
\saveTG{𪣓}{47117}
\saveTG{圯}{47117}
\saveTG{圮}{47117}
\saveTG{坈}{47117}
\saveTG{㘲}{47117}
\saveTG{𪢵}{47117}
\saveTG{𫇤}{47117}
\saveTG{𡏎}{47117}
\saveTG{𡓋}{47117}
\saveTG{𡊬}{47117}
\saveTG{𡓦}{47117}
\saveTG{均}{47120}
\saveTG{圽}{47120}
\saveTG{壛}{47120}
\saveTG{圴}{47120}
\saveTG{𡊶}{47120}
\saveTG{垌}{47120}
\saveTG{𡊁}{47120}
\saveTG{𡊧}{47120}
\saveTG{堈}{47120}
\saveTG{坸}{47120}
\saveTG{墹}{47120}
\saveTG{坰}{47120}
\saveTG{垧}{47120}
\saveTG{埛}{47120}
\saveTG{堋}{47120}
\saveTG{𡍜}{47121}
\saveTG{𡎁}{47121}
\saveTG{㘧}{47121}
\saveTG{𪣱}{47121}
\saveTG{𪻪}{47121}
\saveTG{㙟}{47121}
\saveTG{𪢻}{47121}
\saveTG{𡐐}{47121}
\saveTG{㘭}{47121}
\saveTG{𠜤}{47121}
\saveTG{𡓲}{47122}
\saveTG{𡎾}{47122}
\saveTG{𧰝}{47122}
\saveTG{𡉍}{47122}
\saveTG{𪤗}{47122}
\saveTG{圽}{47122}
\saveTG{㘬}{47123}
\saveTG{𡉳}{47124}
\saveTG{𡊤}{47124}
\saveTG{𡎧}{47125}
\saveTG{垧}{47126}
\saveTG{𡋕}{47126}
\saveTG{𧯼}{47126}
\saveTG{坞}{47127}
\saveTG{埽}{47127}
\saveTG{鵱}{47127}
\saveTG{埆}{47127}
\saveTG{墎}{47127}
\saveTG{堝}{47127}
\saveTG{邽}{47127}
\saveTG{垑}{47127}
\saveTG{场}{47127}
\saveTG{垹}{47127}
\saveTG{𡎮}{47127}
\saveTG{𡉖}{47127}
\saveTG{𡑹}{47127}
\saveTG{𡑏}{47127}
\saveTG{𡍾}{47127}
\saveTG{𡊐}{47127}
\saveTG{𫛷}{47127}
\saveTG{𡋝}{47127}
\saveTG{𡌝}{47127}
\saveTG{𡏠}{47127}
\saveTG{𪅧}{47127}
\saveTG{𤍫}{47127}
\saveTG{𡏭}{47127}
\saveTG{䲧}{47127}
\saveTG{𡏺}{47127}
\saveTG{䳏}{47127}
\saveTG{𪇠}{47127}
\saveTG{𪅀}{47127}
\saveTG{𪈭}{47127}
\saveTG{𡓂}{47127}
\saveTG{𡌈}{47127}
\saveTG{𡏿}{47127}
\saveTG{𡋓}{47127}
\saveTG{𡓏}{47127}
\saveTG{𡌟}{47127}
\saveTG{𡌞}{47127}
\saveTG{𡌲}{47127}
\saveTG{𡉁}{47127}
\saveTG{𨜴}{47127}
\saveTG{𨙭}{47127}
\saveTG{𦐫}{47127}
\saveTG{𨞨}{47127}
\saveTG{𨜨}{47127}
\saveTG{驽}{47127}
\saveTG{埇}{47127}
\saveTG{鄞}{47127}
\saveTG{鷧}{47127}
\saveTG{壻}{47127}
\saveTG{塢}{47127}
\saveTG{𧕱}{47131}
\saveTG{𡐆}{47131}
\saveTG{𧒕}{47131}
\saveTG{𡏈}{47131}
\saveTG{𪤍}{47132}
\saveTG{堟}{47132}
\saveTG{塚}{47132}
\saveTG{塜}{47132}
\saveTG{垠}{47132}
\saveTG{𦫒}{47132}
\saveTG{𡑩}{47132}
\saveTG{𡋤}{47133}
\saveTG{塳}{47135}
\saveTG{𣫣}{47136}
\saveTG{𧍵}{47136}
\saveTG{𡐚}{47136}
\saveTG{𧏌}{47136}
\saveTG{螜}{47136}
\saveTG{𧉭}{47136}
\saveTG{𧐜}{47136}
\saveTG{𧐡}{47136}
\saveTG{塠}{47137}
\saveTG{懿}{47138}
\saveTG{㦤}{47138}
\saveTG{𢤥}{47138}
\saveTG{坍}{47140}
\saveTG{埱}{47140}
\saveTG{𡊈}{47141}
\saveTG{𡊓}{47141}
\saveTG{𪣟}{47141}
\saveTG{𡎰}{47141}
\saveTG{塀}{47141}
\saveTG{𡓹}{47141}
\saveTG{𪤖}{47143}
\saveTG{𡏒}{47143}
\saveTG{𡏷}{47143}
\saveTG{𡑎}{47143}
\saveTG{𡊢}{47144}
\saveTG{𡤜}{47144}
\saveTG{𪤞}{47145}
\saveTG{塅}{47147}
\saveTG{𡎹}{47147}
\saveTG{㙍}{47147}
\saveTG{𪔰}{47147}
\saveTG{𪤩}{47147}
\saveTG{𡉗}{47147}
\saveTG{埠}{47147}
\saveTG{圾}{47147}
\saveTG{埐}{47147}
\saveTG{瑴}{47147}
\saveTG{垊}{47147}
\saveTG{坄}{47147}
\saveTG{𣕒}{47147}
\saveTG{𢃣}{47147}
\saveTG{𪢲}{47147}
\saveTG{𡊋}{47147}
\saveTG{𡐄}{47147}
\saveTG{𡓻}{47147}
\saveTG{𣪝}{47147}
\saveTG{𡑴}{47147}
\saveTG{㲄}{47147}
\saveTG{𣪊}{47147}
\saveTG{𧏚}{47147}
\saveTG{𧯸}{47147}
\saveTG{𧎅}{47147}
\saveTG{𡒂}{47147}
\saveTG{㙾}{47147}
\saveTG{𡒠}{47148}
\saveTG{坶}{47150}
\saveTG{𡑳}{47151}
\saveTG{𡒛}{47152}
\saveTG{堚}{47152}
\saveTG{埄}{47154}
\saveTG{}{47154}
\saveTG{𡉱}{47155}
\saveTG{𪣒}{47157}
\saveTG{埩}{47157}
\saveTG{墀}{47159}
\saveTG{𪣈}{47161}
\saveTG{㙴}{47161}
\saveTG{塯}{47162}
\saveTG{𡏽}{47162}
\saveTG{𡊱}{47162}
\saveTG{𡎣}{47162}
\saveTG{𡓇}{47163}
\saveTG{垎}{47164}
\saveTG{𡍄}{47164}
\saveTG{𡍉}{47164}
\saveTG{𪣣}{47165}
\saveTG{𡒸}{47166}
\saveTG{堳}{47167}
\saveTG{𡕆}{47168}
\saveTG{𡄻}{47168}
\saveTG{垇}{47170}
\saveTG{𡌏}{47171}
\saveTG{𪣦}{47171}
\saveTG{堀}{47172}
\saveTG{𡔐}{47172}
\saveTG{𡑥}{47172}
\saveTG{𡒈}{47172}
\saveTG{𡌷}{47172}
\saveTG{埳}{47177}
\saveTG{垖}{47177}
\saveTG{𡊊}{47177}
\saveTG{塓}{47180}
\saveTG{㘮}{47180}
\saveTG{𪤧}{47181}
\saveTG{𡉌}{47181}
\saveTG{埧}{47181}
\saveTG{㱅}{47182}
\saveTG{㰻}{47182}
\saveTG{𣣹}{47182}
\saveTG{㰪}{47182}
\saveTG{𣣥}{47182}
\saveTG{𣤼}{47182}
\saveTG{坎}{47182}
\saveTG{歏}{47182}
\saveTG{坝}{47182}
\saveTG{𡑨}{47182}
\saveTG{𡎇}{47184}
\saveTG{𪣷}{47184}
\saveTG{墺}{47184}
\saveTG{堠}{47184}
\saveTG{𣡆}{47184}
\saveTG{𡒕}{47186}
\saveTG{𡓒}{47186}
\saveTG{𧹌}{47186}
\saveTG{𡊑}{47192}
\saveTG{𡐤}{47193}
\saveTG{垜}{47194}
\saveTG{𣞙}{47194}
\saveTG{𡕏}{47194}
\saveTG{𪤌}{47194}
\saveTG{垛}{47194}
\saveTG{𡑕}{47194}
\saveTG{堔}{47194}
\saveTG{𪤤}{47199}
\saveTG{𡍖}{47199}
\saveTG{𧔌}{47199}
\saveTG{𠨚}{47200}
\saveTG{𤜿}{47210}
\saveTG{𤞁}{47210}
\saveTG{飌}{47210}
\saveTG{帆}{47210}
\saveTG{猦}{47210}
\saveTG{𩙤}{47210}
\saveTG{𤜢}{47210}
\saveTG{猊}{47212}
\saveTG{狔}{47212}
\saveTG{狃}{47212}
\saveTG{翹}{47212}
\saveTG{狍}{47212}
\saveTG{皰}{47212}
\saveTG{𠒘}{47212}
\saveTG{𠣻}{47212}
\saveTG{𢂈}{47212}
\saveTG{𤝺}{47212}
\saveTG{𫎭}{47212}
\saveTG{𤠏}{47212}
\saveTG{𢁡}{47212}
\saveTG{𤟣}{47212}
\saveTG{犯}{47212}
\saveTG{狙}{47212}
\saveTG{匏}{47212}
\saveTG{𡗉}{47212}
\saveTG{猛}{47212}
\saveTG{𠓈}{47213}
\saveTG{幄}{47214}
\saveTG{𤠿}{47214}
\saveTG{㹩}{47214}
\saveTG{𩁭}{47215}
\saveTG{㺟}{47215}
\saveTG{𩀠}{47215}
\saveTG{𨾯}{47215}
\saveTG{𩁡}{47215}
\saveTG{𩀉}{47215}
\saveTG{𠓏}{47216}
\saveTG{𥀨}{47217}
\saveTG{𤡯}{47217}
\saveTG{𤜦}{47217}
\saveTG{𤜝}{47217}
\saveTG{𤜱}{47217}
\saveTG{㠶}{47217}
\saveTG{𢁒}{47217}
\saveTG{𤢡}{47217}
\saveTG{赩}{47217}
\saveTG{帊}{47217}
\saveTG{𤣚}{47217}
\saveTG{𢂕}{47217}
\saveTG{𦫗}{47217}
\saveTG{㿬}{47217}
\saveTG{𪚣}{47217}
\saveTG{𦫚}{47217}
\saveTG{𦫞}{47217}
\saveTG{𩱼}{47217}
\saveTG{𧹧}{47217}
\saveTG{㡈}{47217}
\saveTG{𫌤}{47217}
\saveTG{𤟛}{47217}
\saveTG{㺥}{47217}
\saveTG{𫆜}{47217}
\saveTG{䞔}{47217}
\saveTG{犳}{47220}
\saveTG{狗}{47220}
\saveTG{𫆺}{47220}
\saveTG{𦛶}{47220}
\saveTG{狥}{47220}
\saveTG{𦑽}{47220}
\saveTG{𢃊}{47220}
\saveTG{𨷎}{47220}
\saveTG{猢}{47220}
\saveTG{狪}{47220}
\saveTG{幱}{47220}
\saveTG{𤝩}{47221}
\saveTG{𤡒}{47221}
\saveTG{𪾆}{47221}
\saveTG{𪤲}{47221}
\saveTG{𫉼}{47221}
\saveTG{𪩴}{47221}
\saveTG{𢂅}{47221}
\saveTG{𤝉}{47221}
\saveTG{𦐢}{47221}
\saveTG{𦑱}{47221}
\saveTG{𢅰}{47221}
\saveTG{𪺿}{47221}
\saveTG{𪖒}{47221}
\saveTG{𤜩}{47222}
\saveTG{𤡌}{47222}
\saveTG{𢄪}{47222}
\saveTG{㺒}{47222}
\saveTG{㠴}{47222}
\saveTG{𤿇}{47222}
\saveTG{𢁞}{47222}
\saveTG{𤿉}{47223}
\saveTG{𤝧}{47223}
\saveTG{𢁕}{47223}
\saveTG{𤡲}{47224}
\saveTG{𢁶}{47224}
\saveTG{𢂓}{47226}
\saveTG{𢃖}{47226}
\saveTG{𢂶}{47226}
\saveTG{𢂁}{47226}
\saveTG{㡄}{47226}
\saveTG{𤟼}{47226}
\saveTG{䞒}{47226}
\saveTG{獡}{47227}
\saveTG{猧}{47227}
\saveTG{鴮}{47227}
\saveTG{郗}{47227}
\saveTG{鵗}{47227}
\saveTG{毊}{47227}
\saveTG{獝}{47227}
\saveTG{郁}{47227}
\saveTG{猾}{47227}
\saveTG{𨚤}{47227}
\saveTG{𨟗}{47227}
\saveTG{𪇓}{47227}
\saveTG{𨟔}{47227}
\saveTG{𨟆}{47227}
\saveTG{𨞫}{47227}
\saveTG{鶴}{47227}
\saveTG{𤝻}{47227}
\saveTG{𤞴}{47227}
\saveTG{𤜛}{47227}
\saveTG{𤜽}{47227}
\saveTG{𤜠}{47227}
\saveTG{𨛊}{47227}
\saveTG{𤜡}{47227}
\saveTG{𤞡}{47227}
\saveTG{𫛦}{47227}
\saveTG{㕁}{47227}
\saveTG{𨛇}{47227}
\saveTG{㡅}{47227}
\saveTG{𢁩}{47227}
\saveTG{𤿦}{47227}
\saveTG{𤿒}{47227}
\saveTG{𨛼}{47227}
\saveTG{𨙴}{47227}
\saveTG{𨜤}{47227}
\saveTG{𨜆}{47227}
\saveTG{𨚲}{47227}
\saveTG{𤡡}{47227}
\saveTG{𢃂}{47227}
\saveTG{𢅂}{47227}
\saveTG{𢁠}{47227}
\saveTG{𢄛}{47227}
\saveTG{𤝠}{47227}
\saveTG{𡰧}{47227}
\saveTG{𦠶}{47227}
\saveTG{𤟠}{47227}
\saveTG{𤟳}{47227}
\saveTG{鹳}{47227}
\saveTG{𨝵}{47227}
\saveTG{鸛}{47227}
\saveTG{郝}{47227}
\saveTG{鹤}{47227}
\saveTG{酄}{47227}
\saveTG{郀}{47227}
\saveTG{犸}{47227}
\saveTG{鸏}{47227}
\saveTG{鹲}{47227}
\saveTG{鶜}{47227}
\saveTG{鄸}{47227}
\saveTG{𨞂}{47227}
\saveTG{胬}{47227}
\saveTG{帑}{47227}
\saveTG{𪆉}{47227}
\saveTG{郩}{47227}
\saveTG{𤣀}{47227}
\saveTG{𤠸}{47227}
\saveTG{𨟤}{47227}
\saveTG{䳑}{47227}
\saveTG{𪄘}{47227}
\saveTG{𪅣}{47227}
\saveTG{䴃}{47227}
\saveTG{𪃢}{47227}
\saveTG{𪁌}{47227}
\saveTG{𩩬}{47227}
\saveTG{𢄓}{47227}
\saveTG{𢄦}{47227}
\saveTG{𫛇}{47227}
\saveTG{𫛆}{47227}
\saveTG{𤢧}{47227}
\saveTG{𤡕}{47227}
\saveTG{㡌}{47227}
\saveTG{𢃞}{47227}
\saveTG{𤠮}{47227}
\saveTG{𤜼}{47227}
\saveTG{𢂬}{47227}
\saveTG{𦜚}{47227}
\saveTG{𧤴}{47227}
\saveTG{𣫉}{47227}
\saveTG{𪵑}{47228}
\saveTG{㹼}{47229}
\saveTG{𤡦}{47229}
\saveTG{𧹷}{47231}
\saveTG{𤠘}{47231}
\saveTG{𤡸}{47232}
\saveTG{𧲇}{47232}
\saveTG{𧘽}{47232}
\saveTG{𧞺}{47232}
\saveTG{𢄐}{47232}
\saveTG{𧞹}{47232}
\saveTG{𢂻}{47232}
\saveTG{㺀}{47232}
\saveTG{狠}{47232}
\saveTG{猭}{47232}
\saveTG{𤢲}{47232}
\saveTG{𢅱}{47232}
\saveTG{𢅗}{47232}
\saveTG{𪻊}{47232}
\saveTG{𢄵}{47232}
\saveTG{𢃭}{47232}
\saveTG{𣼘}{47232}
\saveTG{𤣜}{47232}
\saveTG{𢄟}{47232}
\saveTG{㠽}{47233}
\saveTG{𧹝}{47233}
\saveTG{㹣}{47233}
\saveTG{𢅑}{47234}
\saveTG{㡝}{47235}
\saveTG{幈}{47241}
\saveTG{𤞶}{47241}
\saveTG{𣫄}{47241}
\saveTG{𤝇}{47242}
\saveTG{𤠈}{47244}
\saveTG{𤟻}{47244}
\saveTG{㹪}{47244}
\saveTG{𡳶}{47244}
\saveTG{𧹠}{47247}
\saveTG{𧹮}{47247}
\saveTG{䐨}{47247}
\saveTG{觳}{47247}
\saveTG{𣫔}{47247}
\saveTG{𣹬}{47247}
\saveTG{𥀎}{47247}
\saveTG{𧹲}{47247}
\saveTG{𣪛}{47247}
\saveTG{𢄌}{47247}
\saveTG{𣪩}{47247}
\saveTG{𤝈}{47247}
\saveTG{𤠍}{47247}
\saveTG{𤡘}{47247}
\saveTG{𤣌}{47247}
\saveTG{𣪒}{47247}
\saveTG{𪵈}{47247}
\saveTG{𣪟}{47247}
\saveTG{𠮁}{47247}
\saveTG{豰}{47247}
\saveTG{彀}{47247}
\saveTG{殻}{47247}
\saveTG{猳}{47247}
\saveTG{殸}{47247}
\saveTG{赧}{47247}
\saveTG{殼}{47247}
\saveTG{獀}{47247}
\saveTG{赮}{47247}
\saveTG{殽}{47247}
\saveTG{𡙈}{47247}
\saveTG{𧹜}{47247}
\saveTG{𪱙}{47247}
\saveTG{㠷}{47247}
\saveTG{𤜫}{47247}
\saveTG{𤜯}{47247}
\saveTG{𪺹}{47247}
\saveTG{𣫅}{47247}
\saveTG{𤟚}{47247}
\saveTG{𪵒}{47247}
\saveTG{𣫧}{47247}
\saveTG{𣪚}{47247}
\saveTG{𣪎}{47247}
\saveTG{𣫁}{47247}
\saveTG{㱿}{47247}
\saveTG{𣪗}{47247}
\saveTG{狦}{47250}
\saveTG{皹}{47252}
\saveTG{𤠵}{47252}
\saveTG{獬}{47252}
\saveTG{}{47254}
\saveTG{𦜁}{47254}
\saveTG{皹}{47256}
\saveTG{㡓}{47256}
\saveTG{𤝷}{47257}
\saveTG{𤝕}{47257}
\saveTG{狰}{47257}
\saveTG{𢂰}{47257}
\saveTG{幨}{47261}
\saveTG{𤟁}{47262}
\saveTG{㹦}{47262}
\saveTG{𤠑}{47262}
\saveTG{𢁾}{47262}
\saveTG{𢄭}{47262}
\saveTG{𢄜}{47263}
\saveTG{𤣃}{47263}
\saveTG{𦛃}{47264}
\saveTG{狢}{47264}
\saveTG{𤢊}{47264}
\saveTG{𢂽}{47265}
\saveTG{猸}{47267}
\saveTG{𢃼}{47267}
\saveTG{峱}{47272}
\saveTG{𤟎}{47272}
\saveTG{𤠀}{47272}
\saveTG{𢃗}{47272}
\saveTG{𤝳}{47274}
\saveTG{𤟅}{47277}
\saveTG{𪺸}{47277}
\saveTG{𦜿}{47277}
\saveTG{㡊}{47277}
\saveTG{幎}{47280}
\saveTG{猽}{47280}
\saveTG{㺞}{47281}
\saveTG{𤠳}{47281}
\saveTG{𣤳}{47281}
\saveTG{歡}{47282}
\saveTG{赥}{47282}
\saveTG{欷}{47282}
\saveTG{㰭}{47282}
\saveTG{𣤇}{47282}
\saveTG{𤡃}{47282}
\saveTG{𤡻}{47282}
\saveTG{𣣲}{47282}
\saveTG{𢅟}{47282}
\saveTG{𪴪}{47282}
\saveTG{㱋}{47282}
\saveTG{𢁧}{47282}
\saveTG{狈}{47282}
\saveTG{𤝆}{47282}
\saveTG{𤡋}{47282}
\saveTG{𤢔}{47282}
\saveTG{獭}{47282}
\saveTG{𤠣}{47284}
\saveTG{㺅}{47284}
\saveTG{帿}{47284}
\saveTG{猴}{47284}
\saveTG{猰}{47284}
\saveTG{𢅭}{47286}
\saveTG{獺}{47286}
\saveTG{𤍌}{47289}
\saveTG{𢅽}{47289}
\saveTG{狝}{47292}
\saveTG{𤡔}{47292}
\saveTG{猕}{47292}
\saveTG{𤢘}{47293}
\saveTG{𤞎}{47294}
\saveTG{猱}{47294}
\saveTG{㲆}{47302}
\saveTG{𨗛}{47307}
\saveTG{𩤄}{47312}
\saveTG{𦐼}{47321}
\saveTG{𩿹}{47323}
\saveTG{邿}{47327}
\saveTG{𪅏}{47327}
\saveTG{䲥}{47327}
\saveTG{鴑}{47327}
\saveTG{𪈧}{47327}
\saveTG{𨙯}{47327}
\saveTG{𪄅}{47327}
\saveTG{𪀮}{47327}
\saveTG{𪄎}{47327}
\saveTG{𪇗}{47327}
\saveTG{酀}{47327}
\saveTG{𨟂}{47327}
\saveTG{𩿤}{47327}
\saveTG{䴏}{47327}
\saveTG{𪁓}{47327}
\saveTG{𪆪}{47327}
\saveTG{𪇂}{47327}
\saveTG{𩼮}{47327}
\saveTG{鶦}{47327}
\saveTG{𪆑}{47327}
\saveTG{駑}{47327}
\saveTG{𪫴}{47331}
\saveTG{𢡶}{47332}
\saveTG{𤍂}{47332}
\saveTG{𢘰}{47332}
\saveTG{𢗧}{47332}
\saveTG{𤈔}{47332}
\saveTG{𡢰}{47334}
\saveTG{𢣯}{47334}
\saveTG{慤}{47334}
\saveTG{愨}{47334}
\saveTG{㲇}{47334}
\saveTG{𢢿}{47334}
\saveTG{𢡯}{47334}
\saveTG{𢞒}{47334}
\saveTG{𤍁}{47334}
\saveTG{𢟥}{47334}
\saveTG{怒}{47334}
\saveTG{恏}{47334}
\saveTG{𤋐}{47336}
\saveTG{𩽑}{47336}
\saveTG{𢡱}{47337}
\saveTG{𢦆}{47338}
\saveTG{𢢶}{47338}
\saveTG{𪭁}{47339}
\saveTG{𢢢}{47347}
\saveTG{𪃟}{47347}
\saveTG{鷇}{47347}
\saveTG{𣫠}{47347}
\saveTG{𡬻}{47348}
\saveTG{𩏻}{47357}
\saveTG{夊}{47400}
\saveTG{𦘃}{47401}
\saveTG{聲}{47401}
\saveTG{𢻈}{47402}
\saveTG{翅}{47402}
\saveTG{攳}{47404}
\saveTG{𡢁}{47404}
\saveTG{𪍢}{47404}
\saveTG{𡝓}{47404}
\saveTG{𡠧}{47404}
\saveTG{𣫜}{47406}
\saveTG{𠮑}{47407}
\saveTG{𡕞}{47407}
\saveTG{𠬚}{47407}
\saveTG{𡜢}{47407}
\saveTG{𡖁}{47407}
\saveTG{𪍱}{47407}
\saveTG{孥}{47407}
\saveTG{𡙕}{47410}
\saveTG{㚯}{47410}
\saveTG{姵}{47410}
\saveTG{㚨}{47410}
\saveTG{㜄}{47410}
\saveTG{𡟚}{47412}
\saveTG{𡝹}{47412}
\saveTG{𡤋}{47412}
\saveTG{姐}{47412}
\saveTG{𡤞}{47412}
\saveTG{}{47412}
\saveTG{𠭋}{47412}
\saveTG{妞}{47412}
\saveTG{娩}{47412}
\saveTG{姽}{47412}
\saveTG{䶯}{47412}
\saveTG{妮}{47412}
\saveTG{婗}{47412}
\saveTG{麭}{47412}
\saveTG{嫓}{47412}
\saveTG{爼}{47412}
\saveTG{𡡟}{47412}
\saveTG{𡟳}{47412}
\saveTG{婏}{47413}
\saveTG{媉}{47414}
\saveTG{娓}{47414}
\saveTG{𡠩}{47414}
\saveTG{嬥}{47415}
\saveTG{𠃙}{47417}
\saveTG{妑}{47417}
\saveTG{妃}{47417}
\saveTG{𡚫}{47417}
\saveTG{𪦏}{47417}
\saveTG{㚿}{47417}
\saveTG{𡜆}{47417}
\saveTG{𡤩}{47417}
\saveTG{𠙜}{47417}
\saveTG{𪦝}{47417}
\saveTG{𡤸}{47417}
\saveTG{𪌄}{47417}
\saveTG{𦫛}{47417}
\saveTG{䒎}{47417}
\saveTG{𡤹}{47417}
\saveTG{㜶}{47417}
\saveTG{𡤎}{47417}
\saveTG{𡟰}{47417}
\saveTG{㚮}{47417}
\saveTG{𡢘}{47417}
\saveTG{𡚱}{47417}
\saveTG{𪥪}{47417}
\saveTG{𡟄}{47418}
\saveTG{嫺}{47420}
\saveTG{娴}{47420}
\saveTG{妁}{47420}
\saveTG{媩}{47420}
\saveTG{孄}{47420}
\saveTG{𪦫}{47420}
\saveTG{麴}{47420}
\saveTG{姛}{47420}
\saveTG{婤}{47420}
\saveTG{朝}{47420}
\saveTG{婅}{47420}
\saveTG{㚹}{47420}
\saveTG{姰}{47420}
\saveTG{𡟧}{47420}
\saveTG{嫻}{47420}
\saveTG{𡙳}{47420}
\saveTG{姠}{47420}
\saveTG{姁}{47420}
\saveTG{𦐁}{47420}
\saveTG{𡡲}{47421}
\saveTG{𣎠}{47421}
\saveTG{䎐}{47421}
\saveTG{𪥾}{47421}
\saveTG{𡞇}{47421}
\saveTG{𪥵}{47421}
\saveTG{𦑨}{47421}
\saveTG{㚬}{47421}
\saveTG{𪍒}{47421}
\saveTG{嫪}{47422}
\saveTG{𡛁}{47422}
\saveTG{𪍆}{47422}
\saveTG{䴯}{47422}
\saveTG{妤}{47422}
\saveTG{𡟞}{47422}
\saveTG{𡟅}{47423}
\saveTG{𡚵}{47423}
\saveTG{𡠶}{47423}
\saveTG{𪤽}{47423}
\saveTG{𪌫}{47423}
\saveTG{𡣟}{47424}
\saveTG{𡛾}{47424}
\saveTG{𡢄}{47424}
\saveTG{𡠳}{47424}
\saveTG{𢱬}{47424}
\saveTG{𡢸}{47424}
\saveTG{㚸}{47426}
\saveTG{𡢃}{47426}
\saveTG{㛠}{47426}
\saveTG{𡝆}{47426}
\saveTG{𣪤}{47427}
\saveTG{𡤡}{47427}
\saveTG{𪌼}{47427}
\saveTG{𠡺}{47427}
\saveTG{𡡅}{47427}
\saveTG{𡠄}{47427}
\saveTG{𡟂}{47427}
\saveTG{𡞄}{47427}
\saveTG{𡤝}{47427}
\saveTG{𪦪}{47427}
\saveTG{𡣴}{47427}
\saveTG{𩾾}{47427}
\saveTG{𫚻}{47427}
\saveTG{𫛛}{47427}
\saveTG{𪁡}{47427}
\saveTG{𪂂}{47427}
\saveTG{𪆘}{47427}
\saveTG{䳊}{47427}
\saveTG{𪇡}{47427}
\saveTG{𩾔}{47427}
\saveTG{𪆆}{47427}
\saveTG{𡟦}{47427}
\saveTG{姼}{47427}
\saveTG{𡜙}{47427}
\saveTG{𪥶}{47427}
\saveTG{𨚴}{47427}
\saveTG{㛚}{47427}
\saveTG{𪍌}{47427}
\saveTG{𪌻}{47427}
\saveTG{𧤂}{47427}
\saveTG{𨛨}{47427}
\saveTG{𨝌}{47427}
\saveTG{𨙸}{47427}
\saveTG{𨞼}{47427}
\saveTG{𪄳}{47427}
\saveTG{媧}{47427}
\saveTG{婿}{47427}
\saveTG{嫋}{47427}
\saveTG{奶}{47427}
\saveTG{娜}{47427}
\saveTG{妈}{47427}
\saveTG{嫏}{47427}
\saveTG{婦}{47427}
\saveTG{孎}{47427}
\saveTG{媰}{47427}
\saveTG{鳷}{47427}
\saveTG{鵓}{47427}
\saveTG{郣}{47427}
\saveTG{鹁}{47427}
\saveTG{努}{47427}
\saveTG{鴱}{47427}
\saveTG{𡟩}{47427}
\saveTG{𪍛}{47427}
\saveTG{𡞭}{47427}
\saveTG{邚}{47427}
\saveTG{𪌀}{47427}
\saveTG{𡣽}{47427}
\saveTG{𡠟}{47427}
\saveTG{𡣄}{47427}
\saveTG{𪦭}{47427}
\saveTG{𪤐}{47427}
\saveTG{𡤖}{47427}
\saveTG{𡜃}{47428}
\saveTG{𪦄}{47431}
\saveTG{𡠁}{47431}
\saveTG{𡤥}{47431}
\saveTG{㜯}{47431}
\saveTG{𡜱}{47431}
\saveTG{㜊}{47432}
\saveTG{𡡆}{47432}
\saveTG{𡢤}{47432}
\saveTG{𡝘}{47432}
\saveTG{䴿}{47432}
\saveTG{𪍃}{47432}
\saveTG{𡟼}{47432}
\saveTG{𡝖}{47432}
\saveTG{𡠙}{47432}
\saveTG{𡝲}{47432}
\saveTG{𡟟}{47432}
\saveTG{𡟇}{47432}
\saveTG{嬝}{47432}
\saveTG{𡞷}{47433}
\saveTG{㚵}{47433}
\saveTG{𡠮}{47434}
\saveTG{𡠇}{47436}
\saveTG{𡠵}{47436}
\saveTG{𡟴}{47437}
\saveTG{𡤵}{47438}
\saveTG{姍}{47440}
\saveTG{娵}{47440}
\saveTG{奴}{47440}
\saveTG{姗}{47440}
\saveTG{婌}{47440}
\saveTG{𢍠}{47441}
\saveTG{㛛}{47441}
\saveTG{𡠸}{47441}
\saveTG{𡡘}{47441}
\saveTG{𢍇}{47441}
\saveTG{𡢂}{47441}
\saveTG{𡟭}{47441}
\saveTG{𡠬}{47442}
\saveTG{𡡬}{47442}
\saveTG{𨝜}{47442}
\saveTG{㜦}{47443}
\saveTG{䵁}{47443}
\saveTG{𪦌}{47444}
\saveTG{𡞉}{47444}
\saveTG{㜒}{47444}
\saveTG{𡟛}{47444}
\saveTG{𡛓}{47444}
\saveTG{𡞽}{47447}
\saveTG{㜌}{47447}
\saveTG{㝅}{47447}
\saveTG{𣫌}{47447}
\saveTG{𡦊}{47447}
\saveTG{㚺}{47447}
\saveTG{娺}{47447}
\saveTG{嫂}{47447}
\saveTG{姄}{47447}
\saveTG{婽}{47447}
\saveTG{報}{47447}
\saveTG{𪦓}{47447}
\saveTG{好}{47447}
\saveTG{𣫃}{47447}
\saveTG{𡠆}{47447}
\saveTG{𡢶}{47447}
\saveTG{𡞦}{47447}
\saveTG{𡚾}{47447}
\saveTG{㚫}{47447}
\saveTG{𪦋}{47447}
\saveTG{𪍥}{47447}
\saveTG{𪌓}{47447}
\saveTG{𪥫}{47447}
\saveTG{𡣝}{47448}
\saveTG{姆}{47450}
\saveTG{𡢡}{47451}
\saveTG{㜨}{47452}
\saveTG{媈}{47452}
\saveTG{䴼}{47453}
\saveTG{䴶}{47454}
\saveTG{𡜠}{47454}
\saveTG{㛔}{47454}
\saveTG{𡝾}{47454}
\saveTG{𡜜}{47455}
\saveTG{㚩}{47455}
\saveTG{婙}{47457}
\saveTG{𡤒}{47458}
\saveTG{𪌢}{47461}
\saveTG{㜬}{47461}
\saveTG{姳}{47462}
\saveTG{媹}{47462}
\saveTG{𡜴}{47462}
\saveTG{𪌕}{47462}
\saveTG{𪦞}{47462}
\saveTG{妱}{47462}
\saveTG{𡣁}{47464}
\saveTG{㛰}{47464}
\saveTG{婮}{47464}
\saveTG{𪌣}{47464}
\saveTG{𪌾}{47464}
\saveTG{𪍞}{47464}
\saveTG{𡞕}{47464}
\saveTG{媚}{47467}
\saveTG{𡝗}{47467}
\saveTG{𪌺}{47467}
\saveTG{𡟔}{47468}
\saveTG{妇}{47470}
\saveTG{𪌠}{47472}
\saveTG{𡝛}{47472}
\saveTG{𩍞}{47472}
\saveTG{𡳼}{47472}
\saveTG{𡞿}{47472}
\saveTG{𡜬}{47474}
\saveTG{𡛂}{47474}
\saveTG{𪌤}{47477}
\saveTG{㚶}{47477}
\saveTG{𡜥}{47477}
\saveTG{㛀}{47477}
\saveTG{𪍠}{47478}
\saveTG{奺}{47480}
\saveTG{嫇}{47480}
\saveTG{嬹}{47481}
\saveTG{𡢀}{47481}
\saveTG{𪦉}{47481}
\saveTG{嬩}{47481}
\saveTG{𣣋}{47482}
\saveTG{𣢩}{47482}
\saveTG{嫰}{47482}
\saveTG{㜛}{47482}
\saveTG{𡑟}{47482}
\saveTG{𪴳}{47482}
\saveTG{𤕟}{47482}
\saveTG{𪌒}{47482}
\saveTG{𡣞}{47482}
\saveTG{𣤨}{47482}
\saveTG{𡛫}{47482}
\saveTG{𦤣}{47484}
\saveTG{㜩}{47484}
\saveTG{𡞵}{47484}
\saveTG{𡟑}{47484}
\saveTG{㛟}{47484}
\saveTG{𡞥}{47484}
\saveTG{𪍏}{47486}
\saveTG{嬾}{47486}
\saveTG{媍}{47486}
\saveTG{𪦬}{47486}
\saveTG{𡛄}{47487}
\saveTG{娽}{47488}
\saveTG{𡤚}{47489}
\saveTG{妳}{47492}
\saveTG{㛆}{47494}
\saveTG{媃}{47494}
\saveTG{𡠊}{47494}
\saveTG{㛊}{47494}
\saveTG{𡠿}{47494}
\saveTG{𪍚}{47494}
\saveTG{娽}{47499}
\saveTG{𪍄}{47499}
\saveTG{𦎼}{47501}
\saveTG{拏}{47502}
\saveTG{撀}{47502}
\saveTG{𪺫}{47502}
\saveTG{𢵨}{47502}
\saveTG{𢭋}{47502}
\saveTG{𢴸}{47502}
\saveTG{𢴷}{47502}
\saveTG{𤛉}{47503}
\saveTG{𤛗}{47504}
\saveTG{𩌖}{47504}
\saveTG{𤛓}{47504}
\saveTG{𤚼}{47504}
\saveTG{䡰}{47506}
\saveTG{𩏜}{47506}
\saveTG{𩌥}{47506}
\saveTG{𩍤}{47506}
\saveTG{韋}{47506}
\saveTG{𤛖}{47507}
\saveTG{𩘚}{47510}
\saveTG{𩉜}{47510}
\saveTG{靵}{47510}
\saveTG{𩊛}{47512}
\saveTG{鞔}{47512}
\saveTG{靵}{47512}
\saveTG{鞄}{47512}
\saveTG{靻}{47512}
\saveTG{轻}{47512}
\saveTG{𩉶}{47512}
\saveTG{𩊁}{47512}
\saveTG{𩌟}{47512}
\saveTG{𩌑}{47512}
\saveTG{䪌}{47512}
\saveTG{䩯}{47512}
\saveTG{𩉹}{47512}
\saveTG{𩊨}{47513}
\saveTG{𩍪}{47515}
\saveTG{靶}{47517}
\saveTG{𩊋}{47517}
\saveTG{𩋓}{47517}
\saveTG{𩎘}{47517}
\saveTG{𩎝}{47517}
\saveTG{靮}{47520}
\saveTG{鞠}{47520}
\saveTG{鞫}{47520}
\saveTG{轫}{47520}
\saveTG{䩓}{47520}
\saveTG{𩊎}{47520}
\saveTG{𩊅}{47520}
\saveTG{𩉷}{47520}
\saveTG{𦑜}{47520}
\saveTG{𩉿}{47520}
\saveTG{𩋙}{47520}
\saveTG{𩊗}{47520}
\saveTG{䩗}{47520}
\saveTG{𩋒}{47520}
\saveTG{䩕}{47520}
\saveTG{䩴}{47520}
\saveTG{韌}{47520}
\saveTG{辋}{47520}
\saveTG{靭}{47520}
\saveTG{𩉛}{47520}
\saveTG{𤯐}{47520}
\saveTG{䪍}{47520}
\saveTG{𩋶}{47520}
\saveTG{𩍔}{47520}
\saveTG{𩋃}{47520}
\saveTG{𩋺}{47520}
\saveTG{䢁}{47521}
\saveTG{𦑻}{47521}
\saveTG{𩌭}{47522}
\saveTG{鷨}{47527}
\saveTG{𩊾}{47527}
\saveTG{𩊺}{47527}
\saveTG{𩊥}{47527}
\saveTG{䳬}{47527}
\saveTG{𩌗}{47527}
\saveTG{𩌠}{47527}
\saveTG{𩌄}{47527}
\saveTG{𩊧}{47527}
\saveTG{𩍆}{47527}
\saveTG{䩑}{47527}
\saveTG{鄿}{47527}
\saveTG{鞹}{47527}
\saveTG{郼}{47527}
\saveTG{䪕}{47529}
\saveTG{𩊕}{47529}
\saveTG{𩎤}{47532}
\saveTG{辗}{47532}
\saveTG{鞎}{47532}
\saveTG{𩋚}{47532}
\saveTG{𩊫}{47532}
\saveTG{𩌁}{47532}
\saveTG{𩎷}{47534}
\saveTG{𩎂}{47534}
\saveTG{䩼}{47535}
\saveTG{𩏦}{47536}
\saveTG{𩌝}{47537}
\saveTG{辀}{47540}
\saveTG{靱}{47540}
\saveTG{靫}{47543}
\saveTG{㨌}{47547}
\saveTG{𣪬}{47547}
\saveTG{𨐅}{47547}
\saveTG{𩋄}{47547}
\saveTG{𣪡}{47547}
\saveTG{𩎶}{47547}
\saveTG{𩏇}{47547}
\saveTG{𩎕}{47547}
\saveTG{𩌽}{47547}
\saveTG{𩋁}{47547}
\saveTG{𩋥}{47547}
\saveTG{辍}{47547}
\saveTG{毂}{47547}
\saveTG{轂}{47547}
\saveTG{靸}{47547}
\saveTG{𩌊}{47547}
\saveTG{䩔}{47547}
\saveTG{𩋦}{47547}
\saveTG{𫖋}{47547}
\saveTG{𣫂}{47547}
\saveTG{𤚲}{47547}
\saveTG{𦎯}{47547}
\saveTG{𩍝}{47552}
\saveTG{韗}{47552}
\saveTG{𩊩}{47554}
\saveTG{𪔏}{47554}
\saveTG{䩵}{47558}
\saveTG{䪜}{47561}
\saveTG{韂}{47561}
\saveTG{鞀}{47562}
\saveTG{𩎣}{47562}
\saveTG{轺}{47562}
\saveTG{𩊚}{47564}
\saveTG{辂}{47564}
\saveTG{𩋜}{47564}
\saveTG{𩎬}{47564}
\saveTG{𩌎}{47566}
\saveTG{𩋿}{47567}
\saveTG{𩏄}{47569}
\saveTG{𩋎}{47572}
\saveTG{𩉥}{47574}
\saveTG{𩎗}{47582}
\saveTG{𩌱}{47582}
\saveTG{𩉢}{47582}
\saveTG{软}{47582}
\saveTG{𩋴}{47584}
\saveTG{𩎑}{47589}
\saveTG{𩍧}{47592}
\saveTG{鞣}{47594}
\saveTG{𩊜}{47594}
\saveTG{𩌜}{47594}
\saveTG{䂋}{47594}
\saveTG{𩎫}{47594}
\saveTG{韖}{47594}
\saveTG{䩮}{47599}
\saveTG{謦}{47601}
\saveTG{𠮫}{47601}
\saveTG{𪐕}{47601}
\saveTG{韾}{47601}
\saveTG{𣊮}{47601}
\saveTG{𡄒}{47601}
\saveTG{𤰚}{47601}
\saveTG{𥖗}{47602}
\saveTG{磬}{47602}
\saveTG{㲈}{47602}
\saveTG{砮}{47602}
\saveTG{㿦}{47602}
\saveTG{𥔼}{47602}
\saveTG{𥗚}{47602}
\saveTG{𣫊}{47604}
\saveTG{𡄈}{47604}
\saveTG{𠹢}{47604}
\saveTG{𨢋}{47604}
\saveTG{𨢤}{47604}
\saveTG{𨡷}{47604}
\saveTG{𥅄}{47604}
\saveTG{𥊧}{47604}
\saveTG{𣍆}{47604}
\saveTG{𣪥}{47604}
\saveTG{𥊨}{47604}
\saveTG{𣉑}{47606}
\saveTG{馨}{47609}
\saveTG{𪡘}{47610}
\saveTG{𧀟}{47612}
\saveTG{𫉳}{47617}
\saveTG{𪕮}{47617}
\saveTG{䒐}{47617}
\saveTG{𠙏}{47617}
\saveTG{翿}{47620}
\saveTG{𠝖}{47620}
\saveTG{䎁}{47620}
\saveTG{𠳫}{47620}
\saveTG{翓}{47620}
\saveTG{胡}{47620}
\saveTG{𠄆}{47621}
\saveTG{𠠐}{47621}
\saveTG{𪢣}{47621}
\saveTG{𡕑}{47621}
\saveTG{𡕐}{47626}
\saveTG{鸪}{47627}
\saveTG{𪄖}{47627}
\saveTG{𪇘}{47627}
\saveTG{𨟒}{47627}
\saveTG{𨛢}{47627}
\saveTG{𡗀}{47627}
\saveTG{𨜅}{47627}
\saveTG{𨞪}{47627}
\saveTG{𨛳}{47627}
\saveTG{𨚞}{47627}
\saveTG{鹋}{47627}
\saveTG{都}{47627}
\saveTG{𪇜}{47627}
\saveTG{鶓}{47627}
\saveTG{鄀}{47627}
\saveTG{鵲}{47627}
\saveTG{鹊}{47627}
\saveTG{鵸}{47627}
\saveTG{鴶}{47627}
\saveTG{郆}{47627}
\saveTG{鶘}{47627}
\saveTG{鹕}{47627}
\saveTG{鴣}{47627}
\saveTG{𪆨}{47627}
\saveTG{殾}{47647}
\saveTG{𦒺}{47647}
\saveTG{嗀}{47647}
\saveTG{𠽠}{47647}
\saveTG{𣫐}{47647}
\saveTG{𥔳}{47647}
\saveTG{𣪇}{47647}
\saveTG{𣫈}{47647}
\saveTG{𣫀}{47647}
\saveTG{㿲}{47647}
\saveTG{嘏}{47647}
\saveTG{嗀}{47647}
\saveTG{𣪸}{47647}
\saveTG{瞉}{47647}
\saveTG{𡤟}{47662}
\saveTG{𣫪}{47668}
\saveTG{歖}{47682}
\saveTG{欯}{47682}
\saveTG{欹}{47682}
\saveTG{𪴰}{47682}
\saveTG{㱇}{47682}
\saveTG{𣣿}{47682}
\saveTG{𣤫}{47682}
\saveTG{𣣩}{47682}
\saveTG{𪜆}{47710}
\saveTG{𩛥}{47710}
\saveTG{𠫲}{47710}
\saveTG{𪠜}{47710}
\saveTG{𡙺}{47710}
\saveTG{𠤙}{47712}
\saveTG{𪠝}{47712}
\saveTG{𪓒}{47712}
\saveTG{乸}{47715}
\saveTG{𣭄}{47715}
\saveTG{㐇}{47717}
\saveTG{𠙆}{47717}
\saveTG{𪕷}{47717}
\saveTG{㐐}{47717}
\saveTG{𪚸}{47717}
\saveTG{翃}{47720}
\saveTG{𦐞}{47720}
\saveTG{却}{47720}
\saveTG{切}{47720}
\saveTG{刧}{47720}
\saveTG{刼}{47720}
\saveTG{䳓}{47727}
\saveTG{鵪}{47727}
\saveTG{𨝥}{47727}
\saveTG{郄}{47727}
\saveTG{邯}{47727}
\saveTG{夦}{47727}
\saveTG{䲺}{47727}
\saveTG{𨞛}{47727}
\saveTG{䣍}{47727}
\saveTG{𨚫}{47727}
\saveTG{𠫾}{47727}
\saveTG{𪆰}{47727}
\saveTG{𪃔}{47727}
\saveTG{𪂛}{47727}
\saveTG{𪈟}{47727}
\saveTG{𩿅}{47727}
\saveTG{𪄤}{47727}
\saveTG{𠬒}{47727}
\saveTG{鹌}{47727}
\saveTG{𣫘}{47727}
\saveTG{𢐙}{47727}
\saveTG{𪄽}{47727}
\saveTG{𩾕}{47727}
\saveTG{𩿆}{47727}
\saveTG{𫛑}{47727}
\saveTG{𪇄}{47727}
\saveTG{𩛗}{47732}
\saveTG{袃}{47732}
\saveTG{𨲅}{47732}
\saveTG{𧞡}{47732}
\saveTG{䭌}{47732}
\saveTG{𧞒}{47732}
\saveTG{𩛳}{47732}
\saveTG{𣪹}{47747}
\saveTG{䍍}{47747}
\saveTG{㲉}{47747}
\saveTG{𠫳}{47747}
\saveTG{𫅴}{47747}
\saveTG{叝}{47747}
\saveTG{𠬫}{47747}
\saveTG{𪕸}{47747}
\saveTG{㰥}{47762}
\saveTG{𦓉}{47764}
\saveTG{罄}{47772}
\saveTG{𪙈}{47772}
\saveTG{𪗭}{47772}
\saveTG{𡴴}{47772}
\saveTG{𡹹}{47772}
\saveTG{𤯒}{47774}
\saveTG{㰦}{47782}
\saveTG{𣣚}{47782}
\saveTG{𣤶}{47782}
\saveTG{𣢟}{47782}
\saveTG{𣤟}{47782}
\saveTG{𣢓}{47782}
\saveTG{歁}{47782}
\saveTG{起}{47801}
\saveTG{趯}{47801}
\saveTG{𧽡}{47801}
\saveTG{𧻜}{47801}
\saveTG{𧺋}{47801}
\saveTG{𧻡}{47801}
\saveTG{趘}{47801}
\saveTG{趄}{47801}
\saveTG{𧼜}{47802}
\saveTG{𧻌}{47802}
\saveTG{𧻮}{47802}
\saveTG{䟍}{47802}
\saveTG{𧽪}{47802}
\saveTG{𧽗}{47802}
\saveTG{趐}{47802}
\saveTG{趜}{47802}
\saveTG{趨}{47802}
\saveTG{䞻}{47802}
\saveTG{趍}{47802}
\saveTG{赹}{47802}
\saveTG{𧻹}{47802}
\saveTG{𧺐}{47802}
\saveTG{𫏕}{47802}
\saveTG{䞤}{47802}
\saveTG{𧻛}{47802}
\saveTG{𧺸}{47802}
\saveTG{𧼿}{47802}
\saveTG{𧺕}{47802}
\saveTG{䞼}{47802}
\saveTG{𧺤}{47802}
\saveTG{𧻘}{47802}
\saveTG{𧺥}{47802}
\saveTG{𧺖}{47802}
\saveTG{䞴}{47802}
\saveTG{𧽋}{47802}
\saveTG{𧽻}{47802}
\saveTG{𧽌}{47802}
\saveTG{𧼽}{47802}
\saveTG{𧻠}{47803}
\saveTG{𧼾}{47803}
\saveTG{𧻎}{47804}
\saveTG{𧽁}{47804}
\saveTG{𧼅}{47804}
\saveTG{𤠼}{47804}
\saveTG{𧺉}{47804}
\saveTG{𧼲}{47804}
\saveTG{𧾩}{47804}
\saveTG{𧾣}{47804}
\saveTG{𧺢}{47804}
\saveTG{䞵}{47804}
\saveTG{𧻖}{47804}
\saveTG{𧽏}{47804}
\saveTG{趣}{47804}
\saveTG{𧺙}{47804}
\saveTG{𫎷}{47805}
\saveTG{𧽖}{47806}
\saveTG{䟋}{47806}
\saveTG{𧼹}{47806}
\saveTG{䞦}{47806}
\saveTG{超}{47806}
\saveTG{䞫}{47806}
\saveTG{趦}{47806}
\saveTG{趋}{47807}
\saveTG{𧽎}{47807}
\saveTG{𡳣}{47807}
\saveTG{𧻫}{47807}
\saveTG{𧺽}{47807}
\saveTG{䞷}{47807}
\saveTG{趑}{47808}
\saveTG{赼}{47808}
\saveTG{𧾴}{47808}
\saveTG{𧺫}{47808}
\saveTG{𧾒}{47808}
\saveTG{𧺓}{47808}
\saveTG{𧾰}{47808}
\saveTG{𧷀}{47808}
\saveTG{𧼪}{47808}
\saveTG{𧼵}{47808}
\saveTG{𧾚}{47808}
\saveTG{𧻾}{47808}
\saveTG{𧽓}{47808}
\saveTG{趂}{47809}
\saveTG{趢}{47809}
\saveTG{趓}{47809}
\saveTG{𤎮}{47809}
\saveTG{𧻞}{47809}
\saveTG{㷫}{47809}
\saveTG{𧼇}{47809}
\saveTG{𧼢}{47809}
\saveTG{飆}{47810}
\saveTG{𩙁}{47810}
\saveTG{飙}{47810}
\saveTG{𩘕}{47810}
\saveTG{觌}{47812}
\saveTG{𪎻}{47813}
\saveTG{𦫡}{47817}
\saveTG{𦫜}{47817}
\saveTG{𪎸}{47817}
\saveTG{翸}{47820}
\saveTG{期}{47820}
\saveTG{𠠔}{47821}
\saveTG{𠠠}{47821}
\saveTG{𪥂}{47821}
\saveTG{䵏}{47822}
\saveTG{𧸷}{47823}
\saveTG{䴅}{47827}
\saveTG{𩿛}{47827}
\saveTG{䲪}{47827}
\saveTG{𪀬}{47827}
\saveTG{䳍}{47827}
\saveTG{𪆴}{47827}
\saveTG{𪏝}{47827}
\saveTG{𪏖}{47827}
\saveTG{𪇽}{47827}
\saveTG{𪇼}{47827}
\saveTG{𨟛}{47827}
\saveTG{鶧}{47827}
\saveTG{鷏}{47827}
\saveTG{鷞}{47827}
\saveTG{鄚}{47827}
\saveTG{鵊}{47827}
\saveTG{𪁾}{47827}
\saveTG{郟}{47827}
\saveTG{鷬}{47827}
\saveTG{𪃪}{47827}
\saveTG{𨟉}{47827}
\saveTG{𪄗}{47827}
\saveTG{𨞜}{47827}
\saveTG{𡖳}{47827}
\saveTG{𨝊}{47827}
\saveTG{𫛰}{47827}
\saveTG{𨟅}{47827}
\saveTG{䳢}{47827}
\saveTG{𪅾}{47827}
\saveTG{𪆁}{47827}
\saveTG{𪅐}{47827}
\saveTG{𪄿}{47827}
\saveTG{𨛺}{47827}
\saveTG{𪃇}{47827}
\saveTG{䣐}{47827}
\saveTG{𨝴}{47827}
\saveTG{䢼}{47827}
\saveTG{𪏠}{47832}
\saveTG{艱}{47832}
\saveTG{㷤}{47847}
\saveTG{㺉}{47847}
\saveTG{𣪨}{47847}
\saveTG{𣪳}{47847}
\saveTG{𧼒}{47847}
\saveTG{𪏙}{47847}
\saveTG{𪎳}{47847}
\saveTG{𪏕}{47856}
\saveTG{𤟴}{47856}
\saveTG{𠔴}{47864}
\saveTG{㯝}{47864}
\saveTG{𧷧}{47864}
\saveTG{𧾌}{47881}
\saveTG{𣤥}{47882}
\saveTG{歕}{47882}
\saveTG{欺}{47882}
\saveTG{歎}{47882}
\saveTG{𣤘}{47882}
\saveTG{𣤛}{47882}
\saveTG{㰰}{47882}
\saveTG{𣥁}{47882}
\saveTG{𤜹}{47882}
\saveTG{𣤯}{47882}
\saveTG{𧼍}{47889}
\saveTG{𪏧}{47899}
\saveTG{𤟘}{47899}
\saveTG{𠌊}{47901}
\saveTG{漀}{47902}
\saveTG{𥿚}{47903}
\saveTG{𦅬}{47903}
\saveTG{䋈}{47903}
\saveTG{𣖫}{47904}
\saveTG{𥡛}{47904}
\saveTG{𣐟}{47904}
\saveTG{𣔘}{47904}
\saveTG{杂}{47904}
\saveTG{𣑧}{47904}
\saveTG{𥻦}{47904}
\saveTG{䅽}{47904}
\saveTG{𣫙}{47904}
\saveTG{㯏}{47904}
\saveTG{𣝝}{47906}
\saveTG{𫊈}{47910}
\saveTG{杋}{47910}
\saveTG{楓}{47910}
\saveTG{㭄}{47910}
\saveTG{𣝲}{47910}
\saveTG{𣑲}{47910}
\saveTG{𩙂}{47910}
\saveTG{枫}{47910}
\saveTG{机}{47910}
\saveTG{𪲂}{47910}
\saveTG{𣎷}{47911}
\saveTG{𣛢}{47911}
\saveTG{𣐓}{47911}
\saveTG{𣚟}{47912}
\saveTG{𣚻}{47912}
\saveTG{𣑕}{47912}
\saveTG{柤}{47912}
\saveTG{𣕕}{47912}
\saveTG{楹}{47912}
\saveTG{杻}{47912}
\saveTG{𣜖}{47912}
\saveTG{𣒝}{47912}
\saveTG{梚}{47912}
\saveTG{棿}{47912}
\saveTG{𠣼}{47912}
\saveTG{𣗿}{47912}
\saveTG{桖}{47912}
\saveTG{枹}{47912}
\saveTG{𣟇}{47912}
\saveTG{査}{47912}
\saveTG{柅}{47912}
\saveTG{橻}{47912}
\saveTG{桅}{47912}
\saveTG{𣓶}{47912}
\saveTG{枧}{47912}
\saveTG{𠤃}{47912}
\saveTG{欃}{47913}
\saveTG{楃}{47914}
\saveTG{𡑚}{47914}
\saveTG{㭴}{47914}
\saveTG{柽}{47914}
\saveTG{極}{47914}
\saveTG{樫}{47914}
\saveTG{梶}{47914}
\saveTG{𣕝}{47914}
\saveTG{𣗕}{47914}
\saveTG{櫂}{47915}
\saveTG{𫃙}{47915}
\saveTG{𣑌}{47916}
\saveTG{𪳥}{47917}
\saveTG{梶}{47917}
\saveTG{𪲊}{47917}
\saveTG{𣐊}{47917}
\saveTG{𣏆}{47917}
\saveTG{𪴤}{47917}
\saveTG{杞}{47917}
\saveTG{栬}{47917}
\saveTG{𧡽}{47917}
\saveTG{𣜌}{47917}
\saveTG{𣞠}{47917}
\saveTG{𣏌}{47917}
\saveTG{櫷}{47917}
\saveTG{𣝊}{47917}
\saveTG{㭸}{47917}
\saveTG{𣗽}{47917}
\saveTG{𣘛}{47917}
\saveTG{𣙝}{47917}
\saveTG{杷}{47917}
\saveTG{㯡}{47917}
\saveTG{𣒤}{47917}
\saveTG{枹}{47917}
\saveTG{𣛺}{47920}
\saveTG{𣛬}{47920}
\saveTG{𣒗}{47920}
\saveTG{𣘪}{47920}
\saveTG{𪱽}{47920}
\saveTG{𠝼}{47920}
\saveTG{櫊}{47920}
\saveTG{}{47920}
\saveTG{构}{47920}
\saveTG{棡}{47920}
\saveTG{枸}{47920}
\saveTG{楜}{47920}
\saveTG{楖}{47920}
\saveTG{橌}{47920}
\saveTG{橺}{47920}
\saveTG{枃}{47920}
\saveTG{椈}{47920}
\saveTG{欄}{47920}
\saveTG{樃}{47920}
\saveTG{柳}{47920}
\saveTG{栁}{47920}
\saveTG{榈}{47920}
\saveTG{櫚}{47920}
\saveTG{椚}{47920}
\saveTG{棚}{47920}
\saveTG{杓}{47920}
\saveTG{枊}{47920}
\saveTG{㮝}{47920}
\saveTG{𣒂}{47920}
\saveTG{厀}{47920}
\saveTG{栒}{47920}
\saveTG{橍}{47920}
\saveTG{閷}{47920}
\saveTG{棢}{47920}
\saveTG{枂}{47920}
\saveTG{栩}{47920}
\saveTG{櫩}{47920}
\saveTG{杒}{47920}
\saveTG{桐}{47920}
\saveTG{朷}{47920}
\saveTG{柌}{47920}
\saveTG{椆}{47920}
\saveTG{𣖑}{47921}
\saveTG{𪲣}{47921}
\saveTG{𣐕}{47921}
\saveTG{𣗭}{47921}
\saveTG{㮶}{47921}
\saveTG{𣔂}{47921}
\saveTG{𣛨}{47921}
\saveTG{𣏊}{47921}
\saveTG{𣔒}{47921}
\saveTG{𣜝}{47921}
\saveTG{𪲚}{47921}
\saveTG{𣝰}{47921}
\saveTG{𣒥}{47922}
\saveTG{𣐆}{47922}
\saveTG{杼}{47922}
\saveTG{樛}{47922}
\saveTG{𣑈}{47922}
\saveTG{柕}{47922}
\saveTG{𣠼}{47923}
\saveTG{𣔢}{47923}
\saveTG{𣏐}{47923}
\saveTG{𣘵}{47923}
\saveTG{𪄐}{47923}
\saveTG{𣏯}{47924}
\saveTG{㭎}{47924}
\saveTG{𣛣}{47924}
\saveTG{𣘥}{47924}
\saveTG{𣜊}{47926}
\saveTG{㭵}{47926}
\saveTG{𣙎}{47926}
\saveTG{𣐒}{47926}
\saveTG{鷋}{47927}
\saveTG{桶}{47927}
\saveTG{桏}{47927}
\saveTG{榒}{47927}
\saveTG{梛}{47927}
\saveTG{樢}{47927}
\saveTG{杩}{47927}
\saveTG{桞}{47927}
\saveTG{鷯}{47927}
\saveTG{鹩}{47927}
\saveTG{鶆}{47927}
\saveTG{郲}{47927}
\saveTG{桷}{47927}
\saveTG{梮}{47927}
\saveTG{橘}{47927}
\saveTG{槨}{47927}
\saveTG{楇}{47927}
\saveTG{榾}{47927}
\saveTG{杛}{47927}
\saveTG{鸫}{47927}
\saveTG{槝}{47927}
\saveTG{郴}{47927}
\saveTG{梆}{47927}
\saveTG{𣓱}{47927}
\saveTG{𣜗}{47927}
\saveTG{㭨}{47927}
\saveTG{𣕋}{47927}
\saveTG{𣚑}{47927}
\saveTG{𣚚}{47927}
\saveTG{𣜔}{47927}
\saveTG{𣐢}{47927}
\saveTG{𣚹}{47927}
\saveTG{𣠗}{47927}
\saveTG{𣠸}{47927}
\saveTG{𣑤}{47927}
\saveTG{𣖣}{47927}
\saveTG{㮲}{47927}
\saveTG{𣓍}{47927}
\saveTG{𪳆}{47927}
\saveTG{𪉏}{47927}
\saveTG{𩿯}{47927}
\saveTG{䳵}{47927}
\saveTG{㮧}{47927}
\saveTG{𪄭}{47927}
\saveTG{𪇎}{47927}
\saveTG{𣖼}{47927}
\saveTG{𣓓}{47927}
\saveTG{𪳬}{47927}
\saveTG{𣝗}{47927}
\saveTG{𣎸}{47927}
\saveTG{𪈻}{47927}
\saveTG{𪀣}{47927}
\saveTG{𪇭}{47927}
\saveTG{𪃸}{47927}
\saveTG{𪃏}{47927}
\saveTG{𣒒}{47927}
\saveTG{榔}{47927}
\saveTG{𣜼}{47927}
\saveTG{䣛}{47927}
\saveTG{𣞽}{47927}
\saveTG{𪃥}{47927}
\saveTG{𣚔}{47927}
\saveTG{𣐨}{47927}
\saveTG{𣘙}{47927}
\saveTG{𣔆}{47927}
\saveTG{㭤}{47927}
\saveTG{𪁖}{47927}
\saveTG{𣐶}{47927}
\saveTG{𪴂}{47927}
\saveTG{𣗐}{47927}
\saveTG{𪲐}{47927}
\saveTG{𣛭}{47927}
\saveTG{𣓹}{47927}
\saveTG{㭁}{47927}
\saveTG{㮋}{47927}
\saveTG{䢶}{47927}
\saveTG{𨞢}{47927}
\saveTG{𪉌}{47927}
\saveTG{䣇}{47927}
\saveTG{𨝷}{47927}
\saveTG{𪥑}{47927}
\saveTG{𨝛}{47927}
\saveTG{𨝒}{47927}
\saveTG{欘}{47927}
\saveTG{栘}{47927}
\saveTG{椰}{47927}
\saveTG{杨}{47927}
\saveTG{楈}{47927}
\saveTG{樇}{47927}
\saveTG{榍}{47927}
\saveTG{𣚉}{47928}
\saveTG{㯗}{47929}
\saveTG{𨶓}{47929}
\saveTG{梞}{47931}
\saveTG{𣜠}{47931}
\saveTG{㮻}{47931}
\saveTG{𧎹}{47931}
\saveTG{𪳜}{47931}
\saveTG{𣖗}{47931}
\saveTG{𪲨}{47932}
\saveTG{𣗔}{47932}
\saveTG{櫲}{47932}
\saveTG{𣔲}{47932}
\saveTG{𣙧}{47932}
\saveTG{𣘴}{47932}
\saveTG{榐}{47932}
\saveTG{椽}{47932}
\saveTG{楤}{47932}
\saveTG{根}{47932}
\saveTG{梕}{47932}
\saveTG{樋}{47932}
\saveTG{橡}{47932}
\saveTG{檭}{47932}
\saveTG{橼}{47932}
\saveTG{櫞}{47932}
\saveTG{檛}{47932}
\saveTG{𣡃}{47932}
\saveTG{𣛞}{47932}
\saveTG{𢠗}{47932}
\saveTG{柊}{47933}
\saveTG{𪳖}{47935}
\saveTG{槰}{47935}
\saveTG{欚}{47936}
\saveTG{槌}{47937}
\saveTG{𣟙}{47938}
\saveTG{枬}{47940}
\saveTG{权}{47940}
\saveTG{棷}{47940}
\saveTG{椒}{47940}
\saveTG{柵}{47940}
\saveTG{栅}{47940}
\saveTG{枠}{47941}
\saveTG{𣕍}{47942}
\saveTG{㯍}{47943}
\saveTG{杈}{47943}
\saveTG{𣟪}{47943}
\saveTG{𣖕}{47944}
\saveTG{𣑪}{47944}
\saveTG{𣟗}{47944}
\saveTG{𣖒}{47944}
\saveTG{𣏵}{47944}
\saveTG{𣑮}{47944}
\saveTG{㰘}{47944}
\saveTG{𣐠}{47944}
\saveTG{樱}{47944}
\saveTG{樳}{47946}
\saveTG{殺}{47947}
\saveTG{梫}{47947}
\saveTG{椵}{47947}
\saveTG{极}{47947}
\saveTG{檓}{47947}
\saveTG{縠}{47947}
\saveTG{糓}{47947}
\saveTG{穀}{47947}
\saveTG{𣔖}{47947}
\saveTG{榖}{47947}
\saveTG{棴}{47947}
\saveTG{杸}{47947}
\saveTG{椴}{47947}
\saveTG{㮴}{47947}
\saveTG{𣐔}{47947}
\saveTG{𣞭}{47947}
\saveTG{𣑚}{47947}
\saveTG{棳}{47947}
\saveTG{𣖽}{47947}
\saveTG{㰒}{47947}
\saveTG{𣫎}{47947}
\saveTG{㮽}{47947}
\saveTG{𣫇}{47947}
\saveTG{𣫗}{47947}
\saveTG{𣫓}{47947}
\saveTG{𣪺}{47947}
\saveTG{𣚯}{47947}
\saveTG{𣪋}{47947}
\saveTG{𣒃}{47947}
\saveTG{𣔈}{47947}
\saveTG{樧}{47947}
\saveTG{杍}{47947}
\saveTG{樼}{47947}
\saveTG{桪}{47947}
\saveTG{𣐡}{47947}
\saveTG{榝}{47947}
\saveTG{𣏘}{47947}
\saveTG{𣝦}{47948}
\saveTG{𥼆}{47949}
\saveTG{栂}{47950}
\saveTG{枏}{47950}
\saveTG{𣜘}{47951}
\saveTG{㮮}{47952}
\saveTG{楎}{47952}
\saveTG{欅}{47952}
\saveTG{檞}{47952}
\saveTG{栙}{47954}
\saveTG{桻}{47954}
\saveTG{栂}{47954}
\saveTG{𪲋}{47954}
\saveTG{𣑭}{47955}
\saveTG{𣡴}{47956}
\saveTG{棦}{47957}
\saveTG{櫸}{47958}
\saveTG{𣟱}{47958}
\saveTG{樨}{47959}
\saveTG{𣡞}{47961}
\saveTG{㭣}{47961}
\saveTG{檐}{47961}
\saveTG{橹}{47961}
\saveTG{槢}{47962}
\saveTG{𣔺}{47962}
\saveTG{榴}{47962}
\saveTG{柖}{47962}
\saveTG{𣓗}{47962}
\saveTG{櫓}{47963}
\saveTG{㮭}{47963}
\saveTG{格}{47964}
\saveTG{𣛒}{47964}
\saveTG{椐}{47964}
\saveTG{𦃆}{47964}
\saveTG{𣓂}{47964}
\saveTG{𪳅}{47964}
\saveTG{𣗛}{47964}
\saveTG{𪴕}{47966}
\saveTG{桾}{47967}
\saveTG{楣}{47967}
\saveTG{㮞}{47968}
\saveTG{桕}{47970}
\saveTG{㭾}{47972}
\saveTG{㮀}{47972}
\saveTG{桘}{47977}
\saveTG{㭒}{47977}
\saveTG{榋}{47977}
\saveTG{杦}{47980}
\saveTG{榠}{47980}
\saveTG{椇}{47981}
\saveTG{㯢}{47981}
\saveTG{㯘}{47982}
\saveTG{𣤷}{47982}
\saveTG{𣣙}{47982}
\saveTG{𣞔}{47982}
\saveTG{𣟬}{47982}
\saveTG{𣒱}{47982}
\saveTG{𣝅}{47982}
\saveTG{㭥}{47982}
\saveTG{𪱷}{47982}
\saveTG{𣔙}{47982}
\saveTG{栨}{47982}
\saveTG{款}{47982}
\saveTG{歀}{47982}
\saveTG{杴}{47982}
\saveTG{樕}{47982}
\saveTG{𣐌}{47982}
\saveTG{𣤂}{47982}
\saveTG{㱈}{47982}
\saveTG{𣖬}{47982}
\saveTG{𪴃}{47984}
\saveTG{㮢}{47984}
\saveTG{楔}{47984}
\saveTG{𪲰}{47984}
\saveTG{橮}{47986}
\saveTG{㯯}{47986}
\saveTG{櫴}{47986}
\saveTG{㯧}{47986}
\saveTG{樌}{47986}
\saveTG{𣏧}{47987}
\saveTG{楰}{47987}
\saveTG{𣠪}{47989}
\saveTG{𣡿}{47989}
\saveTG{𣠙}{47989}
\saveTG{𣠜}{47989}
\saveTG{棂}{47989}
\saveTG{𣘤}{47991}
\saveTG{𪱾}{47992}
\saveTG{𣝌}{47993}
\saveTG{𣚎}{47993}
\saveTG{𣝠}{47993}
\saveTG{𣚃}{47993}
\saveTG{𣒺}{47994}
\saveTG{𣟍}{47994}
\saveTG{𣜩}{47994}
\saveTG{棎}{47994}
\saveTG{桗}{47994}
\saveTG{楺}{47994}
\saveTG{槡}{47994}
\saveTG{樤}{47994}
\saveTG{㮪}{47994}
\saveTG{𣑫}{47994}
\saveTG{𣘖}{47994}
\saveTG{𣞰}{47994}
\saveTG{𪏷}{47998}
\saveTG{㰀}{47999}
\saveTG{椂}{47999}
\saveTG{𣝵}{47999}
\saveTG{𪭢}{48008}
\saveTG{𡯨}{48011}
\saveTG{尷}{48011}
\saveTG{尴}{48011}
\saveTG{䃉}{48011}
\saveTG{㞉}{48011}
\saveTG{𠃩}{48012}
\saveTG{𠆷}{48012}
\saveTG{𡯔}{48012}
\saveTG{𡯕}{48012}
\saveTG{尬}{48012}
\saveTG{尲}{48013}
\saveTG{𡯾}{48016}
\saveTG{馗}{48016}
\saveTG{𠄁}{48016}
\saveTG{𢃓}{48031}
\saveTG{𢼈}{48040}
\saveTG{𠧂}{48040}
\saveTG{𨎳}{48054}
\saveTG{𢲹}{48063}
\saveTG{}{48100}
\saveTG{圦}{48100}
\saveTG{𡉇}{48100}
\saveTG{䀇}{48102}
\saveTG{䀋}{48102}
\saveTG{𥂁}{48102}
\saveTG{盩}{48102}
\saveTG{𤨣}{48104}
\saveTG{𡑃}{48104}
\saveTG{𨭻}{48109}
\saveTG{𨯍}{48109}
\saveTG{𡌶}{48112}
\saveTG{𡔑}{48112}
\saveTG{𡏦}{48112}
\saveTG{𡓟}{48112}
\saveTG{塩}{48112}
\saveTG{壏}{48112}
\saveTG{塧}{48112}
\saveTG{𡎛}{48112}
\saveTG{𪉹}{48112}
\saveTG{𪤴}{48112}
\saveTG{𪤒}{48112}
\saveTG{𡊇}{48112}
\saveTG{}{48112}
\saveTG{𡋄}{48114}
\saveTG{㙂}{48117}
\saveTG{𧏙}{48117}
\saveTG{𡍻}{48117}
\saveTG{圪}{48117}
\saveTG{墘}{48117}
\saveTG{𡉛}{48117}
\saveTG{𡌖}{48117}
\saveTG{𡒞}{48117}
\saveTG{圿}{48120}
\saveTG{堬}{48121}
\saveTG{𡒺}{48127}
\saveTG{𤅰}{48127}
\saveTG{𡋁}{48127}
\saveTG{𨯹}{48127}
\saveTG{㘯}{48127}
\saveTG{𫛿}{48127}
\saveTG{𡓞}{48127}
\saveTG{𡌡}{48127}
\saveTG{塲}{48127}
\saveTG{坋}{48127}
\saveTG{埨}{48127}
\saveTG{坅}{48127}
\saveTG{塕}{48127}
\saveTG{𡏆}{48128}
\saveTG{𡋗}{48131}
\saveTG{墲}{48131}
\saveTG{坽}{48132}
\saveTG{𩞎}{48132}
\saveTG{𡒶}{48132}
\saveTG{埝}{48132}
\saveTG{𡑖}{48132}
\saveTG{𡌧}{48133}
\saveTG{𪤣}{48133}
\saveTG{𡑞}{48133}
\saveTG{𡏤}{48134}
\saveTG{螫}{48136}
\saveTG{𧌻}{48136}
\saveTG{蟼}{48136}
\saveTG{𧰋}{48137}
\saveTG{𡏊}{48137}
\saveTG{坆}{48140}
\saveTG{𪤓}{48140}
\saveTG{𡑒}{48140}
\saveTG{𡏼}{48140}
\saveTG{墩}{48140}
\saveTG{墽}{48140}
\saveTG{𡉦}{48140}
\saveTG{垪}{48141}
\saveTG{𡔄}{48141}
\saveTG{㙏}{48142}
\saveTG{𡐣}{48142}
\saveTG{𡑻}{48142}
\saveTG{𣻜}{48143}
\saveTG{𡌚}{48144}
\saveTG{𡓫}{48144}
\saveTG{墫}{48146}
\saveTG{壿}{48146}
\saveTG{垟}{48151}
\saveTG{𡑱}{48151}
\saveTG{𡋂}{48152}
\saveTG{㙚}{48152}
\saveTG{㙿}{48153}
\saveTG{}{48156}
\saveTG{㙁}{48157}
\saveTG{垥}{48161}
\saveTG{墡}{48161}
\saveTG{𡐭}{48161}
\saveTG{𪣺}{48161}
\saveTG{𡌢}{48162}
\saveTG{墖}{48166}
\saveTG{増}{48166}
\saveTG{增}{48166}
\saveTG{𡐈}{48166}
\saveTG{𪤇}{48167}
\saveTG{㙮}{48168}
\saveTG{﨏}{48168}
\saveTG{𡌜}{48172}
\saveTG{𡌙}{48177}
\saveTG{𡒬}{48181}
\saveTG{㙡}{48182}
\saveTG{𪤚}{48184}
\saveTG{𡎤}{48184}
\saveTG{垁}{48184}
\saveTG{𡑯}{48186}
\saveTG{𡊒}{48190}
\saveTG{𦯴}{48200}
\saveTG{𢂃}{48211}
\saveTG{帨}{48212}
\saveTG{猐}{48212}
\saveTG{獈}{48212}
\saveTG{狏}{48212}
\saveTG{兝}{48212}
\saveTG{𤠝}{48212}
\saveTG{𢅡}{48212}
\saveTG{㺝}{48212}
\saveTG{}{48212}
\saveTG{獇}{48213}
\saveTG{𠓎}{48214}
\saveTG{𢂘}{48214}
\saveTG{𤟽}{48217}
\saveTG{𣱬}{48217}
\saveTG{犵}{48217}
\saveTG{𠄊}{48217}
\saveTG{𧹵}{48217}
\saveTG{𠄃}{48217}
\saveTG{𠚮}{48217}
\saveTG{㡃}{48217}
\saveTG{𤣟}{48217}
\saveTG{𤞄}{48217}
\saveTG{𨩧}{48219}
\saveTG{猃}{48219}
\saveTG{㺄}{48220}
\saveTG{㠹}{48220}
\saveTG{㡏}{48220}
\saveTG{㡐}{48221}
\saveTG{𤞆}{48221}
\saveTG{𢁮}{48227}
\saveTG{𤜰}{48227}
\saveTG{𢃪}{48227}
\saveTG{𢅷}{48227}
\saveTG{𢅹}{48227}
\saveTG{𡣤}{48227}
\saveTG{幯}{48227}
\saveTG{帉}{48227}
\saveTG{𧹪}{48227}
\saveTG{㺋}{48227}
\saveTG{𪯗}{48227}
\saveTG{𩱏}{48227}
\saveTG{𤢌}{48227}
\saveTG{𢁷}{48231}
\saveTG{幠}{48231}
\saveTG{𤣞}{48231}
\saveTG{𤝅}{48231}
\saveTG{㺊}{48231}
\saveTG{狑}{48232}
\saveTG{狯}{48232}
\saveTG{𤡀}{48232}
\saveTG{𢅕}{48233}
\saveTG{𢃧}{48233}
\saveTG{㡘}{48237}
\saveTG{𢆁}{48237}
\saveTG{㺌}{48237}
\saveTG{𤠦}{48240}
\saveTG{㺖}{48240}
\saveTG{𤡈}{48240}
\saveTG{𤢉}{48240}
\saveTG{𢿱}{48240}
\saveTG{𪾇}{48240}
\saveTG{獓}{48240}
\saveTG{獙}{48240}
\saveTG{赦}{48240}
\saveTG{獤}{48240}
\saveTG{獥}{48240}
\saveTG{散}{48240}
\saveTG{𢼘}{48240}
\saveTG{𢽔}{48240}
\saveTG{𣁒}{48240}
\saveTG{𢿯}{48240}
\saveTG{𢿲}{48240}
\saveTG{𢽳}{48240}
\saveTG{𡏯}{48240}
\saveTG{𢻾}{48240}
\saveTG{𢄻}{48240}
\saveTG{𢄯}{48240}
\saveTG{𤡏}{48240}
\saveTG{𤡳}{48240}
\saveTG{𤢄}{48240}
\saveTG{𤟸}{48240}
\saveTG{𤡄}{48240}
\saveTG{𤝪}{48241}
\saveTG{帡}{48241}
\saveTG{𤟲}{48244}
\saveTG{𤝴}{48244}
\saveTG{𫆸}{48246}
\saveTG{㬼}{48247}
\saveTG{𢆃}{48247}
\saveTG{𤟱}{48247}
\saveTG{𤣙}{48248}
\saveTG{𣀸}{48249}
\saveTG{𦍕}{48251}
\saveTG{𢆥}{48252}
\saveTG{𢅵}{48253}
\saveTG{𢅶}{48253}
\saveTG{𣼿}{48254}
\saveTG{𤞦}{48254}
\saveTG{𪩷}{48256}
\saveTG{𤢇}{48257}
\saveTG{𢂳}{48257}
\saveTG{帢}{48261}
\saveTG{𤝰}{48261}
\saveTG{𩠗}{48262}
\saveTG{𤠁}{48262}
\saveTG{𧹣}{48262}
\saveTG{𤞻}{48262}
\saveTG{猞}{48264}
\saveTG{猶}{48264}
\saveTG{𢅋}{48266}
\saveTG{獪}{48266}
\saveTG{獊}{48267}
\saveTG{𢂲}{48268}
\saveTG{𤞞}{48268}
\saveTG{𤢋}{48274}
\saveTG{𤜞}{48280}
\saveTG{}{48282}
\saveTG{𤡆}{48282}
\saveTG{𤢮}{48284}
\saveTG{𪻄}{48284}
\saveTG{𢅐}{48286}
\saveTG{𧹅}{48286}
\saveTG{獫}{48286}
\saveTG{𤝝}{48290}
\saveTG{𣘒}{48294}
\saveTG{狳}{48294}
\saveTG{𪁿}{48327}
\saveTG{𪂜}{48327}
\saveTG{驚}{48327}
\saveTG{𪇟}{48327}
\saveTG{憼}{48334}
\saveTG{慦}{48334}
\saveTG{憗}{48334}
\saveTG{𢟻}{48334}
\saveTG{𤎦}{48334}
\saveTG{𢠛}{48334}
\saveTG{𩻯}{48336}
\saveTG{𩸝}{48336}
\saveTG{𤍻}{48337}
\saveTG{𢠜}{48338}
\saveTG{𡅐}{48361}
\saveTG{倝}{48400}
\saveTG{姒}{48400}
\saveTG{娰}{48400}
\saveTG{𠐱}{48400}
\saveTG{𡚭}{48400}
\saveTG{𤕢}{48404}
\saveTG{䵅}{48407}
\saveTG{𠭷}{48407}
\saveTG{𪌟}{48411}
\saveTG{妰}{48411}
\saveTG{嫅}{48412}
\saveTG{孂}{48412}
\saveTG{𡟆}{48412}
\saveTG{㜋}{48412}
\saveTG{䴾}{48412}
\saveTG{𡡇}{48412}
\saveTG{娧}{48412}
\saveTG{㜮}{48412}
\saveTG{𩙶}{48413}
\saveTG{姾}{48414}
\saveTG{㛗}{48414}
\saveTG{𪌴}{48414}
\saveTG{雗}{48415}
\saveTG{𪟴}{48417}
\saveTG{𪌇}{48417}
\saveTG{𪍙}{48417}
\saveTG{𡤱}{48417}
\saveTG{𡠎}{48417}
\saveTG{㛨}{48417}
\saveTG{㲦}{48417}
\saveTG{乾}{48417}
\saveTG{亁}{48417}
\saveTG{麧}{48417}
\saveTG{𡣛}{48417}
\saveTG{𢆨}{48418}
\saveTG{𡢒}{48418}
\saveTG{}{48419}
\saveTG{𪟺}{48420}
\saveTG{𪍍}{48420}
\saveTG{妎}{48420}
\saveTG{𪥥}{48420}
\saveTG{𡛐}{48421}
\saveTG{媊}{48421}
\saveTG{媮}{48421}
\saveTG{𢒨}{48422}
\saveTG{𡛧}{48422}
\saveTG{𢺺}{48427}
\saveTG{𣎍}{48427}
\saveTG{䮧}{48427}
\saveTG{𡠨}{48427}
\saveTG{𪦕}{48427}
\saveTG{翰}{48427}
\saveTG{鶾}{48427}
\saveTG{妗}{48427}
\saveTG{婨}{48427}
\saveTG{嬆}{48427}
\saveTG{妢}{48427}
\saveTG{娣}{48427}
\saveTG{𦒋}{48427}
\saveTG{𪦴}{48427}
\saveTG{𡟸}{48427}
\saveTG{𡡽}{48427}
\saveTG{𠢕}{48427}
\saveTG{𠢇}{48427}
\saveTG{㛋}{48430}
\saveTG{𪌥}{48431}
\saveTG{𡞰}{48431}
\saveTG{𧹳}{48431}
\saveTG{嫵}{48431}
\saveTG{妐}{48432}
\saveTG{𡜅}{48432}
\saveTG{婾}{48432}
\saveTG{𡟝}{48432}
\saveTG{𡡦}{48432}
\saveTG{姈}{48432}
\saveTG{𡡂}{48432}
\saveTG{嬘}{48433}
\saveTG{嬨}{48433}
\saveTG{𪦟}{48433}
\saveTG{𫏮}{48434}
\saveTG{螒}{48436}
\saveTG{𩹼}{48436}
\saveTG{嫌}{48437}
\saveTG{𪌿}{48437}
\saveTG{教}{48440}
\saveTG{嬓}{48440}
\saveTG{敎}{48440}
\saveTG{𪪉}{48440}
\saveTG{嬍}{48440}
\saveTG{嫩}{48440}
\saveTG{𪌑}{48440}
\saveTG{𪍘}{48440}
\saveTG{㪍}{48440}
\saveTG{𡠾}{48440}
\saveTG{𪦚}{48440}
\saveTG{媺}{48440}
\saveTG{𪦐}{48440}
\saveTG{𢼂}{48440}
\saveTG{𢼫}{48440}
\saveTG{𡢝}{48440}
\saveTG{𡡹}{48440}
\saveTG{𡛇}{48440}
\saveTG{㜫}{48440}
\saveTG{𡝒}{48440}
\saveTG{𪦩}{48440}
\saveTG{斡}{48440}
\saveTG{㜟}{48440}
\saveTG{𡠯}{48440}
\saveTG{𡝺}{48440}
\saveTG{㜜}{48440}
\saveTG{𡣷}{48440}
\saveTG{𡟹}{48440}
\saveTG{𪯎}{48440}
\saveTG{𡟁}{48440}
\saveTG{𪍮}{48440}
\saveTG{𢾳}{48440}
\saveTG{𡡵}{48440}
\saveTG{幹}{48441}
\saveTG{姘}{48441}
\saveTG{𡞪}{48442}
\saveTG{𡠞}{48443}
\saveTG{䴵}{48444}
\saveTG{𡟜}{48444}
\saveTG{𡞘}{48444}
\saveTG{㢣}{48444}
\saveTG{媕}{48446}
\saveTG{𦩻}{48447}
\saveTG{𥀐}{48447}
\saveTG{姩}{48450}
\saveTG{𡤷}{48451}
\saveTG{嬟}{48453}
\saveTG{𢧢}{48453}
\saveTG{𡠫}{48454}
\saveTG{婵}{48456}
\saveTG{𩏑}{48457}
\saveTG{𡣆}{48457}
\saveTG{娒}{48457}
\saveTG{𥉏}{48460}
\saveTG{𣉙}{48460}
\saveTG{𡡝}{48461}
\saveTG{𪍶}{48461}
\saveTG{姶}{48461}
\saveTG{嫸}{48461}
\saveTG{𩠙}{48462}
\saveTG{𡞝}{48462}
\saveTG{娢}{48462}
\saveTG{𡣊}{48463}
\saveTG{𡣔}{48463}
\saveTG{𨢈}{48464}
\saveTG{𪍑}{48464}
\saveTG{㜝}{48464}
\saveTG{媨}{48464}
\saveTG{𡞆}{48464}
\saveTG{嬒}{48466}
\saveTG{𡡑}{48466}
\saveTG{𪦔}{48467}
\saveTG{㭲}{48468}
\saveTG{𡜊}{48472}
\saveTG{嫙}{48481}
\saveTG{𡟐}{48482}
\saveTG{㜡}{48482}
\saveTG{媄}{48484}
\saveTG{嬐}{48486}
\saveTG{𤌹}{48489}
\saveTG{㚷}{48490}
\saveTG{𦾠}{48491}
\saveTG{𡝙}{48491}
\saveTG{𡣵}{48493}
\saveTG{𡌘}{48494}
\saveTG{𡝐}{48494}
\saveTG{𡠏}{48494}
\saveTG{𡢥}{48494}
\saveTG{榦}{48494}
\saveTG{𠏉}{48494}
\saveTG{擎}{48502}
\saveTG{𩊀}{48502}
\saveTG{𩉻}{48512}
\saveTG{𩏛}{48512}
\saveTG{轮}{48512}
\saveTG{𩌢}{48512}
\saveTG{𩊭}{48512}
\saveTG{辁}{48514}
\saveTG{𩉳}{48517}
\saveTG{䩐}{48517}
\saveTG{䢀}{48517}
\saveTG{𩎰}{48517}
\saveTG{䩱}{48520}
\saveTG{𩉡}{48520}
\saveTG{𩋳}{48521}
\saveTG{输}{48521}
\saveTG{轸}{48522}
\saveTG{䩺}{48527}
\saveTG{𩉵}{48527}
\saveTG{𩌵}{48527}
\saveTG{𩎖}{48527}
\saveTG{𩍲}{48527}
\saveTG{𩌒}{48527}
\saveTG{靲}{48527}
\saveTG{𩉭}{48530}
\saveTG{𩋏}{48530}
\saveTG{𩍚}{48532}
\saveTG{𩍒}{48540}
\saveTG{𩍉}{48540}
\saveTG{𩉩}{48540}
\saveTG{䪃}{48540}
\saveTG{辙}{48540}
\saveTG{𩋩}{48540}
\saveTG{𩊖}{48541}
\saveTG{鞥}{48546}
\saveTG{𩋟}{48547}
\saveTG{𩊑}{48551}
\saveTG{𩋱}{48556}
\saveTG{𩊱}{48557}
\saveTG{鞈}{48561}
\saveTG{𩍈}{48561}
\saveTG{韐}{48561}
\saveTG{鞧}{48564}
\saveTG{}{48564}
\saveTG{𩌼}{48566}
\saveTG{𩌞}{48566}
\saveTG{𢾝}{48580}
\saveTG{𩍮}{48581}
\saveTG{𩋼}{48584}
\saveTG{𩏩}{48586}
\saveTG{䪖}{48587}
\saveTG{𩌣}{48589}
\saveTG{𩎆}{48593}
\saveTG{䩣}{48594}
\saveTG{警}{48601}
\saveTG{𩠶}{48602}
\saveTG{𠺛}{48604}
\saveTG{𨣻}{48604}
\saveTG{𪡜}{48617}
\saveTG{𢽽}{48640}
\saveTG{敬}{48640}
\saveTG{故}{48640}
\saveTG{㪚}{48640}
\saveTG{𫄿}{48672}
\saveTG{𪓴}{48717}
\saveTG{𨝝}{48717}
\saveTG{𫅸}{48737}
\saveTG{㽐}{48737}
\saveTG{𢾤}{48740}
\saveTG{𫅵}{48740}
\saveTG{𢽱}{48740}
\saveTG{𢻱}{48740}
\saveTG{𠬗}{48763}
\saveTG{𦓊}{48766}
\saveTG{𧮯}{48768}
\saveTG{𪥏}{48800}
\saveTG{龪}{48800}
\saveTG{𧺍}{48800}
\saveTG{𧺌}{48800}
\saveTG{趖}{48801}
\saveTG{趛}{48801}
\saveTG{䟀}{48801}
\saveTG{𧺞}{48801}
\saveTG{䞘}{48801}
\saveTG{𧻤}{48801}
\saveTG{䞢}{48802}
\saveTG{𧼄}{48802}
\saveTG{𧼯}{48802}
\saveTG{𧾟}{48802}
\saveTG{𨅚}{48802}
\saveTG{䟑}{48802}
\saveTG{𧺮}{48802}
\saveTG{趁}{48802}
\saveTG{赺}{48802}
\saveTG{𧽳}{48802}
\saveTG{𧻋}{48803}
\saveTG{趝}{48803}
\saveTG{𧺭}{48803}
\saveTG{𧾘}{48804}
\saveTG{𧼱}{48804}
\saveTG{𧻓}{48804}
\saveTG{𧺴}{48804}
\saveTG{𧾐}{48804}
\saveTG{𧽾}{48804}
\saveTG{𧻝}{48804}
\saveTG{𧾅}{48804}
\saveTG{𧾀}{48806}
\saveTG{䞩}{48806}
\saveTG{𧽜}{48806}
\saveTG{𧾃}{48806}
\saveTG{趥}{48806}
\saveTG{䞱}{48806}
\saveTG{𧺣}{48808}
\saveTG{𧾏}{48808}
\saveTG{𧽵}{48808}
\saveTG{䞮}{48809}
\saveTG{𤎅}{48809}
\saveTG{𠔫}{48811}
\saveTG{𦒙}{48827}
\saveTG{𠔡}{48827}
\saveTG{𫜴}{48827}
\saveTG{黅}{48827}
\saveTG{𢼭}{48840}
\saveTG{𤉧}{48840}
\saveTG{𣛈}{48891}
\saveTG{𤐈}{48891}
\saveTG{朳}{48900}
\saveTG{朲}{48900}
\saveTG{枞}{48900}
\saveTG{杁}{48900}
\saveTG{檠}{48904}
\saveTG{𣘢}{48904}
\saveTG{𥹛}{48904}
\saveTG{𣔿}{48904}
\saveTG{柞}{48911}
\saveTG{栏}{48911}
\saveTG{槎}{48912}
\saveTG{柂}{48912}
\saveTG{榏}{48912}
\saveTG{梲}{48912}
\saveTG{椸}{48912}
\saveTG{棁}{48912}
\saveTG{𣞯}{48912}
\saveTG{𣡓}{48912}
\saveTG{椪}{48912}
\saveTG{欖}{48912}
\saveTG{榄}{48912}
\saveTG{𣝃}{48912}
\saveTG{𣙥}{48912}
\saveTG{𣡶}{48912}
\saveTG{槛}{48912}
\saveTG{檻}{48912}
\saveTG{枪}{48912}
\saveTG{𣙖}{48912}
\saveTG{𣕆}{48912}
\saveTG{𣛩}{48914}
\saveTG{㭫}{48914}
\saveTG{樦}{48914}
\saveTG{栓}{48914}
\saveTG{𣐍}{48914}
\saveTG{𣔫}{48914}
\saveTG{𪳸}{48914}
\saveTG{𣙚}{48915}
\saveTG{権}{48915}
\saveTG{𣒻}{48917}
\saveTG{𪲞}{48917}
\saveTG{杚}{48917}
\saveTG{㰖}{48917}
\saveTG{𣡅}{48917}
\saveTG{𣕗}{48917}
\saveTG{𣘎}{48917}
\saveTG{𣖰}{48917}
\saveTG{𣝳}{48917}
\saveTG{𣏙}{48917}
\saveTG{𣙫}{48918}
\saveTG{𣡔}{48918}
\saveTG{检}{48919}
\saveTG{𣔋}{48919}
\saveTG{𣏎}{48920}
\saveTG{𪲾}{48921}
\saveTG{榆}{48921}
\saveTG{檹}{48921}
\saveTG{椾}{48921}
\saveTG{櫤}{48921}
\saveTG{𣘠}{48921}
\saveTG{𣘀}{48922}
\saveTG{𣟹}{48922}
\saveTG{𣛌}{48927}
\saveTG{𣛖}{48927}
\saveTG{𣞂}{48927}
\saveTG{𣙱}{48927}
\saveTG{𣠰}{48927}
\saveTG{𣘺}{48927}
\saveTG{枌}{48927}
\saveTG{棆}{48927}
\saveTG{櫛}{48927}
\saveTG{枍}{48927}
\saveTG{枔}{48927}
\saveTG{椧}{48927}
\saveTG{檎}{48927}
\saveTG{橁}{48927}
\saveTG{𣛉}{48927}
\saveTG{梯}{48927}
\saveTG{𪴧}{48927}
\saveTG{㰏}{48927}
\saveTG{㮬}{48927}
\saveTG{㯓}{48927}
\saveTG{𣜤}{48929}
\saveTG{𪳋}{48930}
\saveTG{𣗹}{48931}
\saveTG{榚}{48931}
\saveTG{橅}{48931}
\saveTG{𣕜}{48931}
\saveTG{𣖯}{48932}
\saveTG{桧}{48932}
\saveTG{柃}{48932}
\saveTG{𪳹}{48932}
\saveTG{𣜭}{48932}
\saveTG{棯}{48932}
\saveTG{松}{48932}
\saveTG{楡}{48932}
\saveTG{𣔾}{48932}
\saveTG{𣜴}{48932}
\saveTG{㭚}{48932}
\saveTG{𣖺}{48932}
\saveTG{𣞬}{48932}
\saveTG{𣞮}{48932}
\saveTG{𣝡}{48932}
\saveTG{𪟹}{48932}
\saveTG{𪱸}{48933}
\saveTG{棇}{48933}
\saveTG{𪴆}{48933}
\saveTG{棜}{48933}
\saveTG{檖}{48933}
\saveTG{𣜑}{48933}
\saveTG{𣛶}{48934}
\saveTG{𣠑}{48935}
\saveTG{𣜃}{48936}
\saveTG{檤}{48936}
\saveTG{槏}{48937}
\saveTG{㮸}{48938}
\saveTG{㮹}{48940}
\saveTG{橄}{48940}
\saveTG{𣕌}{48940}
\saveTG{櫢}{48940}
\saveTG{𣟺}{48940}
\saveTG{𣏤}{48940}
\saveTG{𣚿}{48940}
\saveTG{枚}{48940}
\saveTG{𣘮}{48940}
\saveTG{𣙕}{48940}
\saveTG{𣗣}{48940}
\saveTG{𣒵}{48940}
\saveTG{檄}{48940}
\saveTG{𣟷}{48940}
\saveTG{橵}{48940}
\saveTG{𣡁}{48940}
\saveTG{𣙙}{48940}
\saveTG{救}{48940}
\saveTG{𪳞}{48940}
\saveTG{㮘}{48940}
\saveTG{㯳}{48940}
\saveTG{𢽓}{48940}
\saveTG{𣞒}{48940}
\saveTG{橔}{48940}
\saveTG{杵}{48940}
\saveTG{𢾣}{48940}
\saveTG{㯙}{48940}
\saveTG{𢽟}{48940}
\saveTG{栟}{48941}
\saveTG{檊}{48941}
\saveTG{𣔼}{48941}
\saveTG{𢿰}{48943}
\saveTG{𣟔}{48943}
\saveTG{𣛆}{48943}
\saveTG{𣘧}{48943}
\saveTG{𣕞}{48944}
\saveTG{𣝶}{48944}
\saveTG{㮵}{48944}
\saveTG{𣝁}{48945}
\saveTG{樽}{48946}
\saveTG{𣜒}{48947}
\saveTG{𣛦}{48947}
\saveTG{𣙨}{48947}
\saveTG{𣒰}{48947}
\saveTG{栴}{48947}
\saveTG{椱}{48947}
\saveTG{𣠘}{48949}
\saveTG{𣠉}{48951}
\saveTG{𣟲}{48951}
\saveTG{㮆}{48951}
\saveTG{样}{48951}
\saveTG{𣑻}{48952}
\saveTG{𣗯}{48952}
\saveTG{檥}{48953}
\saveTG{𣝕}{48953}
\saveTG{𣘷}{48953}
\saveTG{椫}{48956}
\saveTG{梅}{48957}
\saveTG{𪴅}{48957}
\saveTG{㰕}{48957}
\saveTG{𣠊}{48958}
\saveTG{柗}{48960}
\saveTG{橏}{48961}
\saveTG{㭘}{48961}
\saveTG{𣠱}{48961}
\saveTG{㯚}{48961}
\saveTG{𣚞}{48961}
\saveTG{𠎙}{48961}
\saveTG{𣜆}{48961}
\saveTG{𣚴}{48961}
\saveTG{𣚰}{48962}
\saveTG{𪳂}{48962}
\saveTG{梒}{48962}
\saveTG{𣙭}{48962}
\saveTG{𣘉}{48962}
\saveTG{𣘜}{48963}
\saveTG{𣟣}{48963}
\saveTG{楢}{48964}
\saveTG{𣓌}{48964}
\saveTG{櫡}{48964}
\saveTG{𣛗}{48964}
\saveTG{𣘞}{48964}
\saveTG{橧}{48966}
\saveTG{檜}{48966}
\saveTG{槍}{48967}
\saveTG{𪴌}{48977}
\saveTG{樅}{48981}
\saveTG{𣗍}{48981}
\saveTG{檱}{48981}
\saveTG{𣒀}{48982}
\saveTG{檨}{48982}
\saveTG{𪲧}{48982}
\saveTG{㯀}{48982}
\saveTG{𣖙}{48984}
\saveTG{栚}{48984}
\saveTG{橂}{48984}
\saveTG{㮳}{48984}
\saveTG{𣛘}{48984}
\saveTG{𣠺}{48986}
\saveTG{検}{48986}
\saveTG{檢}{48986}
\saveTG{㰓}{48986}
\saveTG{𣝎}{48989}
\saveTG{𣐐}{48990}
\saveTG{𣠓}{48991}
\saveTG{樣}{48992}
\saveTG{様}{48992}
\saveTG{𣠹}{48993}
\saveTG{𣞫}{48994}
\saveTG{𣔌}{48994}
\saveTG{𥻳}{48994}
\saveTG{梌}{48994}
\saveTG{𡯩}{49012}
\saveTG{𠧁}{49066}
\saveTG{𪤅}{49112}
\saveTG{埢}{49112}
\saveTG{垙}{49112}
\saveTG{堘}{49114}
\saveTG{𩁮}{49115}
\saveTG{𡐃}{49117}
\saveTG{㝻}{49120}
\saveTG{埫}{49127}
\saveTG{𡑍}{49127}
\saveTG{𡌔}{49127}
\saveTG{𥊢}{49127}
\saveTG{𪤉}{49132}
\saveTG{𪣴}{49140}
\saveTG{𪣻}{49144}
\saveTG{𡏰}{49144}
\saveTG{𨮰}{49144}
\saveTG{坢}{49150}
\saveTG{𡑄}{49152}
\saveTG{𡑝}{49157}
\saveTG{壋}{49166}
\saveTG{垱}{49177}
\saveTG{𡓴}{49181}
\saveTG{𡏶}{49186}
\saveTG{埮}{49189}
\saveTG{燅}{49189}
\saveTG{㚃}{49198}
\saveTG{狲}{49200}
\saveTG{𪧿}{49200}
\saveTG{𦝘}{49217}
\saveTG{𢃩}{49217}
\saveTG{㹰}{49217}
\saveTG{猟}{49217}
\saveTG{𦙧}{49220}
\saveTG{猀}{49220}
\saveTG{㠺}{49220}
\saveTG{𢄑}{49227}
\saveTG{𤞚}{49227}
\saveTG{𢅲}{49227}
\saveTG{𢅮}{49227}
\saveTG{𦞊}{49227}
\saveTG{𠜓}{49227}
\saveTG{帩}{49227}
\saveTG{𢅎}{49240}
\saveTG{𩀼}{49240}
\saveTG{𤠋}{49244}
\saveTG{𦚓}{49250}
\saveTG{幥}{49252}
\saveTG{獜}{49259}
\saveTG{𤢎}{49266}
\saveTG{狄}{49280}
\saveTG{𢃸}{49280}
\saveTG{𢃔}{49289}
\saveTG{𤟇}{49289}
\saveTG{𤝸}{49294}
\saveTG{悐}{49338}
\saveTG{姯}{49412}
\saveTG{婘}{49412}
\saveTG{𡠠}{49414}
\saveTG{𪦯}{49414}
\saveTG{𪌮}{49420}
\saveTG{妙}{49420}
\saveTG{麨}{49420}
\saveTG{𪌯}{49427}
\saveTG{𡤰}{49427}
\saveTG{嫦}{49427}
\saveTG{𡞀}{49427}
\saveTG{𡡯}{49427}
\saveTG{娋}{49427}
\saveTG{𪍉}{49429}
\saveTG{𡤭}{49431}
\saveTG{𪦱}{49431}
\saveTG{𡣖}{49432}
\saveTG{㜆}{49439}
\saveTG{𡡠}{49440}
\saveTG{𣒁}{49444}
\saveTG{𡞱}{49444}
\saveTG{𡟒}{49444}
\saveTG{姅}{49450}
\saveTG{𪍴}{49457}
\saveTG{嫾}{49459}
\saveTG{𡞞}{49462}
\saveTG{㜭}{49466}
\saveTG{𡣣}{49468}
\saveTG{𡡀}{49468}
\saveTG{𤿖}{49472}
\saveTG{𡟊}{49480}
\saveTG{𪍗}{49480}
\saveTG{𪌌}{49480}
\saveTG{𥽤}{49481}
\saveTG{䵀}{49486}
\saveTG{婒}{49489}
\saveTG{𪌬}{49494}
\saveTG{嬫}{49494}
\saveTG{𡡢}{49494}
\saveTG{𢸧}{49502}
\saveTG{𩊠}{49512}
\saveTG{鞺}{49514}
\saveTG{𨐈}{49517}
\saveTG{𩎃}{49517}
\saveTG{𩎸}{49517}
\saveTG{䩖}{49520}
\saveTG{𩊮}{49520}
\saveTG{𩌨}{49527}
\saveTG{鞝}{49527}
\saveTG{韒}{49527}
\saveTG{鞘}{49527}
\saveTG{𩋇}{49527}
\saveTG{𩊂}{49532}
\saveTG{𩍳}{49547}
\saveTG{靽}{49550}
\saveTG{辚}{49559}
\saveTG{䩧}{49561}
\saveTG{鞦}{49580}
\saveTG{𩌆}{49586}
\saveTG{𣻅}{49620}
\saveTG{𡯀}{49700}
\saveTG{𨾅}{49715}
\saveTG{尠}{49720}
\saveTG{𠫴}{49720}
\saveTG{𤯇}{49789}
\saveTG{𡮱}{49796}
\saveTG{𧼚}{49801}
\saveTG{赻}{49802}
\saveTG{趟}{49802}
\saveTG{趙}{49802}
\saveTG{𧽟}{49802}
\saveTG{𧺾}{49805}
\saveTG{𧼃}{49808}
\saveTG{𧺩}{49808}
\saveTG{𧸝}{49850}
\saveTG{𪏋}{49889}
\saveTG{㭂}{49900}
\saveTG{𣟢}{49911}
\saveTG{桄}{49912}
\saveTG{棬}{49912}
\saveTG{𣙟}{49914}
\saveTG{樘}{49914}
\saveTG{𣘽}{49917}
\saveTG{橩}{49917}
\saveTG{𣗋}{49917}
\saveTG{𪲛}{49919}
\saveTG{𡮘}{49920}
\saveTG{桫}{49920}
\saveTG{杪}{49920}
\saveTG{𣕂}{49927}
\saveTG{𣖷}{49927}
\saveTG{㭞}{49927}
\saveTG{㭻}{49927}
\saveTG{𣔐}{49927}
\saveTG{橯}{49927}
\saveTG{椦}{49927}
\saveTG{梢}{49927}
\saveTG{橳}{49927}
\saveTG{欓}{49931}
\saveTG{𣜷}{49938}
\saveTG{𣗌}{49939}
\saveTG{橕}{49941}
\saveTG{𣒹}{49944}
\saveTG{楼}{49944}
\saveTG{㰔}{49947}
\saveTG{𪱿}{49947}
\saveTG{柈}{49950}
\saveTG{﨔}{49952}
\saveTG{𣛟}{49952}
\saveTG{𪲦}{49952}
\saveTG{榉}{49958}
\saveTG{橉}{49959}
\saveTG{櫿}{49966}
\saveTG{檔}{49966}
\saveTG{档}{49977}
\saveTG{𣏹}{49980}
\saveTG{楸}{49980}
\saveTG{梑}{49980}
\saveTG{𣠂}{49981}
\saveTG{𣑷}{49984}
\saveTG{㰊}{49989}
\saveTG{𣜧}{49989}
\saveTG{棪}{49989}
\saveTG{𣠃}{49989}
\saveTG{橖}{49994}
\saveTG{𣟕}{49994}
\saveTG{𣞁}{49994}
\saveTG{扌}{50000}
\saveTG{扩}{50000}
\saveTG{丯}{50000}
\saveTG{丰}{50000}
\saveTG{𠂖}{50000}
\saveTG{𠀆}{50000}
\saveTG{㐄}{50000}
\saveTG{丈}{50000}
\saveTG{𢪏}{50000}
\saveTG{龶}{50001}
\saveTG{𡗒}{50002}
\saveTG{𢎯}{50002}
\saveTG{𠥽}{50003}
\saveTG{吏}{50006}
\saveTG{申}{50006}
\saveTG{史}{50006}
\saveTG{曳}{50006}
\saveTG{中}{50006}
\saveTG{車}{50006}
\saveTG{𠦆}{50006}
\saveTG{𤰶}{50006}
\saveTG{串}{50006}
\saveTG{肀}{50007}
\saveTG{𦘒}{50007}
\saveTG{事}{50007}
\saveTG{亊}{50007}
\saveTG{聿}{50007}
\saveTG{𠁦}{50011}
\saveTG{攠}{50011}
\saveTG{摝}{50012}
\saveTG{轆}{50012}
\saveTG{摬}{50012}
\saveTG{兂}{50012}
\saveTG{𢬂}{50014}
\saveTG{軴}{50014}
\saveTG{𢸵}{50014}
\saveTG{拄}{50014}
\saveTG{𢷹}{50014}
\saveTG{𪭽}{50014}
\saveTG{摌}{50015}
\saveTG{𢺚}{50015}
\saveTG{𨾳}{50015}
\saveTG{𢹼}{50015}
\saveTG{𢶾}{50015}
\saveTG{𨌴}{50015}
\saveTG{䡴}{50015}
\saveTG{𨾡}{50015}
\saveTG{𢺠}{50015}
\saveTG{𨾓}{50015}
\saveTG{撞}{50015}
\saveTG{擁}{50015}
\saveTG{推}{50015}
\saveTG{攤}{50015}
\saveTG{摊}{50015}
\saveTG{攡}{50015}
\saveTG{擅}{50016}
\saveTG{𢯻}{50017}
\saveTG{𢸎}{50017}
\saveTG{攍}{50017}
\saveTG{丸}{50017}
\saveTG{抗}{50017}
\saveTG{䡉}{50017}
\saveTG{𢴅}{50017}
\saveTG{𢷉}{50017}
\saveTG{𢲣}{50017}
\saveTG{𢺆}{50017}
\saveTG{㧤}{50017}
\saveTG{㧧}{50017}
\saveTG{𢺑}{50017}
\saveTG{𨋢}{50018}
\saveTG{𢯛}{50018}
\saveTG{拉}{50018}
\saveTG{𢱅}{50018}
\saveTG{𢭸}{50021}
\saveTG{𢰤}{50021}
\saveTG{揨}{50021}
\saveTG{𢵲}{50022}
\saveTG{𢱘}{50022}
\saveTG{擠}{50023}
\saveTG{𢹓}{50023}
\saveTG{挤}{50024}
\saveTG{㨈}{50024}
\saveTG{搞}{50027}
\saveTG{韦}{50027}
\saveTG{甹}{50027}
\saveTG{携}{50027}
\saveTG{擕}{50027}
\saveTG{摘}{50027}
\saveTG{㧸}{50027}
\saveTG{㧍}{50027}
\saveTG{𪭫}{50027}
\saveTG{𠃫}{50027}
\saveTG{𨍩}{50027}
\saveTG{𢳣}{50027}
\saveTG{㩦}{50027}
\saveTG{𨌯}{50027}
\saveTG{𢯡}{50027}
\saveTG{𪮧}{50027}
\saveTG{搒}{50027}
\saveTG{摛}{50027}
\saveTG{揥}{50027}
\saveTG{抃}{50030}
\saveTG{𢴒}{50030}
\saveTG{𪢱}{50030}
\saveTG{撨}{50031}
\saveTG{㩠}{50031}
\saveTG{𪮳}{50031}
\saveTG{𪯂}{50031}
\saveTG{𢫔}{50031}
\saveTG{𢸣}{50032}
\saveTG{𨎭}{50032}
\saveTG{𢮯}{50032}
\saveTG{拡}{50032}
\saveTG{𢴶}{50032}
\saveTG{挔}{50032}
\saveTG{擿}{50032}
\saveTG{𢸬}{50032}
\saveTG{㨰}{50032}
\saveTG{攘}{50032}
\saveTG{𨏛}{50032}
\saveTG{𢰜}{50032}
\saveTG{𨍽}{50032}
\saveTG{㩝}{50032}
\saveTG{𪮠}{50033}
\saveTG{𢶶}{50036}
\saveTG{𢳘}{50036}
\saveTG{𢶧}{50037}
\saveTG{𨎷}{50037}
\saveTG{摭}{50037}
\saveTG{捬}{50040}
\saveTG{𢭷}{50040}
\saveTG{抆}{50040}
\saveTG{𨌍}{50041}
\saveTG{㨩}{50041}
\saveTG{𢶥}{50041}
\saveTG{擗}{50041}
\saveTG{𨌮}{50041}
\saveTG{𪮊}{50042}
\saveTG{掋}{50042}
\saveTG{𢲜}{50042}
\saveTG{摔}{50043}
\saveTG{𢪴}{50044}
\saveTG{接}{50044}
\saveTG{𢬵}{50044}
\saveTG{𢷱}{50044}
\saveTG{𢳋}{50045}
\saveTG{}{50047}
\saveTG{掖}{50047}
\saveTG{𢯪}{50047}
\saveTG{𪮿}{50047}
\saveTG{𨍏}{50047}
\saveTG{㩳}{50047}
\saveTG{𨎙}{50047}
\saveTG{𢺁}{50047}
\saveTG{㨦}{50047}
\saveTG{𢱋}{50047}
\saveTG{㨃}{50047}
\saveTG{挍}{50048}
\saveTG{捽}{50048}
\saveTG{較}{50048}
\saveTG{𢬪}{50049}
\saveTG{𢴠}{50050}
\saveTG{擵}{50052}
\saveTG{𨎡}{50053}
\saveTG{撁}{50053}
\saveTG{𢵹}{50054}
\saveTG{𢫨}{50054}
\saveTG{𪮉}{50054}
\saveTG{𢫾}{50060}
\saveTG{掊}{50061}
\saveTG{掂}{50061}
\saveTG{𨌶}{50061}
\saveTG{𨍤}{50061}
\saveTG{揞}{50061}
\saveTG{𨍑}{50061}
\saveTG{𢶹}{50061}
\saveTG{攟}{50061}
\saveTG{𢴨}{50062}
\saveTG{搐}{50063}
\saveTG{𢭹}{50064}
\saveTG{搪}{50065}
\saveTG{𨍴}{50065}
\saveTG{𢹲}{50067}
\saveTG{𢱙}{50082}
\saveTG{㧡}{50082}
\saveTG{輆}{50082}
\saveTG{𢳿}{50085}
\saveTG{擴}{50086}
\saveTG{𨏅}{50086}
\saveTG{掶}{50087}
\saveTG{𢰍}{50089}
\saveTG{𨎺}{50091}
\saveTG{𨏸}{50091}
\saveTG{𢱨}{50093}
\saveTG{𢵸}{50094}
\saveTG{𢳀}{50094}
\saveTG{𨌟}{50094}
\saveTG{𢮕}{50094}
\saveTG{攗}{50094}
\saveTG{攈}{50094}
\saveTG{𢭩}{50094}
\saveTG{𪮾}{50094}
\saveTG{𨎹}{50094}
\saveTG{𢶸}{50094}
\saveTG{𢱊}{50096}
\saveTG{㨂}{50096}
\saveTG{輬}{50096}
\saveTG{𨍡}{50096}
\saveTG{掠}{50096}
\saveTG{𨎍}{50099}
\saveTG{𢳧}{50099}
\saveTG{𢦾}{50100}
\saveTG{𧈴}{50100}
\saveTG{}{50100}
\saveTG{𠀏}{50101}
\saveTG{𠄤}{50101}
\saveTG{蠱}{50102}
\saveTG{衋}{50102}
\saveTG{𣌽}{50102}
\saveTG{盅}{50102}
\saveTG{盎}{50102}
\saveTG{蛊}{50102}
\saveTG{𥁞}{50102}
\saveTG{𧗚}{50102}
\saveTG{䀌}{50102}
\saveTG{𫋫}{50102}
\saveTG{𠁕}{50102}
\saveTG{𧗙}{50102}
\saveTG{𥂛}{50102}
\saveTG{𥁫}{50102}
\saveTG{𧗊}{50102}
\saveTG{䀆}{50102}
\saveTG{𧗁}{50102}
\saveTG{𦘕}{50102}
\saveTG{盡}{50102}
\saveTG{𡎀}{50104}
\saveTG{𦥆}{50104}
\saveTG{𡉫}{50104}
\saveTG{𦦜}{50104}
\saveTG{𦤿}{50104}
\saveTG{𡏑}{50104}
\saveTG{晝}{50106}
\saveTG{畫}{50106}
\saveTG{𫚟}{50106}
\saveTG{𤲯}{50106}
\saveTG{𠀐}{50106}
\saveTG{}{50107}
\saveTG{𧯟}{50108}
\saveTG{𢎶}{50108}
\saveTG{𨰵}{50109}
\saveTG{虻}{50110}
\saveTG{𩇶}{50111}
\saveTG{螰}{50112}
\saveTG{𧒱}{50112}
\saveTG{蛀}{50114}
\saveTG{𧋧}{50114}
\saveTG{𧒐}{50114}
\saveTG{𧔊}{50114}
\saveTG{𧕀}{50114}
\saveTG{𧐻}{50115}
\saveTG{𧕘}{50115}
\saveTG{𧔿}{50115}
\saveTG{𧕮}{50115}
\saveTG{蜼}{50115}
\saveTG{𩁑}{50115}
\saveTG{𧑆}{50115}
\saveTG{蟺}{50116}
\saveTG{𠁰}{50117}
\saveTG{𧐢}{50117}
\saveTG{𧏵}{50117}
\saveTG{䖻}{50117}
\saveTG{𧕺}{50117}
\saveTG{蚢}{50117}
\saveTG{𧕳}{50117}
\saveTG{𧕻}{50117}
\saveTG{𪚻}{50117}
\saveTG{𧍄}{50117}
\saveTG{𧉼}{50118}
\saveTG{𧏑}{50118}
\saveTG{𠄬}{50119}
\saveTG{蝏}{50121}
\saveTG{𧕚}{50123}
\saveTG{蠐}{50123}
\saveTG{蛴}{50124}
\saveTG{𣉽}{50127}
\saveTG{𧉽}{50127}
\saveTG{𧎸}{50127}
\saveTG{𢫮}{50127}
\saveTG{𢬆}{50127}
\saveTG{蜟}{50127}
\saveTG{𧓈}{50127}
\saveTG{𧍝}{50127}
\saveTG{𦑢}{50127}
\saveTG{䗤}{50127}
\saveTG{𧏸}{50127}
\saveTG{螃}{50127}
\saveTG{螭}{50127}
\saveTG{螪}{50127}
\saveTG{鸯}{50127}
\saveTG{蚄}{50127}
\saveTG{𧊤}{50130}
\saveTG{蟭}{50131}
\saveTG{𧐳}{50131}
\saveTG{𧕄}{50131}
\saveTG{䗯}{50131}
\saveTG{𧔄}{50131}
\saveTG{𧖚}{50131}
\saveTG{𧕰}{50131}
\saveTG{𧓕}{50131}
\saveTG{𧔵}{50131}
\saveTG{𧓼}{50131}
\saveTG{𧑖}{50131}
\saveTG{蠰}{50132}
\saveTG{蠔}{50132}
\saveTG{𧏹}{50132}
\saveTG{𧖗}{50132}
\saveTG{蚿}{50132}
\saveTG{蠹}{50136}
\saveTG{䗷}{50136}
\saveTG{𫊭}{50136}
\saveTG{𧉾}{50136}
\saveTG{𧐧}{50136}
\saveTG{𧉉}{50136}
\saveTG{蟗}{50136}
\saveTG{蠢}{50136}
\saveTG{蟲}{50136}
\saveTG{虫}{50136}
\saveTG{𧋌}{50136}
\saveTG{𧏋}{50136}
\saveTG{蟅}{50137}
\saveTG{𧒲}{50137}
\saveTG{蠊}{50137}
\saveTG{蚊}{50140}
\saveTG{𧓄}{50141}
\saveTG{𧔫}{50141}
\saveTG{𫋁}{50141}
\saveTG{𧌊}{50142}
\saveTG{𧐛}{50142}
\saveTG{𧔋}{50143}
\saveTG{𫋔}{50143}
\saveTG{蟀}{50143}
\saveTG{𧊷}{50144}
\saveTG{蟑}{50146}
\saveTG{𧕟}{50147}
\saveTG{蝷}{50147}
\saveTG{蜳}{50147}
\saveTG{蜶}{50148}
\saveTG{蛟}{50148}
\saveTG{𧐤}{50153}
\saveTG{䗈}{50161}
\saveTG{𧍈}{50162}
\saveTG{𣈘}{50162}
\saveTG{𧏷}{50163}
\saveTG{𣜾}{50164}
\saveTG{螗}{50165}
\saveTG{𧊏}{50182}
\saveTG{𧐃}{50184}
\saveTG{螏}{50184}
\saveTG{𧍁}{50185}
\saveTG{𧑦}{50185}
\saveTG{𫋧}{50186}
\saveTG{𧐚}{50189}
\saveTG{䗫}{50194}
\saveTG{𫋝}{50194}
\saveTG{𧌬}{50196}
\saveTG{䗧}{50199}
\saveTG{𠂙}{50200}
\saveTG{𪥃}{50201}
\saveTG{𪫈}{50202}
\saveTG{𦘔}{50207}
\saveTG{尧}{50212}
\saveTG{𫕼}{50212}
\saveTG{𠑸}{50212}
\saveTG{𧢆}{50212}
\saveTG{𠒏}{50212}
\saveTG{兂}{50212}
\saveTG{𨿬}{50215}
\saveTG{𨿌}{50215}
\saveTG{𩇤}{50217}
\saveTG{𠘸}{50217}
\saveTG{𡇗}{50217}
\saveTG{畁}{50221}
\saveTG{肅}{50222}
\saveTG{肃}{50227}
\saveTG{𠂔}{50227}
\saveTG{𪩲}{50227}
\saveTG{𡗚}{50227}
\saveTG{𠕢}{50227}
\saveTG{冑}{50227}
\saveTG{胄}{50227}
\saveTG{粛}{50227}
\saveTG{帇}{50227}
\saveTG{青}{50227}
\saveTG{靑}{50227}
\saveTG{巿}{50227}
\saveTG{𢄖}{50227}
\saveTG{𡘦}{50227}
\saveTG{𤰋}{50227}
\saveTG{𠛸}{50227}
\saveTG{𢄿}{50227}
\saveTG{𦘡}{50227}
\saveTG{𤲥}{50227}
\saveTG{𪥐}{50227}
\saveTG{𦘘}{50227}
\saveTG{𦘝}{50227}
\saveTG{𧉎}{50227}
\saveTG{本}{50230}
\saveTG{𨋈}{50232}
\saveTG{𤰰}{50232}
\saveTG{𣳕}{50232}
\saveTG{𣷛}{50232}
\saveTG{㤗}{50238}
\saveTG{𣠕}{50264}
\saveTG{㫪}{50268}
\saveTG{𠁧}{50277}
\saveTG{𡱋}{50277}
\saveTG{𪲥}{50302}
\saveTG{𪲕}{50302}
\saveTG{专}{50302}
\saveTG{𠗀}{50302}
\saveTG{枣}{50303}
\saveTG{𠖼}{50303}
\saveTG{栆}{50303}
\saveTG{𡘕}{50307}
\saveTG{䏋}{50317}
\saveTG{𠁱}{50326}
\saveTG{𩢥}{50327}
\saveTG{𪆊}{50327}
\saveTG{鴦}{50327}
\saveTG{𦘜}{50327}
\saveTG{𦘛}{50327}
\saveTG{𢙻}{50327}
\saveTG{𪃣}{50327}
\saveTG{𤆖}{50330}
\saveTG{恵}{50330}
\saveTG{𤈂}{50331}
\saveTG{𤊌}{50331}
\saveTG{𢗣}{50331}
\saveTG{𪬵}{50332}
\saveTG{惠}{50333}
\saveTG{𢗱}{50334}
\saveTG{焘}{50334}
\saveTG{𢞯}{50334}
\saveTG{恵}{50336}
\saveTG{患}{50336}
\saveTG{𤆪}{50336}
\saveTG{忠}{50336}
\saveTG{惷}{50336}
\saveTG{𤌲}{50336}
\saveTG{憃}{50337}
\saveTG{𩶓}{50338}
\saveTG{𡗫}{50338}
\saveTG{𡗭}{50338}
\saveTG{𢘲}{50338}
\saveTG{𢗤}{50338}
\saveTG{寿}{50340}
\saveTG{專}{50343}
\saveTG{専}{50346}
\saveTG{𪯡}{50401}
\saveTG{㔼}{50402}
\saveTG{妻}{50404}
\saveTG{婁}{50404}
\saveTG{叓}{50407}
\saveTG{叏}{50407}
\saveTG{𠭄}{50407}
\saveTG{𢜤}{50407}
\saveTG{麦}{50407}
\saveTG{𡕤}{50407}
\saveTG{䍟}{50407}
\saveTG{𡕦}{50407}
\saveTG{𡙹}{50408}
\saveTG{𨿩}{50415}
\saveTG{𩀮}{50415}
\saveTG{𨎊}{50432}
\saveTG{𣁤}{50440}
\saveTG{𣁖}{50440}
\saveTG{𢍣}{50443}
\saveTG{𡝉}{50444}
\saveTG{𡝽}{50446}
\saveTG{𡛽}{50446}
\saveTG{𡝤}{50446}
\saveTG{冉}{50447}
\saveTG{𠬼}{50447}
\saveTG{𨋌}{50447}
\saveTG{𣑜}{50447}
\saveTG{𡝁}{50449}
\saveTG{𡢺}{50480}
\saveTG{𠦴}{50501}
\saveTG{𩊄}{50506}
\saveTG{𤱓}{50506}
\saveTG{奉}{50508}
\saveTG{𠁸}{50548}
\saveTG{轟}{50550}
\saveTG{𨌽}{50552}
\saveTG{𨎑}{50556}
\saveTG{𨌼}{50556}
\saveTG{𢹐}{50594}
\saveTG{𤰔}{50600}
\saveTG{由}{50600}
\saveTG{𠵤}{50601}
\saveTG{𥅌}{50601}
\saveTG{旾}{50601}
\saveTG{書}{50601}
\saveTG{𣆊}{50601}
\saveTG{𥒠}{50601}
\saveTG{砉}{50602}
\saveTG{𣅾}{50604}
\saveTG{𦘠}{50604}
\saveTG{𪠷}{50605}
\saveTG{軎}{50606}
\saveTG{𠲀}{50606}
\saveTG{曺}{50606}
\saveTG{𡈋}{50606}
\saveTG{𤱪}{50607}
\saveTG{𠰫}{50608}
\saveTG{春}{50608}
\saveTG{𨽶}{50609}
\saveTG{唜}{50612}
\saveTG{𩁀}{50615}
\saveTG{㖝}{50617}
\saveTG{𢅺}{50627}
\saveTG{𠢌}{50627}
\saveTG{𦦺}{50627}
\saveTG{𤳐}{50634}
\saveTG{𦦾}{50641}
\saveTG{夀}{50641}
\saveTG{𡭏}{50643}
\saveTG{𥗤}{50662}
\saveTG{㮺}{50663}
\saveTG{𣠖}{50664}
\saveTG{𡄅}{50666}
\saveTG{𣬛}{50710}
\saveTG{𤕣}{50710}
\saveTG{𤰴}{50711}
\saveTG{𤰣}{50711}
\saveTG{𫌷}{50716}
\saveTG{电}{50716}
\saveTG{屯}{50717}
\saveTG{𪩫}{50717}
\saveTG{𪓛}{50717}
\saveTG{𪜃}{50717}
\saveTG{㐕}{50717}
\saveTG{𪓗}{50717}
\saveTG{𨋐}{50717}
\saveTG{㼜}{50717}
\saveTG{𪜑}{50719}
\saveTG{㽕}{50727}
\saveTG{𨊪}{50731}
\saveTG{𨊢}{50731}
\saveTG{𧞢}{50732}
\saveTG{𧘴}{50732}
\saveTG{𪽉}{50732}
\saveTG{表}{50732}
\saveTG{𧚐}{50732}
\saveTG{𠫢}{50732}
\saveTG{嚢}{50732}
\saveTG{囊}{50732}
\saveTG{叀}{50736}
\saveTG{𨎱}{50742}
\saveTG{𠬻}{50747}
\saveTG{毒}{50757}
\saveTG{击}{50772}
\saveTG{畵}{50772}
\saveTG{𦘚}{50772}
\saveTG{𦘙}{50772}
\saveTG{𤲿}{50772}
\saveTG{𡵭}{50772}
\saveTG{𨊥}{50772}
\saveTG{𠦟}{50772}
\saveTG{𡸸}{50773}
\saveTG{𠚆}{50777}
\saveTG{𠳋}{50777}
\saveTG{𦦱}{50777}
\saveTG{舂}{50777}
\saveTG{𤱫}{50786}
\saveTG{央}{50800}
\saveTG{}{50800}
\saveTG{𡗗}{50800}
\saveTG{夹}{50800}
\saveTG{夬}{50800}
\saveTG{夫}{50800}
\saveTG{疌}{50801}
\saveTG{𨾂}{50801}
\saveTG{夷}{50802}
\saveTG{𤴝}{50802}
\saveTG{𠆴}{50802}
\saveTG{𨃲}{50802}
\saveTG{贵}{50802}
\saveTG{赉}{50802}
\saveTG{责}{50802}
\saveTG{𤴛}{50802}
\saveTG{𣚮}{50804}
\saveTG{奏}{50804}
\saveTG{𡙱}{50804}
\saveTG{責}{50806}
\saveTG{𧶁}{50806}
\saveTG{賮}{50806}
\saveTG{𧷇}{50806}
\saveTG{貴}{50806}
\saveTG{㬰}{50806}
\saveTG{𥈜}{50806}
\saveTG{軣}{50808}
\saveTG{𣌼}{50808}
\saveTG{𧈾}{50809}
\saveTG{𤍐}{50809}
\saveTG{㶳}{50809}
\saveTG{𨾚}{50815}
\saveTG{𨾕}{50815}
\saveTG{𡚀}{50843}
\saveTG{𤈮}{50850}
\saveTG{䝴}{50861}
\saveTG{𧵩}{50869}
\saveTG{𠁊}{50888}
\saveTG{末}{50900}
\saveTG{来}{50900}
\saveTG{耒}{50900}
\saveTG{未}{50900}
\saveTG{𥘿}{50901}
\saveTG{棗}{50902}
\saveTG{𠷽}{50902}
\saveTG{朿}{50902}
\saveTG{素}{50903}
\saveTG{𦄯}{50903}
\saveTG{㰆}{50904}
\saveTG{𣒛}{50904}
\saveTG{𣏲}{50904}
\saveTG{𣝐}{50904}
\saveTG{𣡦}{50904}
\saveTG{𣡏}{50904}
\saveTG{𣠔}{50904}
\saveTG{𣚇}{50904}
\saveTG{𣡪}{50904}
\saveTG{𣠀}{50904}
\saveTG{秦}{50904}
\saveTG{𣘯}{50904}
\saveTG{𣓧}{50904}
\saveTG{𣡖}{50904}
\saveTG{㯻}{50904}
\saveTG{𫀜}{50904}
\saveTG{𣐇}{50904}
\saveTG{𣞉}{50904}
\saveTG{櫜}{50904}
\saveTG{𣞈}{50904}
\saveTG{𣟏}{50904}
\saveTG{橐}{50904}
\saveTG{㯱}{50904}
\saveTG{𣐺}{50905}
\saveTG{𣗞}{50905}
\saveTG{𣕤}{50905}
\saveTG{束}{50906}
\saveTG{東}{50906}
\saveTG{柬}{50906}
\saveTG{𣏃}{50907}
\saveTG{泰}{50909}
\saveTG{隶}{50909}
\saveTG{𦔭}{50911}
\saveTG{𨿢}{50915}
\saveTG{𦔛}{50915}
\saveTG{𠅍}{50917}
\saveTG{𦔓}{50927}
\saveTG{𣎺}{50927}
\saveTG{𠀟}{50927}
\saveTG{䎮}{50927}
\saveTG{耪}{50927}
\saveTG{𦔩}{50931}
\saveTG{耲}{50932}
\saveTG{𦔗}{50937}
\saveTG{辢}{50941}
\saveTG{𦆾}{50943}
\saveTG{𣕮}{50946}
\saveTG{𣖢}{50948}
\saveTG{䎧}{50961}
\saveTG{𦔑}{50961}
\saveTG{耱}{50962}
\saveTG{𦔝}{50962}
\saveTG{𣒚}{50966}
\saveTG{𣝯}{50992}
\saveTG{𨏷}{50993}
\saveTG{纛}{50993}
\saveTG{䆐}{50994}
\saveTG{𥠼}{50994}
\saveTG{𣠆}{50995}
\saveTG{𪎎}{51006}
\saveTG{𨋤}{51007}
\saveTG{}{51010}
\saveTG{扯}{51010}
\saveTG{輫}{51011}
\saveTG{攏}{51011}
\saveTG{轣}{51011}
\saveTG{攊}{51011}
\saveTG{擓}{51011}
\saveTG{排}{51011}
\saveTG{𢬤}{51011}
\saveTG{㩑}{51011}
\saveTG{𨏠}{51011}
\saveTG{軭}{51011}
\saveTG{摣}{51012}
\saveTG{攦}{51012}
\saveTG{攎}{51012}
\saveTG{轤}{51012}
\saveTG{輕}{51012}
\saveTG{抏}{51012}
\saveTG{扤}{51012}
\saveTG{挜}{51012}
\saveTG{掗}{51012}
\saveTG{軏}{51012}
\saveTG{輒}{51012}
\saveTG{挋}{51012}
\saveTG{𨊧}{51012}
\saveTG{𢲮}{51012}
\saveTG{𢬥}{51012}
\saveTG{𢬀}{51012}
\saveTG{𣌟}{51012}
\saveTG{𨋷}{51012}
\saveTG{䡿}{51012}
\saveTG{𨏨}{51012}
\saveTG{𢺰}{51012}
\saveTG{𢱉}{51012}
\saveTG{𢴮}{51012}
\saveTG{𢷰}{51012}
\saveTG{㨤}{51012}
\saveTG{𨋆}{51012}
\saveTG{㧜}{51012}
\saveTG{𢴂}{51012}
\saveTG{𢳍}{51012}
\saveTG{𢪔}{51012}
\saveTG{䎠}{51012}
\saveTG{𢶍}{51012}
\saveTG{𢹠}{51012}
\saveTG{扼}{51012}
\saveTG{軛}{51012}
\saveTG{抚}{51012}
\saveTG{摡}{51012}
\saveTG{扛}{51012}
\saveTG{揯}{51012}
\saveTG{挳}{51012}
\saveTG{𪭤}{51012}
\saveTG{𨋔}{51014}
\saveTG{𢹥}{51014}
\saveTG{抠}{51014}
\saveTG{挃}{51014}
\saveTG{捱}{51014}
\saveTG{𨌷}{51014}
\saveTG{𢴬}{51014}
\saveTG{𨌻}{51014}
\saveTG{𢲖}{51014}
\saveTG{䡖}{51014}
\saveTG{𫏷}{51014}
\saveTG{𢫛}{51014}
\saveTG{𢲔}{51014}
\saveTG{𡉈}{51014}
\saveTG{𢭘}{51014}
\saveTG{𨌃}{51014}
\saveTG{揠}{51014}
\saveTG{輊}{51014}
\saveTG{軖}{51014}
\saveTG{抂}{51014}
\saveTG{𨋶}{51014}
\saveTG{𢺓}{51014}
\saveTG{攉}{51015}
\saveTG{摳}{51016}
\saveTG{𢴚}{51016}
\saveTG{𢸉}{51016}
\saveTG{䡱}{51016}
\saveTG{搄}{51016}
\saveTG{㩖}{51016}
\saveTG{𨏃}{51016}
\saveTG{𢬎}{51016}
\saveTG{𢳛}{51016}
\saveTG{䡇}{51017}
\saveTG{𢪑}{51017}
\saveTG{𢴓}{51017}
\saveTG{𢸗}{51017}
\saveTG{𢫣}{51017}
\saveTG{𢮎}{51017}
\saveTG{𢪦}{51017}
\saveTG{𢬴}{51017}
\saveTG{𢰫}{51017}
\saveTG{𢭫}{51017}
\saveTG{拒}{51017}
\saveTG{㧟}{51017}
\saveTG{𢮉}{51017}
\saveTG{𨐀}{51017}
\saveTG{㧚}{51017}
\saveTG{𢪁}{51017}
\saveTG{𨏽}{51017}
\saveTG{𢳌}{51017}
\saveTG{𢪞}{51017}
\saveTG{𢮴}{51017}
\saveTG{𢶐}{51018}
\saveTG{𢵊}{51018}
\saveTG{𢵦}{51018}
\saveTG{𢷴}{51018}
\saveTG{𢭃}{51018}
\saveTG{抷}{51019}
\saveTG{𨊡}{51020}
\saveTG{𢱏}{51020}
\saveTG{打}{51020}
\saveTG{抲}{51020}
\saveTG{軻}{51020}
\saveTG{𨊫}{51021}
\saveTG{𨏎}{51021}
\saveTG{𢫱}{51021}
\saveTG{𢵏}{51021}
\saveTG{捗}{51021}
\saveTG{𢯼}{51021}
\saveTG{𢬲}{51026}
\saveTG{𢺬}{51026}
\saveTG{輌}{51027}
\saveTG{擄}{51027}
\saveTG{𢲐}{51027}
\saveTG{𪭵}{51027}
\saveTG{𢪀}{51027}
\saveTG{𢪰}{51027}
\saveTG{𢩨}{51027}
\saveTG{𢮳}{51027}
\saveTG{擩}{51027}
\saveTG{𢪜}{51027}
\saveTG{摴}{51027}
\saveTG{掳}{51027}
\saveTG{輛}{51027}
\saveTG{𨊮}{51027}
\saveTG{掚}{51027}
\saveTG{扝}{51027}
\saveTG{掯}{51027}
\saveTG{轜}{51027}
\saveTG{搹}{51027}
\saveTG{㧫}{51027}
\saveTG{𢫚}{51027}
\saveTG{𢫲}{51027}
\saveTG{𤭔}{51027}
\saveTG{𨎇}{51027}
\saveTG{擟}{51027}
\saveTG{𨍮}{51027}
\saveTG{𨋣}{51027}
\saveTG{輀}{51027}
\saveTG{𨎪}{51027}
\saveTG{𢲫}{51027}
\saveTG{𢩯}{51027}
\saveTG{𢺞}{51027}
\saveTG{𢮔}{51027}
\saveTG{𢪪}{51027}
\saveTG{𢳃}{51027}
\saveTG{𢯯}{51027}
\saveTG{𫏲}{51027}
\saveTG{抦}{51027}
\saveTG{𨍅}{51027}
\saveTG{𢰚}{51027}
\saveTG{𢩹}{51030}
\saveTG{𢵆}{51031}
\saveTG{拤}{51031}
\saveTG{挊}{51031}
\saveTG{摅}{51031}
\saveTG{𢵣}{51031}
\saveTG{𢱽}{51031}
\saveTG{𨋇}{51031}
\saveTG{𢴾}{51031}
\saveTG{振}{51032}
\saveTG{𢵎}{51032}
\saveTG{據}{51032}
\saveTG{転}{51032}
\saveTG{抎}{51032}
\saveTG{𢳫}{51032}
\saveTG{㩕}{51032}
\saveTG{𢲀}{51032}
\saveTG{掁}{51032}
\saveTG{𢸊}{51032}
\saveTG{㧻}{51032}
\saveTG{𨌑}{51032}
\saveTG{𨎶}{51032}
\saveTG{𢺪}{51032}
\saveTG{𢶘}{51032}
\saveTG{𢸍}{51032}
\saveTG{𢷭}{51032}
\saveTG{𢴊}{51033}
\saveTG{𢺂}{51035}
\saveTG{𢰷}{51036}
\saveTG{攄}{51036}
\saveTG{掭}{51038}
\saveTG{𨊱}{51040}
\saveTG{𢵪}{51040}
\saveTG{挕}{51040}
\saveTG{扜}{51040}
\saveTG{軒}{51040}
\saveTG{㨜}{51040}
\saveTG{扞}{51040}
\saveTG{搟}{51040}
\saveTG{挵}{51041}
\saveTG{攝}{51041}
\saveTG{𪮸}{51041}
\saveTG{𢰵}{51041}
\saveTG{𢪟}{51041}
\saveTG{㨿}{51041}
\saveTG{㧎}{51042}
\saveTG{𨏄}{51042}
\saveTG{𢷛}{51042}
\saveTG{𨏴}{51042}
\saveTG{𨌀}{51042}
\saveTG{搙}{51043}
\saveTG{𢱼}{51044}
\saveTG{𢰳}{51044}
\saveTG{𢶚}{51044}
\saveTG{𨊻}{51044}
\saveTG{𨍥}{51044}
\saveTG{𢴃}{51045}
\saveTG{㩀}{51045}
\saveTG{𨌬}{51046}
\saveTG{𢶁}{51046}
\saveTG{撢}{51046}
\saveTG{掉}{51046}
\saveTG{挭}{51046}
\saveTG{擾}{51047}
\saveTG{摄}{51047}
\saveTG{軙}{51047}
\saveTG{𢺕}{51047}
\saveTG{𢹎}{51047}
\saveTG{𢬌}{51047}
\saveTG{𠁶}{51047}
\saveTG{𢯬}{51047}
\saveTG{𢪊}{51047}
\saveTG{𢯳}{51047}
\saveTG{𢴻}{51047}
\saveTG{𢮽}{51047}
\saveTG{𢫳}{51048}
\saveTG{軯}{51049}
\saveTG{抨}{51049}
\saveTG{𪜉}{51049}
\saveTG{摢}{51049}
\saveTG{𢱎}{51050}
\saveTG{拝}{51050}
\saveTG{𨍼}{51057}
\saveTG{𢲸}{51060}
\saveTG{拈}{51060}
\saveTG{𨍬}{51061}
\saveTG{𢹩}{51061}
\saveTG{𢹝}{51061}
\saveTG{𢸲}{51061}
\saveTG{𨎿}{51061}
\saveTG{捂}{51061}
\saveTG{撍}{51061}
\saveTG{擂}{51061}
\saveTG{搢}{51061}
\saveTG{㩅}{51061}
\saveTG{𢴯}{51061}
\saveTG{䡼}{51061}
\saveTG{𢵚}{51062}
\saveTG{𢫦}{51062}
\saveTG{𨋓}{51062}
\saveTG{㩡}{51062}
\saveTG{𨏒}{51062}
\saveTG{拓}{51062}
\saveTG{𨎂}{51063}
\saveTG{𢱛}{51064}
\saveTG{輏}{51064}
\saveTG{拪}{51064}
\saveTG{𢭳}{51064}
\saveTG{𨟶}{51064}
\saveTG{揊}{51066}
\saveTG{輻}{51066}
\saveTG{㧷}{51068}
\saveTG{㧵}{51069}
\saveTG{𢫘}{51070}
\saveTG{𢫣}{51070}
\saveTG{𢸡}{51072}
\saveTG{𢰔}{51072}
\saveTG{𢳬}{51074}
\saveTG{轌}{51077}
\saveTG{𨌡}{51077}
\saveTG{𪮵}{51081}
\saveTG{𢳜}{51082}
\saveTG{撅}{51082}
\saveTG{撷}{51082}
\saveTG{𢳵}{51082}
\saveTG{㨗}{51082}
\saveTG{擫}{51084}
\saveTG{𢭠}{51084}
\saveTG{𢬍}{51084}
\saveTG{𢮸}{51084}
\saveTG{𢷯}{51084}
\saveTG{㨎}{51084}
\saveTG{輭}{51084}
\saveTG{𢶃}{51086}
\saveTG{𩑚}{51086}
\saveTG{擷}{51086}
\saveTG{揁}{51086}
\saveTG{攧}{51086}
\saveTG{摃}{51086}
\saveTG{𢷧}{51086}
\saveTG{𢹻}{51086}
\saveTG{𢸸}{51086}
\saveTG{𢶄}{51086}
\saveTG{𢴦}{51086}
\saveTG{䡠}{51086}
\saveTG{𢹟}{51086}
\saveTG{𢹉}{51086}
\saveTG{𩒷}{51086}
\saveTG{㩪}{51086}
\saveTG{㩩}{51086}
\saveTG{𢹃}{51086}
\saveTG{𩒅}{51086}
\saveTG{𢹷}{51086}
\saveTG{𩑖}{51086}
\saveTG{摂}{51088}
\saveTG{𢱾}{51089}
\saveTG{𢭅}{51089}
\saveTG{抔}{51090}
\saveTG{摽}{51091}
\saveTG{揼}{51092}
\saveTG{𢵍}{51094}
\saveTG{𢷡}{51094}
\saveTG{𨍫}{51094}
\saveTG{𢳗}{51094}
\saveTG{𢸹}{51094}
\saveTG{搮}{51094}
\saveTG{𢶳}{51094}
\saveTG{𢷽}{51094}
\saveTG{𪮢}{51096}
\saveTG{𪾏}{51102}
\saveTG{𢿫}{51102}
\saveTG{𡑼}{51104}
\saveTG{𫊮}{51107}
\saveTG{鑋}{51109}
\saveTG{蠬}{51111}
\saveTG{䖱}{51111}
\saveTG{𧈵}{51112}
\saveTG{𧔝}{51112}
\saveTG{𧐅}{51112}
\saveTG{𧊺}{51112}
\saveTG{蛵}{51112}
\saveTG{虹}{51112}
\saveTG{𧋹}{51112}
\saveTG{蚖}{51112}
\saveTG{蠦}{51112}
\saveTG{蚅}{51112}
\saveTG{𧖜}{51112}
\saveTG{𧋇}{51112}
\saveTG{𫋋}{51112}
\saveTG{𫊤}{51112}
\saveTG{𧐊}{51113}
\saveTG{蛭}{51114}
\saveTG{𧓋}{51114}
\saveTG{𧔹}{51114}
\saveTG{蚟}{51114}
\saveTG{䗎}{51114}
\saveTG{𧉣}{51114}
\saveTG{𧍊}{51114}
\saveTG{𧋵}{51114}
\saveTG{𫋑}{51114}
\saveTG{蝘}{51114}
\saveTG{𧏺}{51116}
\saveTG{䗵}{51116}
\saveTG{𧊳}{51116}
\saveTG{𧑡}{51117}
\saveTG{䗂}{51117}
\saveTG{𧔀}{51117}
\saveTG{𤭹}{51117}
\saveTG{蚷}{51117}
\saveTG{𧑂}{51117}
\saveTG{𧕯}{51117}
\saveTG{𧑜}{51117}
\saveTG{𧎞}{51117}
\saveTG{𧋖}{51117}
\saveTG{𧈭}{51117}
\saveTG{蚽}{51119}
\saveTG{蚵}{51120}
\saveTG{虰}{51120}
\saveTG{𧊽}{51121}
\saveTG{𧌧}{51121}
\saveTG{𧎘}{51121}
\saveTG{𧌂}{51122}
\saveTG{𧖋}{51126}
\saveTG{𧎺}{51126}
\saveTG{𧑮}{51127}
\saveTG{𧉝}{51127}
\saveTG{𧔸}{51127}
\saveTG{螄}{51127}
\saveTG{𫋃}{51127}
\saveTG{𧓠}{51127}
\saveTG{𧒘}{51127}
\saveTG{蛃}{51127}
\saveTG{蛎}{51127}
\saveTG{蠣}{51127}
\saveTG{螞}{51127}
\saveTG{螎}{51127}
\saveTG{蠕}{51127}
\saveTG{蛳}{51127}
\saveTG{𧊌}{51127}
\saveTG{蜽}{51127}
\saveTG{䗡}{51127}
\saveTG{𧈯}{51127}
\saveTG{𧉄}{51127}
\saveTG{𧔽}{51127}
\saveTG{虾}{51130}
\saveTG{𧏽}{51131}
\saveTG{𧉛}{51131}
\saveTG{𧑕}{51131}
\saveTG{䗅}{51132}
\saveTG{蜄}{51132}
\saveTG{𧓻}{51132}
\saveTG{𧏿}{51132}
\saveTG{䖶}{51132}
\saveTG{𧌮}{51132}
\saveTG{𧒓}{51134}
\saveTG{蟵}{51140}
\saveTG{𧋂}{51140}
\saveTG{虷}{51140}
\saveTG{蚈}{51140}
\saveTG{蚜}{51140}
\saveTG{虶}{51140}
\saveTG{𧓴}{51141}
\saveTG{𧕩}{51142}
\saveTG{𧊗}{51142}
\saveTG{𧏯}{51143}
\saveTG{𧍔}{51144}
\saveTG{𧋼}{51144}
\saveTG{𧋑}{51145}
\saveTG{𧐼}{51146}
\saveTG{𧌸}{51146}
\saveTG{蟫}{51146}
\saveTG{𧋁}{51147}
\saveTG{𧕖}{51147}
\saveTG{𧕡}{51147}
\saveTG{𧏡}{51147}
\saveTG{𧎍}{51147}
\saveTG{𣀂}{51147}
\saveTG{𧔅}{51147}
\saveTG{蚲}{51149}
\saveTG{𧔠}{51151}
\saveTG{蝆}{51152}
\saveTG{𧍲}{51156}
\saveTG{蛅}{51160}
\saveTG{𧋋}{51161}
\saveTG{𧒽}{51161}
\saveTG{𧕅}{51161}
\saveTG{𧎽}{51161}
\saveTG{䖨}{51162}
\saveTG{蛨}{51162}
\saveTG{蝒}{51162}
\saveTG{𫋏}{51163}
\saveTG{蝠}{51166}
\saveTG{𧌀}{51168}
\saveTG{𧉋}{51172}
\saveTG{𪘅}{51172}
\saveTG{𧍹}{51172}
\saveTG{𧌭}{51181}
\saveTG{𧑺}{51182}
\saveTG{蟩}{51182}
\saveTG{𧉆}{51182}
\saveTG{𧐪}{51182}
\saveTG{𧑻}{51182}
\saveTG{蝡}{51184}
\saveTG{𧉂}{51184}
\saveTG{𧔾}{51186}
\saveTG{𫋐}{51186}
\saveTG{蝢}{51186}
\saveTG{𧏦}{51189}
\saveTG{𧉈}{51190}
\saveTG{螵}{51191}
\saveTG{蟝}{51194}
\saveTG{𧋝}{51194}
\saveTG{螈}{51196}
\saveTG{𣯷}{51217}
\saveTG{𠞲}{51227}
\saveTG{𩇗}{51227}
\saveTG{𩇞}{51231}
\saveTG{𩇣}{51234}
\saveTG{𣀾}{51247}
\saveTG{𢿀}{51247}
\saveTG{𢻵}{51247}
\saveTG{㪩}{51247}
\saveTG{𢼹}{51247}
\saveTG{𩈨}{51262}
\saveTG{𩒺}{51286}
\saveTG{𩓨}{51286}
\saveTG{䫨}{51286}
\saveTG{顣}{51286}
\saveTG{顑}{51286}
\saveTG{𩓁}{51286}
\saveTG{甎}{51317}
\saveTG{𪈈}{51327}
\saveTG{䳲}{51327}
\saveTG{𤋒}{51336}
\saveTG{𢜋}{51336}
\saveTG{𢥤}{51338}
\saveTG{𣀨}{51347}
\saveTG{甊}{51417}
\saveTG{㼮}{51417}
\saveTG{𧇑}{51417}
\saveTG{𢿘}{51447}
\saveTG{𠕟}{51462}
\saveTG{𪎋}{51462}
\saveTG{麺}{51462}
\saveTG{𩑺}{51486}
\saveTG{顜}{51486}
\saveTG{䫫}{51486}
\saveTG{𩖅}{51486}
\saveTG{䫔}{51486}
\saveTG{虦}{51517}
\saveTG{𢵿}{51524}
\saveTG{𠁻}{51562}
\saveTG{𧈕}{51612}
\saveTG{𣌹}{51619}
\saveTG{𧺅}{51631}
\saveTG{𢾎}{51647}
\saveTG{𢼰}{51647}
\saveTG{}{51682}
\saveTG{𩕗}{51686}
\saveTG{𩕫}{51686}
\saveTG{頔}{51686}
\saveTG{𩓏}{51686}
\saveTG{𧈖}{51717}
\saveTG{𤭥}{51717}
\saveTG{顿}{51782}
\saveTG{頓}{51786}
\saveTG{𩓳}{51786}
\saveTG{𨆧}{51801}
\saveTG{𨁘}{51802}
\saveTG{𨆪}{51802}
\saveTG{𪥖}{51804}
\saveTG{𤬺}{51817}
\saveTG{𪯋}{51847}
\saveTG{𢻳}{51847}
\saveTG{颊}{51882}
\saveTG{𫖴}{51882}
\saveTG{䫭}{51886}
\saveTG{𩔳}{51886}
\saveTG{𩖀}{51886}
\saveTG{頬}{51886}
\saveTG{𦔙}{51912}
\saveTG{耟}{51917}
\saveTG{㼯}{51917}
\saveTG{𦓪}{51917}
\saveTG{耓}{51920}
\saveTG{𫅺}{51927}
\saveTG{𣜋}{51927}
\saveTG{𣑎}{51930}
\saveTG{耘}{51932}
\saveTG{𦓶}{51932}
\saveTG{𢆬}{51940}
\saveTG{𣜻}{51943}
\saveTG{耨}{51943}
\saveTG{𢍫}{51944}
\saveTG{繛}{51946}
\saveTG{耰}{51947}
\saveTG{𣀪}{51947}
\saveTG{㪝}{51947}
\saveTG{𦓬}{51949}
\saveTG{𪲅}{51949}
\saveTG{㮌}{51962}
\saveTG{𩈘}{51962}
\saveTG{𣔸}{51962}
\saveTG{𦔆}{51966}
\saveTG{㮕}{51984}
\saveTG{𩓕}{51986}
\saveTG{𩔥}{51986}
\saveTG{𩑷}{51986}
\saveTG{𩓋}{51986}
\saveTG{頛}{51986}
\saveTG{頼}{51986}
\saveTG{𢮰}{52000}
\saveTG{𢪉}{52000}
\saveTG{𢷚}{52000}
\saveTG{𢯔}{52000}
\saveTG{𢶏}{52000}
\saveTG{𢮓}{52000}
\saveTG{𢷶}{52000}
\saveTG{𢵷}{52000}
\saveTG{㧃}{52000}
\saveTG{𢪫}{52000}
\saveTG{𪮁}{52000}
\saveTG{𢯩}{52000}
\saveTG{𢳐}{52000}
\saveTG{𢫃}{52000}
\saveTG{𢱦}{52000}
\saveTG{𠚶}{52000}
\saveTG{𪮌}{52000}
\saveTG{𪭼}{52000}
\saveTG{䡅}{52000}
\saveTG{𠜒}{52000}
\saveTG{𨍀}{52000}
\saveTG{𠛜}{52000}
\saveTG{㓝}{52000}
\saveTG{划}{52000}
\saveTG{㨽}{52000}
\saveTG{揦}{52000}
\saveTG{剚}{52000}
\saveTG{挒}{52000}
\saveTG{刜}{52000}
\saveTG{捯}{52000}
\saveTG{刬}{52000}
\saveTG{捌}{52000}
\saveTG{𠜈}{52000}
\saveTG{𠛉}{52000}
\saveTG{𢫧}{52000}
\saveTG{扎}{52010}
\saveTG{批}{52010}
\saveTG{軋}{52010}
\saveTG{𢵞}{52011}
\saveTG{𨊺}{52012}
\saveTG{𢫁}{52012}
\saveTG{𢭊}{52012}
\saveTG{擸}{52012}
\saveTG{拞}{52012}
\saveTG{𢷼}{52012}
\saveTG{挑}{52013}
\saveTG{𪮕}{52013}
\saveTG{𢷖}{52013}
\saveTG{𢱐}{52013}
\saveTG{拰}{52014}
\saveTG{𨌂}{52014}
\saveTG{𢴏}{52014}
\saveTG{𢪭}{52014}
\saveTG{托}{52014}
\saveTG{撬}{52014}
\saveTG{軞}{52014}
\saveTG{軠}{52014}
\saveTG{𢴹}{52014}
\saveTG{𢳸}{52014}
\saveTG{𪮜}{52015}
\saveTG{揰}{52015}
\saveTG{捶}{52015}
\saveTG{摧}{52015}
\saveTG{𢺎}{52016}
\saveTG{𢴫}{52016}
\saveTG{𢬯}{52017}
\saveTG{𢫷}{52017}
\saveTG{𢮤}{52017}
\saveTG{𨌌}{52017}
\saveTG{𨋜}{52017}
\saveTG{𣭂}{52017}
\saveTG{㧗}{52017}
\saveTG{𢩺}{52017}
\saveTG{𢬉}{52017}
\saveTG{𢪲}{52017}
\saveTG{𢰛}{52017}
\saveTG{𢪢}{52017}
\saveTG{𢶼}{52017}
\saveTG{𢳢}{52017}
\saveTG{𢪬}{52017}
\saveTG{搋}{52017}
\saveTG{𢯚}{52017}
\saveTG{𢮊}{52017}
\saveTG{𢬦}{52017}
\saveTG{𢫄}{52017}
\saveTG{𢭖}{52017}
\saveTG{𨋫}{52017}
\saveTG{𢬳}{52017}
\saveTG{𢶤}{52017}
\saveTG{㨢}{52017}
\saveTG{𢮿}{52017}
\saveTG{𢴜}{52017}
\saveTG{𫖖}{52018}
\saveTG{𤳘}{52018}
\saveTG{𨎤}{52018}
\saveTG{㨟}{52018}
\saveTG{撜}{52018}
\saveTG{䡂}{52020}
\saveTG{䡳}{52021}
\saveTG{摲}{52021}
\saveTG{折}{52021}
\saveTG{𣂗}{52021}
\saveTG{斬}{52021}
\saveTG{撕}{52021}
\saveTG{𢒈}{52022}
\saveTG{㣋}{52022}
\saveTG{𨊨}{52022}
\saveTG{𨎧}{52022}
\saveTG{𨎗}{52022}
\saveTG{𢵓}{52022}
\saveTG{𢭺}{52022}
\saveTG{𢳹}{52027}
\saveTG{𢸐}{52027}
\saveTG{𢲽}{52027}
\saveTG{𢭓}{52027}
\saveTG{𢱣}{52027}
\saveTG{𪮤}{52027}
\saveTG{𢺯}{52027}
\saveTG{𢭞}{52027}
\saveTG{𢹣}{52027}
\saveTG{𢭆}{52027}
\saveTG{揹}{52027}
\saveTG{輲}{52027}
\saveTG{揣}{52027}
\saveTG{撝}{52027}
\saveTG{𢳕}{52027}
\saveTG{撟}{52027}
\saveTG{扸}{52027}
\saveTG{攜}{52027}
\saveTG{拶}{52027}
\saveTG{𢳳}{52027}
\saveTG{轎}{52027}
\saveTG{𢶮}{52027}
\saveTG{𢯺}{52027}
\saveTG{𨌳}{52027}
\saveTG{𨏳}{52027}
\saveTG{𢭍}{52027}
\saveTG{㩗}{52027}
\saveTG{𢹂}{52027}
\saveTG{㬞}{52027}
\saveTG{挢}{52028}
\saveTG{𨋲}{52030}
\saveTG{𢬄}{52030}
\saveTG{軱}{52030}
\saveTG{摦}{52030}
\saveTG{抓}{52030}
\saveTG{払}{52030}
\saveTG{𢩻}{52031}
\saveTG{𢫠}{52031}
\saveTG{𢯁}{52031}
\saveTG{𪭞}{52031}
\saveTG{䡏}{52031}
\saveTG{𢱟}{52031}
\saveTG{𢳰}{52031}
\saveTG{𢷠}{52031}
\saveTG{𢩼}{52031}
\saveTG{𢺳}{52031}
\saveTG{𢫻}{52031}
\saveTG{𢫢}{52032}
\saveTG{𢴴}{52032}
\saveTG{抸}{52032}
\saveTG{挀}{52032}
\saveTG{𢸖}{52032}
\saveTG{𪴒}{52033}
\saveTG{𢶎}{52033}
\saveTG{𢮩}{52033}
\saveTG{㧓}{52033}
\saveTG{𢷍}{52034}
\saveTG{𨏱}{52034}
\saveTG{𢶿}{52036}
\saveTG{㩄}{52036}
\saveTG{𢵉}{52037}
\saveTG{𨏈}{52037}
\saveTG{𢫂}{52037}
\saveTG{𨊩}{52037}
\saveTG{𨎒}{52039}
\saveTG{𢴑}{52039}
\saveTG{扺}{52040}
\saveTG{扦}{52040}
\saveTG{軝}{52040}
\saveTG{抵}{52040}
\saveTG{軧}{52040}
\saveTG{挺}{52041}
\saveTG{挻}{52041}
\saveTG{輧}{52041}
\saveTG{拆}{52041}
\saveTG{𢹄}{52041}
\saveTG{𫖒}{52041}
\saveTG{𢮹}{52041}
\saveTG{㨑}{52042}
\saveTG{𢳻}{52043}
\saveTG{㩊}{52043}
\saveTG{㩱}{52043}
\saveTG{捼}{52044}
\saveTG{𢶛}{52044}
\saveTG{𪮓}{52044}
\saveTG{挼}{52044}
\saveTG{𢺲}{52045}
\saveTG{𢯸}{52047}
\saveTG{𪭭}{52047}
\saveTG{𨐂}{52047}
\saveTG{𨍈}{52047}
\saveTG{𫏺}{52047}
\saveTG{𨋻}{52047}
\saveTG{𢺥}{52047}
\saveTG{扳}{52047}
\saveTG{撥}{52047}
\saveTG{援}{52047}
\saveTG{授}{52047}
\saveTG{𢹸}{52047}
\saveTG{䡊}{52047}
\saveTG{𢴔}{52047}
\saveTG{𢳽}{52047}
\saveTG{𨏵}{52047}
\saveTG{𢹚}{52047}
\saveTG{𢱄}{52047}
\saveTG{𢰏}{52047}
\saveTG{𢸻}{52047}
\saveTG{𢯗}{52047}
\saveTG{𢯏}{52047}
\saveTG{捊}{52047}
\saveTG{㧞}{52047}
\saveTG{𩏾}{52047}
\saveTG{𨎀}{52048}
\saveTG{𢷎}{52048}
\saveTG{軤}{52049}
\saveTG{捋}{52049}
\saveTG{抙}{52050}
\saveTG{𫏹}{52050}
\saveTG{𢰱}{52052}
\saveTG{𢰰}{52053}
\saveTG{𪮘}{52054}
\saveTG{𢭣}{52056}
\saveTG{𢭡}{52056}
\saveTG{掙}{52057}
\saveTG{揷}{52057}
\saveTG{挿}{52057}
\saveTG{𨌢}{52057}
\saveTG{𢯈}{52057}
\saveTG{𢴰}{52058}
\saveTG{𢵾}{52061}
\saveTG{指}{52061}
\saveTG{𨌁}{52061}
\saveTG{㧨}{52061}
\saveTG{𪮻}{52062}
\saveTG{揩}{52062}
\saveTG{𢹆}{52062}
\saveTG{䡡}{52062}
\saveTG{𢲃}{52062}
\saveTG{㧺}{52062}
\saveTG{輜}{52063}
\saveTG{𨌭}{52063}
\saveTG{𢲏}{52064}
\saveTG{括}{52064}
\saveTG{捪}{52064}
\saveTG{揗}{52064}
\saveTG{𨌲}{52064}
\saveTG{輴}{52064}
\saveTG{𪮑}{52065}
\saveTG{𨌋}{52067}
\saveTG{㩫}{52068}
\saveTG{播}{52069}
\saveTG{轓}{52069}
\saveTG{𡵎}{52070}
\saveTG{軕}{52070}
\saveTG{㧄}{52070}
\saveTG{𢰠}{52070}
\saveTG{𢩳}{52070}
\saveTG{揺}{52072}
\saveTG{拙}{52072}
\saveTG{捳}{52072}
\saveTG{摇}{52072}
\saveTG{𢹴}{52072}
\saveTG{𢫽}{52074}
\saveTG{𨍳}{52074}
\saveTG{韬}{52077}
\saveTG{搯}{52077}
\saveTG{𢹡}{52077}
\saveTG{插}{52077}
\saveTG{轁}{52077}
\saveTG{𢶽}{52077}
\saveTG{捠}{52081}
\saveTG{𢳄}{52082}
\saveTG{𢹀}{52082}
\saveTG{𢲕}{52084}
\saveTG{𢪺}{52084}
\saveTG{𢯉}{52084}
\saveTG{㨙}{52084}
\saveTG{扷}{52084}
\saveTG{𢮶}{52084}
\saveTG{揆}{52084}
\saveTG{撲}{52085}
\saveTG{𢷏}{52085}
\saveTG{轐}{52085}
\saveTG{𨎉}{52085}
\saveTG{𨎘}{52086}
\saveTG{𨏑}{52086}
\saveTG{㨏}{52089}
\saveTG{𨍱}{52091}
\saveTG{𢲉}{52091}
\saveTG{𢰂}{52091}
\saveTG{𢭁}{52093}
\saveTG{搎}{52093}
\saveTG{𢺈}{52093}
\saveTG{𢬡}{52093}
\saveTG{𢮄}{52094}
\saveTG{𨏋}{52094}
\saveTG{擽}{52094}
\saveTG{𢲇}{52094}
\saveTG{採}{52094}
\saveTG{㧰}{52094}
\saveTG{𢪹}{52094}
\saveTG{轢}{52094}
\saveTG{轈}{52094}
\saveTG{摷}{52094}
\saveTG{𢸫}{52094}
\saveTG{摷}{52095}
\saveTG{擈}{52095}
\saveTG{㨀}{52097}
\saveTG{𧌼}{52100}
\saveTG{蜵}{52100}
\saveTG{𧊞}{52100}
\saveTG{𫊽}{52100}
\saveTG{𧌥}{52100}
\saveTG{𠛿}{52100}
\saveTG{𧍡}{52100}
\saveTG{𧍼}{52100}
\saveTG{蜊}{52100}
\saveTG{𧌵}{52100}
\saveTG{𧎈}{52100}
\saveTG{蛚}{52100}
\saveTG{虯}{52100}
\saveTG{蚓}{52100}
\saveTG{𪟖}{52100}
\saveTG{𫋖}{52100}
\saveTG{𧋀}{52100}
\saveTG{𠜻}{52100}
\saveTG{𠝆}{52100}
\saveTG{𠜚}{52100}
\saveTG{𠟱}{52100}
\saveTG{劃}{52100}
\saveTG{蝲}{52100}
\saveTG{𡌒}{52104}
\saveTG{埑}{52104}
\saveTG{塹}{52104}
\saveTG{𥪭}{52108}
\saveTG{銴}{52109}
\saveTG{鏨}{52109}
\saveTG{鋬}{52109}
\saveTG{虬}{52110}
\saveTG{蚍}{52110}
\saveTG{𧉀}{52111}
\saveTG{蚯}{52112}
\saveTG{蠟}{52112}
\saveTG{𧓓}{52113}
\saveTG{虴}{52114}
\saveTG{蚝}{52114}
\saveTG{𧋭}{52114}
\saveTG{𫊴}{52114}
\saveTG{蜌}{52114}
\saveTG{蝩}{52115}
\saveTG{𧌯}{52115}
\saveTG{蠟}{52116}
\saveTG{𧉔}{52117}
\saveTG{𧉹}{52117}
\saveTG{𧎠}{52117}
\saveTG{𧊑}{52117}
\saveTG{𧐎}{52117}
\saveTG{𧕓}{52117}
\saveTG{𧑵}{52117}
\saveTG{螔}{52117}
\saveTG{䖴}{52117}
\saveTG{螘}{52118}
\saveTG{䗳}{52118}
\saveTG{蜥}{52121}
\saveTG{蟖}{52121}
\saveTG{蚚}{52121}
\saveTG{𧋍}{52121}
\saveTG{𣃂}{52121}
\saveTG{𣃏}{52121}
\saveTG{螹}{52121}
\saveTG{𢒖}{52122}
\saveTG{蟛}{52122}
\saveTG{蟕}{52127}
\saveTG{蜏}{52127}
\saveTG{蝺}{52127}
\saveTG{蟜}{52127}
\saveTG{蟡}{52127}
\saveTG{䘎}{52127}
\saveTG{𧐩}{52127}
\saveTG{𧍮}{52127}
\saveTG{𧈼}{52127}
\saveTG{𧌨}{52127}
\saveTG{𧔍}{52127}
\saveTG{𧔇}{52127}
\saveTG{𧎥}{52127}
\saveTG{𧓯}{52127}
\saveTG{𫋈}{52127}
\saveTG{𧍒}{52127}
\saveTG{蠵}{52127}
\saveTG{𫊸}{52128}
\saveTG{䖣}{52130}
\saveTG{𧏲}{52130}
\saveTG{蛌}{52130}
\saveTG{䗼}{52131}
\saveTG{𧏬}{52131}
\saveTG{𧖣}{52131}
\saveTG{𧈥}{52131}
\saveTG{𫋅}{52132}
\saveTG{䖰}{52132}
\saveTG{𧍆}{52133}
\saveTG{𫊪}{52134}
\saveTG{𧉒}{52135}
\saveTG{𧌐}{52136}
\saveTG{𧐮}{52136}
\saveTG{𧑳}{52136}
\saveTG{蜇}{52136}
\saveTG{𧊉}{52137}
\saveTG{𧒅}{52137}
\saveTG{蟋}{52139}
\saveTG{蚳}{52140}
\saveTG{蚔}{52140}
\saveTG{蜓}{52141}
\saveTG{蚸}{52141}
\saveTG{蜒}{52141}
\saveTG{𧎲}{52141}
\saveTG{䖹}{52141}
\saveTG{𧉲}{52141}
\saveTG{𧊩}{52141}
\saveTG{𫊯}{52142}
\saveTG{𫋜}{52143}
\saveTG{蜲}{52144}
\saveTG{蜉}{52147}
\saveTG{蝬}{52147}
\saveTG{蝂}{52147}
\saveTG{蝯}{52147}
\saveTG{𧓁}{52147}
\saveTG{𧌅}{52147}
\saveTG{𧋏}{52147}
\saveTG{蛶}{52149}
\saveTG{𧖘}{52152}
\saveTG{蟣}{52153}
\saveTG{𧏢}{52154}
\saveTG{𧉳}{52157}
\saveTG{𤄌}{52158}
\saveTG{𧊛}{52161}
\saveTG{𧊙}{52161}
\saveTG{𧌹}{52162}
\saveTG{蝔}{52162}
\saveTG{𧎴}{52162}
\saveTG{𧔜}{52162}
\saveTG{𧌏}{52163}
\saveTG{𧋣}{52164}
\saveTG{蛞}{52164}
\saveTG{𧍎}{52164}
\saveTG{蟠}{52169}
\saveTG{𧍘}{52170}
\saveTG{𧏭}{52172}
\saveTG{䖦}{52172}
\saveTG{𧖖}{52172}
\saveTG{𧍺}{52177}
\saveTG{𧏕}{52181}
\saveTG{螇}{52184}
\saveTG{𧍜}{52184}
\saveTG{䗱}{52185}
\saveTG{𧓳}{52186}
\saveTG{𧐿}{52191}
\saveTG{𫊺}{52193}
\saveTG{𧎤}{52193}
\saveTG{𧋬}{52193}
\saveTG{𧔉}{52194}
\saveTG{蟍}{52194}
\saveTG{𧑀}{52195}
\saveTG{𧑓}{52199}
\saveTG{𧈶}{52199}
\saveTG{𠚷}{52200}
\saveTG{𠛔}{52200}
\saveTG{㔅}{52200}
\saveTG{𠜙}{52200}
\saveTG{𠝜}{52200}
\saveTG{覱}{52212}
\saveTG{𩀧}{52215}
\saveTG{𩴕}{52217}
\saveTG{𣨰}{52217}
\saveTG{𣩂}{52217}
\saveTG{𣰊}{52217}
\saveTG{𣬪}{52217}
\saveTG{𢄤}{52222}
\saveTG{𩇕}{52222}
\saveTG{}{52227}
\saveTG{𢃴}{52227}
\saveTG{𧣯}{52227}
\saveTG{𢂼}{52227}
\saveTG{䯚}{52227}
\saveTG{㿱}{52247}
\saveTG{𢂛}{52247}
\saveTG{靜}{52257}
\saveTG{剸}{52300}
\saveTG{𨔕}{52304}
\saveTG{䳻}{52327}
\saveTG{𪁊}{52327}
\saveTG{𢝗}{52330}
\saveTG{𪑟}{52331}
\saveTG{焎}{52332}
\saveTG{悊}{52332}
\saveTG{慙}{52332}
\saveTG{䱫}{52336}
\saveTG{㓶}{52400}
\saveTG{𠞣}{52400}
\saveTG{𠞭}{52400}
\saveTG{𦗚}{52401}
\saveTG{𦖝}{52401}
\saveTG{𦖨}{52401}
\saveTG{娎}{52404}
\saveTG{𢼺}{52407}
\saveTG{氀}{52414}
\saveTG{𣰟}{52417}
\saveTG{𤬏}{52433}
\saveTG{㜞}{52442}
\saveTG{𢍵}{52449}
\saveTG{剗}{52500}
\saveTG{揧}{52502}
\saveTG{㨻}{52502}
\saveTG{𢳮}{52521}
\saveTG{𣂧}{52521}
\saveTG{𢴛}{52547}
\saveTG{𠟆}{52600}
\saveTG{𠞷}{52600}
\saveTG{𠟷}{52600}
\saveTG{剨}{52600}
\saveTG{𣌺}{52600}
\saveTG{㓢}{52600}
\saveTG{誓}{52601}
\saveTG{磛}{52602}
\saveTG{硩}{52602}
\saveTG{𠼃}{52602}
\saveTG{䀸}{52602}
\saveTG{暫}{52602}
\saveTG{哲}{52602}
\saveTG{晢}{52602}
\saveTG{𩈻}{52602}
\saveTG{𠃦}{52617}
\saveTG{𢆟}{52641}
\saveTG{𤳃}{52684}
\saveTG{䍋}{52715}
\saveTG{𢀄}{52717}
\saveTG{𣭘}{52717}
\saveTG{乴}{52717}
\saveTG{𢒍}{52722}
\saveTG{𢏨}{52727}
\saveTG{䭕}{52732}
\saveTG{裚}{52732}
\saveTG{䭁}{52732}
\saveTG{㼊}{52733}
\saveTG{𡴣}{52757}
\saveTG{𩎡}{52757}
\saveTG{𪘔}{52771}
\saveTG{㟻}{52772}
\saveTG{𪘼}{52772}
\saveTG{刾}{52800}
\saveTG{刔}{52800}
\saveTG{𠜁}{52800}
\saveTG{𠝩}{52800}
\saveTG{踅}{52801}
\saveTG{𣮌}{52801}
\saveTG{蹔}{52801}
\saveTG{䟅}{52802}
\saveTG{𡘭}{52804}
\saveTG{𧷠}{52806}
\saveTG{𡱩}{52807}
\saveTG{烲}{52809}
\saveTG{𣰠}{52817}
\saveTG{}{52827}
\saveTG{𧷑}{52864}
\saveTG{𥡯}{52894}
\saveTG{剌}{52900}
\saveTG{刺}{52900}
\saveTG{𠞓}{52900}
\saveTG{𠠅}{52900}
\saveTG{𠛐}{52900}
\saveTG{𠝁}{52900}
\saveTG{𠛨}{52900}
\saveTG{紥}{52903}
\saveTG{䋢}{52903}
\saveTG{𥻌}{52904}
\saveTG{槧}{52904}
\saveTG{梊}{52904}
\saveTG{𣕔}{52904}
\saveTG{耗}{52914}
\saveTG{𣭵}{52917}
\saveTG{䎰}{52921}
\saveTG{𢒞}{52922}
\saveTG{𦔁}{52927}
\saveTG{𦓺}{52927}
\saveTG{𦔊}{52927}
\saveTG{㼍}{52930}
\saveTG{𤫷}{52933}
\saveTG{㼑}{52933}
\saveTG{𦓽}{52944}
\saveTG{𦅻}{52947}
\saveTG{𨏝}{52947}
\saveTG{䎫}{52947}
\saveTG{耭}{52953}
\saveTG{䎩}{52963}
\saveTG{𦀱}{52993}
\saveTG{𣝞}{52994}
\saveTG{𢫑}{53000}
\saveTG{𢪗}{53000}
\saveTG{𨊳}{53000}
\saveTG{𢦑}{53000}
\saveTG{𢯜}{53000}
\saveTG{戈}{53000}
\saveTG{𢦍}{53000}
\saveTG{㦮}{53000}
\saveTG{𠀋}{53000}
\saveTG{扑}{53000}
\saveTG{𨊣}{53000}
\saveTG{曵}{53000}
\saveTG{抋}{53000}
\saveTG{戋}{53000}
\saveTG{掛}{53000}
\saveTG{㧙}{53004}
\saveTG{𢦣}{53006}
\saveTG{䡎}{53007}
\saveTG{护}{53007}
\saveTG{搾}{53011}
\saveTG{抁}{53012}
\saveTG{𨌵}{53012}
\saveTG{𢯕}{53012}
\saveTG{𢱱}{53012}
\saveTG{拕}{53012}
\saveTG{輐}{53012}
\saveTG{𨌊}{53012}
\saveTG{控}{53012}
\saveTG{扰}{53012}
\saveTG{捖}{53012}
\saveTG{捥}{53012}
\saveTG{𢲼}{53014}
\saveTG{拢}{53014}
\saveTG{挓}{53014}
\saveTG{𢯶}{53014}
\saveTG{㩙}{53014}
\saveTG{㩁}{53015}
\saveTG{𢹬}{53015}
\saveTG{揎}{53016}
\saveTG{攛}{53017}
\saveTG{軶}{53017}
\saveTG{挖}{53017}
\saveTG{㩆}{53017}
\saveTG{㨭}{53017}
\saveTG{𢶊}{53017}
\saveTG{㧖}{53017}
\saveTG{𨋍}{53017}
\saveTG{𢫖}{53017}
\saveTG{𢳙}{53017}
\saveTG{抭}{53017}
\saveTG{𢭔}{53017}
\saveTG{𢲁}{53017}
\saveTG{𢰉}{53017}
\saveTG{䡐}{53017}
\saveTG{䡝}{53017}
\saveTG{𨋠}{53017}
\saveTG{𢭕}{53017}
\saveTG{𢹑}{53018}
\saveTG{擰}{53021}
\saveTG{𢴕}{53021}
\saveTG{拧}{53021}
\saveTG{𢹪}{53022}
\saveTG{掺}{53022}
\saveTG{摻}{53022}
\saveTG{䡪}{53027}
\saveTG{䡢}{53027}
\saveTG{㩷}{53027}
\saveTG{𢮍}{53027}
\saveTG{揙}{53027}
\saveTG{捕}{53027}
\saveTG{輔}{53027}
\saveTG{掮}{53027}
\saveTG{摉}{53027}
\saveTG{书}{53027}
\saveTG{搧}{53027}
\saveTG{𢴧}{53028}
\saveTG{𢮂}{53031}
\saveTG{𨌆}{53031}
\saveTG{攐}{53032}
\saveTG{𢱑}{53032}
\saveTG{搲}{53032}
\saveTG{攨}{53032}
\saveTG{𢬠}{53032}
\saveTG{𢳤}{53032}
\saveTG{𢹞}{53032}
\saveTG{𢲙}{53032}
\saveTG{𢫕}{53032}
\saveTG{䡙}{53032}
\saveTG{𢭗}{53032}
\saveTG{𢳦}{53032}
\saveTG{㩃}{53033}
\saveTG{轗}{53035}
\saveTG{撼}{53035}
\saveTG{𢷾}{53035}
\saveTG{𢰾}{53036}
\saveTG{攇}{53036}
\saveTG{𨏥}{53036}
\saveTG{撡}{53038}
\saveTG{撚}{53038}
\saveTG{𢹇}{53038}
\saveTG{拭}{53040}
\saveTG{軾}{53040}
\saveTG{𢩮}{53040}
\saveTG{拚}{53040}
\saveTG{𢫙}{53040}
\saveTG{𢲟}{53041}
\saveTG{𢯞}{53041}
\saveTG{搏}{53042}
\saveTG{𢬇}{53043}
\saveTG{𨍭}{53043}
\saveTG{𨋒}{53044}
\saveTG{𢺦}{53044}
\saveTG{𨋴}{53044}
\saveTG{按}{53044}
\saveTG{𢱇}{53044}
\saveTG{𢲻}{53047}
\saveTG{𨌘}{53047}
\saveTG{𢯱}{53047}
\saveTG{拔}{53047}
\saveTG{軷}{53047}
\saveTG{拨}{53047}
\saveTG{韨}{53047}
\saveTG{捘}{53047}
\saveTG{𢯴}{53048}
\saveTG{𢳼}{53048}
\saveTG{𢲷}{53048}
\saveTG{輱}{53050}
\saveTG{攕}{53050}
\saveTG{撠}{53050}
\saveTG{找}{53050}
\saveTG{𢬩}{53050}
\saveTG{𢷿}{53050}
\saveTG{𢳴}{53050}
\saveTG{𢹍}{53050}
\saveTG{𢦒}{53050}
\saveTG{𫏳}{53050}
\saveTG{𢲦}{53050}
\saveTG{𪮮}{53050}
\saveTG{𢶩}{53050}
\saveTG{𢶪}{53050}
\saveTG{㩬}{53050}
\saveTG{㧔}{53050}
\saveTG{擮}{53050}
\saveTG{搣}{53050}
\saveTG{擑}{53050}
\saveTG{掝}{53050}
\saveTG{揻}{53050}
\saveTG{摵}{53050}
\saveTG{𢬮}{53051}
\saveTG{𨏓}{53051}
\saveTG{𨏪}{53051}
\saveTG{㩥}{53051}
\saveTG{𢶋}{53051}
\saveTG{𢹖}{53051}
\saveTG{𢭂}{53052}
\saveTG{𢷘}{53052}
\saveTG{𨌚}{53052}
\saveTG{𢯆}{53053}
\saveTG{輚}{53053}
\saveTG{𢬿}{53054}
\saveTG{𢳒}{53054}
\saveTG{㧴}{53055}
\saveTG{𢹊}{53056}
\saveTG{撺}{53056}
\saveTG{𨎕}{53056}
\saveTG{㨔}{53056}
\saveTG{㨓}{53058}
\saveTG{撪}{53058}
\saveTG{𢷄}{53058}
\saveTG{𪮗}{53059}
\saveTG{軩}{53060}
\saveTG{抬}{53060}
\saveTG{𠁯}{53061}
\saveTG{𢲚}{53061}
\saveTG{摍}{53062}
\saveTG{揢}{53064}
\saveTG{𨎣}{53064}
\saveTG{𨍇}{53064}
\saveTG{𢴀}{53064}
\saveTG{揝}{53064}
\saveTG{轄}{53065}
\saveTG{搳}{53065}
\saveTG{㨳}{53065}
\saveTG{𨎜}{53065}
\saveTG{㩈}{53067}
\saveTG{㨧}{53068}
\saveTG{䡥}{53068}
\saveTG{輽}{53068}
\saveTG{搈}{53068}
\saveTG{𢸙}{53069}
\saveTG{𢷈}{53072}
\saveTG{㨸}{53072}
\saveTG{捾}{53077}
\saveTG{輨}{53077}
\saveTG{𢶺}{53077}
\saveTG{𨍵}{53080}
\saveTG{𢳡}{53080}
\saveTG{摈}{53081}
\saveTG{攓}{53081}
\saveTG{掟}{53081}
\saveTG{㧒}{53082}
\saveTG{𢱿}{53082}
\saveTG{𢲘}{53082}
\saveTG{𢺫}{53082}
\saveTG{𢰇}{53084}
\saveTG{挨}{53084}
\saveTG{擜}{53084}
\saveTG{捩}{53084}
\saveTG{揬}{53084}
\saveTG{䡾}{53084}
\saveTG{𢰃}{53084}
\saveTG{𢫯}{53084}
\saveTG{㩎}{53084}
\saveTG{𢵐}{53084}
\saveTG{𢲎}{53084}
\saveTG{㧋}{53084}
\saveTG{㩵}{53084}
\saveTG{𨋩}{53084}
\saveTG{𢵼}{53085}
\saveTG{𢴍}{53086}
\saveTG{𢷤}{53086}
\saveTG{𢺔}{53086}
\saveTG{擯}{53086}
\saveTG{擦}{53091}
\saveTG{𢮱}{53091}
\saveTG{㧠}{53091}
\saveTG{𢴱}{53093}
\saveTG{𨎛}{53094}
\saveTG{㧲}{53094}
\saveTG{㩟}{53094}
\saveTG{𢸘}{53096}
\saveTG{𢮥}{53098}
\saveTG{捄}{53099}
\saveTG{㦯}{53100}
\saveTG{𫊰}{53100}
\saveTG{㦶}{53100}
\saveTG{彧}{53100}
\saveTG{虲}{53100}
\saveTG{戜}{53100}
\saveTG{𪭉}{53100}
\saveTG{𢦴}{53100}
\saveTG{𢦘}{53100}
\saveTG{或}{53100}
\saveTG{𢨜}{53100}
\saveTG{𢨁}{53100}
\saveTG{盞}{53102}
\saveTG{盏}{53102}
\saveTG{盙}{53102}
\saveTG{𪭌}{53102}
\saveTG{盛}{53102}
\saveTG{𥁘}{53102}
\saveTG{䖩}{53104}
\saveTG{𤦂}{53104}
\saveTG{𢧌}{53104}
\saveTG{𡌳}{53104}
\saveTG{𤦒}{53104}
\saveTG{蜿}{53112}
\saveTG{蚘}{53112}
\saveTG{蛇}{53112}
\saveTG{𧰞}{53112}
\saveTG{𧏏}{53112}
\saveTG{𧐷}{53114}
\saveTG{蛖}{53114}
\saveTG{𧍱}{53114}
\saveTG{螲}{53114}
\saveTG{蝖}{53116}
\saveTG{䖳}{53117}
\saveTG{𧑙}{53117}
\saveTG{䖾}{53117}
\saveTG{𧉃}{53117}
\saveTG{𧉵}{53117}
\saveTG{𧉡}{53117}
\saveTG{䗿}{53121}
\saveTG{𧌆}{53121}
\saveTG{𧉞}{53121}
\saveTG{𧑁}{53122}
\saveTG{𧑶}{53127}
\saveTG{𧑗}{53127}
\saveTG{𧓺}{53127}
\saveTG{𧏳}{53127}
\saveTG{𧒠}{53127}
\saveTG{𧒫}{53127}
\saveTG{𧔚}{53127}
\saveTG{蝙}{53127}
\saveTG{蜅}{53127}
\saveTG{𫋙}{53131}
\saveTG{蜋}{53132}
\saveTG{𫋕}{53132}
\saveTG{𧑚}{53133}
\saveTG{𧌒}{53136}
\saveTG{𧕹}{53136}
\saveTG{䘆}{53136}
\saveTG{𧑫}{53138}
\saveTG{蚮}{53140}
\saveTG{𧈺}{53140}
\saveTG{𧊖}{53141}
\saveTG{𧎣}{53141}
\saveTG{𧑰}{53141}
\saveTG{䗚}{53143}
\saveTG{𧍉}{53143}
\saveTG{𧉤}{53144}
\saveTG{𧌃}{53144}
\saveTG{蛂}{53147}
\saveTG{䗏}{53147}
\saveTG{𧉦}{53150}
\saveTG{蛾}{53150}
\saveTG{蟙}{53150}
\saveTG{蛑}{53150}
\saveTG{蜮}{53150}
\saveTG{蠘}{53150}
\saveTG{蝛}{53150}
\saveTG{𧕆}{53151}
\saveTG{䘂}{53151}
\saveTG{𧐦}{53151}
\saveTG{𧔒}{53151}
\saveTG{𧕂}{53151}
\saveTG{𧓥}{53151}
\saveTG{䘋}{53151}
\saveTG{𧒵}{53151}
\saveTG{𧒶}{53153}
\saveTG{𧊥}{53153}
\saveTG{𧔼}{53153}
\saveTG{䗃}{53153}
\saveTG{𧊕}{53154}
\saveTG{𧒿}{53154}
\saveTG{𧍧}{53156}
\saveTG{𧊎}{53157}
\saveTG{䗩}{53159}
\saveTG{𧫊}{53161}
\saveTG{𧉟}{53161}
\saveTG{𧑲}{53162}
\saveTG{𧎡}{53162}
\saveTG{𧎧}{53162}
\saveTG{𧐴}{53162}
\saveTG{𫋍}{53164}
\saveTG{螛}{53165}
\saveTG{䗆}{53177}
\saveTG{𧏖}{53181}
\saveTG{蝊}{53181}
\saveTG{𧉢}{53182}
\saveTG{𫋌}{53182}
\saveTG{𧖃}{53184}
\saveTG{𧏃}{53184}
\saveTG{𤠢}{53184}
\saveTG{蜧}{53184}
\saveTG{𧎛}{53184}
\saveTG{𧑅}{53185}
\saveTG{𧎢}{53186}
\saveTG{𧓍}{53186}
\saveTG{蟘}{53186}
\saveTG{螾}{53186}
\saveTG{蠙}{53186}
\saveTG{𧉱}{53194}
\saveTG{𫋀}{53194}
\saveTG{蛷}{53199}
\saveTG{成}{53200}
\saveTG{戍}{53200}
\saveTG{威}{53200}
\saveTG{咸}{53200}
\saveTG{𢦠}{53200}
\saveTG{𢦵}{53200}
\saveTG{𪱞}{53200}
\saveTG{戚}{53200}
\saveTG{𢧟}{53200}
\saveTG{𢦟}{53200}
\saveTG{戌}{53200}
\saveTG{戊}{53200}
\saveTG{烕}{53200}
\saveTG{𢦨}{53200}
\saveTG{𢦞}{53200}
\saveTG{𪭏}{53204}
\saveTG{𢧩}{53213}
\saveTG{𩳠}{53213}
\saveTG{觱}{53227}
\saveTG{甫}{53227}
\saveTG{旉}{53227}
\saveTG{𢃤}{53227}
\saveTG{膥}{53227}
\saveTG{𧤅}{53227}
\saveTG{𧥑}{53227}
\saveTG{𪻈}{53231}
\saveTG{㳼}{53232}
\saveTG{𢂤}{53247}
\saveTG{𠀽}{53247}
\saveTG{𦛹}{53250}
\saveTG{𢨖}{53250}
\saveTG{𩇚}{53261}
\saveTG{靛}{53281}
\saveTG{𤝮}{53284}
\saveTG{𤜲}{53284}
\saveTG{𪂵}{53327}
\saveTG{𪀵}{53327}
\saveTG{慼}{53330}
\saveTG{惑}{53330}
\saveTG{感}{53330}
\saveTG{𢞿}{53332}
\saveTG{𤒉}{53334}
\saveTG{𤒜}{53335}
\saveTG{慭}{53338}
\saveTG{尃}{53342}
\saveTG{𢨋}{53356}
\saveTG{戎}{53400}
\saveTG{𠨊}{53400}
\saveTG{𪭐}{53400}
\saveTG{戒}{53400}
\saveTG{𡰓}{53417}
\saveTG{𫜒}{53417}
\saveTG{𪎈}{53440}
\saveTG{𢌵}{53440}
\saveTG{𡠃}{53446}
\saveTG{}{53450}
\saveTG{𪎏}{53491}
\saveTG{𢦦}{53500}
\saveTG{戔}{53503}
\saveTG{䰥}{53513}
\saveTG{䪑}{53517}
\saveTG{𨏺}{53547}
\saveTG{𨌏}{53550}
\saveTG{𨏖}{53553}
\saveTG{𢹫}{53572}
\saveTG{𩎧}{53584}
\saveTG{𩎓}{53584}
\saveTG{戓}{53600}
\saveTG{𪭎}{53600}
\saveTG{𢧆}{53600}
\saveTG{㦴}{53600}
\saveTG{𤰭}{53600}
\saveTG{𢧋}{53600}
\saveTG{𥇙}{53600}
\saveTG{𢧔}{53600}
\saveTG{喸}{53612}
\saveTG{𣌱}{53620}
\saveTG{𤱤}{53621}
\saveTG{𦘢}{53632}
\saveTG{𠳆}{53650}
\saveTG{𢨢}{53650}
\saveTG{𢧸}{53650}
\saveTG{𢧬}{53650}
\saveTG{𨠾}{53650}
\saveTG{𡿿}{53700}
\saveTG{𢦙}{53700}
\saveTG{戉}{53700}
\saveTG{𢦭}{53700}
\saveTG{乶}{53717}
\saveTG{𨛚}{53750}
\saveTG{𧚝}{53750}
\saveTG{𧛦}{53750}
\saveTG{㦼}{53750}
\saveTG{𪨯}{53772}
\saveTG{䶠}{53772}
\saveTG{𤝟}{53784}
\saveTG{𢦥}{53800}
\saveTG{𢨇}{53800}
\saveTG{𢦶}{53800}
\saveTG{蹙}{53801}
\saveTG{𠎶}{53805}
\saveTG{𧶤}{53806}
\saveTG{𤉨}{53809}
\saveTG{𤉹}{53809}
\saveTG{𤒓}{53809}
\saveTG{𢨚}{53850}
\saveTG{𧵸}{53860}
\saveTG{𦁓}{53900}
\saveTG{𦓦}{53900}
\saveTG{𥜖}{53901}
\saveTG{𪟾}{53902}
\saveTG{𦔤}{53912}
\saveTG{𡩆}{53917}
\saveTG{𣘾}{53927}
\saveTG{𣏩}{53940}
\saveTG{𦗱}{53941}
\saveTG{𦔍}{53942}
\saveTG{𣒭}{53950}
\saveTG{㦵}{53950}
\saveTG{𢧧}{53954}
\saveTG{𣚺}{53957}
\saveTG{𦔣}{53957}
\saveTG{耛}{53960}
\saveTG{𤳡}{53965}
\saveTG{𪺽}{53984}
\saveTG{𦔅}{53984}
\saveTG{抖}{54000}
\saveTG{轛}{54000}
\saveTG{𢌭}{54000}
\saveTG{𢩩}{54000}
\saveTG{拊}{54000}
\saveTG{𢲥}{54000}
\saveTG{𪭡}{54000}
\saveTG{抍}{54000}
\saveTG{軵}{54000}
\saveTG{𢩱}{54002}
\saveTG{𢱹}{54003}
\saveTG{𢩭}{54003}
\saveTG{𢷮}{54003}
\saveTG{𢷋}{54003}
\saveTG{𢲌}{54003}
\saveTG{𢬭}{54003}
\saveTG{𣁮}{54003}
\saveTG{㩂}{54003}
\saveTG{𢱃}{54003}
\saveTG{𪭸}{54003}
\saveTG{𣁱}{54003}
\saveTG{𡬤}{54003}
\saveTG{𢫊}{54003}
\saveTG{扗}{54010}
\saveTG{𡉘}{54010}
\saveTG{𢩿}{54010}
\saveTG{𢭭}{54010}
\saveTG{𢫟}{54010}
\saveTG{𢭰}{54010}
\saveTG{𪭿}{54010}
\saveTG{𨎽}{54012}
\saveTG{䡷}{54012}
\saveTG{𪭥}{54012}
\saveTG{抛}{54012}
\saveTG{𢷞}{54012}
\saveTG{𢶭}{54012}
\saveTG{𢯿}{54012}
\saveTG{𢹹}{54012}
\saveTG{𢺝}{54012}
\saveTG{㨁}{54012}
\saveTG{𢶉}{54012}
\saveTG{𢱍}{54012}
\saveTG{扡}{54012}
\saveTG{抌}{54012}
\saveTG{搕}{54012}
\saveTG{撓}{54012}
\saveTG{拋}{54012}
\saveTG{𢯽}{54012}
\saveTG{𢵖}{54012}
\saveTG{𨋄}{54012}
\saveTG{𨍰}{54012}
\saveTG{𨋛}{54012}
\saveTG{𢷪}{54013}
\saveTG{䡜}{54014}
\saveTG{㨒}{54014}
\saveTG{挂}{54014}
\saveTG{擡}{54014}
\saveTG{奿}{54014}
\saveTG{𢯅}{54014}
\saveTG{㩲}{54015}
\saveTG{𢹵}{54015}
\saveTG{搉}{54015}
\saveTG{𢳩}{54015}
\saveTG{㨷}{54015}
\saveTG{揸}{54016}
\saveTG{掩}{54016}
\saveTG{𢺍}{54016}
\saveTG{㧥}{54017}
\saveTG{扏}{54017}
\saveTG{𢱓}{54017}
\saveTG{𢶻}{54017}
\saveTG{𨌙}{54017}
\saveTG{𨍠}{54017}
\saveTG{𨎬}{54017}
\saveTG{𢪎}{54017}
\saveTG{𢯓}{54017}
\saveTG{𢭽}{54017}
\saveTG{㧯}{54017}
\saveTG{𢬹}{54017}
\saveTG{𨌧}{54017}
\saveTG{抴}{54017}
\saveTG{𢲅}{54017}
\saveTG{𢭾}{54017}
\saveTG{𢯥}{54017}
\saveTG{𢭜}{54017}
\saveTG{𢩾}{54017}
\saveTG{軌}{54017}
\saveTG{𢲠}{54017}
\saveTG{𢭪}{54017}
\saveTG{𢫌}{54017}
\saveTG{𢯘}{54017}
\saveTG{撎}{54018}
\saveTG{𢵺}{54018}
\saveTG{𨍜}{54018}
\saveTG{揕}{54018}
\saveTG{𢫅}{54021}
\saveTG{輢}{54021}
\saveTG{掎}{54021}
\saveTG{𢸳}{54024}
\saveTG{𢬋}{54024}
\saveTG{𢺊}{54027}
\saveTG{𢫉}{54027}
\saveTG{拗}{54027}
\saveTG{軪}{54027}
\saveTG{抪}{54027}
\saveTG{扐}{54027}
\saveTG{摕}{54027}
\saveTG{轕}{54027}
\saveTG{擖}{54027}
\saveTG{拷}{54027}
\saveTG{挎}{54027}
\saveTG{搚}{54027}
\saveTG{捞}{54027}
\saveTG{轥}{54027}
\saveTG{抐}{54027}
\saveTG{軜}{54027}
\saveTG{揇}{54027}
\saveTG{撱}{54027}
\saveTG{拹}{54027}
\saveTG{𢰒}{54027}
\saveTG{𠡯}{54027}
\saveTG{𨋝}{54027}
\saveTG{𨏼}{54027}
\saveTG{𨎄}{54027}
\saveTG{𨋞}{54027}
\saveTG{䡃}{54027}
\saveTG{𢳝}{54027}
\saveTG{𢲊}{54027}
\saveTG{𢺃}{54027}
\saveTG{㨅}{54027}
\saveTG{㨺}{54027}
\saveTG{㨚}{54027}
\saveTG{𢵘}{54027}
\saveTG{𢵗}{54027}
\saveTG{𢺤}{54027}
\saveTG{𢮷}{54027}
\saveTG{𢱪}{54027}
\saveTG{𢷁}{54027}
\saveTG{𢭟}{54027}
\saveTG{𢬱}{54027}
\saveTG{𪯄}{54027}
\saveTG{𢮌}{54027}
\saveTG{㧆}{54027}
\saveTG{𢫭}{54027}
\saveTG{𢳾}{54027}
\saveTG{㨥}{54027}
\saveTG{㧑}{54027}
\saveTG{𦘓}{54027}
\saveTG{𫖔}{54027}
\saveTG{𨌣}{54027}
\saveTG{𢮻}{54027}
\saveTG{㨊}{54027}
\saveTG{𢲄}{54027}
\saveTG{𢯨}{54027}
\saveTG{𢯎}{54027}
\saveTG{𢬾}{54027}
\saveTG{𪭬}{54027}
\saveTG{𢹜}{54027}
\saveTG{𨊭}{54030}
\saveTG{軚}{54030}
\saveTG{挝}{54030}
\saveTG{𢺉}{54031}
\saveTG{𢷓}{54031}
\saveTG{𢺐}{54031}
\saveTG{𢯖}{54031}
\saveTG{𪮎}{54031}
\saveTG{𢫿}{54031}
\saveTG{捇}{54031}
\saveTG{𢸺}{54031}
\saveTG{𢺭}{54031}
\saveTG{㩚}{54032}
\saveTG{𢳑}{54032}
\saveTG{𢶰}{54032}
\saveTG{㨬}{54032}
\saveTG{𢵇}{54032}
\saveTG{抾}{54032}
\saveTG{轅}{54032}
\saveTG{𪮼}{54032}
\saveTG{𢹗}{54034}
\saveTG{𠁺}{54035}
\saveTG{𨏕}{54035}
\saveTG{撻}{54035}
\saveTG{𢸚}{54035}
\saveTG{𢳥}{54036}
\saveTG{𩏼}{54037}
\saveTG{䡌}{54037}
\saveTG{挞}{54038}
\saveTG{𢯧}{54040}
\saveTG{䡈}{54040}
\saveTG{𢬨}{54040}
\saveTG{擣}{54041}
\saveTG{𢪄}{54041}
\saveTG{持}{54041}
\saveTG{搑}{54041}
\saveTG{}{54041}
\saveTG{𢱕}{54042}
\saveTG{𨍷}{54042}
\saveTG{𪮛}{54043}
\saveTG{𢱜}{54043}
\saveTG{𣡮}{54043}
\saveTG{𢰆}{54043}
\saveTG{𨏫}{54043}
\saveTG{𢳠}{54044}
\saveTG{𢭛}{54044}
\saveTG{捹}{54044}
\saveTG{𢱺}{54047}
\saveTG{輘}{54047}
\saveTG{𢯠}{54047}
\saveTG{拵}{54047}
\saveTG{掕}{54047}
\saveTG{披}{54047}
\saveTG{擭}{54047}
\saveTG{技}{54047}
\saveTG{𨎮}{54047}
\saveTG{䡋}{54047}
\saveTG{𢭦}{54047}
\saveTG{抜}{54047}
\saveTG{挬}{54047}
\saveTG{𢱭}{54048}
\saveTG{𢳎}{54048}
\saveTG{𢹕}{54048}
\saveTG{𢱂}{54051}
\saveTG{㩢}{54053}
\saveTG{𢰝}{54053}
\saveTG{䡸}{54053}
\saveTG{𢶬}{54053}
\saveTG{撶}{54054}
\saveTG{𢸆}{54055}
\saveTG{𢰎}{54055}
\saveTG{𢯹}{54056}
\saveTG{𢮇}{54057}
\saveTG{𢯷}{54057}
\saveTG{描}{54060}
\saveTG{𢫈}{54060}
\saveTG{輺}{54060}
\saveTG{軲}{54060}
\saveTG{𢯐}{54060}
\saveTG{拮}{54061}
\saveTG{轖}{54061}
\saveTG{𪮬}{54061}
\saveTG{𨌒}{54061}
\saveTG{搘}{54061}
\saveTG{措}{54061}
\saveTG{捁}{54061}
\saveTG{搭}{54061}
\saveTG{𢳭}{54062}
\saveTG{𢰥}{54063}
\saveTG{𢬸}{54064}
\saveTG{𢵝}{54064}
\saveTG{㨋}{54064}
\saveTG{𢱁}{54064}
\saveTG{𢱗}{54064}
\saveTG{撦}{54064}
\saveTG{擆}{54064}
\saveTG{掿}{54064}
\saveTG{𢵻}{54064}
\saveTG{𢸓}{54068}
\saveTG{拑}{54070}
\saveTG{𢱝}{54072}
\saveTG{𨏁}{54080}
\saveTG{軑}{54080}
\saveTG{𢪂}{54080}
\saveTG{𢷸}{54080}
\saveTG{𢺗}{54081}
\saveTG{䡩}{54081}
\saveTG{𢷇}{54081}
\saveTG{搷}{54081}
\saveTG{𪮙}{54081}
\saveTG{掑}{54081}
\saveTG{輁}{54081}
\saveTG{拱}{54081}
\saveTG{𢲛}{54082}
\saveTG{𪭴}{54082}
\saveTG{𢷐}{54082}
\saveTG{𢷟}{54082}
\saveTG{攒}{54082}
\saveTG{䡹}{54082}
\saveTG{𨏡}{54082}
\saveTG{𢲧}{54082}
\saveTG{𢪯}{54083}
\saveTG{𢸅}{54084}
\saveTG{摤}{54084}
\saveTG{摸}{54084}
\saveTG{𢱖}{54084}
\saveTG{𢺀}{54084}
\saveTG{𨍞}{54085}
\saveTG{𢰶}{54085}
\saveTG{𨎔}{54085}
\saveTG{𢴁}{54085}
\saveTG{𨏘}{54086}
\saveTG{轒}{54086}
\saveTG{𨏔}{54086}
\saveTG{𢷺}{54086}
\saveTG{撗}{54086}
\saveTG{𨎩}{54086}
\saveTG{𢴢}{54086}
\saveTG{䡽}{54086}
\saveTG{攢}{54086}
\saveTG{𫏶}{54088}
\saveTG{𢲯}{54088}
\saveTG{挾}{54088}
\saveTG{拻}{54089}
\saveTG{𢬺}{54090}
\saveTG{𢪮}{54090}
\saveTG{𢫩}{54090}
\saveTG{㨆}{54090}
\saveTG{𢯒}{54091}
\saveTG{攃}{54091}
\saveTG{捺}{54091}
\saveTG{𢱢}{54093}
\saveTG{𢺌}{54094}
\saveTG{𢵛}{54094}
\saveTG{𢵳}{54094}
\saveTG{搽}{54094}
\saveTG{揲}{54094}
\saveTG{𢱴}{54094}
\saveTG{擛}{54094}
\saveTG{𢫬}{54094}
\saveTG{𨍕}{54094}
\saveTG{𢴉}{54094}
\saveTG{㨲}{54094}
\saveTG{𢵙}{54094}
\saveTG{𢺇}{54094}
\saveTG{𢭒}{54094}
\saveTG{㩰}{54094}
\saveTG{𢵵}{54095}
\saveTG{轑}{54096}
\saveTG{撩}{54096}
\saveTG{𢷌}{54096}
\saveTG{𢰀}{54098}
\saveTG{𢵭}{54098}
\saveTG{𢴙}{54098}
\saveTG{蚪}{54100}
\saveTG{蝌}{54100}
\saveTG{蚹}{54100}
\saveTG{𠢫}{54102}
\saveTG{𠢦}{54102}
\saveTG{䖞}{54103}
\saveTG{𫋚}{54103}
\saveTG{蚁}{54103}
\saveTG{𧐇}{54103}
\saveTG{𧐝}{54103}
\saveTG{𫊥}{54110}
\saveTG{虵}{54112}
\saveTG{𧰚}{54112}
\saveTG{䗘}{54112}
\saveTG{𧕭}{54112}
\saveTG{蟯}{54112}
\saveTG{𧰟}{54112}
\saveTG{蛯}{54112}
\saveTG{蛙}{54114}
\saveTG{蝰}{54114}
\saveTG{𧌉}{54114}
\saveTG{蠸}{54115}
\saveTG{螼}{54115}
\saveTG{𧋽}{54117}
\saveTG{𧏰}{54117}
\saveTG{𧌺}{54117}
\saveTG{𧌄}{54117}
\saveTG{𧊼}{54117}
\saveTG{䗁}{54121}
\saveTG{螮}{54127}
\saveTG{蛕}{54127}
\saveTG{蛠}{54127}
\saveTG{蠇}{54127}
\saveTG{螨}{54127}
\saveTG{蟎}{54127}
\saveTG{蝻}{54127}
\saveTG{蚴}{54127}
\saveTG{蚋}{54127}
\saveTG{蜹}{54127}
\saveTG{蟏}{54127}
\saveTG{蠨}{54127}
\saveTG{𠢾}{54127}
\saveTG{䗶}{54127}
\saveTG{䘈}{54127}
\saveTG{𫋛}{54127}
\saveTG{䗖}{54127}
\saveTG{䖷}{54127}
\saveTG{蜐}{54127}
\saveTG{𧉩}{54127}
\saveTG{𧋒}{54131}
\saveTG{𫋩}{54131}
\saveTG{𧋺}{54131}
\saveTG{𧔦}{54131}
\saveTG{𫋣}{54131}
\saveTG{𧉧}{54131}
\saveTG{𧈽}{54131}
\saveTG{蠓}{54132}
\saveTG{蟽}{54135}
\saveTG{𧓶}{54135}
\saveTG{䗢}{54136}
\saveTG{𧎦}{54138}
\saveTG{𧏥}{54138}
\saveTG{𧐞}{54141}
\saveTG{𫊵}{54141}
\saveTG{𧋪}{54141}
\saveTG{䗀}{54142}
\saveTG{𫋤}{54142}
\saveTG{𧎋}{54143}
\saveTG{𧎔}{54144}
\saveTG{蠎}{54144}
\saveTG{𧋢}{54147}
\saveTG{蠖}{54147}
\saveTG{蚑}{54147}
\saveTG{𧋫}{54147}
\saveTG{𧋃}{54147}
\saveTG{蚾}{54147}
\saveTG{蟒}{54148}
\saveTG{𧕨}{54151}
\saveTG{𧓡}{54151}
\saveTG{蠛}{54153}
\saveTG{𧑍}{54154}
\saveTG{𧍫}{54157}
\saveTG{𧑑}{54158}
\saveTG{𧍏}{54160}
\saveTG{蝫}{54160}
\saveTG{蠩}{54160}
\saveTG{蛄}{54160}
\saveTG{𧑱}{54161}
\saveTG{𫋓}{54161}
\saveTG{𧒗}{54161}
\saveTG{蛣}{54161}
\saveTG{𧋓}{54161}
\saveTG{蜡}{54161}
\saveTG{螧}{54161}
\saveTG{蟢}{54161}
\saveTG{𧍗}{54164}
\saveTG{蠴}{54164}
\saveTG{𧒇}{54164}
\saveTG{蚶}{54170}
\saveTG{𧋄}{54181}
\saveTG{蜞}{54181}
\saveTG{𧋨}{54182}
\saveTG{𧉑}{54183}
\saveTG{𧏜}{54184}
\saveTG{𧏂}{54184}
\saveTG{䗮}{54184}
\saveTG{蟆}{54184}
\saveTG{𧍩}{54184}
\saveTG{𧕥}{54184}
\saveTG{𧋞}{54184}
\saveTG{蝧}{54185}
\saveTG{蟥}{54186}
\saveTG{𧓛}{54186}
\saveTG{𧔖}{54186}
\saveTG{蟦}{54186}
\saveTG{𧍳}{54187}
\saveTG{蛺}{54188}
\saveTG{蚞}{54190}
\saveTG{𧎳}{54193}
\saveTG{蝶}{54194}
\saveTG{䗋}{54194}
\saveTG{𧓜}{54194}
\saveTG{𧔕}{54194}
\saveTG{𫊷}{54194}
\saveTG{蝾}{54194}
\saveTG{}{54194}
\saveTG{蠂}{54194}
\saveTG{蟟}{54196}
\saveTG{𧍍}{54198}
\saveTG{𧍠}{54198}
\saveTG{𩇠}{54212}
\saveTG{𩇜}{54217}
\saveTG{𧇨}{54219}
\saveTG{𧊘}{54227}
\saveTG{𩇡}{54241}
\saveTG{𢺷}{54247}
\saveTG{䨼}{54247}
\saveTG{𩇥}{54247}
\saveTG{𠵳}{54247}
\saveTG{𫜘}{54286}
\saveTG{𡭇}{54327}
\saveTG{𪃠}{54327}
\saveTG{𪬈}{54332}
\saveTG{𢛱}{54332}
\saveTG{𤋰}{54332}
\saveTG{𩹚}{54336}
\saveTG{斠}{54400}
\saveTG{𦕥}{54401}
\saveTG{𦕴}{54401}
\saveTG{㪹}{54403}
\saveTG{𡰍}{54417}
\saveTG{𠢉}{54427}
\saveTG{𩋮}{54500}
\saveTG{𪮣}{54521}
\saveTG{𢺴}{54522}
\saveTG{𢶯}{54527}
\saveTG{𢺏}{54528}
\saveTG{𪯃}{54532}
\saveTG{䪝}{54547}
\saveTG{𨍝}{54556}
\saveTG{𣍕}{54617}
\saveTG{𢆅}{54627}
\saveTG{𣍚}{54647}
\saveTG{𢻅}{54647}
\saveTG{𤉕}{54689}
\saveTG{㔗}{54727}
\saveTG{𪨵}{54772}
\saveTG{𡗻}{54780}
\saveTG{𡙧}{54801}
\saveTG{𣂐}{54803}
\saveTG{𪰯}{54804}
\saveTG{𡘢}{54805}
\saveTG{勣}{54827}
\saveTG{𪟝}{54827}
\saveTG{𧴧}{54827}
\saveTG{𠢥}{54827}
\saveTG{𢺻}{54847}
\saveTG{𤿶}{54847}
\saveTG{㿮}{54847}
\saveTG{𦆫}{54903}
\saveTG{㩽}{54904}
\saveTG{𦔏}{54912}
\saveTG{𥠙}{54912}
\saveTG{䎨}{54915}
\saveTG{𠡫}{54927}
\saveTG{𪎐}{54927}
\saveTG{勅}{54927}
\saveTG{𦔚}{54927}
\saveTG{𠡾}{54927}
\saveTG{𣔩}{54927}
\saveTG{𠡠}{54927}
\saveTG{𣝮}{54927}
\saveTG{耡}{54927}
\saveTG{耢}{54927}
\saveTG{𦓷}{54931}
\saveTG{𫅼}{54935}
\saveTG{䎭}{54936}
\saveTG{𦓮}{54940}
\saveTG{𪱴}{54940}
\saveTG{𦔋}{54941}
\saveTG{𦔠}{54946}
\saveTG{耯}{54947}
\saveTG{耚}{54947}
\saveTG{𦔌}{54961}
\saveTG{耤}{54961}
\saveTG{𣗓}{54964}
\saveTG{𪥊}{54980}
\saveTG{𦓳}{54981}
\saveTG{𦓿}{54981}
\saveTG{䎯}{54985}
\saveTG{𦔃}{54985}
\saveTG{𧆄}{54994}
\saveTG{𦓹}{54998}
\saveTG{扙}{55000}
\saveTG{𢪋}{55000}
\saveTG{丼}{55000}
\saveTG{井}{55000}
\saveTG{𪭝}{55002}
\saveTG{𢵕}{55002}
\saveTG{𣲜}{55002}
\saveTG{𢪧}{55002}
\saveTG{㐩}{55005}
\saveTG{𢪝}{55005}
\saveTG{𢪠}{55006}
\saveTG{𢷦}{55006}
\saveTG{𪮈}{55006}
\saveTG{𪭶}{55006}
\saveTG{丳}{55006}
\saveTG{抻}{55006}
\saveTG{拽}{55006}
\saveTG{捙}{55006}
\saveTG{𠦱}{55006}
\saveTG{𨎼}{55006}
\saveTG{𨏟}{55006}
\saveTG{𨏬}{55006}
\saveTG{𨏆}{55006}
\saveTG{䡛}{55006}
\saveTG{𨋙}{55006}
\saveTG{𨋯}{55006}
\saveTG{𨏲}{55006}
\saveTG{𨎓}{55006}
\saveTG{𨏰}{55006}
\saveTG{𨍙}{55006}
\saveTG{𢬓}{55007}
\saveTG{扥}{55010}
\saveTG{挠}{55012}
\saveTG{𢱮}{55012}
\saveTG{㩇}{55016}
\saveTG{𢯀}{55017}
\saveTG{軘}{55017}
\saveTG{𢦗}{55017}
\saveTG{扽}{55017}
\saveTG{𢳊}{55017}
\saveTG{执}{55017}
\saveTG{𢵶}{55017}
\saveTG{𢴇}{55017}
\saveTG{𢹿}{55018}
\saveTG{㩋}{55024}
\saveTG{𢴘}{55027}
\saveTG{𪮋}{55027}
\saveTG{拂}{55027}
\saveTG{弗}{55027}
\saveTG{𨍘}{55027}
\saveTG{𢭿}{55027}
\saveTG{掅}{55027}
\saveTG{輤}{55027}
\saveTG{𨋥}{55027}
\saveTG{㧊}{55027}
\saveTG{𢴆}{55027}
\saveTG{摙}{55030}
\saveTG{𢫆}{55030}
\saveTG{轋}{55030}
\saveTG{𢮨}{55031}
\saveTG{𢰁}{55031}
\saveTG{𢳨}{55031}
\saveTG{𢮛}{55031}
\saveTG{𢫼}{55031}
\saveTG{擃}{55032}
\saveTG{㧼}{55032}
\saveTG{攮}{55032}
\saveTG{抟}{55032}
\saveTG{𢴥}{55033}
\saveTG{𢰯}{55033}
\saveTG{𨎥}{55033}
\saveTG{㩨}{55037}
\saveTG{䡺}{55037}
\saveTG{𢸦}{55038}
\saveTG{𢳪}{55039}
\saveTG{揵}{55040}
\saveTG{𨌐}{55042}
\saveTG{轉}{55043}
\saveTG{𢷬}{55043}
\saveTG{摶}{55043}
\saveTG{摟}{55044}
\saveTG{捿}{55044}
\saveTG{𢫇}{55047}
\saveTG{𢳞}{55047}
\saveTG{𫖕}{55047}
\saveTG{搆}{55047}
\saveTG{抩}{55047}
\saveTG{捜}{55047}
\saveTG{𢵜}{55052}
\saveTG{撵}{55054}
\saveTG{𢺩}{55056}
\saveTG{攆}{55056}
\saveTG{𨍛}{55057}
\saveTG{捧}{55058}
\saveTG{𢺒}{55058}
\saveTG{軸}{55060}
\saveTG{抽}{55060}
\saveTG{𢳏}{55060}
\saveTG{𨎈}{55061}
\saveTG{𨍟}{55061}
\saveTG{𢬑}{55065}
\saveTG{𢲵}{55066}
\saveTG{𨎝}{55066}
\saveTG{𢰦}{55068}
\saveTG{𢸝}{55068}
\saveTG{𢯰}{55074}
\saveTG{𢫫}{55074}
\saveTG{㨹}{55077}
\saveTG{摏}{55077}
\saveTG{轊}{55077}
\saveTG{抰}{55080}
\saveTG{䡍}{55080}
\saveTG{軮}{55080}
\saveTG{抶}{55080}
\saveTG{軼}{55080}
\saveTG{扶}{55080}
\saveTG{挟}{55080}
\saveTG{抉}{55080}
\saveTG{捷}{55081}
\saveTG{𢶝}{55081}
\saveTG{𠁵}{55081}
\saveTG{捵}{55081}
\saveTG{𪮞}{55082}
\saveTG{挗}{55082}
\saveTG{𨋸}{55082}
\saveTG{𢯵}{55082}
\saveTG{𢭴}{55082}
\saveTG{𢭯}{55082}
\saveTG{揍}{55084}
\saveTG{輳}{55084}
\saveTG{攅}{55086}
\saveTG{撌}{55086}
\saveTG{𢵟}{55086}
\saveTG{𨎾}{55086}
\saveTG{㩌}{55086}
\saveTG{𨎨}{55086}
\saveTG{抺}{55090}
\saveTG{抹}{55090}
\saveTG{𢬗}{55090}
\saveTG{拺}{55092}
\saveTG{𨋵}{55092}
\saveTG{𢲬}{55092}
\saveTG{㧣}{55092}
\saveTG{㨞}{55093}
\saveTG{𢸩}{55094}
\saveTG{搩}{55094}
\saveTG{搸}{55094}
\saveTG{拣}{55094}
\saveTG{轃}{55094}
\saveTG{𢸨}{55094}
\saveTG{朄}{55096}
\saveTG{𨌛}{55096}
\saveTG{𣌾}{55096}
\saveTG{𨌿}{55096}
\saveTG{𢴟}{55096}
\saveTG{𢵽}{55096}
\saveTG{𢱠}{55096}
\saveTG{𢷜}{55096}
\saveTG{捒}{55096}
\saveTG{揀}{55096}
\saveTG{𢸀}{55099}
\saveTG{捸}{55099}
\saveTG{𢰬}{55099}
\saveTG{蚌}{55100}
\saveTG{𠀎}{55100}
\saveTG{𧓞}{55100}
\saveTG{𧉖}{55102}
\saveTG{𥁖}{55102}
\saveTG{𢑹}{55102}
\saveTG{垫}{55104}
\saveTG{坓}{55104}
\saveTG{㻃}{55104}
\saveTG{𫊳}{55106}
\saveTG{𧊍}{55106}
\saveTG{𧍇}{55106}
\saveTG{蛼}{55106}
\saveTG{𧊋}{55106}
\saveTG{蚛}{55106}
\saveTG{𧏶}{55106}
\saveTG{𧊣}{55106}
\saveTG{𧊐}{55107}
\saveTG{𧍶}{55107}
\saveTG{𣍈}{55108}
\saveTG{豊}{55108}
\saveTG{𧯮}{55108}
\saveTG{﨡}{55110}
\saveTG{蛲}{55112}
\saveTG{𧏱}{55112}
\saveTG{𧈮}{55117}
\saveTG{𧉙}{55117}
\saveTG{𧕬}{55118}
\saveTG{𫋠}{55118}
\saveTG{𧑛}{55124}
\saveTG{𦑓}{55127}
\saveTG{𠢞}{55127}
\saveTG{𧉸}{55127}
\saveTG{蜻}{55127}
\saveTG{蟰}{55127}
\saveTG{鸷}{55127}
\saveTG{𦐡}{55127}
\saveTG{𧖤}{55131}
\saveTG{𧏎}{55131}
\saveTG{𧕒}{55131}
\saveTG{䖵}{55131}
\saveTG{𧓅}{55132}
\saveTG{𧖒}{55132}
\saveTG{蟪}{55133}
\saveTG{䗭}{55136}
\saveTG{蛰}{55136}
\saveTG{𧕐}{55136}
\saveTG{𧐖}{55136}
\saveTG{𧒭}{55138}
\saveTG{𧔥}{55138}
\saveTG{𧑔}{55139}
\saveTG{𧐒}{55139}
\saveTG{𧐕}{55143}
\saveTG{𧓆}{55143}
\saveTG{𫋂}{55144}
\saveTG{螻}{55144}
\saveTG{𦘤}{55145}
\saveTG{蚺}{55147}
\saveTG{蝳}{55157}
\saveTG{蜯}{55158}
\saveTG{蛐}{55160}
\saveTG{蚰}{55160}
\saveTG{𧑠}{55162}
\saveTG{螬}{55166}
\saveTG{𣉿}{55166}
\saveTG{蝽}{55168}
\saveTG{𧐍}{55177}
\saveTG{彗}{55177}
\saveTG{蛱}{55180}
\saveTG{蚗}{55180}
\saveTG{蛈}{55180}
\saveTG{蚨}{55180}
\saveTG{蜨}{55181}
\saveTG{𧌎}{55181}
\saveTG{𧐥}{55181}
\saveTG{𧰅}{55182}
\saveTG{𧎯}{55182}
\saveTG{蛦}{55182}
\saveTG{𧋿}{55182}
\saveTG{𫊬}{55182}
\saveTG{𧎝}{55184}
\saveTG{𧑋}{55186}
\saveTG{𧐐}{55186}
\saveTG{𧑈}{55186}
\saveTG{𧊠}{55186}
\saveTG{𧋆}{55190}
\saveTG{蛛}{55190}
\saveTG{𧉿}{55190}
\saveTG{𧊸}{55192}
\saveTG{螦}{55193}
\saveTG{𧔬}{55194}
\saveTG{螓}{55194}
\saveTG{𧎩}{55194}
\saveTG{𧍴}{55196}
\saveTG{𧋐}{55196}
\saveTG{𫋨}{55196}
\saveTG{蝀}{55196}
\saveTG{𧌾}{55201}
\saveTG{𫕹}{55210}
\saveTG{𠓆}{55217}
\saveTG{𠒹}{55217}
\saveTG{𠙐}{55217}
\saveTG{𫌡}{55217}
\saveTG{𠒔}{55217}
\saveTG{䐌}{55227}
\saveTG{𦘨}{55227}
\saveTG{𥝃}{55227}
\saveTG{𩰿}{55227}
\saveTG{𫕻}{55227}
\saveTG{𢂏}{55227}
\saveTG{𧹘}{55230}
\saveTG{農}{55232}
\saveTG{𣲴}{55232}
\saveTG{𣸕}{55232}
\saveTG{𪱘}{55244}
\saveTG{𧭃}{55261}
\saveTG{𪲬}{55292}
\saveTG{𩤰}{55327}
\saveTG{𢜎}{55330}
\saveTG{𪬇}{55330}
\saveTG{热}{55331}
\saveTG{𪬿}{55332}
\saveTG{𢘍}{55332}
\saveTG{𢣡}{55333}
\saveTG{㤟}{55336}
\saveTG{慧}{55337}
\saveTG{𢣱}{55339}
\saveTG{𡬭}{55346}
\saveTG{𫜑}{55400}
\saveTG{𠦈}{55400}
\saveTG{𤯽}{55401}
\saveTG{𡥳}{55407}
\saveTG{𪽂}{55412}
\saveTG{𠢠}{55427}
\saveTG{势}{55427}
\saveTG{𠡂}{55427}
\saveTG{冓}{55447}
\saveTG{㜖}{55466}
\saveTG{麸}{55480}
\saveTG{𠦒}{55500}
\saveTG{挚}{55502}
\saveTG{𫐋}{55504}
\saveTG{辇}{55504}
\saveTG{輦}{55506}
\saveTG{𠁷}{55556}
\saveTG{𨏿}{55556}
\saveTG{𩌉}{55599}
\saveTG{曲}{55600}
\saveTG{𤾞}{55602}
\saveTG{𥄱}{55602}
\saveTG{𥋅}{55603}
\saveTG{𨠷}{55604}
\saveTG{𧟷}{55604}
\saveTG{𨣹}{55604}
\saveTG{𨤈}{55604}
\saveTG{曹}{55606}
\saveTG{𣊄}{55607}
\saveTG{𣇺}{55608}
\saveTG{替}{55608}
\saveTG{𣍘}{55609}
\saveTG{𣋕}{55658}
\saveTG{𧖌}{55662}
\saveTG{𣊛}{55666}
\saveTG{𨐄}{55666}
\saveTG{𣌠}{55668}
\saveTG{𤱁}{55680}
\saveTG{𤲣}{55681}
\saveTG{𤱉}{55682}
\saveTG{𣍖}{55696}
\saveTG{𣮝}{55715}
\saveTG{𢎵}{55727}
\saveTG{𢏪}{55727}
\saveTG{𤲀}{55727}
\saveTG{𢏇}{55727}
\saveTG{𠔘}{55731}
\saveTG{𣍅}{55765}
\saveTG{典}{55801}
\saveTG{费}{55802}
\saveTG{贽}{55802}
\saveTG{𡘉}{55804}
\saveTG{賛}{55806}
\saveTG{費}{55806}
\saveTG{僰}{55809}
\saveTG{熭}{55809}
\saveTG{𤇝}{55809}
\saveTG{㚘}{55880}
\saveTG{𤒍}{55889}
\saveTG{𤒖}{55899}
\saveTG{𤏡}{55899}
\saveTG{耕}{55900}
\saveTG{汬}{55902}
\saveTG{𥿏}{55903}
\saveTG{絷}{55903}
\saveTG{𣓵}{55904}
\saveTG{𣍃}{55906}
\saveTG{𣓤}{55917}
\saveTG{𦔖}{55936}
\saveTG{𣓝}{55942}
\saveTG{𦔪}{55944}
\saveTG{耬}{55944}
\saveTG{耩}{55947}
\saveTG{𥺠}{55949}
\saveTG{𪳭}{55956}
\saveTG{𣒠}{55965}
\saveTG{𫅻}{55965}
\saveTG{𫅽}{55968}
\saveTG{𡘮}{55982}
\saveTG{耫}{55986}
\saveTG{𫅿}{55986}
\saveTG{棘}{55992}
\saveTG{𣡍}{55992}
\saveTG{𣘐}{55993}
\saveTG{𥡀}{55994}
\saveTG{𪲸}{55996}
\saveTG{𣗥}{55996}
\saveTG{㯥}{55996}
\saveTG{𨽼}{55999}
\saveTG{𢯙}{56000}
\saveTG{抇}{56000}
\saveTG{掴}{56000}
\saveTG{摑}{56000}
\saveTG{攌}{56000}
\saveTG{捆}{56000}
\saveTG{𨋳}{56000}
\saveTG{𨍲}{56000}
\saveTG{䡒}{56000}
\saveTG{𢪷}{56000}
\saveTG{㧽}{56000}
\saveTG{㧢}{56000}
\saveTG{㨡}{56000}
\saveTG{𢪾}{56000}
\saveTG{𢬼}{56000}
\saveTG{㩛}{56000}
\saveTG{𢬻}{56000}
\saveTG{𢮖}{56000}
\saveTG{扣}{56000}
\saveTG{拁}{56000}
\saveTG{𢰋}{56000}
\saveTG{𫏴}{56002}
\saveTG{𢫥}{56002}
\saveTG{𢭇}{56002}
\saveTG{拍}{56002}
\saveTG{𢯟}{56010}
\saveTG{𢭱}{56010}
\saveTG{担}{56010}
\saveTG{𢬒}{56011}
\saveTG{輼}{56012}
\saveTG{搵}{56012}
\saveTG{揾}{56012}
\saveTG{軦}{56012}
\saveTG{摫}{56012}
\saveTG{拀}{56012}
\saveTG{擺}{56012}
\saveTG{𨐁}{56012}
\saveTG{㩸}{56012}
\saveTG{㩹}{56012}
\saveTG{𪮆}{56012}
\saveTG{𢱫}{56012}
\saveTG{掍}{56012}
\saveTG{挸}{56012}
\saveTG{輥}{56012}
\saveTG{韫}{56012}
\saveTG{轀}{56012}
\saveTG{𢹮}{56014}
\saveTG{挰}{56014}
\saveTG{揘}{56014}
\saveTG{捏}{56014}
\saveTG{𢶫}{56014}
\saveTG{𨍧}{56014}
\saveTG{捚}{56015}
\saveTG{㩴}{56015}
\saveTG{𣊼}{56015}
\saveTG{攞}{56015}
\saveTG{𢲪}{56015}
\saveTG{𨎅}{56016}
\saveTG{㨪}{56017}
\saveTG{𢷕}{56017}
\saveTG{䡚}{56017}
\saveTG{𨍹}{56017}
\saveTG{𠄺}{56017}
\saveTG{挹}{56017}
\saveTG{邫}{56017}
\saveTG{𢬐}{56017}
\saveTG{𢵅}{56017}
\saveTG{𢹧}{56017}
\saveTG{𢮧}{56021}
\saveTG{擤}{56021}
\saveTG{掦}{56027}
\saveTG{𪯀}{56027}
\saveTG{𢹅}{56027}
\saveTG{䡻}{56027}
\saveTG{揚}{56027}
\saveTG{𣈱}{56027}
\saveTG{捐}{56027}
\saveTG{𪭦}{56027}
\saveTG{𢷒}{56027}
\saveTG{輰}{56027}
\saveTG{搨}{56027}
\saveTG{𢪶}{56027}
\saveTG{揭}{56027}
\saveTG{拐}{56027}
\saveTG{輵}{56027}
\saveTG{暢}{56027}
\saveTG{𢮺}{56027}
\saveTG{𢯇}{56027}
\saveTG{掲}{56027}
\saveTG{𢸼}{56027}
\saveTG{𢯮}{56027}
\saveTG{𢯤}{56027}
\saveTG{𢶔}{56027}
\saveTG{𢬢}{56027}
\saveTG{擉}{56027}
\saveTG{𨌉}{56027}
\saveTG{揌}{56030}
\saveTG{𢵃}{56030}
\saveTG{䡯}{56030}
\saveTG{𢰌}{56030}
\saveTG{摁}{56030}
\saveTG{摠}{56030}
\saveTG{㩏}{56031}
\saveTG{𢺖}{56032}
\saveTG{𨏙}{56032}
\saveTG{摆}{56032}
\saveTG{轘}{56032}
\saveTG{𢶢}{56032}
\saveTG{擐}{56032}
\saveTG{揋}{56032}
\saveTG{𢸃}{56032}
\saveTG{𢷑}{56033}
\saveTG{𨏌}{56033}
\saveTG{摠}{56033}
\saveTG{𢰐}{56034}
\saveTG{𨏀}{56036}
\saveTG{摾}{56036}
\saveTG{𢺱}{56039}
\saveTG{𢮆}{56040}
\saveTG{𫖓}{56040}
\saveTG{捭}{56040}
\saveTG{𢫵}{56040}
\saveTG{𨎦}{56041}
\saveTG{𢶓}{56041}
\saveTG{揖}{56041}
\saveTG{輯}{56041}
\saveTG{擇}{56041}
\saveTG{捍}{56041}
\saveTG{𪮦}{56043}
\saveTG{㧹}{56043}
\saveTG{𢮅}{56044}
\saveTG{攖}{56044}
\saveTG{𢱰}{56044}
\saveTG{𢰘}{56044}
\saveTG{𢲋}{56044}
\saveTG{䡟}{56045}
\saveTG{䡬}{56047}
\saveTG{𨌦}{56047}
\saveTG{𢸶}{56047}
\saveTG{𨏹}{56047}
\saveTG{𢱻}{56047}
\saveTG{𢱩}{56047}
\saveTG{摱}{56047}
\saveTG{攫}{56047}
\saveTG{撮}{56047}
\saveTG{𢺘}{56048}
\saveTG{𢬽}{56050}
\saveTG{押}{56050}
\saveTG{𢳂}{56054}
\saveTG{䡲}{56056}
\saveTG{撣}{56056}
\saveTG{轠}{56060}
\saveTG{攂}{56060}
\saveTG{𢮵}{56060}
\saveTG{𨍆}{56060}
\saveTG{捛}{56060}
\saveTG{撂}{56064}
\saveTG{𨎏}{56064}
\saveTG{𢱧}{56071}
\saveTG{𢭲}{56080}
\saveTG{軹}{56080}
\saveTG{抧}{56080}
\saveTG{提}{56081}
\saveTG{捉}{56081}
\saveTG{𨌓}{56082}
\saveTG{𩏿}{56082}
\saveTG{损}{56082}
\saveTG{𣉄}{56082}
\saveTG{𢮚}{56084}
\saveTG{𢫸}{56084}
\saveTG{捑}{56084}
\saveTG{搝}{56084}
\saveTG{損}{56086}
\saveTG{𢵯}{56086}
\saveTG{𢲗}{56091}
\saveTG{摞}{56093}
\saveTG{𢺢}{56093}
\saveTG{捰}{56094}
\saveTG{𢱌}{56094}
\saveTG{㨐}{56094}
\saveTG{操}{56094}
\saveTG{𢹯}{56094}
\saveTG{輠}{56094}
\saveTG{𪭷}{56094}
\saveTG{𢸠}{56095}
\saveTG{撔}{56096}
\saveTG{㩧}{56099}
\saveTG{𧊁}{56100}
\saveTG{蝈}{56100}
\saveTG{𧋕}{56100}
\saveTG{𧋡}{56100}
\saveTG{𧊀}{56100}
\saveTG{𧓘}{56100}
\saveTG{𧊭}{56100}
\saveTG{蛔}{56100}
\saveTG{𧏈}{56100}
\saveTG{𧉫}{56100}
\saveTG{蜘}{56100}
\saveTG{蟈}{56100}
\saveTG{蚎}{56100}
\saveTG{蜠}{56100}
\saveTG{蜖}{56100}
\saveTG{𫋘}{56100}
\saveTG{𧒪}{56100}
\saveTG{𧋙}{56102}
\saveTG{𡒮}{56104}
\saveTG{䖧}{56110}
\saveTG{蝹}{56112}
\saveTG{螕}{56112}
\saveTG{蜫}{56112}
\saveTG{蜆}{56112}
\saveTG{螝}{56113}
\saveTG{𧋸}{56114}
\saveTG{𧕫}{56114}
\saveTG{蝗}{56114}
\saveTG{蟶}{56114}
\saveTG{䘃}{56114}
\saveTG{䗌}{56115}
\saveTG{蠷}{56115}
\saveTG{𧋎}{56115}
\saveTG{蝇}{56116}
\saveTG{蝿}{56116}
\saveTG{𧊫}{56117}
\saveTG{𧋾}{56117}
\saveTG{𧏩}{56117}
\saveTG{䗑}{56118}
\saveTG{𧓧}{56121}
\saveTG{蜴}{56127}
\saveTG{蜎}{56127}
\saveTG{蠋}{56127}
\saveTG{𧋱}{56127}
\saveTG{蝪}{56127}
\saveTG{𧒹}{56127}
\saveTG{𧒞}{56127}
\saveTG{𧊅}{56127}
\saveTG{𧓭}{56127}
\saveTG{𥃞}{56127}
\saveTG{𧍞}{56127}
\saveTG{蝎}{56127}
\saveTG{𧍪}{56127}
\saveTG{蝟}{56127}
\saveTG{蜗}{56127}
\saveTG{𧎁}{56128}
\saveTG{𧏀}{56130}
\saveTG{𧍤}{56130}
\saveTG{螅}{56130}
\saveTG{蟌}{56130}
\saveTG{蟔}{56131}
\saveTG{𧑩}{56132}
\saveTG{蠉}{56132}
\saveTG{𧎖}{56132}
\saveTG{𧔷}{56132}
\saveTG{𧍥}{56132}
\saveTG{𧔘}{56132}
\saveTG{䗾}{56133}
\saveTG{蜱}{56140}
\saveTG{蠌}{56141}
\saveTG{𧎎}{56142}
\saveTG{𧍋}{56143}
\saveTG{蠳}{56144}
\saveTG{蜱}{56145}
\saveTG{蠼}{56147}
\saveTG{𧏧}{56147}
\saveTG{𧍚}{56147}
\saveTG{蟃}{56147}
\saveTG{䖬}{56150}
\saveTG{𧑷}{56152}
\saveTG{𧏻}{56154}
\saveTG{蟬}{56156}
\saveTG{蠝}{56160}
\saveTG{蝐}{56160}
\saveTG{䗉}{56160}
\saveTG{𧐋}{56164}
\saveTG{𧒊}{56168}
\saveTG{蛽}{56180}
\saveTG{𧑌}{56181}
\saveTG{蝭}{56181}
\saveTG{𧋥}{56182}
\saveTG{𧓝}{56184}
\saveTG{蜈}{56184}
\saveTG{螑}{56184}
\saveTG{𧌚}{56189}
\saveTG{螺}{56193}
\saveTG{𧒮}{56194}
\saveTG{蜾}{56194}
\saveTG{𧊄}{56196}
\saveTG{𧑊}{56196}
\saveTG{𧔙}{56199}
\saveTG{靚}{56212}
\saveTG{𧠎}{56217}
\saveTG{𧠥}{56217}
\saveTG{䙻}{56217}
\saveTG{𠸮}{56227}
\saveTG{鬹}{56227}
\saveTG{𠧃}{56245}
\saveTG{𠷓}{56280}
\saveTG{𪄯}{56327}
\saveTG{𢤙}{56334}
\saveTG{𢤎}{56334}
\saveTG{嫢}{56404}
\saveTG{𡦑}{56407}
\saveTG{覯}{56412}
\saveTG{𧢃}{56417}
\saveTG{𤽝}{56418}
\saveTG{𫜓}{56427}
\saveTG{𢍁}{56440}
\saveTG{𫜔}{56445}
\saveTG{𤱣}{56450}
\saveTG{𥔏}{56602}
\saveTG{𧡲}{56612}
\saveTG{𩳽}{56613}
\saveTG{𤾳}{56614}
\saveTG{𩳄}{56617}
\saveTG{𤳸}{56617}
\saveTG{𤳄}{56631}
\saveTG{𨀲}{56682}
\saveTG{𣍀}{56694}
\saveTG{㕀}{56712}
\saveTG{䚎}{56712}
\saveTG{𧠤}{56717}
\saveTG{𨚭}{56717}
\saveTG{𠔔}{56800}
\saveTG{𨆥}{56802}
\saveTG{𡙭}{56804}
\saveTG{𤍮}{56809}
\saveTG{規}{56812}
\saveTG{覥}{56812}
\saveTG{䚉}{56817}
\saveTG{䚍}{56817}
\saveTG{𧢔}{56817}
\saveTG{𨜣}{56817}
\saveTG{耞}{56900}
\saveTG{𫅹}{56900}
\saveTG{𦓲}{56900}
\saveTG{𦓾}{56900}
\saveTG{𫀪}{56900}
\saveTG{槼}{56904}
\saveTG{䎱}{56912}
\saveTG{𦓼}{56912}
\saveTG{}{56912}
\saveTG{䰤}{56913}
\saveTG{𦓵}{56915}
\saveTG{𦔬}{56915}
\saveTG{𧠵}{56917}
\saveTG{𧡴}{56917}
\saveTG{𣙹}{56917}
\saveTG{𧡍}{56917}
\saveTG{䙿}{56917}
\saveTG{𫌫}{56917}
\saveTG{𦔨}{56927}
\saveTG{𦓻}{56927}
\saveTG{耦}{56927}
\saveTG{䎬}{56931}
\saveTG{𦔧}{56932}
\saveTG{}{56932}
\saveTG{䎪}{56941}
\saveTG{𦔥}{56941}
\saveTG{𡭂}{56943}
\saveTG{𦓸}{56945}
\saveTG{𦔔}{56947}
\saveTG{𦔎}{56947}
\saveTG{𦔕}{56947}
\saveTG{𦔜}{56981}
\saveTG{𦔂}{56982}
\saveTG{𦔈}{56984}
\saveTG{𦔐}{56986}
\saveTG{軐}{57010}
\saveTG{𢦚}{57010}
\saveTG{𢦕}{57010}
\saveTG{㧩}{57010}
\saveTG{𢱚}{57010}
\saveTG{軓}{57010}
\saveTG{扟}{57010}
\saveTG{𢩰}{57010}
\saveTG{𢩧}{57011}
\saveTG{𪭹}{57011}
\saveTG{抯}{57012}
\saveTG{輓}{57012}
\saveTG{抳}{57012}
\saveTG{挽}{57012}
\saveTG{軳}{57012}
\saveTG{揑}{57012}
\saveTG{輗}{57012}
\saveTG{掜}{57012}
\saveTG{擝}{57012}
\saveTG{掹}{57012}
\saveTG{攪}{57012}
\saveTG{扭}{57012}
\saveTG{抱}{57012}
\saveTG{𠷏}{57012}
\saveTG{𢪕}{57012}
\saveTG{𢹳}{57012}
\saveTG{𢭚}{57012}
\saveTG{𢹋}{57012}
\saveTG{𢸔}{57012}
\saveTG{𢳷}{57012}
\saveTG{㨕}{57012}
\saveTG{𢬟}{57012}
\saveTG{𢶣}{57012}
\saveTG{𢭌}{57012}
\saveTG{𢬔}{57012}
\saveTG{䡕}{57013}
\saveTG{攙}{57013}
\saveTG{𨋀}{57014}
\saveTG{𢫞}{57014}
\saveTG{𢯑}{57014}
\saveTG{𢮢}{57014}
\saveTG{摼}{57014}
\saveTG{握}{57014}
\saveTG{軽}{57014}
\saveTG{捤}{57014}
\saveTG{擢}{57015}
\saveTG{䡄}{57017}
\saveTG{撧}{57017}
\saveTG{艳}{57017}
\saveTG{𨋭}{57017}
\saveTG{𢩫}{57017}
\saveTG{𢩵}{57017}
\saveTG{𢩽}{57017}
\saveTG{𨋗}{57017}
\saveTG{𢪚}{57017}
\saveTG{𨊠}{57017}
\saveTG{𨊹}{57017}
\saveTG{𨊲}{57017}
\saveTG{𨊶}{57017}
\saveTG{𪮖}{57017}
\saveTG{𢭨}{57017}
\saveTG{𢪨}{57017}
\saveTG{𢬘}{57017}
\saveTG{㧪}{57017}
\saveTG{𢫎}{57017}
\saveTG{𢲭}{57017}
\saveTG{把}{57017}
\saveTG{艴}{57017}
\saveTG{㨮}{57017}
\saveTG{𢯂}{57017}
\saveTG{𤰷}{57017}
\saveTG{𫏸}{57017}
\saveTG{拯}{57019}
\saveTG{掏}{57020}
\saveTG{輞}{57020}
\saveTG{撊}{57020}
\saveTG{抑}{57020}
\saveTG{拥}{57020}
\saveTG{挧}{57020}
\saveTG{抈}{57020}
\saveTG{攔}{57020}
\saveTG{掬}{57020}
\saveTG{揤}{57020}
\saveTG{輷}{57020}
\saveTG{揈}{57020}
\saveTG{軥}{57020}
\saveTG{拘}{57020}
\saveTG{抅}{57020}
\saveTG{擱}{57020}
\saveTG{搁}{57020}
\saveTG{掆}{57020}
\saveTG{挏}{57020}
\saveTG{扚}{57020}
\saveTG{掤}{57020}
\saveTG{抝}{57020}
\saveTG{𢷃}{57020}
\saveTG{𢯃}{57020}
\saveTG{㧕}{57020}
\saveTG{𢬃}{57020}
\saveTG{𢭙}{57020}
\saveTG{𨍖}{57020}
\saveTG{𨋖}{57020}
\saveTG{扪}{57020}
\saveTG{𨋕}{57020}
\saveTG{㓞}{57020}
\saveTG{𪭠}{57020}
\saveTG{}{57020}
\saveTG{抣}{57020}
\saveTG{畃}{57020}
\saveTG{搠}{57020}
\saveTG{撋}{57020}
\saveTG{軔}{57020}
\saveTG{扨}{57020}
\saveTG{韧}{57020}
\saveTG{輣}{57020}
\saveTG{捫}{57020}
\saveTG{輖}{57020}
\saveTG{𠛝}{57021}
\saveTG{𠜥}{57021}
\saveTG{㧇}{57021}
\saveTG{𢮀}{57021}
\saveTG{㧅}{57021}
\saveTG{𣌨}{57021}
\saveTG{𨊸}{57021}
\saveTG{𨊷}{57021}
\saveTG{𨎫}{57021}
\saveTG{𢴿}{57021}
\saveTG{𪮡}{57021}
\saveTG{𪮯}{57021}
\saveTG{𨋋}{57021}
\saveTG{𨋾}{57021}
\saveTG{𢰮}{57021}
\saveTG{𨏦}{57021}
\saveTG{𪭜}{57022}
\saveTG{𢪃}{57022}
\saveTG{𢪱}{57022}
\saveTG{𢩶}{57022}
\saveTG{𦑬}{57022}
\saveTG{𨋊}{57022}
\saveTG{𢸌}{57022}
\saveTG{轇}{57022}
\saveTG{摎}{57022}
\saveTG{抒}{57022}
\saveTG{𪭾}{57022}
\saveTG{𨊵}{57023}
\saveTG{𢹶}{57023}
\saveTG{𢰹}{57023}
\saveTG{𢯊}{57023}
\saveTG{𪮰}{57023}
\saveTG{𨊾}{57024}
\saveTG{㧏}{57024}
\saveTG{𢪣}{57024}
\saveTG{𢵱}{57024}
\saveTG{𢺛}{57024}
\saveTG{𢪼}{57024}
\saveTG{𨋹}{57024}
\saveTG{𢰄}{57024}
\saveTG{𨋦}{57024}
\saveTG{㨄}{57026}
\saveTG{𢵧}{57026}
\saveTG{䡘}{57026}
\saveTG{𢲰}{57026}
\saveTG{𢸛}{57026}
\saveTG{𢴌}{57026}
\saveTG{㧦}{57026}
\saveTG{𨍐}{57027}
\saveTG{㧈}{57027}
\saveTG{𢰕}{57027}
\saveTG{𢶈}{57027}
\saveTG{𢸴}{57027}
\saveTG{𢭵}{57027}
\saveTG{𪯁}{57027}
\saveTG{𨌖}{57027}
\saveTG{𢹾}{57027}
\saveTG{𢺄}{57027}
\saveTG{𢫤}{57027}
\saveTG{𢲍}{57027}
\saveTG{𢵴}{57027}
\saveTG{𢰭}{57027}
\saveTG{𢵮}{57027}
\saveTG{𨌨}{57027}
\saveTG{𨍋}{57027}
\saveTG{摀}{57027}
\saveTG{捅}{57027}
\saveTG{掃}{57027}
\saveTG{扔}{57027}
\saveTG{搦}{57027}
\saveTG{挪}{57027}
\saveTG{捔}{57027}
\saveTG{揟}{57027}
\saveTG{挶}{57027}
\saveTG{搰}{57027}
\saveTG{搗}{57027}
\saveTG{捣}{57027}
\saveTG{搊}{57027}
\saveTG{拸}{57027}
\saveTG{畅}{57027}
\saveTG{挷}{57027}
\saveTG{𨙷}{57027}
\saveTG{𨚢}{57027}
\saveTG{𣅚}{57027}
\saveTG{𨛩}{57027}
\saveTG{䡔}{57027}
\saveTG{𨝨}{57027}
\saveTG{𨙰}{57027}
\saveTG{𢵋}{57027}
\saveTG{𢰗}{57027}
\saveTG{𢴄}{57027}
\saveTG{𢰸}{57027}
\saveTG{𪮪}{57027}
\saveTG{𪇪}{57027}
\saveTG{𢰻}{57027}
\saveTG{㨶}{57027}
\saveTG{𪁀}{57027}
\saveTG{扬}{57027}
\saveTG{邦}{57027}
\saveTG{揶}{57027}
\saveTG{㨝}{57027}
\saveTG{捓}{57027}
\saveTG{𢸜}{57027}
\saveTG{掷}{57027}
\saveTG{擲}{57027}
\saveTG{𢺡}{57027}
\saveTG{𢮼}{57027}
\saveTG{𪮂}{57027}
\saveTG{𢰓}{57027}
\saveTG{𢴐}{57027}
\saveTG{𢷢}{57027}
\saveTG{𢸞}{57027}
\saveTG{㨯}{57027}
\saveTG{𢲲}{57027}
\saveTG{𨙲}{57027}
\saveTG{𩾷}{57027}
\saveTG{𩾺}{57027}
\saveTG{𪀕}{57027}
\saveTG{𩿀}{57027}
\saveTG{𪀎}{57027}
\saveTG{𨍾}{57027}
\saveTG{䡧}{57027}
\saveTG{𫛒}{57027}
\saveTG{𢩪}{57027}
\saveTG{㨛}{57028}
\saveTG{𢱳}{57028}
\saveTG{𨏭}{57029}
\saveTG{𢵁}{57029}
\saveTG{𢵢}{57029}
\saveTG{刱}{57030}
\saveTG{𢳱}{57031}
\saveTG{𢭄}{57031}
\saveTG{𢮋}{57032}
\saveTG{𢫒}{57032}
\saveTG{𢷝}{57032}
\saveTG{𨎢}{57032}
\saveTG{㧾}{57032}
\saveTG{𢭝}{57032}
\saveTG{𢮞}{57032}
\saveTG{𪮱}{57032}
\saveTG{𢫺}{57032}
\saveTG{𢶑}{57032}
\saveTG{𢷻}{57032}
\saveTG{搌}{57032}
\saveTG{𨍉}{57032}
\saveTG{撾}{57032}
\saveTG{揔}{57032}
\saveTG{輾}{57032}
\saveTG{拫}{57032}
\saveTG{掾}{57032}
\saveTG{𨋨}{57032}
\saveTG{𢳟}{57032}
\saveTG{𢱸}{57033}
\saveTG{𢫝}{57033}
\saveTG{搀}{57033}
\saveTG{𢴋}{57033}
\saveTG{𢹌}{57035}
\saveTG{摓}{57035}
\saveTG{𨏂}{57036}
\saveTG{𢳶}{57036}
\saveTG{䡫}{57036}
\saveTG{攭}{57036}
\saveTG{掻}{57036}
\saveTG{搔}{57036}
\saveTG{搥}{57037}
\saveTG{𢰽}{57037}
\saveTG{𢷫}{57037}
\saveTG{𢸷}{57038}
\saveTG{掫}{57040}
\saveTG{}{57040}
\saveTG{掓}{57040}
\saveTG{輈}{57040}
\saveTG{輙}{57040}
\saveTG{𢩲}{57041}
\saveTG{㧶}{57041}
\saveTG{摒}{57041}
\saveTG{拠}{57041}
\saveTG{搱}{57041}
\saveTG{𨌈}{57041}
\saveTG{摪}{57042}
\saveTG{𪭺}{57043}
\saveTG{扠}{57043}
\saveTG{𢪡}{57044}
\saveTG{撄}{57044}
\saveTG{𪮴}{57044}
\saveTG{𠂴}{57044}
\saveTG{𢱲}{57044}
\saveTG{𢰟}{57044}
\saveTG{𪭪}{57044}
\saveTG{𨍍}{57044}
\saveTG{𢫍}{57045}
\saveTG{𢪤}{57045}
\saveTG{撏}{57046}
\saveTG{𢶙}{57047}
\saveTG{㩭}{57047}
\saveTG{𢯫}{57047}
\saveTG{𢳲}{57047}
\saveTG{𢪩}{57047}
\saveTG{𢬶}{57047}
\saveTG{𢱥}{57047}
\saveTG{𨊿}{57047}
\saveTG{𨍺}{57047}
\saveTG{𢲈}{57047}
\saveTG{𢷳}{57047}
\saveTG{𩐀}{57047}
\saveTG{䡑}{57047}
\saveTG{𨋚}{57047}
\saveTG{𢱈}{57047}
\saveTG{𢮒}{57047}
\saveTG{搬}{57047}
\saveTG{报}{57047}
\saveTG{扱}{57047}
\saveTG{投}{57047}
\saveTG{輟}{57047}
\saveTG{掇}{57047}
\saveTG{抿}{57047}
\saveTG{摋}{57047}
\saveTG{搜}{57047}
\saveTG{軗}{57047}
\saveTG{挦}{57047}
\saveTG{轏}{57047}
\saveTG{𨍪}{57047}
\saveTG{𨋁}{57047}
\saveTG{𢬏}{57047}
\saveTG{𢫓}{57047}
\saveTG{𢲩}{57047}
\saveTG{𨍿}{57047}
\saveTG{𡥅}{57047}
\saveTG{𢩴}{57047}
\saveTG{𢫡}{57047}
\saveTG{𢱷}{57047}
\saveTG{㩔}{57047}
\saveTG{𢵔}{57047}
\saveTG{𢴼}{57047}
\saveTG{𢪈}{57050}
\saveTG{拇}{57050}
\saveTG{𢺟}{57051}
\saveTG{𢶷}{57052}
\saveTG{𢴡}{57052}
\saveTG{㩮}{57052}
\saveTG{揮}{57052}
\saveTG{𢲞}{57052}
\saveTG{𩏽}{57054}
\saveTG{捀}{57054}
\saveTG{挥}{57054}
\saveTG{择}{57054}
\saveTG{䡣}{57056}
\saveTG{㩣}{57056}
\saveTG{挣}{57057}
\saveTG{𨋪}{57057}
\saveTG{𢭎}{57057}
\saveTG{攑}{57058}
\saveTG{摨}{57059}
\saveTG{𪮥}{57061}
\saveTG{𪮇}{57061}
\saveTG{㨱}{57061}
\saveTG{𨎻}{57061}
\saveTG{撸}{57061}
\saveTG{擔}{57061}
\saveTG{㨨}{57062}
\saveTG{摺}{57062}
\saveTG{招}{57062}
\saveTG{𢯾}{57062}
\saveTG{軺}{57062}
\saveTG{𨍸}{57062}
\saveTG{𢬷}{57062}
\saveTG{𨏗}{57063}
\saveTG{擼}{57063}
\saveTG{𨍌}{57064}
\saveTG{挌}{57064}
\saveTG{輅}{57064}
\saveTG{据}{57064}
\saveTG{𢷅}{57064}
\saveTG{㨉}{57064}
\saveTG{𠲱}{57064}
\saveTG{𨎲}{57064}
\saveTG{𨎟}{57064}
\saveTG{𪮔}{57064}
\saveTG{𢰲}{57067}
\saveTG{捃}{57067}
\saveTG{輑}{57067}
\saveTG{𨍢}{57068}
\saveTG{𢱆}{57068}
\saveTG{𢮮}{57070}
\saveTG{扫}{57070}
\saveTG{㧮}{57070}
\saveTG{掘}{57072}
\saveTG{搖}{57072}
\saveTG{𢭈}{57072}
\saveTG{𢫏}{57072}
\saveTG{𨊼}{57074}
\saveTG{𢬞}{57074}
\saveTG{𨋱}{57077}
\saveTG{輡}{57077}
\saveTG{掐}{57077}
\saveTG{}{57077}
\saveTG{𢶵}{57079}
\saveTG{𨎁}{57080}
\saveTG{㨠}{57080}
\saveTG{𢸾}{57081}
\saveTG{𢰅}{57081}
\saveTG{𧽇}{57081}
\saveTG{擬}{57081}
\saveTG{𠁹}{57081}
\saveTG{𢪵}{57081}
\saveTG{𢮭}{57081}
\saveTG{𢰺}{57081}
\saveTG{𢷣}{57081}
\saveTG{𨍄}{57081}
\saveTG{𢷲}{57081}
\saveTG{𢸯}{57081}
\saveTG{撰}{57081}
\saveTG{𣤖}{57082}
\saveTG{𨋰}{57082}
\saveTG{𢶆}{57082}
\saveTG{𢭻}{57082}
\saveTG{𢸈}{57082}
\saveTG{𢶖}{57082}
\saveTG{𢰑}{57082}
\saveTG{𢷗}{57082}
\saveTG{𢹤}{57082}
\saveTG{𣢘}{57082}
\saveTG{掼}{57082}
\saveTG{扻}{57082}
\saveTG{揿}{57082}
\saveTG{撳}{57082}
\saveTG{軟}{57082}
\saveTG{摗}{57082}
\saveTG{擹}{57082}
\saveTG{掀}{57082}
\saveTG{擨}{57082}
\saveTG{𪮶}{57082}
\saveTG{揳}{57084}
\saveTG{換}{57084}
\saveTG{换}{57084}
\saveTG{擙}{57084}
\saveTG{𢰡}{57084}
\saveTG{𥈑}{57084}
\saveTG{𢰼}{57084}
\saveTG{摜}{57086}
\saveTG{𢸒}{57086}
\saveTG{𢵒}{57086}
\saveTG{𢶦}{57086}
\saveTG{攋}{57086}
\saveTG{択}{57087}
\saveTG{𢭑}{57087}
\saveTG{𢸂}{57089}
\saveTG{𢹰}{57089}
\saveTG{𪮭}{57089}
\saveTG{摖}{57091}
\saveTG{𨋎}{57092}
\saveTG{𪭧}{57092}
\saveTG{𢴲}{57093}
\saveTG{搡}{57094}
\saveTG{探}{57094}
\saveTG{𪮨}{57094}
\saveTG{𢳚}{57094}
\saveTG{𢷂}{57094}
\saveTG{挅}{57094}
\saveTG{挆}{57094}
\saveTG{揉}{57094}
\saveTG{輮}{57094}
\saveTG{䡦}{57094}
\saveTG{𢮑}{57099}
\saveTG{𩐒}{57102}
\saveTG{𪾑}{57102}
\saveTG{䪡}{57102}
\saveTG{䪠}{57102}
\saveTG{墼}{57104}
\saveTG{𡐊}{57104}
\saveTG{𤩯}{57104}
\saveTG{𡊷}{57104}
\saveTG{蛜}{57107}
\saveTG{𨩕}{57109}
\saveTG{𦘣}{57110}
\saveTG{𧌡}{57110}
\saveTG{𧍯}{57110}
\saveTG{虮}{57110}
\saveTG{𧈝}{57111}
\saveTG{蛆}{57112}
\saveTG{蚫}{57112}
\saveTG{𫋟}{57112}
\saveTG{䖡}{57112}
\saveTG{蛫}{57112}
\saveTG{蜢}{57112}
\saveTG{蚬}{57112}
\saveTG{蜺}{57112}
\saveTG{蚭}{57112}
\saveTG{𧊴}{57113}
\saveTG{蛏}{57114}
\saveTG{𧎜}{57114}
\saveTG{蠗}{57115}
\saveTG{𧍰}{57115}
\saveTG{𧑟}{57115}
\saveTG{𧓰}{57117}
\saveTG{𧕃}{57117}
\saveTG{𧋦}{57117}
\saveTG{蚬}{57117}
\saveTG{𪩰}{57117}
\saveTG{蚆}{57117}
\saveTG{艶}{57117}
\saveTG{蠅}{57117}
\saveTG{䖠}{57117}
\saveTG{𠙫}{57117}
\saveTG{𧌳}{57117}
\saveTG{𧑾}{57117}
\saveTG{𫊧}{57120}
\saveTG{䖼}{57120}
\saveTG{𧌞}{57120}
\saveTG{𧌋}{57120}
\saveTG{𧍕}{57120}
\saveTG{虭}{57120}
\saveTG{蜔}{57120}
\saveTG{蜩}{57120}
\saveTG{蚼}{57120}
\saveTG{蝴}{57120}
\saveTG{蝍}{57120}
\saveTG{虳}{57120}
\saveTG{蚐}{57120}
\saveTG{蜪}{57120}
\saveTG{蝄}{57120}
\saveTG{蛡}{57120}
\saveTG{蚏}{57120}
\saveTG{蛧}{57120}
\saveTG{𦑍}{57120}
\saveTG{𧈳}{57121}
\saveTG{𧌇}{57121}
\saveTG{𧉠}{57121}
\saveTG{𧍑}{57121}
\saveTG{𧋤}{57121}
\saveTG{𧋈}{57122}
\saveTG{𧉚}{57122}
\saveTG{蟉}{57122}
\saveTG{𧈷}{57123}
\saveTG{𧎒}{57123}
\saveTG{𧖆}{57123}
\saveTG{𧊊}{57124}
\saveTG{𧈿}{57124}
\saveTG{𧊚}{57126}
\saveTG{𧒄}{57126}
\saveTG{䖲}{57126}
\saveTG{䖮}{57126}
\saveTG{𧍂}{57127}
\saveTG{𧐲}{57127}
\saveTG{䗛}{57127}
\saveTG{𧒺}{57127}
\saveTG{鄷}{57127}
\saveTG{蝸}{57127}
\saveTG{螖}{57127}
\saveTG{螂}{57127}
\saveTG{蚂}{57127}
\saveTG{蛥}{57127}
\saveTG{蠾}{57127}
\saveTG{螐}{57127}
\saveTG{蝑}{57127}
\saveTG{蠮}{57127}
\saveTG{蛹}{57127}
\saveTG{𧎟}{57127}
\saveTG{𧓸}{57127}
\saveTG{𧐾}{57127}
\saveTG{𧌴}{57127}
\saveTG{𧑏}{57127}
\saveTG{𧈣}{57127}
\saveTG{𧁄}{57127}
\saveTG{𫋄}{57127}
\saveTG{𧌈}{57127}
\saveTG{𧐹}{57127}
\saveTG{𧐙}{57127}
\saveTG{𧎷}{57127}
\saveTG{𧍛}{57127}
\saveTG{𧑐}{57127}
\saveTG{䳋}{57127}
\saveTG{𪂉}{57127}
\saveTG{𧔛}{57127}
\saveTG{𧒑}{57127}
\saveTG{𩐋}{57128}
\saveTG{䗇}{57129}
\saveTG{𧕗}{57129}
\saveTG{𧎇}{57131}
\saveTG{𧐫}{57131}
\saveTG{𧎓}{57131}
\saveTG{𧋷}{57131}
\saveTG{𧒬}{57132}
\saveTG{𧒖}{57132}
\saveTG{蛝}{57132}
\saveTG{蝝}{57132}
\saveTG{蟓}{57132}
\saveTG{𧐺}{57132}
\saveTG{䗓}{57132}
\saveTG{𧑄}{57132}
\saveTG{𧓏}{57132}
\saveTG{𧓪}{57132}
\saveTG{𧊂}{57133}
\saveTG{𧋅}{57133}
\saveTG{螁}{57133}
\saveTG{𫋢}{57135}
\saveTG{䗦}{57135}
\saveTG{𧑝}{57135}
\saveTG{䗨}{57136}
\saveTG{𧑨}{57136}
\saveTG{蟿}{57136}
\saveTG{蛪}{57136}
\saveTG{蚒}{57140}
\saveTG{𫊾}{57141}
\saveTG{𧎨}{57141}
\saveTG{𧊓}{57144}
\saveTG{蟳}{57146}
\saveTG{蝦}{57147}
\saveTG{螋}{57147}
\saveTG{𧉐}{57147}
\saveTG{𧋶}{57147}
\saveTG{𧌓}{57147}
\saveTG{𧒼}{57147}
\saveTG{𧈻}{57147}
\saveTG{𧓾}{57147}
\saveTG{𧏫}{57147}
\saveTG{𧏘}{57147}
\saveTG{𧎚}{57147}
\saveTG{𧌗}{57147}
\saveTG{𫊻}{57147}
\saveTG{虸}{57147}
\saveTG{𫊩}{57147}
\saveTG{𧉬}{57147}
\saveTG{𧓇}{57147}
\saveTG{蝃}{57147}
\saveTG{𧊨}{57150}
\saveTG{蚦}{57150}
\saveTG{蠏}{57152}
\saveTG{𧉯}{57154}
\saveTG{蜂}{57154}
\saveTG{𧎊}{57156}
\saveTG{𧊵}{57157}
\saveTG{𧌰}{57157}
\saveTG{蟾}{57161}
\saveTG{䗜}{57162}
\saveTG{𧐔}{57162}
\saveTG{𧍌}{57162}
\saveTG{蛁}{57162}
\saveTG{𧔎}{57163}
\saveTG{𧐣}{57164}
\saveTG{𧒌}{57164}
\saveTG{𧕪}{57164}
\saveTG{蛒}{57164}
\saveTG{蜛}{57164}
\saveTG{蝞}{57167}
\saveTG{𧌫}{57170}
\saveTG{𧌑}{57172}
\saveTG{蜬}{57172}
\saveTG{𧎼}{57172}
\saveTG{𧉅}{57174}
\saveTG{蜭}{57177}
\saveTG{螟}{57180}
\saveTG{蟤}{57181}
\saveTG{㰲}{57182}
\saveTG{𣤁}{57182}
\saveTG{𣤐}{57182}
\saveTG{𧐁}{57182}
\saveTG{㰩}{57182}
\saveTG{𧏗}{57182}
\saveTG{𧐄}{57182}
\saveTG{蠍}{57182}
\saveTG{𧓗}{57182}
\saveTG{𫋦}{57184}
\saveTG{𧒳}{57184}
\saveTG{𧎑}{57184}
\saveTG{䗔}{57184}
\saveTG{蝜}{57186}
\saveTG{蠀}{57186}
\saveTG{䠭}{57186}
\saveTG{䗰}{57186}
\saveTG{𧓒}{57186}
\saveTG{𧓩}{57186}
\saveTG{𧔣}{57186}
\saveTG{蚇}{57187}
\saveTG{𧉰}{57192}
\saveTG{𩐑}{57192}
\saveTG{𧓙}{57194}
\saveTG{𧒷}{57194}
\saveTG{𫋇}{57194}
\saveTG{𧑇}{57194}
\saveTG{蟂}{57194}
\saveTG{螩}{57194}
\saveTG{蝚}{57194}
\saveTG{𧊶}{57194}
\saveTG{𧊱}{57194}
\saveTG{𧏠}{57194}
\saveTG{𧌍}{57199}
\saveTG{𩐊}{57212}
\saveTG{䪢}{57212}
\saveTG{靓}{57212}
\saveTG{翘}{57212}
\saveTG{䪟}{57212}
\saveTG{䪣}{57212}
\saveTG{靘}{57217}
\saveTG{𧈗}{57217}
\saveTG{𧈌}{57217}
\saveTG{𪅵}{57217}
\saveTG{𪞁}{57217}
\saveTG{翉}{57220}
\saveTG{𫕸}{57221}
\saveTG{䎘}{57221}
\saveTG{𦚨}{57222}
\saveTG{鶄}{57227}
\saveTG{鹔}{57227}
\saveTG{鷫}{57227}
\saveTG{觢}{57227}
\saveTG{郬}{57227}
\saveTG{𨙶}{57227}
\saveTG{䴖}{57227}
\saveTG{𪆯}{57227}
\saveTG{䳙}{57227}
\saveTG{𦣓}{57227}
\saveTG{帮}{57227}
\saveTG{幚}{57227}
\saveTG{鵏}{57227}
\saveTG{郙}{57227}
\saveTG{鬶}{57227}
\saveTG{}{57227}
\saveTG{郕}{57227}
\saveTG{𣫦}{57241}
\saveTG{𪵐}{57247}
\saveTG{𧓬}{57247}
\saveTG{𤝢}{57254}
\saveTG{静}{57257}
\saveTG{歗}{57282}
\saveTG{𢴪}{57282}
\saveTG{𤎓}{57289}
\saveTG{𦑵}{57321}
\saveTG{鷒}{57327}
\saveTG{𫑘}{57327}
\saveTG{𪈐}{57327}
\saveTG{鄟}{57327}
\saveTG{𪀡}{57327}
\saveTG{𪒨}{57331}
\saveTG{恝}{57332}
\saveTG{𢡴}{57332}
\saveTG{𢛑}{57332}
\saveTG{𢢞}{57334}
\saveTG{𩽓}{57336}
\saveTG{𢜌}{57336}
\saveTG{𪬯}{57338}
\saveTG{𢤿}{57338}
\saveTG{𡬨}{57342}
\saveTG{𨙛}{57382}
\saveTG{}{57404}
\saveTG{觏}{57412}
\saveTG{䶲}{57417}
\saveTG{𫜳}{57417}
\saveTG{𪓚}{57417}
\saveTG{𪎉}{57417}
\saveTG{麹}{57420}
\saveTG{䎃}{57421}
\saveTG{鷜}{57427}
\saveTG{鶈}{57427}
\saveTG{郪}{57427}
\saveTG{𨜒}{57427}
\saveTG{𡗆}{57427}
\saveTG{䣚}{57427}
\saveTG{𪃺}{57427}
\saveTG{𢍆}{57442}
\saveTG{㛃}{57442}
\saveTG{𡢖}{57444}
\saveTG{𪌛}{57477}
\saveTG{𣤋}{57482}
\saveTG{撃}{57502}
\saveTG{㸷}{57502}
\saveTG{挈}{57502}
\saveTG{擊}{57502}
\saveTG{𢮟}{57502}
\saveTG{𤚹}{57506}
\saveTG{𫑢}{57506}
\saveTG{轚}{57506}
\saveTG{𫐐}{57517}
\saveTG{𨚗}{57527}
\saveTG{𫑠}{57527}
\saveTG{鄻}{57527}
\saveTG{䳞}{57527}
\saveTG{䪗}{57547}
\saveTG{㧐}{57547}
\saveTG{𧫷}{57601}
\saveTG{䛚}{57601}
\saveTG{礊}{57602}
\saveTG{䂮}{57602}
\saveTG{㗉}{57604}
\saveTG{𠿉}{57604}
\saveTG{𨣗}{57604}
\saveTG{𠛽}{57621}
\saveTG{𣌴}{57621}
\saveTG{䣠}{57627}
\saveTG{邮}{57627}
\saveTG{𪂹}{57627}
\saveTG{𩿬}{57627}
\saveTG{𢏢}{57627}
\saveTG{𡳧}{57641}
\saveTG{毄}{57647}
\saveTG{𣪭}{57647}
\saveTG{𣌵}{57647}
\saveTG{𣋨}{57647}
\saveTG{𣍜}{57686}
\saveTG{𤳨}{57699}
\saveTG{𡴉}{57711}
\saveTG{𦟒}{57711}
\saveTG{𥂉}{57712}
\saveTG{𣭭}{57715}
\saveTG{𤮈}{57717}
\saveTG{𤮛}{57717}
\saveTG{㐝}{57717}
\saveTG{㼤}{57717}
\saveTG{𩿰}{57727}
\saveTG{𪅖}{57727}
\saveTG{邨}{57727}
\saveTG{𫑛}{57727}
\saveTG{𨚓}{57727}
\saveTG{𩞌}{57732}
\saveTG{𠟳}{57741}
\saveTG{𣪠}{57747}
\saveTG{𫜩}{57771}
\saveTG{罊}{57772}
\saveTG{𪘂}{57772}
\saveTG{齧}{57772}
\saveTG{𣤢}{57782}
\saveTG{契}{57804}
\saveTG{𡘱}{57805}
\saveTG{𠜵}{57805}
\saveTG{𧷄}{57806}
\saveTG{𩖬}{57810}
\saveTG{规}{57812}
\saveTG{觍}{57812}
\saveTG{䒋}{57817}
\saveTG{𦐋}{57821}
\saveTG{𫛥}{57827}
\saveTG{䣒}{57827}
\saveTG{𩿶}{57827}
\saveTG{鄪}{57827}
\saveTG{邞}{57827}
\saveTG{鳺}{57827}
\saveTG{鴂}{57827}
\saveTG{郏}{57827}
\saveTG{鴺}{57827}
\saveTG{𪄸}{57827}
\saveTG{鄼}{57827}
\saveTG{𪀙}{57827}
\saveTG{𪃈}{57827}
\saveTG{𧷕}{57868}
\saveTG{𣤈}{57882}
\saveTG{𣢧}{57882}
\saveTG{𤉟}{57892}
\saveTG{洯}{57902}
\saveTG{絜}{57903}
\saveTG{繋}{57903}
\saveTG{繫}{57903}
\saveTG{椝}{57904}
\saveTG{檕}{57904}
\saveTG{栔}{57904}
\saveTG{𢦻}{57910}
\saveTG{耝}{57912}
\saveTG{耙}{57917}
\saveTG{𦫕}{57917}
\saveTG{𦒰}{57919}
\saveTG{𦓥}{57920}
\saveTG{𠣩}{57921}
\saveTG{𦐾}{57921}
\saveTG{䋤}{57923}
\saveTG{𨚘}{57927}
\saveTG{𣗧}{57927}
\saveTG{𨛙}{57927}
\saveTG{䎤}{57927}
\saveTG{𪅸}{57927}
\saveTG{𩿲}{57927}
\saveTG{𪅙}{57927}
\saveTG{𪈎}{57927}
\saveTG{𪀜}{57927}
\saveTG{𩿣}{57927}
\saveTG{𫛍}{57927}
\saveTG{𪂩}{57927}
\saveTG{鶒}{57927}
\saveTG{䣢}{57927}
\saveTG{䣂}{57927}
\saveTG{鶫}{57927}
\saveTG{鵣}{57927}
\saveTG{鶇}{57927}
\saveTG{𦓨}{57927}
\saveTG{𧔓}{57931}
\saveTG{𦫎}{57932}
\saveTG{𪳘}{57932}
\saveTG{𦔘}{57941}
\saveTG{𦓭}{57945}
\saveTG{𡥒}{57947}
\saveTG{𦔀}{57947}
\saveTG{耔}{57947}
\saveTG{𫅾}{57962}
\saveTG{𦔢}{57964}
\saveTG{𦃅}{57964}
\saveTG{𦓱}{57964}
\saveTG{耜}{57977}
\saveTG{𦔫}{57981}
\saveTG{欶}{57982}
\saveTG{𤴚}{57982}
\saveTG{𣢹}{57982}
\saveTG{赖}{57982}
\saveTG{賴}{57986}
\saveTG{𦔇}{57994}
\saveTG{扖}{58000}
\saveTG{扒}{58000}
\saveTG{𨊤}{58000}
\saveTG{拟}{58000}
\saveTG{𨋘}{58011}
\saveTG{𢭢}{58011}
\saveTG{𢸿}{58011}
\saveTG{𢬕}{58011}
\saveTG{拃}{58011}
\saveTG{拦}{58011}
\saveTG{抢}{58012}
\saveTG{搓}{58012}
\saveTG{𢷙}{58012}
\saveTG{𢱔}{58012}
\saveTG{𢺙}{58012}
\saveTG{𪮝}{58012}
\saveTG{㨫}{58012}
\saveTG{㩜}{58012}
\saveTG{䡭}{58012}
\saveTG{𪭻}{58012}
\saveTG{𪺷}{58012}
\saveTG{𢬖}{58012}
\saveTG{攬}{58012}
\saveTG{拖}{58012}
\saveTG{抡}{58012}
\saveTG{掽}{58012}
\saveTG{揓}{58012}
\saveTG{挩}{58012}
\saveTG{捝}{58012}
\saveTG{𢫜}{58012}
\saveTG{揽}{58012}
\saveTG{轞}{58012}
\saveTG{搤}{58012}
\saveTG{𢶅}{58012}
\saveTG{拴}{58014}
\saveTG{挫}{58014}
\saveTG{輇}{58014}
\saveTG{𢬝}{58014}
\saveTG{𢺜}{58015}
\saveTG{𢴈}{58016}
\saveTG{𨊰}{58017}
\saveTG{扢}{58017}
\saveTG{㧉}{58017}
\saveTG{㨴}{58017}
\saveTG{𣇋}{58017}
\saveTG{𨌔}{58017}
\saveTG{𢴎}{58017}
\saveTG{𢭉}{58017}
\saveTG{𢺮}{58017}
\saveTG{𢮾}{58017}
\saveTG{𢬁}{58017}
\saveTG{𢹈}{58017}
\saveTG{𢮲}{58017}
\saveTG{捡}{58019}
\saveTG{捦}{58019}
\saveTG{扴}{58020}
\saveTG{擶}{58021}
\saveTG{𢯋}{58021}
\saveTG{𨌺}{58021}
\saveTG{𢷔}{58021}
\saveTG{輸}{58021}
\saveTG{揄}{58021}
\saveTG{揃}{58021}
\saveTG{抮}{58022}
\saveTG{軫}{58022}
\saveTG{𢹱}{58024}
\saveTG{輪}{58027}
\saveTG{掄}{58027}
\saveTG{扮}{58027}
\saveTG{𢶠}{58027}
\saveTG{㩶}{58027}
\saveTG{𨏤}{58027}
\saveTG{𢺣}{58027}
\saveTG{𪮟}{58027}
\saveTG{𢵀}{58027}
\saveTG{𢸄}{58027}
\saveTG{𢭐}{58027}
\saveTG{𢪆}{58027}
\saveTG{㩉}{58027}
\saveTG{㨣}{58027}
\saveTG{𨋂}{58027}
\saveTG{𨎰}{58027}
\saveTG{𨍚}{58027}
\saveTG{𨎃}{58027}
\saveTG{𫖑}{58027}
\saveTG{㨵}{58027}
\saveTG{擳}{58027}
\saveTG{挮}{58027}
\saveTG{摥}{58027}
\saveTG{擒}{58027}
\saveTG{軡}{58027}
\saveTG{扲}{58027}
\saveTG{掵}{58027}
\saveTG{𢶌}{58029}
\saveTG{𢮈}{58030}
\saveTG{撫}{58031}
\saveTG{𢸥}{58031}
\saveTG{䡆}{58031}
\saveTG{𢪌}{58031}
\saveTG{𢰩}{58031}
\saveTG{𢶕}{58032}
\saveTG{軨}{58032}
\saveTG{𢬜}{58032}
\saveTG{㨾}{58032}
\saveTG{𢸢}{58032}
\saveTG{𨍨}{58032}
\saveTG{搇}{58032}
\saveTG{拎}{58032}
\saveTG{攁}{58032}
\saveTG{𢷀}{58032}
\saveTG{捻}{58032}
\saveTG{捴}{58033}
\saveTG{𢷊}{58033}
\saveTG{䡵}{58033}
\saveTG{𨌪}{58033}
\saveTG{𢱶}{58033}
\saveTG{扵}{58033}
\saveTG{𢮁}{58033}
\saveTG{𨌕}{58034}
\saveTG{𨌝}{58034}
\saveTG{𢶞}{58034}
\saveTG{搃}{58036}
\saveTG{搛}{58037}
\saveTG{𢱤}{58038}
\saveTG{𪮫}{58040}
\saveTG{撖}{58040}
\saveTG{𢲳}{58040}
\saveTG{撒}{58040}
\saveTG{撴}{58040}
\saveTG{撤}{58040}
\saveTG{撇}{58040}
\saveTG{𢸑}{58040}
\saveTG{𢪛}{58040}
\saveTG{𢳺}{58040}
\saveTG{𢭮}{58040}
\saveTG{𢳆}{58040}
\saveTG{𢲆}{58040}
\saveTG{𢳯}{58040}
\saveTG{㩤}{58040}
\saveTG{㨖}{58040}
\saveTG{𢰴}{58040}
\saveTG{𨊽}{58040}
\saveTG{擞}{58040}
\saveTG{𪭟}{58040}
\saveTG{𨎞}{58040}
\saveTG{轍}{58040}
\saveTG{擻}{58040}
\saveTG{擏}{58040}
\saveTG{撽}{58040}
\saveTG{軿}{58041}
\saveTG{擀}{58041}
\saveTG{拼}{58041}
\saveTG{龫}{58041}
\saveTG{𢷩}{58041}
\saveTG{𪮷}{58041}
\saveTG{𢵂}{58042}
\saveTG{㩐}{58043}
\saveTG{𨋼}{58043}
\saveTG{𢲑}{58044}
\saveTG{𢱒}{58045}
\saveTG{撙}{58046}
\saveTG{揜}{58046}
\saveTG{𢫹}{58047}
\saveTG{𢰧}{58047}
\saveTG{輹}{58047}
\saveTG{𢬧}{58050}
\saveTG{𢮡}{58051}
\saveTG{𢹛}{58051}
\saveTG{𨋽}{58051}
\saveTG{𢷨}{58052}
\saveTG{𢲡}{58052}
\saveTG{𢹘}{58052}
\saveTG{𢯌}{58052}
\saveTG{轙}{58053}
\saveTG{𢸕}{58053}
\saveTG{𢸽}{58053}
\saveTG{𨏢}{58054}
\saveTG{𢲨}{58054}
\saveTG{轙}{58054}
\saveTG{㩘}{58055}
\saveTG{轙}{58055}
\saveTG{𨏚}{58056}
\saveTG{掸}{58056}
\saveTG{𢹺}{58056}
\saveTG{挴}{58057}
\saveTG{𢷵}{58057}
\saveTG{𪮹}{58058}
\saveTG{𢸤}{58059}
\saveTG{𢵈}{58061}
\saveTG{𢵰}{58061}
\saveTG{𢹽}{58061}
\saveTG{撘}{58061}
\saveTG{摿}{58061}
\saveTG{拾}{58061}
\saveTG{𨍣}{58062}
\saveTG{𢰢}{58062}
\saveTG{𢰈}{58062}
\saveTG{𢷷}{58063}
\saveTG{揂}{58064}
\saveTG{捨}{58064}
\saveTG{輶}{58064}
\saveTG{𢴖}{58064}
\saveTG{撯}{58065}
\saveTG{𢴣}{58066}
\saveTG{𢶒}{58066}
\saveTG{搶}{58067}
\saveTG{輍}{58068}
\saveTG{𢵄}{58068}
\saveTG{𢭧}{58072}
\saveTG{𢴩}{58074}
\saveTG{𢰖}{58077}
\saveTG{𢮪}{58081}
\saveTG{摐}{58081}
\saveTG{𢭀}{58081}
\saveTG{𢹢}{58081}
\saveTG{𨌰}{58082}
\saveTG{㧿}{58082}
\saveTG{䡮}{58082}
\saveTG{𪮲}{58082}
\saveTG{𢵑}{58084}
\saveTG{𢬈}{58084}
\saveTG{𢲂}{58084}
\saveTG{𪭨}{58084}
\saveTG{𢵫}{58084}
\saveTG{𢳇}{58084}
\saveTG{𢮦}{58085}
\saveTG{撿}{58086}
\saveTG{𢺅}{58086}
\saveTG{𢸟}{58086}
\saveTG{擌}{58088}
\saveTG{𢴵}{58089}
\saveTG{𢶨}{58089}
\saveTG{𨋏}{58090}
\saveTG{㩍}{58092}
\saveTG{㩯}{58093}
\saveTG{攥}{58093}
\saveTG{𪮅}{58094}
\saveTG{𠄜}{58094}
\saveTG{𢲢}{58094}
\saveTG{𨌎}{58094}
\saveTG{𢸮}{58094}
\saveTG{𨎸}{58094}
\saveTG{𢶱}{58094}
\saveTG{捈}{58094}
\saveTG{𢲺}{58098}
\saveTG{𧈢}{58100}
\saveTG{整}{58101}
\saveTG{𣦔}{58102}
\saveTG{𢿋}{58102}
\saveTG{𢿄}{58102}
\saveTG{𥂢}{58102}
\saveTG{𩐈}{58102}
\saveTG{𩐎}{58102}
\saveTG{𡏴}{58104}
\saveTG{𡑁}{58104}
\saveTG{鳌}{58106}
\saveTG{鏊}{58109}
\saveTG{蚱}{58111}
\saveTG{蛻}{58112}
\saveTG{螠}{58112}
\saveTG{蜣}{58112}
\saveTG{𧓦}{58112}
\saveTG{蜕}{58112}
\saveTG{䗒}{58112}
\saveTG{𧏞}{58112}
\saveTG{𧊲}{58114}
\saveTG{𫋆}{58114}
\saveTG{𧋮}{58114}
\saveTG{𧎉}{58117}
\saveTG{𧒋}{58117}
\saveTG{𫊨}{58117}
\saveTG{虼}{58117}
\saveTG{𧉮}{58117}
\saveTG{𧖐}{58117}
\saveTG{𧎵}{58117}
\saveTG{蚧}{58120}
\saveTG{䗄}{58121}
\saveTG{蝓}{58121}
\saveTG{𧍿}{58122}
\saveTG{豑}{58127}
\saveTG{𫊦}{58127}
\saveTG{𫋞}{58127}
\saveTG{𧐀}{58127}
\saveTG{𪉑}{58127}
\saveTG{䗻}{58127}
\saveTG{蠄}{58127}
\saveTG{蚡}{58127}
\saveTG{骜}{58127}
\saveTG{𧋘}{58127}
\saveTG{𧕋}{58127}
\saveTG{蜦}{58127}
\saveTG{蚙}{58127}
\saveTG{螉}{58127}
\saveTG{𧏮}{58131}
\saveTG{𧌷}{58131}
\saveTG{蟱}{58131}
\saveTG{𫊹}{58131}
\saveTG{蜙}{58131}
\saveTG{螆}{58132}
\saveTG{蚣}{58132}
\saveTG{蛉}{58132}
\saveTG{蜙}{58132}
\saveTG{𧓲}{58132}
\saveTG{䗹}{58133}
\saveTG{𧐟}{58134}
\saveTG{𧒴}{58136}
\saveTG{螯}{58136}
\saveTG{螊}{58137}
\saveTG{𧑃}{58140}
\saveTG{𧐭}{58140}
\saveTG{𧎄}{58140}
\saveTG{𧑒}{58140}
\saveTG{𧒰}{58140}
\saveTG{𧒀}{58140}
\saveTG{蚥}{58140}
\saveTG{䖫}{58141}
\saveTG{䗴}{58141}
\saveTG{蛢}{58141}
\saveTG{𫊿}{58143}
\saveTG{𧒆}{58143}
\saveTG{𧍬}{58144}
\saveTG{䗗}{58144}
\saveTG{𧌕}{58147}
\saveTG{蝮}{58147}
\saveTG{蝣}{58147}
\saveTG{𧊜}{58147}
\saveTG{蛘}{58151}
\saveTG{𧕇}{58151}
\saveTG{𧒻}{58151}
\saveTG{䘊}{58153}
\saveTG{𧋟}{58153}
\saveTG{蟻}{58153}
\saveTG{蝉}{58156}
\saveTG{𤁙}{58156}
\saveTG{蟮}{58161}
\saveTG{蛤}{58161}
\saveTG{𧑹}{58161}
\saveTG{蛿}{58162}
\saveTG{𫋎}{58162}
\saveTG{𧌖}{58164}
\saveTG{𧏓}{58164}
\saveTG{蝤}{58164}
\saveTG{𧒯}{58166}
\saveTG{螥}{58167}
\saveTG{𧕉}{58168}
\saveTG{𧋉}{58168}
\saveTG{𧊦}{58172}
\saveTG{𧌤}{58177}
\saveTG{蜁}{58181}
\saveTG{𧐗}{58182}
\saveTG{䗥}{58182}
\saveTG{𧏣}{58184}
\saveTG{蜍}{58194}
\saveTG{𠩺}{58201}
\saveTG{氂}{58214}
\saveTG{釐}{58215}
\saveTG{𨤸}{58215}
\saveTG{靝}{58217}
\saveTG{靔}{58217}
\saveTG{𩘮}{58217}
\saveTG{𪅗}{58227}
\saveTG{剺}{58227}
\saveTG{𫆰}{58227}
\saveTG{𠞪}{58227}
\saveTG{𩪋}{58227}
\saveTG{𢄡}{58227}
\saveTG{𢡷}{58231}
\saveTG{𢟤}{58231}
\saveTG{𩇙}{58232}
\saveTG{𣸗}{58232}
\saveTG{𩺸}{58236}
\saveTG{敷}{58240}
\saveTG{𢽊}{58240}
\saveTG{𣁟}{58240}
\saveTG{𢼆}{58240}
\saveTG{敖}{58240}
\saveTG{𢾾}{58240}
\saveTG{𢼋}{58240}
\saveTG{𢿂}{58241}
\saveTG{𤘒}{58241}
\saveTG{嫠}{58244}
\saveTG{𣁛}{58246}
\saveTG{𠭰}{58247}
\saveTG{𢿍}{58247}
\saveTG{孷}{58247}
\saveTG{犛}{58251}
\saveTG{𢂷}{58261}
\saveTG{𧫬}{58261}
\saveTG{漦}{58292}
\saveTG{𣘬}{58294}
\saveTG{𥼋}{58294}
\saveTG{斄}{58298}
\saveTG{驁}{58327}
\saveTG{鷔}{58327}
\saveTG{鷘}{58327}
\saveTG{𢚅}{58330}
\saveTG{𢜉}{58331}
\saveTG{㥿}{58334}
\saveTG{熬}{58334}
\saveTG{慗}{58334}
\saveTG{懯}{58334}
\saveTG{𩺾}{58336}
\saveTG{𩷢}{58336}
\saveTG{鰲}{58336}
\saveTG{𢾭}{58340}
\saveTG{聱}{58401}
\saveTG{嫯}{58404}
\saveTG{𡥽}{58407}
\saveTG{𢪅}{58427}
\saveTG{𢿙}{58440}
\saveTG{數}{58440}
\saveTG{𡠼}{58444}
\saveTG{𡟋}{58444}
\saveTG{𢍛}{58444}
\saveTG{𦪈}{58447}
\saveTG{㧳}{58502}
\saveTG{摮}{58502}
\saveTG{㹈}{58504}
\saveTG{謷}{58601}
\saveTG{𧫣}{58601}
\saveTG{𧩥}{58601}
\saveTG{嗸}{58604}
\saveTG{𣊁}{58604}
\saveTG{𪯓}{58640}
\saveTG{𢾜}{58648}
\saveTG{𦧶}{58664}
\saveTG{𡴢}{58711}
\saveTG{㲠}{58715}
\saveTG{鼇}{58717}
\saveTG{𨝩}{58717}
\saveTG{𩞕}{58732}
\saveTG{𧜌}{58732}
\saveTG{㪄}{58740}
\saveTG{𪙠}{58771}
\saveTG{嶅}{58772}
\saveTG{𪘻}{58772}
\saveTG{赘}{58802}
\saveTG{獒}{58804}
\saveTG{贅}{58806}
\saveTG{𤏠}{58831}
\saveTG{敟}{58840}
\saveTG{𪯈}{58840}
\saveTG{𩕀}{58862}
\saveTG{𠇠}{58900}
\saveTG{䎣}{58900}
\saveTG{𠓩}{58900}
\saveTG{𦓧}{58900}
\saveTG{𦇆}{58903}
\saveTG{𦓰}{58914}
\saveTG{䎢}{58917}
\saveTG{𦔟}{58927}
\saveTG{𦅫}{58927}
\saveTG{𪱼}{58927}
\saveTG{耣}{58927}
\saveTG{𦔒}{58932}
\saveTG{𢶴}{58933}
\saveTG{敕}{58940}
\saveTG{敇}{58940}
\saveTG{𪯑}{58940}
\saveTG{𢽬}{58940}
\saveTG{𦔞}{58940}
\saveTG{耠}{58961}
\saveTG{𣞄}{58966}
\saveTG{𦔦}{58966}
\saveTG{䎥}{58968}
\saveTG{𦔡}{58968}
\saveTG{𢖗}{58982}
\saveTG{𢮬}{59000}
\saveTG{𢲶}{59012}
\saveTG{捲}{59012}
\saveTG{撹}{59012}
\saveTG{搅}{59012}
\saveTG{輄}{59012}
\saveTG{挄}{59012}
\saveTG{𨎋}{59014}
\saveTG{摚}{59014}
\saveTG{毮}{59014}
\saveTG{𨌫}{59017}
\saveTG{𢶇}{59017}
\saveTG{搅}{59017}
\saveTG{挱}{59020}
\saveTG{𢭼}{59020}
\saveTG{抄}{59020}
\saveTG{𢮙}{59027}
\saveTG{撈}{59027}
\saveTG{𨌩}{59027}
\saveTG{𢮐}{59027}
\saveTG{輎}{59027}
\saveTG{挘}{59027}
\saveTG{𢯭}{59027}
\saveTG{捎}{59027}
\saveTG{𢪍}{59030}
\saveTG{攩}{59031}
\saveTG{𨏻}{59031}
\saveTG{𢵌}{59032}
\saveTG{𢶲}{59038}
\saveTG{撐}{59041}
\saveTG{搂}{59044}
\saveTG{𨍦}{59044}
\saveTG{𢬰}{59044}
\saveTG{𢳓}{59044}
\saveTG{𢹒}{59047}
\saveTG{拌}{59050}
\saveTG{𢭬}{59050}
\saveTG{𢯝}{59050}
\saveTG{搼}{59052}
\saveTG{撑}{59052}
\saveTG{𢸏}{59056}
\saveTG{𨏏}{59057}
\saveTG{轔}{59059}
\saveTG{撛}{59059}
\saveTG{𨎚}{59060}
\saveTG{𢹙}{59062}
\saveTG{㨘}{59062}
\saveTG{攚}{59066}
\saveTG{擋}{59066}
\saveTG{𨎴}{59066}
\saveTG{𢰪}{59068}
\saveTG{挡}{59077}
\saveTG{揪}{59080}
\saveTG{𢫐}{59080}
\saveTG{𤆎}{59080}
\saveTG{𢹔}{59081}
\saveTG{𢸋}{59082}
\saveTG{𢬅}{59084}
\saveTG{𢷥}{59085}
\saveTG{𨍊}{59086}
\saveTG{𢱡}{59086}
\saveTG{掞}{59089}
\saveTG{𢴗}{59089}
\saveTG{𢸱}{59089}
\saveTG{𨌹}{59089}
\saveTG{㩒}{59091}
\saveTG{𢴤}{59094}
\saveTG{㩞}{59094}
\saveTG{𢬊}{59094}
\saveTG{𨏍}{59094}
\saveTG{蜷}{59112}
\saveTG{𫊶}{59112}
\saveTG{螳}{59114}
\saveTG{𫋗}{59115}
\saveTG{蝋}{59117}
\saveTG{䖢}{59120}
\saveTG{蟧}{59127}
\saveTG{𧌽}{59127}
\saveTG{𫋥}{59127}
\saveTG{𧔟}{59127}
\saveTG{蟐}{59127}
\saveTG{𧐬}{59127}
\saveTG{蛸}{59127}
\saveTG{𧉍}{59130}
\saveTG{𧑞}{59131}
\saveTG{𧒸}{59138}
\saveTG{蝼}{59144}
\saveTG{𧕊}{59147}
\saveTG{𧉻}{59150}
\saveTG{䗲}{59157}
\saveTG{𧰢}{59157}
\saveTG{𧕍}{59162}
\saveTG{𧍖}{59162}
\saveTG{蟷}{59166}
\saveTG{𧒩}{59174}
\saveTG{𨙨}{59177}
\saveTG{𧎐}{59180}
\saveTG{𧏉}{59185}
\saveTG{𧎫}{59186}
\saveTG{䗊}{59189}
\saveTG{𧐽}{59189}
\saveTG{𧓌}{59189}
\saveTG{蠑}{59194}
\saveTG{𧏪}{59194}
\saveTG{𧐑}{59196}
\saveTG{𧎾}{59196}
\saveTG{靗}{59212}
\saveTG{𫕺}{59250}
\saveTG{𩇢}{59252}
\saveTG{𠡸}{59427}
\saveTG{𡮦}{59620}
\saveTG{耖}{59920}
\saveTG{𦓴}{59927}
\saveTG{耥}{59927}
\saveTG{耮}{59927}
\saveTG{耧}{59944}
\saveTG{𦔄}{59962}
\saveTG{䎦}{59989}
\saveTG{𣅂}{60000}
\saveTG{罒}{60000}
\saveTG{旷}{60000}
\saveTG{口}{60000}
\saveTG{囗}{60000}
\saveTG{〇}{60000}
\saveTG{𠁴}{60006}
\saveTG{盳}{60010}
\saveTG{甿}{60010}
\saveTG{𠵌}{60012}
\saveTG{𠰍}{60014}
\saveTG{𥅾}{60014}
\saveTG{𣆮}{60014}
\saveTG{𥇁}{60014}
\saveTG{𥆜}{60014}
\saveTG{𥌬}{60014}
\saveTG{暀}{60014}
\saveTG{𥅖}{60014}
\saveTG{𡄐}{60014}
\saveTG{𡃚}{60014}
\saveTG{𣈐}{60014}
\saveTG{𠲕}{60014}
\saveTG{𣊌}{60015}
\saveTG{㫿}{60015}
\saveTG{囃}{60015}
\saveTG{睢}{60015}
\saveTG{噇}{60015}
\saveTG{𠼲}{60015}
\saveTG{𠻿}{60015}
\saveTG{𠾒}{60015}
\saveTG{𡅧}{60015}
\saveTG{𡄪}{60015}
\saveTG{𥍃}{60015}
\saveTG{噰}{60015}
\saveTG{𣌖}{60015}
\saveTG{𥊓}{60015}
\saveTG{𥊖}{60015}
\saveTG{𡄸}{60015}
\saveTG{唯}{60015}
\saveTG{疃}{60015}
\saveTG{曈}{60015}
\saveTG{瞳}{60015}
\saveTG{囄}{60015}
\saveTG{䁴}{60016}
\saveTG{𡅹}{60016}
\saveTG{𣋊}{60016}
\saveTG{𠿞}{60016}
\saveTG{㽘}{60017}
\saveTG{吭}{60017}
\saveTG{喨}{60017}
\saveTG{䀮}{60017}
\saveTG{𥋦}{60017}
\saveTG{𥉶}{60017}
\saveTG{𥆨}{60017}
\saveTG{𥉫}{60017}
\saveTG{𠳓}{60017}
\saveTG{𠿾}{60017}
\saveTG{𥄦}{60017}
\saveTG{㬯}{60017}
\saveTG{㖢}{60017}
\saveTG{𠺩}{60017}
\saveTG{𥅻}{60017}
\saveTG{𪽒}{60017}
\saveTG{𠵉}{60017}
\saveTG{𥋨}{60017}
\saveTG{𠺠}{60017}
\saveTG{𠹂}{60017}
\saveTG{𠵷}{60017}
\saveTG{𪡁}{60017}
\saveTG{𣊈}{60017}
\saveTG{𣆖}{60017}
\saveTG{𣅇}{60017}
\saveTG{𥅈}{60018}
\saveTG{𠴖}{60018}
\saveTG{啦}{60018}
\saveTG{㕸}{60018}
\saveTG{𠴹}{60018}
\saveTG{𠼵}{60021}
\saveTG{𠲗}{60021}
\saveTG{𡄘}{60021}
\saveTG{䁎}{60021}
\saveTG{𠷥}{60021}
\saveTG{喭}{60022}
\saveTG{嚌}{60023}
\saveTG{𣋠}{60023}
\saveTG{哜}{60024}
\saveTG{哼}{60027}
\saveTG{𪡦}{60027}
\saveTG{𤳬}{60027}
\saveTG{𡆽}{60027}
\saveTG{𡆸}{60027}
\saveTG{𠿘}{60027}
\saveTG{𡂗}{60027}
\saveTG{𦉶}{60027}
\saveTG{𪾷}{60027}
\saveTG{𠴽}{60027}
\saveTG{嘀}{60027}
\saveTG{𠻗}{60027}
\saveTG{𠼬}{60027}
\saveTG{𡂀}{60027}
\saveTG{𠶴}{60027}
\saveTG{𣉞}{60027}
\saveTG{唷}{60027}
\saveTG{啼}{60027}
\saveTG{嗃}{60027}
\saveTG{𥊔}{60027}
\saveTG{䁤}{60027}
\saveTG{号}{60027}
\saveTG{眆}{60027}
\saveTG{昉}{60027}
\saveTG{嘃}{60027}
\saveTG{瞝}{60027}
\saveTG{嗙}{60027}
\saveTG{𡇳}{60027}
\saveTG{㕫}{60027}
\saveTG{𡄡}{60027}
\saveTG{𠳻}{60027}
\saveTG{𠻊}{60027}
\saveTG{𪿅}{60027}
\saveTG{𠻫}{60030}
\saveTG{𠯴}{60030}
\saveTG{𠲔}{60030}
\saveTG{𡃼}{60031}
\saveTG{𡂘}{60031}
\saveTG{㗹}{60031}
\saveTG{𡄖}{60031}
\saveTG{𥌜}{60031}
\saveTG{𡄯}{60031}
\saveTG{瞧}{60031}
\saveTG{噍}{60031}
\saveTG{𥌾}{60031}
\saveTG{𣋳}{60031}
\saveTG{𠲖}{60032}
\saveTG{𪱒}{60032}
\saveTG{𡁺}{60032}
\saveTG{𠽓}{60032}
\saveTG{呟}{60032}
\saveTG{昿}{60032}
\saveTG{嚷}{60032}
\saveTG{昡}{60032}
\saveTG{㗒}{60032}
\saveTG{畩}{60032}
\saveTG{𧛧}{60032}
\saveTG{眩}{60032}
\saveTG{𠼁}{60032}
\saveTG{𦌶}{60032}
\saveTG{嚎}{60032}
\saveTG{𡃩}{60032}
\saveTG{𠵱}{60032}
\saveTG{嚒}{60032}
\saveTG{𣋟}{60032}
\saveTG{𡅬}{60032}
\saveTG{𡂓}{60032}
\saveTG{𡂪}{60033}
\saveTG{𤳌}{60034}
\saveTG{噫}{60036}
\saveTG{𪱍}{60036}
\saveTG{𣆬}{60037}
\saveTG{𥋲}{60037}
\saveTG{𠿳}{60037}
\saveTG{嗻}{60037}
\saveTG{𡆡}{60037}
\saveTG{𥈚}{60038}
\saveTG{𡂭}{60038}
\saveTG{呅}{60040}
\saveTG{旼}{60040}
\saveTG{𪽌}{60040}
\saveTG{盿}{60040}
\saveTG{𪢨}{60040}
\saveTG{𡃥}{60041}
\saveTG{䡶}{60041}
\saveTG{𡅼}{60041}
\saveTG{𡁈}{60041}
\saveTG{𥋑}{60041}
\saveTG{㖕}{60041}
\saveTG{䁹}{60041}
\saveTG{䁄}{60041}
\saveTG{𥌲}{60041}
\saveTG{噼}{60041}
\saveTG{𣈋}{60042}
\saveTG{𠻎}{60042}
\saveTG{㖡}{60042}
\saveTG{𠵒}{60043}
\saveTG{𠻜}{60043}
\saveTG{𪡸}{60044}
\saveTG{𥇒}{60044}
\saveTG{唼}{60044}
\saveTG{暲}{60046}
\saveTG{瞕}{60046}
\saveTG{𠼀}{60046}
\saveTG{𥇜}{60047}
\saveTG{𥌍}{60047}
\saveTG{𣉲}{60047}
\saveTG{𡂻}{60047}
\saveTG{啍}{60047}
\saveTG{喥}{60047}
\saveTG{㬀}{60047}
\saveTG{𤲠}{60048}
\saveTG{晈}{60048}
\saveTG{睟}{60048}
\saveTG{咬}{60048}
\saveTG{啐}{60048}
\saveTG{晬}{60048}
\saveTG{𡀬}{60048}
\saveTG{㘐}{60048}
\saveTG{𥅟}{60048}
\saveTG{𠱪}{60049}
\saveTG{𠲲}{60050}
\saveTG{㫠}{60050}
\saveTG{嚤}{60052}
\saveTG{𪰽}{60056}
\saveTG{𠺟}{60056}
\saveTG{𠽪}{60061}
\saveTG{𪾲}{60061}
\saveTG{㖣}{60061}
\saveTG{喑}{60061}
\saveTG{唁}{60061}
\saveTG{暗}{60061}
\saveTG{𥇋}{60061}
\saveTG{𠶧}{60062}
\saveTG{𠽜}{60062}
\saveTG{嚰}{60062}
\saveTG{㗜}{60063}
\saveTG{𠸠}{60063}
\saveTG{𠳺}{60064}
\saveTG{𠹔}{60065}
\saveTG{𪰼}{60065}
\saveTG{𠶦}{60065}
\saveTG{瞊}{60065}
\saveTG{𥈛}{60068}
\saveTG{𠷄}{60072}
\saveTG{𡁫}{60077}
\saveTG{𠯿}{60080}
\saveTG{𣅶}{60080}
\saveTG{𡀲}{60082}
\saveTG{畡}{60082}
\saveTG{晐}{60082}
\saveTG{咳}{60082}
\saveTG{䀭}{60082}
\saveTG{𠹋}{60084}
\saveTG{𥊆}{60085}
\saveTG{曠}{60086}
\saveTG{嚝}{60086}
\saveTG{矌}{60086}
\saveTG{𤳱}{60086}
\saveTG{𡀀}{60091}
\saveTG{𡈸}{60093}
\saveTG{𡄁}{60094}
\saveTG{𡀫}{60094}
\saveTG{𥋶}{60094}
\saveTG{㗱}{60094}
\saveTG{𠾍}{60094}
\saveTG{𣊍}{60094}
\saveTG{𥊚}{60094}
\saveTG{𠳹}{60094}
\saveTG{嘛}{60094}
\saveTG{𠴬}{60094}
\saveTG{𥇱}{60094}
\saveTG{𡅁}{60095}
\saveTG{𡅓}{60095}
\saveTG{䁁}{60096}
\saveTG{𥈘}{60096}
\saveTG{𠶛}{60096}
\saveTG{晾}{60096}
\saveTG{𥉽}{60099}
\saveTG{𠻞}{60099}
\saveTG{𧾷}{60100}
\saveTG{曰}{60100}
\saveTG{𧿈}{60100}
\saveTG{旦}{60100}
\saveTG{日}{60100}
\saveTG{𣄾}{60101}
\saveTG{罡}{60101}
\saveTG{𠮡}{60101}
\saveTG{𡆤}{60101}
\saveTG{𡇉}{60101}
\saveTG{𡇆}{60101}
\saveTG{}{60101}
\saveTG{目}{60101}
\saveTG{囸}{60101}
\saveTG{昰}{60101}
\saveTG{𥁕}{60102}
\saveTG{𥃇}{60102}
\saveTG{𤴁}{60102}
\saveTG{𦊂}{60102}
\saveTG{𥂮}{60102}
\saveTG{𡈮}{60102}
\saveTG{𠱄}{60102}
\saveTG{𡀉}{60102}
\saveTG{𥁯}{60102}
\saveTG{𣈍}{60102}
\saveTG{罝}{60102}
\saveTG{𣌬}{60102}
\saveTG{𣅣}{60102}
\saveTG{㬪}{60102}
\saveTG{𣋣}{60102}
\saveTG{𪽤}{60102}
\saveTG{𡇬}{60102}
\saveTG{𦋳}{60102}
\saveTG{𪢭}{60102}
\saveTG{𡈀}{60102}
\saveTG{𨂞}{60102}
\saveTG{㫫}{60102}
\saveTG{𠹜}{60102}
\saveTG{𣆞}{60102}
\saveTG{𥃿}{60102}
\saveTG{𦊆}{60102}
\saveTG{㫔}{60102}
\saveTG{𣆹}{60102}
\saveTG{曡}{60102}
\saveTG{置}{60102}
\saveTG{圔}{60102}
\saveTG{显}{60102}
\saveTG{畳}{60102}
\saveTG{昷}{60102}
\saveTG{疊}{60102}
\saveTG{疉}{60102}
\saveTG{疂}{60102}
\saveTG{𧖸}{60102}
\saveTG{𥁰}{60102}
\saveTG{𡆬}{60102}
\saveTG{国}{60103}
\saveTG{罣}{60104}
\saveTG{囯}{60104}
\saveTG{囶}{60104}
\saveTG{塁}{60104}
\saveTG{壘}{60104}
\saveTG{墨}{60104}
\saveTG{圼}{60104}
\saveTG{罜}{60104}
\saveTG{𦊄}{60104}
\saveTG{𡇓}{60104}
\saveTG{𪰑}{60104}
\saveTG{𡍸}{60104}
\saveTG{𡇽}{60104}
\saveTG{𡈁}{60104}
\saveTG{𦌋}{60104}
\saveTG{𡋘}{60104}
\saveTG{𡈞}{60104}
\saveTG{𡆮}{60104}
\saveTG{𡋵}{60104}
\saveTG{𪰦}{60104}
\saveTG{𤲲}{60104}
\saveTG{𡔌}{60104}
\saveTG{𦉷}{60104}
\saveTG{𡒀}{60104}
\saveTG{呈}{60104}
\saveTG{星}{60105}
\saveTG{𣊹}{60105}
\saveTG{𠻖}{60105}
\saveTG{𨤥}{60105}
\saveTG{𡈈}{60105}
\saveTG{量}{60105}
\saveTG{㽮}{60105}
\saveTG{曐}{60105}
\saveTG{里}{60105}
\saveTG{𡈩}{60105}
\saveTG{罿}{60105}
\saveTG{𦊥}{60106}
\saveTG{𡆷}{60106}
\saveTG{𤳏}{60106}
\saveTG{𦋶}{60106}
\saveTG{𦋠}{60106}
\saveTG{𦊢}{60108}
\saveTG{𡇧}{60108}
\saveTG{昱}{60108}
\saveTG{𣅋}{60108}
\saveTG{𨯔}{60109}
\saveTG{𫅇}{60109}
\saveTG{𡈚}{60111}
\saveTG{暃}{60111}
\saveTG{𨇻}{60111}
\saveTG{罪}{60111}
\saveTG{𤱲}{60112}
\saveTG{蹗}{60112}
\saveTG{𪾍}{60112}
\saveTG{晁}{60113}
\saveTG{𨀵}{60114}
\saveTG{躔}{60114}
\saveTG{跓}{60114}
\saveTG{踓}{60115}
\saveTG{蹱}{60115}
\saveTG{雖}{60115}
\saveTG{𡃵}{60115}
\saveTG{𨄉}{60115}
\saveTG{𩁍}{60115}
\saveTG{𪰹}{60115}
\saveTG{𨆁}{60116}
\saveTG{䟽}{60117}
\saveTG{𨃸}{60117}
\saveTG{𤰉}{60117}
\saveTG{𨇹}{60117}
\saveTG{䟘}{60117}
\saveTG{𨀟}{60117}
\saveTG{𡇸}{60117}
\saveTG{䟲}{60117}
\saveTG{𣌫}{60117}
\saveTG{𣅬}{60117}
\saveTG{𣆷}{60117}
\saveTG{𥄀}{60117}
\saveTG{𨂖}{60118}
\saveTG{𨅀}{60118}
\saveTG{𨀎}{60118}
\saveTG{㘤}{60120}
\saveTG{𦌧}{60120}
\saveTG{𪰊}{60120}
\saveTG{𣉛}{60121}
\saveTG{𨂪}{60122}
\saveTG{躋}{60123}
\saveTG{跻}{60124}
\saveTG{𨂋}{60124}
\saveTG{䠙}{60127}
\saveTG{𨃤}{60127}
\saveTG{𫏤}{60127}
\saveTG{𪾫}{60127}
\saveTG{𦋅}{60127}
\saveTG{昮}{60127}
\saveTG{勗}{60127}
\saveTG{蜀}{60127}
\saveTG{𦐇}{60127}
\saveTG{𣆃}{60127}
\saveTG{𨂔}{60127}
\saveTG{𨂺}{60127}
\saveTG{𦊃}{60127}
\saveTG{昻}{60127}
\saveTG{蹄}{60127}
\saveTG{蹢}{60127}
\saveTG{趽}{60127}
\saveTG{跡}{60130}
\saveTG{罫}{60130}
\saveTG{𨆽}{60131}
\saveTG{𨄭}{60131}
\saveTG{𤴂}{60131}
\saveTG{𧒛}{60131}
\saveTG{躟}{60132}
\saveTG{𡈵}{60132}
\saveTG{𨄄}{60133}
\saveTG{𧋩}{60136}
\saveTG{虽}{60136}
\saveTG{𧐌}{60136}
\saveTG{𧏒}{60136}
\saveTG{𧎌}{60136}
\saveTG{𧌪}{60136}
\saveTG{𧋯}{60136}
\saveTG{𨆩}{60137}
\saveTG{蹠}{60137}
\saveTG{𠘀}{60138}
\saveTG{躕}{60140}
\saveTG{𨆒}{60141}
\saveTG{𨀠}{60141}
\saveTG{𨁅}{60141}
\saveTG{躃}{60141}
\saveTG{𨂒}{60142}
\saveTG{𨃛}{60142}
\saveTG{𨁵}{60143}
\saveTG{𨄮}{60143}
\saveTG{𨁐}{60144}
\saveTG{踥}{60144}
\saveTG{𨈂}{60147}
\saveTG{𨂫}{60147}
\saveTG{踱}{60147}
\saveTG{𪔒}{60147}
\saveTG{𨇯}{60147}
\saveTG{跤}{60148}
\saveTG{踤}{60148}
\saveTG{晸}{60148}
\saveTG{𨄢}{60152}
\saveTG{國}{60153}
\saveTG{罭}{60153}
\saveTG{𫏚}{60161}
\saveTG{㬁}{60161}
\saveTG{踣}{60161}
\saveTG{踮}{60161}
\saveTG{𣈔}{60162}
\saveTG{𨅙}{60162}
\saveTG{𨇢}{60162}
\saveTG{𨃕}{60163}
\saveTG{𦌕}{60164}
\saveTG{𪱅}{60164}
\saveTG{𨁮}{60164}
\saveTG{𨃠}{60165}
\saveTG{𡆴}{60174}
\saveTG{𡆰}{60175}
\saveTG{囙}{60177}
\saveTG{𤱗}{60180}
\saveTG{𨄀}{60181}
\saveTG{𦍂}{60181}
\saveTG{𨀖}{60182}
\saveTG{𨇁}{60186}
\saveTG{𨇾}{60189}
\saveTG{𪢩}{60191}
\saveTG{𨁤}{60194}
\saveTG{𨁻}{60194}
\saveTG{𨂙}{60196}
\saveTG{𨄗}{60199}
\saveTG{𡆹}{60201}
\saveTG{甼}{60201}
\saveTG{𣄿}{60201}
\saveTG{𦉬}{60201}
\saveTG{𠻝}{60202}
\saveTG{𡆱}{60202}
\saveTG{𦌀}{60202}
\saveTG{曑}{60202}
\saveTG{𣆪}{60206}
\saveTG{𣆚}{60207}
\saveTG{罗}{60207}
\saveTG{𣌤}{60209}
\saveTG{𧵁}{60209}
\saveTG{园}{60212}
\saveTG{四}{60212}
\saveTG{𣈳}{60212}
\saveTG{𦋾}{60212}
\saveTG{䍥}{60212}
\saveTG{罷}{60212}
\saveTG{晃}{60212}
\saveTG{見}{60212}
\saveTG{兄}{60212}
\saveTG{囮}{60214}
\saveTG{𧠭}{60214}
\saveTG{𪐾}{60214}
\saveTG{𠲉}{60214}
\saveTG{𣌷}{60214}
\saveTG{䍜}{60215}
\saveTG{𩀇}{60215}
\saveTG{𨿔}{60215}
\saveTG{𩀍}{60215}
\saveTG{𩀭}{60215}
\saveTG{𩁯}{60215}
\saveTG{䍦}{60215}
\saveTG{𧠻}{60216}
\saveTG{㫯}{60217}
\saveTG{㫕}{60217}
\saveTG{𣋍}{60217}
\saveTG{𡇩}{60217}
\saveTG{囥}{60217}
\saveTG{𪢯}{60217}
\saveTG{𥃺}{60217}
\saveTG{𣆎}{60217}
\saveTG{𣆧}{60217}
\saveTG{𣅷}{60217}
\saveTG{𣆍}{60217}
\saveTG{𠯚}{60217}
\saveTG{𡇰}{60217}
\saveTG{𡈦}{60217}
\saveTG{𦌿}{60217}
\saveTG{𦊫}{60217}
\saveTG{㘢}{60217}
\saveTG{𩴻}{60217}
\saveTG{䍡}{60217}
\saveTG{𠰨}{60217}
\saveTG{𦋤}{60217}
\saveTG{𥃯}{60217}
\saveTG{𣈙}{60220}
\saveTG{𣉦}{60220}
\saveTG{𦋞}{60220}
\saveTG{𣋃}{60220}
\saveTG{𪰮}{60220}
\saveTG{𣌻}{60220}
\saveTG{𣅕}{60220}
\saveTG{畀}{60221}
\saveTG{𣌲}{60221}
\saveTG{𦋒}{60221}
\saveTG{㫹}{60221}
\saveTG{𣈈}{60221}
\saveTG{𥇴}{60221}
\saveTG{𦌫}{60221}
\saveTG{𥄊}{60221}
\saveTG{罽}{60221}
\saveTG{囬}{60221}
\saveTG{罞}{60222}
\saveTG{𡈝}{60222}
\saveTG{𡈖}{60222}
\saveTG{𡈪}{60222}
\saveTG{𫅅}{60224}
\saveTG{𡇇}{60227}
\saveTG{𣌥}{60227}
\saveTG{䍤}{60227}
\saveTG{𡆵}{60227}
\saveTG{𣋥}{60227}
\saveTG{𣇅}{60227}
\saveTG{𡇊}{60227}
\saveTG{𡈌}{60227}
\saveTG{𠱮}{60227}
\saveTG{𡇏}{60227}
\saveTG{𡈘}{60227}
\saveTG{𡆭}{60227}
\saveTG{𢑢}{60227}
\saveTG{㬅}{60227}
\saveTG{𣌉}{60227}
\saveTG{𥅙}{60227}
\saveTG{𡇅}{60227}
\saveTG{䍠}{60227}
\saveTG{𥅫}{60227}
\saveTG{𦞅}{60227}
\saveTG{𦊤}{60227}
\saveTG{𡆫}{60227}
\saveTG{㘣}{60227}
\saveTG{𤲳}{60227}
\saveTG{𦚋}{60227}
\saveTG{𦊈}{60227}
\saveTG{𦍇}{60227}
\saveTG{𡇮}{60227}
\saveTG{𠮠}{60227}
\saveTG{𪔅}{60227}
\saveTG{𠚙}{60227}
\saveTG{昘}{60227}
\saveTG{𥝆}{60227}
\saveTG{𣃸}{60227}
\saveTG{𣆻}{60227}
\saveTG{𣅻}{60227}
\saveTG{𣈦}{60227}
\saveTG{𣌰}{60227}
\saveTG{肙}{60227}
\saveTG{禺}{60227}
\saveTG{囿}{60227}
\saveTG{易}{60227}
\saveTG{昜}{60227}
\saveTG{晑}{60227}
\saveTG{胃}{60227}
\saveTG{圊}{60227}
\saveTG{圃}{60227}
\saveTG{羃}{60227}
\saveTG{冐}{60227}
\saveTG{圇}{60227}
\saveTG{罱}{60227}
\saveTG{罤}{60227}
\saveTG{晜}{60227}
\saveTG{圐}{60227}
\saveTG{罥}{60227}
\saveTG{囧}{60227}
\saveTG{囫}{60227}
\saveTG{圀}{60227}
\saveTG{呙}{60227}
\saveTG{叧}{60227}
\saveTG{暠}{60227}
\saveTG{吊}{60227}
\saveTG{圌}{60227}
\saveTG{昺}{60227}
\saveTG{𥊽}{60227}
\saveTG{𣍓}{60227}
\saveTG{𣈯}{60228}
\saveTG{界}{60228}
\saveTG{羃}{60228}
\saveTG{昦}{60228}
\saveTG{𠕰}{60229}
\saveTG{𡆥}{60230}
\saveTG{㬄}{60231}
\saveTG{㫱}{60231}
\saveTG{𦍄}{60231}
\saveTG{罛}{60232}
\saveTG{圂}{60232}
\saveTG{𣌪}{60232}
\saveTG{園}{60232}
\saveTG{晨}{60232}
\saveTG{𨑌}{60232}
\saveTG{𦊽}{60232}
\saveTG{𥅀}{60232}
\saveTG{㫤}{60232}
\saveTG{囦}{60232}
\saveTG{𤲆}{60232}
\saveTG{𧰮}{60232}
\saveTG{曟}{60232}
\saveTG{𦋯}{60233}
\saveTG{曧}{60236}
\saveTG{𦋰}{60237}
\saveTG{团}{60240}
\saveTG{罻}{60240}
\saveTG{䍓}{60241}
\saveTG{𦌠}{60241}
\saveTG{𡈬}{60243}
\saveTG{𣍑}{60243}
\saveTG{𠰎}{60247}
\saveTG{𣇗}{60247}
\saveTG{𣫕}{60247}
\saveTG{𦌩}{60247}
\saveTG{𡅾}{60247}
\saveTG{𣎞}{60247}
\saveTG{𦋉}{60247}
\saveTG{𡅝}{60247}
\saveTG{曻}{60250}
\saveTG{晟}{60253}
\saveTG{𥅛}{60253}
\saveTG{𣆭}{60253}
\saveTG{𦌬}{60257}
\saveTG{𣌂}{60264}
\saveTG{𡇘}{60270}
\saveTG{𡇕}{60271}
\saveTG{𣇯}{60277}
\saveTG{昃}{60281}
\saveTG{𦌇}{60282}
\saveTG{𡙷}{60284}
\saveTG{𡈭}{60284}
\saveTG{𣋷}{60286}
\saveTG{𡈷}{60293}
\saveTG{𡈳}{60294}
\saveTG{𣉹}{60294}
\saveTG{𡂾}{60299}
\saveTG{𣅭}{60300}
\saveTG{𠮚}{60300}
\saveTG{囹}{60302}
\saveTG{𥈁}{60302}
\saveTG{𦊓}{60302}
\saveTG{图}{60303}
\saveTG{㫡}{60303}
\saveTG{𠯣}{60307}
\saveTG{罖}{60308}
\saveTG{𠘗}{60308}
\saveTG{𠯁}{60308}
\saveTG{𪐴}{60314}
\saveTG{𪐦}{60317}
\saveTG{𪒏}{60317}
\saveTG{𦌥}{60320}
\saveTG{𡈛}{60320}
\saveTG{𣌮}{60320}
\saveTG{𪒌}{60321}
\saveTG{𪑬}{60321}
\saveTG{𪒔}{60327}
\saveTG{䴎}{60327}
\saveTG{𡈙}{60327}
\saveTG{𪒒}{60327}
\saveTG{罵}{60327}
\saveTG{鷕}{60327}
\saveTG{𡈊}{60327}
\saveTG{𡈎}{60327}
\saveTG{思}{60330}
\saveTG{𥄧}{60330}
\saveTG{𢗭}{60330}
\saveTG{𢗀}{60330}
\saveTG{𢠝}{60330}
\saveTG{𢜹}{60330}
\saveTG{𢝮}{60330}
\saveTG{𢙍}{60330}
\saveTG{㤙}{60330}
\saveTG{𤆗}{60330}
\saveTG{𢚊}{60330}
\saveTG{𪫩}{60330}
\saveTG{恩}{60330}
\saveTG{慁}{60330}
\saveTG{䍢}{60331}
\saveTG{𢘇}{60331}
\saveTG{羆}{60331}
\saveTG{黑}{60331}
\saveTG{𪒕}{60331}
\saveTG{𡈒}{60331}
\saveTG{𪐷}{60331}
\saveTG{𡇺}{60331}
\saveTG{𢜧}{60331}
\saveTG{𢝕}{60331}
\saveTG{黒}{60331}
\saveTG{愚}{60332}
\saveTG{惖}{60332}
\saveTG{𤋁}{60332}
\saveTG{𪓃}{60332}
\saveTG{𤐍}{60332}
\saveTG{罴}{60333}
\saveTG{㬎}{60333}
\saveTG{𪸣}{60334}
\saveTG{𠼿}{60334}
\saveTG{𢙶}{60334}
\saveTG{㘠}{60334}
\saveTG{𡇫}{60334}
\saveTG{𩶔}{60336}
\saveTG{𪽟}{60336}
\saveTG{𪫰}{60336}
\saveTG{𩶊}{60336}
\saveTG{𢛝}{60336}
\saveTG{罳}{60336}
\saveTG{𠮰}{60337}
\saveTG{𡿯}{60337}
\saveTG{𢗻}{60338}
\saveTG{𢝝}{60339}
\saveTG{𡈻}{60339}
\saveTG{団}{60340}
\saveTG{𥃷}{60340}
\saveTG{㝵}{60341}
\saveTG{䙷}{60341}
\saveTG{㝶}{60341}
\saveTG{圑}{60342}
\saveTG{團}{60343}
\saveTG{𪑗}{60344}
\saveTG{𡈣}{60346}
\saveTG{𦋲}{60347}
\saveTG{𪒊}{60347}
\saveTG{𪑒}{60347}
\saveTG{黯}{60361}
\saveTG{噫}{60361}
\saveTG{𦌸}{60382}
\saveTG{𩼄}{60389}
\saveTG{𪑘}{60394}
\saveTG{黥}{60396}
\saveTG{𤰞}{60400}
\saveTG{早}{60400}
\saveTG{田}{60400}
\saveTG{旻}{60400}
\saveTG{龱}{60400}
\saveTG{𦉴}{60400}
\saveTG{𥄐}{60400}
\saveTG{𥃪}{60400}
\saveTG{𨈗}{60401}
\saveTG{𣇛}{60401}
\saveTG{𪡒}{60401}
\saveTG{𤲜}{60401}
\saveTG{𥇡}{60401}
\saveTG{睪}{60401}
\saveTG{旱}{60401}
\saveTG{咠}{60401}
\saveTG{圛}{60401}
\saveTG{圉}{60401}
\saveTG{𣅐}{60401}
\saveTG{𠯊}{60401}
\saveTG{𤰠}{60401}
\saveTG{𤰤}{60401}
\saveTG{𣊴}{60401}
\saveTG{㫭}{60401}
\saveTG{曱}{60401}
\saveTG{𦋁}{60401}
\saveTG{𥇝}{60401}
\saveTG{䍐}{60401}
\saveTG{図}{60403}
\saveTG{囡}{60404}
\saveTG{妟}{60404}
\saveTG{嘦}{60404}
\saveTG{晏}{60404}
\saveTG{𡈠}{60406}
\saveTG{𦌃}{60406}
\saveTG{罩}{60406}
\saveTG{𣆉}{60406}
\saveTG{𠵀}{60406}
\saveTG{畟}{60407}
\saveTG{𠭁}{60407}
\saveTG{𤰯}{60407}
\saveTG{𣅝}{60407}
\saveTG{𠵬}{60407}
\saveTG{𥄎}{60407}
\saveTG{㫗}{60407}
\saveTG{𡕭}{60407}
\saveTG{罦}{60407}
\saveTG{囝}{60407}
\saveTG{曼}{60407}
\saveTG{㘝}{60407}
\saveTG{𪠲}{60407}
\saveTG{𡕥}{60407}
\saveTG{𡥔}{60407}
\saveTG{𡦬}{60407}
\saveTG{𡦩}{60407}
\saveTG{𠯂}{60407}
\saveTG{𡇀}{60407}
\saveTG{𤲋}{60407}
\saveTG{㬊}{60407}
\saveTG{𣋦}{60408}
\saveTG{𣉱}{60408}
\saveTG{𣊖}{60408}
\saveTG{𡇻}{60408}
\saveTG{曓}{60408}
\saveTG{冕}{60412}
\saveTG{䨃}{60415}
\saveTG{𨿑}{60415}
\saveTG{𪟦}{60415}
\saveTG{㬮}{60415}
\saveTG{䨉}{60415}
\saveTG{𩁇}{60415}
\saveTG{𡆶}{60417}
\saveTG{旯}{60417}
\saveTG{𤲈}{60417}
\saveTG{𡇲}{60417}
\saveTG{𡇹}{60417}
\saveTG{另}{60427}
\saveTG{𡇨}{60427}
\saveTG{𪟧}{60427}
\saveTG{𠢮}{60427}
\saveTG{𣋶}{60427}
\saveTG{㘞}{60427}
\saveTG{𠢿}{60427}
\saveTG{𣋂}{60427}
\saveTG{朂}{60427}
\saveTG{男}{60427}
\saveTG{䍔}{60431}
\saveTG{𦋆}{60433}
\saveTG{𡆪}{60440}
\saveTG{㘟}{60440}
\saveTG{𢌿}{60440}
\saveTG{𠮽}{60440}
\saveTG{𣅹}{60440}
\saveTG{㫒}{60440}
\saveTG{𥃲}{60440}
\saveTG{𡆼}{60440}
\saveTG{𡜑}{60440}
\saveTG{㚻}{60440}
\saveTG{昇}{60440}
\saveTG{𡜰}{60440}
\saveTG{𡜸}{60440}
\saveTG{𥅳}{60441}
\saveTG{𧠋}{60441}
\saveTG{𡇭}{60441}
\saveTG{𡇦}{60441}
\saveTG{𢍧}{60443}
\saveTG{昪}{60443}
\saveTG{𡝎}{60444}
\saveTG{𢍰}{60444}
\saveTG{𡇔}{60444}
\saveTG{𡢑}{60444}
\saveTG{嘦}{60446}
\saveTG{𦌁}{60446}
\saveTG{罬}{60447}
\saveTG{最}{60447}
\saveTG{𡇄}{60447}
\saveTG{𦌖}{60447}
\saveTG{𠯧}{60447}
\saveTG{𣈠}{60448}
\saveTG{𦌙}{60448}
\saveTG{𪰘}{60460}
\saveTG{𠭒}{60476}
\saveTG{𤘵}{60500}
\saveTG{畢}{60500}
\saveTG{甲}{60500}
\saveTG{㽚}{60500}
\saveTG{𦎝}{60501}
\saveTG{𪢪}{60501}
\saveTG{𢸇}{60502}
\saveTG{暈}{60502}
\saveTG{𠰅}{60502}
\saveTG{𡆺}{60502}
\saveTG{𣇟}{60502}
\saveTG{𡇐}{60503}
\saveTG{𦊚}{60503}
\saveTG{罼}{60504}
\saveTG{𦌂}{60504}
\saveTG{晕}{60504}
\saveTG{曅}{60504}
\saveTG{𣈝}{60504}
\saveTG{𣅼}{60506}
\saveTG{𠦤}{60506}
\saveTG{𠲁}{60506}
\saveTG{𦌭}{60506}
\saveTG{𪽓}{60506}
\saveTG{圍}{60506}
\saveTG{𨍯}{60506}
\saveTG{䡞}{60506}
\saveTG{𪢫}{60506}
\saveTG{𤲃}{60507}
\saveTG{𣌦}{60507}
\saveTG{𡆎}{60516}
\saveTG{羇}{60521}
\saveTG{𣇄}{60521}
\saveTG{𣆽}{60521}
\saveTG{䁀}{60521}
\saveTG{羁}{60527}
\saveTG{𦋱}{60527}
\saveTG{围}{60527}
\saveTG{羈}{60527}
\saveTG{𦋈}{60535}
\saveTG{𦌈}{60544}
\saveTG{𣈓}{60547}
\saveTG{囲}{60550}
\saveTG{𦌹}{60556}
\saveTG{𦍈}{60556}
\saveTG{𤳅}{60556}
\saveTG{㘡}{60560}
\saveTG{𦊔}{60600}
\saveTG{𣅍}{60600}
\saveTG{𥄲}{60600}
\saveTG{𡇌}{60600}
\saveTG{𠯭}{60600}
\saveTG{𣆢}{60600}
\saveTG{𣅊}{60600}
\saveTG{昌}{60600}
\saveTG{吕}{60600}
\saveTG{畕}{60600}
\saveTG{冒}{60600}
\saveTG{𠯮}{60600}
\saveTG{回}{60600}
\saveTG{𦊸}{60600}
\saveTG{𠰝}{60600}
\saveTG{䍛}{60600}
\saveTG{𡇍}{60600}
\saveTG{謈}{60601}
\saveTG{𡇞}{60601}
\saveTG{𦊴}{60601}
\saveTG{𡇾}{60601}
\saveTG{圁}{60601}
\saveTG{𣆗}{60601}
\saveTG{𡈐}{60601}
\saveTG{罯}{60601}
\saveTG{𡇪}{60601}
\saveTG{𧦊}{60601}
\saveTG{詈}{60601}
\saveTG{𧪔}{60601}
\saveTG{圄}{60601}
\saveTG{𧬉}{60601}
\saveTG{𥅏}{60601}
\saveTG{圕}{60601}
\saveTG{圙}{60602}
\saveTG{𡈔}{60602}
\saveTG{𡇢}{60602}
\saveTG{𡇚}{60602}
\saveTG{𡇵}{60602}
\saveTG{𦊼}{60602}
\saveTG{礨}{60602}
\saveTG{罶}{60602}
\saveTG{呂}{60602}
\saveTG{𠴿}{60602}
\saveTG{碞}{60602}
\saveTG{𡇝}{60602}
\saveTG{𠷰}{60602}
\saveTG{𡇈}{60602}
\saveTG{𡈇}{60603}
\saveTG{囼}{60603}
\saveTG{䍝}{60603}
\saveTG{䍣}{60604}
\saveTG{𡇜}{60604}
\saveTG{𣌿}{60604}
\saveTG{𡈄}{60604}
\saveTG{𡈉}{60604}
\saveTG{𡈥}{60604}
\saveTG{𦊲}{60604}
\saveTG{𡇣}{60604}
\saveTG{啚}{60604}
\saveTG{罸}{60604}
\saveTG{固}{60604}
\saveTG{罟}{60604}
\saveTG{晷}{60604}
\saveTG{畧}{60604}
\saveTG{署}{60604}
\saveTG{圖}{60604}
\saveTG{圗}{60604}
\saveTG{暑}{60604}
\saveTG{𠰬}{60605}
\saveTG{罾}{60606}
\saveTG{𦋿}{60606}
\saveTG{𡇱}{60606}
\saveTG{𫅋}{60606}
\saveTG{𪱈}{60606}
\saveTG{𤲉}{60608}
\saveTG{𡈕}{60608}
\saveTG{𡇡}{60608}
\saveTG{𠱑}{60608}
\saveTG{𦊾}{60609}
\saveTG{𩀏}{60615}
\saveTG{𩁜}{60615}
\saveTG{𨇛}{60617}
\saveTG{𫅉}{60617}
\saveTG{罰}{60620}
\saveTG{𪽍}{60621}
\saveTG{𥉣}{60627}
\saveTG{𣇵}{60627}
\saveTG{𦊛}{60627}
\saveTG{𦊒}{60627}
\saveTG{𪽢}{60631}
\saveTG{𡈟}{60632}
\saveTG{𣊕}{60632}
\saveTG{𦊯}{60641}
\saveTG{冔}{60641}
\saveTG{𦋂}{60643}
\saveTG{𦋷}{60648}
\saveTG{品}{60660}
\saveTG{瞐}{60660}
\saveTG{畾}{60660}
\saveTG{𡈲}{60660}
\saveTG{𦋹}{60660}
\saveTG{晶}{60660}
\saveTG{𦌽}{60661}
\saveTG{𧠮}{60661}
\saveTG{𡈰}{60661}
\saveTG{𧮀}{60661}
\saveTG{𧭿}{60661}
\saveTG{𣌆}{60664}
\saveTG{𡈨}{60666}
\saveTG{𦌵}{60666}
\saveTG{𩉉}{60677}
\saveTG{𣋚}{60682}
\saveTG{𣋾}{60695}
\saveTG{𣌚}{60696}
\saveTG{𣅖}{60710}
\saveTG{𦉾}{60710}
\saveTG{𡆲}{60710}
\saveTG{𦊺}{60711}
\saveTG{𠯕}{60711}
\saveTG{昆}{60712}
\saveTG{毘}{60712}
\saveTG{𥃩}{60712}
\saveTG{㫐}{60712}
\saveTG{𡇖}{60712}
\saveTG{𡇠}{60712}
\saveTG{䍖}{60712}
\saveTG{𦋓}{60712}
\saveTG{圈}{60712}
\saveTG{囵}{60712}
\saveTG{𦉱}{60714}
\saveTG{𣬦}{60715}
\saveTG{𣰭}{60715}
\saveTG{𣰸}{60715}
\saveTG{㫣}{60715}
\saveTG{𦋸}{60715}
\saveTG{𤳍}{60715}
\saveTG{𨿪}{60715}
\saveTG{罨}{60716}
\saveTG{黾}{60716}
\saveTG{鼌}{60716}
\saveTG{𪓕}{60716}
\saveTG{𣇣}{60716}
\saveTG{𣰱}{60716}
\saveTG{𢁐}{60717}
\saveTG{囤}{60717}
\saveTG{𡇃}{60717}
\saveTG{鼂}{60717}
\saveTG{𫜞}{60717}
\saveTG{𦊳}{60717}
\saveTG{𦌡}{60717}
\saveTG{𡆳}{60717}
\saveTG{邑}{60717}
\saveTG{𦊐}{60717}
\saveTG{𪓲}{60717}
\saveTG{𦋋}{60717}
\saveTG{𪢬}{60717}
\saveTG{𦊁}{60717}
\saveTG{𣆠}{60717}
\saveTG{𡆠}{60717}
\saveTG{圏}{60717}
\saveTG{𤰵}{60718}
\saveTG{罚}{60720}
\saveTG{𠰡}{60727}
\saveTG{𣌸}{60727}
\saveTG{昂}{60727}
\saveTG{𡇼}{60727}
\saveTG{𠮲}{60727}
\saveTG{𫅀}{60727}
\saveTG{𡈆}{60727}
\saveTG{𠰇}{60727}
\saveTG{𠵛}{60727}
\saveTG{𣋓}{60727}
\saveTG{𣅘}{60727}
\saveTG{𦊑}{60727}
\saveTG{曷}{60727}
\saveTG{𪰩}{60727}
\saveTG{𣇎}{60727}
\saveTG{䀚}{60727}
\saveTG{昴}{60727}
\saveTG{曇}{60731}
\saveTG{𡇁}{60731}
\saveTG{𡆩}{60731}
\saveTG{𡆻}{60731}
\saveTG{𡇎}{60731}
\saveTG{䍗}{60731}
\saveTG{𡇟}{60731}
\saveTG{𡆾}{60731}
\saveTG{𣆂}{60731}
\saveTG{𧘗}{60732}
\saveTG{𧚣}{60732}
\saveTG{𡇴}{60732}
\saveTG{囩}{60732}
\saveTG{畏}{60732}
\saveTG{昙}{60732}
\saveTG{睘}{60732}
\saveTG{囜}{60732}
\saveTG{曩}{60732}
\saveTG{瞏}{60732}
\saveTG{圜}{60732}
\saveTG{褁}{60732}
\saveTG{罢}{60732}
\saveTG{䍚}{60732}
\saveTG{𩜻}{60732}
\saveTG{𤲡}{60732}
\saveTG{𦊷}{60732}
\saveTG{𤲐}{60732}
\saveTG{𧝖}{60732}
\saveTG{𡈃}{60732}
\saveTG{𡈂}{60732}
\saveTG{𡆢}{60740}
\saveTG{𡆿}{60742}
\saveTG{䍕}{60743}
\saveTG{囻}{60747}
\saveTG{𠷛}{60747}
\saveTG{𠵶}{60747}
\saveTG{罠}{60747}
\saveTG{𡈑}{60753}
\saveTG{䍙}{60754}
\saveTG{𦊏}{60754}
\saveTG{𡶸}{60770}
\saveTG{旵}{60772}
\saveTG{罍}{60772}
\saveTG{𣅽}{60772}
\saveTG{𡾔}{60772}
\saveTG{嵒}{60772}
\saveTG{喦}{60772}
\saveTG{𠼧}{60772}
\saveTG{𡆯}{60772}
\saveTG{𡶩}{60772}
\saveTG{𦋣}{60772}
\saveTG{𠚎}{60772}
\saveTG{𣆵}{60777}
\saveTG{𦊿}{60777}
\saveTG{𡇑}{60777}
\saveTG{㘥}{60793}
\saveTG{𡈱}{60793}
\saveTG{𡈫}{60793}
\saveTG{𡆦}{60800}
\saveTG{𦉵}{60800}
\saveTG{囚}{60800}
\saveTG{𡆣}{60800}
\saveTG{昗}{60800}
\saveTG{只}{60800}
\saveTG{貝}{60800}
\saveTG{𠔓}{60801}
\saveTG{𣉮}{60801}
\saveTG{𠯵}{60801}
\saveTG{𠶵}{60801}
\saveTG{𡈓}{60801}
\saveTG{𡇥}{60801}
\saveTG{𣇳}{60801}
\saveTG{𦋊}{60801}
\saveTG{𠔱}{60801}
\saveTG{𦌔}{60801}
\saveTG{𨅇}{60801}
\saveTG{足}{60801}
\saveTG{是}{60801}
\saveTG{異}{60801}
\saveTG{𫅈}{60801}
\saveTG{𧻻}{60802}
\saveTG{𦋨}{60802}
\saveTG{𤲊}{60802}
\saveTG{𧹒}{60802}
\saveTG{员}{60802}
\saveTG{圆}{60802}
\saveTG{𣈩}{60803}
\saveTG{㕦}{60804}
\saveTG{𦉼}{60804}
\saveTG{𣉉}{60804}
\saveTG{𠴩}{60804}
\saveTG{𡗽}{60804}
\saveTG{𦊊}{60804}
\saveTG{𪰳}{60804}
\saveTG{𤣂}{60804}
\saveTG{𡈗}{60804}
\saveTG{𤳦}{60804}
\saveTG{𦌚}{60804}
\saveTG{𪰖}{60804}
\saveTG{奰}{60804}
\saveTG{昊}{60804}
\saveTG{狊}{60804}
\saveTG{旲}{60804}
\saveTG{吴}{60804}
\saveTG{因}{60804}
\saveTG{𪰔}{60805}
\saveTG{𠱵}{60805}
\saveTG{𧵗}{60806}
\saveTG{𠽷}{60806}
\saveTG{𧵞}{60806}
\saveTG{𣊇}{60806}
\saveTG{𦌷}{60806}
\saveTG{圚}{60806}
\saveTG{買}{60806}
\saveTG{員}{60806}
\saveTG{圎}{60806}
\saveTG{圓}{60806}
\saveTG{𣊂}{60808}
\saveTG{𣊜}{60808}
\saveTG{𤇀}{60809}
\saveTG{𡇂}{60809}
\saveTG{㬃}{60809}
\saveTG{𧿮}{60809}
\saveTG{炅}{60809}
\saveTG{賘}{60813}
\saveTG{䝬}{60814}
\saveTG{𧸳}{60814}
\saveTG{賍}{60814}
\saveTG{𦍀}{60815}
\saveTG{𩀎}{60815}
\saveTG{𨿎}{60815}
\saveTG{𧸌}{60815}
\saveTG{𠺯}{60817}
\saveTG{貥}{60817}
\saveTG{𣋽}{60820}
\saveTG{𦋺}{60820}
\saveTG{𧶺}{60821}
\saveTG{员}{60827}
\saveTG{𦋴}{60827}
\saveTG{䝮}{60831}
\saveTG{𧵾}{60832}
\saveTG{眾}{60832}
\saveTG{𧸫}{60832}
\saveTG{𦌣}{60832}
\saveTG{𧸜}{60837}
\saveTG{𧸖}{60837}
\saveTG{𧵻}{60841}
\saveTG{𧶴}{60847}
\saveTG{賋}{60848}
\saveTG{賥}{60848}
\saveTG{𧴴}{60860}
\saveTG{𧷯}{60861}
\saveTG{賠}{60861}
\saveTG{𧶆}{60864}
\saveTG{𣊋}{60864}
\saveTG{𨆬}{60882}
\saveTG{𣌋}{60882}
\saveTG{𦌗}{60882}
\saveTG{賅}{60882}
\saveTG{𡚤}{60884}
\saveTG{𡈍}{60884}
\saveTG{贔}{60886}
\saveTG{𪓇}{60886}
\saveTG{𧸼}{60886}
\saveTG{䝶}{60896}
\saveTG{𣇪}{60897}
\saveTG{𦋎}{60898}
\saveTG{𪐗}{60898}
\saveTG{𡆧}{60900}
\saveTG{𣌢}{60900}
\saveTG{𡭨}{60900}
\saveTG{𦊎}{60901}
\saveTG{囨}{60901}
\saveTG{𫀝}{60901}
\saveTG{罘}{60901}
\saveTG{𡈜}{60901}
\saveTG{囦}{60902}
\saveTG{㬧}{60903}
\saveTG{𦃙}{60903}
\saveTG{纍}{60903}
\saveTG{累}{60903}
\saveTG{䋰}{60903}
\saveTG{𡈢}{60903}
\saveTG{𦍁}{60903}
\saveTG{𦅔}{60903}
\saveTG{𦀤}{60903}
\saveTG{𧹈}{60904}
\saveTG{𣏁}{60904}
\saveTG{𦊧}{60904}
\saveTG{䍒}{60904}
\saveTG{𣐮}{60904}
\saveTG{𡇙}{60904}
\saveTG{𡮧}{60904}
\saveTG{𣇆}{60904}
\saveTG{𠰓}{60904}
\saveTG{𪐺}{60904}
\saveTG{𦊜}{60904}
\saveTG{𪢮}{60904}
\saveTG{𣈎}{60904}
\saveTG{𦌮}{60904}
\saveTG{𦋡}{60904}
\saveTG{𣊻}{60904}
\saveTG{𥅼}{60904}
\saveTG{𡇒}{60904}
\saveTG{㫧}{60904}
\saveTG{䍘}{60904}
\saveTG{圞}{60904}
\saveTG{呆}{60904}
\saveTG{罺}{60904}
\saveTG{杲}{60904}
\saveTG{果}{60904}
\saveTG{困}{60904}
\saveTG{櫐}{60904}
\saveTG{喿}{60904}
\saveTG{囷}{60904}
\saveTG{𦌆}{60905}
\saveTG{𣉰}{60905}
\saveTG{景}{60906}
\saveTG{𦌒}{60906}
\saveTG{㬌}{60906}
\saveTG{眔}{60909}
\saveTG{暴}{60909}
\saveTG{𡇋}{60909}
\saveTG{𣉴}{60914}
\saveTG{罹}{60915}
\saveTG{𩀚}{60915}
\saveTG{羅}{60915}
\saveTG{𩁠}{60915}
\saveTG{𩀺}{60915}
\saveTG{𦊻}{60917}
\saveTG{𡈏}{60917}
\saveTG{𣉶}{60917}
\saveTG{𣌯}{60917}
\saveTG{𦊵}{60917}
\saveTG{𡇿}{60920}
\saveTG{𤳝}{60920}
\saveTG{𦋽}{60921}
\saveTG{𪰬}{60921}
\saveTG{𦍉}{60921}
\saveTG{𦌰}{60921}
\saveTG{羂}{60927}
\saveTG{𦌢}{60927}
\saveTG{𦋥}{60927}
\saveTG{䯫}{60927}
\saveTG{𦌾}{60932}
\saveTG{𦍆}{60932}
\saveTG{𡈡}{60933}
\saveTG{𦌦}{60933}
\saveTG{𣐥}{60940}
\saveTG{𦋝}{60943}
\saveTG{羄}{60946}
\saveTG{𪳿}{60947}
\saveTG{𦌅}{60947}
\saveTG{𣈕}{60948}
\saveTG{𦌉}{60948}
\saveTG{㮂}{60948}
\saveTG{㬥}{60949}
\saveTG{𡇯}{60956}
\saveTG{𦌊}{60982}
\saveTG{𣡚}{60986}
\saveTG{𣌛}{60986}
\saveTG{𦌪}{60989}
\saveTG{𥄳}{60990}
\saveTG{𦋏}{60991}
\saveTG{𥍙}{60991}
\saveTG{圝}{60993}
\saveTG{羉}{60993}
\saveTG{𡈴}{60993}
\saveTG{𣊡}{60993}
\saveTG{罧}{60994}
\saveTG{𡈹}{60994}
\saveTG{𤳥}{60994}
\saveTG{𤳂}{60994}
\saveTG{𣇰}{60994}
\saveTG{𣡗}{60995}
\saveTG{𦌟}{60999}
\saveTG{眐}{61011}
\saveTG{𡁭}{61011}
\saveTG{𠼕}{61011}
\saveTG{哐}{61011}
\saveTG{㫵}{61011}
\saveTG{𠯔}{61011}
\saveTG{𥇪}{61011}
\saveTG{眶}{61011}
\saveTG{啡}{61011}
\saveTG{曨}{61011}
\saveTG{矓}{61011}
\saveTG{嚨}{61011}
\saveTG{嚦}{61011}
\saveTG{𥆦}{61011}
\saveTG{𤳽}{61011}
\saveTG{𠳧}{61011}
\saveTG{㗺}{61011}
\saveTG{𠯽}{61012}
\saveTG{𠮵}{61012}
\saveTG{𠰪}{61012}
\saveTG{𥃽}{61012}
\saveTG{𥄬}{61012}
\saveTG{䁦}{61012}
\saveTG{𥋖}{61012}
\saveTG{𣇥}{61012}
\saveTG{𣈶}{61012}
\saveTG{𠸳}{61012}
\saveTG{叿}{61012}
\saveTG{𠳍}{61012}
\saveTG{𠿙}{61012}
\saveTG{曬}{61012}
\saveTG{㕶}{61012}
\saveTG{𠻺}{61012}
\saveTG{𡄊}{61012}
\saveTG{𠳒}{61012}
\saveTG{𠿓}{61012}
\saveTG{𥌮}{61012}
\saveTG{呃}{61012}
\saveTG{𪰆}{61012}
\saveTG{哑}{61012}
\saveTG{啞}{61012}
\saveTG{呒}{61012}
\saveTG{嘅}{61012}
\saveTG{囇}{61012}
\saveTG{矖}{61012}
\saveTG{嚧}{61012}
\saveTG{矑}{61012}
\saveTG{𥆀}{61012}
\saveTG{咂}{61012}
\saveTG{盶}{61012}
\saveTG{曥}{61012}
\saveTG{𠱸}{61012}
\saveTG{𣇁}{61012}
\saveTG{䀴}{61012}
\saveTG{嘘}{61012}
\saveTG{𪰈}{61012}
\saveTG{𪡿}{61012}
\saveTG{𠸣}{61012}
\saveTG{𠳏}{61012}
\saveTG{𠲞}{61012}
\saveTG{𤱻}{61012}
\saveTG{𠲬}{61012}
\saveTG{𥇞}{61012}
\saveTG{𠿅}{61012}
\saveTG{𤱥}{61012}
\saveTG{𠾜}{61012}
\saveTG{𠲮}{61012}
\saveTG{㗏}{61012}
\saveTG{𠸟}{61012}
\saveTG{㖶}{61014}
\saveTG{𣈿}{61014}
\saveTG{眰}{61014}
\saveTG{𠸯}{61014}
\saveTG{𠼸}{61014}
\saveTG{𥉢}{61014}
\saveTG{𣉓}{61014}
\saveTG{𠴝}{61014}
\saveTG{𠶖}{61014}
\saveTG{睚}{61014}
\saveTG{𥆚}{61014}
\saveTG{旺}{61014}
\saveTG{呕}{61014}
\saveTG{眍}{61014}
\saveTG{啀}{61014}
\saveTG{咥}{61014}
\saveTG{𥈴}{61014}
\saveTG{𥈂}{61014}
\saveTG{㖸}{61014}
\saveTG{𠽎}{61014}
\saveTG{晊}{61014}
\saveTG{𡂥}{61014}
\saveTG{𠽥}{61014}
\saveTG{𠽧}{61014}
\saveTG{𠹿}{61014}
\saveTG{𠰧}{61014}
\saveTG{𥈔}{61014}
\saveTG{唖}{61015}
\saveTG{嚯}{61015}
\saveTG{矐}{61015}
\saveTG{𪡄}{61015}
\saveTG{𪰥}{61015}
\saveTG{𣌓}{61015}
\saveTG{㕵}{61015}
\saveTG{曤}{61015}
\saveTG{喱}{61015}
\saveTG{咺}{61016}
\saveTG{疅}{61016}
\saveTG{䁥}{61016}
\saveTG{𠽋}{61016}
\saveTG{𣉾}{61016}
\saveTG{暅}{61016}
\saveTG{瞘}{61016}
\saveTG{嘔}{61016}
\saveTG{晅}{61016}
\saveTG{暱}{61016}
\saveTG{𠵏}{61016}
\saveTG{𠷐}{61016}
\saveTG{𠰐}{61016}
\saveTG{𣉖}{61016}
\saveTG{𥅨}{61016}
\saveTG{𡃢}{61016}
\saveTG{𠰂}{61017}
\saveTG{𠶃}{61017}
\saveTG{𠳟}{61017}
\saveTG{𠮾}{61017}
\saveTG{𡀱}{61017}
\saveTG{咓}{61017}
\saveTG{噓}{61017}
\saveTG{昛}{61017}
\saveTG{矖}{61017}
\saveTG{𥆆}{61017}
\saveTG{𡁙}{61017}
\saveTG{𠻒}{61017}
\saveTG{唬}{61017}
\saveTG{𥌯}{61017}
\saveTG{𥄷}{61017}
\saveTG{𠰠}{61017}
\saveTG{𡁓}{61017}
\saveTG{𡅃}{61017}
\saveTG{𡄼}{61017}
\saveTG{𠿜}{61017}
\saveTG{𠳔}{61017}
\saveTG{𤭳}{61017}
\saveTG{𤮋}{61017}
\saveTG{𥉅}{61017}
\saveTG{𣅥}{61017}
\saveTG{𠯞}{61017}
\saveTG{𠰆}{61017}
\saveTG{𠽴}{61017}
\saveTG{𪽊}{61017}
\saveTG{𥆍}{61017}
\saveTG{𥆮}{61017}
\saveTG{𥄙}{61017}
\saveTG{𥉿}{61017}
\saveTG{𥋳}{61017}
\saveTG{𣅪}{61017}
\saveTG{𠴉}{61017}
\saveTG{𠲷}{61017}
\saveTG{𠳠}{61017}
\saveTG{𥆖}{61018}
\saveTG{哣}{61018}
\saveTG{𣆏}{61019}
\saveTG{呸}{61019}
\saveTG{呵}{61020}
\saveTG{咑}{61020}
\saveTG{叮}{61020}
\saveTG{町}{61020}
\saveTG{哬}{61020}
\saveTG{𪰏}{61020}
\saveTG{盯}{61020}
\saveTG{𠯼}{61020}
\saveTG{啊}{61020}
\saveTG{䀪}{61021}
\saveTG{𠼫}{61021}
\saveTG{𠯸}{61021}
\saveTG{𠷢}{61021}
\saveTG{𡆚}{61021}
\saveTG{㘅}{61021}
\saveTG{㗸}{61021}
\saveTG{𠾑}{61021}
\saveTG{𠹭}{61021}
\saveTG{哘}{61021}
\saveTG{𣆯}{61021}
\saveTG{𠳤}{61022}
\saveTG{𡅿}{61026}
\saveTG{𥇍}{61026}
\saveTG{𠶚}{61026}
\saveTG{𠶄}{61026}
\saveTG{𪽎}{61026}
\saveTG{曘}{61027}
\saveTG{嘕}{61027}
\saveTG{噖}{61027}
\saveTG{𥊕}{61027}
\saveTG{𡂖}{61027}
\saveTG{𡁑}{61027}
\saveTG{𥇑}{61027}
\saveTG{啃}{61027}
\saveTG{嗝}{61027}
\saveTG{昞}{61027}
\saveTG{𡄣}{61027}
\saveTG{𡁠}{61027}
\saveTG{𠿛}{61027}
\saveTG{𣈺}{61027}
\saveTG{𠿎}{61027}
\saveTG{眪}{61027}
\saveTG{𠻢}{61027}
\saveTG{䀎}{61027}
\saveTG{𣆝}{61027}
\saveTG{㬏}{61027}
\saveTG{㖇}{61027}
\saveTG{𠯗}{61027}
\saveTG{𠰳}{61027}
\saveTG{𠯪}{61027}
\saveTG{𪾟}{61027}
\saveTG{𥉊}{61027}
\saveTG{𥋃}{61027}
\saveTG{𥌎}{61027}
\saveTG{𣊵}{61027}
\saveTG{𣅙}{61027}
\saveTG{𡃌}{61027}
\saveTG{𠮱}{61027}
\saveTG{𠰁}{61027}
\saveTG{𠶪}{61027}
\saveTG{𠹮}{61027}
\saveTG{𡃽}{61027}
\saveTG{𤳜}{61027}
\saveTG{㽭}{61027}
\saveTG{𥌋}{61027}
\saveTG{𥍗}{61027}
\saveTG{𠼉}{61027}
\saveTG{𥃳}{61027}
\saveTG{矋}{61027}
\saveTG{呖}{61027}
\saveTG{曞}{61027}
\saveTG{唡}{61027}
\saveTG{啢}{61027}
\saveTG{嗎}{61027}
\saveTG{眄}{61027}
\saveTG{嚅}{61027}
\saveTG{𥅡}{61027}
\saveTG{𥄚}{61030}
\saveTG{吓}{61030}
\saveTG{𣊯}{61031}
\saveTG{𣊾}{61031}
\saveTG{𠾦}{61031}
\saveTG{𤱂}{61031}
\saveTG{𤳟}{61031}
\saveTG{𥉖}{61031}
\saveTG{𥄆}{61031}
\saveTG{噁}{61031}
\saveTG{咔}{61031}
\saveTG{𠯝}{61031}
\saveTG{}{61031}
\saveTG{𨌇}{61032}
\saveTG{𠳃}{61032}
\saveTG{㖘}{61032}
\saveTG{𠺲}{61032}
\saveTG{𤲘}{61032}
\saveTG{𤱼}{61032}
\saveTG{𥆋}{61032}
\saveTG{䀼}{61032}
\saveTG{㬡}{61032}
\saveTG{𥋞}{61032}
\saveTG{眃}{61032}
\saveTG{噱}{61032}
\saveTG{𠴴}{61032}
\saveTG{啄}{61032}
\saveTG{呍}{61032}
\saveTG{𠽖}{61033}
\saveTG{𠿗}{61034}
\saveTG{𡃖}{61036}
\saveTG{𠻹}{61038}
\saveTG{㖭}{61038}
\saveTG{𤲖}{61038}
\saveTG{盱}{61040}
\saveTG{旴}{61040}
\saveTG{呀}{61040}
\saveTG{盰}{61040}
\saveTG{旰}{61040}
\saveTG{咞}{61040}
\saveTG{𣆙}{61040}
\saveTG{𠱢}{61040}
\saveTG{𠻃}{61040}
\saveTG{𤰟}{61040}
\saveTG{㗔}{61040}
\saveTG{眲}{61040}
\saveTG{咡}{61040}
\saveTG{吁}{61040}
\saveTG{哶}{61041}
\saveTG{𥍉}{61041}
\saveTG{𥆊}{61041}
\saveTG{哢}{61041}
\saveTG{𠰑}{61041}
\saveTG{囁}{61041}
\saveTG{𡆄}{61042}
\saveTG{𣌍}{61042}
\saveTG{𡂩}{61042}
\saveTG{𠾇}{61043}
\saveTG{嗕}{61043}
\saveTG{𠶡}{61044}
\saveTG{𣉋}{61044}
\saveTG{𠱻}{61044}
\saveTG{𥇼}{61044}
\saveTG{䀘}{61044}
\saveTG{喓}{61044}
\saveTG{𠻔}{61045}
\saveTG{𠼥}{61045}
\saveTG{𠷊}{61045}
\saveTG{𠼰}{61045}
\saveTG{𣆳}{61045}
\saveTG{𡁇}{61046}
\saveTG{𤲤}{61046}
\saveTG{䁏}{61046}
\saveTG{嘾}{61046}
\saveTG{曋}{61046}
\saveTG{瞫}{61046}
\saveTG{啅}{61046}
\saveTG{晫}{61046}
\saveTG{哽}{61046}
\saveTG{𡄟}{61047}
\saveTG{嗄}{61047}
\saveTG{嗫}{61047}
\saveTG{嚘}{61047}
\saveTG{𣌘}{61047}
\saveTG{𠯘}{61047}
\saveTG{𡁞}{61047}
\saveTG{𪰲}{61047}
\saveTG{𠷋}{61047}
\saveTG{𠶺}{61047}
\saveTG{𠽁}{61047}
\saveTG{𠳰}{61047}
\saveTG{𡁁}{61047}
\saveTG{𡆇}{61047}
\saveTG{𡅒}{61047}
\saveTG{𣌊}{61047}
\saveTG{𠹅}{61048}
\saveTG{𠹶}{61049}
\saveTG{𡀛}{61049}
\saveTG{嘑}{61049}
\saveTG{嚹}{61049}
\saveTG{呯}{61049}
\saveTG{㗑}{61050}
\saveTG{𠴟}{61052}
\saveTG{噦}{61053}
\saveTG{嘎}{61053}
\saveTG{嘠}{61053}
\saveTG{𡅺}{61054}
\saveTG{𡂣}{61055}
\saveTG{𣊢}{61056}
\saveTG{𠱏}{61057}
\saveTG{呫}{61060}
\saveTG{晤}{61061}
\saveTG{唔}{61061}
\saveTG{噆}{61061}
\saveTG{𡂂}{61061}
\saveTG{𠼘}{61061}
\saveTG{𥌼}{61061}
\saveTG{𥋸}{61061}
\saveTG{𥆐}{61061}
\saveTG{㬐}{61061}
\saveTG{㘊}{61061}
\saveTG{𥅭}{61061}
\saveTG{𡅈}{61062}
\saveTG{𠳱}{61062}
\saveTG{𠰴}{61062}
\saveTG{𡂳}{61062}
\saveTG{喕}{61062}
\saveTG{𠸸}{61062}
\saveTG{嚸}{61062}
\saveTG{䀡}{61062}
\saveTG{𥈅}{61062}
\saveTG{咟}{61062}
\saveTG{𡀂}{61063}
\saveTG{晒}{61064}
\saveTG{𥆺}{61064}
\saveTG{唒}{61064}
\saveTG{嗮}{61064}
\saveTG{哂}{61064}
\saveTG{𠸢}{61066}
\saveTG{䁮}{61067}
\saveTG{㖔}{61068}
\saveTG{𥌑}{61068}
\saveTG{𠳝}{61069}
\saveTG{𠵠}{61069}
\saveTG{𣇊}{61069}
\saveTG{𥄝}{61072}
\saveTG{𠵾}{61072}
\saveTG{嚙}{61072}
\saveTG{噛}{61072}
\saveTG{啮}{61072}
\saveTG{𠶲}{61072}
\saveTG{𠷭}{61072}
\saveTG{𣆐}{61077}
\saveTG{𣋭}{61077}
\saveTG{𪾦}{61077}
\saveTG{𠰷}{61077}
\saveTG{𠽌}{61077}
\saveTG{嘥}{61081}
\saveTG{噘}{61082}
\saveTG{𠯬}{61082}
\saveTG{唝}{61082}
\saveTG{𪢋}{61082}
\saveTG{𠸝}{61082}
\saveTG{𥊂}{61082}
\saveTG{嚈}{61084}
\saveTG{𤲬}{61084}
\saveTG{𠷀}{61084}
\saveTG{㬉}{61084}
\saveTG{噳}{61084}
\saveTG{𥈇}{61084}
\saveTG{䀖}{61084}
\saveTG{𩓽}{61086}
\saveTG{𩒱}{61086}
\saveTG{𠼴}{61086}
\saveTG{𩒻}{61086}
\saveTG{㘖}{61086}
\saveTG{𡄶}{61086}
\saveTG{𠽒}{61086}
\saveTG{𡅅}{61086}
\saveTG{𡆘}{61086}
\saveTG{𡅑}{61086}
\saveTG{𠸩}{61086}
\saveTG{𥌨}{61086}
\saveTG{𡂄}{61086}
\saveTG{𣌔}{61086}
\saveTG{𥋣}{61086}
\saveTG{嚬}{61086}
\saveTG{䁰}{61086}
\saveTG{𥈗}{61086}
\saveTG{𩖇}{61086}
\saveTG{𡅥}{61086}
\saveTG{𠾿}{61086}
\saveTG{㖽}{61086}
\saveTG{𡅪}{61086}
\saveTG{𠾫}{61086}
\saveTG{𥈿}{61086}
\saveTG{𩔠}{61086}
\saveTG{暊}{61086}
\saveTG{嗊}{61086}
\saveTG{噸}{61086}
\saveTG{𠿪}{61086}
\saveTG{𥋄}{61089}
\saveTG{𣊠}{61089}
\saveTG{𣊥}{61089}
\saveTG{吥}{61090}
\saveTG{㬓}{61091}
\saveTG{𠱙}{61091}
\saveTG{眎}{61091}
\saveTG{呩}{61091}
\saveTG{嘌}{61091}
\saveTG{瞟}{61091}
\saveTG{𡁼}{61091}
\saveTG{𠹀}{61093}
\saveTG{𠹦}{61093}
\saveTG{𠹱}{61094}
\saveTG{㗚}{61094}
\saveTG{𠺿}{61096}
\saveTG{𨄏}{61101}
\saveTG{㙱}{61101}
\saveTG{𨇎}{61104}
\saveTG{𡏔}{61104}
\saveTG{趾}{61110}
\saveTG{𫕾}{61111}
\saveTG{𧍃}{61111}
\saveTG{䠊}{61111}
\saveTG{𨀕}{61111}
\saveTG{躘}{61111}
\saveTG{踁}{61112}
\saveTG{踂}{61112}
\saveTG{躧}{61112}
\saveTG{䠡}{61112}
\saveTG{𧿟}{61112}
\saveTG{𨇗}{61112}
\saveTG{𨁶}{61112}
\saveTG{𨇖}{61112}
\saveTG{𨀹}{61112}
\saveTG{𨂿}{61112}
\saveTG{𨀿}{61112}
\saveTG{𣌜}{61112}
\saveTG{𧊻}{61114}
\saveTG{𧿷}{61114}
\saveTG{𨇠}{61114}
\saveTG{𪻹}{61114}
\saveTG{𨂉}{61114}
\saveTG{𨁨}{61114}
\saveTG{跮}{61114}
\saveTG{𨂷}{61115}
\saveTG{𫏣}{61116}
\saveTG{𨄅}{61116}
\saveTG{𨆱}{61116}
\saveTG{𧿁}{61117}
\saveTG{蹮}{61117}
\saveTG{𨇆}{61117}
\saveTG{距}{61117}
\saveTG{𤭺}{61117}
\saveTG{𨂜}{61117}
\saveTG{𤭼}{61117}
\saveTG{𨀄}{61117}
\saveTG{𧿙}{61117}
\saveTG{𨅂}{61117}
\saveTG{䥤}{61117}
\saveTG{𧿠}{61117}
\saveTG{𧿊}{61117}
\saveTG{𫏁}{61118}
\saveTG{𨁋}{61118}
\saveTG{跒}{61120}
\saveTG{䟓}{61120}
\saveTG{𧿄}{61120}
\saveTG{𨇙}{61121}
\saveTG{踄}{61121}
\saveTG{䟰}{61121}
\saveTG{䟚}{61121}
\saveTG{𪱊}{61126}
\saveTG{𨂎}{61127}
\saveTG{𫏧}{61127}
\saveTG{𧿉}{61127}
\saveTG{䠃}{61127}
\saveTG{𨅲}{61127}
\saveTG{𣋮}{61127}
\saveTG{躎}{61127}
\saveTG{}{61127}
\saveTG{𨂶}{61127}
\saveTG{𨅥}{61131}
\saveTG{躚}{61131}
\saveTG{躆}{61132}
\saveTG{𨄰}{61132}
\saveTG{䠆}{61132}
\saveTG{𨁿}{61132}
\saveTG{䟴}{61132}
\saveTG{𨂽}{61132}
\saveTG{𨆖}{61132}
\saveTG{𨄃}{61133}
\saveTG{𨆙}{61134}
\saveTG{𨃊}{61136}
\saveTG{𧖙}{61136}
\saveTG{𨈌}{61139}
\saveTG{𨆸}{61139}
\saveTG{趼}{61140}
\saveTG{趶}{61140}
\saveTG{蹰}{61140}
\saveTG{𧿂}{61140}
\saveTG{躡}{61141}
\saveTG{𨃽}{61143}
\saveTG{𨆼}{61143}
\saveTG{𨄖}{61144}
\saveTG{趼}{61144}
\saveTG{𨁦}{61144}
\saveTG{𨄵}{61146}
\saveTG{𨅭}{61146}
\saveTG{踔}{61146}
\saveTG{𨈉}{61147}
\saveTG{𧿋}{61147}
\saveTG{斀}{61147}
\saveTG{𨅅}{61147}
\saveTG{𨇄}{61147}
\saveTG{蹑}{61147}
\saveTG{𨄥}{61149}
\saveTG{𨃅}{61150}
\saveTG{䠩}{61152}
\saveTG{𨁈}{61156}
\saveTG{𨂯}{61156}
\saveTG{𧿓}{61157}
\saveTG{跕}{61160}
\saveTG{𨅔}{61161}
\saveTG{𨇒}{61162}
\saveTG{跖}{61162}
\saveTG{跴}{61164}
\saveTG{𨆢}{61164}
\saveTG{𨁪}{61164}
\saveTG{踾}{61166}
\saveTG{𧿽}{61166}
\saveTG{𨁇}{61168}
\saveTG{踎}{61169}
\saveTG{𪘏}{61172}
\saveTG{𨂵}{61177}
\saveTG{蹝}{61181}
\saveTG{蹶}{61182}
\saveTG{𨁾}{61182}
\saveTG{𩓶}{61186}
\saveTG{𩒇}{61186}
\saveTG{𨇪}{61186}
\saveTG{𨆌}{61186}
\saveTG{𨆆}{61186}
\saveTG{𨅑}{61186}
\saveTG{𩔐}{61186}
\saveTG{𩔰}{61186}
\saveTG{𨈀}{61186}
\saveTG{𣌌}{61186}
\saveTG{𩑰}{61186}
\saveTG{𫖥}{61186}
\saveTG{𧰑}{61186}
\saveTG{𨂠}{61186}
\saveTG{顕}{61186}
\saveTG{蹞}{61186}
\saveTG{䫳}{61186}
\saveTG{𨆵}{61186}
\saveTG{𨂑}{61187}
\saveTG{𨅟}{61189}
\saveTG{𥅊}{61190}
\saveTG{𧿤}{61190}
\saveTG{𨀚}{61191}
\saveTG{𨆺}{61191}
\saveTG{𨃙}{61194}
\saveTG{𥍕}{61212}
\saveTG{𧡄}{61212}
\saveTG{𧢁}{61214}
\saveTG{𧠖}{61214}
\saveTG{𪼺}{61217}
\saveTG{號}{61217}
\saveTG{瓹}{61217}
\saveTG{㼴}{61217}
\saveTG{𧇼}{61218}
\saveTG{𫌧}{61218}
\saveTG{𢽕}{61219}
\saveTG{𡂲}{61221}
\saveTG{𥝂}{61227}
\saveTG{𩧚}{61227}
\saveTG{𤲂}{61230}
\saveTG{𨲖}{61231}
\saveTG{𢾙}{61247}
\saveTG{敡}{61247}
\saveTG{𩉘}{61262}
\saveTG{𨇡}{61266}
\saveTG{颚}{61282}
\saveTG{颙}{61282}
\saveTG{顒}{61286}
\saveTG{𩓷}{61286}
\saveTG{𩕩}{61286}
\saveTG{𩕾}{61286}
\saveTG{顎}{61286}
\saveTG{黖}{61312}
\saveTG{𪓀}{61312}
\saveTG{黸}{61312}
\saveTG{黫}{61314}
\saveTG{𪐬}{61317}
\saveTG{𪐚}{61327}
\saveTG{𪑷}{61327}
\saveTG{𢙸}{61331}
\saveTG{𪒝}{61331}
\saveTG{𢤃}{61331}
\saveTG{㸃}{61336}
\saveTG{𪹭}{61336}
\saveTG{䵟}{61340}
\saveTG{䵦}{61340}
\saveTG{𦖷}{61342}
\saveTG{𪑾}{61343}
\saveTG{𪒤}{61346}
\saveTG{𪑉}{61347}
\saveTG{𣌏}{61347}
\saveTG{𪒩}{61352}
\saveTG{點}{61362}
\saveTG{𪒽}{61362}
\saveTG{顯}{61386}
\saveTG{顋}{61386}
\saveTG{𪑸}{61386}
\saveTG{𪑳}{61386}
\saveTG{𩒳}{61386}
\saveTG{𪒳}{61396}
\saveTG{𦥁}{61414}
\saveTG{𪟡}{61426}
\saveTG{𩧡}{61427}
\saveTG{𣀇}{61447}
\saveTG{㪋}{61447}
\saveTG{𤳢}{61486}
\saveTG{𩖍}{61486}
\saveTG{𡁉}{61506}
\saveTG{𩎍}{61506}
\saveTG{𧈍}{61517}
\saveTG{辴}{61532}
\saveTG{眶}{61611}
\saveTG{𠯡}{61617}
\saveTG{𤳠}{61627}
\saveTG{𥌃}{61627}
\saveTG{𥇔}{61632}
\saveTG{瞫}{61646}
\saveTG{𣀜}{61647}
\saveTG{𧈐}{61717}
\saveTG{𤮦}{61717}
\saveTG{𧇷}{61717}
\saveTG{𩞂}{61732}
\saveTG{饕}{61732}
\saveTG{𢆜}{61740}
\saveTG{𩕪}{61786}
\saveTG{𩒁}{61786}
\saveTG{𩓓}{61786}
\saveTG{𩒝}{61786}
\saveTG{䫘}{61786}
\saveTG{䪽}{61786}
\saveTG{㼵}{61801}
\saveTG{𣊰}{61801}
\saveTG{𢆝}{61804}
\saveTG{𪰾}{61806}
\saveTG{𦧪}{61806}
\saveTG{題}{61808}
\saveTG{题}{61808}
\saveTG{㷡}{61809}
\saveTG{煚}{61809}
\saveTG{贚}{61811}
\saveTG{貦}{61812}
\saveTG{𧸰}{61812}
\saveTG{𧴽}{61814}
\saveTG{䝽}{61814}
\saveTG{䞁}{61814}
\saveTG{𧸪}{61814}
\saveTG{𧶋}{61817}
\saveTG{𧴷}{61822}
\saveTG{𧵛}{61826}
\saveTG{𠾟}{61826}
\saveTG{𧸱}{61827}
\saveTG{𧸀}{61827}
\saveTG{𧶊}{61831}
\saveTG{𧵇}{61831}
\saveTG{賑}{61832}
\saveTG{賬}{61832}
\saveTG{贉}{61846}
\saveTG{𧶶}{61847}
\saveTG{𣀕}{61847}
\saveTG{𧸗}{61852}
\saveTG{貼}{61860}
\saveTG{𧵀}{61860}
\saveTG{䞅}{61863}
\saveTG{𧵳}{61864}
\saveTG{𧸩}{61868}
\saveTG{䞊}{61868}
\saveTG{𧶵}{61877}
\saveTG{𫖲}{61882}
\saveTG{䞂}{61884}
\saveTG{𧶸}{61886}
\saveTG{𩓺}{61886}
\saveTG{䫟}{61886}
\saveTG{𩓌}{61886}
\saveTG{𧵋}{61891}
\saveTG{𣜍}{61904}
\saveTG{𣔣}{61914}
\saveTG{𣞃}{61916}
\saveTG{𤮉}{61917}
\saveTG{𤭋}{61917}
\saveTG{𡅗}{61917}
\saveTG{㼫}{61917}
\saveTG{𣘫}{61917}
\saveTG{敤}{61947}
\saveTG{𢿾}{61947}
\saveTG{𣀛}{61947}
\saveTG{𠹘}{61947}
\saveTG{𠸞}{61962}
\saveTG{𥈤}{61982}
\saveTG{颢}{61982}
\saveTG{颗}{61982}
\saveTG{𩕃}{61986}
\saveTG{𩔏}{61986}
\saveTG{𩓢}{61986}
\saveTG{𡀶}{61986}
\saveTG{𫖢}{61986}
\saveTG{𩑵}{61986}
\saveTG{顆}{61986}
\saveTG{顥}{61986}
\saveTG{𥃹}{62000}
\saveTG{𥃧}{62000}
\saveTG{𪰢}{62000}
\saveTG{𣉇}{62000}
\saveTG{㫼}{62000}
\saveTG{𣉤}{62000}
\saveTG{唰}{62000}
\saveTG{吲}{62000}
\saveTG{甽}{62000}
\saveTG{唎}{62000}
\saveTG{𠴼}{62000}
\saveTG{喇}{62000}
\saveTG{叫}{62000}
\saveTG{瞓}{62000}
\saveTG{哵}{62000}
\saveTG{𠜠}{62000}
\saveTG{𠛭}{62000}
\saveTG{㖄}{62000}
\saveTG{𡃍}{62000}
\saveTG{𠜟}{62000}
\saveTG{𠵯}{62000}
\saveTG{𠶜}{62000}
\saveTG{𠺓}{62000}
\saveTG{𠷌}{62000}
\saveTG{𠯀}{62000}
\saveTG{𠸑}{62000}
\saveTG{𥈙}{62000}
\saveTG{𠮝}{62000}
\saveTG{𣈛}{62000}
\saveTG{㘌}{62000}
\saveTG{𠰙}{62000}
\saveTG{咧}{62000}
\saveTG{𪰝}{62000}
\saveTG{嚠}{62000}
\saveTG{𥆁}{62000}
\saveTG{𠯖}{62007}
\saveTG{吼}{62010}
\saveTG{呲}{62010}
\saveTG{吡}{62010}
\saveTG{毗}{62010}
\saveTG{眦}{62010}
\saveTG{啂}{62010}
\saveTG{咝}{62011}
\saveTG{𥋕}{62012}
\saveTG{𠳥}{62012}
\saveTG{𤳶}{62012}
\saveTG{𥅔}{62012}
\saveTG{𠰋}{62012}
\saveTG{𠽲}{62012}
\saveTG{𥆙}{62012}
\saveTG{眺}{62013}
\saveTG{咷}{62013}
\saveTG{晀}{62013}
\saveTG{𠱁}{62013}
\saveTG{𥊅}{62014}
\saveTG{𠰃}{62014}
\saveTG{𥆂}{62014}
\saveTG{𥄮}{62014}
\saveTG{嘊}{62014}
\saveTG{𠴯}{62014}
\saveTG{𠾊}{62014}
\saveTG{𥋍}{62014}
\saveTG{𪾮}{62014}
\saveTG{𥆯}{62014}
\saveTG{吒}{62014}
\saveTG{𠲏}{62014}
\saveTG{眊}{62014}
\saveTG{𠽍}{62014}
\saveTG{𥅸}{62014}
\saveTG{𠹯}{62014}
\saveTG{𣇦}{62015}
\saveTG{畽}{62015}
\saveTG{嗺}{62015}
\saveTG{𣈧}{62015}
\saveTG{喠}{62015}
\saveTG{唾}{62015}
\saveTG{𡀰}{62015}
\saveTG{睡}{62015}
\saveTG{䁗}{62016}
\saveTG{䁽}{62016}
\saveTG{𣋲}{62016}
\saveTG{𡅘}{62016}
\saveTG{𡂏}{62016}
\saveTG{𤱩}{62017}
\saveTG{𠿯}{62017}
\saveTG{𠹌}{62017}
\saveTG{䀝}{62017}
\saveTG{𥉘}{62017}
\saveTG{𥊋}{62017}
\saveTG{𪾬}{62017}
\saveTG{𠃪}{62017}
\saveTG{𠴄}{62017}
\saveTG{嗁}{62017}
\saveTG{𠮜}{62017}
\saveTG{𣈃}{62017}
\saveTG{𠽶}{62017}
\saveTG{𥍋}{62017}
\saveTG{㕰}{62017}
\saveTG{𠵺}{62017}
\saveTG{𠳶}{62017}
\saveTG{𠶉}{62017}
\saveTG{𤰽}{62017}
\saveTG{嗈}{62017}
\saveTG{𠲪}{62017}
\saveTG{𥃤}{62017}
\saveTG{𠵣}{62017}
\saveTG{𠮪}{62017}
\saveTG{𠲥}{62017}
\saveTG{𠯩}{62017}
\saveTG{𠸕}{62017}
\saveTG{𠱹}{62017}
\saveTG{𣅒}{62017}
\saveTG{𣭃}{62017}
\saveTG{𤰦}{62017}
\saveTG{暟}{62018}
\saveTG{噔}{62018}
\saveTG{瞪}{62018}
\saveTG{𠹛}{62018}
\saveTG{𠺣}{62020}
\saveTG{䵞}{62020}
\saveTG{𥋠}{62020}
\saveTG{𥇂}{62020}
\saveTG{𥇕}{62020}
\saveTG{䀕}{62020}
\saveTG{晰}{62021}
\saveTG{听}{62021}
\saveTG{昕}{62021}
\saveTG{𪰇}{62021}
\saveTG{𠼹}{62021}
\saveTG{𠼗}{62021}
\saveTG{𣊙}{62021}
\saveTG{噺}{62021}
\saveTG{𥇦}{62021}
\saveTG{𣂽}{62021}
\saveTG{䀿}{62021}
\saveTG{𥇢}{62021}
\saveTG{䁪}{62021}
\saveTG{晣}{62021}
\saveTG{哳}{62021}
\saveTG{盺}{62021}
\saveTG{唽}{62021}
\saveTG{嘶}{62021}
\saveTG{𡂵}{62022}
\saveTG{䀐}{62022}
\saveTG{𪡊}{62022}
\saveTG{嘭}{62022}
\saveTG{𢒯}{62022}
\saveTG{𠷨}{62027}
\saveTG{𥍑}{62027}
\saveTG{𥆱}{62027}
\saveTG{㬙}{62027}
\saveTG{𡄴}{62027}
\saveTG{𠯯}{62027}
\saveTG{嘴}{62027}
\saveTG{𠿩}{62027}
\saveTG{嘺}{62027}
\saveTG{唀}{62027}
\saveTG{噅}{62027}
\saveTG{𠰗}{62027}
\saveTG{喘}{62027}
\saveTG{嘣}{62027}
\saveTG{𡁹}{62027}
\saveTG{哕}{62027}
\saveTG{㖗}{62027}
\saveTG{𠼩}{62027}
\saveTG{㽯}{62027}
\saveTG{𡆌}{62027}
\saveTG{𪢤}{62027}
\saveTG{𡅫}{62027}
\saveTG{𠼽}{62027}
\saveTG{𣇂}{62027}
\saveTG{𣋌}{62027}
\saveTG{𡀢}{62027}
\saveTG{𣈭}{62027}
\saveTG{𥊪}{62027}
\saveTG{𥆎}{62027}
\saveTG{𥄩}{62027}
\saveTG{𥈋}{62027}
\saveTG{𣌄}{62027}
\saveTG{𠾩}{62027}
\saveTG{𡆝}{62027}
\saveTG{𥋊}{62027}
\saveTG{畖}{62030}
\saveTG{呱}{62030}
\saveTG{𠮺}{62030}
\saveTG{𥄄}{62030}
\saveTG{𤱄}{62030}
\saveTG{𤔗}{62030}
\saveTG{𠲍}{62031}
\saveTG{𠱚}{62031}
\saveTG{𥃼}{62031}
\saveTG{嚑}{62031}
\saveTG{𥉍}{62031}
\saveTG{䂅}{62031}
\saveTG{𣊳}{62031}
\saveTG{𤰜}{62031}
\saveTG{𠽟}{62031}
\saveTG{𠰈}{62031}
\saveTG{曛}{62031}
\saveTG{吆}{62031}
\saveTG{矄}{62031}
\saveTG{𣋡}{62032}
\saveTG{𠶌}{62032}
\saveTG{𠻯}{62032}
\saveTG{𠸁}{62032}
\saveTG{𤱰}{62032}
\saveTG{𠵙}{62032}
\saveTG{哌}{62032}
\saveTG{眽}{62032}
\saveTG{眨}{62032}
\saveTG{𠺅}{62032}
\saveTG{𥇐}{62032}
\saveTG{吆}{62032}
\saveTG{𥄼}{62034}
\saveTG{𡁋}{62034}
\saveTG{嗤}{62036}
\saveTG{𠻉}{62036}
\saveTG{𡀷}{62036}
\saveTG{𤰝}{62037}
\saveTG{𡅯}{62037}
\saveTG{𣆅}{62037}
\saveTG{𠻅}{62037}
\saveTG{𠸾}{62037}
\saveTG{𠹰}{62037}
\saveTG{𠰏}{62037}
\saveTG{𤰕}{62037}
\saveTG{𡆕}{62039}
\saveTG{㗭}{62039}
\saveTG{呧}{62040}
\saveTG{䀒}{62040}
\saveTG{眡}{62040}
\saveTG{眂}{62040}
\saveTG{吀}{62040}
\saveTG{𣆴}{62041}
\saveTG{䀽}{62041}
\saveTG{𥆑}{62041}
\saveTG{𠵚}{62041}
\saveTG{𤲒}{62041}
\saveTG{哔}{62041}
\saveTG{唌}{62041}
\saveTG{䁓}{62042}
\saveTG{㽟}{62043}
\saveTG{㫝}{62043}
\saveTG{䂃}{62043}
\saveTG{㬭}{62043}
\saveTG{𠴓}{62043}
\saveTG{𥈒}{62044}
\saveTG{唩}{62044}
\saveTG{𠶽}{62044}
\saveTG{𠼈}{62044}
\saveTG{𡃲}{62044}
\saveTG{哸}{62044}
\saveTG{𪾶}{62045}
\saveTG{嚼}{62046}
\saveTG{𡄨}{62046}
\saveTG{𥈦}{62047}
\saveTG{𣉁}{62047}
\saveTG{暧}{62047}
\saveTG{嗳}{62047}
\saveTG{噯}{62047}
\saveTG{曖}{62047}
\saveTG{𠲺}{62047}
\saveTG{㖟}{62047}
\saveTG{瞹}{62047}
\saveTG{昄}{62047}
\saveTG{哹}{62047}
\saveTG{喛}{62047}
\saveTG{暖}{62047}
\saveTG{眅}{62047}
\saveTG{𥆬}{62047}
\saveTG{𣉀}{62047}
\saveTG{𠸷}{62047}
\saveTG{𡅇}{62047}
\saveTG{𠷑}{62047}
\saveTG{㗶}{62047}
\saveTG{𠷴}{62047}
\saveTG{䁔}{62047}
\saveTG{𤲫}{62047}
\saveTG{𡂅}{62047}
\saveTG{𠲐}{62047}
\saveTG{畈}{62047}
\saveTG{𥋀}{62047}
\saveTG{𥆹}{62047}
\saveTG{𥅢}{62047}
\saveTG{㫞}{62047}
\saveTG{𥍚}{62048}
\saveTG{𡆑}{62048}
\saveTG{哷}{62049}
\saveTG{呼}{62049}
\saveTG{暖}{62049}
\saveTG{瞬}{62052}
\saveTG{𠼐}{62052}
\saveTG{嘰}{62053}
\saveTG{睜}{62057}
\saveTG{𪰭}{62057}
\saveTG{𣊬}{62057}
\saveTG{𠲜}{62057}
\saveTG{𠼜}{62060}
\saveTG{𠽹}{62061}
\saveTG{𡂉}{62061}
\saveTG{𠻂}{62061}
\saveTG{㖃}{62061}
\saveTG{啱}{62062}
\saveTG{𥇷}{62062}
\saveTG{喈}{62062}
\saveTG{𥉭}{62062}
\saveTG{𥅠}{62062}
\saveTG{𡃞}{62062}
\saveTG{𠹗}{62062}
\saveTG{𠴲}{62063}
\saveTG{䀨}{62064}
\saveTG{𠸉}{62064}
\saveTG{𣇲}{62064}
\saveTG{㖧}{62064}
\saveTG{𠳂}{62064}
\saveTG{咶}{62064}
\saveTG{睧}{62064}
\saveTG{瞃}{62064}
\saveTG{𥆠}{62064}
\saveTG{𠸓}{62064}
\saveTG{𥈧}{62065}
\saveTG{𠸦}{62065}
\saveTG{𥋎}{62067}
\saveTG{㗍}{62069}
\saveTG{𡃓}{62069}
\saveTG{噃}{62069}
\saveTG{𥈺}{62069}
\saveTG{𣊩}{62069}
\saveTG{𠮿}{62070}
\saveTG{𠶑}{62070}
\saveTG{㕳}{62070}
\saveTG{㗀}{62070}
\saveTG{昢}{62072}
\saveTG{咄}{62072}
\saveTG{𤱟}{62072}
\saveTG{䁘}{62072}
\saveTG{𠶯}{62072}
\saveTG{𥇸}{62072}
\saveTG{𥅒}{62072}
\saveTG{暚}{62072}
\saveTG{嗂}{62072}
\saveTG{𠹓}{62074}
\saveTG{𥉰}{62077}
\saveTG{𥈊}{62077}
\saveTG{𠿂}{62077}
\saveTG{喢}{62077}
\saveTG{㗖}{62077}
\saveTG{𠺍}{62081}
\saveTG{𠴇}{62081}
\saveTG{𠽭}{62082}
\saveTG{𠾨}{62082}
\saveTG{㕭}{62084}
\saveTG{𠴎}{62084}
\saveTG{暌}{62084}
\saveTG{嗘}{62084}
\saveTG{𤲺}{62084}
\saveTG{𠴃}{62084}
\saveTG{𥉐}{62084}
\saveTG{𥍛}{62084}
\saveTG{睽}{62084}
\saveTG{𡃒}{62085}
\saveTG{𣊪}{62085}
\saveTG{𡃾}{62085}
\saveTG{𡂈}{62085}
\saveTG{𠾲}{62085}
\saveTG{瞨}{62085}
\saveTG{噗}{62085}
\saveTG{𠼠}{62086}
\saveTG{𡂒}{62086}
\saveTG{𠿹}{62086}
\saveTG{𥉚}{62088}
\saveTG{𡀕}{62089}
\saveTG{𥊠}{62091}
\saveTG{𤳁}{62091}
\saveTG{𠼾}{62091}
\saveTG{噝}{62093}
\saveTG{㗪}{62093}
\saveTG{𡀚}{62093}
\saveTG{𥌣}{62093}
\saveTG{喺}{62093}
\saveTG{𣋵}{62094}
\saveTG{𤱛}{62094}
\saveTG{𠷘}{62094}
\saveTG{𥈐}{62094}
\saveTG{䀥}{62094}
\saveTG{𪠸}{62094}
\saveTG{𣈄}{62094}
\saveTG{啋}{62094}
\saveTG{喍}{62094}
\saveTG{咊}{62094}
\saveTG{嚛}{62094}
\saveTG{睬}{62094}
\saveTG{𠱟}{62094}
\saveTG{𥊈}{62094}
\saveTG{𠼝}{62094}
\saveTG{䁻}{62094}
\saveTG{𥊌}{62094}
\saveTG{𠾋}{62095}
\saveTG{曗}{62095}
\saveTG{𠻥}{62095}
\saveTG{𡁖}{62095}
\saveTG{𥋙}{62095}
\saveTG{㗼}{62095}
\saveTG{𠶏}{62097}
\saveTG{𤲎}{62098}
\saveTG{𠠡}{62100}
\saveTG{𠛣}{62100}
\saveTG{𨄩}{62100}
\saveTG{𧾻}{62100}
\saveTG{𧿯}{62100}
\saveTG{㓻}{62100}
\saveTG{𠠯}{62100}
\saveTG{𨀺}{62100}
\saveTG{劅}{62100}
\saveTG{𥂄}{62102}
\saveTG{𡋾}{62104}
\saveTG{䟤}{62104}
\saveTG{𡋏}{62104}
\saveTG{𪽥}{62108}
\saveTG{跐}{62110}
\saveTG{䟬}{62112}
\saveTG{𨂇}{62112}
\saveTG{𫏘}{62112}
\saveTG{𨆂}{62112}
\saveTG{躐}{62112}
\saveTG{𨇏}{62112}
\saveTG{䠪}{62113}
\saveTG{跳}{62113}
\saveTG{毾}{62114}
\saveTG{𫏊}{62114}
\saveTG{𨂘}{62114}
\saveTG{氎}{62114}
\saveTG{𨄍}{62115}
\saveTG{踵}{62115}
\saveTG{𧾹}{62117}
\saveTG{𨃑}{62117}
\saveTG{𨁓}{62117}
\saveTG{𨀈}{62117}
\saveTG{𧿥}{62117}
\saveTG{𫏦}{62117}
\saveTG{𣭻}{62117}
\saveTG{㲲}{62117}
\saveTG{𣯎}{62117}
\saveTG{𧿌}{62117}
\saveTG{𨅏}{62117}
\saveTG{蹏}{62117}
\saveTG{𫏓}{62117}
\saveTG{𨀭}{62117}
\saveTG{蹬}{62118}
\saveTG{𧿧}{62121}
\saveTG{𫏞}{62121}
\saveTG{𨇰}{62121}
\saveTG{䟷}{62121}
\saveTG{𢒡}{62122}
\saveTG{𧐸}{62122}
\saveTG{𨃳}{62127}
\saveTG{𫏎}{62127}
\saveTG{𨇋}{62127}
\saveTG{𨁊}{62127}
\saveTG{踹}{62127}
\saveTG{𨅌}{62127}
\saveTG{𨈎}{62127}
\saveTG{𨀨}{62127}
\saveTG{𨄪}{62127}
\saveTG{蹦}{62127}
\saveTG{踽}{62127}
\saveTG{蹻}{62127}
\saveTG{蹐}{62127}
\saveTG{𨇊}{62127}
\saveTG{𨇕}{62127}
\saveTG{𨅳}{62127}
\saveTG{𨁧}{62127}
\saveTG{䠌}{62127}
\saveTG{𫏋}{62128}
\saveTG{爴}{62130}
\saveTG{𨃻}{62130}
\saveTG{𨀰}{62130}
\saveTG{𧖎}{62131}
\saveTG{𧔆}{62131}
\saveTG{𧖈}{62131}
\saveTG{𨂼}{62132}
\saveTG{𨂗}{62133}
\saveTG{𨃆}{62133}
\saveTG{㼔}{62133}
\saveTG{𫏃}{62134}
\saveTG{跹}{62134}
\saveTG{𨆲}{62134}
\saveTG{𨆛}{62136}
\saveTG{𧔤}{62136}
\saveTG{䟪}{62137}
\saveTG{𨄠}{62139}
\saveTG{𧿐}{62140}
\saveTG{跅}{62141}
\saveTG{𨁗}{62141}
\saveTG{𨁆}{62141}
\saveTG{跸}{62141}
\saveTG{䟡}{62143}
\saveTG{䟹}{62143}
\saveTG{𨇺}{62143}
\saveTG{𨁡}{62144}
\saveTG{𨃦}{62144}
\saveTG{𫏙}{62144}
\saveTG{踒}{62144}
\saveTG{䟥}{62147}
\saveTG{𫏖}{62147}
\saveTG{蹳}{62147}
\saveTG{䟗}{62147}
\saveTG{䟦}{62147}
\saveTG{𧿨}{62147}
\saveTG{𨂩}{62147}
\saveTG{踲}{62164}
\saveTG{踏}{62169}
\saveTG{𨂻}{62169}
\saveTG{蹯}{62169}
\saveTG{躖}{62170}
\saveTG{𨇔}{62170}
\saveTG{𧿖}{62170}
\saveTG{䟖}{62172}
\saveTG{𧿺}{62172}
\saveTG{䠛}{62174}
\saveTG{蹈}{62177}
\saveTG{𨈊}{62179}
\saveTG{踬}{62182}
\saveTG{𪴯}{62182}
\saveTG{蹊}{62184}
\saveTG{䠏}{62184}
\saveTG{䟯}{62184}
\saveTG{跃}{62184}
\saveTG{𨁸}{62184}
\saveTG{𨃵}{62184}
\saveTG{蹼}{62185}
\saveTG{𨆯}{62185}
\saveTG{躓}{62186}
\saveTG{𨅃}{62191}
\saveTG{𨄐}{62193}
\saveTG{𨇼}{62193}
\saveTG{躒}{62194}
\saveTG{跞}{62194}
\saveTG{䠕}{62194}
\saveTG{𥣷}{62194}
\saveTG{𨃮}{62194}
\saveTG{踩}{62194}
\saveTG{𨄓}{62195}
\saveTG{𨇉}{62196}
\saveTG{𠞋}{62200}
\saveTG{𠰤}{62200}
\saveTG{𠞍}{62200}
\saveTG{𠜏}{62200}
\saveTG{剐}{62200}
\saveTG{𠞫}{62200}
\saveTG{剈}{62200}
\saveTG{剔}{62200}
\saveTG{別}{62200}
\saveTG{𠟔}{62200}
\saveTG{𠝹}{62200}
\saveTG{𠞵}{62200}
\saveTG{𧠑}{62211}
\saveTG{𧠘}{62213}
\saveTG{氍}{62214}
\saveTG{𧡒}{62217}
\saveTG{𣭖}{62217}
\saveTG{𣉷}{62217}
\saveTG{㓵}{62220}
\saveTG{𣂿}{62221}
\saveTG{𣃊}{62221}
\saveTG{𣃖}{62221}
\saveTG{𣂨}{62221}
\saveTG{𢒠}{62222}
\saveTG{𢒗}{62222}
\saveTG{哛}{62227}
\saveTG{㖰}{62227}
\saveTG{㖎}{62228}
\saveTG{𢒧}{62232}
\saveTG{𥀦}{62247}
\saveTG{𡕺}{62247}
\saveTG{𡥺}{62247}
\saveTG{𥆞}{62260}
\saveTG{𤐁}{62289}
\saveTG{𥟙}{62294}
\saveTG{𥟟}{62294}
\saveTG{𠠁}{62300}
\saveTG{䵯}{62315}
\saveTG{𪐞}{62317}
\saveTG{𣮊}{62317}
\saveTG{𪒐}{62317}
\saveTG{𪒘}{62318}
\saveTG{𪐹}{62327}
\saveTG{𡃸}{62327}
\saveTG{𪅚}{62327}
\saveTG{𪈉}{62327}
\saveTG{𪑴}{62327}
\saveTG{䵝}{62327}
\saveTG{𢝔}{62330}
\saveTG{𢞈}{62330}
\saveTG{𥉬}{62330}
\saveTG{𢥽}{62331}
\saveTG{𤋇}{62334}
\saveTG{𡬷}{62342}
\saveTG{𪑋}{62343}
\saveTG{𪒱}{62347}
\saveTG{䵬}{62363}
\saveTG{𪑕}{62364}
\saveTG{黜}{62372}
\saveTG{𪒢}{62385}
\saveTG{𪑽}{62394}
\saveTG{𪒲}{62395}
\saveTG{别}{62400}
\saveTG{𠞅}{62400}
\saveTG{𠟎}{62400}
\saveTG{𤕔}{62408}
\saveTG{𣭸}{62417}
\saveTG{𪫉}{62422}
\saveTG{𣉡}{62431}
\saveTG{𦋪}{62441}
\saveTG{𣱍}{62447}
\saveTG{𦋦}{62472}
\saveTG{𠔦}{62481}
\saveTG{𤚧}{62502}
\saveTG{𣃓}{62521}
\saveTG{𣢗}{62582}
\saveTG{𠟑}{62600}
\saveTG{𠟟}{62600}
\saveTG{𠳇}{62602}
\saveTG{𠷒}{62603}
\saveTG{毷}{62614}
\saveTG{㫀}{62621}
\saveTG{𤬆}{62633}
\saveTG{𠭤}{62647}
\saveTG{𥌭}{62693}
\saveTG{㓭}{62700}
\saveTG{𩙽}{62713}
\saveTG{毼}{62714}
\saveTG{𥇾}{62715}
\saveTG{𣂰}{62721}
\saveTG{𧞈}{62732}
\saveTG{𩞣}{62732}
\saveTG{𣡯}{62758}
\saveTG{𪙶}{62771}
\saveTG{𧿀}{62800}
\saveTG{𠜹}{62800}
\saveTG{㓳}{62800}
\saveTG{𠟻}{62800}
\saveTG{則}{62800}
\saveTG{匙}{62801}
\saveTG{𡚇}{62804}
\saveTG{𡺔}{62807}
\saveTG{㲘}{62817}
\saveTG{𧵮}{62817}
\saveTG{𧵆}{62821}
\saveTG{𧶇}{62821}
\saveTG{𫛸}{62827}
\saveTG{𧶲}{62827}
\saveTG{𧸕}{62827}
\saveTG{𧸅}{62827}
\saveTG{貶}{62832}
\saveTG{𧵬}{62832}
\saveTG{𧶫}{62832}
\saveTG{𧵟}{62833}
\saveTG{貾}{62840}
\saveTG{𧸻}{62847}
\saveTG{販}{62847}
\saveTG{𧵄}{62847}
\saveTG{𧵃}{62850}
\saveTG{𪢛}{62857}
\saveTG{𧸥}{62868}
\saveTG{𧷊}{62872}
\saveTG{贌}{62885}
\saveTG{𧸲}{62886}
\saveTG{𧷣}{62895}
\saveTG{𧸢}{62895}
\saveTG{㔀}{62900}
\saveTG{𠟘}{62900}
\saveTG{𠜴}{62900}
\saveTG{劋}{62900}
\saveTG{𣚧}{62904}
\saveTG{氉}{62914}
\saveTG{𦔉}{62915}
\saveTG{𢀍}{62917}
\saveTG{𣮰}{62917}
\saveTG{𣮔}{62917}
\saveTG{𣯉}{62917}
\saveTG{𢒙}{62922}
\saveTG{影}{62922}
\saveTG{𤔖}{62930}
\saveTG{𤬁}{62933}
\saveTG{𢒳}{62947}
\saveTG{𣚤}{62947}
\saveTG{𣊉}{62958}
\saveTG{𥋮}{62961}
\saveTG{𠚢}{62972}
\saveTG{𡆞}{62992}
\saveTG{𣅃}{63000}
\saveTG{𣅵}{63000}
\saveTG{卟}{63000}
\saveTG{啩}{63000}
\saveTG{吣}{63000}
\saveTG{𤰘}{63000}
\saveTG{𥃨}{63000}
\saveTG{𠴤}{63001}
\saveTG{咇}{63004}
\saveTG{昈}{63007}
\saveTG{𠴧}{63011}
\saveTG{𠽱}{63012}
\saveTG{吮}{63012}
\saveTG{睕}{63012}
\saveTG{晼}{63012}
\saveTG{唍}{63012}
\saveTG{畹}{63012}
\saveTG{啘}{63012}
\saveTG{𥋱}{63012}
\saveTG{𥊱}{63012}
\saveTG{𣈞}{63012}
\saveTG{𪾣}{63012}
\saveTG{𥉓}{63012}
\saveTG{啌}{63012}
\saveTG{睆}{63012}
\saveTG{晥}{63012}
\saveTG{噈}{63012}
\saveTG{咜}{63012}
\saveTG{咤}{63014}
\saveTG{昽}{63014}
\saveTG{咙}{63014}
\saveTG{㗧}{63014}
\saveTG{噻}{63014}
\saveTG{𠾰}{63014}
\saveTG{㗌}{63014}
\saveTG{𥉺}{63014}
\saveTG{哤}{63014}
\saveTG{眬}{63014}
\saveTG{𪱑}{63014}
\saveTG{睻}{63016}
\saveTG{暄}{63016}
\saveTG{喧}{63016}
\saveTG{𣅸}{63017}
\saveTG{𠰽}{63017}
\saveTG{㕱}{63017}
\saveTG{𡂷}{63017}
\saveTG{𠻩}{63017}
\saveTG{𠰲}{63017}
\saveTG{𪾡}{63017}
\saveTG{㽙}{63017}
\saveTG{呝}{63017}
\saveTG{𤲏}{63017}
\saveTG{𪡱}{63017}
\saveTG{䵨}{63017}
\saveTG{𥄻}{63017}
\saveTG{𣇈}{63017}
\saveTG{眝}{63021}
\saveTG{嚀}{63021}
\saveTG{咛}{63021}
\saveTG{矃}{63021}
\saveTG{𠳽}{63021}
\saveTG{𥊘}{63021}
\saveTG{𥊀}{63022}
\saveTG{嘇}{63022}
\saveTG{㽩}{63022}
\saveTG{𠿄}{63027}
\saveTG{𣇬}{63027}
\saveTG{𪢜}{63027}
\saveTG{𥉡}{63027}
\saveTG{𡅶}{63027}
\saveTG{𪿀}{63027}
\saveTG{䀯}{63027}
\saveTG{𡃕}{63027}
\saveTG{𠱝}{63027}
\saveTG{𠴛}{63027}
\saveTG{𡀨}{63027}
\saveTG{𡃂}{63027}
\saveTG{晡}{63027}
\saveTG{哺}{63027}
\saveTG{𠶥}{63027}
\saveTG{𡁮}{63027}
\saveTG{𠰻}{63030}
\saveTG{𠼢}{63031}
\saveTG{𪰧}{63031}
\saveTG{𠺨}{63031}
\saveTG{𠹁}{63032}
\saveTG{𠺘}{63032}
\saveTG{㫰}{63032}
\saveTG{𠺢}{63032}
\saveTG{䀶}{63032}
\saveTG{哴}{63032}
\saveTG{𡂔}{63035}
\saveTG{𡄏}{63035}
\saveTG{𠷂}{63036}
\saveTG{𠷖}{63037}
\saveTG{嘫}{63038}
\saveTG{𤳒}{63038}
\saveTG{㬗}{63038}
\saveTG{𥊶}{63038}
\saveTG{𣉐}{63039}
\saveTG{𠰺}{63040}
\saveTG{𥅞}{63040}
\saveTG{𪰜}{63040}
\saveTG{𠯅}{63040}
\saveTG{𠱌}{63041}
\saveTG{𥄜}{63041}
\saveTG{𪾩}{63041}
\saveTG{𠽢}{63041}
\saveTG{𥈹}{63041}
\saveTG{𠹼}{63041}
\saveTG{𠲧}{63041}
\saveTG{𠱶}{63041}
\saveTG{嚩}{63042}
\saveTG{𪡟}{63042}
\saveTG{𠱔}{63042}
\saveTG{䀣}{63043}
\saveTG{㬍}{63043}
\saveTG{嚩}{63043}
\saveTG{㗘}{63043}
\saveTG{咹}{63044}
\saveTG{𠻈}{63044}
\saveTG{㫨}{63044}
\saveTG{𥅥}{63044}
\saveTG{㕹}{63047}
\saveTG{唆}{63047}
\saveTG{𠻕}{63047}
\saveTG{𠷃}{63047}
\saveTG{𥈟}{63047}
\saveTG{𥊁}{63047}
\saveTG{𠿆}{63047}
\saveTG{晙}{63047}
\saveTG{畯}{63047}
\saveTG{睃}{63047}
\saveTG{𥈃}{63047}
\saveTG{𠽬}{63048}
\saveTG{𠸂}{63048}
\saveTG{喊}{63050}
\saveTG{嚱}{63050}
\saveTG{晠}{63050}
\saveTG{嘁}{63050}
\saveTG{哞}{63050}
\saveTG{𪾤}{63050}
\saveTG{眸}{63050}
\saveTG{睋}{63050}
\saveTG{哦}{63050}
\saveTG{𢧷}{63050}
\saveTG{𠸹}{63050}
\saveTG{𠲎}{63050}
\saveTG{𣆈}{63050}
\saveTG{𥅩}{63050}
\saveTG{眓}{63050}
\saveTG{喴}{63050}
\saveTG{睵}{63050}
\saveTG{𡃹}{63050}
\saveTG{𠿠}{63050}
\saveTG{𠽤}{63050}
\saveTG{𠺵}{63050}
\saveTG{𠯫}{63050}
\saveTG{𣆒}{63050}
\saveTG{𥇿}{63051}
\saveTG{㖪}{63051}
\saveTG{㽣}{63051}
\saveTG{𠿑}{63051}
\saveTG{㖅}{63051}
\saveTG{𥅜}{63051}
\saveTG{𡄑}{63051}
\saveTG{𡃰}{63051}
\saveTG{㘍}{63051}
\saveTG{𡁶}{63051}
\saveTG{𡁧}{63051}
\saveTG{𥍀}{63051}
\saveTG{𪡂}{63052}
\saveTG{𠻏}{63052}
\saveTG{哰}{63052}
\saveTG{𥆏}{63052}
\saveTG{𠲌}{63053}
\saveTG{𠽈}{63053}
\saveTG{𠵖}{63053}
\saveTG{𠲦}{63054}
\saveTG{㖑}{63054}
\saveTG{𥅯}{63054}
\saveTG{𥊬}{63054}
\saveTG{𨎵}{63054}
\saveTG{𣇕}{63055}
\saveTG{𠹷}{63055}
\saveTG{𥋏}{63056}
\saveTG{䁍}{63056}
\saveTG{𠽦}{63056}
\saveTG{𣈻}{63056}
\saveTG{𠷧}{63058}
\saveTG{𠸴}{63058}
\saveTG{𥉷}{63059}
\saveTG{𠷼}{63059}
\saveTG{眙}{63060}
\saveTG{咍}{63060}
\saveTG{𠾘}{63061}
\saveTG{𡃊}{63061}
\saveTG{𠷪}{63061}
\saveTG{𥊉}{63061}
\saveTG{噾}{63061}
\saveTG{𣅿}{63061}
\saveTG{𠹒}{63062}
\saveTG{𠶠}{63063}
\saveTG{喒}{63064}
\saveTG{喀}{63064}
\saveTG{𥇻}{63065}
\saveTG{瞎}{63065}
\saveTG{嗐}{63065}
\saveTG{𠾙}{63066}
\saveTG{𠹍}{63068}
\saveTG{𥈾}{63068}
\saveTG{𠺀}{63068}
\saveTG{𡂹}{63069}
\saveTG{𥉴}{63071}
\saveTG{嘧}{63072}
\saveTG{𤲁}{63072}
\saveTG{𡀙}{63072}
\saveTG{𠴨}{63077}
\saveTG{𥋺}{63077}
\saveTG{𡄓}{63081}
\saveTG{啶}{63081}
\saveTG{𠱇}{63082}
\saveTG{𥇓}{63082}
\saveTG{𥄴}{63082}
\saveTG{𠽰}{63082}
\saveTG{㗷}{63082}
\saveTG{𪾸}{63083}
\saveTG{𥈳}{63084}
\saveTG{䁭}{63084}
\saveTG{𥋰}{63084}
\saveTG{𣅤}{63084}
\saveTG{𣈚}{63084}
\saveTG{䀵}{63084}
\saveTG{唉}{63084}
\saveTG{囐}{63084}
\saveTG{吠}{63084}
\saveTG{唳}{63084}
\saveTG{睙}{63084}
\saveTG{畎}{63084}
\saveTG{𥇺}{63084}
\saveTG{㘔}{63086}
\saveTG{𣋪}{63086}
\saveTG{𡁟}{63086}
\saveTG{𪱄}{63086}
\saveTG{𡁃}{63086}
\saveTG{矉}{63086}
\saveTG{瞚}{63086}
\saveTG{𠻤}{63086}
\saveTG{𣇼}{63091}
\saveTG{嚓}{63091}
\saveTG{𥌀}{63091}
\saveTG{𠵻}{63091}
\saveTG{咏}{63092}
\saveTG{昹}{63092}
\saveTG{眿}{63092}
\saveTG{𥊻}{63093}
\saveTG{𥄵}{63094}
\saveTG{𠳼}{63094}
\saveTG{𥌢}{63096}
\saveTG{曢}{63096}
\saveTG{𤳑}{63098}
\saveTG{𥆿}{63099}
\saveTG{䟔}{63100}
\saveTG{盢}{63102}
\saveTG{䠁}{63111}
\saveTG{蹴}{63112}
\saveTG{踠}{63112}
\saveTG{跎}{63112}
\saveTG{𨄺}{63115}
\saveTG{𨇳}{63117}
\saveTG{𨁚}{63117}
\saveTG{𨀁}{63117}
\saveTG{𨀸}{63117}
\saveTG{躥}{63117}
\saveTG{𨀉}{63121}
\saveTG{𤴍}{63121}
\saveTG{𨄾}{63121}
\saveTG{𨁏}{63127}
\saveTG{𡄽}{63127}
\saveTG{𨃩}{63127}
\saveTG{𨀮}{63127}
\saveTG{蹁}{63127}
\saveTG{𤔝}{63130}
\saveTG{䟩}{63131}
\saveTG{踉}{63132}
\saveTG{𨃹}{63132}
\saveTG{䠚}{63133}
\saveTG{𧿼}{63133}
\saveTG{蠈}{63136}
\saveTG{𨅶}{63138}
\saveTG{蹨}{63138}
\saveTG{𧿑}{63140}
\saveTG{䟼}{63141}
\saveTG{𨇶}{63147}
\saveTG{𫏆}{63147}
\saveTG{踆}{63147}
\saveTG{跋}{63147}
\saveTG{𨄒}{63147}
\saveTG{𤢜}{63148}
\saveTG{𨀳}{63150}
\saveTG{𢧄}{63150}
\saveTG{戥}{63150}
\saveTG{𪭓}{63150}
\saveTG{𢧫}{63150}
\saveTG{𨁟}{63150}
\saveTG{践}{63150}
\saveTG{𨇦}{63151}
\saveTG{踐}{63153}
\saveTG{𨀻}{63154}
\saveTG{𨆎}{63154}
\saveTG{𨃂}{63156}
\saveTG{蹿}{63156}
\saveTG{䟠}{63157}
\saveTG{𨈄}{63157}
\saveTG{䠞}{63159}
\saveTG{𨃭}{63159}
\saveTG{跆}{63160}
\saveTG{𨅻}{63161}
\saveTG{蹜}{63162}
\saveTG{𨂥}{63164}
\saveTG{䠉}{63177}
\saveTG{𨂌}{63182}
\saveTG{𨇥}{63182}
\saveTG{𧿡}{63184}
\saveTG{𨃍}{63184}
\saveTG{𤡴}{63184}
\saveTG{𤟩}{63184}
\saveTG{䟮}{63184}
\saveTG{𨅿}{63185}
\saveTG{𨬓}{63185}
\saveTG{𨄻}{63186}
\saveTG{踪}{63191}
\saveTG{𨆾}{63191}
\saveTG{𨄈}{63194}
\saveTG{䟣}{63194}
\saveTG{𡑅}{63194}
\saveTG{䟵}{63199}
\saveTG{𡒦}{63214}
\saveTG{𥍄}{63216}
\saveTG{𧥀}{63227}
\saveTG{𠶇}{63232}
\saveTG{𢎖}{63240}
\saveTG{𢨐}{63250}
\saveTG{戵}{63250}
\saveTG{㦹}{63250}
\saveTG{𣉝}{63254}
\saveTG{猒}{63284}
\saveTG{𤞣}{63284}
\saveTG{𪐙}{63300}
\saveTG{黦}{63312}
\saveTG{𪐤}{63317}
\saveTG{黪}{63322}
\saveTG{黲}{63322}
\saveTG{𪒞}{63331}
\saveTG{𥌷}{63336}
\saveTG{𢣉}{63337}
\saveTG{黙}{63338}
\saveTG{㦔}{63338}
\saveTG{黓}{63340}
\saveTG{黢}{63347}
\saveTG{黬}{63350}
\saveTG{𪒃}{63351}
\saveTG{𪑝}{63351}
\saveTG{𪑆}{63353}
\saveTG{𪐶}{63357}
\saveTG{𪑥}{63368}
\saveTG{默}{63384}
\saveTG{𪒥}{63385}
\saveTG{𠨃}{63400}
\saveTG{𠴆}{63413}
\saveTG{𥉛}{63446}
\saveTG{𢧀}{63450}
\saveTG{戢}{63450}
\saveTG{𤢕}{63484}
\saveTG{𠿽}{63527}
\saveTG{辴}{63532}
\saveTG{𤱢}{63540}
\saveTG{𡁳}{63547}
\saveTG{戰}{63550}
\saveTG{𩉎}{63602}
\saveTG{𩉂}{63602}
\saveTG{𠷳}{63604}
\saveTG{䁿}{63608}
\saveTG{𥊷}{63608}
\saveTG{𨢽}{63647}
\saveTG{𡃣}{63650}
\saveTG{𠺳}{63667}
\saveTG{獸}{63684}
\saveTG{𩞹}{63732}
\saveTG{𧴤}{63800}
\saveTG{𪰿}{63804}
\saveTG{䝹}{63817}
\saveTG{貯}{63821}
\saveTG{𧸎}{63821}
\saveTG{贂}{63822}
\saveTG{𧷘}{63827}
\saveTG{𧸺}{63827}
\saveTG{䝵}{63827}
\saveTG{𧸿}{63827}
\saveTG{賦}{63840}
\saveTG{𧴰}{63840}
\saveTG{𧵴}{63841}
\saveTG{𧸔}{63841}
\saveTG{賻}{63842}
\saveTG{𧵤}{63843}
\saveTG{𧵨}{63844}
\saveTG{賐}{63847}
\saveTG{𧸐}{63848}
\saveTG{贓}{63850}
\saveTG{賳}{63850}
\saveTG{賎}{63850}
\saveTG{戝}{63850}
\saveTG{𧶒}{63850}
\saveTG{賊}{63850}
\saveTG{𧵶}{63851}
\saveTG{𧷢}{63851}
\saveTG{𧵪}{63852}
\saveTG{𧶍}{63853}
\saveTG{賤}{63853}
\saveTG{𧸄}{63853}
\saveTG{𧶂}{63853}
\saveTG{𧶕}{63855}
\saveTG{賳}{63856}
\saveTG{𧵝}{63857}
\saveTG{𧷲}{63857}
\saveTG{𧸉}{63858}
\saveTG{貽}{63860}
\saveTG{𡤔}{63861}
\saveTG{𧷦}{63872}
\saveTG{贆}{63884}
\saveTG{賩}{63891}
\saveTG{𧸴}{63896}
\saveTG{賕}{63899}
\saveTG{𠾮}{63941}
\saveTG{𣇚}{63947}
\saveTG{𠽨}{63961}
\saveTG{𪡣}{63965}
\saveTG{呏}{64000}
\saveTG{时}{64000}
\saveTG{唞}{64000}
\saveTG{旪}{64000}
\saveTG{嚉}{64000}
\saveTG{叶}{64000}
\saveTG{嘝}{64000}
\saveTG{呌}{64000}
\saveTG{𪡮}{64000}
\saveTG{𠯰}{64000}
\saveTG{吋}{64000}
\saveTG{咐}{64000}
\saveTG{𠸥}{64000}
\saveTG{𠮹}{64002}
\saveTG{䀞}{64003}
\saveTG{𠷵}{64003}
\saveTG{嘝}{64003}
\saveTG{𠺫}{64003}
\saveTG{𡀃}{64003}
\saveTG{𠲝}{64003}
\saveTG{㬣}{64003}
\saveTG{𣅮}{64004}
\saveTG{𠯢}{64004}
\saveTG{吐}{64010}
\saveTG{𠴗}{64010}
\saveTG{吪}{64010}
\saveTG{𥃾}{64010}
\saveTG{𠴏}{64010}
\saveTG{𠱊}{64010}
\saveTG{𠶈}{64010}
\saveTG{𥇀}{64010}
\saveTG{叱}{64010}
\saveTG{𠷆}{64011}
\saveTG{暁}{64012}
\saveTG{嘵}{64012}
\saveTG{咃}{64012}
\saveTG{咾}{64012}
\saveTG{嗑}{64012}
\saveTG{哋}{64012}
\saveTG{眈}{64012}
\saveTG{𠷦}{64012}
\saveTG{𡀄}{64012}
\saveTG{𠶗}{64012}
\saveTG{㗐}{64012}
\saveTG{𡀽}{64012}
\saveTG{㘕}{64012}
\saveTG{𣌐}{64012}
\saveTG{𥃸}{64012}
\saveTG{瞌}{64012}
\saveTG{𣋞}{64012}
\saveTG{咗}{64012}
\saveTG{吔}{64012}
\saveTG{曉}{64012}
\saveTG{𠳭}{64012}
\saveTG{𥍍}{64012}
\saveTG{𠼻}{64014}
\saveTG{𠵕}{64014}
\saveTG{哇}{64014}
\saveTG{𥉯}{64014}
\saveTG{嚡}{64014}
\saveTG{眭}{64014}
\saveTG{睳}{64014}
\saveTG{喹}{64014}
\saveTG{晆}{64014}
\saveTG{睦}{64014}
\saveTG{畦}{64014}
\saveTG{𣈹}{64014}
\saveTG{𡃏}{64014}
\saveTG{𠱽}{64014}
\saveTG{𠺺}{64014}
\saveTG{㘆}{64014}
\saveTG{𥈻}{64014}
\saveTG{𪽘}{64014}
\saveTG{𥋾}{64015}
\saveTG{𠻨}{64015}
\saveTG{𡄾}{64015}
\saveTG{矔}{64015}
\saveTG{嚾}{64015}
\saveTG{𡀇}{64015}
\saveTG{𥉑}{64015}
\saveTG{𣉒}{64015}
\saveTG{𡃜}{64015}
\saveTG{𠻓}{64015}
\saveTG{𡄳}{64015}
\saveTG{晻}{64016}
\saveTG{喳}{64016}
\saveTG{唵}{64016}
\saveTG{𠾵}{64016}
\saveTG{𣉎}{64016}
\saveTG{𣋢}{64017}
\saveTG{𣊀}{64017}
\saveTG{𥍘}{64017}
\saveTG{𣉪}{64017}
\saveTG{䁱}{64017}
\saveTG{𥉂}{64017}
\saveTG{𥆈}{64017}
\saveTG{𥍇}{64017}
\saveTG{吪}{64017}
\saveTG{𥉱}{64017}
\saveTG{𥅂}{64017}
\saveTG{𠵅}{64017}
\saveTG{䁆}{64017}
\saveTG{𠺄}{64017}
\saveTG{㽢}{64017}
\saveTG{𥈩}{64017}
\saveTG{𡀥}{64017}
\saveTG{𠳗}{64017}
\saveTG{𠴋}{64017}
\saveTG{㕤}{64017}
\saveTG{𪽇}{64017}
\saveTG{𪡝}{64017}
\saveTG{𤳙}{64017}
\saveTG{呓}{64017}
\saveTG{呭}{64017}
\saveTG{𡅍}{64017}
\saveTG{𡆜}{64017}
\saveTG{𠮟}{64017}
\saveTG{㗾}{64017}
\saveTG{𥄒}{64017}
\saveTG{𥅋}{64017}
\saveTG{𠯾}{64017}
\saveTG{𪠻}{64017}
\saveTG{㕪}{64017}
\saveTG{𣉊}{64017}
\saveTG{𡆁}{64017}
\saveTG{𠶍}{64017}
\saveTG{𠷸}{64018}
\saveTG{𪾼}{64018}
\saveTG{啿}{64018}
\saveTG{噎}{64018}
\saveTG{𪠶}{64018}
\saveTG{𠾻}{64018}
\saveTG{𠻼}{64018}
\saveTG{𡃻}{64018}
\saveTG{曀}{64018}
\saveTG{嗬}{64021}
\saveTG{𠵇}{64021}
\saveTG{𥇚}{64021}
\saveTG{𠶾}{64021}
\saveTG{畸}{64021}
\saveTG{𡂕}{64022}
\saveTG{𠯦}{64022}
\saveTG{𠴳}{64022}
\saveTG{𡁏}{64023}
\saveTG{𠸚}{64026}
\saveTG{㗣}{64027}
\saveTG{𠱼}{64027}
\saveTG{𡀠}{64027}
\saveTG{𡀵}{64027}
\saveTG{𪾹}{64027}
\saveTG{𣎏}{64027}
\saveTG{䀷}{64027}
\saveTG{𥈶}{64027}
\saveTG{𥅇}{64027}
\saveTG{𥇟}{64027}
\saveTG{𣈰}{64027}
\saveTG{𣇷}{64027}
\saveTG{㫑}{64027}
\saveTG{𣅺}{64027}
\saveTG{𠰱}{64027}
\saveTG{𡁣}{64027}
\saveTG{𡀣}{64027}
\saveTG{𡅳}{64027}
\saveTG{𠸔}{64027}
\saveTG{𠲵}{64027}
\saveTG{𠷺}{64027}
\saveTG{𪢠}{64027}
\saveTG{㗢}{64027}
\saveTG{𠼮}{64027}
\saveTG{𠯠}{64027}
\saveTG{𠽆}{64027}
\saveTG{𤱷}{64027}
\saveTG{㖴}{64027}
\saveTG{𠿸}{64027}
\saveTG{𠰖}{64027}
\saveTG{𥅚}{64027}
\saveTG{𥈞}{64027}
\saveTG{𤱎}{64027}
\saveTG{𤱅}{64027}
\saveTG{㽖}{64027}
\saveTG{𠡛}{64027}
\saveTG{𡁦}{64027}
\saveTG{𠻋}{64027}
\saveTG{哊}{64027}
\saveTG{呦}{64027}
\saveTG{噧}{64027}
\saveTG{嗋}{64027}
\saveTG{睎}{64027}
\saveTG{晞}{64027}
\saveTG{唏}{64027}
\saveTG{暔}{64027}
\saveTG{喃}{64027}
\saveTG{吶}{64027}
\saveTG{暪}{64027}
\saveTG{瞞}{64027}
\saveTG{瞒}{64027}
\saveTG{呐}{64027}
\saveTG{嘞}{64027}
\saveTG{叻}{64027}
\saveTG{囒}{64027}
\saveTG{晇}{64027}
\saveTG{咵}{64027}
\saveTG{嚆}{64027}
\saveTG{噶}{64027}
\saveTG{唠}{64027}
\saveTG{咘}{64027}
\saveTG{眑}{64027}
\saveTG{𠾞}{64027}
\saveTG{𠱿}{64027}
\saveTG{𡁐}{64028}
\saveTG{𠶋}{64028}
\saveTG{𠾢}{64030}
\saveTG{呔}{64030}
\saveTG{𤰥}{64030}
\saveTG{𠷻}{64030}
\saveTG{𠴘}{64030}
\saveTG{嚇}{64031}
\saveTG{哧}{64031}
\saveTG{𠳷}{64031}
\saveTG{㖁}{64031}
\saveTG{𠵐}{64031}
\saveTG{𠵽}{64031}
\saveTG{𡃺}{64031}
\saveTG{𥍂}{64031}
\saveTG{𥌩}{64031}
\saveTG{𣇐}{64031}
\saveTG{㫢}{64031}
\saveTG{嚥}{64031}
\saveTG{𠺆}{64031}
\saveTG{𠹤}{64031}
\saveTG{𣌕}{64031}
\saveTG{𪢧}{64031}
\saveTG{𥊸}{64031}
\saveTG{𡆖}{64031}
\saveTG{曣}{64031}
\saveTG{囈}{64031}
\saveTG{㬨}{64031}
\saveTG{𥋿}{64031}
\saveTG{𣇌}{64031}
\saveTG{曚}{64032}
\saveTG{𪡫}{64032}
\saveTG{𠶆}{64032}
\saveTG{𠺂}{64032}
\saveTG{矇}{64032}
\saveTG{呿}{64032}
\saveTG{吰}{64032}
\saveTG{𠶫}{64032}
\saveTG{𠸼}{64032}
\saveTG{𡃴}{64032}
\saveTG{𠸖}{64032}
\saveTG{𣉏}{64032}
\saveTG{哝}{64032}
\saveTG{𤴃}{64034}
\saveTG{㘃}{64034}
\saveTG{噠}{64035}
\saveTG{𡂴}{64035}
\saveTG{𡃿}{64035}
\saveTG{𡂫}{64035}
\saveTG{𪡏}{64035}
\saveTG{囆}{64036}
\saveTG{𡃨}{64036}
\saveTG{𠻛}{64036}
\saveTG{𪢆}{64036}
\saveTG{𡂁}{64037}
\saveTG{哒}{64038}
\saveTG{𤲻}{64038}
\saveTG{𠹎}{64039}
\saveTG{𣅓}{64040}
\saveTG{𪰤}{64040}
\saveTG{𠲤}{64040}
\saveTG{𠯆}{64040}
\saveTG{𡜭}{64040}
\saveTG{哎}{64040}
\saveTG{𠽿}{64041}
\saveTG{時}{64041}
\saveTG{晔}{64041}
\saveTG{𠯥}{64041}
\saveTG{𣈑}{64041}
\saveTG{𠵃}{64041}
\saveTG{哗}{64041}
\saveTG{𠴈}{64041}
\saveTG{𥄌}{64041}
\saveTG{𠱾}{64041}
\saveTG{啈}{64041}
\saveTG{畤}{64041}
\saveTG{疇}{64041}
\saveTG{嚋}{64041}
\saveTG{𠺊}{64042}
\saveTG{𠵝}{64042}
\saveTG{𠰵}{64042}
\saveTG{𠻱}{64042}
\saveTG{𠸊}{64042}
\saveTG{𣋬}{64043}
\saveTG{𡀗}{64043}
\saveTG{𠸤}{64043}
\saveTG{𤲵}{64043}
\saveTG{𡆋}{64043}
\saveTG{𠺮}{64043}
\saveTG{𥌆}{64043}
\saveTG{𤲔}{64043}
\saveTG{𠼖}{64044}
\saveTG{𠾭}{64044}
\saveTG{𪰫}{64044}
\saveTG{𠻵}{64044}
\saveTG{𠴑}{64044}
\saveTG{㬒}{64044}
\saveTG{䁳}{64044}
\saveTG{𠻶}{64044}
\saveTG{喯}{64044}
\saveTG{𠿺}{64044}
\saveTG{𠱥}{64044}
\saveTG{嘙}{64044}
\saveTG{𠶎}{64047}
\saveTG{𠵿}{64047}
\saveTG{𠱀}{64047}
\saveTG{𠶊}{64047}
\saveTG{㗞}{64047}
\saveTG{𥅗}{64047}
\saveTG{𥆔}{64047}
\saveTG{𠹐}{64047}
\saveTG{𠽑}{64047}
\saveTG{𠲊}{64047}
\saveTG{𠱜}{64047}
\saveTG{𠼼}{64047}
\saveTG{𠿤}{64047}
\saveTG{𣇠}{64047}
\saveTG{嚄}{64047}
\saveTG{𥍜}{64047}
\saveTG{矆}{64047}
\saveTG{哮}{64047}
\saveTG{㫲}{64047}
\saveTG{𣆱}{64047}
\saveTG{哱}{64047}
\saveTG{𠴜}{64047}
\saveTG{𥄏}{64047}
\saveTG{㬦}{64047}
\saveTG{𡅲}{64047}
\saveTG{嘜}{64047}
\saveTG{𠳩}{64047}
\saveTG{𤱍}{64047}
\saveTG{㫴}{64047}
\saveTG{㖫}{64047}
\saveTG{𤲪}{64047}
\saveTG{啵}{64047}
\saveTG{睖}{64047}
\saveTG{吱}{64047}
\saveTG{𨏣}{64048}
\saveTG{𥊊}{64048}
\saveTG{𠶿}{64051}
\saveTG{𥈖}{64051}
\saveTG{𡅆}{64051}
\saveTG{𠺈}{64052}
\saveTG{𥆶}{64052}
\saveTG{𡀌}{64052}
\saveTG{𡃙}{64053}
\saveTG{𣋻}{64053}
\saveTG{䁾}{64053}
\saveTG{瞱}{64054}
\saveTG{嘩}{64054}
\saveTG{曄}{64054}
\saveTG{𡃈}{64056}
\saveTG{㗆}{64056}
\saveTG{暐}{64056}
\saveTG{喡}{64056}
\saveTG{𪾳}{64057}
\saveTG{𪠳}{64057}
\saveTG{䀦}{64060}
\saveTG{𠸏}{64060}
\saveTG{𠵎}{64060}
\saveTG{睹}{64060}
\saveTG{咕}{64060}
\saveTG{瞄}{64060}
\saveTG{喵}{64060}
\saveTG{啫}{64060}
\saveTG{喖}{64060}
\saveTG{暏}{64060}
\saveTG{𪢍}{64061}
\saveTG{𠽻}{64061}
\saveTG{晧}{64061}
\saveTG{唶}{64061}
\saveTG{瞦}{64061}
\saveTG{暿}{64061}
\saveTG{嘻}{64061}
\saveTG{嗜}{64061}
\saveTG{咭}{64061}
\saveTG{哠}{64061}
\saveTG{𥉌}{64061}
\saveTG{𣉟}{64061}
\saveTG{𥉙}{64061}
\saveTG{𥊏}{64061}
\saveTG{𠻧}{64061}
\saveTG{𣈏}{64061}
\saveTG{𥈣}{64061}
\saveTG{軩}{64061}
\saveTG{𠺦}{64061}
\saveTG{嗒}{64061}
\saveTG{𡀁}{64061}
\saveTG{矒}{64062}
\saveTG{𠾏}{64064}
\saveTG{𠸋}{64064}
\saveTG{𥋷}{64064}
\saveTG{喏}{64064}
\saveTG{𣋛}{64064}
\saveTG{𡀩}{64064}
\saveTG{𠺴}{64064}
\saveTG{睰}{64064}
\saveTG{𣈴}{64064}
\saveTG{𪱂}{64064}
\saveTG{𠹲}{64064}
\saveTG{𣌁}{64064}
\saveTG{𠳈}{64064}
\saveTG{𥌳}{64067}
\saveTG{𡄋}{64067}
\saveTG{𠼙}{64068}
\saveTG{𥆪}{64069}
\saveTG{𣇏}{64069}
\saveTG{𠽳}{64069}
\saveTG{𣇿}{64069}
\saveTG{咁}{64070}
\saveTG{𠷅}{64072}
\saveTG{𠴢}{64077}
\saveTG{嚿}{64077}
\saveTG{𠯈}{64080}
\saveTG{𠼏}{64080}
\saveTG{哄}{64081}
\saveTG{𥈰}{64081}
\saveTG{䀧}{64081}
\saveTG{嚏}{64081}
\saveTG{嚔}{64081}
\saveTG{𥆥}{64081}
\saveTG{唭}{64081}
\saveTG{晎}{64081}
\saveTG{唗}{64081}
\saveTG{𤱨}{64081}
\saveTG{瞋}{64081}
\saveTG{嗔}{64081}
\saveTG{𥌪}{64081}
\saveTG{𪡥}{64081}
\saveTG{𠺹}{64082}
\saveTG{𠿝}{64082}
\saveTG{𡁲}{64082}
\saveTG{喷}{64082}
\saveTG{𠻀}{64082}
\saveTG{𣈒}{64084}
\saveTG{𥌞}{64084}
\saveTG{𡃎}{64084}
\saveTG{暯}{64084}
\saveTG{瞙}{64084}
\saveTG{嗼}{64084}
\saveTG{𡃗}{64084}
\saveTG{𣋒}{64084}
\saveTG{暵}{64085}
\saveTG{𠸄}{64085}
\saveTG{暎}{64085}
\saveTG{𡁚}{64085}
\saveTG{嘆}{64085}
\saveTG{䁧}{64085}
\saveTG{𤳉}{64085}
\saveTG{䁐}{64085}
\saveTG{𥌚}{64086}
\saveTG{𡀺}{64086}
\saveTG{𡄙}{64086}
\saveTG{𠾛}{64086}
\saveTG{𥋢}{64086}
\saveTG{噴}{64086}
\saveTG{曂}{64086}
\saveTG{囋}{64086}
\saveTG{𡀅}{64086}
\saveTG{𡂝}{64086}
\saveTG{𣋺}{64086}
\saveTG{𠼄}{64087}
\saveTG{𣈵}{64087}
\saveTG{㽠}{64088}
\saveTG{𣇍}{64088}
\saveTG{䀹}{64088}
\saveTG{唊}{64088}
\saveTG{咴}{64089}
\saveTG{𡃄}{64089}
\saveTG{𥄢}{64090}
\saveTG{咻}{64090}
\saveTG{𪾭}{64090}
\saveTG{晽}{64090}
\saveTG{啉}{64090}
\saveTG{㕲}{64090}
\saveTG{𤱃}{64090}
\saveTG{𥇧}{64091}
\saveTG{噤}{64091}
\saveTG{㖠}{64091}
\saveTG{𠶙}{64091}
\saveTG{𥋴}{64091}
\saveTG{𡅢}{64091}
\saveTG{𡂠}{64091}
\saveTG{𣋜}{64091}
\saveTG{𥈡}{64091}
\saveTG{嗦}{64093}
\saveTG{囌}{64094}
\saveTG{瞸}{64094}
\saveTG{𠱞}{64094}
\saveTG{𡁿}{64094}
\saveTG{喋}{64094}
\saveTG{𠿚}{64094}
\saveTG{𣊝}{64094}
\saveTG{𣈽}{64094}
\saveTG{𣋑}{64094}
\saveTG{䁋}{64094}
\saveTG{𠻬}{64094}
\saveTG{𠾣}{64094}
\saveTG{𥍐}{64094}
\saveTG{𪱉}{64094}
\saveTG{𠾯}{64094}
\saveTG{㖼}{64094}
\saveTG{𠿃}{64094}
\saveTG{𡁽}{64094}
\saveTG{𠵟}{64094}
\saveTG{嗏}{64094}
\saveTG{𡃔}{64096}
\saveTG{瞭}{64096}
\saveTG{暸}{64096}
\saveTG{嘹}{64096}
\saveTG{𪰪}{64098}
\saveTG{睞}{64098}
\saveTG{𢯦}{64098}
\saveTG{唻}{64098}
\saveTG{𤲓}{64098}
\saveTG{𠻟}{64099}
\saveTG{𫏀}{64100}
\saveTG{𧾽}{64100}
\saveTG{斣}{64100}
\saveTG{跗}{64100}
\saveTG{𥁻}{64102}
\saveTG{𨃁}{64102}
\saveTG{𧿫}{64103}
\saveTG{𡬺}{64103}
\saveTG{𡭀}{64103}
\saveTG{𡭈}{64103}
\saveTG{𨆷}{64103}
\saveTG{𨅒}{64103}
\saveTG{𧿘}{64104}
\saveTG{𨀋}{64110}
\saveTG{𧿇}{64112}
\saveTG{𨁷}{64112}
\saveTG{跣}{64112}
\saveTG{蹺}{64112}
\saveTG{𨄎}{64114}
\saveTG{𨁠}{64114}
\saveTG{𨀬}{64114}
\saveTG{𨅣}{64114}
\saveTG{𨅆}{64114}
\saveTG{𧿱}{64114}
\saveTG{踛}{64114}
\saveTG{跬}{64114}
\saveTG{䠑}{64114}
\saveTG{䠰}{64115}
\saveTG{𨆟}{64115}
\saveTG{蹅}{64116}
\saveTG{𧿒}{64117}
\saveTG{𨆚}{64117}
\saveTG{𨃈}{64117}
\saveTG{跇}{64117}
\saveTG{𧿢}{64117}
\saveTG{𠴥}{64117}
\saveTG{𨂁}{64117}
\saveTG{𨄁}{64117}
\saveTG{𨀼}{64117}
\saveTG{𧿕}{64117}
\saveTG{踸}{64118}
\saveTG{踦}{64121}
\saveTG{蹒}{64127}
\saveTG{蹣}{64127}
\saveTG{躏}{64127}
\saveTG{躪}{64127}
\saveTG{𠡎}{64127}
\saveTG{跨}{64127}
\saveTG{蹛}{64127}
\saveTG{𨅤}{64127}
\saveTG{𨆶}{64127}
\saveTG{𨄳}{64127}
\saveTG{𨀒}{64127}
\saveTG{𨆣}{64127}
\saveTG{𨆍}{64127}
\saveTG{𨇿}{64127}
\saveTG{𨈆}{64127}
\saveTG{𨃋}{64127}
\saveTG{𫏥}{64127}
\saveTG{𨂍}{64127}
\saveTG{𨂾}{64127}
\saveTG{𨈅}{64127}
\saveTG{𨃧}{64127}
\saveTG{𨃖}{64127}
\saveTG{𨃺}{64127}
\saveTG{䟜}{64127}
\saveTG{𧾼}{64127}
\saveTG{𨁕}{64130}
\saveTG{𨁯}{64131}
\saveTG{𨇟}{64131}
\saveTG{𨅫}{64132}
\saveTG{躂}{64135}
\saveTG{𨄹}{64136}
\saveTG{𧔺}{64136}
\saveTG{跶}{64138}
\saveTG{𨀾}{64140}
\saveTG{𣀈}{64140}
\saveTG{𨆳}{64141}
\saveTG{跱}{64141}
\saveTG{躊}{64141}
\saveTG{𨂛}{64141}
\saveTG{𨇨}{64141}
\saveTG{䠜}{64142}
\saveTG{𨇞}{64143}
\saveTG{𨃯}{64143}
\saveTG{𨃌}{64143}
\saveTG{𨂄}{64143}
\saveTG{𡌍}{64144}
\saveTG{𨁼}{64144}
\saveTG{踜}{64147}
\saveTG{跛}{64147}
\saveTG{𨄣}{64147}
\saveTG{𨁝}{64147}
\saveTG{𪯇}{64147}
\saveTG{𢻧}{64147}
\saveTG{𢻡}{64147}
\saveTG{跂}{64147}
\saveTG{𨀛}{64147}
\saveTG{𤿮}{64147}
\saveTG{踍}{64147}
\saveTG{𨅢}{64148}
\saveTG{𨂧}{64151}
\saveTG{𨂨}{64151}
\saveTG{𨀗}{64158}
\saveTG{踷}{64160}
\saveTG{跍}{64160}
\saveTG{躤}{64161}
\saveTG{𨃚}{64161}
\saveTG{踖}{64161}
\saveTG{𨄘}{64161}
\saveTG{𨁒}{64161}
\saveTG{𨆮}{64161}
\saveTG{𨀙}{64161}
\saveTG{𨃐}{64163}
\saveTG{蹃}{64164}
\saveTG{𨅓}{64164}
\saveTG{躇}{64164}
\saveTG{𨅱}{64167}
\saveTG{𫏄}{64170}
\saveTG{𧿰}{64170}
\saveTG{}{64180}
\saveTG{跿}{64181}
\saveTG{𫏇}{64181}
\saveTG{蹎}{64181}
\saveTG{踑}{64181}
\saveTG{𨆄}{64182}
\saveTG{𨃝}{64182}
\saveTG{𨇈}{64182}
\saveTG{𨈃}{64182}
\saveTG{躜}{64182}
\saveTG{𨆠}{64184}
\saveTG{𨇇}{64186}
\saveTG{躦}{64186}
\saveTG{𨄷}{64188}
\saveTG{𨁂}{64188}
\saveTG{𨀡}{64189}
\saveTG{𨂕}{64190}
\saveTG{𨃓}{64194}
\saveTG{𨆡}{64194}
\saveTG{𨅈}{64194}
\saveTG{蹀}{64194}
\saveTG{𨂏}{64194}
\saveTG{𨅾}{64194}
\saveTG{𨅼}{64194}
\saveTG{蹽}{64196}
\saveTG{𨂐}{64198}
\saveTG{𣋇}{64203}
\saveTG{𣌒}{64204}
\saveTG{𫌢}{64212}
\saveTG{𧡉}{64213}
\saveTG{𡬩}{64213}
\saveTG{𡬣}{64213}
\saveTG{𧠕}{64213}
\saveTG{䙸}{64213}
\saveTG{𧢤}{64214}
\saveTG{覐}{64214}
\saveTG{𧠴}{64214}
\saveTG{𧠌}{64214}
\saveTG{𠤩}{64217}
\saveTG{𠢃}{64227}
\saveTG{㔥}{64227}
\saveTG{𪟣}{64227}
\saveTG{𪔈}{64240}
\saveTG{𧂓}{64243}
\saveTG{𫌣}{64294}
\saveTG{𪐻}{64303}
\saveTG{𪒶}{64303}
\saveTG{𪓁}{64303}
\saveTG{𪐢}{64310}
\saveTG{黕}{64312}
\saveTG{𪒖}{64314}
\saveTG{𪑭}{64314}
\saveTG{𪒴}{64314}
\saveTG{黤}{64316}
\saveTG{𪒇}{64316}
\saveTG{黮}{64318}
\saveTG{𪓊}{64322}
\saveTG{𪑮}{64327}
\saveTG{𪒪}{64327}
\saveTG{𢣢}{64327}
\saveTG{黝}{64327}
\saveTG{𪑿}{64331}
\saveTG{𪒻}{64331}
\saveTG{𠳢}{64331}
\saveTG{𤉏}{64332}
\saveTG{𤏩}{64332}
\saveTG{𪬩}{64332}
\saveTG{𪓉}{64343}
\saveTG{𪑖}{64344}
\saveTG{𪒈}{64344}
\saveTG{𪓈}{64347}
\saveTG{𪓅}{64351}
\saveTG{𪒯}{64356}
\saveTG{黠}{64361}
\saveTG{䵱}{64361}
\saveTG{䵭}{64364}
\saveTG{黚}{64370}
\saveTG{𪐝}{64380}
\saveTG{黰}{64381}
\saveTG{𪐥}{64383}
\saveTG{黩}{64384}
\saveTG{黷}{64386}
\saveTG{𪒰}{64386}
\saveTG{𪑀}{64389}
\saveTG{𪑨}{64391}
\saveTG{𪒼}{64391}
\saveTG{𪑧}{64394}
\saveTG{𪑂}{64394}
\saveTG{𪑞}{64394}
\saveTG{𪑚}{64398}
\saveTG{𠦫}{64400}
\saveTG{𪟞}{64420}
\saveTG{𡚎}{64421}
\saveTG{𠢝}{64427}
\saveTG{𫅌}{64443}
\saveTG{𥀗}{64447}
\saveTG{𤿱}{64447}
\saveTG{𤿧}{64447}
\saveTG{𢻣}{64447}
\saveTG{𤳆}{64498}
\saveTG{𠒛}{64517}
\saveTG{𤲍}{64588}
\saveTG{𨅩}{64601}
\saveTG{𪯮}{64603}
\saveTG{𤴀}{64617}
\saveTG{𡄀}{64627}
\saveTG{勖}{64627}
\saveTG{㔣}{64627}
\saveTG{𢻪}{64647}
\saveTG{𥀚}{64647}
\saveTG{𥀴}{64647}
\saveTG{𥖕}{64647}
\saveTG{㔠}{64727}
\saveTG{𨚼}{64727}
\saveTG{𢅄}{64727}
\saveTG{𢻌}{64747}
\saveTG{財}{64800}
\saveTG{𪧺}{64803}
\saveTG{𧴼}{64803}
\saveTG{𧴫}{64803}
\saveTG{𣊒}{64804}
\saveTG{𢻖}{64804}
\saveTG{𢅨}{64805}
\saveTG{韙}{64805}
\saveTG{𪹶}{64809}
\saveTG{𧶿}{64811}
\saveTG{貤}{64812}
\saveTG{𧴸}{64817}
\saveTG{𧹀}{64824}
\saveTG{勛}{64827}
\saveTG{𧶰}{64827}
\saveTG{𧶖}{64827}
\saveTG{賄}{64827}
\saveTG{贎}{64827}
\saveTG{勋}{64827}
\saveTG{𧵧}{64831}
\saveTG{𧷹}{64836}
\saveTG{𧷚}{64840}
\saveTG{𧴱}{64840}
\saveTG{𧴶}{64841}
\saveTG{䝰}{64843}
\saveTG{𧶱}{64843}
\saveTG{𧶭}{64844}
\saveTG{𪔡}{64847}
\saveTG{貱}{64847}
\saveTG{䝸}{64847}
\saveTG{贜}{64853}
\saveTG{𧶩}{64855}
\saveTG{𧵑}{64860}
\saveTG{賭}{64860}
\saveTG{𧵊}{64870}
\saveTG{𡁩}{64877}
\saveTG{𧷒}{64881}
\saveTG{𧷸}{64884}
\saveTG{𧸊}{64886}
\saveTG{𧹏}{64886}
\saveTG{贖}{64886}
\saveTG{𧹎}{64886}
\saveTG{𧷻}{64886}
\saveTG{𧶛}{64898}
\saveTG{𡎯}{64910}
\saveTG{𠺜}{64917}
\saveTG{𣜣}{64917}
\saveTG{𢾟}{64940}
\saveTG{𣅢}{64941}
\saveTG{𠹊}{64945}
\saveTG{𢻥}{64947}
\saveTG{㿺}{64947}
\saveTG{𣇾}{64947}
\saveTG{𢻔}{64947}
\saveTG{𪍻}{64947}
\saveTG{𩌺}{64956}
\saveTG{𣛤}{64998}
\saveTG{吽}{65000}
\saveTG{畊}{65000}
\saveTG{𠮼}{65000}
\saveTG{盽}{65000}
\saveTG{㕩}{65000}
\saveTG{𠰜}{65002}
\saveTG{𠲟}{65002}
\saveTG{𤰼}{65002}
\saveTG{𠰼}{65002}
\saveTG{𣅫}{65002}
\saveTG{𠲽}{65002}
\saveTG{𠵮}{65002}
\saveTG{㖀}{65004}
\saveTG{𠯤}{65005}
\saveTG{𪰉}{65005}
\saveTG{𠵡}{65005}
\saveTG{唓}{65006}
\saveTG{𠻆}{65006}
\saveTG{眒}{65006}
\saveTG{𥄡}{65006}
\saveTG{㖂}{65006}
\saveTG{𠳡}{65006}
\saveTG{𣆘}{65006}
\saveTG{呻}{65006}
\saveTG{𠶝}{65007}
\saveTG{𠷈}{65007}
\saveTG{甠}{65010}
\saveTG{𠹃}{65012}
\saveTG{哓}{65012}
\saveTG{晓}{65012}
\saveTG{嚍}{65012}
\saveTG{𡂍}{65012}
\saveTG{𤯠}{65012}
\saveTG{𣇭}{65012}
\saveTG{𣅟}{65012}
\saveTG{𣉗}{65012}
\saveTG{𠰮}{65015}
\saveTG{𥊮}{65016}
\saveTG{㗲}{65016}
\saveTG{吨}{65017}
\saveTG{旽}{65017}
\saveTG{𪢰}{65017}
\saveTG{盹}{65017}
\saveTG{𥅿}{65017}
\saveTG{䀓}{65017}
\saveTG{𪠺}{65017}
\saveTG{𨊯}{65017}
\saveTG{𥊝}{65017}
\saveTG{𣊎}{65017}
\saveTG{𠮬}{65017}
\saveTG{𡅏}{65018}
\saveTG{𤲦}{65021}
\saveTG{㬘}{65024}
\saveTG{晴}{65027}
\saveTG{𤲟}{65027}
\saveTG{𥄔}{65027}
\saveTG{𡃑}{65027}
\saveTG{昲}{65027}
\saveTG{啨}{65027}
\saveTG{䀻}{65027}
\saveTG{𡁔}{65027}
\saveTG{䀟}{65027}
\saveTG{}{65027}
\saveTG{昁}{65027}
\saveTG{睛}{65027}
\saveTG{嘯}{65027}
\saveTG{嘨}{65027}
\saveTG{啸}{65027}
\saveTG{咈}{65027}
\saveTG{𠰀}{65027}
\saveTG{嗹}{65030}
\saveTG{㬩}{65030}
\saveTG{𠵢}{65030}
\saveTG{呠}{65030}
\saveTG{𠼭}{65030}
\saveTG{𤱙}{65030}
\saveTG{𡂞}{65031}
\saveTG{𣆣}{65031}
\saveTG{䁸}{65032}
\saveTG{䁃}{65032}
\saveTG{囔}{65032}
\saveTG{𣋏}{65032}
\saveTG{𠶓}{65032}
\saveTG{噥}{65032}
\saveTG{𠳜}{65032}
\saveTG{啭}{65032}
\saveTG{𪾵}{65033}
\saveTG{𪱇}{65033}
\saveTG{𠽡}{65033}
\saveTG{𥊩}{65036}
\saveTG{瞣}{65036}
\saveTG{嚖}{65037}
\saveTG{𡆂}{65038}
\saveTG{𥌰}{65038}
\saveTG{曃}{65039}
\saveTG{𥊵}{65039}
\saveTG{睷}{65040}
\saveTG{畴}{65040}
\saveTG{𠼯}{65040}
\saveTG{𠿶}{65041}
\saveTG{暷}{65043}
\saveTG{䁣}{65043}
\saveTG{𡁯}{65043}
\saveTG{囀}{65043}
\saveTG{𡀯}{65043}
\saveTG{𠶭}{65044}
\saveTG{啛}{65044}
\saveTG{㗕}{65044}
\saveTG{瞜}{65044}
\saveTG{𠳴}{65044}
\saveTG{嘍}{65044}
\saveTG{畘}{65047}
\saveTG{唛}{65047}
\saveTG{𠸻}{65047}
\saveTG{𡄧}{65047}
\saveTG{𥉇}{65047}
\saveTG{𥅆}{65047}
\saveTG{𣆀}{65047}
\saveTG{𪰷}{65047}
\saveTG{呥}{65047}
\saveTG{𥆓}{65050}
\saveTG{𡆀}{65056}
\saveTG{𥌦}{65056}
\saveTG{𠷮}{65057}
\saveTG{𪽙}{65058}
\saveTG{㖺}{65058}
\saveTG{𠺭}{65058}
\saveTG{唪}{65058}
\saveTG{𠾴}{65058}
\saveTG{𣈖}{65058}
\saveTG{眒}{65060}
\saveTG{𠱋}{65060}
\saveTG{𠶢}{65060}
\saveTG{㖆}{65065}
\saveTG{嘈}{65066}
\saveTG{𥋋}{65068}
\saveTG{暙}{65068}
\saveTG{𠾱}{65068}
\saveTG{睶}{65068}
\saveTG{𡅰}{65069}
\saveTG{𥊎}{65077}
\saveTG{暳}{65077}
\saveTG{𠽅}{65077}
\saveTG{嘒}{65077}
\saveTG{眏}{65080}
\saveTG{𥆭}{65080}
\saveTG{眣}{65080}
\saveTG{映}{65080}
\saveTG{咉}{65080}
\saveTG{畉}{65080}
\saveTG{呋}{65080}
\saveTG{昳}{65080}
\saveTG{吷}{65080}
\saveTG{呹}{65080}
\saveTG{𥄑}{65080}
\saveTG{㫙}{65080}
\saveTG{㫸}{65081}
\saveTG{睓}{65081}
\saveTG{𠶁}{65081}
\saveTG{𡄱}{65081}
\saveTG{𠽕}{65081}
\saveTG{唺}{65081}
\saveTG{晪}{65081}
\saveTG{啑}{65081}
\saveTG{睫}{65081}
\saveTG{𪰶}{65082}
\saveTG{眱}{65082}
\saveTG{𥊜}{65082}
\saveTG{䀗}{65082}
\saveTG{咦}{65082}
\saveTG{啧}{65082}
\saveTG{瞆}{65082}
\saveTG{𤰮}{65082}
\saveTG{𠺋}{65082}
\saveTG{𪡺}{65082}
\saveTG{𠷬}{65082}
\saveTG{𣅡}{65082}
\saveTG{𣆰}{65082}
\saveTG{𣉅}{65084}
\saveTG{𠸫}{65084}
\saveTG{嘳}{65086}
\saveTG{𠾚}{65086}
\saveTG{𡂐}{65086}
\saveTG{曊}{65086}
\saveTG{瞶}{65086}
\saveTG{嘖}{65086}
\saveTG{瞔}{65086}
\saveTG{𤳎}{65086}
\saveTG{𠻣}{65089}
\saveTG{咮}{65090}
\saveTG{昩}{65090}
\saveTG{眜}{65090}
\saveTG{眛}{65090}
\saveTG{味}{65090}
\saveTG{昧}{65090}
\saveTG{睐}{65090}
\saveTG{𠵈}{65090}
\saveTG{𥅦}{65090}
\saveTG{𠰌}{65090}
\saveTG{𠱤}{65090}
\saveTG{𥅲}{65092}
\saveTG{𣆦}{65092}
\saveTG{嗉}{65093}
\saveTG{𪰸}{65093}
\saveTG{𠹳}{65094}
\saveTG{𥉜}{65094}
\saveTG{嗪}{65094}
\saveTG{𠺾}{65095}
\saveTG{𡄮}{65096}
\saveTG{𪢘}{65096}
\saveTG{𤲚}{65096}
\saveTG{暕}{65096}
\saveTG{𣋩}{65096}
\saveTG{𠼂}{65096}
\saveTG{㖦}{65096}
\saveTG{𠲿}{65096}
\saveTG{𠽔}{65096}
\saveTG{𥈵}{65096}
\saveTG{㘑}{65099}
\saveTG{𣇨}{65099}
\saveTG{𥌤}{65099}
\saveTG{𥌿}{65099}
\saveTG{𧿣}{65100}
\saveTG{𨁬}{65102}
\saveTG{跩}{65106}
\saveTG{𨀞}{65107}
\saveTG{跷}{65112}
\saveTG{𨄴}{65117}
\saveTG{𧿬}{65117}
\saveTG{𨆨}{65118}
\saveTG{𨀦}{65122}
\saveTG{𧿲}{65127}
\saveTG{𫒁}{65127}
\saveTG{𫏏}{65127}
\saveTG{𧿳}{65127}
\saveTG{䟛}{65127}
\saveTG{𧿾}{65130}
\saveTG{蹥}{65130}
\saveTG{𨆞}{65132}
\saveTG{𨄲}{65136}
\saveTG{蹥}{65136}
\saveTG{𨇀}{65137}
\saveTG{𫏩}{65138}
\saveTG{𨄞}{65139}
\saveTG{踌}{65140}
\saveTG{踺}{65140}
\saveTG{𨄔}{65143}
\saveTG{𨄜}{65144}
\saveTG{𨀱}{65147}
\saveTG{𨀆}{65147}
\saveTG{𨀅}{65147}
\saveTG{𨇍}{65156}
\saveTG{𨂭}{65157}
\saveTG{𨀪}{65165}
\saveTG{蹧}{65166}
\saveTG{𨅕}{65168}
\saveTG{踳}{65168}
\saveTG{蹖}{65177}
\saveTG{跌}{65180}
\saveTG{趹}{65180}
\saveTG{趺}{65180}
\saveTG{𨁜}{65180}
\saveTG{踕}{65181}
\saveTG{䠄}{65181}
\saveTG{趹}{65182}
\saveTG{跠}{65182}
\saveTG{𨂡}{65184}
\saveTG{𨇃}{65186}
\saveTG{蹟}{65186}
\saveTG{蹪}{65186}
\saveTG{跊}{65190}
\saveTG{𨀤}{65190}
\saveTG{跦}{65190}
\saveTG{𧿴}{65190}
\saveTG{𫏌}{65192}
\saveTG{䟱}{65192}
\saveTG{𨃥}{65194}
\saveTG{踈}{65196}
\saveTG{𨃀}{65196}
\saveTG{𨆹}{65196}
\saveTG{𪽑}{65202}
\saveTG{𧢙}{65211}
\saveTG{𧡝}{65218}
\saveTG{𠽘}{65227}
\saveTG{𤳷}{65260}
\saveTG{𣉺}{65282}
\saveTG{𪱖}{65282}
\saveTG{𫏅}{65282}
\saveTG{𪒆}{65304}
\saveTG{𪑯}{65307}
\saveTG{黗}{65317}
\saveTG{𪐠}{65317}
\saveTG{𪐟}{65317}
\saveTG{𪒬}{65332}
\saveTG{𪒜}{65333}
\saveTG{𪒡}{65339}
\saveTG{𪑼}{65347}
\saveTG{𪒑}{65386}
\saveTG{𪒦}{65386}
\saveTG{䵢}{65390}
\saveTG{𪑠}{65399}
\saveTG{㕜}{65407}
\saveTG{㽒}{65412}
\saveTG{𢾔}{65482}
\saveTG{𥆸}{65482}
\saveTG{𢆦}{65496}
\saveTG{𪡽}{65508}
\saveTG{𣫹}{65555}
\saveTG{𨎠}{65605}
\saveTG{𤳭}{65617}
\saveTG{𤲼}{65706}
\saveTG{𪽝}{65760}
\saveTG{𣈡}{65800}
\saveTG{韪}{65802}
\saveTG{賗}{65806}
\saveTG{𧸯}{65808}
\saveTG{貹}{65810}
\saveTG{贐}{65812}
\saveTG{}{65827}
\saveTG{䝼}{65827}
\saveTG{贃}{65836}
\saveTG{𧸽}{65838}
\saveTG{𧵘}{65844}
\saveTG{𧷡}{65844}
\saveTG{𧵷}{65847}
\saveTG{購}{65847}
\saveTG{𧶥}{65855}
\saveTG{賰}{65868}
\saveTG{賟}{65881}
\saveTG{𧵌}{65882}
\saveTG{𧸃}{65886}
\saveTG{𧵖}{65890}
\saveTG{𧵺}{65892}
\saveTG{𨾄}{65899}
\saveTG{𠻇}{65906}
\saveTG{𠶞}{65915}
\saveTG{䀳}{65960}
\saveTG{𡀖}{65993}
\saveTG{䀠}{66000}
\saveTG{𥇘}{66000}
\saveTG{𥊞}{66000}
\saveTG{𥇈}{66000}
\saveTG{𠷹}{66000}
\saveTG{𤲴}{66000}
\saveTG{𤲭}{66000}
\saveTG{呬}{66000}
\saveTG{啯}{66000}
\saveTG{嘓}{66000}
\saveTG{啝}{66000}
\saveTG{睏}{66000}
\saveTG{咽}{66000}
\saveTG{吅}{66000}
\saveTG{咖}{66000}
\saveTG{𠰸}{66000}
\saveTG{昍}{66000}
\saveTG{𥇭}{66000}
\saveTG{𠷝}{66000}
\saveTG{𠳁}{66000}
\saveTG{𠲛}{66000}
\saveTG{𡁴}{66000}
\saveTG{𠴱}{66000}
\saveTG{𠱠}{66000}
\saveTG{㖥}{66000}
\saveTG{𠯐}{66000}
\saveTG{𣋗}{66000}
\saveTG{𣇖}{66000}
\saveTG{𥆃}{66000}
\saveTG{𥌘}{66000}
\saveTG{嗰}{66000}
\saveTG{咱}{66002}
\saveTG{啪}{66002}
\saveTG{𠱖}{66002}
\saveTG{㕷}{66002}
\saveTG{𣆆}{66002}
\saveTG{𪡈}{66002}
\saveTG{𠵆}{66010}
\saveTG{呾}{66010}
\saveTG{𠶒}{66010}
\saveTG{㫜}{66010}
\saveTG{𥅃}{66010}
\saveTG{𣋖}{66011}
\saveTG{眖}{66012}
\saveTG{𠷚}{66012}
\saveTG{㬈}{66012}
\saveTG{嚫}{66012}
\saveTG{瞡}{66012}
\saveTG{嗢}{66012}
\saveTG{呪}{66012}
\saveTG{睍}{66012}
\saveTG{哯}{66012}
\saveTG{𠽇}{66012}
\saveTG{晛}{66012}
\saveTG{𪢎}{66014}
\saveTG{䁼}{66014}
\saveTG{睈}{66014}
\saveTG{㖏}{66014}
\saveTG{喤}{66014}
\saveTG{唣}{66014}
\saveTG{嚜}{66014}
\saveTG{𠴔}{66014}
\saveTG{𠹡}{66014}
\saveTG{𣈷}{66014}
\saveTG{睲}{66015}
\saveTG{𡆆}{66015}
\saveTG{𠸨}{66015}
\saveTG{㬬}{66015}
\saveTG{𥈯}{66015}
\saveTG{囉}{66015}
\saveTG{𥆼}{66015}
\saveTG{曪}{66015}
\saveTG{暒}{66015}
\saveTG{哩}{66015}
\saveTG{𥇊}{66017}
\saveTG{䁜}{66017}
\saveTG{𥆢}{66017}
\saveTG{𥆩}{66017}
\saveTG{𩴈}{66017}
\saveTG{𥉨}{66017}
\saveTG{䁛}{66017}
\saveTG{𥆰}{66017}
\saveTG{𣈀}{66017}
\saveTG{𠻷}{66017}
\saveTG{𪽔}{66017}
\saveTG{唈}{66017}
\saveTG{𠽞}{66017}
\saveTG{𣈣}{66017}
\saveTG{𧠲}{66017}
\saveTG{𧡻}{66017}
\saveTG{𠹇}{66017}
\saveTG{𠴺}{66017}
\saveTG{𣆿}{66017}
\saveTG{𠺌}{66017}
\saveTG{𠺷}{66017}
\saveTG{𪡓}{66017}
\saveTG{𡃁}{66017}
\saveTG{𣉨}{66017}
\saveTG{𡂃}{66017}
\saveTG{㫛}{66017}
\saveTG{𧡆}{66017}
\saveTG{𥉎}{66017}
\saveTG{𥆒}{66017}
\saveTG{𠴡}{66017}
\saveTG{喅}{66018}
\saveTG{𣈫}{66018}
\saveTG{𡁪}{66020}
\saveTG{嚊}{66021}
\saveTG{睤}{66021}
\saveTG{𪾥}{66027}
\saveTG{𡁰}{66027}
\saveTG{𡃉}{66027}
\saveTG{𠵭}{66027}
\saveTG{㗁}{66027}
\saveTG{𡄃}{66027}
\saveTG{啺}{66027}
\saveTG{𥈭}{66027}
\saveTG{䁑}{66027}
\saveTG{𠽀}{66027}
\saveTG{㘄}{66027}
\saveTG{𡄎}{66027}
\saveTG{𡃶}{66027}
\saveTG{㗄}{66027}
\saveTG{𠿥}{66027}
\saveTG{𠲸}{66027}
\saveTG{𡂼}{66027}
\saveTG{㖞}{66027}
\saveTG{𠲢}{66027}
\saveTG{𠴭}{66027}
\saveTG{𠵫}{66027}
\saveTG{𠿐}{66027}
\saveTG{𡂮}{66027}
\saveTG{𣉌}{66027}
\saveTG{㬂}{66027}
\saveTG{矈}{66027}
\saveTG{𥌹}{66027}
\saveTG{𥈮}{66027}
\saveTG{𥋛}{66027}
\saveTG{䁌}{66027}
\saveTG{𥈎}{66027}
\saveTG{𥈬}{66027}
\saveTG{}{66027}
\saveTG{喁}{66027}
\saveTG{晹}{66027}
\saveTG{暍}{66027}
\saveTG{暘}{66027}
\saveTG{睗}{66027}
\saveTG{矏}{66027}
\saveTG{矊}{66027}
\saveTG{啰}{66027}
\saveTG{睊}{66027}
\saveTG{喟}{66027}
\saveTG{喝}{66027}
\saveTG{呺}{66027}
\saveTG{咢}{66027}
\saveTG{噣}{66027}
\saveTG{畼}{66027}
\saveTG{𥌗}{66027}
\saveTG{𪾨}{66027}
\saveTG{𠷟}{66028}
\saveTG{嗯}{66030}
\saveTG{𠺒}{66030}
\saveTG{𠷣}{66031}
\saveTG{𣉳}{66031}
\saveTG{䁫}{66031}
\saveTG{嘿}{66031}
\saveTG{𥌡}{66032}
\saveTG{𡆗}{66032}
\saveTG{𡀧}{66032}
\saveTG{喂}{66032}
\saveTG{𠷉}{66032}
\saveTG{𣌝}{66032}
\saveTG{𣊐}{66032}
\saveTG{𪢥}{66032}
\saveTG{𠾪}{66032}
\saveTG{䁵}{66032}
\saveTG{𪡠}{66032}
\saveTG{𠼊}{66032}
\saveTG{𪱆}{66032}
\saveTG{噮}{66032}
\saveTG{𣉍}{66032}
\saveTG{𥌁}{66033}
\saveTG{𡃱}{66033}
\saveTG{嚺}{66033}
\saveTG{𡄤}{66037}
\saveTG{𠶼}{66037}
\saveTG{𠸵}{66037}
\saveTG{𡀾}{66039}
\saveTG{嚃}{66039}
\saveTG{睥}{66040}
\saveTG{𠳣}{66040}
\saveTG{啤}{66040}
\saveTG{唕}{66040}
\saveTG{晘}{66041}
\saveTG{曎}{66041}
\saveTG{睅}{66041}
\saveTG{哻}{66041}
\saveTG{嘚}{66041}
\saveTG{㘁}{66041}
\saveTG{䁺}{66041}
\saveTG{𥋭}{66041}
\saveTG{𡀞}{66041}
\saveTG{𠵄}{66041}
\saveTG{𠳲}{66041}
\saveTG{𡅵}{66041}
\saveTG{𥊙}{66042}
\saveTG{𡃃}{66042}
\saveTG{𠶻}{66042}
\saveTG{䁒}{66042}
\saveTG{𠿴}{66043}
\saveTG{𣈜}{66043}
\saveTG{𥊤}{66043}
\saveTG{𠵨}{66043}
\saveTG{䁙}{66044}
\saveTG{暥}{66044}
\saveTG{𠹵}{66044}
\saveTG{𥌽}{66044}
\saveTG{嚶}{66044}
\saveTG{𣈢}{66045}
\saveTG{𥋽}{66046}
\saveTG{𪰺}{66047}
\saveTG{𠳦}{66047}
\saveTG{𥊴}{66047}
\saveTG{𣌗}{66047}
\saveTG{𣋁}{66047}
\saveTG{𠷾}{66047}
\saveTG{𪢚}{66047}
\saveTG{𠼦}{66047}
\saveTG{𠻦}{66047}
\saveTG{𠿬}{66047}
\saveTG{嘬}{66047}
\saveTG{噑}{66048}
\saveTG{𥍓}{66048}
\saveTG{㘙}{66048}
\saveTG{嗥}{66048}
\saveTG{嘷}{66048}
\saveTG{𠷙}{66048}
\saveTG{暭}{66048}
\saveTG{曍}{66048}
\saveTG{曮}{66048}
\saveTG{暤}{66048}
\saveTG{𣈾}{66050}
\saveTG{呷}{66050}
\saveTG{𠶟}{66050}
\saveTG{𣊊}{66053}
\saveTG{𤳣}{66054}
\saveTG{嗶}{66054}
\saveTG{暺}{66056}
\saveTG{嘽}{66056}
\saveTG{𡃐}{66056}
\saveTG{晿}{66060}
\saveTG{唱}{66060}
\saveTG{𥉞}{66060}
\saveTG{𥈆}{66060}
\saveTG{𡅜}{66061}
\saveTG{𡄗}{66061}
\saveTG{𣌈}{66061}
\saveTG{𠴊}{66062}
\saveTG{𥆻}{66062}
\saveTG{曙}{66064}
\saveTG{龧}{66064}
\saveTG{𥌓}{66064}
\saveTG{𡃆}{66066}
\saveTG{𥋒}{66068}
\saveTG{唄}{66080}
\saveTG{𣇜}{66080}
\saveTG{呮}{66080}
\saveTG{㖷}{66081}
\saveTG{哫}{66081}
\saveTG{睼}{66081}
\saveTG{𠽮}{66081}
\saveTG{𣉆}{66082}
\saveTG{𠽯}{66082}
\saveTG{𡄷}{66082}
\saveTG{嗅}{66084}
\saveTG{𠱐}{66084}
\saveTG{瞁}{66084}
\saveTG{𠲂}{66084}
\saveTG{䁲}{66086}
\saveTG{䁚}{66086}
\saveTG{𠹚}{66086}
\saveTG{𧷝}{66086}
\saveTG{𡅙}{66086}
\saveTG{嘪}{66086}
\saveTG{𥇏}{66089}
\saveTG{𪱀}{66093}
\saveTG{𥉹}{66093}
\saveTG{𥍔}{66093}
\saveTG{𤴈}{66093}
\saveTG{𠼱}{66093}
\saveTG{𣋝}{66094}
\saveTG{𠹑}{66094}
\saveTG{𣇫}{66094}
\saveTG{𠳳}{66094}
\saveTG{𣈁}{66094}
\saveTG{噪}{66094}
\saveTG{𠼎}{66094}
\saveTG{𠸒}{66094}
\saveTG{暞}{66094}
\saveTG{㗎}{66094}
\saveTG{矂}{66094}
\saveTG{𥉒}{66094}
\saveTG{𥆫}{66094}
\saveTG{𣌑}{66094}
\saveTG{𥌥}{66095}
\saveTG{暻}{66096}
\saveTG{𠾶}{66096}
\saveTG{𥋓}{66096}
\saveTG{曝}{66099}
\saveTG{嚗}{66099}
\saveTG{𥉀}{66099}
\saveTG{𧿭}{66100}
\saveTG{䟧}{66100}
\saveTG{𨁉}{66100}
\saveTG{䠅}{66100}
\saveTG{跏}{66100}
\saveTG{𥆤}{66100}
\saveTG{𫏜}{66100}
\saveTG{踟}{66100}
\saveTG{瞾}{66102}
\saveTG{𥂍}{66102}
\saveTG{𠱯}{66102}
\saveTG{𤦉}{66104}
\saveTG{𡋐}{66104}
\saveTG{壨}{66104}
\saveTG{𡓶}{66104}
\saveTG{𡓿}{66104}
\saveTG{𡀓}{66104}
\saveTG{𠸈}{66104}
\saveTG{𡅚}{66104}
\saveTG{𠹈}{66105}
\saveTG{鑍}{66109}
\saveTG{𨀏}{66110}
\saveTG{𨈈}{66112}
\saveTG{𨁎}{66114}
\saveTG{𣇒}{66114}
\saveTG{𠶬}{66114}
\saveTG{𨂈}{66114}
\saveTG{㘗}{66115}
\saveTG{躣}{66115}
\saveTG{𨁫}{66115}
\saveTG{𨇽}{66115}
\saveTG{𨃷}{66117}
\saveTG{䠘}{66117}
\saveTG{𨞰}{66117}
\saveTG{𪛄}{66117}
\saveTG{𡄞}{66117}
\saveTG{𨁲}{66117}
\saveTG{𨁍}{66117}
\saveTG{𨛜}{66117}
\saveTG{𨞠}{66117}
\saveTG{𨇷}{66117}
\saveTG{𨇑}{66117}
\saveTG{𨁞}{66117}
\saveTG{𧡈}{66117}
\saveTG{𧡶}{66117}
\saveTG{𧡰}{66117}
\saveTG{𨄸}{66121}
\saveTG{𫏑}{66127}
\saveTG{骂}{66127}
\saveTG{躅}{66127}
\saveTG{踼}{66127}
\saveTG{踢}{66127}
\saveTG{𨆭}{66127}
\saveTG{𨆗}{66127}
\saveTG{𨀌}{66127}
\saveTG{𨀽}{66127}
\saveTG{𨃃}{66127}
\saveTG{蹋}{66127}
\saveTG{𨃡}{66130}
\saveTG{𧕛}{66131}
\saveTG{𧕦}{66131}
\saveTG{𨃄}{66132}
\saveTG{𨇱}{66132}
\saveTG{𨆈}{66132}
\saveTG{㬤}{66133}
\saveTG{𨆰}{66133}
\saveTG{𧍨}{66136}
\saveTG{𠵪}{66140}
\saveTG{𨆅}{66141}
\saveTG{𨁄}{66141}
\saveTG{𨁽}{66143}
\saveTG{䠋}{66145}
\saveTG{𠹺}{66145}
\saveTG{躩}{66147}
\saveTG{𨈍}{66147}
\saveTG{𨅎}{66147}
\saveTG{𨫙}{66147}
\saveTG{𠹣}{66148}
\saveTG{𧿵}{66150}
\saveTG{蹕}{66154}
\saveTG{䠤}{66156}
\saveTG{䠦}{66168}
\saveTG{踶}{66181}
\saveTG{踀}{66181}
\saveTG{𨅜}{66181}
\saveTG{𨇬}{66182}
\saveTG{䠐}{66184}
\saveTG{䠗}{66184}
\saveTG{䠝}{66186}
\saveTG{𨄱}{66193}
\saveTG{躁}{66194}
\saveTG{踝}{66194}
\saveTG{𨃔}{66194}
\saveTG{𨇅}{66199}
\saveTG{𧠗}{66210}
\saveTG{𧠦}{66210}
\saveTG{𠻰}{66211}
\saveTG{𧡭}{66211}
\saveTG{𠼛}{66211}
\saveTG{覨}{66212}
\saveTG{𫌝}{66212}
\saveTG{覞}{66212}
\saveTG{𧡇}{66212}
\saveTG{𡆛}{66215}
\saveTG{瞿}{66215}
\saveTG{𪱁}{66215}
\saveTG{𨾴}{66215}
\saveTG{𧢩}{66217}
\saveTG{𠺐}{66217}
\saveTG{㒭}{66217}
\saveTG{咒}{66217}
\saveTG{𧢛}{66217}
\saveTG{𨜝}{66217}
\saveTG{𣃒}{66221}
\saveTG{𣊷}{66222}
\saveTG{𥋡}{66227}
\saveTG{𢃋}{66227}
\saveTG{𥝉}{66227}
\saveTG{𩫳}{66227}
\saveTG{𠾧}{66227}
\saveTG{𠳯}{66227}
\saveTG{𥍈}{66227}
\saveTG{𥌖}{66227}
\saveTG{𩱴}{66227}
\saveTG{𦡺}{66227}
\saveTG{𥌂}{66227}
\saveTG{𥉁}{66228}
\saveTG{𧱄}{66232}
\saveTG{𡀈}{66241}
\saveTG{𥈍}{66247}
\saveTG{𡃫}{66247}
\saveTG{𡅔}{66247}
\saveTG{𡅮}{66247}
\saveTG{𡅴}{66247}
\saveTG{𡆉}{66247}
\saveTG{𠸅}{66247}
\saveTG{嚴}{66248}
\saveTG{䑝}{66257}
\saveTG{𪑜}{66300}
\saveTG{𣈗}{66300}
\saveTG{䵣}{66310}
\saveTG{𪐪}{66310}
\saveTG{𡁂}{66314}
\saveTG{𪑈}{66317}
\saveTG{𫜙}{66317}
\saveTG{䵪}{66317}
\saveTG{喌}{66320}
\saveTG{𦉹}{66327}
\saveTG{𪑦}{66327}
\saveTG{鸎}{66327}
\saveTG{駡}{66327}
\saveTG{鷪}{66327}
\saveTG{𪒎}{66327}
\saveTG{䵮}{66327}
\saveTG{𪄙}{66327}
\saveTG{𪑹}{66328}
\saveTG{愳}{66330}
\saveTG{𪹬}{66331}
\saveTG{𪓂}{66331}
\saveTG{𧢅}{66331}
\saveTG{𢠸}{66334}
\saveTG{𤑄}{66338}
\saveTG{𪒙}{66347}
\saveTG{𪓆}{66384}
\saveTG{𪒾}{66384}
\saveTG{䵲}{66394}
\saveTG{𪡔}{66400}
\saveTG{𠳵}{66400}
\saveTG{𢆔}{66401}
\saveTG{斝}{66402}
\saveTG{嬰}{66404}
\saveTG{𡂽}{66406}
\saveTG{𥊐}{66407}
\saveTG{㖾}{66407}
\saveTG{𥆣}{66407}
\saveTG{䂄}{66407}
\saveTG{矍}{66407}
\saveTG{𡂇}{66407}
\saveTG{𥈫}{66407}
\saveTG{𥊿}{66407}
\saveTG{𣉬}{66407}
\saveTG{𤕧}{66408}
\saveTG{𡁢}{66412}
\saveTG{𥋥}{66417}
\saveTG{𧢭}{66417}
\saveTG{𥯽}{66421}
\saveTG{𢣲}{66427}
\saveTG{嬲}{66427}
\saveTG{𤲶}{66427}
\saveTG{𡜓}{66440}
\saveTG{𥈪}{66441}
\saveTG{𡣠}{66442}
\saveTG{㗗}{66445}
\saveTG{單}{66500}
\saveTG{𢦸}{66503}
\saveTG{𢧮}{66503}
\saveTG{𡁘}{66504}
\saveTG{𩲣}{66517}
\saveTG{𨞏}{66517}
\saveTG{𤳵}{66556}
\saveTG{㖐}{66574}
\saveTG{㽞}{66600}
\saveTG{𨤷}{66601}
\saveTG{𧮥}{66601}
\saveTG{嘼}{66601}
\saveTG{譻}{66601}
\saveTG{𡃷}{66604}
\saveTG{𤴑}{66604}
\saveTG{䂂}{66615}
\saveTG{𨝚}{66617}
\saveTG{𡆔}{66621}
\saveTG{𤳴}{66626}
\saveTG{嘂}{66660}
\saveTG{𤳳}{66660}
\saveTG{㗊}{66660}
\saveTG{𡄹}{66660}
\saveTG{𡅽}{66660}
\saveTG{𡂨}{66660}
\saveTG{𣊫}{66660}
\saveTG{𣊭}{66660}
\saveTG{嚻}{66660}
\saveTG{𡈶}{66660}
\saveTG{𤳹}{66661}
\saveTG{𩁁}{66661}
\saveTG{𡅻}{66661}
\saveTG{嚚}{66661}
\saveTG{噐}{66661}
\saveTG{𪉀}{66662}
\saveTG{𧮣}{66662}
\saveTG{𤴒}{66662}
\saveTG{𡄛}{66663}
\saveTG{𤴌}{66663}
\saveTG{𠾅}{66666}
\saveTG{𤴐}{66666}
\saveTG{𤴇}{66666}
\saveTG{𦉩}{66666}
\saveTG{𣡺}{66666}
\saveTG{𤴊}{66667}
\saveTG{𠼨}{66667}
\saveTG{𡼚}{66667}
\saveTG{囂}{66668}
\saveTG{𤴄}{66668}
\saveTG{器}{66668}
\saveTG{𠾖}{66668}
\saveTG{嚣}{66668}
\saveTG{𡅱}{66668}
\saveTG{𤳧}{66681}
\saveTG{𤳻}{66693}
\saveTG{𠸶}{66710}
\saveTG{𠻌}{66712}
\saveTG{𪓪}{66714}
\saveTG{鼍}{66715}
\saveTG{𣰧}{66715}
\saveTG{𣰣}{66715}
\saveTG{𥋁}{66715}
\saveTG{𡀋}{66715}
\saveTG{𣰡}{66715}
\saveTG{𠼅}{66715}
\saveTG{鼍}{66716}
\saveTG{鼉}{66717}
\saveTG{甖}{66717}
\saveTG{𫜝}{66717}
\saveTG{䣈}{66717}
\saveTG{𨞔}{66717}
\saveTG{𨞸}{66717}
\saveTG{𩳺}{66717}
\saveTG{𡁥}{66725}
\saveTG{㕺}{66727}
\saveTG{𪡰}{66727}
\saveTG{𩝵}{66732}
\saveTG{𡄿}{66732}
\saveTG{𠽽}{66744}
\saveTG{𤕦}{66744}
\saveTG{𡄉}{66744}
\saveTG{罌}{66772}
\saveTG{𡷇}{66772}
\saveTG{𡼯}{66772}
\saveTG{𦉍}{66774}
\saveTG{𦈯}{66774}
\saveTG{𦦄}{66777}
\saveTG{𦤪}{66784}
\saveTG{𠸭}{66800}
\saveTG{𧶮}{66800}
\saveTG{𧶞}{66800}
\saveTG{𠽝}{66801}
\saveTG{𨤱}{66801}
\saveTG{𧡨}{66801}
\saveTG{𥋫}{66802}
\saveTG{𥈀}{66804}
\saveTG{𥈏}{66804}
\saveTG{哭}{66804}
\saveTG{𪡼}{66804}
\saveTG{𠽸}{66806}
\saveTG{焸}{66809}
\saveTG{煛}{66809}
\saveTG{𧶬}{66810}
\saveTG{貺}{66812}
\saveTG{𧶔}{66814}
\saveTG{𤴏}{66815}
\saveTG{𧹐}{66815}
\saveTG{䚋}{66817}
\saveTG{䚑}{66817}
\saveTG{𨜯}{66817}
\saveTG{𧷛}{66817}
\saveTG{䞋}{66817}
\saveTG{𩴬}{66817}
\saveTG{𧸨}{66827}
\saveTG{𧶽}{66827}
\saveTG{賜}{66827}
\saveTG{𧶨}{66830}
\saveTG{𧵐}{66837}
\saveTG{賵}{66860}
\saveTG{𧶧}{66860}
\saveTG{賏}{66880}
\saveTG{𧵙}{66880}
\saveTG{𧸏}{66886}
\saveTG{𧷰}{66886}
\saveTG{𤑽}{66889}
\saveTG{𧷳}{66893}
\saveTG{𠸱}{66901}
\saveTG{𣜢}{66927}
\saveTG{𪳐}{66940}
\saveTG{𥇬}{66945}
\saveTG{槑}{66994}
\saveTG{𣛕}{66995}
\saveTG{𣡾}{66995}
\saveTG{𤰨}{67007}
\saveTG{吚}{67007}
\saveTG{㕧}{67007}
\saveTG{咿}{67007}
\saveTG{叽}{67010}
\saveTG{𠱨}{67010}
\saveTG{𠷕}{67010}
\saveTG{𣈼}{67010}
\saveTG{𠳑}{67010}
\saveTG{𥃴}{67010}
\saveTG{𠮙}{67011}
\saveTG{𥄿}{67011}
\saveTG{𣄻}{67011}
\saveTG{𠰘}{67012}
\saveTG{𠸡}{67012}
\saveTG{𥄨}{67012}
\saveTG{眤}{67012}
\saveTG{晲}{67012}
\saveTG{睨}{67012}
\saveTG{呢}{67012}
\saveTG{睌}{67012}
\saveTG{唲}{67012}
\saveTG{吜}{67012}
\saveTG{𣆓}{67012}
\saveTG{唨}{67012}
\saveTG{𠱲}{67012}
\saveTG{咆}{67012}
\saveTG{𥅧}{67012}
\saveTG{𠵼}{67012}
\saveTG{𤱌}{67012}
\saveTG{𠲣}{67012}
\saveTG{𡂑}{67012}
\saveTG{}{67012}
\saveTG{𥄹}{67012}
\saveTG{晩}{67012}
\saveTG{𩐉}{67012}
\saveTG{𠽼}{67012}
\saveTG{𠱎}{67012}
\saveTG{𠵋}{67012}
\saveTG{咀}{67012}
\saveTG{𥅣}{67012}
\saveTG{晚}{67012}
\saveTG{𠳉}{67012}
\saveTG{昵}{67012}
\saveTG{𥉝}{67012}
\saveTG{𥈱}{67012}
\saveTG{䁅}{67012}
\saveTG{𠱺}{67013}
\saveTG{嚵}{67013}
\saveTG{𠶔}{67014}
\saveTG{𠴾}{67014}
\saveTG{𠱱}{67014}
\saveTG{喔}{67014}
\saveTG{𠱅}{67014}
\saveTG{𥊛}{67014}
\saveTG{𠻾}{67014}
\saveTG{𪢅}{67014}
\saveTG{𥉸}{67014}
\saveTG{𣅴}{67014}
\saveTG{𠼤}{67014}
\saveTG{𠴐}{67014}
\saveTG{曜}{67015}
\saveTG{矅}{67015}
\saveTG{嚁}{67015}
\saveTG{𠾐}{67015}
\saveTG{𠷶}{67015}
\saveTG{𣊏}{67016}
\saveTG{𠾼}{67017}
\saveTG{𤱯}{67017}
\saveTG{𠲅}{67017}
\saveTG{吧}{67017}
\saveTG{𪡶}{67017}
\saveTG{䂁}{67017}
\saveTG{眤}{67017}
\saveTG{𥋝}{67017}
\saveTG{𠯇}{67017}
\saveTG{𠯎}{67017}
\saveTG{㕨}{67017}
\saveTG{𠾁}{67017}
\saveTG{𣅗}{67017}
\saveTG{𥃵}{67017}
\saveTG{𡁅}{67017}
\saveTG{𠲨}{67017}
\saveTG{𪚩}{67017}
\saveTG{𥉗}{67017}
\saveTG{𠳿}{67017}
\saveTG{𠼓}{67017}
\saveTG{𥅺}{67017}
\saveTG{𥃶}{67017}
\saveTG{𣆜}{67017}
\saveTG{𣆡}{67017}
\saveTG{𪱔}{67017}
\saveTG{𪾢}{67017}
\saveTG{𡆙}{67017}
\saveTG{𡅠}{67017}
\saveTG{𪡭}{67017}
\saveTG{𠴶}{67017}
\saveTG{𠲶}{67017}
\saveTG{𡅛}{67017}
\saveTG{𠲙}{67017}
\saveTG{𠯟}{67017}
\saveTG{𠹆}{67017}
\saveTG{𣋋}{67017}
\saveTG{𣉸}{67017}
\saveTG{𠷤}{67017}
\saveTG{𡅩}{67017}
\saveTG{𠰾}{67017}
\saveTG{㕴}{67017}
\saveTG{𠱓}{67017}
\saveTG{𠱍}{67017}
\saveTG{𠴙}{67017}
\saveTG{𪡴}{67017}
\saveTG{𪢄}{67018}
\saveTG{𠽵}{67018}
\saveTG{𠷍}{67019}
\saveTG{𠲫}{67020}
\saveTG{眀}{67020}
\saveTG{呁}{67020}
\saveTG{眗}{67020}
\saveTG{瞷}{67020}
\saveTG{𪠴}{67020}
\saveTG{瞯}{67020}
\saveTG{喞}{67020}
\saveTG{唧}{67020}
\saveTG{哅}{67020}
\saveTG{呴}{67020}
\saveTG{啕}{67020}
\saveTG{𠴷}{67020}
\saveTG{𠲃}{67020}
\saveTG{𠰭}{67020}
\saveTG{𠶸}{67020}
\saveTG{晌}{67020}
\saveTG{𨌞}{67020}
\saveTG{𠿷}{67020}
\saveTG{嘲}{67020}
\saveTG{叨}{67020}
\saveTG{啁}{67020}
\saveTG{嗍}{67020}
\saveTG{}{67020}
\saveTG{}{67020}
\saveTG{畇}{67020}
\saveTG{旫}{67020}
\saveTG{哃}{67020}
\saveTG{晍}{67020}
\saveTG{眮}{67020}
\saveTG{吻}{67020}
\saveTG{啲}{67020}
\saveTG{叼}{67020}
\saveTG{响}{67020}
\saveTG{叩}{67020}
\saveTG{嚪}{67020}
\saveTG{睭}{67020}
\saveTG{晭}{67020}
\saveTG{昀}{67020}
\saveTG{喲}{67020}
\saveTG{昫}{67020}
\saveTG{哟}{67020}
\saveTG{啣}{67020}
\saveTG{旳}{67020}
\saveTG{𥄸}{67020}
\saveTG{𣊗}{67020}
\saveTG{㗦}{67020}
\saveTG{矙}{67020}
\saveTG{明}{67020}
\saveTG{盷}{67020}
\saveTG{朙}{67020}
\saveTG{呞}{67020}
\saveTG{}{67020}
\saveTG{咰}{67020}
\saveTG{眴}{67020}
\saveTG{瞤}{67020}
\saveTG{昒}{67020}
\saveTG{𪡙}{67020}
\saveTG{𥄠}{67021}
\saveTG{䀔}{67021}
\saveTG{𣋆}{67021}
\saveTG{𠾽}{67021}
\saveTG{𡀊}{67021}
\saveTG{𡃦}{67021}
\saveTG{𠵜}{67021}
\saveTG{𡂛}{67021}
\saveTG{𠺉}{67021}
\saveTG{䀙}{67021}
\saveTG{𣅉}{67021}
\saveTG{𠯻}{67021}
\saveTG{𥉮}{67021}
\saveTG{𣊿}{67021}
\saveTG{㗅}{67021}
\saveTG{𠯲}{67021}
\saveTG{㗴}{67021}
\saveTG{𠵘}{67021}
\saveTG{𥆧}{67021}
\saveTG{𪰰}{67021}
\saveTG{𤰾}{67021}
\saveTG{𠻴}{67021}
\saveTG{𪰣}{67021}
\saveTG{𠰄}{67021}
\saveTG{𤰩}{67021}
\saveTG{呁}{67021}
\saveTG{盷}{67021}
\saveTG{𥄛}{67021}
\saveTG{𠯄}{67021}
\saveTG{𥉾}{67022}
\saveTG{疁}{67022}
\saveTG{㬔}{67022}
\saveTG{𡆏}{67022}
\saveTG{㗿}{67022}
\saveTG{𪡕}{67022}
\saveTG{䀛}{67022}
\saveTG{𥈌}{67022}
\saveTG{𤰿}{67022}
\saveTG{𪡛}{67022}
\saveTG{嘐}{67022}
\saveTG{嘢}{67022}
\saveTG{𠮭}{67023}
\saveTG{𥅬}{67023}
\saveTG{𠯜}{67023}
\saveTG{𪾠}{67023}
\saveTG{𠹕}{67023}
\saveTG{𣌙}{67024}
\saveTG{㘎}{67024}
\saveTG{㘚}{67024}
\saveTG{𪾽}{67024}
\saveTG{𠽫}{67024}
\saveTG{𠻻}{67024}
\saveTG{𤱆}{67024}
\saveTG{𡁤}{67024}
\saveTG{𥄞}{67024}
\saveTG{𡄢}{67024}
\saveTG{𠰩}{67024}
\saveTG{𠺔}{67024}
\saveTG{𡁡}{67024}
\saveTG{𠼋}{67026}
\saveTG{㽛}{67026}
\saveTG{𨋮}{67026}
\saveTG{㫬}{67026}
\saveTG{𠾠}{67026}
\saveTG{𠵑}{67026}
\saveTG{𠴁}{67026}
\saveTG{啁}{67026}
\saveTG{𪡡}{67026}
\saveTG{𣊺}{67026}
\saveTG{𥄶}{67026}
\saveTG{𤱬}{67026}
\saveTG{𪀟}{67027}
\saveTG{𪆌}{67027}
\saveTG{𫛗}{67027}
\saveTG{瞲}{67027}
\saveTG{𥇯}{67027}
\saveTG{𥇳}{67027}
\saveTG{𣉵}{67027}
\saveTG{㫶}{67027}
\saveTG{𠵹}{67027}
\saveTG{㗙}{67027}
\saveTG{鴫}{67027}
\saveTG{曏}{67027}
\saveTG{哪}{67027}
\saveTG{鳴}{67027}
\saveTG{鸣}{67027}
\saveTG{吗}{67027}
\saveTG{啷}{67027}
\saveTG{嗗}{67027}
\saveTG{唃}{67027}
\saveTG{嘟}{67027}
\saveTG{眵}{67027}
\saveTG{𫛐}{67027}
\saveTG{𤲹}{67027}
\saveTG{𤲮}{67027}
\saveTG{𠱳}{67027}
\saveTG{𠿋}{67027}
\saveTG{𠺁}{67027}
\saveTG{𠱣}{67027}
\saveTG{嚪}{67027}
\saveTG{𡀡}{67027}
\saveTG{㖻}{67027}
\saveTG{𥌸}{67027}
\saveTG{𥉧}{67027}
\saveTG{𡀭}{67027}
\saveTG{㗈}{67027}
\saveTG{𪰵}{67027}
\saveTG{𠴂}{67027}
\saveTG{𥉈}{67027}
\saveTG{𥅹}{67027}
\saveTG{𥊦}{67027}
\saveTG{𠴵}{67027}
\saveTG{𡅭}{67027}
\saveTG{㕼}{67027}
\saveTG{哅}{67027}
\saveTG{𠲾}{67027}
\saveTG{㬑}{67027}
\saveTG{𪄼}{67027}
\saveTG{𡄠}{67027}
\saveTG{𪅺}{67027}
\saveTG{𡀴}{67027}
\saveTG{𠼇}{67027}
\saveTG{𠿭}{67027}
\saveTG{𪄨}{67027}
\saveTG{𡅂}{67027}
\saveTG{𣉃}{67027}
\saveTG{䳟}{67027}
\saveTG{𥇩}{67027}
\saveTG{𥉄}{67027}
\saveTG{𪃦}{67027}
\saveTG{𪄊}{67027}
\saveTG{𨎎}{67027}
\saveTG{㗥}{67027}
\saveTG{𠺶}{67027}
\saveTG{𡂱}{67027}
\saveTG{𠯹}{67027}
\saveTG{𠹴}{67027}
\saveTG{𡂸}{67027}
\saveTG{𠼪}{67027}
\saveTG{哆}{67027}
\saveTG{𠻑}{67027}
\saveTG{𠲄}{67027}
\saveTG{𡁜}{67027}
\saveTG{𡂰}{67027}
\saveTG{𠮨}{67027}
\saveTG{𠯷}{67027}
\saveTG{𠳐}{67027}
\saveTG{𠳀}{67027}
\saveTG{𨙫}{67027}
\saveTG{㖿}{67027}
\saveTG{𣅅}{67027}
\saveTG{䁨}{67027}
\saveTG{𥆌}{67027}
\saveTG{瞩}{67027}
\saveTG{𥃥}{67027}
\saveTG{𥄋}{67027}
\saveTG{𥈓}{67027}
\saveTG{𠮤}{67027}
\saveTG{嘱}{67027}
\saveTG{瞗}{67027}
\saveTG{矚}{67027}
\saveTG{曯}{67027}
\saveTG{囑}{67027}
\saveTG{噊}{67027}
\saveTG{喐}{67027}
\saveTG{旸}{67027}
\saveTG{嗚}{67027}
\saveTG{呜}{67027}
\saveTG{喎}{67027}
\saveTG{𠹥}{67027}
\saveTG{}{67027}
\saveTG{䁡}{67028}
\saveTG{𣊱}{67028}
\saveTG{𪢌}{67029}
\saveTG{𥊺}{67029}
\saveTG{㽤}{67029}
\saveTG{𥌻}{67029}
\saveTG{㘓}{67029}
\saveTG{𠸽}{67031}
\saveTG{𠹏}{67031}
\saveTG{𣉔}{67031}
\saveTG{𠺱}{67031}
\saveTG{𣇡}{67031}
\saveTG{𡁱}{67031}
\saveTG{𠴍}{67032}
\saveTG{𠹖}{67032}
\saveTG{唿}{67032}
\saveTG{喙}{67032}
\saveTG{嗵}{67032}
\saveTG{𥆾}{67032}
\saveTG{眼}{67032}
\saveTG{𥉕}{67032}
\saveTG{哏}{67032}
\saveTG{𥇰}{67032}
\saveTG{𥉲}{67032}
\saveTG{𥉤}{67032}
\saveTG{𠿵}{67032}
\saveTG{𡄄}{67032}
\saveTG{㗻}{67032}
\saveTG{𡀮}{67032}
\saveTG{𠰿}{67032}
\saveTG{𥈝}{67032}
\saveTG{𣈬}{67032}
\saveTG{𥈥}{67032}
\saveTG{𡁬}{67032}
\saveTG{𠺙}{67033}
\saveTG{𡄩}{67033}
\saveTG{昸}{67033}
\saveTG{咚}{67033}
\saveTG{𤱞}{67033}
\saveTG{𣇤}{67033}
\saveTG{𠿔}{67033}
\saveTG{𡂙}{67035}
\saveTG{𥊒}{67035}
\saveTG{𣊘}{67036}
\saveTG{𠽐}{67036}
\saveTG{𥌐}{67036}
\saveTG{䁩}{67036}
\saveTG{𪢖}{67037}
\saveTG{𠶐}{67037}
\saveTG{𥈲}{67037}
\saveTG{㗓}{67037}
\saveTG{喼}{67037}
\saveTG{呶}{67040}
\saveTG{叹}{67040}
\saveTG{𠳾}{67041}
\saveTG{𠲳}{67041}
\saveTG{𥋉}{67042}
\saveTG{𠺏}{67042}
\saveTG{𡄔}{67042}
\saveTG{𠶩}{67042}
\saveTG{𥋂}{67042}
\saveTG{𠳘}{67042}
\saveTG{𥄗}{67043}
\saveTG{𡁷}{67043}
\saveTG{𤳚}{67043}
\saveTG{𠷲}{67043}
\saveTG{䀑}{67043}
\saveTG{㖊}{67043}
\saveTG{嘤}{67044}
\saveTG{𡀑}{67044}
\saveTG{𣆔}{67044}
\saveTG{𡁕}{67044}
\saveTG{𡀿}{67044}
\saveTG{𠰯}{67045}
\saveTG{噚}{67046}
\saveTG{𥃫}{67047}
\saveTG{𥄫}{67047}
\saveTG{㖩}{67047}
\saveTG{𠼶}{67047}
\saveTG{𠴫}{67047}
\saveTG{𠴪}{67047}
\saveTG{𠳖}{67047}
\saveTG{𤰖}{67047}
\saveTG{𨋃}{67047}
\saveTG{𪽜}{67047}
\saveTG{𥆷}{67047}
\saveTG{𥈨}{67047}
\saveTG{𪰴}{67047}
\saveTG{𣋙}{67047}
\saveTG{𣉜}{67047}
\saveTG{𣉥}{67047}
\saveTG{𣆲}{67047}
\saveTG{𠲚}{67047}
\saveTG{𠱆}{67047}
\saveTG{𠿊}{67047}
\saveTG{𡁒}{67047}
\saveTG{𠿍}{67047}
\saveTG{𠺽}{67047}
\saveTG{𠻍}{67047}
\saveTG{𠲴}{67047}
\saveTG{𠲡}{67047}
\saveTG{𡄆}{67047}
\saveTG{𠸬}{67047}
\saveTG{𠺻}{67047}
\saveTG{𥆉}{67047}
\saveTG{𠾷}{67047}
\saveTG{𣇽}{67047}
\saveTG{𪱏}{67047}
\saveTG{睱}{67047}
\saveTG{㗇}{67047}
\saveTG{㖬}{67047}
\saveTG{𠽙}{67047}
\saveTG{𤱕}{67047}
\saveTG{啜}{67047}
\saveTG{吺}{67047}
\saveTG{吇}{67047}
\saveTG{暇}{67047}
\saveTG{眠}{67047}
\saveTG{呡}{67047}
\saveTG{唚}{67047}
\saveTG{嗖}{67047}
\saveTG{瞍}{67047}
\saveTG{吸}{67047}
\saveTG{矎}{67047}
\saveTG{畷}{67047}
\saveTG{𥉟}{67047}
\saveTG{昅}{67047}
\saveTG{噿}{67048}
\saveTG{𠸀}{67048}
\saveTG{呣}{67050}
\saveTG{𠹙}{67052}
\saveTG{𠿇}{67052}
\saveTG{𠸎}{67052}
\saveTG{𠾆}{67052}
\saveTG{睴}{67052}
\saveTG{暉}{67052}
\saveTG{喗}{67052}
\saveTG{䀱}{67054}
\saveTG{㖓}{67054}
\saveTG{𣇔}{67054}
\saveTG{晖}{67054}
\saveTG{𠱡}{67055}
\saveTG{𣅧}{67055}
\saveTG{𠯍}{67055}
\saveTG{𪰠}{67057}
\saveTG{𠲓}{67057}
\saveTG{睁}{67057}
\saveTG{𠹄}{67060}
\saveTG{𠻙}{67060}
\saveTG{㫟}{67061}
\saveTG{瞻}{67061}
\saveTG{𥋯}{67061}
\saveTG{噡}{67061}
\saveTG{𠿌}{67061}
\saveTG{𠾈}{67061}
\saveTG{噜}{67061}
\saveTG{曕}{67061}
\saveTG{𤱠}{67062}
\saveTG{𥉳}{67062}
\saveTG{𡄇}{67062}
\saveTG{𠶕}{67062}
\saveTG{㗩}{67062}
\saveTG{𠺕}{67062}
\saveTG{𠶅}{67062}
\saveTG{𡂆}{67062}
\saveTG{𠰉}{67062}
\saveTG{𠾓}{67062}
\saveTG{𠳪}{67062}
\saveTG{㫥}{67062}
\saveTG{𠱷}{67062}
\saveTG{眧}{67062}
\saveTG{眳}{67062}
\saveTG{昭}{67062}
\saveTG{𥌧}{67063}
\saveTG{𣋼}{67063}
\saveTG{嚕}{67063}
\saveTG{䀩}{67064}
\saveTG{𠻐}{67064}
\saveTG{𠸪}{67064}
\saveTG{𡀔}{67064}
\saveTG{𠸧}{67064}
\saveTG{𠺝}{67064}
\saveTG{䁊}{67064}
\saveTG{𥇄}{67064}
\saveTG{𡄵}{67064}
\saveTG{𠼣}{67064}
\saveTG{𡂜}{67064}
\saveTG{㬆}{67064}
\saveTG{㗃}{67064}
\saveTG{咯}{67064}
\saveTG{啹}{67064}
\saveTG{略}{67064}
\saveTG{𥇗}{67065}
\saveTG{睸}{67067}
\saveTG{𣈲}{67067}
\saveTG{𣇉}{67067}
\saveTG{𠲰}{67067}
\saveTG{𠷯}{67067}
\saveTG{𠳞}{67070}
\saveTG{𠱃}{67070}
\saveTG{𠲼}{67070}
\saveTG{㖤}{67072}
\saveTG{𥇣}{67072}
\saveTG{𡆟}{67072}
\saveTG{𥆲}{67072}
\saveTG{𥇤}{67072}
\saveTG{𥇵}{67072}
\saveTG{啒}{67072}
\saveTG{囓}{67072}
\saveTG{𡁄}{67073}
\saveTG{𥇌}{67077}
\saveTG{𠽏}{67077}
\saveTG{𠱂}{67077}
\saveTG{𥊥}{67077}
\saveTG{𠽣}{67077}
\saveTG{𥅑}{67077}
\saveTG{啗}{67077}
\saveTG{畂}{67080}
\saveTG{瞑}{67080}
\saveTG{暝}{67080}
\saveTG{𠰊}{67081}
\saveTG{𣋱}{67081}
\saveTG{㖵}{67081}
\saveTG{𡁎}{67081}
\saveTG{𡁝}{67081}
\saveTG{噀}{67081}
\saveTG{𠸃}{67081}
\saveTG{𡃳}{67081}
\saveTG{𠺡}{67082}
\saveTG{㘈}{67082}
\saveTG{𠲭}{67082}
\saveTG{𠸰}{67082}
\saveTG{𠿮}{67082}
\saveTG{𡄥}{67082}
\saveTG{𠿣}{67082}
\saveTG{𠾬}{67082}
\saveTG{𡁍}{67082}
\saveTG{𠺼}{67082}
\saveTG{㗵}{67082}
\saveTG{𡁌}{67082}
\saveTG{𠿒}{67082}
\saveTG{𡃧}{67082}
\saveTG{𡁀}{67082}
\saveTG{𠾗}{67082}
\saveTG{𡂿}{67082}
\saveTG{𠿁}{67082}
\saveTG{𣊞}{67082}
\saveTG{𣋸}{67082}
\saveTG{𥋵}{67082}
\saveTG{𥊡}{67082}
\saveTG{𠺖}{67082}
\saveTG{𪾪}{67082}
\saveTG{𪾯}{67082}
\saveTG{𤱐}{67082}
\saveTG{𥍌}{67082}
\saveTG{𪢐}{67082}
\saveTG{𠮻}{67082}
\saveTG{呗}{67082}
\saveTG{吹}{67082}
\saveTG{噷}{67082}
\saveTG{欥}{67082}
\saveTG{嗽}{67082}
\saveTG{欭}{67082}
\saveTG{}{67082}
\saveTG{𣋉}{67084}
\saveTG{𠸇}{67084}
\saveTG{𪡃}{67084}
\saveTG{㬋}{67084}
\saveTG{㗋}{67084}
\saveTG{噢}{67084}
\saveTG{喫}{67084}
\saveTG{喉}{67084}
\saveTG{睺}{67084}
\saveTG{唤}{67084}
\saveTG{喚}{67084}
\saveTG{𥈉}{67084}
\saveTG{㬇}{67085}
\saveTG{𥌶}{67086}
\saveTG{𥊫}{67086}
\saveTG{𥌔}{67086}
\saveTG{㘋}{67086}
\saveTG{𡃤}{67086}
\saveTG{𠾸}{67086}
\saveTG{𥋻}{67086}
\saveTG{𠺬}{67087}
\saveTG{呎}{67087}
\saveTG{𠶨}{67089}
\saveTG{𣇧}{67089}
\saveTG{𡄺}{67089}
\saveTG{𥌝}{67089}
\saveTG{𠴌}{67089}
\saveTG{暩}{67091}
\saveTG{㗫}{67091}
\saveTG{𥉻}{67091}
\saveTG{𠵸}{67092}
\saveTG{𠺰}{67092}
\saveTG{𪡇}{67092}
\saveTG{𥇎}{67092}
\saveTG{𥅘}{67092}
\saveTG{𠲋}{67092}
\saveTG{噄}{67093}
\saveTG{𥊯}{67093}
\saveTG{𡁵}{67093}
\saveTG{𥉉}{67094}
\saveTG{𠳸}{67094}
\saveTG{𪾴}{67094}
\saveTG{嗓}{67094}
\saveTG{哚}{67094}
\saveTG{嘄}{67094}
\saveTG{㘀}{67094}
\saveTG{𪡨}{67094}
\saveTG{𠽄}{67094}
\saveTG{𠶀}{67094}
\saveTG{𠽉}{67094}
\saveTG{𠺧}{67094}
\saveTG{𣉕}{67094}
\saveTG{𠹠}{67094}
\saveTG{𤱧}{67094}
\saveTG{𥉼}{67094}
\saveTG{𡂢}{67094}
\saveTG{㽥}{67094}
\saveTG{𡀐}{67094}
\saveTG{㖨}{67099}
\saveTG{㫽}{67099}
\saveTG{𡂎}{67099}
\saveTG{睩}{67099}
\saveTG{嚟}{67099}
\saveTG{𥌛}{67099}
\saveTG{𨌠}{67099}
\saveTG{𠴀}{67102}
\saveTG{𥁴}{67102}
\saveTG{盟}{67102}
\saveTG{曌}{67102}
\saveTG{𧖽}{67102}
\saveTG{𥂗}{67102}
\saveTG{琞}{67103}
\saveTG{𤧩}{67104}
\saveTG{𡊙}{67104}
\saveTG{𡌛}{67104}
\saveTG{𤦼}{67104}
\saveTG{𣈂}{67104}
\saveTG{㙒}{67104}
\saveTG{墅}{67104}
\saveTG{𧿅}{67110}
\saveTG{𨅰}{67112}
\saveTG{𨃎}{67112}
\saveTG{𨇫}{67112}
\saveTG{跑}{67112}
\saveTG{跪}{67112}
\saveTG{𧿏}{67112}
\saveTG{𫏝}{67112}
\saveTG{跙}{67112}
\saveTG{𨂀}{67112}
\saveTG{跜}{67112}
\saveTG{𨀧}{67113}
\saveTG{𨂓}{67114}
\saveTG{𨀃}{67114}
\saveTG{𧿔}{67114}
\saveTG{䠎}{67114}
\saveTG{躍}{67115}
\saveTG{𨇴}{67117}
\saveTG{𨁩}{67117}
\saveTG{跁}{67117}
\saveTG{𧿆}{67117}
\saveTG{𪛅}{67117}
\saveTG{𧾾}{67117}
\saveTG{𨀓}{67117}
\saveTG{𧿻}{67117}
\saveTG{𨁱}{67117}
\saveTG{𨤤}{67117}
\saveTG{𨇩}{67117}
\saveTG{𨜖}{67117}
\saveTG{𨁙}{67117}
\saveTG{跔}{67120}
\saveTG{𨇲}{67120}
\saveTG{趵}{67120}
\saveTG{踋}{67120}
\saveTG{踘}{67120}
\saveTG{躙}{67120}
\saveTG{躢}{67120}
\saveTG{跀}{67120}
\saveTG{𧎙}{67120}
\saveTG{躝}{67120}
\saveTG{𨅉}{67121}
\saveTG{𨂃}{67121}
\saveTG{䠒}{67121}
\saveTG{𤴉}{67121}
\saveTG{𨆀}{67121}
\saveTG{𦡉}{67121}
\saveTG{䟙}{67121}
\saveTG{𨅹}{67121}
\saveTG{野}{67122}
\saveTG{蹘}{67122}
\saveTG{𨆿}{67124}
\saveTG{𨀍}{67124}
\saveTG{𨅋}{67124}
\saveTG{𨆇}{67126}
\saveTG{𨅍}{67126}
\saveTG{𨁴}{67126}
\saveTG{𨄑}{67126}
\saveTG{𨀜}{67126}
\saveTG{𨀴}{67126}
\saveTG{𨂊}{67126}
\saveTG{𨂹}{67127}
\saveTG{𨜵}{67127}
\saveTG{𨃴}{67127}
\saveTG{𨄋}{67127}
\saveTG{𨂆}{67127}
\saveTG{鸀}{67127}
\saveTG{𨝪}{67127}
\saveTG{𨄼}{67127}
\saveTG{𨞚}{67127}
\saveTG{𨛋}{67127}
\saveTG{𨃒}{67127}
\saveTG{𨀯}{67127}
\saveTG{𨃘}{67127}
\saveTG{𨇵}{67127}
\saveTG{𨀩}{67127}
\saveTG{𨁳}{67127}
\saveTG{𨁌}{67127}
\saveTG{䠬}{67127}
\saveTG{𨄂}{67127}
\saveTG{𨞱}{67127}
\saveTG{𨝾}{67127}
\saveTG{𨞕}{67127}
\saveTG{𪄚}{67127}
\saveTG{𧾿}{67127}
\saveTG{𨅮}{67127}
\saveTG{𨄙}{67127}
\saveTG{䴑}{67127}
\saveTG{𪈦}{67127}
\saveTG{躑}{67127}
\saveTG{踯}{67127}
\saveTG{踴}{67127}
\saveTG{踊}{67127}
\saveTG{鹭}{67127}
\saveTG{蹫}{67127}
\saveTG{跼}{67127}
\saveTG{踻}{67127}
\saveTG{鴠}{67127}
\saveTG{跢}{67127}
\saveTG{郢}{67127}
\saveTG{𨅡}{67127}
\saveTG{𨅵}{67127}
\saveTG{𨇧}{67127}
\saveTG{𨇝}{67127}
\saveTG{𨅛}{67127}
\saveTG{𪆽}{67127}
\saveTG{䠱}{67127}
\saveTG{𨀇}{67127}
\saveTG{䠖}{67128}
\saveTG{𨅽}{67129}
\saveTG{踘}{67129}
\saveTG{𨃣}{67131}
\saveTG{𧕌}{67131}
\saveTG{跽}{67131}
\saveTG{蹍}{67132}
\saveTG{𫏢}{67132}
\saveTG{𨂦}{67132}
\saveTG{𫏡}{67132}
\saveTG{䠫}{67132}
\saveTG{𨄶}{67132}
\saveTG{𨆔}{67132}
\saveTG{跟}{67132}
\saveTG{𨂴}{67132}
\saveTG{𨄿}{67133}
\saveTG{蹆}{67133}
\saveTG{𨀐}{67133}
\saveTG{𧐓}{67136}
\saveTG{𫙣}{67136}
\saveTG{𧐯}{67136}
\saveTG{𧓃}{67136}
\saveTG{𧒍}{67136}
\saveTG{𨃬}{67137}
\saveTG{踙}{67140}
\saveTG{跚}{67140}
\saveTG{踧}{67140}
\saveTG{𧿎}{67141}
\saveTG{蹡}{67142}
\saveTG{𨂤}{67142}
\saveTG{䠥}{67142}
\saveTG{蹡}{67143}
\saveTG{𧿜}{67143}
\saveTG{䟸}{67144}
\saveTG{𨀔}{67144}
\saveTG{𨂲}{67144}
\saveTG{𨃿}{67144}
\saveTG{趿}{67147}
\saveTG{䟾}{67147}
\saveTG{䟨}{67147}
\saveTG{䠍}{67147}
\saveTG{𨃏}{67147}
\saveTG{𪵉}{67147}
\saveTG{𫏗}{67147}
\saveTG{𨀝}{67147}
\saveTG{𧿸}{67147}
\saveTG{䟝}{67147}
\saveTG{𨀊}{67147}
\saveTG{䟕}{67147}
\saveTG{𨃟}{67147}
\saveTG{𠺸}{67147}
\saveTG{𨃜}{67149}
\saveTG{}{67150}
\saveTG{跭}{67154}
\saveTG{𨀢}{67155}
\saveTG{𨂱}{67156}
\saveTG{𧿹}{67157}
\saveTG{踭}{67157}
\saveTG{𨅧}{67161}
\saveTG{䠨}{67161}
\saveTG{𨄌}{67162}
\saveTG{蹓}{67162}
\saveTG{𫏉}{67162}
\saveTG{𨃶}{67164}
\saveTG{路}{67164}
\saveTG{䠧}{67164}
\saveTG{踞}{67164}
\saveTG{䠇}{67172}
\saveTG{𨄽}{67177}
\saveTG{𨁺}{67181}
\saveTG{𨂳}{67181}
\saveTG{䠣}{67181}
\saveTG{𣣆}{67182}
\saveTG{𠿖}{67182}
\saveTG{𨁰}{67182}
\saveTG{歜}{67182}
\saveTG{𨅄}{67182}
\saveTG{𧿞}{67182}
\saveTG{𨀥}{67182}
\saveTG{𨇮}{67182}
\saveTG{𣤞}{67182}
\saveTG{𨂰}{67184}
\saveTG{𨂸}{67184}
\saveTG{躀}{67186}
\saveTG{𨇚}{67188}
\saveTG{𨂂}{67189}
\saveTG{𨁣}{67189}
\saveTG{𨄊}{67191}
\saveTG{䟢}{67192}
\saveTG{𨤧}{67192}
\saveTG{跺}{67194}
\saveTG{跥}{67194}
\saveTG{蹂}{67194}
\saveTG{𨄛}{67199}
\saveTG{䠈}{67199}
\saveTG{䟿}{67199}
\saveTG{𣈟}{67210}
\saveTG{𩘍}{67210}
\saveTG{䬗}{67210}
\saveTG{𧠐}{67212}
\saveTG{𧡕}{67214}
\saveTG{覎}{67214}
\saveTG{𩁐}{67215}
\saveTG{𧡜}{67217}
\saveTG{𧡷}{67217}
\saveTG{𦫩}{67217}
\saveTG{𧢊}{67218}
\saveTG{卾}{67220}
\saveTG{翤}{67220}
\saveTG{嗣}{67220}
\saveTG{𦐽}{67221}
\saveTG{𦑉}{67221}
\saveTG{𧢑}{67221}
\saveTG{𡀒}{67222}
\saveTG{嗣}{67226}
\saveTG{鹃}{67227}
\saveTG{鵑}{67227}
\saveTG{鶍}{67227}
\saveTG{鸮}{67227}
\saveTG{𨚙}{67227}
\saveTG{𨝿}{67227}
\saveTG{𨛡}{67227}
\saveTG{𪉐}{67227}
\saveTG{𫑤}{67227}
\saveTG{鷐}{67227}
\saveTG{𡗏}{67227}
\saveTG{𫜀}{67227}
\saveTG{𪈱}{67227}
\saveTG{𪃌}{67227}
\saveTG{𪃍}{67227}
\saveTG{𪈀}{67227}
\saveTG{𪁼}{67227}
\saveTG{𢄾}{67227}
\saveTG{鄂}{67227}
\saveTG{鹗}{67227}
\saveTG{鴞}{67227}
\saveTG{鶚}{67227}
\saveTG{鸜}{67227}
\saveTG{𣵯}{67232}
\saveTG{𣪆}{67247}
\saveTG{𣪓}{67247}
\saveTG{𣪉}{67247}
\saveTG{𨟠}{67257}
\saveTG{𠮩}{67270}
\saveTG{𠤁}{67277}
\saveTG{䟳}{67302}
\saveTG{𪐵}{67312}
\saveTG{𪑎}{67312}
\saveTG{𪑺}{67314}
\saveTG{𪑱}{67314}
\saveTG{𪓄}{67317}
\saveTG{䵥}{67317}
\saveTG{𪐨}{67317}
\saveTG{𪑍}{67317}
\saveTG{𪒣}{67317}
\saveTG{𪐘}{67317}
\saveTG{𪐼}{67317}
\saveTG{𪑲}{67317}
\saveTG{䵴}{67317}
\saveTG{𪑣}{67317}
\saveTG{𣊤}{67321}
\saveTG{𪐧}{67321}
\saveTG{𪐛}{67321}
\saveTG{𪐯}{67323}
\saveTG{䵠}{67323}
\saveTG{黟}{67327}
\saveTG{鷺}{67327}
\saveTG{𫘘}{67327}
\saveTG{𩣶}{67327}
\saveTG{𪅅}{67327}
\saveTG{𪂡}{67327}
\saveTG{𨜐}{67327}
\saveTG{𪆤}{67327}
\saveTG{𪃄}{67327}
\saveTG{𠲹}{67331}
\saveTG{𢜏}{67332}
\saveTG{煦}{67332}
\saveTG{喣}{67332}
\saveTG{𤈌}{67332}
\saveTG{𢙟}{67332}
\saveTG{𫜚}{67332}
\saveTG{𥈈}{67332}
\saveTG{𪹕}{67332}
\saveTG{𢢤}{67334}
\saveTG{𢢨}{67335}
\saveTG{照}{67336}
\saveTG{𤒭}{67338}
\saveTG{𪑃}{67344}
\saveTG{𪑰}{67344}
\saveTG{𪒮}{67347}
\saveTG{𪐣}{67347}
\saveTG{𠮊}{67347}
\saveTG{𥋹}{67347}
\saveTG{𪐮}{67347}
\saveTG{𪒋}{67347}
\saveTG{𪐿}{67357}
\saveTG{黵}{67361}
\saveTG{𪒧}{67361}
\saveTG{𪒂}{67368}
\saveTG{𪑁}{67374}
\saveTG{𪒄}{67380}
\saveTG{𪑩}{67381}
\saveTG{㱄}{67382}
\saveTG{𣣦}{67382}
\saveTG{𣢱}{67382}
\saveTG{㱎}{67382}
\saveTG{歞}{67382}
\saveTG{𪑻}{67384}
\saveTG{𪒗}{67394}
\saveTG{𪑶}{67394}
\saveTG{𪑔}{67399}
\saveTG{𡥕}{67401}
\saveTG{𦦿}{67417}
\saveTG{𪱕}{67421}
\saveTG{𦒡}{67421}
\saveTG{䳛}{67427}
\saveTG{𪄍}{67427}
\saveTG{𨞍}{67427}
\saveTG{鷃}{67427}
\saveTG{鸚}{67427}
\saveTG{鸅}{67427}
\saveTG{鄤}{67427}
\saveTG{𨛎}{67427}
\saveTG{𫑦}{67427}
\saveTG{𨟙}{67427}
\saveTG{𪈴}{67427}
\saveTG{䳚}{67427}
\saveTG{䣤}{67427}
\saveTG{𨚰}{67427}
\saveTG{𠬱}{67447}
\saveTG{𣤵}{67482}
\saveTG{歝}{67482}
\saveTG{𤜔}{67502}
\saveTG{㨼}{67502}
\saveTG{𢯲}{67502}
\saveTG{𪓽}{67517}
\saveTG{翈}{67520}
\saveTG{鷤}{67527}
\saveTG{鷝}{67527}
\saveTG{鄲}{67527}
\saveTG{鸭}{67527}
\saveTG{鴨}{67527}
\saveTG{䣞}{67527}
\saveTG{𨞎}{67527}
\saveTG{囅}{67532}
\saveTG{𦪣}{67544}
\saveTG{𤱋}{67544}
\saveTG{𥀕}{67547}
\saveTG{𤲰}{67562}
\saveTG{㿢}{67602}
\saveTG{𣋐}{67604}
\saveTG{𥉦}{67604}
\saveTG{𪡉}{67621}
\saveTG{𩩪}{67627}
\saveTG{𣊧}{67627}
\saveTG{𪁳}{67627}
\saveTG{郘}{67627}
\saveTG{𫑣}{67627}
\saveTG{𨞽}{67627}
\saveTG{𪂇}{67627}
\saveTG{鸓}{67627}
\saveTG{鄙}{67627}
\saveTG{𪽈}{67627}
\saveTG{䁕}{67664}
\saveTG{𣣘}{67682}
\saveTG{𤭲}{67717}
\saveTG{𪓮}{67717}
\saveTG{翾}{67720}
\saveTG{𦒬}{67721}
\saveTG{𫛂}{67727}
\saveTG{鹖}{67727}
\saveTG{鵾}{67727}
\saveTG{鹍}{67727}
\saveTG{鹮}{67727}
\saveTG{鶡}{67727}
\saveTG{䴉}{67727}
\saveTG{𨛗}{67727}
\saveTG{𫑡}{67727}
\saveTG{䬭}{67732}
\saveTG{𣍊}{67733}
\saveTG{歇}{67782}
\saveTG{𢁈}{67801}
\saveTG{鶗}{67802}
\saveTG{𣇴}{67802}
\saveTG{𧤘}{67802}
\saveTG{𦑡}{67802}
\saveTG{𪥒}{67804}
\saveTG{𧷥}{67806}
\saveTG{𤋚}{67809}
\saveTG{𤈳}{67809}
\saveTG{𤋜}{67809}
\saveTG{𤋗}{67809}
\saveTG{焽}{67809}
\saveTG{𤉵}{67809}
\saveTG{焁}{67809}
\saveTG{𤍓}{67809}
\saveTG{賉}{67812}
\saveTG{𧸭}{67815}
\saveTG{𦫮}{67817}
\saveTG{𧵢}{67817}
\saveTG{𤓅}{67817}
\saveTG{𧵅}{67817}
\saveTG{𧵥}{67817}
\saveTG{賙}{67820}
\saveTG{𧵂}{67821}
\saveTG{䝧}{67821}
\saveTG{𧴬}{67822}
\saveTG{賿}{67822}
\saveTG{𧵈}{67823}
\saveTG{𧵣}{67826}
\saveTG{䝭}{67826}
\saveTG{𧹇}{67827}
\saveTG{}{67827}
\saveTG{𫛫}{67827}
\saveTG{𩿦}{67827}
\saveTG{䴍}{67827}
\saveTG{𫚾}{67827}
\saveTG{𧷾}{67827}
\saveTG{鶰}{67827}
\saveTG{郥}{67827}
\saveTG{郹}{67827}
\saveTG{鵙}{67827}
\saveTG{鶪}{67827}
\saveTG{鷶}{67827}
\saveTG{郧}{67827}
\saveTG{鄖}{67827}
\saveTG{𧶳}{67827}
\saveTG{䴗}{67827}
\saveTG{𨛴}{67827}
\saveTG{𧷿}{67827}
\saveTG{䝲}{67833}
\saveTG{𧵡}{67844}
\saveTG{𧴯}{67847}
\saveTG{賯}{67847}
\saveTG{𪒸}{67847}
\saveTG{賱}{67852}
\saveTG{𧶄}{67857}
\saveTG{𧸸}{67861}
\saveTG{贍}{67861}
\saveTG{𧵓}{67862}
\saveTG{㷖}{67862}
\saveTG{賂}{67864}
\saveTG{䝻}{67864}
\saveTG{𧵫}{67877}
\saveTG{𧷉}{67882}
\saveTG{䝪}{67882}
\saveTG{㰨}{67882}
\saveTG{䞀}{67884}
\saveTG{賝}{67894}
\saveTG{䌎}{67903}
\saveTG{𥺔}{67904}
\saveTG{𩙕}{67910}
\saveTG{𡂟}{67917}
\saveTG{𣋰}{67917}
\saveTG{𦑲}{67921}
\saveTG{𠟯}{67921}
\saveTG{𪁣}{67927}
\saveTG{𪆣}{67927}
\saveTG{𪂠}{67927}
\saveTG{𪈚}{67927}
\saveTG{𪃷}{67927}
\saveTG{鄵}{67927}
\saveTG{𨜍}{67927}
\saveTG{夥}{67927}
\saveTG{䳴}{67927}
\saveTG{𪈰}{67927}
\saveTG{𪇰}{67927}
\saveTG{𨜑}{67927}
\saveTG{𣣴}{67982}
\saveTG{𣡙}{67986}
\saveTG{叺}{68000}
\saveTG{叭}{68000}
\saveTG{㽗}{68000}
\saveTG{㕥}{68000}
\saveTG{𥃱}{68000}
\saveTG{𣅁}{68000}
\saveTG{𠳎}{68007}
\saveTG{㕽}{68007}
\saveTG{𥅁}{68011}
\saveTG{咋}{68011}
\saveTG{昨}{68011}
\saveTG{𠴚}{68011}
\saveTG{呛}{68012}
\saveTG{嗟}{68012}
\saveTG{唴}{68012}
\saveTG{𥈕}{68012}
\saveTG{𡅞}{68012}
\saveTG{𡅋}{68012}
\saveTG{𠰹}{68012}
\saveTG{㖉}{68012}
\saveTG{㖹}{68012}
\saveTG{嗌}{68012}
\saveTG{𤱴}{68012}
\saveTG{䁟}{68012}
\saveTG{𠵔}{68012}
\saveTG{嚂}{68012}
\saveTG{㽨}{68012}
\saveTG{𤱝}{68012}
\saveTG{𥌈}{68012}
\saveTG{暛}{68012}
\saveTG{哾}{68012}
\saveTG{囕}{68012}
\saveTG{𣉼}{68012}
\saveTG{暆}{68012}
\saveTG{嗴}{68013}
\saveTG{䀬}{68014}
\saveTG{𠱴}{68014}
\saveTG{𪰡}{68014}
\saveTG{睉}{68014}
\saveTG{唑}{68014}
\saveTG{囖}{68015}
\saveTG{𣅠}{68017}
\saveTG{𤰢}{68017}
\saveTG{䁯}{68017}
\saveTG{盵}{68017}
\saveTG{暣}{68017}
\saveTG{吃}{68017}
\saveTG{㘛}{68017}
\saveTG{𪿄}{68017}
\saveTG{𠷇}{68017}
\saveTG{𠴻}{68017}
\saveTG{𥅓}{68017}
\saveTG{𡁾}{68017}
\saveTG{𠱕}{68017}
\saveTG{𠵰}{68017}
\saveTG{𥇉}{68017}
\saveTG{𥍖}{68017}
\saveTG{𥆟}{68017}
\saveTG{𠾔}{68017}
\saveTG{𠶹}{68017}
\saveTG{𠺪}{68017}
\saveTG{𠯏}{68017}
\saveTG{𠼳}{68017}
\saveTG{㫓}{68017}
\saveTG{噬}{68018}
\saveTG{睑}{68019}
\saveTG{唫}{68019}
\saveTG{𪡋}{68019}
\saveTG{𥇶}{68019}
\saveTG{畍}{68020}
\saveTG{𥄍}{68020}
\saveTG{𣈥}{68020}
\saveTG{𪠿}{68020}
\saveTG{𠷁}{68020}
\saveTG{𠮶}{68020}
\saveTG{𡈅}{68020}
\saveTG{吤}{68020}
\saveTG{睮}{68021}
\saveTG{喻}{68021}
\saveTG{𠱉}{68022}
\saveTG{畛}{68022}
\saveTG{昣}{68022}
\saveTG{眕}{68022}
\saveTG{𡅣}{68024}
\saveTG{𤰪}{68027}
\saveTG{軡}{68027}
\saveTG{𥅎}{68027}
\saveTG{𠯋}{68027}
\saveTG{𠴠}{68027}
\saveTG{𡆈}{68027}
\saveTG{㖒}{68027}
\saveTG{𥆽}{68027}
\saveTG{𥄯}{68027}
\saveTG{𪢉}{68027}
\saveTG{𠿰}{68027}
\saveTG{吩}{68027}
\saveTG{𡅌}{68027}
\saveTG{𤳈}{68027}
\saveTG{𠸐}{68027}
\saveTG{㬛}{68027}
\saveTG{㫻}{68027}
\saveTG{𥋼}{68027}
\saveTG{𤲕}{68027}
\saveTG{噏}{68027}
\saveTG{瞈}{68027}
\saveTG{暡}{68027}
\saveTG{嗡}{68027}
\saveTG{昑}{68027}
\saveTG{噙}{68027}
\saveTG{盻}{68027}
\saveTG{睇}{68027}
\saveTG{𥌉}{68027}
\saveTG{吟}{68027}
\saveTG{睔}{68027}
\saveTG{盼}{68027}
\saveTG{昐}{68027}
\saveTG{嗲}{68027}
\saveTG{吖}{68027}
\saveTG{𡈺}{68027}
\saveTG{𠿫}{68027}
\saveTG{𥋘}{68027}
\saveTG{𪡆}{68027}
\saveTG{𡃮}{68027}
\saveTG{𥌺}{68027}
\saveTG{㘉}{68027}
\saveTG{𡅉}{68027}
\saveTG{𠹹}{68027}
\saveTG{𠵴}{68027}
\saveTG{㖮}{68027}
\saveTG{𠴣}{68027}
\saveTG{𡄍}{68027}
\saveTG{𡀝}{68030}
\saveTG{𠴮}{68030}
\saveTG{𠽛}{68031}
\saveTG{㽧}{68031}
\saveTG{嘸}{68031}
\saveTG{瞴}{68031}
\saveTG{㗝}{68031}
\saveTG{𠷿}{68031}
\saveTG{㕬}{68031}
\saveTG{𣊲}{68031}
\saveTG{𠿏}{68032}
\saveTG{𠸌}{68032}
\saveTG{𪐲}{68032}
\saveTG{𡂊}{68032}
\saveTG{呤}{68032}
\saveTG{𪾧}{68032}
\saveTG{喰}{68032}
\saveTG{唸}{68032}
\saveTG{哙}{68032}
\saveTG{唥}{68032}
\saveTG{昤}{68032}
\saveTG{昖}{68032}
\saveTG{喩}{68032}
\saveTG{嗞}{68032}
\saveTG{𠹸}{68032}
\saveTG{𠾕}{68032}
\saveTG{𡀍}{68032}
\saveTG{𠲈}{68032}
\saveTG{𡂺}{68032}
\saveTG{𥌫}{68032}
\saveTG{𪽏}{68032}
\saveTG{𣇝}{68032}
\saveTG{𣈮}{68032}
\saveTG{𠻠}{68032}
\saveTG{𠴒}{68032}
\saveTG{𠼒}{68033}
\saveTG{𡀟}{68033}
\saveTG{唹}{68033}
\saveTG{㘂}{68033}
\saveTG{㬠}{68034}
\saveTG{𠸺}{68034}
\saveTG{𥉥}{68034}
\saveTG{𠿼}{68034}
\saveTG{𥍒}{68035}
\saveTG{𠿱}{68036}
\saveTG{𡆍}{68036}
\saveTG{嗛}{68037}
\saveTG{𡄫}{68037}
\saveTG{䁠}{68037}
\saveTG{喩}{68037}
\saveTG{𥋧}{68038}
\saveTG{瞮}{68040}
\saveTG{嗷}{68040}
\saveTG{𪾰}{68040}
\saveTG{𨊴}{68040}
\saveTG{𤲧}{68040}
\saveTG{𠾀}{68040}
\saveTG{㕮}{68040}
\saveTG{𡁛}{68040}
\saveTG{𠽊}{68040}
\saveTG{𡀎}{68040}
\saveTG{𠶣}{68040}
\saveTG{𠼍}{68040}
\saveTG{𠶂}{68040}
\saveTG{𠳚}{68040}
\saveTG{曒}{68040}
\saveTG{𠼚}{68040}
\saveTG{𠼌}{68040}
\saveTG{𡂡}{68040}
\saveTG{𪢀}{68040}
\saveTG{𠾝}{68040}
\saveTG{𠻳}{68040}
\saveTG{𠻲}{68040}
\saveTG{𠾎}{68040}
\saveTG{㬚}{68040}
\saveTG{𣊶}{68040}
\saveTG{𪰱}{68040}
\saveTG{𣊟}{68040}
\saveTG{𪱃}{68040}
\saveTG{𥄖}{68040}
\saveTG{𥋗}{68040}
\saveTG{𥄭}{68040}
\saveTG{䁶}{68040}
\saveTG{𥅪}{68040}
\saveTG{𥋔}{68040}
\saveTG{𠲠}{68040}
\saveTG{𪰋}{68040}
\saveTG{𥋆}{68040}
\saveTG{嘋}{68040}
\saveTG{旿}{68040}
\saveTG{呚}{68040}
\saveTG{暾}{68040}
\saveTG{畋}{68040}
\saveTG{𥋌}{68040}
\saveTG{矀}{68040}
\saveTG{噋}{68040}
\saveTG{𡀏}{68040}
\saveTG{噭}{68040}
\saveTG{噉}{68040}
\saveTG{瞰}{68040}
\saveTG{曔}{68040}
\saveTG{吘}{68040}
\saveTG{𣈨}{68041}
\saveTG{𥄺}{68041}
\saveTG{𠿨}{68041}
\saveTG{𡄭}{68042}
\saveTG{𥊭}{68043}
\saveTG{𠾡}{68043}
\saveTG{𤴆}{68043}
\saveTG{𣉂}{68044}
\saveTG{𡃇}{68044}
\saveTG{𠹞}{68044}
\saveTG{𣋹}{68044}
\saveTG{𠷩}{68044}
\saveTG{𡀜}{68044}
\saveTG{噂}{68046}
\saveTG{啽}{68046}
\saveTG{𡆅}{68047}
\saveTG{𠱘}{68047}
\saveTG{𡃛}{68047}
\saveTG{哖}{68050}
\saveTG{㬕}{68051}
\saveTG{𣉚}{68051}
\saveTG{𡀳}{68051}
\saveTG{咩}{68051}
\saveTG{𪢟}{68051}
\saveTG{眻}{68051}
\saveTG{𣉫}{68051}
\saveTG{嗱}{68052}
\saveTG{𪽦}{68053}
\saveTG{䂀}{68053}
\saveTG{𥋟}{68053}
\saveTG{曦}{68053}
\saveTG{𠿿}{68055}
\saveTG{㬢}{68055}
\saveTG{𡅷}{68055}
\saveTG{𠽂}{68056}
\saveTG{𨍓}{68056}
\saveTG{啴}{68056}
\saveTG{𠼺}{68056}
\saveTG{𥉪}{68057}
\saveTG{䀲}{68057}
\saveTG{𥌇}{68057}
\saveTG{𠻽}{68057}
\saveTG{𠽩}{68057}
\saveTG{𠳨}{68057}
\saveTG{嗨}{68057}
\saveTG{晦}{68057}
\saveTG{畮}{68057}
\saveTG{𣌀}{68059}
\saveTG{𪱋}{68061}
\saveTG{𠾺}{68061}
\saveTG{𪾾}{68061}
\saveTG{𥊳}{68061}
\saveTG{𪾿}{68061}
\saveTG{哈}{68061}
\saveTG{𠽾}{68061}
\saveTG{䀫}{68061}
\saveTG{𡅨}{68061}
\saveTG{𠼟}{68062}
\saveTG{晗}{68062}
\saveTG{𪢂}{68062}
\saveTG{𥆡}{68062}
\saveTG{𪢑}{68062}
\saveTG{唅}{68062}
\saveTG{𠴰}{68064}
\saveTG{𪡧}{68064}
\saveTG{㬖}{68064}
\saveTG{啥}{68064}
\saveTG{䁬}{68066}
\saveTG{𣋘}{68066}
\saveTG{㬝}{68066}
\saveTG{𡄦}{68066}
\saveTG{囎}{68066}
\saveTG{瞺}{68066}
\saveTG{噲}{68066}
\saveTG{噌}{68066}
\saveTG{㽪}{68066}
\saveTG{嗆}{68067}
\saveTG{𪰻}{68067}
\saveTG{𡅎}{68067}
\saveTG{㗳}{68068}
\saveTG{𠸘}{68068}
\saveTG{䀰}{68068}
\saveTG{𪑌}{68068}
\saveTG{唂}{68068}
\saveTG{𠾌}{68077}
\saveTG{𠯱}{68080}
\saveTG{𪢒}{68080}
\saveTG{暰}{68081}
\saveTG{瞛}{68081}
\saveTG{暶}{68081}
\saveTG{𪢕}{68082}
\saveTG{䁢}{68082}
\saveTG{𠿢}{68082}
\saveTG{𥅴}{68082}
\saveTG{㗰}{68082}
\saveTG{𠱈}{68084}
\saveTG{嗾}{68084}
\saveTG{咲}{68084}
\saveTG{眹}{68084}
\saveTG{𣉧}{68084}
\saveTG{䀢}{68084}
\saveTG{㗛}{68084}
\saveTG{𥈢}{68084}
\saveTG{𣈸}{68084}
\saveTG{𠹻}{68084}
\saveTG{𠸍}{68084}
\saveTG{𠺑}{68084}
\saveTG{𥇥}{68085}
\saveTG{𡀹}{68086}
\saveTG{𡃪}{68086}
\saveTG{嚽}{68086}
\saveTG{瞼}{68086}
\saveTG{嗿}{68086}
\saveTG{噞}{68086}
\saveTG{𥊗}{68089}
\saveTG{𤱏}{68090}
\saveTG{𠰒}{68090}
\saveTG{𣆋}{68090}
\saveTG{𠰚}{68090}
\saveTG{𠲇}{68091}
\saveTG{𣇞}{68094}
\saveTG{㘇}{68094}
\saveTG{𪡷}{68094}
\saveTG{唋}{68094}
\saveTG{畭}{68094}
\saveTG{趴}{68100}
\saveTG{𪾜}{68102}
\saveTG{𧿛}{68108}
\saveTG{𨃼}{68111}
\saveTG{䟭}{68111}
\saveTG{𨁔}{68111}
\saveTG{𨀣}{68112}
\saveTG{𧿶}{68112}
\saveTG{跄}{68112}
\saveTG{蹉}{68112}
\saveTG{踫}{68112}
\saveTG{跧}{68114}
\saveTG{䟶}{68114}
\saveTG{𨁑}{68117}
\saveTG{𨂟}{68117}
\saveTG{𨈇}{68117}
\saveTG{𨇣}{68117}
\saveTG{𫏠}{68117}
\saveTG{趷}{68117}
\saveTG{𫏒}{68117}
\saveTG{𧿦}{68117}
\saveTG{𨀫}{68117}
\saveTG{𨅁}{68118}
\saveTG{𧿩}{68120}
\saveTG{𨄫}{68120}
\saveTG{𪱎}{68120}
\saveTG{踰}{68121}
\saveTG{跈}{68122}
\saveTG{𧿝}{68126}
\saveTG{趻}{68127}
\saveTG{踚}{68127}
\saveTG{蹹}{68127}
\saveTG{𨁃}{68127}
\saveTG{𨆓}{68127}
\saveTG{𧿚}{68127}
\saveTG{𨄆}{68127}
\saveTG{𧿪}{68127}
\saveTG{䠯}{68127}
\saveTG{𨈋}{68127}
\saveTG{𫏟}{68127}
\saveTG{踗}{68132}
\saveTG{𨅷}{68132}
\saveTG{跉}{68132}
\saveTG{䠔}{68132}
\saveTG{𨆏}{68133}
\saveTG{𨃰}{68137}
\saveTG{躈}{68140}
\saveTG{𫏂}{68140}
\saveTG{𨅖}{68140}
\saveTG{𨄨}{68140}
\saveTG{𨇂}{68140}
\saveTG{𨅊}{68140}
\saveTG{𫏍}{68140}
\saveTG{蹾}{68140}
\saveTG{跰}{68141}
\saveTG{𨅺}{68142}
\saveTG{𨅸}{68143}
\saveTG{𨃇}{68144}
\saveTG{蹲}{68146}
\saveTG{躨}{68147}
\saveTG{𫏨}{68151}
\saveTG{𨇤}{68151}
\saveTG{𨀘}{68151}
\saveTG{躌}{68151}
\saveTG{𨆤}{68151}
\saveTG{𨆋}{68154}
\saveTG{𨅗}{68156}
\saveTG{踇}{68157}
\saveTG{𨁖}{68160}
\saveTG{𪑇}{68161}
\saveTG{𨇸}{68161}
\saveTG{𨈁}{68161}
\saveTG{𨅞}{68161}
\saveTG{跲}{68161}
\saveTG{䠓}{68164}
\saveTG{𨆝}{68166}
\saveTG{蹭}{68166}
\saveTG{蹌}{68167}
\saveTG{𨅦}{68168}
\saveTG{䟺}{68180}
\saveTG{蹤}{68181}
\saveTG{𨂚}{68181}
\saveTG{𨆑}{68181}
\saveTG{踨}{68181}
\saveTG{𨁁}{68182}
\saveTG{𨄕}{68184}
\saveTG{𨆘}{68186}
\saveTG{𨆦}{68186}
\saveTG{𨀀}{68190}
\saveTG{䟻}{68194}
\saveTG{𥂺}{68212}
\saveTG{𧢡}{68214}
\saveTG{𧢠}{68214}
\saveTG{𢼙}{68214}
\saveTG{飸}{68232}
\saveTG{覹}{68240}
\saveTG{敭}{68240}
\saveTG{𠭲}{68247}
\saveTG{𥏬}{68284}
\saveTG{𠊛}{68300}
\saveTG{𫐴}{68303}
\saveTG{𪐜}{68317}
\saveTG{𪒉}{68317}
\saveTG{𪑙}{68319}
\saveTG{𪐱}{68320}
\saveTG{𪒛}{68323}
\saveTG{𪑑}{68327}
\saveTG{𪐭}{68327}
\saveTG{黔}{68327}
\saveTG{𪒭}{68327}
\saveTG{䵰}{68327}
\saveTG{𪑡}{68330}
\saveTG{𪑅}{68331}
\saveTG{𪑫}{68332}
\saveTG{𪑄}{68332}
\saveTG{𪐸}{68332}
\saveTG{𪒓}{68332}
\saveTG{𢡑}{68334}
\saveTG{𢣮}{68334}
\saveTG{𪐫}{68340}
\saveTG{𪒠}{68340}
\saveTG{黭}{68346}
\saveTG{𪒹}{68356}
\saveTG{黣}{68357}
\saveTG{𪒟}{68366}
\saveTG{䵳}{68366}
\saveTG{𪒫}{68386}
\saveTG{𪐳}{68390}
\saveTG{𪑏}{68394}
\saveTG{𤱘}{68417}
\saveTG{𢿜}{68440}
\saveTG{𢽎}{68440}
\saveTG{斁}{68440}
\saveTG{𣀌}{68440}
\saveTG{𢽛}{68480}
\saveTG{𢼓}{68540}
\saveTG{㪤}{68540}
\saveTG{𠠩}{68620}
\saveTG{𣀬}{68640}
\saveTG{𣀡}{68640}
\saveTG{𪯟}{68640}
\saveTG{𧴩}{68800}
\saveTG{𣌎}{68804}
\saveTG{䝫}{68811}
\saveTG{賹}{68812}
\saveTG{𧸦}{68812}
\saveTG{𧷆}{68817}
\saveTG{䝯}{68817}
\saveTG{䝩}{68822}
\saveTG{𧷺}{68827}
\saveTG{𧷜}{68827}
\saveTG{𧹊}{68827}
\saveTG{𧸙}{68833}
\saveTG{賺}{68837}
\saveTG{贁}{68840}
\saveTG{敗}{68840}
\saveTG{𧸮}{68840}
\saveTG{𧸂}{68840}
\saveTG{䞃}{68840}
\saveTG{𢿃}{68840}
\saveTG{賆}{68841}
\saveTG{𧸡}{68855}
\saveTG{𧶅}{68857}
\saveTG{𫎖}{68861}
\saveTG{𧶗}{68862}
\saveTG{𧸚}{68864}
\saveTG{𧶟}{68864}
\saveTG{𧷶}{68865}
\saveTG{贈}{68866}
\saveTG{𧸤}{68866}
\saveTG{賶}{68867}
\saveTG{𧶾}{68885}
\saveTG{𧸘}{68886}
\saveTG{𧵉}{68890}
\saveTG{賒}{68891}
\saveTG{賖}{68894}
\saveTG{𣀠}{68940}
\saveTG{𣀉}{68940}
\saveTG{㪙}{68940}
\saveTG{𣞅}{68984}
\saveTG{啳}{69012}
\saveTG{𥆄}{69012}
\saveTG{咣}{69012}
\saveTG{睠}{69012}
\saveTG{晄}{69012}
\saveTG{𡃅}{69014}
\saveTG{瞠}{69014}
\saveTG{嘡}{69014}
\saveTG{畻}{69014}
\saveTG{𠻘}{69015}
\saveTG{𪡚}{69015}
\saveTG{𤱦}{69017}
\saveTG{𤱳}{69017}
\saveTG{𡆓}{69017}
\saveTG{𤲨}{69017}
\saveTG{𥇆}{69019}
\saveTG{吵}{69020}
\saveTG{𤰬}{69020}
\saveTG{𠳕}{69020}
\saveTG{𡂌}{69020}
\saveTG{眇}{69020}
\saveTG{唦}{69020}
\saveTG{𠴕}{69020}
\saveTG{𥇇}{69021}
\saveTG{㫾}{69027}
\saveTG{𠿀}{69027}
\saveTG{𠶤}{69027}
\saveTG{𣇢}{69027}
\saveTG{哨}{69027}
\saveTG{𣆺}{69027}
\saveTG{睄}{69027}
\saveTG{𪱌}{69027}
\saveTG{嘮}{69027}
\saveTG{𠯙}{69030}
\saveTG{矘}{69031}
\saveTG{曭}{69031}
\saveTG{𥌴}{69031}
\saveTG{𥇫}{69032}
\saveTG{𠿈}{69038}
\saveTG{矁}{69038}
\saveTG{𣉢}{69039}
\saveTG{𠺗}{69039}
\saveTG{瞇}{69039}
\saveTG{喽}{69044}
\saveTG{𡀼}{69044}
\saveTG{}{69044}
\saveTG{𡁊}{69044}
\saveTG{䁖}{69044}
\saveTG{𡀘}{69044}
\saveTG{𥊼}{69047}
\saveTG{𡄕}{69047}
\saveTG{𪱓}{69047}
\saveTG{𥍆}{69047}
\saveTG{𠰢}{69050}
\saveTG{畔}{69050}
\saveTG{眫}{69050}
\saveTG{𠴞}{69050}
\saveTG{𡁆}{69052}
\saveTG{𥋇}{69052}
\saveTG{𤳩}{69057}
\saveTG{𥌌}{69057}
\saveTG{疄}{69059}
\saveTG{瞵}{69059}
\saveTG{噒}{69059}
\saveTG{暽}{69059}
\saveTG{嚐}{69061}
\saveTG{㗂}{69062}
\saveTG{𥋐}{69062}
\saveTG{𪾱}{69062}
\saveTG{噹}{69066}
\saveTG{𡂚}{69067}
\saveTG{𪠽}{69077}
\saveTG{}{69077}
\saveTG{吙}{69080}
\saveTG{炚}{69080}
\saveTG{𥄣}{69080}
\saveTG{𥇃}{69080}
\saveTG{瞅}{69080}
\saveTG{唙}{69080}
\saveTG{啾}{69080}
\saveTG{𤰹}{69080}
\saveTG{𥊰}{69082}
\saveTG{𡁻}{69082}
\saveTG{唢}{69082}
\saveTG{嗩}{69086}
\saveTG{𤲩}{69089}
\saveTG{𥍏}{69089}
\saveTG{𡂋}{69089}
\saveTG{𠻪}{69089}
\saveTG{𡃘}{69089}
\saveTG{啖}{69089}
\saveTG{晱}{69089}
\saveTG{睒}{69089}
\saveTG{咪}{69094}
\saveTG{眯}{69094}
\saveTG{𥊲}{69094}
\saveTG{𡅊}{69094}
\saveTG{𡁸}{69096}
\saveTG{𥉋}{69099}
\saveTG{𫏈}{69112}
\saveTG{踡}{69112}
\saveTG{蹚}{69114}
\saveTG{𨄤}{69115}
\saveTG{䟞}{69120}
\saveTG{𨁭}{69120}
\saveTG{䠀}{69127}
\saveTG{𨂅}{69127}
\saveTG{𫛹}{69127}
\saveTG{䠮}{69127}
\saveTG{踃}{69127}
\saveTG{𨅝}{69142}
\saveTG{𠼔}{69146}
\saveTG{躞}{69147}
\saveTG{跘}{69150}
\saveTG{𨃪}{69152}
\saveTG{𨆴}{69157}
\saveTG{蹸}{69159}
\saveTG{𨆉}{69166}
\saveTG{踿}{69180}
\saveTG{𨆐}{69182}
\saveTG{𨤮}{69189}
\saveTG{𨤵}{69189}
\saveTG{𨁹}{69189}
\saveTG{𨆃}{69191}
\saveTG{𨀷}{69194}
\saveTG{𨅨}{69194}
\saveTG{𨄇}{69199}
\saveTG{𠵗}{69217}
\saveTG{𫆿}{69239}
\saveTG{𤿌}{69247}
\saveTG{𪃧}{69327}
\saveTG{𪑊}{69327}
\saveTG{𪒿}{69327}
\saveTG{𢡸}{69332}
\saveTG{𪐩}{69380}
\saveTG{𪑓}{69389}
\saveTG{𣊚}{69598}
\saveTG{𤏒}{69689}
\saveTG{𣰍}{69715}
\saveTG{𣋔}{69715}
\saveTG{𡮨}{69800}
\saveTG{尟}{69802}
\saveTG{𤓐}{69803}
\saveTG{𧵦}{69817}
\saveTG{𡮳}{69820}
\saveTG{𧸣}{69827}
\saveTG{𧸍}{69857}
\saveTG{𧷬}{69868}
\saveTG{𧷂}{69880}
\saveTG{䞉}{69886}
\saveTG{贘}{69886}
\saveTG{䞆}{69886}
\saveTG{賧}{69889}
\saveTG{𧷼}{69889}
\saveTG{𧵵}{69894}
\saveTG{𫃆}{69894}
\saveTG{𠄌}{70000}
\saveTG{𠃏}{70017}
\saveTG{乀}{70030}
\saveTG{𣦢}{70102}
\saveTG{璧}{70103}
\saveTG{𤩹}{70104}
\saveTG{𨐬}{70104}
\saveTG{埅}{70104}
\saveTG{壁}{70104}
\saveTG{鐾}{70109}
\saveTG{驻}{70114}
\saveTG{骓}{70115}
\saveTG{𨾦}{70115}
\saveTG{雎}{70115}
\saveTG{𩧧}{70118}
\saveTG{𩨆}{70121}
\saveTG{𫘮}{70127}
\saveTG{骧}{70132}
\saveTG{𫘜}{70140}
\saveTG{𨐧}{70141}
\saveTG{骍}{70141}
\saveTG{骇}{70182}
\saveTG{㐃}{70200}
\saveTG{𨸘}{70200}
\saveTG{𣩩}{70207}
\saveTG{𠙅}{70211}
\saveTG{䬃}{70211}
\saveTG{𦣆}{70211}
\saveTG{𩙮}{70212}
\saveTG{䬜}{70212}
\saveTG{䬘}{70212}
\saveTG{膔}{70212}
\saveTG{𦠦}{70212}
\saveTG{𡳪}{70212}
\saveTG{𩗁}{70213}
\saveTG{𩗶}{70214}
\saveTG{𡳝}{70214}
\saveTG{脏}{70214}
\saveTG{𠙤}{70214}
\saveTG{𩘱}{70214}
\saveTG{𨼁}{70214}
\saveTG{𨹛}{70214}
\saveTG{𡲈}{70214}
\saveTG{𪨚}{70214}
\saveTG{𩖰}{70214}
\saveTG{𩨻}{70214}
\saveTG{𦙴}{70214}
\saveTG{𩗒}{70214}
\saveTG{隡}{70215}
\saveTG{脽}{70215}
\saveTG{雅}{70215}
\saveTG{雕}{70215}
\saveTG{𨿁}{70215}
\saveTG{𦟗}{70215}
\saveTG{𦟿}{70215}
\saveTG{𦣎}{70215}
\saveTG{𩁊}{70215}
\saveTG{䐮}{70215}
\saveTG{𩀜}{70215}
\saveTG{𨿺}{70215}
\saveTG{臃}{70215}
\saveTG{膧}{70215}
\saveTG{𨾫}{70215}
\saveTG{陮}{70215}
\saveTG{𨾋}{70215}
\saveTG{𩀐}{70215}
\saveTG{𨻫}{70215}
\saveTG{𩪘}{70215}
\saveTG{朣}{70215}
\saveTG{𩘜}{70216}
\saveTG{膻}{70216}
\saveTG{𫗊}{70216}
\saveTG{䬓}{70216}
\saveTG{䬏}{70216}
\saveTG{𡲼}{70216}
\saveTG{𪱩}{70217}
\saveTG{𠒱}{70217}
\saveTG{𩪉}{70217}
\saveTG{骯}{70217}
\saveTG{𩩁}{70217}
\saveTG{肮}{70217}
\saveTG{𠙱}{70217}
\saveTG{𪱧}{70217}
\saveTG{𨺥}{70217}
\saveTG{𠙮}{70217}
\saveTG{䑉}{70217}
\saveTG{𪱨}{70217}
\saveTG{阬}{70217}
\saveTG{䏠}{70218}
\saveTG{𩗬}{70219}
\saveTG{𨸲}{70220}
\saveTG{𦝞}{70221}
\saveTG{𨺱}{70221}
\saveTG{𣂴}{70221}
\saveTG{𨹘}{70221}
\saveTG{𨼛}{70221}
\saveTG{𨺊}{70221}
\saveTG{𦞎}{70222}
\saveTG{隮}{70223}
\saveTG{臍}{70223}
\saveTG{𫕅}{70224}
\saveTG{}{70224}
\saveTG{脐}{70224}
\saveTG{𦜢}{70227}
\saveTG{𦟛}{70227}
\saveTG{𦡜}{70227}
\saveTG{䐱}{70227}
\saveTG{𨐯}{70227}
\saveTG{膀}{70227}
\saveTG{䧛}{70227}
\saveTG{臇}{70227}
\saveTG{䧚}{70227}
\saveTG{𦜶}{70227}
\saveTG{䐧}{70227}
\saveTG{𩥔}{70227}
\saveTG{𩖫}{70227}
\saveTG{𫕒}{70227}
\saveTG{髈}{70227}
\saveTG{臂}{70227}
\saveTG{腣}{70227}
\saveTG{防}{70227}
\saveTG{肪}{70227}
\saveTG{脝}{70227}
\saveTG{𩨣}{70227}
\saveTG{幦}{70227}
\saveTG{劈}{70227}
\saveTG{髇}{70227}
\saveTG{䐪}{70227}
\saveTG{𦞾}{70227}
\saveTG{𦝑}{70227}
\saveTG{𣎪}{70227}
\saveTG{𩩣}{70227}
\saveTG{䧡}{70227}
\saveTG{膲}{70231}
\saveTG{𩨬}{70231}
\saveTG{𩪮}{70231}
\saveTG{臕}{70231}
\saveTG{𩪣}{70231}
\saveTG{𨸫}{70232}
\saveTG{𩩷}{70232}
\saveTG{㵨}{70232}
\saveTG{𧲜}{70232}
\saveTG{𨻷}{70232}
\saveTG{𪱭}{70232}
\saveTG{䧇}{70232}
\saveTG{𨽢}{70232}
\saveTG{𨽖}{70232}
\saveTG{䧫}{70232}
\saveTG{䑋}{70232}
\saveTG{𤬞}{70232}
\saveTG{胘}{70232}
\saveTG{𤬚}{70233}
\saveTG{㼎}{70234}
\saveTG{臆}{70236}
\saveTG{𩪬}{70237}
\saveTG{臁}{70237}
\saveTG{腑}{70240}
\saveTG{𫔻}{70240}
\saveTG{𨐫}{70241}
\saveTG{辟}{70241}
\saveTG{隦}{70241}
\saveTG{𨻕}{70241}
\saveTG{𦡤}{70241}
\saveTG{𦞠}{70241}
\saveTG{𦛛}{70241}
\saveTG{𩪧}{70241}
\saveTG{𨐛}{70241}
\saveTG{𦡍}{70241}
\saveTG{膟}{70243}
\saveTG{𦟸}{70243}
\saveTG{𦝉}{70243}
\saveTG{䧪}{70244}
\saveTG{𨼻}{70244}
\saveTG{𨽘}{70244}
\saveTG{障}{70246}
\saveTG{朜}{70247}
\saveTG{𩩮}{70247}
\saveTG{䧐}{70247}
\saveTG{腋}{70247}
\saveTG{𦢝}{70247}
\saveTG{𠫏}{70248}
\saveTG{㬵}{70248}
\saveTG{𩩠}{70248}
\saveTG{脺}{70248}
\saveTG{胶}{70248}
\saveTG{骹}{70248}
\saveTG{𨼀}{70253}
\saveTG{𫔤}{70254}
\saveTG{𦟏}{70256}
\saveTG{𥌙}{70256}
\saveTG{𧦗}{70261}
\saveTG{䏽}{70261}
\saveTG{陪}{70261}
\saveTG{𨻓}{70261}
\saveTG{𦟋}{70261}
\saveTG{隌}{70261}
\saveTG{腤}{70261}
\saveTG{𩩿}{70261}
\saveTG{𩪙}{70262}
\saveTG{膪}{70262}
\saveTG{𣎀}{70262}
\saveTG{𣍵}{70264}
\saveTG{𩪟}{70265}
\saveTG{𩪩}{70265}
\saveTG{膅}{70265}
\saveTG{𦜟}{70269}
\saveTG{脑}{70272}
\saveTG{𦟉}{70274}
\saveTG{𦟷}{70281}
\saveTG{胲}{70282}
\saveTG{陔}{70282}
\saveTG{骸}{70282}
\saveTG{䧜}{70285}
\saveTG{𨽏}{70286}
\saveTG{𨽔}{70286}
\saveTG{𩓥}{70286}
\saveTG{髍}{70294}
\saveTG{𦡣}{70294}
\saveTG{𦟟}{70294}
\saveTG{𦢵}{70294}
\saveTG{𦜈}{70294}
\saveTG{𩣇}{70312}
\saveTG{𩦳}{70312}
\saveTG{騼}{70312}
\saveTG{駐}{70314}
\saveTG{騅}{70315}
\saveTG{𩥤}{70315}
\saveTG{𩥮}{70315}
\saveTG{䮵}{70315}
\saveTG{𩦼}{70315}
\saveTG{驙}{70316}
\saveTG{𩧣}{70317}
\saveTG{䯁}{70317}
\saveTG{𩤙}{70321}
\saveTG{䮰}{70321}
\saveTG{䮺}{70323}
\saveTG{𩦩}{70327}
\saveTG{鷿}{70327}
\saveTG{𩥬}{70327}
\saveTG{𩥻}{70327}
\saveTG{騯}{70327}
\saveTG{䮦}{70327}
\saveTG{𩤢}{70327}
\saveTG{𩡼}{70330}
\saveTG{𩣴}{70331}
\saveTG{䮽}{70331}
\saveTG{䮄}{70332}
\saveTG{𩧝}{70332}
\saveTG{𩦪}{70332}
\saveTG{驤}{70332}
\saveTG{𢜑}{70334}
\saveTG{憵}{70334}
\saveTG{𩼢}{70336}
\saveTG{𩥝}{70338}
\saveTG{馼}{70340}
\saveTG{騂}{70341}
\saveTG{𩣰}{70343}
\saveTG{騿}{70346}
\saveTG{駮}{70348}
\saveTG{𩤏}{70348}
\saveTG{𩤑}{70351}
\saveTG{𩤎}{70362}
\saveTG{𩣖}{70364}
\saveTG{𩥁}{70365}
\saveTG{駭}{70382}
\saveTG{𩧉}{70386}
\saveTG{䮶}{70394}
\saveTG{𩦆}{70394}
\saveTG{䢃}{70404}
\saveTG{嬖}{70404}
\saveTG{孹}{70407}
\saveTG{𨿷}{70415}
\saveTG{难}{70415}
\saveTG{𣄷}{70418}
\saveTG{𣁜}{70440}
\saveTG{𨐢}{70444}
\saveTG{𠮃}{70447}
\saveTG{𡣀}{70448}
\saveTG{𨹍}{70448}
\saveTG{𡥪}{70461}
\saveTG{𣄴}{70496}
\saveTG{擘}{70502}
\saveTG{𨐨}{70554}
\saveTG{𨐴}{70564}
\saveTG{譬}{70601}
\saveTG{礕}{70602}
\saveTG{𩀃}{70615}
\saveTG{𪩯}{70711}
\saveTG{𠄔}{70713}
\saveTG{𨾧}{70715}
\saveTG{𨲈}{70715}
\saveTG{𨾛}{70715}
\saveTG{䶆}{70715}
\saveTG{𨾥}{70715}
\saveTG{𨾹}{70715}
\saveTG{𩀀}{70715}
\saveTG{𩀫}{70715}
\saveTG{𨲷}{70716}
\saveTG{𨲵}{70716}
\saveTG{𪕇}{70717}
\saveTG{甓}{70717}
\saveTG{鼊}{70717}
\saveTG{𢐦}{70727}
\saveTG{𪕬}{70727}
\saveTG{𪕃}{70727}
\saveTG{𪖂}{70727}
\saveTG{𦣳}{70731}
\saveTG{𨳃}{70732}
\saveTG{𨳀}{70732}
\saveTG{襞}{70732}
\saveTG{鼤}{70740}
\saveTG{𦟨}{70747}
\saveTG{𫇈}{70748}
\saveTG{𧧃}{70761}
\saveTG{䛗}{70761}
\saveTG{𪩥}{70764}
\saveTG{𪕹}{70765}
\saveTG{𨐽}{70777}
\saveTG{躄}{70801}
\saveTG{𤴣}{70802}
\saveTG{𢣹}{70812}
\saveTG{赃}{70814}
\saveTG{赔}{70861}
\saveTG{赅}{70882}
\saveTG{繴}{70903}
\saveTG{糪}{70904}
\saveTG{檗}{70904}
\saveTG{𥽱}{70915}
\saveTG{巪}{71027}
\saveTG{𢸰}{71061}
\saveTG{䀈}{71102}
\saveTG{𣦯}{71102}
\saveTG{𧗖}{71102}
\saveTG{𠿦}{71104}
\saveTG{𡍱}{71104}
\saveTG{𡏇}{71104}
\saveTG{𪣗}{71104}
\saveTG{𡏃}{71104}
\saveTG{𡐷}{71104}
\saveTG{塈}{71104}
\saveTG{𦣠}{71104}
\saveTG{𠽺}{71104}
\saveTG{𨲣}{71104}
\saveTG{}{71106}
\saveTG{暨}{71106}
\saveTG{䇒}{71108}
\saveTG{𨬐}{71109}
\saveTG{}{71111}
\saveTG{𠪜}{71112}
\saveTG{𩧾}{71114}
\saveTG{𫘠}{71114}
\saveTG{𨻼}{71114}
\saveTG{驱}{71114}
\saveTG{𩾇}{71117}
\saveTG{𤭁}{71117}
\saveTG{}{71119}
\saveTG{骘}{71127}
\saveTG{骊}{71127}
\saveTG{𢀨}{71127}
\saveTG{𧓨}{71136}
\saveTG{𧏾}{71136}
\saveTG{螶}{71136}
\saveTG{𧓱}{71136}
\saveTG{瓥}{71136}
\saveTG{𧔞}{71136}
\saveTG{䘌}{71136}
\saveTG{𧏐}{71136}
\saveTG{𧒉}{71136}
\saveTG{𧒏}{71136}
\saveTG{𧖄}{71136}
\saveTG{𧐆}{71136}
\saveTG{𨪚}{71140}
\saveTG{𫘛}{71140}
\saveTG{䯅}{71147}
\saveTG{𢾌}{71147}
\saveTG{骦}{71161}
\saveTG{𩧿}{71166}
\saveTG{頚}{71186}
\saveTG{𩒍}{71186}
\saveTG{𩒗}{71186}
\saveTG{𩔡}{71186}
\saveTG{骠}{71191}
\saveTG{𫘪}{71196}
\saveTG{厂}{71200}
\saveTG{胪}{71207}
\saveTG{阯}{71210}
\saveTG{歴}{71211}
\saveTG{陫}{71211}
\saveTG{厞}{71211}
\saveTG{腓}{71211}
\saveTG{阷}{71211}
\saveTG{𩙛}{71211}
\saveTG{𧆡}{71211}
\saveTG{𩙓}{71211}
\saveTG{𩙘}{71211}
\saveTG{𦚞}{71211}
\saveTG{𪠐}{71211}
\saveTG{𩙙}{71211}
\saveTG{𩗽}{71211}
\saveTG{䫹}{71211}
\saveTG{䏘}{71211}
\saveTG{𡳕}{71211}
\saveTG{𦢫}{71211}
\saveTG{𨻃}{71211}
\saveTG{𩙖}{71211}
\saveTG{厏}{71211}
\saveTG{龎}{71211}
\saveTG{隴}{71211}
\saveTG{朧}{71211}
\saveTG{歷}{71211}
\saveTG{𦡕}{71212}
\saveTG{𦙗}{71212}
\saveTG{𦜖}{71212}
\saveTG{𣍴}{71212}
\saveTG{𣍭}{71212}
\saveTG{𦚹}{71212}
\saveTG{𦙁}{71212}
\saveTG{𣎄}{71212}
\saveTG{𠨽}{71212}
\saveTG{𦛖}{71212}
\saveTG{𦙫}{71212}
\saveTG{}{71212}
\saveTG{𨼋}{71212}
\saveTG{𠪉}{71212}
\saveTG{𦚠}{71212}
\saveTG{䏓}{71212}
\saveTG{𦠊}{71212}
\saveTG{𩩋}{71212}
\saveTG{𠩧}{71212}
\saveTG{𩗛}{71212}
\saveTG{𪠙}{71212}
\saveTG{㕎}{71212}
\saveTG{𠪈}{71212}
\saveTG{𦚟}{71212}
\saveTG{𨹟}{71212}
\saveTG{𨽜}{71212}
\saveTG{𠨱}{71212}
\saveTG{厩}{71212}
\saveTG{陘}{71212}
\saveTG{脛}{71212}
\saveTG{肛}{71212}
\saveTG{厄}{71212}
\saveTG{兏}{71212}
\saveTG{旤}{71212}
\saveTG{阨}{71212}
\saveTG{厑}{71212}
\saveTG{𩪸}{71212}
\saveTG{𫇀}{71212}
\saveTG{𦠿}{71212}
\saveTG{𠩎}{71212}
\saveTG{𩩤}{71212}
\saveTG{𤬜}{71212}
\saveTG{𪨞}{71212}
\saveTG{𩗾}{71212}
\saveTG{𩗿}{71212}
\saveTG{䬚}{71212}
\saveTG{𩗠}{71212}
\saveTG{阢}{71212}
\saveTG{朊}{71212}
\saveTG{阮}{71212}
\saveTG{髗}{71212}
\saveTG{臚}{71212}
\saveTG{陋}{71212}
\saveTG{𠩫}{71212}
\saveTG{𠩆}{71212}
\saveTG{𨸖}{71212}
\saveTG{𫆻}{71212}
\saveTG{𠫂}{71212}
\saveTG{𪠋}{71212}
\saveTG{魘}{71213}
\saveTG{压}{71213}
\saveTG{飋}{71213}
\saveTG{𩴣}{71213}
\saveTG{𫗋}{71213}
\saveTG{𩗢}{71213}
\saveTG{𠩓}{71213}
\saveTG{魇}{71213}
\saveTG{𠩥}{71214}
\saveTG{𠪪}{71214}
\saveTG{𠩜}{71214}
\saveTG{𠪖}{71214}
\saveTG{𡐰}{71214}
\saveTG{胵}{71214}
\saveTG{圧}{71214}
\saveTG{厐}{71214}
\saveTG{厖}{71214}
\saveTG{厔}{71214}
\saveTG{𠪤}{71214}
\saveTG{厓}{71214}
\saveTG{𨸟}{71214}
\saveTG{𠩟}{71214}
\saveTG{壓}{71214}
\saveTG{隁}{71214}
\saveTG{𠩑}{71214}
\saveTG{𩘀}{71214}
\saveTG{𩙩}{71214}
\saveTG{𦝪}{71214}
\saveTG{𡳊}{71214}
\saveTG{𠫀}{71214}
\saveTG{𪨛}{71214}
\saveTG{𤧤}{71214}
\saveTG{𦞥}{71214}
\saveTG{𠪽}{71214}
\saveTG{𡓸}{71214}
\saveTG{㕓}{71214}
\saveTG{𨻋}{71214}
\saveTG{𨻐}{71214}
\saveTG{𪺾}{71214}
\saveTG{𩗸}{71214}
\saveTG{𦞐}{71214}
\saveTG{颬}{71214}
\saveTG{𡱭}{71214}
\saveTG{𩰋}{71214}
\saveTG{𦚗}{71214}
\saveTG{𠩈}{71214}
\saveTG{㻺}{71214}
\saveTG{陻}{71214}
\saveTG{厇}{71214}
\saveTG{𠪨}{71214}
\saveTG{𠩔}{71214}
\saveTG{臛}{71215}
\saveTG{𩀾}{71215}
\saveTG{㕍}{71215}
\saveTG{𩪼}{71215}
\saveTG{𨽅}{71215}
\saveTG{𠪵}{71215}
\saveTG{𪠕}{71215}
\saveTG{𠪎}{71215}
\saveTG{𪠈}{71215}
\saveTG{雁}{71215}
\saveTG{厘}{71215}
\saveTG{𨽧}{71215}
\saveTG{厪}{71215}
\saveTG{𦝟}{71215}
\saveTG{𤬠}{71215}
\saveTG{𠫈}{71215}
\saveTG{𠨰}{71215}
\saveTG{厜}{71215}
\saveTG{膒}{71216}
\saveTG{飐}{71216}
\saveTG{𨸭}{71216}
\saveTG{㕇}{71216}
\saveTG{𦚸}{71216}
\saveTG{飅}{71216}
\saveTG{𠪶}{71216}
\saveTG{𩘆}{71216}
\saveTG{䧢}{71216}
\saveTG{𦟻}{71216}
\saveTG{颭}{71216}
\saveTG{𫕘}{71216}
\saveTG{𦞌}{71216}
\saveTG{𠪍}{71217}
\saveTG{𠩳}{71217}
\saveTG{𪠄}{71217}
\saveTG{𦡏}{71217}
\saveTG{𡲽}{71217}
\saveTG{𦡻}{71217}
\saveTG{𫇃}{71217}
\saveTG{䯈}{71217}
\saveTG{㒫}{71217}
\saveTG{𦟰}{71217}
\saveTG{㽁}{71217}
\saveTG{𠩅}{71217}
\saveTG{𠪹}{71217}
\saveTG{𠨷}{71217}
\saveTG{䖎}{71217}
\saveTG{𨼘}{71217}
\saveTG{𦘵}{71217}
\saveTG{𦘰}{71217}
\saveTG{㼩}{71217}
\saveTG{𨸧}{71217}
\saveTG{𠪑}{71217}
\saveTG{𦘸}{71217}
\saveTG{䖐}{71217}
\saveTG{𤭸}{71217}
\saveTG{𤭛}{71217}
\saveTG{𦡑}{71217}
\saveTG{𤭄}{71217}
\saveTG{𤬶}{71217}
\saveTG{脰}{71218}
\saveTG{𦟚}{71218}
\saveTG{𨹜}{71218}
\saveTG{𨻭}{71218}
\saveTG{𨻯}{71218}
\saveTG{𩗟}{71218}
\saveTG{陿}{71218}
\saveTG{𨽲}{71218}
\saveTG{𠫎}{71219}
\saveTG{飃}{71219}
\saveTG{胚}{71219}
\saveTG{𩘟}{71219}
\saveTG{㕋}{71219}
\saveTG{𠩽}{71219}
\saveTG{𨸹}{71219}
\saveTG{𨬑}{71219}
\saveTG{𠪒}{71219}
\saveTG{䦺}{71220}
\saveTG{𣍠}{71220}
\saveTG{𦘭}{71220}
\saveTG{𩨑}{71220}
\saveTG{𠩪}{71220}
\saveTG{阿}{71220}
\saveTG{厕}{71220}
\saveTG{厠}{71220}
\saveTG{胢}{71220}
\saveTG{𠪋}{71220}
\saveTG{𫔿}{71220}
\saveTG{厮}{71221}
\saveTG{䯒}{71221}
\saveTG{陟}{71221}
\saveTG{胻}{71221}
\saveTG{𢀦}{71221}
\saveTG{厅}{71221}
\saveTG{𨽇}{71221}
\saveTG{厛}{71221}
\saveTG{𠩇}{71222}
\saveTG{𠪟}{71224}
\saveTG{𠵲}{71226}
\saveTG{䯊}{71226}
\saveTG{𠪠}{71226}
\saveTG{𩪍}{71226}
\saveTG{𠩝}{71227}
\saveTG{𠨳}{71227}
\saveTG{𠪲}{71227}
\saveTG{𨸑}{71227}
\saveTG{𨻛}{71227}
\saveTG{𣍷}{71227}
\saveTG{𣧚}{71227}
\saveTG{𠀞}{71227}
\saveTG{𪠃}{71227}
\saveTG{𨽥}{71227}
\saveTG{𩣹}{71227}
\saveTG{𫕆}{71227}
\saveTG{𩢉}{71227}
\saveTG{𠪂}{71227}
\saveTG{𨼏}{71227}
\saveTG{𨻳}{71227}
\saveTG{𩱔}{71227}
\saveTG{𪅼}{71227}
\saveTG{𩪜}{71227}
\saveTG{𩪰}{71227}
\saveTG{𪓍}{71227}
\saveTG{𦢈}{71227}
\saveTG{脼}{71227}
\saveTG{𨽨}{71227}
\saveTG{䐴}{71227}
\saveTG{𨽚}{71227}
\saveTG{陑}{71227}
\saveTG{𧱔}{71227}
\saveTG{𦟐}{71227}
\saveTG{𠨼}{71227}
\saveTG{𧥚}{71227}
\saveTG{鴈}{71227}
\saveTG{鳫}{71227}
\saveTG{肟}{71227}
\saveTG{隬}{71227}
\saveTG{厉}{71227}
\saveTG{𣃞}{71227}
\saveTG{𣍾}{71227}
\saveTG{𩨗}{71227}
\saveTG{䧞}{71227}
\saveTG{𣍪}{71227}
\saveTG{𠫑}{71227}
\saveTG{㕊}{71227}
\saveTG{𦢚}{71227}
\saveTG{𣎩}{71227}
\saveTG{𢅠}{71227}
\saveTG{𠪺}{71227}
\saveTG{𦠒}{71227}
\saveTG{𦠴}{71227}
\saveTG{𦠓}{71227}
\saveTG{𠟄}{71227}
\saveTG{陃}{71227}
\saveTG{脣}{71227}
\saveTG{胹}{71227}
\saveTG{历}{71227}
\saveTG{臑}{71227}
\saveTG{隭}{71227}
\saveTG{隔}{71227}
\saveTG{膈}{71227}
\saveTG{鷢}{71227}
\saveTG{厲}{71227}
\saveTG{㕐}{71227}
\saveTG{𦠌}{71227}
\saveTG{㕂}{71227}
\saveTG{𠨴}{71228}
\saveTG{𠩄}{71228}
\saveTG{𦠩}{71229}
\saveTG{𢢸}{71231}
\saveTG{𤔋}{71231}
\saveTG{胩}{71231}
\saveTG{黡}{71231}
\saveTG{黶}{71231}
\saveTG{𨼚}{71231}
\saveTG{𨑃}{71231}
\saveTG{𠩂}{71231}
\saveTG{𦠲}{71231}
\saveTG{𠪩}{71231}
\saveTG{𦤻}{71231}
\saveTG{𩪪}{71231}
\saveTG{𠫇}{71231}
\saveTG{𪠇}{71231}
\saveTG{㕔}{71231}
\saveTG{𠫊}{71231}
\saveTG{𩨤}{71231}
\saveTG{脹}{71232}
\saveTG{𠪥}{71232}
\saveTG{𩪥}{71232}
\saveTG{𤬅}{71232}
\saveTG{𨸞}{71232}
\saveTG{𠨸}{71232}
\saveTG{𦟥}{71232}
\saveTG{𪺑}{71232}
\saveTG{𠫒}{71232}
\saveTG{𠪏}{71232}
\saveTG{𩞨}{71232}
\saveTG{𠫄}{71232}
\saveTG{𠫐}{71232}
\saveTG{𦢄}{71232}
\saveTG{𣱷}{71232}
\saveTG{𧝏}{71232}
\saveTG{䐁}{71232}
\saveTG{𩩒}{71232}
\saveTG{𨼫}{71232}
\saveTG{𨑄}{71232}
\saveTG{㕄}{71232}
\saveTG{𪠍}{71232}
\saveTG{𠩐}{71232}
\saveTG{𦙘}{71232}
\saveTG{𠩸}{71232}
\saveTG{臄}{71232}
\saveTG{𧲋}{71232}
\saveTG{饜}{71232}
\saveTG{餍}{71232}
\saveTG{脤}{71232}
\saveTG{豚}{71232}
\saveTG{陙}{71232}
\saveTG{辰}{71232}
\saveTG{𠪰}{71233}
\saveTG{𦟙}{71233}
\saveTG{𠩁}{71233}
\saveTG{𪠎}{71233}
\saveTG{𠪿}{71236}
\saveTG{𤔜}{71236}
\saveTG{𨽳}{71236}
\saveTG{蟨}{71236}
\saveTG{蜃}{71236}
\saveTG{𦢛}{71236}
\saveTG{𦝿}{71236}
\saveTG{𧓽}{71236}
\saveTG{𠪊}{71237}
\saveTG{𨼆}{71237}
\saveTG{懕}{71238}
\saveTG{𠪾}{71239}
\saveTG{厯}{71239}
\saveTG{愿}{71239}
\saveTG{𠪱}{71239}
\saveTG{肝}{71240}
\saveTG{牙}{71240}
\saveTG{𩩄}{71240}
\saveTG{䏏}{71240}
\saveTG{㕏}{71240}
\saveTG{骬}{71240}
\saveTG{骭}{71240}
\saveTG{䏪}{71240}
\saveTG{㕑}{71240}
\saveTG{𠪆}{71240}
\saveTG{𨸦}{71240}
\saveTG{𨸗}{71240}
\saveTG{厨}{71240}
\saveTG{𠩀}{71240}
\saveTG{𪠆}{71241}
\saveTG{厊}{71241}
\saveTG{㕃}{71241}
\saveTG{厗}{71241}
\saveTG{厈}{71241}
\saveTG{𠪯}{71241}
\saveTG{𪠒}{71241}
\saveTG{㕌}{71241}
\saveTG{𠪮}{71241}
\saveTG{𨽦}{71241}
\saveTG{𨽕}{71241}
\saveTG{𠩢}{71241}
\saveTG{𦣀}{71241}
\saveTG{𨸶}{71242}
\saveTG{𠨿}{71242}
\saveTG{厎}{71242}
\saveTG{𫔾}{71242}
\saveTG{𨻲}{71242}
\saveTG{𪠑}{71242}
\saveTG{𩨠}{71242}
\saveTG{𦞹}{71243}
\saveTG{腰}{71244}
\saveTG{嬮}{71244}
\saveTG{𣍲}{71244}
\saveTG{𩩖}{71244}
\saveTG{𨺿}{71244}
\saveTG{𩧏}{71245}
\saveTG{𦛟}{71245}
\saveTG{𠩶}{71246}
\saveTG{𠩲}{71246}
\saveTG{𨺑}{71246}
\saveTG{骾}{71246}
\saveTG{䐺}{71246}
\saveTG{𠩰}{71246}
\saveTG{𠪀}{71246}
\saveTG{𦜰}{71246}
\saveTG{𩪭}{71247}
\saveTG{厚}{71247}
\saveTG{𫆯}{71247}
\saveTG{𪱠}{71247}
\saveTG{𦜒}{71247}
\saveTG{𠩹}{71247}
\saveTG{𡲠}{71247}
\saveTG{𠪇}{71247}
\saveTG{𠪅}{71247}
\saveTG{𦞶}{71247}
\saveTG{𠩴}{71247}
\saveTG{𠩒}{71247}
\saveTG{𪠏}{71247}
\saveTG{𠨹}{71247}
\saveTG{𩩹}{71247}
\saveTG{𣀻}{71247}
\saveTG{𠩞}{71247}
\saveTG{𠩭}{71247}
\saveTG{𠩻}{71247}
\saveTG{𠪘}{71247}
\saveTG{𠨯}{71247}
\saveTG{𠩨}{71247}
\saveTG{𠪬}{71247}
\saveTG{𠨻}{71247}
\saveTG{𠩍}{71247}
\saveTG{𪠔}{71247}
\saveTG{𠬡}{71247}
\saveTG{𢾥}{71247}
\saveTG{𢼞}{71247}
\saveTG{𢾂}{71247}
\saveTG{𦛍}{71247}
\saveTG{𤔮}{71247}
\saveTG{㪒}{71247}
\saveTG{厦}{71247}
\saveTG{𣀥}{71247}
\saveTG{𣀣}{71247}
\saveTG{𣁀}{71247}
\saveTG{𢾯}{71247}
\saveTG{𢿺}{71247}
\saveTG{𢽢}{71247}
\saveTG{𢼱}{71247}
\saveTG{𧥙}{71247}
\saveTG{𢽚}{71247}
\saveTG{𢽧}{71247}
\saveTG{𢻻}{71247}
\saveTG{𢼖}{71247}
\saveTG{𢽜}{71247}
\saveTG{厰}{71248}
\saveTG{𠪚}{71248}
\saveTG{𪠛}{71248}
\saveTG{𦞬}{71248}
\saveTG{𠨺}{71248}
\saveTG{𠪫}{71248}
\saveTG{𠪭}{71248}
\saveTG{𠪣}{71248}
\saveTG{厫}{71248}
\saveTG{胓}{71249}
\saveTG{𠨵}{71250}
\saveTG{𠨭}{71250}
\saveTG{䑞}{71251}
\saveTG{𠨾}{71252}
\saveTG{𪠘}{71252}
\saveTG{擪}{71252}
\saveTG{𠩙}{71253}
\saveTG{㕒}{71253}
\saveTG{厍}{71254}
\saveTG{𠩿}{71254}
\saveTG{厴}{71256}
\saveTG{厙}{71256}
\saveTG{厣}{71256}
\saveTG{𠩯}{71256}
\saveTG{㕅}{71256}
\saveTG{𠪷}{71256}
\saveTG{𠩌}{71260}
\saveTG{阽}{71260}
\saveTG{𣅛}{71260}
\saveTG{𦜷}{71260}
\saveTG{胋}{71260}
\saveTG{厢}{71260}
\saveTG{䐶}{71261}
\saveTG{䏸}{71261}
\saveTG{𫕁}{71261}
\saveTG{𪠅}{71261}
\saveTG{𥐘}{71261}
\saveTG{𨼄}{71261}
\saveTG{𠩊}{71261}
\saveTG{𪩳}{71261}
\saveTG{𨼐}{71261}
\saveTG{厝}{71261}
\saveTG{㕉}{71261}
\saveTG{𩩑}{71261}
\saveTG{𠩮}{71262}
\saveTG{𦚈}{71262}
\saveTG{𩩰}{71262}
\saveTG{靥}{71262}
\saveTG{陌}{71262}
\saveTG{𦠝}{71262}
\saveTG{脜}{71262}
\saveTG{𣍽}{71262}
\saveTG{𦟶}{71262}
\saveTG{𪞲}{71262}
\saveTG{靨}{71262}
\saveTG{磿}{71262}
\saveTG{𠩡}{71262}
\saveTG{𠪐}{71262}
\saveTG{腼}{71262}
\saveTG{𠵧}{71263}
\saveTG{𦟢}{71263}
\saveTG{唇}{71263}
\saveTG{厬}{71264}
\saveTG{𦟕}{71264}
\saveTG{𦚵}{71264}
\saveTG{𠩚}{71264}
\saveTG{䧈}{71264}
\saveTG{𠪄}{71264}
\saveTG{㕆}{71264}
\saveTG{𠪡}{71264}
\saveTG{𨺤}{71266}
\saveTG{腷}{71266}
\saveTG{𠪃}{71267}
\saveTG{𠪞}{71268}
\saveTG{𧯏}{71268}
\saveTG{𨹙}{71268}
\saveTG{𪠚}{71269}
\saveTG{脴}{71269}
\saveTG{曆}{71269}
\saveTG{暦}{71269}
\saveTG{𠫁}{71269}
\saveTG{𨹭}{71269}
\saveTG{𪙪}{71271}
\saveTG{𪙇}{71272}
\saveTG{𠩉}{71272}
\saveTG{𡳒}{71272}
\saveTG{𠩋}{71272}
\saveTG{𡸡}{71272}
\saveTG{厒}{71272}
\saveTG{𪙌}{71272}
\saveTG{𩪲}{71272}
\saveTG{𦢠}{71272}
\saveTG{𠪦}{71274}
\saveTG{𤯍}{71274}
\saveTG{膤}{71277}
\saveTG{仄}{71280}
\saveTG{𠪙}{71281}
\saveTG{厧}{71281}
\saveTG{𠩷}{71281}
\saveTG{蹷}{71281}
\saveTG{赝}{71282}
\saveTG{顾}{71282}
\saveTG{𩪗}{71282}
\saveTG{𫖹}{71282}
\saveTG{𠪗}{71282}
\saveTG{𠪁}{71282}
\saveTG{颀}{71282}
\saveTG{𠪢}{71282}
\saveTG{𠩦}{71282}
\saveTG{𠩗}{71282}
\saveTG{𠩱}{71282}
\saveTG{𠪴}{71282}
\saveTG{𣣾}{71282}
\saveTG{𪠌}{71282}
\saveTG{𨺗}{71282}
\saveTG{厥}{71282}
\saveTG{𠩩}{71284}
\saveTG{厭}{71284}
\saveTG{𪠊}{71284}
\saveTG{陾}{71284}
\saveTG{腝}{71284}
\saveTG{厌}{71284}
\saveTG{𥎦}{71284}
\saveTG{𠪝}{71284}
\saveTG{𠨮}{71284}
\saveTG{𡙽}{71284}
\saveTG{𤟐}{71284}
\saveTG{𨽀}{71284}
\saveTG{𠨶}{71285}
\saveTG{𪠖}{71285}
\saveTG{頎}{71286}
\saveTG{𠫍}{71286}
\saveTG{𪠓}{71286}
\saveTG{䫢}{71286}
\saveTG{𩓃}{71286}
\saveTG{𩒒}{71286}
\saveTG{𩓛}{71286}
\saveTG{𩒫}{71286}
\saveTG{𩒣}{71286}
\saveTG{䫗}{71286}
\saveTG{𩓦}{71286}
\saveTG{䫚}{71286}
\saveTG{𩑏}{71286}
\saveTG{𫖞}{71286}
\saveTG{頋}{71286}
\saveTG{𩓒}{71286}
\saveTG{𦢶}{71286}
\saveTG{䪶}{71286}
\saveTG{𦡋}{71286}
\saveTG{𠫉}{71286}
\saveTG{𩖄}{71286}
\saveTG{𩖒}{71286}
\saveTG{𩔟}{71286}
\saveTG{𩒃}{71286}
\saveTG{䫆}{71286}
\saveTG{𩕁}{71286}
\saveTG{䫃}{71286}
\saveTG{𨻺}{71286}
\saveTG{𨺟}{71286}
\saveTG{𨽗}{71286}
\saveTG{𣅦}{71286}
\saveTG{𠩠}{71286}
\saveTG{顝}{71286}
\saveTG{願}{71286}
\saveTG{贗}{71286}
\saveTG{贋}{71286}
\saveTG{厱}{71286}
\saveTG{𠪳}{71287}
\saveTG{𠩘}{71288}
\saveTG{㷴}{71289}
\saveTG{𠩾}{71289}
\saveTG{𤑤}{71289}
\saveTG{𨻢}{71289}
\saveTG{𤈲}{71289}
\saveTG{𠪛}{71289}
\saveTG{𦙻}{71289}
\saveTG{𠪌}{71289}
\saveTG{阫}{71290}
\saveTG{肧}{71290}
\saveTG{䧣}{71291}
\saveTG{𩪊}{71291}
\saveTG{䏡}{71291}
\saveTG{𠩏}{71291}
\saveTG{际}{71291}
\saveTG{膘}{71291}
\saveTG{厡}{71292}
\saveTG{𫆮}{71293}
\saveTG{𦃢}{71293}
\saveTG{𠪓}{71293}
\saveTG{}{71294}
\saveTG{𠪸}{71294}
\saveTG{𠫋}{71294}
\saveTG{𦞰}{71294}
\saveTG{𠫃}{71294}
\saveTG{𠩬}{71294}
\saveTG{檿}{71294}
\saveTG{𠩕}{71294}
\saveTG{厤}{71294}
\saveTG{𠩵}{71295}
\saveTG{𠩤}{71296}
\saveTG{原}{71296}
\saveTG{𨻣}{71296}
\saveTG{厵}{71296}
\saveTG{𠗁}{71303}
\saveTG{𩢬}{71307}
\saveTG{𩢼}{71311}
\saveTG{䮾}{71311}
\saveTG{騑}{71311}
\saveTG{𩣾}{71312}
\saveTG{𩣘}{71312}
\saveTG{𩣕}{71312}
\saveTG{𩢄}{71312}
\saveTG{𩣪}{71312}
\saveTG{𩧥}{71312}
\saveTG{𩣄}{71312}
\saveTG{𩤃}{71312}
\saveTG{驪}{71312}
\saveTG{驢}{71312}
\saveTG{驉}{71312}
\saveTG{駤}{71314}
\saveTG{駆}{71314}
\saveTG{𩤴}{71314}
\saveTG{𩢤}{71314}
\saveTG{驅}{71316}
\saveTG{𩢘}{71316}
\saveTG{𫘇}{71317}
\saveTG{駏}{71317}
\saveTG{鮔}{71317}
\saveTG{𫘌}{71317}
\saveTG{駓}{71319}
\saveTG{𩧤}{71320}
\saveTG{𩡯}{71320}
\saveTG{䭶}{71320}
\saveTG{𩣝}{71321}
\saveTG{䭼}{71321}
\saveTG{𩡸}{71327}
\saveTG{𩤷}{71327}
\saveTG{騳}{71327}
\saveTG{𤇟}{71327}
\saveTG{𩤐}{71327}
\saveTG{𩧃}{71327}
\saveTG{𩧢}{71327}
\saveTG{𩥋}{71327}
\saveTG{𩦣}{71327}
\saveTG{驫}{71327}
\saveTG{𪄵}{71327}
\saveTG{𩥐}{71327}
\saveTG{𩤬}{71327}
\saveTG{馬}{71327}
\saveTG{騭}{71327}
\saveTG{𩿔}{71327}
\saveTG{䮥}{71327}
\saveTG{慝}{71331}
\saveTG{㥦}{71331}
\saveTG{𤎝}{71331}
\saveTG{𢟪}{71331}
\saveTG{㤅}{71331}
\saveTG{𢙞}{71331}
\saveTG{悘}{71331}
\saveTG{𤇸}{71331}
\saveTG{䮱}{71332}
\saveTG{𢟀}{71332}
\saveTG{𢛚}{71333}
\saveTG{𢟲}{71334}
\saveTG{𤋧}{71336}
\saveTG{鱀}{71336}
\saveTG{𩼴}{71336}
\saveTG{憠}{71338}
\saveTG{𢥧}{71338}
\saveTG{𢟍}{71339}
\saveTG{𢡪}{71339}
\saveTG{馯}{71340}
\saveTG{駬}{71340}
\saveTG{𩦎}{71340}
\saveTG{𩤾}{71340}
\saveTG{䯀}{71341}
\saveTG{辱}{71343}
\saveTG{騕}{71344}
\saveTG{驔}{71346}
\saveTG{䮓}{71346}
\saveTG{𩤳}{71347}
\saveTG{駍}{71349}
\saveTG{䮆}{71349}
\saveTG{䮹}{71353}
\saveTG{𩣙}{71356}
\saveTG{䮠}{71360}
\saveTG{驦}{71361}
\saveTG{䮏}{71361}
\saveTG{驑}{71362}
\saveTG{𩧍}{71362}
\saveTG{𫘏}{71362}
\saveTG{𩢷}{71362}
\saveTG{𫘊}{71364}
\saveTG{𩣚}{71369}
\saveTG{駵}{71377}
\saveTG{𩦒}{71382}
\saveTG{𩦢}{71384}
\saveTG{𩥰}{71384}
\saveTG{𩕮}{71386}
\saveTG{䫲}{71386}
\saveTG{𩦈}{71389}
\saveTG{驃}{71391}
\saveTG{𩤋}{71391}
\saveTG{𢀱}{71394}
\saveTG{騵}{71396}
\saveTG{䭴}{71402}
\saveTG{娿}{71404}
\saveTG{𡕣}{71407}
\saveTG{𪚘}{71411}
\saveTG{𠮔}{71412}
\saveTG{𣄭}{71412}
\saveTG{𨼼}{71412}
\saveTG{𣄰}{71412}
\saveTG{旡}{71412}
\saveTG{𦗛}{71416}
\saveTG{㲣}{71417}
\saveTG{𪩤}{71420}
\saveTG{𠡿}{71427}
\saveTG{𩡧}{71427}
\saveTG{𠢤}{71427}
\saveTG{𢀭}{71427}
\saveTG{𧱝}{71432}
\saveTG{𦕇}{71432}
\saveTG{𠪼}{71432}
\saveTG{馵}{71442}
\saveTG{𡝌}{71443}
\saveTG{𩢏}{71446}
\saveTG{斅}{71447}
\saveTG{𢿕}{71447}
\saveTG{㪢}{71447}
\saveTG{敠}{71447}
\saveTG{𡡕}{71448}
\saveTG{𦖦}{71486}
\saveTG{𩒄}{71486}
\saveTG{𩓔}{71486}
\saveTG{𩑌}{71486}
\saveTG{𢴺}{71502}
\saveTG{馽}{71506}
\saveTG{𦘟}{71507}
\saveTG{𢀥}{71527}
\saveTG{𩑞}{71586}
\saveTG{𦟧}{71601}
\saveTG{𧫜}{71601}
\saveTG{𥕳}{71602}
\saveTG{𩡳}{71602}
\saveTG{㫳}{71603}
\saveTG{䢈}{71606}
\saveTG{𥌅}{71608}
\saveTG{𤳪}{71608}
\saveTG{𩤹}{71609}
\saveTG{𩥖}{71627}
\saveTG{𫕖}{71627}
\saveTG{𦖫}{71642}
\saveTG{㪞}{71647}
\saveTG{敯}{71647}
\saveTG{𩔉}{71686}
\saveTG{䫒}{71686}
\saveTG{𩒲}{71686}
\saveTG{𠤬}{71710}
\saveTG{匚}{71710}
\saveTG{匸}{71710}
\saveTG{匟}{71711}
\saveTG{匡}{71711}
\saveTG{匩}{71711}
\saveTG{匭}{71711}
\saveTG{𠄧}{71711}
\saveTG{匦}{71711}
\saveTG{匪}{71711}
\saveTG{匨}{71711}
\saveTG{𪟯}{71711}
\saveTG{𠤵}{71711}
\saveTG{匹}{71711}
\saveTG{匤}{71711}
\saveTG{𠥔}{71711}
\saveTG{龨}{71711}
\saveTG{𠥥}{71711}
\saveTG{𠤲}{71711}
\saveTG{𠥶}{71711}
\saveTG{㔸}{71711}
\saveTG{㔳}{71711}
\saveTG{㔲}{71711}
\saveTG{匯}{71711}
\saveTG{匞}{71711}
\saveTG{匷}{71711}
\saveTG{𠀧}{71711}
\saveTG{匜}{71711}
\saveTG{𠥕}{71711}
\saveTG{𠥳}{71711}
\saveTG{𠤷}{71711}
\saveTG{𠥟}{71711}
\saveTG{𠥞}{71711}
\saveTG{𠥆}{71711}
\saveTG{𠤹}{71711}
\saveTG{𠥃}{71712}
\saveTG{𠥰}{71712}
\saveTG{匢}{71712}
\saveTG{𠥮}{71712}
\saveTG{𠤾}{71712}
\saveTG{𠥴}{71712}
\saveTG{𦣦}{71712}
\saveTG{㔷}{71712}
\saveTG{𠥬}{71712}
\saveTG{𦣞}{71712}
\saveTG{𨳁}{71712}
\saveTG{𪕣}{71712}
\saveTG{𫇆}{71712}
\saveTG{𠥧}{71712}
\saveTG{𠥪}{71712}
\saveTG{𠥍}{71712}
\saveTG{𠥩}{71712}
\saveTG{㔰}{71712}
\saveTG{匠}{71712}
\saveTG{匝}{71712}
\saveTG{匬}{71712}
\saveTG{𦣝}{71712}
\saveTG{既}{71712}
\saveTG{臣}{71712}
\saveTG{匾}{71712}
\saveTG{𪖍}{71712}
\saveTG{㱘}{71712}
\saveTG{𪟬}{71712}
\saveTG{𠨕}{71712}
\saveTG{𠤶}{71712}
\saveTG{𠤮}{71712}
\saveTG{𠥭}{71712}
\saveTG{𠥣}{71712}
\saveTG{𠤺}{71713}
\saveTG{𤇴}{71713}
\saveTG{𧈟}{71713}
\saveTG{𠥨}{71713}
\saveTG{𠤰}{71713}
\saveTG{𠥉}{71714}
\saveTG{𠥙}{71714}
\saveTG{𠥝}{71714}
\saveTG{鼴}{71714}
\saveTG{匽}{71714}
\saveTG{匴}{71714}
\saveTG{区}{71714}
\saveTG{𠤼}{71714}
\saveTG{𣱐}{71714}
\saveTG{𠥌}{71714}
\saveTG{𠥚}{71714}
\saveTG{𡠷}{71714}
\saveTG{匥}{71714}
\saveTG{𨲢}{71714}
\saveTG{𨱶}{71714}
\saveTG{𠥖}{71714}
\saveTG{𠥯}{71714}
\saveTG{匣}{71715}
\saveTG{匰}{71715}
\saveTG{𠥎}{71715}
\saveTG{𠥅}{71715}
\saveTG{𠥲}{71715}
\saveTG{𠥠}{71715}
\saveTG{𠥢}{71715}
\saveTG{匿}{71716}
\saveTG{區}{71716}
\saveTG{叵}{71716}
\saveTG{匲}{71716}
\saveTG{𪟰}{71716}
\saveTG{𪟱}{71716}
\saveTG{𠤽}{71716}
\saveTG{㔯}{71716}
\saveTG{𠥇}{71716}
\saveTG{𠥵}{71716}
\saveTG{𠥋}{71716}
\saveTG{匫}{71716}
\saveTG{𠤿}{71716}
\saveTG{𠥤}{71716}
\saveTG{𠤳}{71716}
\saveTG{㔱}{71716}
\saveTG{𠥏}{71716}
\saveTG{𠥡}{71716}
\saveTG{𠥺}{71716}
\saveTG{𦟾}{71716}
\saveTG{𠥀}{71716}
\saveTG{匼}{71716}
\saveTG{瓯}{71717}
\saveTG{𠹬}{71717}
\saveTG{𪕀}{71717}
\saveTG{𤬴}{71717}
\saveTG{𤬵}{71717}
\saveTG{瓺}{71717}
\saveTG{𠥒}{71717}
\saveTG{甌}{71717}
\saveTG{匶}{71717}
\saveTG{巨}{71717}
\saveTG{乬}{71717}
\saveTG{𪓧}{71717}
\saveTG{𤮖}{71717}
\saveTG{㼢}{71717}
\saveTG{𠥱}{71717}
\saveTG{𠤸}{71717}
\saveTG{𠤭}{71718}
\saveTG{𠥗}{71718}
\saveTG{𠥊}{71718}
\saveTG{㔴}{71718}
\saveTG{𠥦}{71718}
\saveTG{㔵}{71718}
\saveTG{𠥈}{71718}
\saveTG{𠥜}{71718}
\saveTG{𪟲}{71718}
\saveTG{𠥛}{71718}
\saveTG{𠤱}{71718}
\saveTG{𠥘}{71718}
\saveTG{𠤴}{71718}
\saveTG{𠥁}{71718}
\saveTG{𠥂}{71718}
\saveTG{𠥐}{71718}
\saveTG{匵}{71718}
\saveTG{匮}{71718}
\saveTG{匱}{71718}
\saveTG{匛}{71718}
\saveTG{匳}{71718}
\saveTG{匧}{71718}
\saveTG{医}{71718}
\saveTG{㔶}{71718}
\saveTG{𠤯}{71719}
\saveTG{𠤻}{71719}
\saveTG{𠥑}{71719}
\saveTG{𩢓}{71723}
\saveTG{𢀬}{71727}
\saveTG{镾}{71727}
\saveTG{𩢧}{71731}
\saveTG{䮍}{71732}
\saveTG{㕈}{71732}
\saveTG{𨲍}{71732}
\saveTG{𨱗}{71732}
\saveTG{長}{71732}
\saveTG{镸}{71732}
\saveTG{𡿴}{71737}
\saveTG{𡿰}{71737}
\saveTG{𨲙}{71738}
\saveTG{𫔗}{71738}
\saveTG{𨲀}{71741}
\saveTG{𪕔}{71742}
\saveTG{𦕁}{71742}
\saveTG{𪔾}{71744}
\saveTG{𢼧}{71747}
\saveTG{𢿚}{71747}
\saveTG{𢻷}{71747}
\saveTG{𢻰}{71747}
\saveTG{㪆}{71747}
\saveTG{𪔿}{71747}
\saveTG{𥀬}{71747}
\saveTG{𢾻}{71747}
\saveTG{敺}{71747}
\saveTG{䶄}{71749}
\saveTG{鼯}{71761}
\saveTG{𨱬}{71762}
\saveTG{𪕐}{71762}
\saveTG{鼫}{71762}
\saveTG{𪕲}{71766}
\saveTG{𨑅}{71774}
\saveTG{𪕚}{71777}
\saveTG{𦦈}{71777}
\saveTG{颐}{71782}
\saveTG{𫖵}{71782}
\saveTG{𫖱}{71782}
\saveTG{𩑝}{71786}
\saveTG{䫀}{71786}
\saveTG{𩒦}{71786}
\saveTG{𪕯}{71786}
\saveTG{頣}{71786}
\saveTG{頤}{71786}
\saveTG{䪸}{71786}
\saveTG{𩒥}{71786}
\saveTG{𩓴}{71786}
\saveTG{𩔸}{71786}
\saveTG{𩔿}{71786}
\saveTG{𩑼}{71786}
\saveTG{𩒑}{71786}
\saveTG{𩕿}{71786}
\saveTG{𩓆}{71786}
\saveTG{𩑾}{71786}
\saveTG{𩑥}{71786}
\saveTG{䦊}{71791}
\saveTG{𨱘}{71801}
\saveTG{𡘡}{71804}
\saveTG{𪥎}{71804}
\saveTG{𡗬}{71804}
\saveTG{𤋝}{71809}
\saveTG{𤌮}{71809}
\saveTG{𧹔}{71832}
\saveTG{赈}{71832}
\saveTG{𢿒}{71847}
\saveTG{𣀒}{71847}
\saveTG{贴}{71860}
\saveTG{𪩪}{71882}
\saveTG{𩕠}{71886}
\saveTG{𩖑}{71886}
\saveTG{㷳}{71891}
\saveTG{𥘐}{71901}
\saveTG{𡭳}{71902}
\saveTG{𥾩}{71903}
\saveTG{𩧌}{71904}
\saveTG{橜}{71904}
\saveTG{㯺}{71904}
\saveTG{𤬾}{71917}
\saveTG{㮣}{71941}
\saveTG{䯂}{71942}
\saveTG{𥤡}{71942}
\saveTG{𢿈}{71947}
\saveTG{𢾪}{71947}
\saveTG{䊠}{71949}
\saveTG{𣐃}{71981}
\saveTG{颡}{71982}
\saveTG{𠩼}{71982}
\saveTG{𢀲}{71986}
\saveTG{顙}{71986}
\saveTG{𪩦}{71991}
\saveTG{𠞄}{72000}
\saveTG{𠚴}{72000}
\saveTG{髩}{72027}
\saveTG{乓}{72031}
\saveTG{巛}{72037}
\saveTG{㔋}{72100}
\saveTG{𩨉}{72100}
\saveTG{㓸}{72100}
\saveTG{𠠇}{72100}
\saveTG{刞}{72100}
\saveTG{驯}{72100}
\saveTG{劉}{72100}
\saveTG{𠞈}{72100}
\saveTG{䯶}{72101}
\saveTG{𠀉}{72101}
\saveTG{𩬣}{72102}
\saveTG{丘}{72102}
\saveTG{𠀈}{72102}
\saveTG{䰐}{72102}
\saveTG{䰔}{72102}
\saveTG{𩮨}{72102}
\saveTG{𩯎}{72102}
\saveTG{𩯸}{72102}
\saveTG{䰃}{72102}
\saveTG{䰈}{72102}
\saveTG{𩬡}{72102}
\saveTG{𩭴}{72102}
\saveTG{𩬧}{72102}
\saveTG{𩬾}{72102}
\saveTG{𩭙}{72102}
\saveTG{𩭯}{72102}
\saveTG{𡌌}{72104}
\saveTG{𡊣}{72104}
\saveTG{𡋴}{72104}
\saveTG{𩮢}{72104}
\saveTG{𡍮}{72104}
\saveTG{坕}{72104}
\saveTG{髽}{72104}
\saveTG{𩬎}{72104}
\saveTG{𡏐}{72104}
\saveTG{𩫽}{72104}
\saveTG{𣂓}{72104}
\saveTG{𩭦}{72105}
\saveTG{𩭇}{72105}
\saveTG{𩯤}{72106}
\saveTG{𠠊}{72106}
\saveTG{𩮎}{72106}
\saveTG{𥪨}{72108}
\saveTG{𩯇}{72108}
\saveTG{𩮖}{72108}
\saveTG{𩬦}{72108}
\saveTG{髬}{72109}
\saveTG{髭}{72112}
\saveTG{𩯕}{72112}
\saveTG{𩨐}{72112}
\saveTG{𩬱}{72113}
\saveTG{𡑊}{72114}
\saveTG{𣱄}{72115}
\saveTG{㲯}{72115}
\saveTG{𩬰}{72117}
\saveTG{𩧱}{72117}
\saveTG{𩬛}{72117}
\saveTG{𢁎}{72117}
\saveTG{𩮗}{72118}
\saveTG{𠨬}{72120}
\saveTG{斵}{72121}
\saveTG{𣃃}{72121}
\saveTG{𣃋}{72121}
\saveTG{斲}{72121}
\saveTG{𠤅}{72122}
\saveTG{𩭅}{72127}
\saveTG{𩯮}{72127}
\saveTG{𨬞}{72127}
\saveTG{䢸}{72127}
\saveTG{𫘱}{72127}
\saveTG{𩮬}{72128}
\saveTG{骄}{72128}
\saveTG{髿}{72129}
\saveTG{𧎪}{72136}
\saveTG{𧓢}{72136}
\saveTG{𧌟}{72136}
\saveTG{𧈪}{72136}
\saveTG{𩮚}{72136}
\saveTG{𩭃}{72136}
\saveTG{𥁔}{72137}
\saveTG{𩰁}{72153}
\saveTG{𩯶}{72153}
\saveTG{𩥪}{72154}
\saveTG{𤔷}{72157}
\saveTG{𩭶}{72165}
\saveTG{𩨏}{72169}
\saveTG{骥}{72181}
\saveTG{䯸}{72182}
\saveTG{䰌}{72182}
\saveTG{骙}{72184}
\saveTG{𫘬}{72184}
\saveTG{𦢟}{72185}
\saveTG{𩰄}{72186}
\saveTG{𨻩}{72196}
\saveTG{𦛺}{72200}
\saveTG{𦘶}{72200}
\saveTG{𠜳}{72200}
\saveTG{𠠝}{72200}
\saveTG{㓹}{72200}
\saveTG{𠝄}{72200}
\saveTG{𠠏}{72200}
\saveTG{𠛚}{72200}
\saveTG{𠚾}{72200}
\saveTG{}{72200}
\saveTG{㓮}{72200}
\saveTG{剮}{72200}
\saveTG{刚}{72200}
\saveTG{𠜺}{72200}
\saveTG{䏖}{72200}
\saveTG{𠛒}{72200}
\saveTG{𫆠}{72200}
\saveTG{𠞂}{72200}
\saveTG{𠠮}{72200}
\saveTG{𠝭}{72200}
\saveTG{㔉}{72200}
\saveTG{𠝑}{72200}
\saveTG{𠜾}{72200}
\saveTG{𠝴}{72200}
\saveTG{𠛧}{72200}
\saveTG{𠜛}{72200}
\saveTG{𠝾}{72200}
\saveTG{𠛰}{72200}
\saveTG{𪱢}{72200}
\saveTG{㓾}{72200}
\saveTG{𠂆}{72200}
\saveTG{劚}{72200}
\saveTG{刖}{72200}
\saveTG{剭}{72200}
\saveTG{刷}{72200}
\saveTG{脷}{72200}
\saveTG{劂}{72200}
\saveTG{剧}{72200}
\saveTG{𠂋}{72201}
\saveTG{乒}{72201}
\saveTG{鬡}{72201}
\saveTG{鬖}{72202}
\saveTG{𦜋}{72202}
\saveTG{𩭹}{72202}
\saveTG{𩬖}{72202}
\saveTG{𩮜}{72202}
\saveTG{𩫶}{72207}
\saveTG{𩫾}{72207}
\saveTG{𩮅}{72207}
\saveTG{䯯}{72209}
\saveTG{肶}{72210}
\saveTG{阰}{72210}
\saveTG{骴}{72210}
\saveTG{颲}{72210}
\saveTG{𦙥}{72210}
\saveTG{𦜘}{72210}
\saveTG{𢨩}{72211}
\saveTG{䫽}{72211}
\saveTG{𩩆}{72211}
\saveTG{𩰀}{72211}
\saveTG{𪱚}{72211}
\saveTG{𩗰}{72211}
\saveTG{𩙑}{72211}
\saveTG{䬟}{72211}
\saveTG{𩘼}{72211}
\saveTG{𩬟}{72211}
\saveTG{𩘥}{72211}
\saveTG{䰕}{72212}
\saveTG{𨺾}{72212}
\saveTG{𩬌}{72212}
\saveTG{𩯆}{72212}
\saveTG{𩭂}{72212}
\saveTG{𩭕}{72212}
\saveTG{𩨼}{72212}
\saveTG{𡲌}{72212}
\saveTG{𫆞}{72212}
\saveTG{𢒝}{72212}
\saveTG{𦙜}{72212}
\saveTG{𩗦}{72212}
\saveTG{𩘃}{72212}
\saveTG{𩘛}{72212}
\saveTG{𩪳}{72212}
\saveTG{𠨗}{72212}
\saveTG{𩘉}{72212}
\saveTG{𪠭}{72212}
\saveTG{𡳲}{72212}
\saveTG{𩮷}{72212}
\saveTG{𠂘}{72212}
\saveTG{𢨷}{72212}
\saveTG{䝈}{72212}
\saveTG{𢨹}{72212}
\saveTG{𩯺}{72212}
\saveTG{颩}{72212}
\saveTG{髡}{72212}
\saveTG{髨}{72212}
\saveTG{臘}{72212}
\saveTG{卮}{72212}
\saveTG{𢩘}{72212}
\saveTG{𦚚}{72212}
\saveTG{𨹀}{72212}
\saveTG{𨸾}{72212}
\saveTG{𦠭}{72212}
\saveTG{𦚃}{72212}
\saveTG{𢨼}{72212}
\saveTG{𨹢}{72212}
\saveTG{脁}{72213}
\saveTG{𩖱}{72213}
\saveTG{𩗝}{72213}
\saveTG{朓}{72213}
\saveTG{𧲢}{72214}
\saveTG{𦨄}{72214}
\saveTG{𦘴}{72214}
\saveTG{𨼅}{72214}
\saveTG{䏕}{72214}
\saveTG{𩗌}{72214}
\saveTG{陛}{72214}
\saveTG{䬐}{72214}
\saveTG{𢩈}{72214}
\saveTG{𪱬}{72214}
\saveTG{𢨸}{72214}
\saveTG{𨺄}{72214}
\saveTG{膬}{72214}
\saveTG{𨼦}{72214}
\saveTG{垕}{72214}
\saveTG{䯗}{72214}
\saveTG{𦡘}{72215}
\saveTG{𤰀}{72215}
\saveTG{𣭽}{72215}
\saveTG{𩯬}{72215}
\saveTG{𦨃}{72215}
\saveTG{𩗲}{72215}
\saveTG{𩮲}{72215}
\saveTG{腫}{72215}
\saveTG{陲}{72215}
\saveTG{膗}{72215}
\saveTG{腄}{72215}
\saveTG{𣯱}{72215}
\saveTG{𩪓}{72215}
\saveTG{𦙤}{72215}
\saveTG{隀}{72215}
\saveTG{𩮞}{72215}
\saveTG{𩩞}{72215}
\saveTG{𩩳}{72215}
\saveTG{𩮴}{72215}
\saveTG{𩗮}{72216}
\saveTG{𨻵}{72216}
\saveTG{𪨝}{72216}
\saveTG{颳}{72216}
\saveTG{𩗇}{72216}
\saveTG{𠂬}{72217}
\saveTG{𩩾}{72217}
\saveTG{𦞡}{72217}
\saveTG{𧱒}{72217}
\saveTG{𦛐}{72217}
\saveTG{𣰢}{72217}
\saveTG{𣰝}{72217}
\saveTG{𣮮}{72217}
\saveTG{𣮱}{72217}
\saveTG{䬙}{72217}
\saveTG{𢩢}{72217}
\saveTG{𢨤}{72217}
\saveTG{𠑿}{72217}
\saveTG{𫙁}{72217}
\saveTG{𦙏}{72217}
\saveTG{䯆}{72217}
\saveTG{𩖞}{72217}
\saveTG{𩘈}{72217}
\saveTG{𠃤}{72217}
\saveTG{𩬜}{72217}
\saveTG{𩨛}{72217}
\saveTG{𨻆}{72217}
\saveTG{𩯓}{72217}
\saveTG{𩫼}{72217}
\saveTG{𩮓}{72217}
\saveTG{𠃘}{72217}
\saveTG{𠂮}{72217}
\saveTG{𦠼}{72217}
\saveTG{𢀴}{72217}
\saveTG{}{72217}
\saveTG{𫘻}{72217}
\saveTG{𢨴}{72217}
\saveTG{𢩒}{72217}
\saveTG{𢨪}{72217}
\saveTG{髠}{72217}
\saveTG{虒}{72217}
\saveTG{巵}{72217}
\saveTG{𧠨}{72217}
\saveTG{𤬕}{72217}
\saveTG{𩖷}{72217}
\saveTG{𩨨}{72217}
\saveTG{𩨷}{72217}
\saveTG{𪨕}{72217}
\saveTG{𦣂}{72218}
\saveTG{𨽆}{72218}
\saveTG{𩪂}{72218}
\saveTG{隑}{72218}
\saveTG{𩗭}{72218}
\saveTG{膯}{72218}
\saveTG{隥}{72218}
\saveTG{䐩}{72218}
\saveTG{𨤔}{72219}
\saveTG{𩗞}{72219}
\saveTG{䬆}{72219}
\saveTG{𩘊}{72219}
\saveTG{鬋}{72221}
\saveTG{所}{72221}
\saveTG{肵}{72221}
\saveTG{𨸢}{72221}
\saveTG{𣂤}{72221}
\saveTG{𦠠}{72221}
\saveTG{䏳}{72221}
\saveTG{䰑}{72221}
\saveTG{𩬒}{72221}
\saveTG{𩭿}{72221}
\saveTG{斦}{72221}
\saveTG{斸}{72221}
\saveTG{𩪽}{72221}
\saveTG{𦠀}{72221}
\saveTG{斤}{72221}
\saveTG{𣂥}{72221}
\saveTG{𣃁}{72221}
\saveTG{𢒁}{72222}
\saveTG{𩭁}{72222}
\saveTG{彫}{72222}
\saveTG{髳}{72222}
\saveTG{膨}{72222}
\saveTG{𣍼}{72222}
\saveTG{𩬏}{72222}
\saveTG{𨽑}{72222}
\saveTG{肜}{72222}
\saveTG{𩯨}{72223}
\saveTG{𩭉}{72224}
\saveTG{𣎂}{72224}
\saveTG{䏒}{72227}
\saveTG{𦢥}{72227}
\saveTG{𦢿}{72227}
\saveTG{𩭠}{72227}
\saveTG{𩭫}{72227}
\saveTG{𩬝}{72227}
\saveTG{𩫿}{72227}
\saveTG{𩯚}{72227}
\saveTG{𩬀}{72227}
\saveTG{㐆}{72227}
\saveTG{𢁺}{72227}
\saveTG{𫛲}{72227}
\saveTG{𫕌}{72227}
\saveTG{𦠽}{72227}
\saveTG{𩮘}{72227}
\saveTG{𩭔}{72227}
\saveTG{䰓}{72227}
\saveTG{𩮇}{72227}
\saveTG{𩮒}{72227}
\saveTG{𩭎}{72227}
\saveTG{𨺲}{72227}
\saveTG{𢩁}{72227}
\saveTG{䧦}{72227}
\saveTG{𦡓}{72227}
\saveTG{𩯜}{72227}
\saveTG{𩭗}{72227}
\saveTG{𩭀}{72227}
\saveTG{𩮼}{72227}
\saveTG{𨸴}{72227}
\saveTG{𩩼}{72227}
\saveTG{𦞇}{72227}
\saveTG{𠂰}{72227}
\saveTG{𩭬}{72227}
\saveTG{𩭛}{72227}
\saveTG{𩭊}{72227}
\saveTG{𦟹}{72227}
\saveTG{帋}{72227}
\saveTG{鬀}{72227}
\saveTG{腨}{72227}
\saveTG{髾}{72227}
\saveTG{鬝}{72227}
\saveTG{鬜}{72227}
\saveTG{鬅}{72227}
\saveTG{鬗}{72227}
\saveTG{膌}{72227}
\saveTG{乕}{72227}
\saveTG{髣}{72227}
\saveTG{髵}{72227}
\saveTG{鬄}{72227}
\saveTG{鬌}{72227}
\saveTG{𦢣}{72227}
\saveTG{䚩}{72227}
\saveTG{𦠚}{72227}
\saveTG{𩬉}{72227}
\saveTG{𢨺}{72227}
\saveTG{𦚌}{72227}
\saveTG{𩩯}{72227}
\saveTG{𧴑}{72227}
\saveTG{䯮}{72227}
\saveTG{𨹳}{72227}
\saveTG{𨻱}{72227}
\saveTG{𩬳}{72227}
\saveTG{𩮑}{72227}
\saveTG{䯾}{72227}
\saveTG{𩫹}{72227}
\saveTG{𩮐}{72227}
\saveTG{𩬮}{72227}
\saveTG{𢁹}{72227}
\saveTG{𫙃}{72227}
\saveTG{𦓝}{72227}
\saveTG{𦓚}{72227}
\saveTG{𦓙}{72227}
\saveTG{𦢕}{72227}
\saveTG{𩬓}{72228}
\saveTG{䯰}{72228}
\saveTG{𩯘}{72228}
\saveTG{𩮦}{72228}
\saveTG{𨸯}{72230}
\saveTG{𦙙}{72230}
\saveTG{𩨢}{72230}
\saveTG{𤓰}{72230}
\saveTG{爪}{72230}
\saveTG{瓜}{72230}
\saveTG{胍}{72230}
\saveTG{𦘯}{72230}
\saveTG{𦚝}{72230}
\saveTG{𣂙}{72230}
\saveTG{𦘷}{72231}
\saveTG{𩨯}{72231}
\saveTG{𩭑}{72231}
\saveTG{臐}{72231}
\saveTG{𢇏}{72231}
\saveTG{𦢜}{72231}
\saveTG{𦜙}{72231}
\saveTG{𩭐}{72232}
\saveTG{𧞐}{72232}
\saveTG{𧝦}{72232}
\saveTG{䰒}{72232}
\saveTG{𦚖}{72232}
\saveTG{鬞}{72232}
\saveTG{脈}{72232}
\saveTG{𢩆}{72232}
\saveTG{𠃄}{72232}
\saveTG{𦜥}{72232}
\saveTG{𢩠}{72232}
\saveTG{𩩺}{72232}
\saveTG{𩯙}{72232}
\saveTG{𤔙}{72232}
\saveTG{𠂢}{72232}
\saveTG{𩨶}{72232}
\saveTG{𩬐}{72232}
\saveTG{𪶰}{72232}
\saveTG{𦞓}{72232}
\saveTG{𫘾}{72232}
\saveTG{𩮡}{72232}
\saveTG{𤬑}{72233}
\saveTG{㼌}{72233}
\saveTG{胀}{72234}
\saveTG{𦞲}{72236}
\saveTG{𤫵}{72236}
\saveTG{䧝}{72236}
\saveTG{𨹤}{72236}
\saveTG{鬑}{72237}
\saveTG{隠}{72237}
\saveTG{隱}{72237}
\saveTG{䧙}{72238}
\saveTG{𦣛}{72239}
\saveTG{阺}{72240}
\saveTG{骶}{72240}
\saveTG{胝}{72240}
\saveTG{𣂑}{72240}
\saveTG{𦙆}{72240}
\saveTG{阡}{72240}
\saveTG{𨻾}{72240}
\saveTG{䯕}{72241}
\saveTG{𦅴}{72241}
\saveTG{𨺘}{72241}
\saveTG{𢩇}{72241}
\saveTG{𩭢}{72241}
\saveTG{脡}{72241}
\saveTG{腁}{72241}
\saveTG{斥}{72241}
\saveTG{脠}{72241}
\saveTG{𢩓}{72241}
\saveTG{𩬙}{72242}
\saveTG{脮}{72244}
\saveTG{腇}{72244}
\saveTG{骽}{72244}
\saveTG{䧌}{72244}
\saveTG{𢨭}{72244}
\saveTG{𢨬}{72244}
\saveTG{𦣅}{72246}
\saveTG{𩨩}{72247}
\saveTG{𦡙}{72247}
\saveTG{𦙀}{72247}
\saveTG{𠪔}{72247}
\saveTG{𨺡}{72247}
\saveTG{𢩄}{72247}
\saveTG{𧱭}{72247}
\saveTG{脬}{72247}
\saveTG{𢩑}{72247}
\saveTG{𡥭}{72247}
\saveTG{𣍶}{72247}
\saveTG{𠭙}{72247}
\saveTG{𨹴}{72247}
\saveTG{𫕉}{72247}
\saveTG{𩮤}{72247}
\saveTG{𡳜}{72247}
\saveTG{𩯌}{72247}
\saveTG{𩮕}{72247}
\saveTG{𨻎}{72247}
\saveTG{朡}{72247}
\saveTG{𢨦}{72247}
\saveTG{反}{72247}
\saveTG{髲}{72247}
\saveTG{阪}{72247}
\saveTG{𩮯}{72248}
\saveTG{𩯛}{72248}
\saveTG{𩭾}{72248}
\saveTG{𢨽}{72249}
\saveTG{𨸝}{72249}
\saveTG{脟}{72249}
\saveTG{𦞿}{72252}
\saveTG{𩰅}{72253}
\saveTG{𩯩}{72253}
\saveTG{𩯰}{72253}
\saveTG{䰏}{72253}
\saveTG{𩬕}{72253}
\saveTG{𩮏}{72253}
\saveTG{𦢍}{72256}
\saveTG{𫕋}{72257}
\saveTG{𦠄}{72258}
\saveTG{脂}{72261}
\saveTG{骺}{72261}
\saveTG{后}{72261}
\saveTG{𢩅}{72261}
\saveTG{𢩋}{72261}
\saveTG{𣍬}{72261}
\saveTG{𫇂}{72262}
\saveTG{腦}{72262}
\saveTG{階}{72262}
\saveTG{𦝨}{72262}
\saveTG{䐉}{72263}
\saveTG{𦡅}{72264}
\saveTG{𩩎}{72264}
\saveTG{盾}{72264}
\saveTG{腯}{72264}
\saveTG{𨺠}{72264}
\saveTG{䯏}{72264}
\saveTG{𩩻}{72264}
\saveTG{𦧍}{72264}
\saveTG{䏦}{72264}
\saveTG{𦞖}{72265}
\saveTG{𢩎}{72267}
\saveTG{𦠙}{72267}
\saveTG{𩯔}{72267}
\saveTG{𩮩}{72267}
\saveTG{膰}{72269}
\saveTG{𠩛}{72269}
\saveTG{𨼠}{72269}
\saveTG{𨺻}{72269}
\saveTG{𠨒}{72270}
\saveTG{𦘹}{72270}
\saveTG{𦙞}{72270}
\saveTG{𦝶}{72272}
\saveTG{𩭪}{72272}
\saveTG{𦞼}{72272}
\saveTG{𩨸}{72272}
\saveTG{𤝒}{72272}
\saveTG{𨺭}{72272}
\saveTG{朏}{72272}
\saveTG{胐}{72272}
\saveTG{𩩀}{72273}
\saveTG{𩯾}{72274}
\saveTG{𤬔}{72274}
\saveTG{𨺪}{72274}
\saveTG{䧟}{72277}
\saveTG{𢨯}{72277}
\saveTG{𢩕}{72277}
\saveTG{𢩝}{72277}
\saveTG{𦝥}{72277}
\saveTG{𢩖}{72277}
\saveTG{𢑚}{72277}
\saveTG{𢩙}{72277}
\saveTG{𨻘}{72281}
\saveTG{𦛼}{72281}
\saveTG{𨽯}{72281}
\saveTG{㰮}{72282}
\saveTG{质}{72282}
\saveTG{𦢾}{72282}
\saveTG{𣢬}{72282}
\saveTG{膎}{72284}
\saveTG{𦢂}{72284}
\saveTG{𦝢}{72284}
\saveTG{𢨻}{72284}
\saveTG{𢨾}{72284}
\saveTG{䧤}{72285}
\saveTG{𨽂}{72285}
\saveTG{𩪇}{72286}
\saveTG{𠃅}{72286}
\saveTG{鬚}{72286}
\saveTG{貭}{72286}
\saveTG{䑇}{72286}
\saveTG{𨻨}{72286}
\saveTG{𩯭}{72286}
\saveTG{𨻌}{72288}
\saveTG{䯼}{72289}
\saveTG{阥}{72290}
\saveTG{𨻍}{72291}
\saveTG{𦣋}{72293}
\saveTG{𩩌}{72293}
\saveTG{𤫺}{72293}
\saveTG{𦆩}{72293}
\saveTG{䧰}{72293}
\saveTG{𦟇}{72294}
\saveTG{𦢦}{72294}
\saveTG{𣍩}{72294}
\saveTG{𨹼}{72294}
\saveTG{𨺉}{72294}
\saveTG{𩭼}{72294}
\saveTG{𧳥}{72294}
\saveTG{髹}{72294}
\saveTG{𨽌}{72294}
\saveTG{隟}{72294}
\saveTG{䧨}{72295}
\saveTG{𩪑}{72295}
\saveTG{𦟳}{72295}
\saveTG{𦡧}{72295}
\saveTG{馴}{72300}
\saveTG{𩤲}{72300}
\saveTG{𩣫}{72300}
\saveTG{駲}{72300}
\saveTG{𩬔}{72302}
\saveTG{鬔}{72305}
\saveTG{𩬪}{72307}
\saveTG{𩯀}{72309}
\saveTG{駈}{72312}
\saveTG{𩧆}{72312}
\saveTG{𩢩}{72312}
\saveTG{駣}{72312}
\saveTG{𩣌}{72312}
\saveTG{𩦁}{72312}
\saveTG{𩣛}{72312}
\saveTG{𩢭}{72312}
\saveTG{駣}{72313}
\saveTG{騛}{72313}
\saveTG{馲}{72314}
\saveTG{䭷}{72315}
\saveTG{𫘐}{72315}
\saveTG{䮔}{72315}
\saveTG{𩤗}{72315}
\saveTG{䮴}{72316}
\saveTG{𩤽}{72317}
\saveTG{𩣤}{72317}
\saveTG{𩥉}{72318}
\saveTG{𩦠}{72320}
\saveTG{䮋}{72320}
\saveTG{馸}{72321}
\saveTG{𩣩}{72321}
\saveTG{𩢝}{72327}
\saveTG{𩧎}{72327}
\saveTG{𩧒}{72327}
\saveTG{𩦏}{72327}
\saveTG{𩮮}{72327}
\saveTG{驨}{72327}
\saveTG{驕}{72327}
\saveTG{𩥨}{72327}
\saveTG{𪃻}{72327}
\saveTG{𪄒}{72327}
\saveTG{𩤚}{72327}
\saveTG{𩦰}{72327}
\saveTG{𤇡}{72330}
\saveTG{𢤐}{72330}
\saveTG{𩢍}{72330}
\saveTG{𩯗}{72331}
\saveTG{𩯵}{72331}
\saveTG{𢗹}{72331}
\saveTG{𤌩}{72332}
\saveTG{𩥜}{72332}
\saveTG{𩭳}{72332}
\saveTG{𩮀}{72332}
\saveTG{𪫽}{72333}
\saveTG{䰄}{72336}
\saveTG{𩮰}{72336}
\saveTG{𩺄}{72336}
\saveTG{𩮪}{72337}
\saveTG{𩡬}{72337}
\saveTG{𢜄}{72337}
\saveTG{𩭤}{72338}
\saveTG{𩭮}{72338}
\saveTG{𩯑}{72338}
\saveTG{𩯲}{72338}
\saveTG{𩯥}{72338}
\saveTG{懸}{72339}
\saveTG{𩢈}{72341}
\saveTG{駳}{72341}
\saveTG{騈}{72341}
\saveTG{𡬼}{72342}
\saveTG{𩣧}{72344}
\saveTG{𩯄}{72346}
\saveTG{䮗}{72347}
\saveTG{𩢚}{72347}
\saveTG{𩣞}{72347}
\saveTG{𩦲}{72347}
\saveTG{騣}{72347}
\saveTG{驋}{72347}
\saveTG{𩥘}{72348}
\saveTG{䮑}{72349}
\saveTG{𩡴}{72349}
\saveTG{𩥴}{72350}
\saveTG{𩦋}{72353}
\saveTG{𩥸}{72353}
\saveTG{𩤠}{72362}
\saveTG{𩣯}{72363}
\saveTG{𩤘}{72363}
\saveTG{䮳}{72369}
\saveTG{𩣃}{72371}
\saveTG{𩢎}{72372}
\saveTG{𩤪}{72372}
\saveTG{𩥣}{72374}
\saveTG{䮢}{72377}
\saveTG{𩥅}{72377}
\saveTG{驥}{72381}
\saveTG{騤}{72384}
\saveTG{騱}{72384}
\saveTG{𩡻}{72384}
\saveTG{𩤻}{72385}
\saveTG{𩧄}{72386}
\saveTG{𩥗}{72388}
\saveTG{𩤧}{72389}
\saveTG{騬}{72391}
\saveTG{𩧂}{72391}
\saveTG{𩥷}{72391}
\saveTG{𩧖}{72393}
\saveTG{𩦀}{72394}
\saveTG{驜}{72395}
\saveTG{刪}{72400}
\saveTG{𠟉}{72400}
\saveTG{𠚹}{72400}
\saveTG{𠚫}{72400}
\saveTG{𠚿}{72400}
\saveTG{刐}{72400}
\saveTG{删}{72400}
\saveTG{剟}{72400}
\saveTG{𩮉}{72401}
\saveTG{髶}{72401}
\saveTG{𩮙}{72401}
\saveTG{𪦑}{72404}
\saveTG{𩬃}{72404}
\saveTG{𫘺}{72404}
\saveTG{𩬋}{72404}
\saveTG{氒}{72404}
\saveTG{𩭟}{72406}
\saveTG{髮}{72407}
\saveTG{鬉}{72407}
\saveTG{𩯣}{72407}
\saveTG{䯴}{72407}
\saveTG{䯹}{72407}
\saveTG{䯭}{72407}
\saveTG{𩮃}{72407}
\saveTG{髪}{72407}
\saveTG{鬘}{72407}
\saveTG{䯿}{72408}
\saveTG{𩬴}{72409}
\saveTG{髧}{72412}
\saveTG{𩭒}{72412}
\saveTG{𩯯}{72412}
\saveTG{𩰃}{72412}
\saveTG{𩬤}{72414}
\saveTG{毲}{72414}
\saveTG{𠃈}{72417}
\saveTG{𩮿}{72417}
\saveTG{𩫴}{72417}
\saveTG{𣃅}{72417}
\saveTG{彤}{72422}
\saveTG{𩮺}{72427}
\saveTG{𩫸}{72427}
\saveTG{𠭺}{72434}
\saveTG{𩰆}{72441}
\saveTG{𩯻}{72441}
\saveTG{𩬅}{72442}
\saveTG{𩬥}{72442}
\saveTG{𠦛}{72442}
\saveTG{䰋}{72442}
\saveTG{䯵}{72444}
\saveTG{𩯁}{72446}
\saveTG{𩮣}{72447}
\saveTG{𩯍}{72447}
\saveTG{髯}{72447}
\saveTG{䰉}{72447}
\saveTG{𦖵}{72447}
\saveTG{𩯉}{72447}
\saveTG{𫙂}{72449}
\saveTG{䰀}{72449}
\saveTG{𩭏}{72449}
\saveTG{𩯋}{72453}
\saveTG{䯷}{72453}
\saveTG{𡵕}{72470}
\saveTG{𩬸}{72471}
\saveTG{𠭴}{72472}
\saveTG{𠮐}{72477}
\saveTG{𦗌}{72477}
\saveTG{䫎}{72486}
\saveTG{𩭧}{72496}
\saveTG{𠜂}{72500}
\saveTG{𠛹}{72500}
\saveTG{𪮚}{72502}
\saveTG{𢮣}{72502}
\saveTG{髼}{72504}
\saveTG{𩮔}{72506}
\saveTG{𩬲}{72506}
\saveTG{鬇}{72507}
\saveTG{𩬶}{72507}
\saveTG{𣬭}{72517}
\saveTG{髴}{72527}
\saveTG{𩭝}{72553}
\saveTG{髥}{72557}
\saveTG{𠜜}{72600}
\saveTG{𩮋}{72601}
\saveTG{𩯼}{72601}
\saveTG{𩯐}{72601}
\saveTG{䯻}{72601}
\saveTG{𩭱}{72601}
\saveTG{𧮐}{72601}
\saveTG{𩬺}{72601}
\saveTG{髻}{72601}
\saveTG{鬐}{72601}
\saveTG{𩭡}{72601}
\saveTG{𩭆}{72601}
\saveTG{䯽}{72601}
\saveTG{𩭋}{72602}
\saveTG{𩠷}{72602}
\saveTG{𩬑}{72602}
\saveTG{𩭈}{72602}
\saveTG{𩯢}{72602}
\saveTG{䭮}{72602}
\saveTG{𩬞}{72602}
\saveTG{啠}{72602}
\saveTG{髫}{72602}
\saveTG{𩬼}{72602}
\saveTG{𩭣}{72603}
\saveTG{𩬠}{72603}
\saveTG{䯺}{72604}
\saveTG{䰇}{72604}
\saveTG{𩭽}{72604}
\saveTG{𩬩}{72604}
\saveTG{昏}{72604}
\saveTG{𥄇}{72604}
\saveTG{𩭓}{72604}
\saveTG{髺}{72604}
\saveTG{𩮈}{72604}
\saveTG{𨢏}{72604}
\saveTG{𫘼}{72605}
\saveTG{髷}{72605}
\saveTG{𩮝}{72605}
\saveTG{𩮟}{72605}
\saveTG{𩭺}{72606}
\saveTG{鬠}{72606}
\saveTG{鬙}{72606}
\saveTG{𩯅}{72606}
\saveTG{𫘿}{72607}
\saveTG{𩯠}{72608}
\saveTG{𩭻}{72608}
\saveTG{𩮠}{72608}
\saveTG{鬊}{72608}
\saveTG{𩭍}{72609}
\saveTG{𩭸}{72609}
\saveTG{𩭖}{72609}
\saveTG{𩭰}{72609}
\saveTG{𠃭}{72619}
\saveTG{鬍}{72627}
\saveTG{䃣}{72627}
\saveTG{𩭲}{72627}
\saveTG{𩯦}{72641}
\saveTG{𩰇}{72647}
\saveTG{𩯈}{72653}
\saveTG{𩰂}{72661}
\saveTG{𩯹}{72662}
\saveTG{𩭜}{72682}
\saveTG{𥓎}{72684}
\saveTG{𠛓}{72700}
\saveTG{𠠜}{72700}
\saveTG{𠛵}{72700}
\saveTG{刡}{72700}
\saveTG{剛}{72700}
\saveTG{𠜃}{72700}
\saveTG{𠚸}{72700}
\saveTG{剾}{72700}
\saveTG{𠛊}{72700}
\saveTG{𠛅}{72700}
\saveTG{𠛋}{72700}
\saveTG{𨲞}{72700}
\saveTG{𠞡}{72700}
\saveTG{𩙗}{72711}
\saveTG{𠂩}{72711}
\saveTG{𩬹}{72711}
\saveTG{𪖌}{72712}
\saveTG{髱}{72712}
\saveTG{髢}{72712}
\saveTG{鬣}{72712}
\saveTG{鬈}{72712}
\saveTG{𨱲}{72712}
\saveTG{𨱵}{72712}
\saveTG{𠄑}{72712}
\saveTG{𫘸}{72712}
\saveTG{𢁍}{72712}
\saveTG{𩗙}{72712}
\saveTG{𨱷}{72712}
\saveTG{𩰈}{72712}
\saveTG{𩬈}{72712}
\saveTG{𩯂}{72712}
\saveTG{𩗱}{72712}
\saveTG{𨲻}{72712}
\saveTG{𩭭}{72712}
\saveTG{鬛}{72712}
\saveTG{𩬬}{72712}
\saveTG{𩮻}{72712}
\saveTG{髦}{72714}
\saveTG{𦣨}{72714}
\saveTG{𩫷}{72714}
\saveTG{𨲚}{72714}
\saveTG{𣬮}{72715}
\saveTG{𨱞}{72715}
\saveTG{𨲝}{72715}
\saveTG{𨲉}{72715}
\saveTG{𦥻}{72715}
\saveTG{𣯅}{72715}
\saveTG{𨲦}{72716}
\saveTG{乺}{72717}
\saveTG{䯲}{72717}
\saveTG{𩫻}{72717}
\saveTG{𩯡}{72717}
\saveTG{𨚬}{72717}
\saveTG{鼶}{72717}
\saveTG{𣱎}{72717}
\saveTG{𩮊}{72717}
\saveTG{髰}{72717}
\saveTG{𨽰}{72717}
\saveTG{𪕑}{72717}
\saveTG{𫜣}{72718}
\saveTG{𠂑}{72720}
\saveTG{𪖉}{72721}
\saveTG{𣂻}{72721}
\saveTG{𣂜}{72721}
\saveTG{髟}{72722}
\saveTG{𩬵}{72727}
\saveTG{𩮂}{72727}
\saveTG{𨲭}{72727}
\saveTG{𦛎}{72727}
\saveTG{𨲲}{72727}
\saveTG{𩮭}{72727}
\saveTG{}{72727}
\saveTG{𨲰}{72727}
\saveTG{𩬆}{72727}
\saveTG{𩯝}{72727}
\saveTG{𩬗}{72727}
\saveTG{𩪱}{72731}
\saveTG{𩬨}{72731}
\saveTG{𫘽}{72731}
\saveTG{𣱋}{72732}
\saveTG{鬤}{72732}
\saveTG{𩞺}{72732}
\saveTG{𩯴}{72732}
\saveTG{鬟}{72732}
\saveTG{𩰉}{72732}
\saveTG{𩭨}{72732}
\saveTG{𩬿}{72732}
\saveTG{𩮍}{72732}
\saveTG{䯳}{72738}
\saveTG{𨱡}{72740}
\saveTG{氐}{72740}
\saveTG{氏}{72740}
\saveTG{𨲊}{72741}
\saveTG{鼮}{72741}
\saveTG{𩬁}{72742}
\saveTG{骶}{72743}
\saveTG{𨱩}{72744}
\saveTG{𨱳}{72744}
\saveTG{𩮾}{72747}
\saveTG{𠬿}{72747}
\saveTG{𪕽}{72747}
\saveTG{𪖆}{72747}
\saveTG{孵}{72747}
\saveTG{𪕉}{72749}
\saveTG{𩬍}{72757}
\saveTG{骺}{72761}
\saveTG{𩠞}{72762}
\saveTG{𩠢}{72762}
\saveTG{𫙀}{72764}
\saveTG{𪽡}{72769}
\saveTG{𪖇}{72769}
\saveTG{𪘷}{72772}
\saveTG{𨱦}{72772}
\saveTG{岳}{72772}
\saveTG{𩫺}{72772}
\saveTG{𩬢}{72772}
\saveTG{𩬷}{72773}
\saveTG{𩬚}{72774}
\saveTG{𤯁}{72774}
\saveTG{𩭵}{72777}
\saveTG{𩭥}{72777}
\saveTG{𦥝}{72777}
\saveTG{𩮱}{72777}
\saveTG{𩬂}{72782}
\saveTG{镺}{72784}
\saveTG{鼷}{72784}
\saveTG{䑑}{72785}
\saveTG{𩪛}{72785}
\saveTG{𪖈}{72785}
\saveTG{𩯫}{72786}
\saveTG{𪪀}{72794}
\saveTG{𪖊}{72795}
\saveTG{𩪤}{72795}
\saveTG{则}{72800}
\saveTG{劕}{72800}
\saveTG{𠠧}{72800}
\saveTG{𠛾}{72800}
\saveTG{鬓}{72801}
\saveTG{鬂}{72801}
\saveTG{兵}{72801}
\saveTG{髸}{72801}
\saveTG{鬒}{72801}
\saveTG{𣥉}{72801}
\saveTG{𠔊}{72801}
\saveTG{𪞋}{72801}
\saveTG{𩯒}{72802}
\saveTG{𩬄}{72802}
\saveTG{𩫵}{72802}
\saveTG{𩮁}{72804}
\saveTG{𫀯}{72804}
\saveTG{鬕}{72804}
\saveTG{𩭌}{72804}
\saveTG{𩬭}{72805}
\saveTG{𩯏}{72805}
\saveTG{𩬫}{72806}
\saveTG{𩯃}{72806}
\saveTG{䰎}{72806}
\saveTG{䰖}{72806}
\saveTG{𩯳}{72806}
\saveTG{𩯿}{72806}
\saveTG{鬢}{72806}
\saveTG{質}{72806}
\saveTG{䰅}{72806}
\saveTG{𫘹}{72807}
\saveTG{𡱸}{72807}
\saveTG{𩬊}{72809}
\saveTG{𢒵}{72822}
\saveTG{𩮆}{72822}
\saveTG{贬}{72832}
\saveTG{𤬡}{72833}
\saveTG{账}{72834}
\saveTG{𩬇}{72843}
\saveTG{贩}{72847}
\saveTG{𩮽}{72856}
\saveTG{䞌}{72864}
\saveTG{剁}{72900}
\saveTG{𩮳}{72901}
\saveTG{䯱}{72901}
\saveTG{鬃}{72901}
\saveTG{𩮛}{72903}
\saveTG{𩮹}{72903}
\saveTG{𥝌}{72904}
\saveTG{𩬻}{72904}
\saveTG{䰆}{72904}
\saveTG{𩬽}{72904}
\saveTG{𩮧}{72904}
\saveTG{䰂}{72904}
\saveTG{𩯞}{72904}
\saveTG{𩭄}{72904}
\saveTG{櫽}{72904}
\saveTG{檃}{72904}
\saveTG{髤}{72904}
\saveTG{乐}{72904}
\saveTG{𩭩}{72905}
\saveTG{𩯊}{72906}
\saveTG{𩬯}{72908}
\saveTG{𩬘}{72908}
\saveTG{𩯱}{72909}
\saveTG{𩭞}{72909}
\saveTG{䰁}{72909}
\saveTG{𩮄}{72915}
\saveTG{鬁}{72922}
\saveTG{鬎}{72922}
\saveTG{鬆}{72932}
\saveTG{𣓯}{72942}
\saveTG{𩭷}{72942}
\saveTG{𩭘}{72942}
\saveTG{𩯧}{72942}
\saveTG{𩮫}{72942}
\saveTG{𩭚}{72946}
\saveTG{𩯟}{72946}
\saveTG{𩮸}{72948}
\saveTG{𩮌}{72962}
\saveTG{𩯪}{72977}
\saveTG{𩮶}{72982}
\saveTG{𩯽}{72986}
\saveTG{鬏}{72989}
\saveTG{𩯖}{72991}
\saveTG{縣}{72993}
\saveTG{𨲁}{72998}
\saveTG{䰍}{72998}
\saveTG{𩮥}{72998}
\saveTG{𡍵}{73104}
\saveTG{𡋊}{73104}
\saveTG{䭺}{73107}
\saveTG{驴}{73107}
\saveTG{𣹹}{73111}
\saveTG{𩧻}{73112}
\saveTG{驼}{73112}
\saveTG{骖}{73122}
\saveTG{𩢺}{73127}
\saveTG{骗}{73127}
\saveTG{骟}{73127}
\saveTG{}{73132}
\saveTG{𦠸}{73132}
\saveTG{𧌌}{73136}
\saveTG{𧐶}{73136}
\saveTG{䯃}{73144}
\saveTG{𩧪}{73144}
\saveTG{骏}{73147}
\saveTG{𩧯}{73147}
\saveTG{𫘤}{73148}
\saveTG{𨪎}{73150}
\saveTG{𩧩}{73150}
\saveTG{䮙}{73150}
\saveTG{骀}{73160}
\saveTG{骔}{73191}
\saveTG{𦙦}{73200}
\saveTG{𦘱}{73200}
\saveTG{𨸼}{73204}
\saveTG{𩨲}{73204}
\saveTG{䏟}{73204}
\saveTG{𦙅}{73207}
\saveTG{䫾}{73211}
\saveTG{𩖶}{73211}
\saveTG{𨹶}{73211}
\saveTG{䬁}{73211}
\saveTG{陀}{73212}
\saveTG{𡳔}{73212}
\saveTG{𨺋}{73212}
\saveTG{𦠢}{73212}
\saveTG{𨺖}{73212}
\saveTG{𣍥}{73212}
\saveTG{𠧮}{73212}
\saveTG{𦚐}{73212}
\saveTG{𣎑}{73212}
\saveTG{院}{73212}
\saveTG{肬}{73212}
\saveTG{阭}{73212}
\saveTG{腕}{73212}
\saveTG{髋}{73212}
\saveTG{脘}{73212}
\saveTG{腔}{73212}
\saveTG{𩖻}{73213}
\saveTG{髖}{73213}
\saveTG{臗}{73213}
\saveTG{𩗖}{73213}
\saveTG{𩙥}{73214}
\saveTG{𪞒}{73214}
\saveTG{𨹏}{73214}
\saveTG{𩗎}{73214}
\saveTG{膣}{73214}
\saveTG{胧}{73214}
\saveTG{陇}{73214}
\saveTG{腟}{73214}
\saveTG{颰}{73214}
\saveTG{𡬉}{73214}
\saveTG{䬒}{73214}
\saveTG{䬎}{73215}
\saveTG{䬂}{73215}
\saveTG{𨿱}{73215}
\saveTG{䬄}{73215}
\saveTG{颱}{73216}
\saveTG{𩘎}{73216}
\saveTG{𩘰}{73216}
\saveTG{𫕍}{73216}
\saveTG{𦝻}{73216}
\saveTG{}{73216}
\saveTG{𫕂}{73217}
\saveTG{𨹕}{73217}
\saveTG{䖙}{73217}
\saveTG{䯛}{73217}
\saveTG{𨼭}{73217}
\saveTG{䯔}{73217}
\saveTG{𩨭}{73217}
\saveTG{䯘}{73217}
\saveTG{𨸵}{73217}
\saveTG{阸}{73217}
\saveTG{𩖮}{73218}
\saveTG{飇}{73218}
\saveTG{𡳓}{73219}
\saveTG{𩗕}{73219}
\saveTG{𪺖}{73221}
\saveTG{𦡲}{73221}
\saveTG{𦜸}{73222}
\saveTG{𦜝}{73224}
\saveTG{𨹆}{73227}
\saveTG{𤬊}{73227}
\saveTG{𣃐}{73227}
\saveTG{陠}{73227}
\saveTG{𦞏}{73227}
\saveTG{䐔}{73227}
\saveTG{䯙}{73227}
\saveTG{𢨲}{73227}
\saveTG{脯}{73227}
\saveTG{𦜛}{73227}
\saveTG{𦠤}{73230}
\saveTG{𩨓}{73230}
\saveTG{𦟺}{73230}
\saveTG{朖}{73232}
\saveTG{𦞭}{73232}
\saveTG{䯖}{73232}
\saveTG{𣍱}{73234}
\saveTG{㸕}{73234}
\saveTG{𦢑}{73235}
\saveTG{𤫳}{73236}
\saveTG{𦢉}{73236}
\saveTG{䧮}{73236}
\saveTG{𦠨}{73238}
\saveTG{㢥}{73240}
\saveTG{骮}{73240}
\saveTG{𫆛}{73240}
\saveTG{陚}{73240}
\saveTG{腻}{73240}
\saveTG{膩}{73240}
\saveTG{脦}{73240}
\saveTG{𢎐}{73240}
\saveTG{𦞤}{73241}
\saveTG{髆}{73242}
\saveTG{膊}{73242}
\saveTG{𠪻}{73243}
\saveTG{𦞷}{73244}
\saveTG{胺}{73244}
\saveTG{𨹖}{73247}
\saveTG{𥋜}{73247}
\saveTG{䯋}{73247}
\saveTG{𫕊}{73247}
\saveTG{𨺦}{73247}
\saveTG{𨸽}{73247}
\saveTG{胈}{73247}
\saveTG{瞂}{73247}
\saveTG{朘}{73247}
\saveTG{脧}{73247}
\saveTG{陖}{73247}
\saveTG{𨺮}{73247}
\saveTG{䧕}{73250}
\saveTG{𤋆}{73250}
\saveTG{隵}{73250}
\saveTG{𢦿}{73250}
\saveTG{戙}{73250}
\saveTG{𩯷}{73250}
\saveTG{㦺}{73250}
\saveTG{𢦡}{73250}
\saveTG{𢦹}{73250}
\saveTG{𢧯}{73250}
\saveTG{𤬹}{73250}
\saveTG{𢧿}{73250}
\saveTG{隇}{73250}
\saveTG{膱}{73250}
\saveTG{𢧘}{73250}
\saveTG{𢨛}{73250}
\saveTG{𢧕}{73250}
\saveTG{𢘱}{73250}
\saveTG{𨹒}{73250}
\saveTG{𨹚}{73250}
\saveTG{𨻒}{73250}
\saveTG{𢦲}{73251}
\saveTG{䑎}{73251}
\saveTG{𦢙}{73251}
\saveTG{𦛙}{73252}
\saveTG{𢦓}{73252}
\saveTG{䏬}{73252}
\saveTG{𢧓}{73253}
\saveTG{䏼}{73253}
\saveTG{䧖}{73253}
\saveTG{𢧦}{73254}
\saveTG{𢦩}{73254}
\saveTG{𦞁}{73256}
\saveTG{𩩴}{73256}
\saveTG{𢦽}{73256}
\saveTG{𢧒}{73256}
\saveTG{𢧊}{73258}
\saveTG{𫀄}{73259}
\saveTG{𦟠}{73259}
\saveTG{胎}{73260}
\saveTG{𨼈}{73261}
\saveTG{𨻴}{73261}
\saveTG{𪠗}{73261}
\saveTG{𦞨}{73262}
\saveTG{𦟱}{73262}
\saveTG{髂}{73264}
\saveTG{𦝣}{73264}
\saveTG{𦜵}{73264}
\saveTG{𦟈}{73265}
\saveTG{𩪃}{73265}
\saveTG{𦞳}{73268}
\saveTG{𩘪}{73268}
\saveTG{𦟽}{73272}
\saveTG{𨻅}{73272}
\saveTG{𦡆}{73272}
\saveTG{㞝}{73277}
\saveTG{𦜐}{73277}
\saveTG{𢩗}{73277}
\saveTG{膑}{73281}
\saveTG{髌}{73281}
\saveTG{腚}{73281}
\saveTG{𩩏}{73281}
\saveTG{䧑}{73282}
\saveTG{𦙮}{73282}
\saveTG{𦜀}{73284}
\saveTG{𤟢}{73284}
\saveTG{𤟵}{73284}
\saveTG{𤡵}{73284}
\saveTG{肰}{73284}
\saveTG{𦠎}{73284}
\saveTG{𦝬}{73284}
\saveTG{𦢴}{73284}
\saveTG{𦜏}{73284}
\saveTG{䧬}{73286}
\saveTG{髕}{73286}
\saveTG{臏}{73286}
\saveTG{𩪯}{73286}
\saveTG{𦢎}{73286}
\saveTG{𦟘}{73286}
\saveTG{腙}{73291}
\saveTG{脉}{73292}
\saveTG{𦟅}{73294}
\saveTG{𨽒}{73296}
\saveTG{脙}{73299}
\saveTG{𩡭}{73300}
\saveTG{駜}{73304}
\saveTG{馿}{73307}
\saveTG{𩤒}{73312}
\saveTG{𩦝}{73312}
\saveTG{馻}{73312}
\saveTG{駀}{73312}
\saveTG{駝}{73312}
\saveTG{𩣵}{73312}
\saveTG{駹}{73314}
\saveTG{𩧑}{73314}
\saveTG{𩢵}{73314}
\saveTG{驂}{73322}
\saveTG{𩣼}{73322}
\saveTG{騙}{73327}
\saveTG{䮒}{73327}
\saveTG{騸}{73327}
\saveTG{𢛶}{73330}
\saveTG{駺}{73332}
\saveTG{𩦊}{73333}
\saveTG{㥻}{73336}
\saveTG{䮁}{73341}
\saveTG{䮨}{73341}
\saveTG{𫘒}{73343}
\saveTG{𩣑}{73344}
\saveTG{䮟}{73347}
\saveTG{䮂}{73347}
\saveTG{駿}{73347}
\saveTG{𩢅}{73350}
\saveTG{𩤭}{73350}
\saveTG{𩥳}{73350}
\saveTG{䮅}{73350}
\saveTG{𩤥}{73350}
\saveTG{𩦕}{73350}
\saveTG{驖}{73350}
\saveTG{駥}{73350}
\saveTG{駴}{73350}
\saveTG{𩧀}{73350}
\saveTG{騀}{73350}
\saveTG{𩦷}{73350}
\saveTG{𩦤}{73350}
\saveTG{𩥼}{73350}
\saveTG{𩣊}{73350}
\saveTG{𩤊}{73353}
\saveTG{駘}{73360}
\saveTG{𩥿}{73362}
\saveTG{𩥥}{73362}
\saveTG{𩥌}{73365}
\saveTG{䮿}{73382}
\saveTG{䭾}{73384}
\saveTG{𩢰}{73384}
\saveTG{騃}{73384}
\saveTG{𩦿}{73386}
\saveTG{驞}{73386}
\saveTG{騌}{73391}
\saveTG{𩥵}{73398}
\saveTG{𡞣}{73445}
\saveTG{戏}{73450}
\saveTG{𢵤}{73502}
\saveTG{𪉳}{73602}
\saveTG{㗤}{73605}
\saveTG{𤊸}{73609}
\saveTG{𦦐}{73643}
\saveTG{卧}{73700}
\saveTG{𦣥}{73704}
\saveTG{𨳅}{73711}
\saveTG{鼧}{73712}
\saveTG{𪕎}{73717}
\saveTG{𦥣}{73717}
\saveTG{𨲸}{73721}
\saveTG{𨲜}{73727}
\saveTG{䭆}{73737}
\saveTG{𨲱}{73738}
\saveTG{𪕞}{73742}
\saveTG{𨲄}{73742}
\saveTG{䶈}{73743}
\saveTG{鼥}{73747}
\saveTG{𦣣}{73750}
\saveTG{𫇇}{73750}
\saveTG{𨱿}{73750}
\saveTG{𨲹}{73750}
\saveTG{𪖎}{73751}
\saveTG{𪖋}{73751}
\saveTG{𨲟}{73768}
\saveTG{𪕏}{73782}
\saveTG{𦥨}{73782}
\saveTG{鼣}{73784}
\saveTG{𪕟}{73784}
\saveTG{鼵}{73784}
\saveTG{𨲺}{73786}
\saveTG{𨲇}{73791}
\saveTG{贮}{73812}
\saveTG{赋}{73840}
\saveTG{赙}{73842}
\saveTG{贼}{73850}
\saveTG{贱}{73850}
\saveTG{𪭖}{73850}
\saveTG{贻}{73860}
\saveTG{赇}{73899}
\saveTG{𦄉}{73903}
\saveTG{𥼿}{73904}
\saveTG{𢎊}{73940}
\saveTG{𣙅}{73945}
\saveTG{𢧱}{73950}
\saveTG{𣜽}{73950}
\saveTG{𢰙}{74043}
\saveTG{驸}{74100}
\saveTG{𥁺}{74102}
\saveTG{𥂠}{74102}
\saveTG{𧖺}{74102}
\saveTG{𪣪}{74104}
\saveTG{𡏹}{74104}
\saveTG{𨧦}{74104}
\saveTG{堕}{74104}
\saveTG{𡍃}{74104}
\saveTG{𡓗}{74104}
\saveTG{墮}{74104}
\saveTG{𡎶}{74104}
\saveTG{𥪻}{74108}
\saveTG{驰}{74112}
\saveTG{}{74112}
\saveTG{}{74112}
\saveTG{}{74115}
\saveTG{骑}{74121}
\saveTG{𪟘}{74127}
\saveTG{𠡍}{74127}
\saveTG{𨭛}{74127}
\saveTG{助}{74127}
\saveTG{𧕈}{74131}
\saveTG{𧏆}{74136}
\saveTG{𧉜}{74136}
\saveTG{𧐏}{74136}
\saveTG{螱}{74136}
\saveTG{驳}{74140}
\saveTG{骅}{74141}
\saveTG{𩧼}{74144}
\saveTG{𪣽}{74147}
\saveTG{𫗈}{74147}
\saveTG{𫘟}{74147}
\saveTG{𩣡}{74147}
\saveTG{𤿚}{74147}
\saveTG{𩧵}{74161}
\saveTG{𩦇}{74161}
\saveTG{𫘢}{74169}
\saveTG{𩢨}{74170}
\saveTG{驮}{74180}
\saveTG{骐}{74181}
\saveTG{𩨋}{74181}
\saveTG{𫘭}{74184}
\saveTG{𩧈}{74186}
\saveTG{𨯓}{74191}
\saveTG{胕}{74200}
\saveTG{阩}{74200}
\saveTG{尉}{74200}
\saveTG{𦝗}{74200}
\saveTG{𦡷}{74200}
\saveTG{附}{74200}
\saveTG{肘}{74200}
\saveTG{阧}{74200}
\saveTG{䦹}{74200}
\saveTG{𣂀}{74203}
\saveTG{𡭁}{74203}
\saveTG{㷉}{74203}
\saveTG{𦛻}{74203}
\saveTG{𤓺}{74203}
\saveTG{𣂄}{74203}
\saveTG{𪯬}{74203}
\saveTG{𦙒}{74203}
\saveTG{𪯯}{74203}
\saveTG{𦙰}{74210}
\saveTG{𦘪}{74210}
\saveTG{肚}{74210}
\saveTG{屗}{74210}
\saveTG{𠘰}{74210}
\saveTG{𪨙}{74211}
\saveTG{𡳖}{74211}
\saveTG{𩗷}{74211}
\saveTG{䐦}{74212}
\saveTG{䐈}{74212}
\saveTG{𩙇}{74212}
\saveTG{䬀}{74212}
\saveTG{䬅}{74212}
\saveTG{𩖙}{74212}
\saveTG{𩘡}{74212}
\saveTG{𩖯}{74212}
\saveTG{𡲫}{74212}
\saveTG{𠡨}{74212}
\saveTG{𡳱}{74212}
\saveTG{肔}{74212}
\saveTG{隓}{74212}
\saveTG{隢}{74212}
\saveTG{阤}{74212}
\saveTG{膮}{74212}
\saveTG{髐}{74212}
\saveTG{𦛿}{74212}
\saveTG{𦞟}{74212}
\saveTG{䐠}{74212}
\saveTG{㬻}{74212}
\saveTG{𨽐}{74212}
\saveTG{𨸪}{74212}
\saveTG{𦚳}{74212}
\saveTG{䏙}{74212}
\saveTG{𩪒}{74212}
\saveTG{𩪦}{74212}
\saveTG{𩪄}{74212}
\saveTG{𩩝}{74212}
\saveTG{𩘖}{74212}
\saveTG{𦚱}{74212}
\saveTG{𩙚}{74212}
\saveTG{𨽈}{74212}
\saveTG{𫆖}{74213}
\saveTG{䬉}{74213}
\saveTG{𡲨}{74213}
\saveTG{𩖢}{74213}
\saveTG{䬋}{74214}
\saveTG{𨻊}{74214}
\saveTG{𨼂}{74214}
\saveTG{陞}{74214}
\saveTG{䬞}{74214}
\saveTG{𥖑}{74214}
\saveTG{𦢀}{74214}
\saveTG{胿}{74214}
\saveTG{䯓}{74214}
\saveTG{𦜣}{74214}
\saveTG{陸}{74214}
\saveTG{𦛏}{74214}
\saveTG{㬸}{74214}
\saveTG{𩖠}{74214}
\saveTG{𨼰}{74214}
\saveTG{𩪝}{74214}
\saveTG{𡐠}{74214}
\saveTG{𩖽}{74214}
\saveTG{𩗓}{74214}
\saveTG{𩘧}{74214}
\saveTG{𨺍}{74215}
\saveTG{𨻜}{74215}
\saveTG{𨼜}{74215}
\saveTG{𦞦}{74215}
\saveTG{䑏}{74215}
\saveTG{𩀶}{74215}
\saveTG{𨿭}{74215}
\saveTG{𩁌}{74215}
\saveTG{𦡂}{74215}
\saveTG{𨻚}{74215}
\saveTG{𦡦}{74215}
\saveTG{腌}{74215}
\saveTG{𦞴}{74215}
\saveTG{颹}{74215}
\saveTG{髉}{74215}
\saveTG{𦟩}{74215}
\saveTG{𩗊}{74216}
\saveTG{腌}{74216}
\saveTG{𦠉}{74216}
\saveTG{𪞴}{74216}
\saveTG{骩}{74217}
\saveTG{𨼡}{74217}
\saveTG{𨸒}{74217}
\saveTG{𩨥}{74217}
\saveTG{𪢏}{74217}
\saveTG{𩖺}{74217}
\saveTG{朑}{74217}
\saveTG{肍}{74217}
\saveTG{𠙙}{74217}
\saveTG{𩗄}{74218}
\saveTG{𩘑}{74218}
\saveTG{䐍}{74218}
\saveTG{𦡽}{74218}
\saveTG{𩙯}{74218}
\saveTG{䬝}{74218}
\saveTG{䬊}{74218}
\saveTG{𩗆}{74219}
\saveTG{𩙭}{74219}
\saveTG{𩘝}{74219}
\saveTG{𩗉}{74219}
\saveTG{飉}{74219}
\saveTG{𩘏}{74219}
\saveTG{𢄂}{74220}
\saveTG{䐀}{74221}
\saveTG{陭}{74221}
\saveTG{𨺈}{74221}
\saveTG{𦟑}{74221}
\saveTG{𩩛}{74221}
\saveTG{𦟬}{74222}
\saveTG{𦢩}{74224}
\saveTG{𩩨}{74224}
\saveTG{䯇}{74227}
\saveTG{𪓏}{74227}
\saveTG{𩪏}{74227}
\saveTG{骻}{74227}
\saveTG{𦜤}{74227}
\saveTG{𦝦}{74227}
\saveTG{𨼢}{74227}
\saveTG{𩩜}{74227}
\saveTG{𨽋}{74227}
\saveTG{𫆼}{74227}
\saveTG{𠡰}{74227}
\saveTG{䯐}{74227}
\saveTG{𦜮}{74227}
\saveTG{𦛕}{74227}
\saveTG{𦛒}{74227}
\saveTG{𣎙}{74227}
\saveTG{𦜬}{74227}
\saveTG{䐽}{74227}
\saveTG{𦟯}{74227}
\saveTG{𠢭}{74227}
\saveTG{𠠿}{74227}
\saveTG{𠡲}{74227}
\saveTG{𠡮}{74227}
\saveTG{𫕕}{74227}
\saveTG{𣍫}{74227}
\saveTG{𠢑}{74227}
\saveTG{肋}{74227}
\saveTG{劶}{74227}
\saveTG{𨹯}{74227}
\saveTG{脇}{74227}
\saveTG{陓}{74227}
\saveTG{𨼞}{74227}
\saveTG{𧱼}{74227}
\saveTG{𨹌}{74227}
\saveTG{腩}{74227}
\saveTG{朒}{74227}
\saveTG{肭}{74227}
\saveTG{勵}{74227}
\saveTG{励}{74227}
\saveTG{阞}{74227}
\saveTG{胯}{74227}
\saveTG{𫇁}{74227}
\saveTG{𦝐}{74227}
\saveTG{䐒}{74227}
\saveTG{脪}{74227}
\saveTG{陏}{74227}
\saveTG{隋}{74227}
\saveTG{臈}{74227}
\saveTG{劤}{74227}
\saveTG{䑅}{74227}
\saveTG{𦛨}{74227}
\saveTG{䐭}{74227}
\saveTG{𢃃}{74227}
\saveTG{𤬎}{74227}
\saveTG{𠓔}{74227}
\saveTG{䯝}{74227}
\saveTG{𨺕}{74227}
\saveTG{𠠻}{74227}
\saveTG{䧀}{74230}
\saveTG{𫆘}{74230}
\saveTG{胁}{74230}
\saveTG{肽}{74230}
\saveTG{臙}{74231}
\saveTG{䏯}{74231}
\saveTG{𦛘}{74231}
\saveTG{𨼲}{74231}
\saveTG{𨻝}{74231}
\saveTG{𨽞}{74231}
\saveTG{𣎰}{74231}
\saveTG{𦝌}{74231}
\saveTG{𦢲}{74231}
\saveTG{𨹮}{74231}
\saveTG{𤫸}{74232}
\saveTG{𫆹}{74232}
\saveTG{肱}{74232}
\saveTG{朦}{74232}
\saveTG{脓}{74232}
\saveTG{胠}{74232}
\saveTG{阹}{74232}
\saveTG{随}{74232}
\saveTG{𩪷}{74232}
\saveTG{髓}{74232}
\saveTG{膸}{74232}
\saveTG{隨}{74232}
\saveTG{髄}{74232}
\saveTG{𨼿}{74232}
\saveTG{𧱞}{74232}
\saveTG{䑃}{74232}
\saveTG{𤔽}{74232}
\saveTG{䏮}{74232}
\saveTG{𣎨}{74232}
\saveTG{𦢪}{74232}
\saveTG{𦢬}{74233}
\saveTG{𦡯}{74235}
\saveTG{𩼏}{74236}
\saveTG{𦢐}{74238}
\saveTG{𦞃}{74238}
\saveTG{𦚁}{74238}
\saveTG{𦠐}{74240}
\saveTG{𩨟}{74240}
\saveTG{𢻮}{74240}
\saveTG{𡜝}{74240}
\saveTG{肗}{74240}
\saveTG{𠭜}{74240}
\saveTG{𩨚}{74240}
\saveTG{𦡴}{74241}
\saveTG{𡊃}{74241}
\saveTG{𩩃}{74241}
\saveTG{髒}{74241}
\saveTG{𨼣}{74241}
\saveTG{䐛}{74241}
\saveTG{隯}{74241}
\saveTG{𦙈}{74241}
\saveTG{𦡰}{74242}
\saveTG{𦢸}{74242}
\saveTG{䏢}{74242}
\saveTG{𦞒}{74243}
\saveTG{𩪎}{74244}
\saveTG{𦜭}{74244}
\saveTG{𦡢}{74244}
\saveTG{𨻤}{74244}
\saveTG{𦡐}{74245}
\saveTG{𦞻}{74246}
\saveTG{𦝄}{74247}
\saveTG{𩩡}{74247}
\saveTG{脖}{74247}
\saveTG{臒}{74247}
\saveTG{骳}{74247}
\saveTG{𦙓}{74247}
\saveTG{肢}{74247}
\saveTG{陵}{74247}
\saveTG{陂}{74247}
\saveTG{𥀮}{74247}
\saveTG{㿭}{74247}
\saveTG{𨸠}{74247}
\saveTG{𢻀}{74247}
\saveTG{𢺼}{74247}
\saveTG{𢻄}{74247}
\saveTG{臌}{74247}
\saveTG{𢻠}{74247}
\saveTG{𠫌}{74247}
\saveTG{𦟵}{74247}
\saveTG{𦛊}{74247}
\saveTG{𧱐}{74247}
\saveTG{𩩉}{74247}
\saveTG{𩨝}{74247}
\saveTG{𡦝}{74247}
\saveTG{𩩟}{74247}
\saveTG{𨺎}{74247}
\saveTG{𫆲}{74247}
\saveTG{𨺶}{74247}
\saveTG{䐻}{74248}
\saveTG{𦟮}{74248}
\saveTG{𦚄}{74248}
\saveTG{膵}{74248}
\saveTG{𦚬}{74249}
\saveTG{阵}{74250}
\saveTG{𣍻}{74251}
\saveTG{臓}{74253}
\saveTG{臟}{74253}
\saveTG{䯦}{74253}
\saveTG{𦛥}{74253}
\saveTG{𦠜}{74254}
\saveTG{䐙}{74256}
\saveTG{𩩸}{74256}
\saveTG{𦝛}{74257}
\saveTG{𢩍}{74260}
\saveTG{陹}{74260}
\saveTG{𡳵}{74260}
\saveTG{𦙶}{74260}
\saveTG{骷}{74260}
\saveTG{𡳰}{74260}
\saveTG{陼}{74260}
\saveTG{𣎯}{74261}
\saveTG{䧊}{74261}
\saveTG{𦛋}{74261}
\saveTG{腊}{74261}
\saveTG{𦞂}{74261}
\saveTG{𫆾}{74261}
\saveTG{𧨧}{74261}
\saveTG{𦢧}{74261}
\saveTG{𨻇}{74261}
\saveTG{䚛}{74261}
\saveTG{𦛩}{74261}
\saveTG{㬶}{74261}
\saveTG{𦞯}{74261}
\saveTG{𡳛}{74261}
\saveTG{𨼩}{74261}
\saveTG{䜐}{74261}
\saveTG{𦢃}{74261}
\saveTG{𦛚}{74262}
\saveTG{𦠏}{74264}
\saveTG{䐗}{74264}
\saveTG{𦢁}{74267}
\saveTG{𨹞}{74268}
\saveTG{𦚕}{74270}
\saveTG{𡽃}{74272}
\saveTG{𦧃}{74277}
\saveTG{𪼳}{74280}
\saveTG{𫔺}{74280}
\saveTG{陡}{74281}
\saveTG{𨺌}{74281}
\saveTG{䐜}{74281}
\saveTG{㬴}{74281}
\saveTG{䧆}{74281}
\saveTG{𦜞}{74281}
\saveTG{𦝁}{74281}
\saveTG{𨼪}{74282}
\saveTG{𦟍}{74282}
\saveTG{臜}{74282}
\saveTG{𪱥}{74282}
\saveTG{𦛣}{74282}
\saveTG{𡳁}{74283}
\saveTG{𦚫}{74284}
\saveTG{𫕐}{74284}
\saveTG{膜}{74284}
\saveTG{𦟫}{74285}
\saveTG{朠}{74285}
\saveTG{𦢒}{74286}
\saveTG{膹}{74286}
\saveTG{䐵}{74286}
\saveTG{𩪹}{74286}
\saveTG{臢}{74286}
\saveTG{隫}{74286}
\saveTG{𦢌}{74286}
\saveTG{𦠻}{74286}
\saveTG{𨽤}{74286}
\saveTG{𨽍}{74286}
\saveTG{𪎼}{74286}
\saveTG{𪏎}{74286}
\saveTG{𪏍}{74286}
\saveTG{脥}{74288}
\saveTG{陝}{74288}
\saveTG{陜}{74288}
\saveTG{𣎧}{74289}
\saveTG{脄}{74289}
\saveTG{𩨿}{74289}
\saveTG{𦙣}{74290}
\saveTG{䏫}{74290}
\saveTG{𦝆}{74290}
\saveTG{𦡞}{74291}
\saveTG{𦝀}{74291}
\saveTG{䐑}{74294}
\saveTG{𦣑}{74294}
\saveTG{𨼙}{74294}
\saveTG{𨽬}{74294}
\saveTG{𩩥}{74294}
\saveTG{𨽣}{74294}
\saveTG{胨}{74294}
\saveTG{脎}{74294}
\saveTG{腜}{74294}
\saveTG{陈}{74294}
\saveTG{𦜔}{74294}
\saveTG{𦣌}{74294}
\saveTG{𩪫}{74295}
\saveTG{𫕔}{74296}
\saveTG{𩪚}{74296}
\saveTG{膫}{74296}
\saveTG{䧒}{74298}
\saveTG{𣍿}{74298}
\saveTG{膝}{74299}
\saveTG{𦝃}{74299}
\saveTG{𦡀}{74299}
\saveTG{𦢘}{74299}
\saveTG{駙}{74300}
\saveTG{𩡶}{74310}
\saveTG{𩥒}{74312}
\saveTG{馳}{74312}
\saveTG{馾}{74312}
\saveTG{駪}{74312}
\saveTG{驍}{74312}
\saveTG{𩤉}{74312}
\saveTG{𩥏}{74314}
\saveTG{𩦽}{74314}
\saveTG{𩣱}{74314}
\saveTG{𩦘}{74315}
\saveTG{䮤}{74315}
\saveTG{𩥕}{74315}
\saveTG{𩤔}{74315}
\saveTG{驩}{74315}
\saveTG{騹}{74315}
\saveTG{騎}{74321}
\saveTG{𩧓}{74324}
\saveTG{𩣋}{74327}
\saveTG{𩢆}{74327}
\saveTG{𩣁}{74327}
\saveTG{𩦞}{74327}
\saveTG{䮎}{74327}
\saveTG{𩢊}{74327}
\saveTG{𩣔}{74327}
\saveTG{𪅿}{74327}
\saveTG{𩢒}{74327}
\saveTG{𩣀}{74327}
\saveTG{𢗪}{74330}
\saveTG{駄}{74330}
\saveTG{怼}{74330}
\saveTG{慰}{74330}
\saveTG{𢢠}{74331}
\saveTG{驠}{74331}
\saveTG{䮃}{74331}
\saveTG{𢡢}{74332}
\saveTG{𩦺}{74332}
\saveTG{𩦌}{74332}
\saveTG{𩻍}{74336}
\saveTG{𩷷}{74336}
\saveTG{隳}{74338}
\saveTG{𩦾}{74340}
\saveTG{駁}{74340}
\saveTG{䮻}{74341}
\saveTG{𩦦}{74341}
\saveTG{𩣺}{74344}
\saveTG{騲}{74346}
\saveTG{䮮}{74347}
\saveTG{馶}{74347}
\saveTG{䮚}{74347}
\saveTG{駊}{74347}
\saveTG{𩦗}{74348}
\saveTG{驊}{74354}
\saveTG{𩤮}{74356}
\saveTG{𩢪}{74360}
\saveTG{𩤈}{74361}
\saveTG{𩥂}{74361}
\saveTG{𩢴}{74361}
\saveTG{𩥠}{74361}
\saveTG{𩥟}{74362}
\saveTG{𩢟}{74362}
\saveTG{𩤜}{74364}
\saveTG{𩤱}{74365}
\saveTG{馱}{74380}
\saveTG{騏}{74381}
\saveTG{𩥄}{74381}
\saveTG{𩧅}{74384}
\saveTG{𩦟}{74384}
\saveTG{䮬}{74384}
\saveTG{騻}{74384}
\saveTG{𩦥}{74386}
\saveTG{䮲}{74386}
\saveTG{䮌}{74390}
\saveTG{𩧇}{74391}
\saveTG{𩦭}{74394}
\saveTG{䮜}{74394}
\saveTG{驝}{74394}
\saveTG{𫘗}{74394}
\saveTG{𩦚}{74396}
\saveTG{騋}{74398}
\saveTG{对}{74400}
\saveTG{𨹱}{74402}
\saveTG{𣂊}{74403}
\saveTG{𡟨}{74404}
\saveTG{𪦠}{74404}
\saveTG{𡡙}{74404}
\saveTG{𠬜}{74410}
\saveTG{𡥰}{74412}
\saveTG{𠭑}{74421}
\saveTG{劝}{74427}
\saveTG{𦗟}{74431}
\saveTG{𢼪}{74440}
\saveTG{𢻍}{74447}
\saveTG{𤿔}{74447}
\saveTG{雘}{74447}
\saveTG{𤿵}{74447}
\saveTG{䵊}{74486}
\saveTG{犚}{74500}
\saveTG{𢲴}{74502}
\saveTG{𡬥}{74503}
\saveTG{𡬬}{74503}
\saveTG{𠢟}{74517}
\saveTG{𠯑}{74602}
\saveTG{𩡛}{74609}
\saveTG{𫇙}{74612}
\saveTG{𣋯}{74617}
\saveTG{𥕧}{74627}
\saveTG{𢽹}{74640}
\saveTG{𡝪}{74640}
\saveTG{𥀓}{74647}
\saveTG{𣊽}{74672}
\saveTG{𠥿}{74700}
\saveTG{𡬡}{74700}
\saveTG{𡬿}{74700}
\saveTG{尀}{74700}
\saveTG{𪯭}{74703}
\saveTG{𣁸}{74703}
\saveTG{𡬠}{74703}
\saveTG{毑}{74712}
\saveTG{𦟭}{74712}
\saveTG{䦈}{74712}
\saveTG{𥀩}{74714}
\saveTG{𦥟}{74717}
\saveTG{𪔽}{74717}
\saveTG{𠢔}{74727}
\saveTG{𠡗}{74727}
\saveTG{𨲴}{74727}
\saveTG{𪟙}{74727}
\saveTG{劻}{74727}
\saveTG{𨲕}{74727}
\saveTG{𨲂}{74731}
\saveTG{褽}{74732}
\saveTG{𫔖}{74732}
\saveTG{鼭}{74741}
\saveTG{𨱺}{74741}
\saveTG{𪕵}{74743}
\saveTG{𨱣}{74747}
\saveTG{𤿕}{74747}
\saveTG{㝀}{74747}
\saveTG{𢺾}{74747}
\saveTG{𨱜}{74747}
\saveTG{𠷞}{74760}
\saveTG{𨲤}{74761}
\saveTG{𪕖}{74761}
\saveTG{𨱻}{74761}
\saveTG{𨲘}{74764}
\saveTG{𨱫}{74770}
\saveTG{𡺆}{74772}
\saveTG{𡼐}{74772}
\saveTG{嶞}{74772}
\saveTG{𦣫}{74781}
\saveTG{𩠃}{74782}
\saveTG{𪖅}{74786}
\saveTG{贕}{74786}
\saveTG{𨳄}{74786}
\saveTG{镽}{74796}
\saveTG{𪩭}{74798}
\saveTG{财}{74800}
\saveTG{斞}{74800}
\saveTG{𨄯}{74801}
\saveTG{𦣶}{74801}
\saveTG{𡭓}{74803}
\saveTG{𡬰}{74807}
\saveTG{熨}{74809}
\saveTG{𧹑}{74827}
\saveTG{𠣆}{74827}
\saveTG{贿}{74827}
\saveTG{赌}{74860}
\saveTG{𪨐}{74883}
\saveTG{赎}{74884}
\saveTG{㯐}{74904}
\saveTG{𣞵}{74911}
\saveTG{𢻚}{74947}
\saveTG{𧋗}{75104}
\saveTG{塦}{75104}
\saveTG{𤦨}{75104}
\saveTG{驶}{75106}
\saveTG{骁}{75112}
\saveTG{䏝}{75122}
\saveTG{骕}{75127}
\saveTG{骋}{75127}
\saveTG{螴}{75136}
\saveTG{𦝏}{75136}
\saveTG{𩨃}{75140}
\saveTG{𨧬}{75144}
\saveTG{𩧬}{75147}
\saveTG{𩧨}{75160}
\saveTG{}{75162}
\saveTG{𩧭}{75180}
\saveTG{𫘝}{75182}
\saveTG{𩧫}{75182}
\saveTG{𥁍}{75182}
\saveTG{肼}{75200}
\saveTG{𤫬}{75200}
\saveTG{肨}{75200}
\saveTG{阱}{75200}
\saveTG{𨸡}{75202}
\saveTG{𠁲}{75205}
\saveTG{胂}{75206}
\saveTG{𡲭}{75206}
\saveTG{𦛈}{75206}
\saveTG{𣦆}{75206}
\saveTG{𨸬}{75206}
\saveTG{肿}{75206}
\saveTG{𡲝}{75206}
\saveTG{䦿}{75206}
\saveTG{陣}{75206}
\saveTG{𫆧}{75207}
\saveTG{𦛌}{75207}
\saveTG{𧰭}{75210}
\saveTG{胜}{75210}
\saveTG{𡲥}{75211}
\saveTG{𩖤}{75211}
\saveTG{𩖭}{75212}
\saveTG{𩗼}{75212}
\saveTG{𩘹}{75212}
\saveTG{𦙛}{75212}
\saveTG{𩙨}{75212}
\saveTG{𦠖}{75212}
\saveTG{𫗆}{75213}
\saveTG{𠙍}{75213}
\saveTG{𩗻}{75213}
\saveTG{𩗡}{75214}
\saveTG{𨑉}{75214}
\saveTG{𫕇}{75214}
\saveTG{𩘯}{75214}
\saveTG{𩙧}{75214}
\saveTG{𩖼}{75215}
\saveTG{𫗉}{75215}
\saveTG{𩗴}{75215}
\saveTG{}{75215}
\saveTG{䐸}{75216}
\saveTG{𩘳}{75216}
\saveTG{𩘤}{75216}
\saveTG{𣍁}{75216}
\saveTG{𦘧}{75216}
\saveTG{肫}{75217}
\saveTG{肒}{75217}
\saveTG{骫}{75217}
\saveTG{𣎖}{75217}
\saveTG{䫼}{75218}
\saveTG{𩙬}{75218}
\saveTG{𩗳}{75218}
\saveTG{颫}{75218}
\saveTG{𩘺}{75218}
\saveTG{𦡊}{75218}
\saveTG{𨼷}{75218}
\saveTG{體}{75218}
\saveTG{𫖼}{75219}
\saveTG{𡲧}{75219}
\saveTG{𩗣}{75219}
\saveTG{𩗂}{75219}
\saveTG{𩘢}{75219}
\saveTG{𩘣}{75219}
\saveTG{腗}{75221}
\saveTG{䐹}{75224}
\saveTG{肺}{75227}
\saveTG{胇}{75227}
\saveTG{胏}{75227}
\saveTG{𩩍}{75227}
\saveTG{腈}{75227}
\saveTG{骵}{75230}
\saveTG{𧎏}{75231}
\saveTG{𦜯}{75231}
\saveTG{𦝊}{75231}
\saveTG{𤫭}{75231}
\saveTG{脿}{75232}
\saveTG{膿}{75232}
\saveTG{}{75232}
\saveTG{𦣘}{75232}
\saveTG{𩩩}{75232}
\saveTG{𤫰}{75232}
\saveTG{𫆵}{75233}
\saveTG{䧥}{75233}
\saveTG{𧏁}{75234}
\saveTG{𨹁}{75236}
\saveTG{𦚭}{75236}
\saveTG{𦜦}{75236}
\saveTG{𦟪}{75236}
\saveTG{𩹙}{75236}
\saveTG{𦢔}{75237}
\saveTG{𤔅}{75238}
\saveTG{𨽟}{75238}
\saveTG{瓞}{75238}
\saveTG{䑊}{75238}
\saveTG{𤔏}{75239}
\saveTG{𨺩}{75240}
\saveTG{陦}{75240}
\saveTG{腱}{75240}
\saveTG{𡭍}{75243}
\saveTG{𡭐}{75243}
\saveTG{䧠}{75243}
\saveTG{膞}{75243}
\saveTG{𨻻}{75244}
\saveTG{𨹷}{75244}
\saveTG{膢}{75244}
\saveTG{髏}{75244}
\saveTG{𫆩}{75246}
\saveTG{䐟}{75247}
\saveTG{𫘳}{75247}
\saveTG{𨸱}{75247}
\saveTG{䏥}{75247}
\saveTG{𨹧}{75247}
\saveTG{𨸸}{75260}
\saveTG{𤉓}{75260}
\saveTG{𫆚}{75260}
\saveTG{𫔼}{75260}
\saveTG{𡳩}{75261}
\saveTG{𦚼}{75265}
\saveTG{𣍍}{75265}
\saveTG{䐬}{75266}
\saveTG{䐏}{75268}
\saveTG{𦠛}{75268}
\saveTG{𨺞}{75272}
\saveTG{陆}{75272}
\saveTG{𨺇}{75280}
\saveTG{肤}{75280}
\saveTG{陕}{75280}
\saveTG{胦}{75280}
\saveTG{胅}{75280}
\saveTG{脻}{75281}
\saveTG{𨹻}{75281}
\saveTG{䝊}{75281}
\saveTG{腆}{75281}
\saveTG{胰}{75282}
\saveTG{𩨰}{75282}
\saveTG{䦼}{75282}
\saveTG{䏐}{75282}
\saveTG{}{75282}
\saveTG{腠}{75284}
\saveTG{䯣}{75286}
\saveTG{膭}{75286}
\saveTG{隤}{75286}
\saveTG{𦚤}{75286}
\saveTG{𦟜}{75286}
\saveTG{𣍰}{75289}
\saveTG{䏞}{75290}
\saveTG{𦚜}{75290}
\saveTG{陎}{75290}
\saveTG{𨻰}{75290}
\saveTG{膆}{75293}
\saveTG{𦢭}{75294}
\saveTG{䏭}{75295}
\saveTG{陳}{75296}
\saveTG{脨}{75296}
\saveTG{腖}{75296}
\saveTG{𩢕}{75303}
\saveTG{𩢶}{75306}
\saveTG{𩢲}{75306}
\saveTG{駛}{75306}
\saveTG{䮇}{75307}
\saveTG{𫙟}{75307}
\saveTG{𩢀}{75317}
\saveTG{䭽}{75320}
\saveTG{騁}{75327}
\saveTG{馷}{75327}
\saveTG{驌}{75327}
\saveTG{𫘋}{75327}
\saveTG{𩥶}{75327}
\saveTG{𤊯}{75331}
\saveTG{𩤕}{75332}
\saveTG{騝}{75340}
\saveTG{𩧜}{75343}
\saveTG{䮫}{75344}
\saveTG{𩢡}{75347}
\saveTG{𩣬}{75347}
\saveTG{駎}{75360}
\saveTG{騞}{75362}
\saveTG{䮞}{75368}
\saveTG{𩨁}{75368}
\saveTG{𩥫}{75377}
\saveTG{駃}{75380}
\saveTG{䭿}{75380}
\saveTG{駚}{75380}
\saveTG{𩥲}{75380}
\saveTG{𫘆}{75380}
\saveTG{𩣲}{75381}
\saveTG{䮊}{75387}
\saveTG{駯}{75390}
\saveTG{𩢖}{75390}
\saveTG{䮪}{75394}
\saveTG{𩥚}{75394}
\saveTG{𩧐}{75396}
\saveTG{駷}{75396}
\saveTG{𦕊}{75427}
\saveTG{𦪱}{75447}
\saveTG{𤰁}{75512}
\saveTG{𨋺}{75706}
\saveTG{𨱽}{75706}
\saveTG{肆}{75707}
\saveTG{鼪}{75710}
\saveTG{𧏔}{75714}
\saveTG{𪕅}{75717}
\saveTG{𦣢}{75727}
\saveTG{䑔}{75727}
\saveTG{𦥥}{75727}
\saveTG{䶇}{75727}
\saveTG{𨱰}{75727}
\saveTG{鼱}{75727}
\saveTG{𪕕}{75731}
\saveTG{𨳆}{75732}
\saveTG{𨲳}{75732}
\saveTG{𨱸}{75757}
\saveTG{鼬}{75760}
\saveTG{镻}{75780}
\saveTG{䑖}{75781}
\saveTG{㲳}{75782}
\saveTG{赜}{75782}
\saveTG{𦣱}{75786}
\saveTG{𨲪}{75786}
\saveTG{𨲽}{75786}
\saveTG{賾}{75786}
\saveTG{𨱾}{75787}
\saveTG{𦥦}{75790}
\saveTG{𩠄}{75792}
\saveTG{𪨑}{75806}
\saveTG{贒}{75806}
\saveTG{𡱛}{75808}
\saveTG{}{75827}
\saveTG{䞍}{75827}
\saveTG{𧹗}{75836}
\saveTG{䞐}{75868}
\saveTG{䧅}{75877}
\saveTG{𧹖}{75881}
\saveTG{𩣳}{75960}
\saveTG{𪳫}{75960}
\saveTG{骃}{76100}
\saveTG{驷}{76100}
\saveTG{驲}{76100}
\saveTG{𩧽}{76100}
\saveTG{𪤝}{76104}
\saveTG{𫗇}{76110}
\saveTG{𫘩}{76114}
\saveTG{𧢗}{76117}
\saveTG{𫘥}{76117}
\saveTG{𣉠}{76127}
\saveTG{𧡸}{76127}
\saveTG{𩨅}{76127}
\saveTG{𩨀}{76127}
\saveTG{䯄}{76127}
\saveTG{𩨌}{76127}
\saveTG{骢}{76130}
\saveTG{𨭆}{76131}
\saveTG{𫓣}{76133}
\saveTG{𧕸}{76136}
\saveTG{𧓟}{76136}
\saveTG{𫘣}{76141}
\saveTG{騴}{76144}
\saveTG{𫘫}{76144}
\saveTG{𩧹}{76145}
\saveTG{𦐖}{76180}
\saveTG{𩢢}{76180}
\saveTG{𩨎}{76181}
\saveTG{𫘨}{76182}
\saveTG{𨆫}{76182}
\saveTG{骡}{76193}
\saveTG{骒}{76194}
\saveTG{㬷}{76200}
\saveTG{𦛢}{76200}
\saveTG{𣎫}{76200}
\saveTG{𦞢}{76200}
\saveTG{𧲥}{76200}
\saveTG{䦉}{76200}
\saveTG{𨹰}{76200}
\saveTG{䧃}{76200}
\saveTG{𦝔}{76200}
\saveTG{𪽛}{76200}
\saveTG{𤔩}{76200}
\saveTG{腘}{76200}
\saveTG{膕}{76200}
\saveTG{阳}{76200}
\saveTG{胭}{76200}
\saveTG{䐃}{76200}
\saveTG{𩪐}{76200}
\saveTG{胉}{76202}
\saveTG{𨺁}{76204}
\saveTG{䫻}{76210}
\saveTG{𦝇}{76210}
\saveTG{胆}{76210}
\saveTG{𡲉}{76210}
\saveTG{䬖}{76211}
\saveTG{𨹎}{76211}
\saveTG{𩗥}{76211}
\saveTG{𩙎}{76211}
\saveTG{䧋}{76212}
\saveTG{䏹}{76212}
\saveTG{𠩣}{76212}
\saveTG{𧡋}{76212}
\saveTG{𧢝}{76212}
\saveTG{𨹨}{76212}
\saveTG{𣎔}{76212}
\saveTG{䬑}{76212}
\saveTG{𩖸}{76212}
\saveTG{颺}{76212}
\saveTG{𠒰}{76212}
\saveTG{覵}{76212}
\saveTG{覸}{76212}
\saveTG{𨻀}{76212}
\saveTG{覛}{76212}
\saveTG{膍}{76212}
\saveTG{腽}{76212}
\saveTG{膃}{76212}
\saveTG{𩗺}{76212}
\saveTG{𧠢}{76212}
\saveTG{𦚢}{76212}
\saveTG{䐊}{76212}
\saveTG{𣍦}{76212}
\saveTG{𦙿}{76212}
\saveTG{𦛮}{76212}
\saveTG{𧡯}{76212}
\saveTG{𩙦}{76212}
\saveTG{𩗀}{76212}
\saveTG{𦤲}{76212}
\saveTG{𨻥}{76212}
\saveTG{䯠}{76212}
\saveTG{𩙊}{76212}
\saveTG{颸}{76213}
\saveTG{飔}{76213}
\saveTG{𩘘}{76213}
\saveTG{𦢮}{76213}
\saveTG{𦞙}{76213}
\saveTG{隗}{76213}
\saveTG{𠙣}{76213}
\saveTG{陧}{76214}
\saveTG{隍}{76214}
\saveTG{脭}{76214}
\saveTG{𩘙}{76214}
\saveTG{𪱦}{76214}
\saveTG{𦛠}{76214}
\saveTG{𨼳}{76214}
\saveTG{𦢓}{76214}
\saveTG{𩗫}{76214}
\saveTG{𩗤}{76214}
\saveTG{𦡃}{76215}
\saveTG{䬛}{76215}
\saveTG{𩘾}{76215}
\saveTG{䧉}{76215}
\saveTG{腥}{76215}
\saveTG{臞}{76215}
\saveTG{䤚}{76215}
\saveTG{𧠙}{76217}
\saveTG{𩴦}{76217}
\saveTG{𧢈}{76217}
\saveTG{𩪁}{76217}
\saveTG{𡳺}{76217}
\saveTG{𨹝}{76217}
\saveTG{𦜧}{76217}
\saveTG{𧡎}{76217}
\saveTG{𦚽}{76217}
\saveTG{𫘲}{76217}
\saveTG{𧡤}{76217}
\saveTG{𩳥}{76217}
\saveTG{𩩕}{76217}
\saveTG{𩪅}{76217}
\saveTG{𡳹}{76217}
\saveTG{𧠏}{76217}
\saveTG{𠙂}{76218}
\saveTG{𩗨}{76218}
\saveTG{𩗗}{76218}
\saveTG{𩙰}{76219}
\saveTG{𩙈}{76219}
\saveTG{𨻙}{76219}
\saveTG{𨺐}{76221}
\saveTG{𩩚}{76221}
\saveTG{𦜠}{76221}
\saveTG{隅}{76227}
\saveTG{𨺷}{76227}
\saveTG{𦢷}{76227}
\saveTG{䧎}{76227}
\saveTG{𦛭}{76227}
\saveTG{𣎅}{76227}
\saveTG{𦝲}{76227}
\saveTG{𦝩}{76227}
\saveTG{𩩲}{76227}
\saveTG{䯜}{76227}
\saveTG{𩩐}{76227}
\saveTG{髃}{76227}
\saveTG{臅}{76227}
\saveTG{𫕈}{76227}
\saveTG{𨺨}{76227}
\saveTG{𦚊}{76227}
\saveTG{㬽}{76227}
\saveTG{𦝖}{76227}
\saveTG{𩨴}{76227}
\saveTG{𨻠}{76227}
\saveTG{腸}{76227}
\saveTG{斶}{76227}
\saveTG{𨺸}{76227}
\saveTG{髑}{76227}
\saveTG{腭}{76227}
\saveTG{脶}{76227}
\saveTG{腢}{76227}
\saveTG{陽}{76227}
\saveTG{𨺬}{76228}
\saveTG{𨺯}{76230}
\saveTG{𦡵}{76230}
\saveTG{䚡}{76230}
\saveTG{𨻁}{76230}
\saveTG{䧭}{76230}
\saveTG{𣎗}{76230}
\saveTG{腮}{76230}
\saveTG{𦞜}{76230}
\saveTG{𤫽}{76231}
\saveTG{𩪺}{76231}
\saveTG{𤬒}{76231}
\saveTG{𣎚}{76231}
\saveTG{𪑐}{76231}
\saveTG{𩙢}{76232}
\saveTG{腲}{76232}
\saveTG{㼒}{76232}
\saveTG{隈}{76232}
\saveTG{𦣙}{76232}
\saveTG{𨻟}{76232}
\saveTG{𣎦}{76232}
\saveTG{𤂞}{76232}
\saveTG{䯥}{76233}
\saveTG{𦡹}{76233}
\saveTG{隰}{76233}
\saveTG{𦞗}{76234}
\saveTG{膙}{76236}
\saveTG{𤔑}{76236}
\saveTG{𦟣}{76240}
\saveTG{髀}{76240}
\saveTG{𦠣}{76240}
\saveTG{脾}{76240}
\saveTG{𨺒}{76240}
\saveTG{陴}{76240}
\saveTG{𩩙}{76240}
\saveTG{𨼸}{76241}
\saveTG{𦡡}{76241}
\saveTG{𨼍}{76241}
\saveTG{䏷}{76241}
\saveTG{䐾}{76241}
\saveTG{𥓓}{76242}
\saveTG{𦜉}{76242}
\saveTG{䑍}{76244}
\saveTG{𨻂}{76244}
\saveTG{𠕩}{76245}
\saveTG{𠧅}{76245}
\saveTG{𨼥}{76247}
\saveTG{𫆪}{76247}
\saveTG{𫆳}{76247}
\saveTG{𦠍}{76247}
\saveTG{𦚀}{76247}
\saveTG{𦜄}{76247}
\saveTG{𦞵}{76248}
\saveTG{𦟞}{76249}
\saveTG{𩨹}{76250}
\saveTG{胛}{76250}
\saveTG{𨸨}{76250}
\saveTG{𪨖}{76250}
\saveTG{𨸺}{76250}
\saveTG{𩪖}{76254}
\saveTG{䐷}{76256}
\saveTG{𨼒}{76256}
\saveTG{𨼗}{76260}
\saveTG{𣎇}{76260}
\saveTG{𩩫}{76260}
\saveTG{𦢏}{76260}
\saveTG{𦛗}{76262}
\saveTG{𨹬}{76262}
\saveTG{𡳎}{76262}
\saveTG{𨽉}{76264}
\saveTG{𡂶}{76266}
\saveTG{𨻞}{76268}
\saveTG{𨼓}{76268}
\saveTG{𨻽}{76269}
\saveTG{胑}{76280}
\saveTG{𩨵}{76280}
\saveTG{𦝈}{76281}
\saveTG{隄}{76281}
\saveTG{䟟}{76282}
\saveTG{𩩔}{76282}
\saveTG{𦛤}{76282}
\saveTG{陨}{76282}
\saveTG{䐎}{76282}
\saveTG{𨺆}{76284}
\saveTG{脵}{76284}
\saveTG{𣎱}{76284}
\saveTG{𨻬}{76284}
\saveTG{𦜃}{76284}
\saveTG{䐣}{76286}
\saveTG{𦢤}{76286}
\saveTG{隕}{76286}
\saveTG{𣎈}{76291}
\saveTG{腺}{76292}
\saveTG{𨻄}{76294}
\saveTG{𨹦}{76294}
\saveTG{臊}{76294}
\saveTG{髁}{76294}
\saveTG{腂}{76294}
\saveTG{𦣃}{76294}
\saveTG{𠪧}{76294}
\saveTG{𦢊}{76299}
\saveTG{駰}{76300}
\saveTG{駟}{76300}
\saveTG{馹}{76300}
\saveTG{𩤁}{76300}
\saveTG{𩢱}{76300}
\saveTG{𠺎}{76300}
\saveTG{𩣢}{76302}
\saveTG{𩤖}{76312}
\saveTG{𩥈}{76312}
\saveTG{騉}{76312}
\saveTG{騩}{76313}
\saveTG{𩥧}{76314}
\saveTG{騜}{76314}
\saveTG{𩧘}{76315}
\saveTG{𩤵}{76315}
\saveTG{䣖}{76317}
\saveTG{𫘍}{76326}
\saveTG{𩤸}{76327}
\saveTG{䮷}{76327}
\saveTG{𩤶}{76327}
\saveTG{𩥑}{76327}
\saveTG{𩤛}{76327}
\saveTG{𩦧}{76327}
\saveTG{騔}{76327}
\saveTG{駽}{76327}
\saveTG{𫚼}{76327}
\saveTG{𩤟}{76327}
\saveTG{騦}{76330}
\saveTG{驄}{76330}
\saveTG{𤓑}{76332}
\saveTG{𩦮}{76332}
\saveTG{𢠬}{76334}
\saveTG{駻}{76341}
\saveTG{驛}{76341}
\saveTG{𩦯}{76341}
\saveTG{騴}{76344}
\saveTG{𫘓}{76348}
\saveTG{驆}{76354}
\saveTG{驒}{76356}
\saveTG{䮖}{76360}
\saveTG{𩥺}{76366}
\saveTG{龭}{76381}
\saveTG{騠}{76381}
\saveTG{騡}{76392}
\saveTG{騾}{76393}
\saveTG{騍}{76394}
\saveTG{𧡏}{76412}
\saveTG{覌}{76412}
\saveTG{𠂄}{76427}
\saveTG{𦦣}{76447}
\saveTG{𣍔}{76466}
\saveTG{𨚉}{76517}
\saveTG{𥄤}{76700}
\saveTG{𪕈}{76700}
\saveTG{𩲰}{76713}
\saveTG{𦣇}{76715}
\saveTG{𪖏}{76715}
\saveTG{𧠟}{76717}
\saveTG{𧠠}{76717}
\saveTG{𨲎}{76727}
\saveTG{𨲾}{76727}
\saveTG{𪕭}{76727}
\saveTG{𪕩}{76727}
\saveTG{𨱭}{76727}
\saveTG{䑗}{76727}
\saveTG{𪕫}{76727}
\saveTG{𦥶}{76727}
\saveTG{𦦊}{76727}
\saveTG{𪕳}{76730}
\saveTG{㥸}{76730}
\saveTG{𦢺}{76731}
\saveTG{𢼑}{76740}
\saveTG{𨲩}{76742}
\saveTG{𨱮}{76742}
\saveTG{鼹}{76744}
\saveTG{𪕤}{76744}
\saveTG{𨲋}{76745}
\saveTG{𥏏}{76747}
\saveTG{鼺}{76760}
\saveTG{鼰}{76780}
\saveTG{𪕝}{76782}
\saveTG{鼳}{76784}
\saveTG{𪕜}{76784}
\saveTG{䐯}{76792}
\saveTG{𨲃}{76794}
\saveTG{咫}{76808}
\saveTG{贶}{76812}
\saveTG{𧹓}{76814}
\saveTG{𧢨}{76817}
\saveTG{赐}{76827}
\saveTG{䞏}{76830}
\saveTG{𤑹}{76841}
\saveTG{赗}{76860}
\saveTG{䞎}{76860}
\saveTG{𪲈}{76900}
\saveTG{𧠸}{76912}
\saveTG{𦣲}{76912}
\saveTG{門}{77001}
\saveTG{鬥}{77001}
\saveTG{𦥯}{77002}
\saveTG{𦉰}{77002}
\saveTG{𠃛}{77021}
\saveTG{𢸪}{77031}
\saveTG{閆}{77101}
\saveTG{閂}{77101}
\saveTG{𠬴}{77101}
\saveTG{𩐂}{77101}
\saveTG{閸}{77102}
\saveTG{𥂼}{77102}
\saveTG{𥂻}{77102}
\saveTG{𠁁}{77102}
\saveTG{𠨣}{77102}
\saveTG{𦊨}{77102}
\saveTG{𦊩}{77102}
\saveTG{𦊍}{77102}
\saveTG{𦊕}{77102}
\saveTG{𧖳}{77102}
\saveTG{𥁎}{77102}
\saveTG{𥂀}{77102}
\saveTG{𣦪}{77102}
\saveTG{𦦛}{77102}
\saveTG{皿}{77102}
\saveTG{𠀃}{77102}
\saveTG{𠀇}{77102}
\saveTG{𦚡}{77102}
\saveTG{𥁚}{77102}
\saveTG{舋}{77102}
\saveTG{𥃘}{77102}
\saveTG{𨷽}{77102}
\saveTG{闔}{77102}
\saveTG{盥}{77102}
\saveTG{叠}{77102}
\saveTG{且}{77102}
\saveTG{𨷆}{77102}
\saveTG{𣥙}{77102}
\saveTG{𥁉}{77102}
\saveTG{𥃜}{77102}
\saveTG{𥃖}{77102}
\saveTG{𥃝}{77102}
\saveTG{𧗕}{77102}
\saveTG{𥂞}{77102}
\saveTG{𥃔}{77102}
\saveTG{𥂤}{77102}
\saveTG{𨶂}{77102}
\saveTG{𨶩}{77102}
\saveTG{䦗}{77102}
\saveTG{𨵷}{77102}
\saveTG{𪾙}{77102}
\saveTG{𨵵}{77102}
\saveTG{𨵲}{77102}
\saveTG{𨵯}{77102}
\saveTG{瑿}{77103}
\saveTG{璺}{77103}
\saveTG{閠}{77103}
\saveTG{}{77104}
\saveTG{𪤀}{77104}
\saveTG{𡉮}{77104}
\saveTG{𡊎}{77104}
\saveTG{𨳝}{77104}
\saveTG{𨶾}{77104}
\saveTG{𤥔}{77104}
\saveTG{𨴗}{77104}
\saveTG{㼂}{77104}
\saveTG{𤪐}{77104}
\saveTG{㙠}{77104}
\saveTG{𡌭}{77104}
\saveTG{𤩱}{77104}
\saveTG{𦦍}{77104}
\saveTG{𡑺}{77104}
\saveTG{𨶴}{77104}
\saveTG{𦦋}{77104}
\saveTG{𨴘}{77104}
\saveTG{𨳞}{77104}
\saveTG{𨷙}{77104}
\saveTG{𡓓}{77104}
\saveTG{𨷅}{77104}
\saveTG{䦟}{77104}
\saveTG{𡒊}{77104}
\saveTG{圣}{77104}
\saveTG{𦦯}{77104}
\saveTG{垦}{77104}
\saveTG{堅}{77104}
\saveTG{閨}{77104}
\saveTG{壂}{77104}
\saveTG{墬}{77104}
\saveTG{堲}{77104}
\saveTG{闛}{77104}
\saveTG{壆}{77104}
\saveTG{閏}{77104}
\saveTG{朢}{77104}
\saveTG{𦦮}{77104}
\saveTG{𡉷}{77104}
\saveTG{𦤵}{77104}
\saveTG{𤪭}{77104}
\saveTG{𦊱}{77104}
\saveTG{𡊴}{77104}
\saveTG{䦌}{77104}
\saveTG{𤦁}{77104}
\saveTG{闉}{77104}
\saveTG{𡓵}{77104}
\saveTG{𦊝}{77104}
\saveTG{𨳳}{77104}
\saveTG{㻨}{77104}
\saveTG{𨵮}{77105}
\saveTG{𨶻}{77105}
\saveTG{𦌜}{77105}
\saveTG{𨴻}{77105}
\saveTG{𤯲}{77105}
\saveTG{𨶬}{77106}
\saveTG{𦋘}{77106}
\saveTG{𡱮}{77106}
\saveTG{昼}{77106}
\saveTG{𨷗}{77106}
\saveTG{䦔}{77106}
\saveTG{闓}{77108}
\saveTG{䝂}{77108}
\saveTG{𨶮}{77108}
\saveTG{𨶿}{77108}
\saveTG{竪}{77108}
\saveTG{閚}{77108}
\saveTG{毉}{77108}
\saveTG{豎}{77108}
\saveTG{闦}{77108}
\saveTG{𨴈}{77109}
\saveTG{𨮖}{77109}
\saveTG{𨮆}{77109}
\saveTG{䥣}{77109}
\saveTG{䦦}{77109}
\saveTG{𨥫}{77109}
\saveTG{𨩁}{77109}
\saveTG{𨦴}{77109}
\saveTG{𨴇}{77109}
\saveTG{鋻}{77109}
\saveTG{𫓓}{77109}
\saveTG{𨯜}{77109}
\saveTG{𩙔}{77110}
\saveTG{𧋜}{77110}
\saveTG{𨳡}{77111}
\saveTG{𨳎}{77111}
\saveTG{𨵈}{77111}
\saveTG{𦋛}{77111}
\saveTG{驵}{77112}
\saveTG{屔}{77112}
\saveTG{𨷺}{77112}
\saveTG{𨵨}{77112}
\saveTG{𠁠}{77112}
\saveTG{𨷇}{77112}
\saveTG{𨶚}{77113}
\saveTG{䦞}{77114}
\saveTG{𣫫}{77114}
\saveTG{𫔽}{77114}
\saveTG{𨴡}{77117}
\saveTG{𨴤}{77117}
\saveTG{𨵞}{77117}
\saveTG{鬮}{77117}
\saveTG{𪚴}{77117}
\saveTG{𠤫}{77117}
\saveTG{驹}{77120}
\saveTG{𩧲}{77120}
\saveTG{}{77120}
\saveTG{𠕥}{77120}
\saveTG{鬭}{77121}
\saveTG{鬬}{77121}
\saveTG{𨷖}{77121}
\saveTG{𨷵}{77121}
\saveTG{𨥔}{77121}
\saveTG{𪴿}{77121}
\saveTG{𧐰}{77123}
\saveTG{𪅤}{77127}
\saveTG{𪇖}{77127}
\saveTG{䮩}{77127}
\saveTG{𪀭}{77127}
\saveTG{马}{77127}
\saveTG{𪆱}{77127}
\saveTG{𨜹}{77127}
\saveTG{𨜏}{77127}
\saveTG{𨞌}{77127}
\saveTG{𠢈}{77127}
\saveTG{𩿨}{77127}
\saveTG{骉}{77127}
\saveTG{𫘦}{77127}
\saveTG{𩧺}{77127}
\saveTG{𨶵}{77127}
\saveTG{}{77127}
\saveTG{闒}{77127}
\saveTG{鴡}{77127}
\saveTG{𪃫}{77127}
\saveTG{𨞐}{77127}
\saveTG{闟}{77127}
\saveTG{邱}{77127}
\saveTG{翳}{77127}
\saveTG{鹥}{77127}
\saveTG{閯}{77129}
\saveTG{𨵗}{77131}
\saveTG{𧓹}{77131}
\saveTG{𨪊}{77131}
\saveTG{䦢}{77131}
\saveTG{𧔡}{77131}
\saveTG{𧔳}{77131}
\saveTG{𧕵}{77131}
\saveTG{𨶗}{77132}
\saveTG{𩨍}{77132}
\saveTG{骤}{77132}
\saveTG{𨷌}{77136}
\saveTG{𧔱}{77136}
\saveTG{𧕧}{77136}
\saveTG{𧖥}{77136}
\saveTG{蜰}{77136}
\saveTG{閩}{77136}
\saveTG{骚}{77136}
\saveTG{蟁}{77136}
\saveTG{𧎱}{77136}
\saveTG{𧈡}{77136}
\saveTG{𧊈}{77136}
\saveTG{𧎮}{77136}
\saveTG{𧌘}{77136}
\saveTG{𧈦}{77136}
\saveTG{𧎂}{77136}
\saveTG{𧎰}{77136}
\saveTG{䗟}{77136}
\saveTG{蚤}{77136}
\saveTG{蜸}{77136}
\saveTG{𨷛}{77136}
\saveTG{𨵕}{77136}
\saveTG{𨷷}{77136}
\saveTG{𨶣}{77136}
\saveTG{驭}{77140}
\saveTG{鬪}{77141}
\saveTG{闘}{77141}
\saveTG{𨶜}{77143}
\saveTG{𡝼}{77144}
\saveTG{𩨄}{77147}
\saveTG{𪔳}{77147}
\saveTG{𣪷}{77147}
\saveTG{𨷕}{77147}
\saveTG{𩧳}{77147}
\saveTG{}{77147}
\saveTG{𩾈}{77147}
\saveTG{𦋵}{77147}
\saveTG{䦯}{77147}
\saveTG{骎}{77147}
\saveTG{毀}{77147}
\saveTG{毁}{77147}
\saveTG{骣}{77147}
\saveTG{𠭊}{77147}
\saveTG{𨷡}{77151}
\saveTG{閾}{77153}
\saveTG{䍞}{77153}
\saveTG{驿}{77154}
\saveTG{𩧰}{77154}
\saveTG{𦤋}{77162}
\saveTG{𨵍}{77162}
\saveTG{骝}{77162}
\saveTG{𨫧}{77162}
\saveTG{𨵔}{77163}
\saveTG{𫇓}{77164}
\saveTG{𨶖}{77164}
\saveTG{骆}{77164}
\saveTG{闊}{77164}
\saveTG{𡰪}{77167}
\saveTG{𩢹}{77170}
\saveTG{𦦁}{77172}
\saveTG{𨶟}{77177}
\saveTG{驺}{77177}
\saveTG{𦍅}{77181}
\saveTG{𨷩}{77186}
\saveTG{𡲷}{77194}
\saveTG{𣬷}{77195}
\saveTG{𫘧}{77199}
\saveTG{𫔘}{77200}
\saveTG{閁}{77201}
\saveTG{閅}{77201}
\saveTG{𦦞}{77201}
\saveTG{𠬯}{77201}
\saveTG{𠕊}{77201}
\saveTG{𦦡}{77201}
\saveTG{𨶪}{77202}
\saveTG{𠕝}{77207}
\saveTG{𠃜}{77207}
\saveTG{𠁣}{77207}
\saveTG{𦥗}{77207}
\saveTG{𦦼}{77207}
\saveTG{尸}{77207}
\saveTG{𨴢}{77207}
\saveTG{凤}{77210}
\saveTG{风}{77210}
\saveTG{凨}{77210}
\saveTG{凬}{77210}
\saveTG{凮}{77210}
\saveTG{𨸔}{77210}
\saveTG{鳯}{77210}
\saveTG{鳳}{77210}
\saveTG{凲}{77210}
\saveTG{凰}{77210}
\saveTG{几}{77210}
\saveTG{肌}{77210}
\saveTG{凩}{77210}
\saveTG{夙}{77210}
\saveTG{骪}{77210}
\saveTG{阠}{77210}
\saveTG{風}{77210}
\saveTG{凧}{77210}
\saveTG{凪}{77210}
\saveTG{𨺢}{77210}
\saveTG{𩗧}{77210}
\saveTG{𦢗}{77210}
\saveTG{䏎}{77210}
\saveTG{𦣕}{77210}
\saveTG{𠘼}{77210}
\saveTG{𠙕}{77210}
\saveTG{𠫗}{77210}
\saveTG{肊}{77210}
\saveTG{𩡫}{77210}
\saveTG{𠱰}{77210}
\saveTG{𠘱}{77210}
\saveTG{𩨖}{77210}
\saveTG{𠘾}{77210}
\saveTG{𠂡}{77210}
\saveTG{𠙗}{77210}
\saveTG{𠙒}{77210}
\saveTG{㓘}{77210}
\saveTG{䥚}{77210}
\saveTG{𠘳}{77210}
\saveTG{㶡}{77210}
\saveTG{𡖆}{77210}
\saveTG{𠙔}{77210}
\saveTG{𠙈}{77210}
\saveTG{𠙄}{77210}
\saveTG{𠘴}{77210}
\saveTG{䬕}{77210}
\saveTG{𠘨}{77210}
\saveTG{𠙌}{77210}
\saveTG{𠙘}{77210}
\saveTG{𠘵}{77210}
\saveTG{𠘪}{77210}
\saveTG{𠙬}{77210}
\saveTG{𠝈}{77210}
\saveTG{𪞵}{77210}
\saveTG{凡}{77210}
\saveTG{𩗘}{77211}
\saveTG{𩖛}{77211}
\saveTG{𡰭}{77211}
\saveTG{𡳀}{77211}
\saveTG{𫕏}{77211}
\saveTG{𨴰}{77211}
\saveTG{颮}{77211}
\saveTG{屝}{77211}
\saveTG{飍}{77211}
\saveTG{𨹿}{77211}
\saveTG{飑}{77211}
\saveTG{𩖾}{77211}
\saveTG{𩗜}{77211}
\saveTG{𩖜}{77211}
\saveTG{𥜀}{77211}
\saveTG{𡲤}{77211}
\saveTG{𡱻}{77211}
\saveTG{𩗃}{77211}
\saveTG{𩗑}{77211}
\saveTG{𨴃}{77211}
\saveTG{𩘂}{77211}
\saveTG{𩙠}{77211}
\saveTG{𩖹}{77211}
\saveTG{𡰥}{77211}
\saveTG{陒}{77212}
\saveTG{兒}{77212}
\saveTG{兜}{77212}
\saveTG{脆}{77212}
\saveTG{骲}{77212}
\saveTG{胞}{77212}
\saveTG{𨷦}{77212}
\saveTG{𡳸}{77212}
\saveTG{𣌅}{77212}
\saveTG{𨴼}{77212}
\saveTG{𡳬}{77212}
\saveTG{𩗅}{77212}
\saveTG{𩘲}{77212}
\saveTG{𩘿}{77212}
\saveTG{𩘻}{77212}
\saveTG{𦙠}{77212}
\saveTG{𦡿}{77212}
\saveTG{閲}{77212}
\saveTG{閱}{77212}
\saveTG{飏}{77212}
\saveTG{脕}{77212}
\saveTG{见}{77212}
\saveTG{腉}{77212}
\saveTG{屁}{77212}
\saveTG{隉}{77212}
\saveTG{屍}{77212}
\saveTG{胒}{77212}
\saveTG{尼}{77212}
\saveTG{𨺙}{77212}
\saveTG{𩘦}{77212}
\saveTG{飂}{77212}
\saveTG{陉}{77212}
\saveTG{胫}{77212}
\saveTG{阻}{77212}
\saveTG{甩}{77212}
\saveTG{兕}{77212}
\saveTG{𧢂}{77212}
\saveTG{𨺛}{77212}
\saveTG{𩘇}{77212}
\saveTG{覺}{77212}
\saveTG{𧠈}{77212}
\saveTG{䧯}{77212}
\saveTG{鬩}{77212}
\saveTG{𨽊}{77212}
\saveTG{}{77212}
\saveTG{𧖴}{77212}
\saveTG{𩖴}{77212}
\saveTG{𩖚}{77212}
\saveTG{𩖩}{77212}
\saveTG{𩗈}{77212}
\saveTG{𩗪}{77212}
\saveTG{𩖨}{77212}
\saveTG{𩘐}{77212}
\saveTG{𡲊}{77212}
\saveTG{𡲀}{77212}
\saveTG{𡳆}{77212}
\saveTG{𡰱}{77212}
\saveTG{䫸}{77212}
\saveTG{𩗍}{77212}
\saveTG{𫌩}{77212}
\saveTG{𨻉}{77212}
\saveTG{𩖡}{77212}
\saveTG{𦚦}{77213}
\saveTG{𩙫}{77213}
\saveTG{䬍}{77213}
\saveTG{颾}{77213}
\saveTG{𡲙}{77214}
\saveTG{𡥲}{77214}
\saveTG{𡳈}{77214}
\saveTG{𡳑}{77214}
\saveTG{𨷠}{77214}
\saveTG{𨷭}{77214}
\saveTG{冦}{77214}
\saveTG{屘}{77214}
\saveTG{飕}{77214}
\saveTG{颼}{77214}
\saveTG{腛}{77214}
\saveTG{屋}{77214}
\saveTG{觃}{77214}
\saveTG{𨹡}{77214}
\saveTG{䏔}{77214}
\saveTG{𨺼}{77214}
\saveTG{尾}{77214}
\saveTG{𡦥}{77214}
\saveTG{𡲪}{77214}
\saveTG{𪨋}{77214}
\saveTG{𡲃}{77214}
\saveTG{𡳫}{77214}
\saveTG{𫘴}{77214}
\saveTG{𩖪}{77214}
\saveTG{𨵱}{77214}
\saveTG{𦙾}{77214}
\saveTG{𩗋}{77214}
\saveTG{𡲋}{77214}
\saveTG{𦚲}{77214}
\saveTG{𦞆}{77214}
\saveTG{𠘫}{77214}
\saveTG{閵}{77215}
\saveTG{𨾈}{77215}
\saveTG{𦋜}{77215}
\saveTG{㒿}{77215}
\saveTG{𦡱}{77215}
\saveTG{𨼇}{77215}
\saveTG{雤}{77215}
\saveTG{𩁕}{77215}
\saveTG{𨿀}{77215}
\saveTG{隆}{77215}
\saveTG{𨺓}{77215}
\saveTG{𪨒}{77215}
\saveTG{𪚲}{77215}
\saveTG{𡱌}{77216}
\saveTG{飗}{77216}
\saveTG{飀}{77216}
\saveTG{𩘴}{77216}
\saveTG{𨵀}{77216}
\saveTG{䫿}{77216}
\saveTG{𨳻}{77217}
\saveTG{𨴐}{77217}
\saveTG{䦧}{77217}
\saveTG{𨼨}{77217}
\saveTG{𡲏}{77217}
\saveTG{㞍}{77217}
\saveTG{𡱲}{77217}
\saveTG{𡲵}{77217}
\saveTG{𠙧}{77217}
\saveTG{𨳋}{77217}
\saveTG{𨴓}{77217}
\saveTG{𦦩}{77217}
\saveTG{脃}{77217}
\saveTG{肥}{77217}
\saveTG{凥}{77217}
\saveTG{閌}{77217}
\saveTG{尻}{77217}
\saveTG{闏}{77217}
\saveTG{屉}{77217}
\saveTG{屜}{77217}
\saveTG{𦦳}{77217}
\saveTG{𣩺}{77217}
\saveTG{𠒂}{77217}
\saveTG{𧠞}{77217}
\saveTG{𦙉}{77217}
\saveTG{㒾}{77217}
\saveTG{𪨌}{77217}
\saveTG{𡱤}{77217}
\saveTG{𣭼}{77217}
\saveTG{𡲔}{77217}
\saveTG{𨹔}{77217}
\saveTG{𨼎}{77217}
\saveTG{𦙷}{77217}
\saveTG{𢁁}{77217}
\saveTG{𦊗}{77217}
\saveTG{𩨜}{77217}
\saveTG{䯌}{77217}
\saveTG{𩨾}{77217}
\saveTG{𦌼}{77217}
\saveTG{𦊦}{77217}
\saveTG{䑕}{77217}
\saveTG{㒻}{77217}
\saveTG{𣬑}{77217}
\saveTG{𠘻}{77217}
\saveTG{㞎}{77217}
\saveTG{𡱓}{77217}
\saveTG{𡱂}{77217}
\saveTG{𡱧}{77217}
\saveTG{𡱹}{77217}
\saveTG{𨹣}{77217}
\saveTG{𦟊}{77217}
\saveTG{𩨒}{77217}
\saveTG{𠘬}{77217}
\saveTG{𠘧}{77217}
\saveTG{𦡔}{77217}
\saveTG{𠁽}{77217}
\saveTG{𧇜}{77217}
\saveTG{𨸰}{77217}
\saveTG{𦜻}{77217}
\saveTG{𩙡}{77217}
\saveTG{𡰩}{77217}
\saveTG{䳔}{77217}
\saveTG{𨼉}{77217}
\saveTG{𦋌}{77217}
\saveTG{𩨕}{77217}
\saveTG{𩩢}{77217}
\saveTG{𦉭}{77217}
\saveTG{𧠊}{77217}
\saveTG{䴘}{77217}
\saveTG{𦋼}{77217}
\saveTG{𦌏}{77217}
\saveTG{𩖝}{77217}
\saveTG{𧠓}{77217}
\saveTG{𡰼}{77217}
\saveTG{𡱷}{77217}
\saveTG{𡰽}{77217}
\saveTG{𡲐}{77217}
\saveTG{𡱕}{77217}
\saveTG{㞑}{77217}
\saveTG{𠒅}{77217}
\saveTG{𡱘}{77217}
\saveTG{𡱜}{77217}
\saveTG{𡲩}{77217}
\saveTG{𡳟}{77217}
\saveTG{𡲮}{77217}
\saveTG{𪨔}{77217}
\saveTG{𡳅}{77217}
\saveTG{𫔛}{77217}
\saveTG{𠑼}{77217}
\saveTG{𫌪}{77217}
\saveTG{𨷝}{77217}
\saveTG{𨶙}{77217}
\saveTG{𨷁}{77217}
\saveTG{䦎}{77217}
\saveTG{𨵘}{77217}
\saveTG{𨵄}{77217}
\saveTG{𡳂}{77218}
\saveTG{𦡪}{77218}
\saveTG{飓}{77218}
\saveTG{颶}{77218}
\saveTG{㞐}{77218}
\saveTG{𩘋}{77218}
\saveTG{月}{77220}
\saveTG{同}{77220}
\saveTG{网}{77220}
\saveTG{罔}{77220}
\saveTG{肳}{77220}
\saveTG{胸}{77220}
\saveTG{円}{77220}
\saveTG{用}{77220}
\saveTG{周}{77220}
\saveTG{𨹹}{77220}
\saveTG{䧓}{77220}
\saveTG{𨹽}{77220}
\saveTG{𨺃}{77220}
\saveTG{𨹋}{77220}
\saveTG{𨹐}{77220}
\saveTG{陱}{77220}
\saveTG{𨼝}{77220}
\saveTG{𨸓}{77220}
\saveTG{𠨖}{77220}
\saveTG{𩧛}{77220}
\saveTG{𦑘}{77220}
\saveTG{𦐴}{77220}
\saveTG{𦑆}{77220}
\saveTG{𣎢}{77220}
\saveTG{𧱊}{77220}
\saveTG{𦝓}{77220}
\saveTG{䏤}{77220}
\saveTG{𦡩}{77220}
\saveTG{𦛪}{77220}
\saveTG{䐚}{77220}
\saveTG{𡆨}{77220}
\saveTG{𠕀}{77220}
\saveTG{𠃢}{77220}
\saveTG{𦉫}{77220}
\saveTG{𫆗}{77220}
\saveTG{冋}{77220}
\saveTG{冂}{77220}
\saveTG{腳}{77220}
\saveTG{脚}{77220}
\saveTG{囘}{77220}
\saveTG{罓}{77220}
\saveTG{阴}{77220}
\saveTG{冃}{77220}
\saveTG{翢}{77220}
\saveTG{朋}{77220}
\saveTG{卪}{77220}
\saveTG{肕}{77220}
\saveTG{膶}{77220}
\saveTG{卩}{77220}
\saveTG{冈}{77220}
\saveTG{胴}{77220}
\saveTG{陶}{77220}
\saveTG{胊}{77220}
\saveTG{朐}{77220}
\saveTG{肑}{77220}
\saveTG{𦁒}{77220}
\saveTG{䏛}{77221}
\saveTG{𦋟}{77221}
\saveTG{𨳜}{77221}
\saveTG{𦌞}{77221}
\saveTG{䍏}{77221}
\saveTG{𨵅}{77221}
\saveTG{𨴠}{77221}
\saveTG{𦝋}{77221}
\saveTG{𨵌}{77221}
\saveTG{𫔪}{77221}
\saveTG{𦉸}{77221}
\saveTG{𦉪}{77221}
\saveTG{𡰨}{77221}
\saveTG{𠔼}{77221}
\saveTG{𨷮}{77221}
\saveTG{𨷾}{77221}
\saveTG{䦽}{77221}
\saveTG{𨶨}{77221}
\saveTG{㒺}{77221}
\saveTG{𠕈}{77221}
\saveTG{𫔡}{77221}
\saveTG{屙}{77221}
\saveTG{𠕏}{77221}
\saveTG{𪨜}{77221}
\saveTG{𣎁}{77221}
\saveTG{𠕙}{77221}
\saveTG{𦠥}{77221}
\saveTG{𠝋}{77221}
\saveTG{𠕃}{77221}
\saveTG{䰗}{77221}
\saveTG{𪱣}{77221}
\saveTG{𣫲}{77221}
\saveTG{𨶽}{77221}
\saveTG{𦉯}{77221}
\saveTG{𠄗}{77221}
\saveTG{𩰌}{77221}
\saveTG{𡱣}{77221}
\saveTG{𦑫}{77221}
\saveTG{𡳇}{77222}
\saveTG{𠔽}{77222}
\saveTG{𠕪}{77222}
\saveTG{𨵬}{77222}
\saveTG{𡰹}{77222}
\saveTG{𦛦}{77222}
\saveTG{𨹲}{77222}
\saveTG{𨴟}{77222}
\saveTG{𠕕}{77222}
\saveTG{𠕲}{77222}
\saveTG{髎}{77222}
\saveTG{膠}{77222}
\saveTG{𦊰}{77223}
\saveTG{𢛀}{77223}
\saveTG{𧰴}{77223}
\saveTG{𦝂}{77223}
\saveTG{𠕡}{77223}
\saveTG{𤓲}{77223}
\saveTG{𦟼}{77224}
\saveTG{𦙱}{77224}
\saveTG{𠕁}{77224}
\saveTG{𠮇}{77224}
\saveTG{𦉳}{77224}
\saveTG{𢃹}{77224}
\saveTG{𪠰}{77224}
\saveTG{𨳠}{77224}
\saveTG{𨴏}{77226}
\saveTG{𩩅}{77226}
\saveTG{𠕑}{77226}
\saveTG{𣇱}{77226}
\saveTG{𠕬}{77226}
\saveTG{𨷉}{77226}
\saveTG{𠱬}{77226}
\saveTG{𠕛}{77226}
\saveTG{𪞎}{77226}
\saveTG{𠕌}{77226}
\saveTG{𦊖}{77226}
\saveTG{𡳏}{77226}
\saveTG{𦚧}{77226}
\saveTG{𨴀}{77226}
\saveTG{𠵁}{77226}
\saveTG{鸤}{77227}
\saveTG{屬}{77227}
\saveTG{鷉}{77227}
\saveTG{鷵}{77227}
\saveTG{臀}{77227}
\saveTG{臋}{77227}
\saveTG{鷳}{77227}
\saveTG{鷴}{77227}
\saveTG{鷼}{77227}
\saveTG{膷}{77227}
\saveTG{屑}{77227}
\saveTG{邪}{77227}
\saveTG{邤}{77227}
\saveTG{釁}{77227}
\saveTG{臖}{77227}
\saveTG{鸦}{77227}
\saveTG{鴉}{77227}
\saveTG{帠}{77227}
\saveTG{鶂}{77227}
\saveTG{郮}{77227}
\saveTG{𪉄}{77227}
\saveTG{𤬙}{77227}
\saveTG{𤬋}{77227}
\saveTG{𫑮}{77227}
\saveTG{𡳍}{77227}
\saveTG{𡲇}{77227}
\saveTG{𨳈}{77227}
\saveTG{𨜘}{77227}
\saveTG{䣝}{77227}
\saveTG{𨛮}{77227}
\saveTG{𨙮}{77227}
\saveTG{𨚯}{77227}
\saveTG{䯞}{77227}
\saveTG{𦊌}{77227}
\saveTG{䢳}{77227}
\saveTG{𪡢}{77227}
\saveTG{𦚰}{77227}
\saveTG{䏧}{77227}
\saveTG{𦛲}{77227}
\saveTG{𡭘}{77227}
\saveTG{𠂏}{77227}
\saveTG{𦡄}{77227}
\saveTG{𤰌}{77227}
\saveTG{𦜳}{77227}
\saveTG{𢒀}{77227}
\saveTG{𦜅}{77227}
\saveTG{𦛸}{77227}
\saveTG{𨟟}{77227}
\saveTG{𨟑}{77227}
\saveTG{𨝂}{77227}
\saveTG{𨜸}{77227}
\saveTG{𢑅}{77227}
\saveTG{䣑}{77227}
\saveTG{𨜠}{77227}
\saveTG{䣅}{77227}
\saveTG{𨞋}{77227}
\saveTG{𨞺}{77227}
\saveTG{𦘬}{77227}
\saveTG{䣕}{77227}
\saveTG{𨽛}{77227}
\saveTG{𨸐}{77227}
\saveTG{𨸙}{77227}
\saveTG{𨛞}{77227}
\saveTG{𨞬}{77227}
\saveTG{𦜜}{77227}
\saveTG{}{77227}
\saveTG{}{77227}
\saveTG{﨩}{77227}
\saveTG{鳲}{77227}
\saveTG{腎}{77227}
\saveTG{郳}{77227}
\saveTG{鬧}{77227}
\saveTG{閙}{77227}
\saveTG{鶥}{77227}
\saveTG{郿}{77227}
\saveTG{鹛}{77227}
\saveTG{腡}{77227}
\saveTG{屚}{77227}
\saveTG{朤}{77227}
\saveTG{鶌}{77227}
\saveTG{屫}{77227}
\saveTG{鶋}{77227}
\saveTG{局}{77227}
\saveTG{閒}{77227}
\saveTG{觷}{77227}
\saveTG{郈}{77227}
\saveTG{咼}{77227}
\saveTG{冎}{77227}
\saveTG{鶻}{77227}
\saveTG{骨}{77227}
\saveTG{鵩}{77227}
\saveTG{阝}{77227}
\saveTG{鹏}{77227}
\saveTG{鵬}{77227}
\saveTG{陊}{77227}
\saveTG{鵰}{77227}
\saveTG{屌}{77227}
\saveTG{隝}{77227}
\saveTG{鶞}{77227}
\saveTG{肠}{77227}
\saveTG{鸊}{77227}
\saveTG{閍}{77227}
\saveTG{闁}{77227}
\saveTG{𨷬}{77227}
\saveTG{𨴷}{77227}
\saveTG{䦵}{77227}
\saveTG{𦞚}{77227}
\saveTG{𦦑}{77227}
\saveTG{𦢰}{77227}
\saveTG{𨳵}{77227}
\saveTG{𨳪}{77227}
\saveTG{𨷘}{77227}
\saveTG{䦱}{77227}
\saveTG{𨴎}{77227}
\saveTG{𨴭}{77227}
\saveTG{𩿪}{77227}
\saveTG{𫚺}{77227}
\saveTG{𫚯}{77227}
\saveTG{䲿}{77227}
\saveTG{䲩}{77227}
\saveTG{𪅱}{77227}
\saveTG{𪁵}{77227}
\saveTG{𩪔}{77227}
\saveTG{𣪍}{77227}
\saveTG{𪈺}{77227}
\saveTG{𡱆}{77227}
\saveTG{𪃮}{77227}
\saveTG{𩪡}{77227}
\saveTG{𪄡}{77227}
\saveTG{𠕔}{77227}
\saveTG{𩪆}{77227}
\saveTG{𦜼}{77227}
\saveTG{𩿾}{77227}
\saveTG{𪆒}{77227}
\saveTG{𩾸}{77227}
\saveTG{𪆙}{77227}
\saveTG{𪄁}{77227}
\saveTG{𪁋}{77227}
\saveTG{𪂶}{77227}
\saveTG{𩿫}{77227}
\saveTG{𪁧}{77227}
\saveTG{𪁢}{77227}
\saveTG{隖}{77227}
\saveTG{𨼟}{77227}
\saveTG{𪅋}{77227}
\saveTG{𨵂}{77227}
\saveTG{𨜽}{77227}
\saveTG{𦦝}{77227}
\saveTG{𥝅}{77227}
\saveTG{𦦔}{77227}
\saveTG{𦥤}{77227}
\saveTG{𨷓}{77227}
\saveTG{𨷳}{77227}
\saveTG{䴙}{77227}
\saveTG{𦦟}{77227}
\saveTG{𦦸}{77227}
\saveTG{𧤑}{77227}
\saveTG{𨳚}{77227}
\saveTG{𨴶}{77227}
\saveTG{𡲜}{77227}
\saveTG{㞕}{77227}
\saveTG{屩}{77227}
\saveTG{𡳯}{77227}
\saveTG{𡲺}{77227}
\saveTG{𦜂}{77227}
\saveTG{𦞩}{77227}
\saveTG{𥝋}{77227}
\saveTG{𥝇}{77227}
\saveTG{𢄚}{77227}
\saveTG{𢃵}{77227}
\saveTG{𦦷}{77227}
\saveTG{𥝈}{77227}
\saveTG{𩿊}{77227}
\saveTG{𦥱}{77227}
\saveTG{䦳}{77227}
\saveTG{𨳭}{77227}
\saveTG{𪔃}{77227}
\saveTG{𡰬}{77227}
\saveTG{𡲳}{77227}
\saveTG{𡱺}{77227}
\saveTG{属}{77227}
\saveTG{㞔}{77227}
\saveTG{𡱃}{77227}
\saveTG{𡱈}{77227}
\saveTG{𡰺}{77227}
\saveTG{𡰳}{77227}
\saveTG{𡳚}{77227}
\saveTG{𡱊}{77227}
\saveTG{𡱎}{77227}
\saveTG{𡰯}{77227}
\saveTG{𡲍}{77227}
\saveTG{𡱿}{77227}
\saveTG{𡰴}{77227}
\saveTG{𡲻}{77227}
\saveTG{𩨙}{77227}
\saveTG{𠄷}{77227}
\saveTG{𪨓}{77227}
\saveTG{䐞}{77227}
\saveTG{𦙨}{77227}
\saveTG{𦢨}{77227}
\saveTG{𦛳}{77227}
\saveTG{𦡥}{77227}
\saveTG{𠭾}{77227}
\saveTG{𨷲}{77227}
\saveTG{𦌄}{77227}
\saveTG{𦥡}{77227}
\saveTG{𨴪}{77227}
\saveTG{𨵴}{77227}
\saveTG{𨴜}{77227}
\saveTG{𨷈}{77227}
\saveTG{𡰾}{77227}
\saveTG{𡱄}{77227}
\saveTG{𢅝}{77227}
\saveTG{𪠥}{77227}
\saveTG{䐰}{77227}
\saveTG{𦟓}{77227}
\saveTG{𦜴}{77227}
\saveTG{𠭽}{77227}
\saveTG{𨹓}{77227}
\saveTG{㬾}{77227}
\saveTG{𢃥}{77227}
\saveTG{𡱑}{77227}
\saveTG{𥝊}{77227}
\saveTG{𦦻}{77227}
\saveTG{𦡭}{77227}
\saveTG{𦟲}{77227}
\saveTG{𢂗}{77227}
\saveTG{𦋀}{77227}
\saveTG{𡶬}{77227}
\saveTG{𣍸}{77227}
\saveTG{𦛂}{77227}
\saveTG{鹘}{77227}
\saveTG{𠕐}{77227}
\saveTG{𦛴}{77227}
\saveTG{䐢}{77227}
\saveTG{𦞽}{77227}
\saveTG{𦠈}{77227}
\saveTG{𣎛}{77227}
\saveTG{𫛵}{77227}
\saveTG{𨼖}{77227}
\saveTG{𨺳}{77227}
\saveTG{𩿇}{77227}
\saveTG{𩱬}{77227}
\saveTG{䑁}{77227}
\saveTG{𦦧}{77227}
\saveTG{𦡌}{77227}
\saveTG{𦥩}{77227}
\saveTG{𨵾}{77227}
\saveTG{𨳙}{77227}
\saveTG{𨵧}{77227}
\saveTG{𨵶}{77227}
\saveTG{𦋢}{77228}
\saveTG{𨵼}{77228}
\saveTG{𠕎}{77228}
\saveTG{𪨗}{77228}
\saveTG{𠕦}{77228}
\saveTG{䦏}{77228}
\saveTG{𨵦}{77228}
\saveTG{𠔿}{77228}
\saveTG{㒳}{77228}
\saveTG{䧁}{77229}
\saveTG{𪩱}{77229}
\saveTG{䑌}{77229}
\saveTG{𦠯}{77229}
\saveTG{𦝹}{77229}
\saveTG{𠕘}{77229}
\saveTG{𠕚}{77229}
\saveTG{𡱳}{77229}
\saveTG{𨸤}{77229}
\saveTG{𨳲}{77230}
\saveTG{𨳢}{77231}
\saveTG{𨳿}{77231}
\saveTG{𢛩}{77231}
\saveTG{𤓱}{77231}
\saveTG{𤔢}{77231}
\saveTG{𧒝}{77231}
\saveTG{𧖏}{77231}
\saveTG{𧖉}{77231}
\saveTG{䦝}{77231}
\saveTG{𨵰}{77231}
\saveTG{𦞪}{77231}
\saveTG{𨴿}{77231}
\saveTG{㼖}{77231}
\saveTG{爮}{77231}
\saveTG{爬}{77231}
\saveTG{瓟}{77231}
\saveTG{閇}{77231}
\saveTG{𤕊}{77231}
\saveTG{𤔈}{77231}
\saveTG{𡱦}{77231}
\saveTG{𡱅}{77231}
\saveTG{𡱗}{77231}
\saveTG{㞡}{77232}
\saveTG{𡱰}{77232}
\saveTG{㞘}{77232}
\saveTG{𥊟}{77232}
\saveTG{𡲘}{77232}
\saveTG{𨴳}{77232}
\saveTG{𤫮}{77232}
\saveTG{𨷼}{77232}
\saveTG{𨴯}{77232}
\saveTG{𨳥}{77232}
\saveTG{䢅}{77232}
\saveTG{䢉}{77232}
\saveTG{𦜩}{77232}
\saveTG{𦜎}{77232}
\saveTG{𡱴}{77232}
\saveTG{𤓵}{77232}
\saveTG{𨶺}{77232}
\saveTG{屒}{77232}
\saveTG{𦝰}{77232}
\saveTG{闤}{77232}
\saveTG{𨼯}{77232}
\saveTG{𧱢}{77232}
\saveTG{𦚣}{77232}
\saveTG{𧰽}{77232}
\saveTG{𡳋}{77232}
\saveTG{𡱽}{77232}
\saveTG{𨴖}{77232}
\saveTG{䬤}{77232}
\saveTG{瓝}{77232}
\saveTG{层}{77232}
\saveTG{腞}{77232}
\saveTG{冡}{77232}
\saveTG{限}{77232}
\saveTG{展}{77232}
\saveTG{膼}{77232}
\saveTG{䧘}{77232}
\saveTG{𧜷}{77232}
\saveTG{𩛻}{77232}
\saveTG{䦠}{77232}
\saveTG{㞙}{77232}
\saveTG{𦟌}{77232}
\saveTG{䏰}{77232}
\saveTG{𧱃}{77232}
\saveTG{𦊡}{77233}
\saveTG{𫗁}{77233}
\saveTG{𦦶}{77233}
\saveTG{𤬢}{77233}
\saveTG{閼}{77233}
\saveTG{腿}{77233}
\saveTG{𩘩}{77233}
\saveTG{𦦬}{77233}
\saveTG{𤬣}{77233}
\saveTG{𦙭}{77233}
\saveTG{𨹠}{77234}
\saveTG{𨸻}{77234}
\saveTG{𪺐}{77234}
\saveTG{𤔀}{77234}
\saveTG{𤓷}{77234}
\saveTG{𩪌}{77235}
\saveTG{膖}{77235}
\saveTG{𦞣}{77236}
\saveTG{𡲦}{77236}
\saveTG{䐳}{77236}
\saveTG{𨻦}{77236}
\saveTG{𨻡}{77237}
\saveTG{𩪀}{77237}
\saveTG{㞏}{77237}
\saveTG{隐}{77237}
\saveTG{膇}{77237}
\saveTG{𨴧}{77237}
\saveTG{𢣟}{77238}
\saveTG{𦠧}{77238}
\saveTG{𤕉}{77238}
\saveTG{𨸕}{77240}
\saveTG{𨺏}{77240}
\saveTG{陬}{77240}
\saveTG{叞}{77240}
\saveTG{𠬨}{77240}
\saveTG{闢}{77241}
\saveTG{閕}{77241}
\saveTG{屖}{77241}
\saveTG{屛}{77241}
\saveTG{屏}{77241}
\saveTG{𡳡}{77241}
\saveTG{𡲱}{77241}
\saveTG{𡱡}{77241}
\saveTG{𦝷}{77241}
\saveTG{𡱞}{77242}
\saveTG{肞}{77243}
\saveTG{𦌍}{77243}
\saveTG{𣎟}{77243}
\saveTG{𡰷}{77243}
\saveTG{𫔟}{77243}
\saveTG{𦠅}{77243}
\saveTG{𦚪}{77244}
\saveTG{𡱯}{77244}
\saveTG{𡱚}{77244}
\saveTG{𡲁}{77244}
\saveTG{𡲟}{77244}
\saveTG{𡲾}{77244}
\saveTG{𡲹}{77244}
\saveTG{屦}{77244}
\saveTG{屨}{77244}
\saveTG{屢}{77244}
\saveTG{屡}{77244}
\saveTG{㞜}{77244}
\saveTG{𡲎}{77245}
\saveTG{𨵸}{77245}
\saveTG{𠙊}{77246}
\saveTG{𨼔}{77246}
\saveTG{䦻}{77247}
\saveTG{𡱟}{77247}
\saveTG{𨶞}{77247}
\saveTG{𨸚}{77247}
\saveTG{𦜌}{77247}
\saveTG{𦝒}{77247}
\saveTG{𩨞}{77247}
\saveTG{𡱾}{77247}
\saveTG{𠬮}{77247}
\saveTG{𤈫}{77247}
\saveTG{㕞}{77247}
\saveTG{𪠦}{77247}
\saveTG{𤓶}{77247}
\saveTG{𨸜}{77247}
\saveTG{𨺣}{77247}
\saveTG{㲀}{77247}
\saveTG{𦣒}{77247}
\saveTG{𦠳}{77247}
\saveTG{𪨍}{77247}
\saveTG{𣪵}{77247}
\saveTG{㞌}{77247}
\saveTG{𡥷}{77247}
\saveTG{𣪔}{77247}
\saveTG{㞋}{77247}
\saveTG{𠬩}{77247}
\saveTG{𣪫}{77247}
\saveTG{𡰵}{77247}
\saveTG{𡥸}{77247}
\saveTG{𡰸}{77247}
\saveTG{𡱪}{77247}
\saveTG{𡲆}{77247}
\saveTG{𡲅}{77247}
\saveTG{𡳐}{77247}
\saveTG{𪠪}{77247}
\saveTG{𡳷}{77247}
\saveTG{履}{77247}
\saveTG{𨺽}{77247}
\saveTG{𨹺}{77247}
\saveTG{𩩱}{77247}
\saveTG{𡰫}{77247}
\saveTG{𡰻}{77247}
\saveTG{𢁋}{77247}
\saveTG{𡱥}{77247}
\saveTG{𡳉}{77247}
\saveTG{𡲲}{77247}
\saveTG{𡰿}{77247}
\saveTG{服}{77247}
\saveTG{閉}{77247}
\saveTG{孱}{77247}
\saveTG{腏}{77247}
\saveTG{殿}{77247}
\saveTG{腶}{77247}
\saveTG{股}{77247}
\saveTG{骰}{77247}
\saveTG{𨸳}{77247}
\saveTG{叚}{77247}
\saveTG{腵}{77247}
\saveTG{膄}{77247}
\saveTG{䮕}{77247}
\saveTG{𦚂}{77247}
\saveTG{𫖽}{77247}
\saveTG{𦘿}{77247}
\saveTG{𦋕}{77247}
\saveTG{𡳌}{77247}
\saveTG{𨵟}{77247}
\saveTG{𥀣}{77247}
\saveTG{𨺝}{77247}
\saveTG{屐}{77247}
\saveTG{𦜺}{77247}
\saveTG{𠬝}{77247}
\saveTG{𡲯}{77247}
\saveTG{𤿳}{77247}
\saveTG{𪵓}{77247}
\saveTG{𨵐}{77247}
\saveTG{𫕃}{77247}
\saveTG{𨷣}{77248}
\saveTG{𡱙}{77248}
\saveTG{臎}{77248}
\saveTG{𡳥}{77248}
\saveTG{𨷋}{77248}
\saveTG{胟}{77250}
\saveTG{𦙇}{77250}
\saveTG{羼}{77251}
\saveTG{𡲄}{77251}
\saveTG{𡱝}{77251}
\saveTG{𤚌}{77251}
\saveTG{腪}{77252}
\saveTG{𨼬}{77252}
\saveTG{𠕠}{77253}
\saveTG{𤑂}{77253}
\saveTG{閥}{77253}
\saveTG{𡱫}{77253}
\saveTG{降}{77254}
\saveTG{胮}{77254}
\saveTG{䧏}{77254}
\saveTG{䏺}{77254}
\saveTG{𦙽}{77255}
\saveTG{𡲕}{77256}
\saveTG{𡲿}{77256}
\saveTG{𡲛}{77257}
\saveTG{𡲓}{77257}
\saveTG{𡲣}{77257}
\saveTG{𨷧}{77257}
\saveTG{㬹}{77257}
\saveTG{𣍯}{77257}
\saveTG{𡳞}{77257}
\saveTG{𦌳}{77257}
\saveTG{犀}{77259}
\saveTG{𡲖}{77260}
\saveTG{𡳄}{77260}
\saveTG{𨼧}{77261}
\saveTG{㞛}{77261}
\saveTG{膽}{77261}
\saveTG{𨼮}{77261}
\saveTG{𡱔}{77261}
\saveTG{㞓}{77261}
\saveTG{䦲}{77261}
\saveTG{脗}{77262}
\saveTG{陥}{77262}
\saveTG{䧂}{77262}
\saveTG{𨼌}{77262}
\saveTG{𨻧}{77262}
\saveTG{𦚔}{77262}
\saveTG{𡱇}{77262}
\saveTG{㞒}{77262}
\saveTG{𨻸}{77262}
\saveTG{䐇}{77262}
\saveTG{𦞧}{77262}
\saveTG{𡱶}{77262}
\saveTG{𨵁}{77262}
\saveTG{𦞞}{77263}
\saveTG{𨼑}{77263}
\saveTG{𦢞}{77263}
\saveTG{胳}{77264}
\saveTG{𦛵}{77264}
\saveTG{𡲚}{77264}
\saveTG{𦝮}{77264}
\saveTG{䧄}{77264}
\saveTG{𡲢}{77264}
\saveTG{屠}{77264}
\saveTG{腒}{77264}
\saveTG{居}{77264}
\saveTG{骼}{77264}
\saveTG{届}{77265}
\saveTG{𨶈}{77265}
\saveTG{𪨘}{77265}
\saveTG{層}{77266}
\saveTG{𡰶}{77267}
\saveTG{𣅩}{77267}
\saveTG{}{77267}
\saveTG{𨵻}{77267}
\saveTG{眉}{77267}
\saveTG{屇}{77267}
\saveTG{𨶆}{77267}
\saveTG{𡳻}{77267}
\saveTG{𣎊}{77267}
\saveTG{𦛓}{77270}
\saveTG{𤳰}{77271}
\saveTG{𡳘}{77272}
\saveTG{屆}{77272}
\saveTG{屈}{77272}
\saveTG{𡲗}{77272}
\saveTG{𨵡}{77272}
\saveTG{𡲒}{77272}
\saveTG{𦜇}{77272}
\saveTG{𨺚}{77272}
\saveTG{𨺂}{77272}
\saveTG{𨵠}{77272}
\saveTG{𪨉}{77272}
\saveTG{𡱞}{77272}
\saveTG{𡲶}{77272}
\saveTG{𡲬}{77272}
\saveTG{𩨫}{77274}
\saveTG{𡰲}{77277}
\saveTG{䐄}{77277}
\saveTG{𨴌}{77277}
\saveTG{㞚}{77277}
\saveTG{䏨}{77277}
\saveTG{陷}{77277}
\saveTG{𩨽}{77277}
\saveTG{𨹅}{77277}
\saveTG{𧱴}{77280}
\saveTG{𪨡}{77281}
\saveTG{𦢯}{77281}
\saveTG{𫕄}{77281}
\saveTG{𦠆}{77281}
\saveTG{䑂}{77281}
\saveTG{屣}{77281}
\saveTG{𧿃}{77281}
\saveTG{𡱒}{77281}
\saveTG{𩪞}{77281}
\saveTG{𣤀}{77282}
\saveTG{歋}{77282}
\saveTG{欣}{77282}
\saveTG{屃}{77282}
\saveTG{肷}{77282}
\saveTG{歄}{77282}
\saveTG{屄}{77282}
\saveTG{㞞}{77282}
\saveTG{𣢂}{77282}
\saveTG{𡱐}{77282}
\saveTG{䰘}{77282}
\saveTG{𣢛}{77282}
\saveTG{𣢵}{77282}
\saveTG{𣣉}{77282}
\saveTG{𪴫}{77282}
\saveTG{𣢥}{77282}
\saveTG{𣢨}{77282}
\saveTG{𣢚}{77282}
\saveTG{𡱍}{77282}
\saveTG{𣢞}{77282}
\saveTG{𣢁}{77282}
\saveTG{𣥀}{77282}
\saveTG{㰺}{77282}
\saveTG{𣤅}{77282}
\saveTG{𣣣}{77282}
\saveTG{𩪢}{77282}
\saveTG{䯉}{77282}
\saveTG{𦡸}{77282}
\saveTG{𦙬}{77282}
\saveTG{㽰}{77282}
\saveTG{𦜓}{77282}
\saveTG{𧰫}{77282}
\saveTG{𣤔}{77282}
\saveTG{𣣭}{77282}
\saveTG{𣢭}{77282}
\saveTG{㰹}{77282}
\saveTG{𣣃}{77282}
\saveTG{𣢀}{77282}
\saveTG{𨼤}{77282}
\saveTG{䐿}{77284}
\saveTG{𨺺}{77284}
\saveTG{𩩵}{77284}
\saveTG{𦞈}{77284}
\saveTG{𦞕}{77284}
\saveTG{𦣜}{77284}
\saveTG{㞟}{77284}
\saveTG{𦝝}{77284}
\saveTG{𦝜}{77284}
\saveTG{𩩽}{77284}
\saveTG{𡱢}{77284}
\saveTG{𡱁}{77284}
\saveTG{𡱱}{77284}
\saveTG{隩}{77284}
\saveTG{𡲑}{77284}
\saveTG{𡳮}{77286}
\saveTG{𥌏}{77286}
\saveTG{𨷚}{77286}
\saveTG{𡳠}{77286}
\saveTG{屭}{77286}
\saveTG{臔}{77286}
\saveTG{屓}{77286}
\saveTG{腴}{77287}
\saveTG{}{77287}
\saveTG{𪨊}{77287}
\saveTG{𡰦}{77287}
\saveTG{𦣁}{77289}
\saveTG{𨹾}{77289}
\saveTG{𦜡}{77289}
\saveTG{𩀲}{77289}
\saveTG{𦛧}{77289}
\saveTG{際}{77291}
\saveTG{脲}{77292}
\saveTG{𣍨}{77292}
\saveTG{尿}{77292}
\saveTG{𨽁}{77292}
\saveTG{𡱏}{77293}
\saveTG{䐼}{77293}
\saveTG{𨵺}{77293}
\saveTG{屎}{77294}
\saveTG{屧}{77294}
\saveTG{𡲴}{77294}
\saveTG{屟}{77294}
\saveTG{䐖}{77294}
\saveTG{𦜊}{77294}
\saveTG{𣙰}{77294}
\saveTG{𡳨}{77294}
\saveTG{𦟄}{77294}
\saveTG{腬}{77294}
\saveTG{杘}{77294}
\saveTG{𣐉}{77294}
\saveTG{𦚩}{77294}
\saveTG{𨹄}{77294}
\saveTG{𨹃}{77294}
\saveTG{𨻗}{77294}
\saveTG{𡳙}{77294}
\saveTG{𡲰}{77294}
\saveTG{㞖}{77295}
\saveTG{𡱼}{77295}
\saveTG{𡱖}{77295}
\saveTG{㞠}{77296}
\saveTG{屪}{77296}
\saveTG{𣫥}{77299}
\saveTG{㞗}{77299}
\saveTG{䐂}{77299}
\saveTG{𨳉}{77301}
\saveTG{𩙋}{77302}
\saveTG{閝}{77302}
\saveTG{𨷃}{77302}
\saveTG{𩙌}{77302}
\saveTG{𫗂}{77303}
\saveTG{闧}{77303}
\saveTG{尽}{77303}
\saveTG{𩘬}{77303}
\saveTG{闥}{77305}
\saveTG{𨶐}{77306}
\saveTG{𩖳}{77307}
\saveTG{𦥧}{77307}
\saveTG{𨘘}{77309}
\saveTG{𨶌}{77309}
\saveTG{䭵}{77310}
\saveTG{颿}{77310}
\saveTG{𫘙}{77310}
\saveTG{𩡰}{77310}
\saveTG{䮀}{77312}
\saveTG{䮘}{77312}
\saveTG{𩣦}{77312}
\saveTG{駔}{77312}
\saveTG{𩣮}{77312}
\saveTG{𩣿}{77314}
\saveTG{𩢋}{77320}
\saveTG{𩦂}{77320}
\saveTG{𩦓}{77320}
\saveTG{駨}{77320}
\saveTG{𩦔}{77320}
\saveTG{𩦃}{77320}
\saveTG{駉}{77320}
\saveTG{駧}{77320}
\saveTG{𩤍}{77320}
\saveTG{𩣽}{77320}
\saveTG{騆}{77320}
\saveTG{騊}{77320}
\saveTG{駠}{77320}
\saveTG{驧}{77320}
\saveTG{駒}{77320}
\saveTG{馰}{77320}
\saveTG{䭹}{77320}
\saveTG{䮐}{77320}
\saveTG{𩢞}{77320}
\saveTG{𠨝}{77320}
\saveTG{𦒝}{77320}
\saveTG{𩣍}{77320}
\saveTG{𩡲}{77320}
\saveTG{𩦴}{77320}
\saveTG{𩢛}{77320}
\saveTG{𩥯}{77324}
\saveTG{𩦡}{77327}
\saveTG{闖}{77327}
\saveTG{騧}{77327}
\saveTG{駶}{77327}
\saveTG{𩘵}{77327}
\saveTG{𩢂}{77327}
\saveTG{鸒}{77327}
\saveTG{驈}{77327}
\saveTG{鷖}{77327}
\saveTG{鷽}{77327}
\saveTG{𢘖}{77327}
\saveTG{舄}{77327}
\saveTG{舃}{77327}
\saveTG{騶}{77327}
\saveTG{𪂖}{77327}
\saveTG{𨶠}{77327}
\saveTG{䮈}{77327}
\saveTG{𩣒}{77327}
\saveTG{𨜇}{77327}
\saveTG{𤉡}{77327}
\saveTG{𩿏}{77327}
\saveTG{𩡿}{77327}
\saveTG{𦋻}{77327}
\saveTG{𨶇}{77327}
\saveTG{鄏}{77327}
\saveTG{𪃹}{77327}
\saveTG{𪾖}{77327}
\saveTG{䮸}{77327}
\saveTG{𩥩}{77327}
\saveTG{𩥎}{77327}
\saveTG{𩣅}{77327}
\saveTG{𩤺}{77327}
\saveTG{𢡥}{77330}
\saveTG{𨶡}{77330}
\saveTG{熙}{77331}
\saveTG{黳}{77331}
\saveTG{悶}{77331}
\saveTG{𨴮}{77331}
\saveTG{𨴽}{77331}
\saveTG{𦌲}{77331}
\saveTG{𤋮}{77331}
\saveTG{𢞍}{77331}
\saveTG{𦧀}{77331}
\saveTG{𪒵}{77331}
\saveTG{𨶲}{77331}
\saveTG{𨷔}{77331}
\saveTG{𨶯}{77331}
\saveTG{𦏗}{77331}
\saveTG{𤉂}{77331}
\saveTG{𨶅}{77331}
\saveTG{𨶭}{77331}
\saveTG{𢚥}{77331}
\saveTG{𢢎}{77331}
\saveTG{𢘒}{77331}
\saveTG{煕}{77331}
\saveTG{𢘁}{77332}
\saveTG{𢘼}{77332}
\saveTG{𢗯}{77332}
\saveTG{𢛇}{77332}
\saveTG{𪬋}{77332}
\saveTG{騘}{77332}
\saveTG{𤍥}{77332}
\saveTG{驟}{77332}
\saveTG{㤻}{77332}
\saveTG{𢝯}{77332}
\saveTG{𢡻}{77332}
\saveTG{𩥃}{77332}
\saveTG{𩥇}{77332}
\saveTG{𩦫}{77332}
\saveTG{闗}{77333}
\saveTG{恳}{77333}
\saveTG{𩢦}{77333}
\saveTG{悬}{77333}
\saveTG{閟}{77334}
\saveTG{𢞰}{77334}
\saveTG{𢠶}{77334}
\saveTG{𢝪}{77334}
\saveTG{𫔠}{77334}
\saveTG{𢝅}{77334}
\saveTG{𢟬}{77334}
\saveTG{𢘓}{77335}
\saveTG{騷}{77336}
\saveTG{𢣠}{77336}
\saveTG{𢡙}{77336}
\saveTG{𨶢}{77336}
\saveTG{𩥭}{77336}
\saveTG{𦋮}{77336}
\saveTG{𢞭}{77336}
\saveTG{騒}{77336}
\saveTG{𩼹}{77336}
\saveTG{𨵖}{77336}
\saveTG{𢛥}{77336}
\saveTG{鱟}{77336}
\saveTG{𨳖}{77337}
\saveTG{𢤒}{77337}
\saveTG{𢙦}{77337}
\saveTG{𩾨}{77337}
\saveTG{𫚃}{77338}
\saveTG{𢤀}{77338}
\saveTG{𫗄}{77338}
\saveTG{㦛}{77338}
\saveTG{𤊑}{77338}
\saveTG{𢜛}{77338}
\saveTG{𢚪}{77338}
\saveTG{惥}{77338}
\saveTG{馭}{77340}
\saveTG{𨷜}{77340}
\saveTG{𨵏}{77341}
\saveTG{导}{77341}
\saveTG{𩢇}{77341}
\saveTG{䦙}{77341}
\saveTG{𦥘}{77347}
\saveTG{𩤿}{77347}
\saveTG{𩤨}{77347}
\saveTG{馺}{77347}
\saveTG{𩢃}{77347}
\saveTG{𡬯}{77347}
\saveTG{𩤣}{77347}
\saveTG{𫘈}{77347}
\saveTG{𩣸}{77347}
\saveTG{駸}{77347}
\saveTG{𩢸}{77347}
\saveTG{𫘅}{77347}
\saveTG{驏}{77347}
\saveTG{騢}{77347}
\saveTG{騪}{77347}
\saveTG{𩧊}{77348}
\saveTG{𩦄}{77352}
\saveTG{𩧠}{77352}
\saveTG{䮝}{77356}
\saveTG{𩧞}{77357}
\saveTG{𩘫}{77357}
\saveTG{騽}{77362}
\saveTG{駋}{77362}
\saveTG{騮}{77362}
\saveTG{駱}{77364}
\saveTG{𩤅}{77364}
\saveTG{𩤫}{77371}
\saveTG{𩦑}{77371}
\saveTG{𩤓}{77372}
\saveTG{𩤂}{77377}
\saveTG{𩦖}{77381}
\saveTG{龮}{77381}
\saveTG{𣤮}{77382}
\saveTG{𩧁}{77384}
\saveTG{駅}{77387}
\saveTG{𩧙}{77389}
\saveTG{𩤞}{77394}
\saveTG{𩧋}{77394}
\saveTG{䮣}{77394}
\saveTG{騥}{77394}
\saveTG{𩦅}{77394}
\saveTG{騄}{77399}
\saveTG{閺}{77400}
\saveTG{又}{77400}
\saveTG{閔}{77400}
\saveTG{閈}{77401}
\saveTG{𨴲}{77401}
\saveTG{閗}{77401}
\saveTG{鬦}{77401}
\saveTG{閮}{77401}
\saveTG{聞}{77401}
\saveTG{𦗶}{77401}
\saveTG{䍑}{77401}
\saveTG{𦉻}{77401}
\saveTG{𥽳}{77401}
\saveTG{𠧑}{77401}
\saveTG{𦋃}{77401}
\saveTG{𨵉}{77401}
\saveTG{䦴}{77401}
\saveTG{𨉖}{77401}
\saveTG{叉}{77403}
\saveTG{媐}{77404}
\saveTG{𨶁}{77404}
\saveTG{𨴣}{77404}
\saveTG{婴}{77404}
\saveTG{嫛}{77404}
\saveTG{𡢗}{77404}
\saveTG{婜}{77404}
\saveTG{𨵋}{77404}
\saveTG{𣫻}{77404}
\saveTG{𡡋}{77404}
\saveTG{𡡼}{77404}
\saveTG{闄}{77404}
\saveTG{𡣿}{77404}
\saveTG{𡝨}{77404}
\saveTG{𡢕}{77404}
\saveTG{𨳐}{77404}
\saveTG{𨵭}{77405}
\saveTG{𦋚}{77406}
\saveTG{𨶤}{77406}
\saveTG{𨴝}{77406}
\saveTG{𦋇}{77406}
\saveTG{𨳕}{77407}
\saveTG{𨴫}{77407}
\saveTG{𨳛}{77407}
\saveTG{𨴹}{77407}
\saveTG{𡕯}{77407}
\saveTG{𡥘}{77407}
\saveTG{𠬸}{77407}
\saveTG{𡕻}{77407}
\saveTG{學}{77407}
\saveTG{㕛}{77407}
\saveTG{𨳧}{77407}
\saveTG{䦩}{77407}
\saveTG{𪠨}{77407}
\saveTG{𦥼}{77407}
\saveTG{𪩮}{77407}
\saveTG{𪌲}{77407}
\saveTG{𠬥}{77407}
\saveTG{夓}{77407}
\saveTG{孯}{77407}
\saveTG{殳}{77407}
\saveTG{叟}{77407}
\saveTG{閿}{77407}
\saveTG{闅}{77407}
\saveTG{𠭭}{77407}
\saveTG{𡥗}{77407}
\saveTG{𪞐}{77407}
\saveTG{𡱬}{77407}
\saveTG{𨳰}{77408}
\saveTG{閛}{77409}
\saveTG{𣄲}{77411}
\saveTG{𦔶}{77412}
\saveTG{𠬭}{77412}
\saveTG{观}{77412}
\saveTG{𨶝}{77417}
\saveTG{𨳊}{77417}
\saveTG{𠕤}{77417}
\saveTG{𨵃}{77417}
\saveTG{𠨏}{77420}
\saveTG{𪠂}{77420}
\saveTG{𠣝}{77421}
\saveTG{𨴾}{77421}
\saveTG{𨴸}{77422}
\saveTG{鵽}{77427}
\saveTG{鴅}{77427}
\saveTG{鸡}{77427}
\saveTG{鳮}{77427}
\saveTG{𠡋}{77427}
\saveTG{𪈔}{77427}
\saveTG{𪵌}{77427}
\saveTG{𨙳}{77427}
\saveTG{䢷}{77427}
\saveTG{𠮀}{77427}
\saveTG{邓}{77427}
\saveTG{𠭬}{77427}
\saveTG{𩿑}{77427}
\saveTG{𨟋}{77427}
\saveTG{𫛝}{77427}
\saveTG{鹦}{77427}
\saveTG{鄋}{77427}
\saveTG{舅}{77427}
\saveTG{艰}{77432}
\saveTG{册}{77440}
\saveTG{冊}{77440}
\saveTG{双}{77440}
\saveTG{丹}{77440}
\saveTG{開}{77441}
\saveTG{异}{77441}
\saveTG{䦕}{77441}
\saveTG{𨶘}{77441}
\saveTG{𨴂}{77441}
\saveTG{𢌴}{77441}
\saveTG{𡝑}{77441}
\saveTG{𡝞}{77441}
\saveTG{𢍯}{77442}
\saveTG{𢁅}{77442}
\saveTG{𢍲}{77442}
\saveTG{𢍱}{77442}
\saveTG{𢌺}{77442}
\saveTG{𨵿}{77443}
\saveTG{閞}{77443}
\saveTG{𢍭}{77443}
\saveTG{𠬺}{77444}
\saveTG{𡛒}{77444}
\saveTG{𡚩}{77444}
\saveTG{𡞔}{77445}
\saveTG{𠕋}{77445}
\saveTG{𦌑}{77445}
\saveTG{𨵒}{77446}
\saveTG{𨴔}{77447}
\saveTG{𣫳}{77447}
\saveTG{𨵑}{77447}
\saveTG{𦋖}{77447}
\saveTG{𠖎}{77447}
\saveTG{䦤}{77447}
\saveTG{𨷂}{77447}
\saveTG{𦥸}{77447}
\saveTG{𨳦}{77447}
\saveTG{𨵹}{77447}
\saveTG{舁}{77447}
\saveTG{叒}{77447}
\saveTG{叕}{77447}
\saveTG{段}{77447}
\saveTG{𢍹}{77447}
\saveTG{𢍀}{77447}
\saveTG{𦌤}{77447}
\saveTG{𣆼}{77447}
\saveTG{𣪂}{77447}
\saveTG{闞}{77448}
\saveTG{鬫}{77448}
\saveTG{𪨎}{77468}
\saveTG{𦦖}{77477}
\saveTG{𠮌}{77477}
\saveTG{欢}{77482}
\saveTG{欼}{77482}
\saveTG{𣢕}{77482}
\saveTG{闕}{77482}
\saveTG{𨵇}{77482}
\saveTG{㯩}{77482}
\saveTG{𠭫}{77491}
\saveTG{𨶉}{77493}
\saveTG{𦋄}{77497}
\saveTG{𠭂}{77497}
\saveTG{𦏜}{77501}
\saveTG{𨳯}{77501}
\saveTG{𢲓}{77502}
\saveTG{𤛱}{77502}
\saveTG{䦐}{77502}
\saveTG{㩓}{77502}
\saveTG{𢫪}{77502}
\saveTG{𢮗}{77502}
\saveTG{掔}{77502}
\saveTG{擧}{77502}
\saveTG{𢸁}{77502}
\saveTG{𢹏}{77502}
\saveTG{𨵊}{77503}
\saveTG{𨴍}{77503}
\saveTG{𨳮}{77503}
\saveTG{𦌎}{77504}
\saveTG{㹂}{77504}
\saveTG{𠬤}{77504}
\saveTG{𤘩}{77504}
\saveTG{𨶱}{77504}
\saveTG{𫔚}{77506}
\saveTG{𨴁}{77506}
\saveTG{闡}{77506}
\saveTG{𦦠}{77506}
\saveTG{𦌱}{77506}
\saveTG{𩋆}{77506}
\saveTG{轝}{77506}
\saveTG{𩍂}{77506}
\saveTG{𨏮}{77506}
\saveTG{𨏐}{77506}
\saveTG{閳}{77506}
\saveTG{閘}{77506}
\saveTG{闈}{77506}
\saveTG{𫔢}{77506}
\saveTG{}{77507}
\saveTG{䦛}{77507}
\saveTG{舉}{77508}
\saveTG{𦧁}{77508}
\saveTG{𦦙}{77508}
\saveTG{𨷯}{77508}
\saveTG{𨴞}{77508}
\saveTG{𠭌}{77515}
\saveTG{𪓘}{77517}
\saveTG{𪚮}{77517}
\saveTG{𨴦}{77520}
\saveTG{𨷶}{77520}
\saveTG{𨴆}{77521}
\saveTG{𦐎}{77521}
\saveTG{𨚿}{77527}
\saveTG{𨴒}{77527}
\saveTG{𨳤}{77527}
\saveTG{𦌯}{77527}
\saveTG{𨙻}{77527}
\saveTG{𢁊}{77543}
\saveTG{𨶔}{77547}
\saveTG{毋}{77550}
\saveTG{冄}{77550}
\saveTG{𨳩}{77551}
\saveTG{𨴕}{77555}
\saveTG{𨵓}{77556}
\saveTG{𦍊}{77556}
\saveTG{𨳬}{77574}
\saveTG{𨶕}{77582}
\saveTG{誾}{77601}
\saveTG{闇}{77601}
\saveTG{閫}{77601}
\saveTG{嚳}{77601}
\saveTG{䦖}{77601}
\saveTG{間}{77601}
\saveTG{問}{77601}
\saveTG{閤}{77601}
\saveTG{𧫦}{77601}
\saveTG{𫌵}{77601}
\saveTG{𫔝}{77601}
\saveTG{𥗑}{77601}
\saveTG{𧭒}{77601}
\saveTG{礐}{77601}
\saveTG{𧦥}{77601}
\saveTG{䦜}{77601}
\saveTG{𥕭}{77601}
\saveTG{䁷}{77601}
\saveTG{𠿟}{77601}
\saveTG{𨴬}{77601}
\saveTG{𨳸}{77601}
\saveTG{𦋫}{77601}
\saveTG{𦊪}{77601}
\saveTG{𨶀}{77601}
\saveTG{䦣}{77601}
\saveTG{𧧧}{77601}
\saveTG{譽}{77601}
\saveTG{䃜}{77602}
\saveTG{礜}{77602}
\saveTG{留}{77602}
\saveTG{硻}{77602}
\saveTG{𨵥}{77602}
\saveTG{礐}{77602}
\saveTG{䦓}{77602}
\saveTG{羀}{77602}
\saveTG{𠉁}{77602}
\saveTG{䦒}{77602}
\saveTG{䃧}{77602}
\saveTG{𨵝}{77603}
\saveTG{瞖}{77604}
\saveTG{闙}{77604}
\saveTG{閽}{77604}
\saveTG{昬}{77604}
\saveTG{閣}{77604}
\saveTG{闍}{77604}
\saveTG{𠶳}{77604}
\saveTG{𨵽}{77604}
\saveTG{𦊹}{77604}
\saveTG{𠭀}{77604}
\saveTG{𨶶}{77604}
\saveTG{𨵫}{77604}
\saveTG{𦋧}{77604}
\saveTG{𣆑}{77604}
\saveTG{𣌧}{77604}
\saveTG{䦚}{77604}
\saveTG{𦊟}{77604}
\saveTG{𨳾}{77604}
\saveTG{𫔜}{77604}
\saveTG{𠮢}{77604}
\saveTG{𤲗}{77604}
\saveTG{䁂}{77604}
\saveTG{閪}{77604}
\saveTG{醫}{77604}
\saveTG{𣆩}{77605}
\saveTG{𦥪}{77605}
\saveTG{𠰔}{77605}
\saveTG{𣎭}{77605}
\saveTG{𦦤}{77605}
\saveTG{𨵩}{77606}
\saveTG{𨶍}{77606}
\saveTG{𨶧}{77606}
\saveTG{𦌝}{77606}
\saveTG{𦌘}{77606}
\saveTG{閶}{77606}
\saveTG{閭}{77606}
\saveTG{闣}{77606}
\saveTG{𤱚}{77607}
\saveTG{𣆫}{77607}
\saveTG{𣇀}{77607}
\saveTG{䦮}{77608}
\saveTG{冏}{77608}
\saveTG{𦋑}{77609}
\saveTG{䦭}{77609}
\saveTG{𨶸}{77609}
\saveTG{巼}{77612}
\saveTG{𨶃}{77617}
\saveTG{㖯}{77617}
\saveTG{𦊭}{77618}
\saveTG{𥔰}{77620}
\saveTG{閜}{77621}
\saveTG{𨵎}{77621}
\saveTG{𨵤}{77621}
\saveTG{𦑹}{77621}
\saveTG{𪉎}{77627}
\saveTG{𫑥}{77627}
\saveTG{𪃑}{77627}
\saveTG{𪃯}{77627}
\saveTG{𪂆}{77627}
\saveTG{𨵛}{77627}
\saveTG{𪅉}{77627}
\saveTG{𪁝}{77627}
\saveTG{𦋬}{77627}
\saveTG{鶹}{77627}
\saveTG{鹠}{77627}
\saveTG{𡳢}{77627}
\saveTG{𨛠}{77627}
\saveTG{𨚨}{77627}
\saveTG{𧦔}{77641}
\saveTG{㗨}{77647}
\saveTG{𣫏}{77647}
\saveTG{𫔞}{77651}
\saveTG{𨷰}{77661}
\saveTG{𣉈}{77662}
\saveTG{𠭎}{77664}
\saveTG{𨷄}{77664}
\saveTG{闆}{77666}
\saveTG{𣤌}{77682}
\saveTG{𣤕}{77682}
\saveTG{𣣤}{77682}
\saveTG{歠}{77682}
\saveTG{𣤒}{77682}
\saveTG{𣣏}{77682}
\saveTG{𣆾}{77682}
\saveTG{𣤑}{77682}
\saveTG{𨷸}{77693}
\saveTG{𨷴}{77693}
\saveTG{𠥷}{77710}
\saveTG{𦦭}{77710}
\saveTG{𨳑}{77710}
\saveTG{𪕴}{77710}
\saveTG{𦉽}{77710}
\saveTG{𦊀}{77710}
\saveTG{𠙉}{77710}
\saveTG{𨱙}{77710}
\saveTG{𦉺}{77710}
\saveTG{𦦦}{77710}
\saveTG{𨴑}{77711}
\saveTG{𫜢}{77711}
\saveTG{𨳷}{77712}
\saveTG{𪓢}{77712}
\saveTG{𠑹}{77712}
\saveTG{𦋍}{77712}
\saveTG{𨳂}{77712}
\saveTG{㔾}{77712}
\saveTG{𠨎}{77712}
\saveTG{𢑏}{77712}
\saveTG{鼠}{77712}
\saveTG{}{77712}
\saveTG{闂}{77712}
\saveTG{𠨨}{77712}
\saveTG{𠨩}{77712}
\saveTG{𦊠}{77712}
\saveTG{𪓬}{77712}
\saveTG{𧥔}{77712}
\saveTG{𪓞}{77712}
\saveTG{𨴛}{77712}
\saveTG{𦉮}{77712}
\saveTG{𨴥}{77714}
\saveTG{𨳍}{77714}
\saveTG{𣱒}{77714}
\saveTG{閐}{77714}
\saveTG{𨷍}{77714}
\saveTG{𨳘}{77715}
\saveTG{𦊉}{77715}
\saveTG{𨵜}{77715}
\saveTG{𨴄}{77715}
\saveTG{𦌌}{77715}
\saveTG{𦋙}{77715}
\saveTG{䦰}{77715}
\saveTG{𦊣}{77715}
\saveTG{𣮈}{77715}
\saveTG{𣮫}{77715}
\saveTG{𧤭}{77715}
\saveTG{閹}{77716}
\saveTG{𣫰}{77717}
\saveTG{𤭠}{77717}
\saveTG{𨱤}{77717}
\saveTG{𪕦}{77717}
\saveTG{𪕨}{77717}
\saveTG{㐦}{77717}
\saveTG{𪓭}{77717}
\saveTG{𪓙}{77717}
\saveTG{𣱈}{77717}
\saveTG{𪓩}{77717}
\saveTG{𨳽}{77717}
\saveTG{𨷟}{77717}
\saveTG{𪕁}{77717}
\saveTG{㽇}{77717}
\saveTG{䦍}{77717}
\saveTG{巴}{77717}
\saveTG{闀}{77717}
\saveTG{黽}{77717}
\saveTG{巳}{77717}
\saveTG{巸}{77717}
\saveTG{𪓸}{77717}
\saveTG{乮}{77717}
\saveTG{𢀳}{77717}
\saveTG{𠨧}{77718}
\saveTG{𣌩}{77718}
\saveTG{𨲐}{77719}
\saveTG{𨱠}{77720}
\saveTG{𠨥}{77720}
\saveTG{𪕋}{77720}
\saveTG{卬}{77720}
\saveTG{即}{77720}
\saveTG{卵}{77720}
\saveTG{卯}{77720}
\saveTG{𫇅}{77720}
\saveTG{卿}{77720}
\saveTG{𠁾}{77720}
\saveTG{鼩}{77720}
\saveTG{印}{77720}
\saveTG{𨱢}{77721}
\saveTG{𪔹}{77721}
\saveTG{𪔺}{77721}
\saveTG{𪕱}{77721}
\saveTG{𪔸}{77721}
\saveTG{𪩬}{77721}
\saveTG{𨱨}{77721}
\saveTG{𩮵}{77722}
\saveTG{𩰪}{77722}
\saveTG{䶂}{77723}
\saveTG{𪕺}{77723}
\saveTG{𪕍}{77726}
\saveTG{𪕙}{77726}
\saveTG{𪕥}{77726}
\saveTG{𨜮}{77727}
\saveTG{𨜊}{77727}
\saveTG{𨰾}{77727}
\saveTG{𨙪}{77727}
\saveTG{𫑧}{77727}
\saveTG{𫚵}{77727}
\saveTG{䳇}{77727}
\saveTG{䳎}{77727}
\saveTG{𦥚}{77727}
\saveTG{䳭}{77727}
\saveTG{𨳣}{77727}
\saveTG{𨴅}{77727}
\saveTG{䦪}{77727}
\saveTG{𫑝}{77727}
\saveTG{𥀿}{77727}
\saveTG{𪀘}{77727}
\saveTG{𩿝}{77727}
\saveTG{𠥹}{77727}
\saveTG{䢹}{77727}
\saveTG{𨟦}{77727}
\saveTG{𫑗}{77727}
\saveTG{𪟮}{77727}
\saveTG{䢻}{77727}
\saveTG{𨝇}{77727}
\saveTG{𪕾}{77727}
\saveTG{𦉿}{77727}
\saveTG{𦦎}{77727}
\saveTG{鵖}{77727}
\saveTG{鸱}{77727}
\saveTG{鴟}{77727}
\saveTG{邸}{77727}
\saveTG{邼}{77727}
\saveTG{鄳}{77727}
\saveTG{鴖}{77727}
\saveTG{鸥}{77727}
\saveTG{鴎}{77727}
\saveTG{鷗}{77727}
\saveTG{鴄}{77727}
\saveTG{郾}{77727}
\saveTG{鶠}{77727}
\saveTG{𨴋}{77731}
\saveTG{𨲑}{77731}
\saveTG{𨴵}{77731}
\saveTG{𨳗}{77731}
\saveTG{䦘}{77732}
\saveTG{𦊶}{77732}
\saveTG{𦊬}{77732}
\saveTG{𧛽}{77732}
\saveTG{𧛣}{77732}
\saveTG{𧝷}{77732}
\saveTG{𧞜}{77732}
\saveTG{𠬾}{77732}
\saveTG{𦣴}{77732}
\saveTG{𨷤}{77732}
\saveTG{𩜬}{77732}
\saveTG{㒽}{77732}
\saveTG{𩛁}{77732}
\saveTG{县}{77732}
\saveTG{袰}{77732}
\saveTG{褜}{77732}
\saveTG{閬}{77732}
\saveTG{闤}{77732}
\saveTG{閎}{77732}
\saveTG{艮}{77732}
\saveTG{𫔨}{77732}
\saveTG{𧜤}{77732}
\saveTG{鼨}{77733}
\saveTG{𠬟}{77734}
\saveTG{𨲫}{77735}
\saveTG{𩶿}{77736}
\saveTG{𠭃}{77740}
\saveTG{臤}{77740}
\saveTG{毌}{77740}
\saveTG{𨱪}{77740}
\saveTG{𣱓}{77741}
\saveTG{𨱚}{77742}
\saveTG{𪔻}{77742}
\saveTG{𪕄}{77744}
\saveTG{𪕂}{77747}
\saveTG{𨳟}{77747}
\saveTG{𨴉}{77747}
\saveTG{㱼}{77747}
\saveTG{𪠯}{77747}
\saveTG{民}{77747}
\saveTG{殹}{77747}
\saveTG{殴}{77747}
\saveTG{毆}{77747}
\saveTG{毈}{77747}
\saveTG{𦦒}{77747}
\saveTG{𪕰}{77747}
\saveTG{𦥠}{77747}
\saveTG{𨳶}{77747}
\saveTG{𦊞}{77747}
\saveTG{𦣯}{77747}
\saveTG{𪖃}{77747}
\saveTG{𦣧}{77750}
\saveTG{母}{77750}
\saveTG{鼲}{77752}
\saveTG{𣫬}{77754}
\saveTG{𨱱}{77757}
\saveTG{𣫯}{77757}
\saveTG{䶉}{77762}
\saveTG{鼦}{77762}
\saveTG{𨱴}{77764}
\saveTG{䶅}{77764}
\saveTG{臼}{77770}
\saveTG{𦣩}{77770}
\saveTG{𤕪}{77770}
\saveTG{𠚜}{77770}
\saveTG{𠕄}{77770}
\saveTG{𤔼}{77770}
\saveTG{𠚞}{77770}
\saveTG{𦥔}{77770}
\saveTG{𦥑}{77770}
\saveTG{𦥓}{77770}
\saveTG{凹}{77770}
\saveTG{𡴺}{77772}
\saveTG{嶨}{77772}
\saveTG{關}{77772}
\saveTG{岡}{77772}
\saveTG{𡽬}{77772}
\saveTG{𪽗}{77772}
\saveTG{𦉝}{77772}
\saveTG{𡸧}{77772}
\saveTG{𡷦}{77772}
\saveTG{𡶓}{77772}
\saveTG{𨳹}{77772}
\saveTG{𨷒}{77772}
\saveTG{𡵒}{77772}
\saveTG{𡵹}{77772}
\saveTG{𡸭}{77772}
\saveTG{𠥫}{77772}
\saveTG{𪘦}{77772}
\saveTG{𡵠}{77772}
\saveTG{岊}{77772}
\saveTG{閊}{77772}
\saveTG{镼}{77772}
\saveTG{𡹩}{77772}
\saveTG{罂}{77772}
\saveTG{𨳼}{77774}
\saveTG{𦦥}{77774}
\saveTG{𡶘}{77774}
\saveTG{閰}{77777}
\saveTG{𨺫}{77777}
\saveTG{𦦀}{77777}
\saveTG{𦥮}{77777}
\saveTG{𦦂}{77777}
\saveTG{𨶒}{77777}
\saveTG{凸}{77777}
\saveTG{𠕣}{77777}
\saveTG{𠥓}{77777}
\saveTG{𦣤}{77777}
\saveTG{㠯}{77777}
\saveTG{䦡}{77777}
\saveTG{閻}{77777}
\saveTG{𠂼}{77777}
\saveTG{镹}{77780}
\saveTG{𪔼}{77781}
\saveTG{𨱟}{77782}
\saveTG{㰽}{77782}
\saveTG{𣢮}{77782}
\saveTG{㰼}{77782}
\saveTG{𣤆}{77782}
\saveTG{㰝}{77782}
\saveTG{𣢔}{77782}
\saveTG{𣢣}{77782}
\saveTG{欧}{77782}
\saveTG{歐}{77782}
\saveTG{𣢎}{77782}
\saveTG{𦥞}{77782}
\saveTG{𪕆}{77782}
\saveTG{𨶏}{77785}
\saveTG{𨲿}{77786}
\saveTG{𪕠}{77789}
\saveTG{𨷱}{77793}
\saveTG{闐}{77801}
\saveTG{𦋭}{77801}
\saveTG{興}{77801}
\saveTG{巽}{77801}
\saveTG{舆}{77801}
\saveTG{𨵙}{77801}
\saveTG{輿}{77801}
\saveTG{𨶷}{77801}
\saveTG{𠔹}{77801}
\saveTG{與}{77801}
\saveTG{𨷨}{77801}
\saveTG{𨴴}{77801}
\saveTG{𦥷}{77801}
\saveTG{𠔜}{77801}
\saveTG{𦦪}{77801}
\saveTG{𠭨}{77801}
\saveTG{𢁌}{77801}
\saveTG{𫔥}{77801}
\saveTG{𨃨}{77801}
\saveTG{𦦲}{77801}
\saveTG{閧}{77801}
\saveTG{鬨}{77801}
\saveTG{具}{77801}
\saveTG{閃}{77801}
\saveTG{巺}{77801}
\saveTG{㒷}{77801}
\saveTG{𨂢}{77802}
\saveTG{𣎡}{77802}
\saveTG{𡰰}{77802}
\saveTG{䦑}{77802}
\saveTG{閡}{77802}
\saveTG{贝}{77802}
\saveTG{贯}{77802}
\saveTG{贸}{77802}
\saveTG{𨷊}{77804}
\saveTG{䦬}{77804}
\saveTG{𫔫}{77804}
\saveTG{𨳨}{77804}
\saveTG{𨶥}{77804}
\saveTG{𡘑}{77804}
\saveTG{𦧂}{77804}
\saveTG{𥏈}{77804}
\saveTG{𨳓}{77804}
\saveTG{𪡲}{77804}
\saveTG{𨴊}{77804}
\saveTG{𡚣}{77804}
\saveTG{関}{77804}
\saveTG{闋}{77804}
\saveTG{𪥠}{77804}
\saveTG{闃}{77804}
\saveTG{𨷏}{77804}
\saveTG{𠖉}{77804}
\saveTG{𨵢}{77804}
\saveTG{𫔩}{77804}
\saveTG{𨶑}{77804}
\saveTG{𦦉}{77804}
\saveTG{𨴱}{77804}
\saveTG{䦫}{77805}
\saveTG{𨳺}{77805}
\saveTG{𨶎}{77806}
\saveTG{𧴾}{77806}
\saveTG{𨷪}{77806}
\saveTG{𧸧}{77806}
\saveTG{𨶰}{77806}
\saveTG{闠}{77806}
\saveTG{貿}{77806}
\saveTG{閴}{77806}
\saveTG{賢}{77806}
\saveTG{贀}{77806}
\saveTG{𧷅}{77806}
\saveTG{𨶛}{77806}
\saveTG{貫}{77806}
\saveTG{𧷈}{77806}
\saveTG{𨵪}{77806}
\saveTG{𧷙}{77806}
\saveTG{𧵽}{77806}
\saveTG{𦦴}{77806}
\saveTG{黌}{77806}
\saveTG{𦦫}{77807}
\saveTG{尺}{77807}
\saveTG{𦦚}{77807}
\saveTG{𠨠}{77807}
\saveTG{臾}{77807}
\saveTG{𦌨}{77808}
\saveTG{𤆆}{77809}
\saveTG{𤌨}{77809}
\saveTG{𤏿}{77809}
\saveTG{𤊏}{77809}
\saveTG{㸑}{77809}
\saveTG{爂}{77809}
\saveTG{爨}{77809}
\saveTG{焛}{77809}
\saveTG{燢}{77809}
\saveTG{𤑇}{77809}
\saveTG{𤓕}{77809}
\saveTG{𤌇}{77809}
\saveTG{𤐫}{77809}
\saveTG{𤌑}{77809}
\saveTG{𨷹}{77809}
\saveTG{𩙞}{77810}
\saveTG{𨷐}{77812}
\saveTG{闚}{77812}
\saveTG{𦌓}{77817}
\saveTG{𧶹}{77817}
\saveTG{𥐉}{77818}
\saveTG{赒}{77820}
\saveTG{购}{77820}
\saveTG{𨞾}{77827}
\saveTG{𨝑}{77827}
\saveTG{𪇬}{77827}
\saveTG{鄮}{77827}
\saveTG{𤆙}{77827}
\saveTG{𡳤}{77827}
\saveTG{𨛪}{77827}
\saveTG{𨟊}{77827}
\saveTG{𤬟}{77833}
\saveTG{赆}{77833}
\saveTG{𠮆}{77848}
\saveTG{闝}{77848}
\saveTG{赡}{77861}
\saveTG{赂}{77864}
\saveTG{𧹕}{77864}
\saveTG{𦧅}{77881}
\saveTG{𠔻}{77881}
\saveTG{歟}{77882}
\saveTG{赑}{77882}
\saveTG{𨷞}{77882}
\saveTG{𨶦}{77882}
\saveTG{𨷥}{77885}
\saveTG{閦}{77888}
\saveTG{𫔧}{77888}
\saveTG{𨷢}{77889}
\saveTG{𤓟}{77889}
\saveTG{𤆄}{77892}
\saveTG{㷂}{77894}
\saveTG{䵖}{77900}
\saveTG{𨶋}{77901}
\saveTG{𨳫}{77901}
\saveTG{𨶫}{77901}
\saveTG{𡱵}{77901}
\saveTG{𨳒}{77901}
\saveTG{闎}{77902}
\saveTG{閖}{77902}
\saveTG{澩}{77902}
\saveTG{𨳴}{77902}
\saveTG{𦂟}{77903}
\saveTG{𦀥}{77903}
\saveTG{𦦆}{77903}
\saveTG{繄}{77903}
\saveTG{𨵆}{77903}
\saveTG{䌠}{77903}
\saveTG{緊}{77903}
\saveTG{𠬹}{77903}
\saveTG{𦊮}{77904}
\saveTG{𣐻}{77904}
\saveTG{𠕨}{77904}
\saveTG{𣔚}{77904}
\saveTG{闑}{77904}
\saveTG{𣟰}{77904}
\saveTG{𦊋}{77904}
\saveTG{𦊘}{77904}
\saveTG{𣝑}{77904}
\saveTG{𠕖}{77904}
\saveTG{𣏷}{77904}
\saveTG{桑}{77904}
\saveTG{朵}{77904}
\saveTG{䊆}{77904}
\saveTG{𥽂}{77904}
\saveTG{𥹹}{77904}
\saveTG{閑}{77904}
\saveTG{𨵳}{77904}
\saveTG{𣑸}{77904}
\saveTG{𣐷}{77904}
\saveTG{𦦗}{77904}
\saveTG{𨵚}{77905}
\saveTG{𨴨}{77905}
\saveTG{𠬽}{77905}
\saveTG{䦨}{77905}
\saveTG{𣡈}{77905}
\saveTG{闌}{77906}
\saveTG{𦌴}{77915}
\saveTG{𦌐}{77915}
\saveTG{𨶳}{77917}
\saveTG{𨑁}{77921}
\saveTG{𫔣}{77921}
\saveTG{𦋩}{77923}
\saveTG{𪁁}{77927}
\saveTG{𦦵}{77931}
\saveTG{𦍃}{77932}
\saveTG{𣎿}{77941}
\saveTG{𠕮}{77941}
\saveTG{𦀒}{77941}
\saveTG{𨴩}{77941}
\saveTG{𣑍}{77942}
\saveTG{𣎻}{77942}
\saveTG{𨴺}{77943}
\saveTG{𣘦}{77944}
\saveTG{𠕭}{77946}
\saveTG{毇}{77947}
\saveTG{㭆}{77952}
\saveTG{𨷀}{77964}
\saveTG{𣤜}{77982}
\saveTG{𣤭}{77982}
\saveTG{𣡉}{77984}
\saveTG{𢁑}{77986}
\saveTG{𦋔}{77991}
\saveTG{𨶹}{77993}
\saveTG{𨷻}{77993}
\saveTG{𦈇}{77993}
\saveTG{𦋗}{77994}
\saveTG{𡳭}{77994}
\saveTG{䦥}{77994}
\saveTG{𠙲}{77999}
\saveTG{𩡩}{78000}
\saveTG{𩡮}{78000}
\saveTG{盬}{78102}
\saveTG{𩐌}{78102}
\saveTG{𥂭}{78102}
\saveTG{𥃡}{78102}
\saveTG{監}{78102}
\saveTG{𥃆}{78102}
\saveTG{𥃉}{78102}
\saveTG{鹽}{78102}
\saveTG{𧗄}{78102}
\saveTG{坠}{78104}
\saveTG{𤪋}{78104}
\saveTG{𡑘}{78104}
\saveTG{𤪈}{78104}
\saveTG{𡒓}{78104}
\saveTG{墜}{78104}
\saveTG{𥪡}{78108}
\saveTG{䜿}{78108}
\saveTG{鐆}{78109}
\saveTG{鑒}{78109}
\saveTG{鍳}{78109}
\saveTG{𨧋}{78109}
\saveTG{𩈁}{78111}
\saveTG{𩦹}{78112}
\saveTG{𦡶}{78112}
\saveTG{𩧴}{78114}
\saveTG{𩨦}{78117}
\saveTG{𫘕}{78117}
\saveTG{验}{78119}
\saveTG{𩨈}{78120}
\saveTG{𩧦}{78120}
\saveTG{𩨊}{78121}
\saveTG{𫗃}{78132}
\saveTG{𩨂}{78136}
\saveTG{𢼕}{78140}
\saveTG{𨯃}{78140}
\saveTG{骈}{78141}
\saveTG{𫘯}{78166}
\saveTG{}{78194}
\saveTG{𫒺}{78194}
\saveTG{𣍞}{78200}
\saveTG{𦘲}{78200}
\saveTG{队}{78200}
\saveTG{𦘩}{78200}
\saveTG{閄}{78200}
\saveTG{𠹨}{78211}
\saveTG{䬈}{78211}
\saveTG{胙}{78211}
\saveTG{阼}{78211}
\saveTG{𨽮}{78211}
\saveTG{𩘞}{78211}
\saveTG{𩖦}{78211}
\saveTG{𩘭}{78211}
\saveTG{覽}{78212}
\saveTG{髊}{78212}
\saveTG{隘}{78212}
\saveTG{䐤}{78212}
\saveTG{𦝤}{78212}
\saveTG{𪻁}{78212}
\saveTG{𨻹}{78212}
\saveTG{𦙐}{78212}
\saveTG{䬔}{78212}
\saveTG{𧇬}{78212}
\saveTG{𦟤}{78212}
\saveTG{㭀}{78212}
\saveTG{胣}{78212}
\saveTG{膉}{78212}
\saveTG{陁}{78212}
\saveTG{脱}{78212}
\saveTG{脫}{78212}
\saveTG{覧}{78212}
\saveTG{𩖵}{78213}
\saveTG{𡳃}{78213}
\saveTG{𨹫}{78214}
\saveTG{𦙋}{78214}
\saveTG{𨻮}{78214}
\saveTG{𢽙}{78214}
\saveTG{𩙉}{78214}
\saveTG{𩘓}{78214}
\saveTG{脞}{78214}
\saveTG{𨹑}{78214}
\saveTG{𩗚}{78214}
\saveTG{𨺴}{78214}
\saveTG{𩘄}{78214}
\saveTG{𡳳}{78216}
\saveTG{𪨟}{78216}
\saveTG{𫗀}{78216}
\saveTG{𨼃}{78217}
\saveTG{𡳦}{78217}
\saveTG{阣}{78217}
\saveTG{肐}{78217}
\saveTG{𡳗}{78217}
\saveTG{𩨺}{78217}
\saveTG{𨹪}{78217}
\saveTG{𩨘}{78217}
\saveTG{𦞝}{78217}
\saveTG{𨸛}{78217}
\saveTG{䏗}{78217}
\saveTG{𨽭}{78218}
\saveTG{𨼹}{78218}
\saveTG{𨽙}{78218}
\saveTG{𩘶}{78218}
\saveTG{𫖿}{78218}
\saveTG{颴}{78218}
\saveTG{𫆤}{78219}
\saveTG{险}{78219}
\saveTG{脸}{78219}
\saveTG{阶}{78220}
\saveTG{骱}{78220}
\saveTG{腧}{78221}
\saveTG{𦙝}{78221}
\saveTG{隃}{78221}
\saveTG{𫆨}{78221}
\saveTG{胗}{78222}
\saveTG{𨼾}{78223}
\saveTG{𦜑}{78227}
\saveTG{肦}{78227}
\saveTG{𦢡}{78227}
\saveTG{𦚿}{78227}
\saveTG{朌}{78227}
\saveTG{𨻔}{78227}
\saveTG{膓}{78227}
\saveTG{肣}{78227}
\saveTG{腀}{78227}
\saveTG{𫕎}{78227}
\saveTG{𩨪}{78227}
\saveTG{肹}{78227}
\saveTG{𤫼}{78227}
\saveTG{陯}{78227}
\saveTG{𩨧}{78227}
\saveTG{𦡬}{78227}
\saveTG{𦙕}{78227}
\saveTG{䐥}{78227}
\saveTG{䏲}{78227}
\saveTG{𨸣}{78227}
\saveTG{𨹥}{78227}
\saveTG{𦛽}{78227}
\saveTG{𩪶}{78227}
\saveTG{𢃘}{78227}
\saveTG{䧔}{78230}
\saveTG{𤔉}{78230}
\saveTG{𦜲}{78231}
\saveTG{𩩈}{78231}
\saveTG{𦛰}{78231}
\saveTG{𨼊}{78231}
\saveTG{𦝴}{78231}
\saveTG{膴}{78231}
\saveTG{𤓼}{78232}
\saveTG{陰}{78232}
\saveTG{朎}{78232}
\saveTG{𩪴}{78232}
\saveTG{䑆}{78232}
\saveTG{𤄈}{78232}
\saveTG{𤂟}{78232}
\saveTG{𩩶}{78232}
\saveTG{阾}{78232}
\saveTG{隊}{78232}
\saveTG{腍}{78232}
\saveTG{隂}{78232}
\saveTG{䯍}{78232}
\saveTG{脍}{78232}
\saveTG{𤀩}{78232}
\saveTG{𣷙}{78232}
\saveTG{䯟}{78232}
\saveTG{𦠵}{78232}
\saveTG{𨹉}{78232}
\saveTG{𩙍}{78232}
\saveTG{䐋}{78233}
\saveTG{𩩦}{78233}
\saveTG{隧}{78233}
\saveTG{𦞮}{78234}
\saveTG{𦞸}{78234}
\saveTG{𨼶}{78236}
\saveTG{𦜱}{78236}
\saveTG{𦢅}{78236}
\saveTG{𦝅}{78237}
\saveTG{䯡}{78237}
\saveTG{㼓}{78237}
\saveTG{隒}{78237}
\saveTG{膁}{78237}
\saveTG{𢡣}{78238}
\saveTG{𪯚}{78240}
\saveTG{𢽷}{78240}
\saveTG{𣀼}{78240}
\saveTG{𩪵}{78240}
\saveTG{䧩}{78240}
\saveTG{𩪕}{78240}
\saveTG{𨻖}{78240}
\saveTG{敶}{78240}
\saveTG{𦞀}{78240}
\saveTG{𨸩}{78240}
\saveTG{𪯱}{78240}
\saveTG{肸}{78240}
\saveTG{𢽵}{78240}
\saveTG{𦜍}{78240}
\saveTG{㬿}{78240}
\saveTG{𤉺}{78240}
\saveTG{敐}{78240}
\saveTG{㬳}{78240}
\saveTG{𦟔}{78240}
\saveTG{隞}{78240}
\saveTG{𢾹}{78240}
\saveTG{𢽲}{78240}
\saveTG{䯎}{78241}
\saveTG{胼}{78241}
\saveTG{𨹗}{78241}
\saveTG{骿}{78241}
\saveTG{𦣈}{78242}
\saveTG{𦝡}{78244}
\saveTG{𣍮}{78247}
\saveTG{䧗}{78247}
\saveTG{腹}{78247}
\saveTG{脌}{78250}
\saveTG{𦟃}{78251}
\saveTG{𦎨}{78251}
\saveTG{羘}{78251}
\saveTG{䧧}{78253}
\saveTG{𦠱}{78253}
\saveTG{𩪻}{78253}
\saveTG{𦡫}{78253}
\saveTG{脢}{78257}
\saveTG{𤔯}{78257}
\saveTG{𣎮}{78259}
\saveTG{𨸮}{78260}
\saveTG{膳}{78261}
\saveTG{𦡮}{78261}
\saveTG{𨼵}{78261}
\saveTG{𩩂}{78261}
\saveTG{䏩}{78261}
\saveTG{𦛜}{78262}
\saveTG{𩩊}{78262}
\saveTG{𩩗}{78264}
\saveTG{𦝱}{78264}
\saveTG{膾}{78266}
\saveTG{𦠇}{78266}
\saveTG{䯤}{78266}
\saveTG{𦞛}{78267}
\saveTG{𦛱}{78268}
\saveTG{䧍}{78268}
\saveTG{𦝵}{78269}
\saveTG{𨹊}{78280}
\saveTG{𣎣}{78282}
\saveTG{𣎓}{78282}
\saveTG{𦛔}{78282}
\saveTG{𧲓}{78282}
\saveTG{䐫}{78282}
\saveTG{𦝺}{78284}
\saveTG{朕}{78284}
\saveTG{𨺰}{78284}
\saveTG{𫕀}{78284}
\saveTG{𦢋}{78285}
\saveTG{𦣍}{78285}
\saveTG{𦢹}{78285}
\saveTG{𨹩}{78285}
\saveTG{險}{78286}
\saveTG{険}{78286}
\saveTG{臉}{78286}
\saveTG{𨹈}{78289}
\saveTG{𦙳}{78290}
\saveTG{𪱡}{78291}
\saveTG{𫕗}{78293}
\saveTG{除}{78294}
\saveTG{𤬀}{78294}
\saveTG{𩢐}{78311}
\saveTG{駞}{78312}
\saveTG{駾}{78312}
\saveTG{𩤀}{78312}
\saveTG{𩢯}{78312}
\saveTG{𩤼}{78312}
\saveTG{𩥙}{78312}
\saveTG{𩣜}{78314}
\saveTG{駩}{78314}
\saveTG{𩡹}{78317}
\saveTG{𩥀}{78317}
\saveTG{𩡺}{78320}
\saveTG{騚}{78321}
\saveTG{騟}{78321}
\saveTG{駗}{78322}
\saveTG{𩥛}{78327}
\saveTG{𩢣}{78327}
\saveTG{䭻}{78327}
\saveTG{𫘔}{78331}
\saveTG{𩧗}{78331}
\saveTG{䮉}{78332}
\saveTG{𩥆}{78332}
\saveTG{騐}{78332}
\saveTG{駖}{78332}
\saveTG{𩣭}{78333}
\saveTG{𪬒}{78334}
\saveTG{𢚓}{78334}
\saveTG{愍}{78334}
\saveTG{𩸻}{78336}
\saveTG{𨽎}{78338}
\saveTG{𩥽}{78339}
\saveTG{䭸}{78340}
\saveTG{䮯}{78340}
\saveTG{𩥹}{78340}
\saveTG{𩤝}{78340}
\saveTG{駇}{78340}
\saveTG{驐}{78340}
\saveTG{𫌭}{78340}
\saveTG{駢}{78341}
\saveTG{䮡}{78347}
\saveTG{𩣓}{78347}
\saveTG{𩣆}{78351}
\saveTG{𩥍}{78352}
\saveTG{䮼}{78357}
\saveTG{𩦐}{78361}
\saveTG{騇}{78364}
\saveTG{驓}{78366}
\saveTG{𩦱}{78366}
\saveTG{𩣥}{78368}
\saveTG{𩢿}{78372}
\saveTG{𩦸}{78381}
\saveTG{𩦙}{78384}
\saveTG{𫘑}{78384}
\saveTG{験}{78386}
\saveTG{𩥾}{78386}
\saveTG{驗}{78386}
\saveTG{𩢜}{78390}
\saveTG{駼}{78394}
\saveTG{𫆈}{78401}
\saveTG{𢿭}{78406}
\saveTG{𡦋}{78407}
\saveTG{𦣵}{78407}
\saveTG{𠮄}{78412}
\saveTG{㥖}{78430}
\saveTG{𢼅}{78440}
\saveTG{斆}{78440}
\saveTG{敪}{78440}
\saveTG{𣀭}{78440}
\saveTG{䑫}{78441}
\saveTG{𨺧}{78464}
\saveTG{𪧂}{78497}
\saveTG{𢱯}{78502}
\saveTG{𢯄}{78502}
\saveTG{𢰞}{78502}
\saveTG{擥}{78502}
\saveTG{𨍒}{78506}
\saveTG{𨏊}{78508}
\saveTG{譼}{78601}
\saveTG{𧩾}{78601}
\saveTG{𧨭}{78601}
\saveTG{𥊇}{78601}
\saveTG{𩈵}{78602}
\saveTG{𣇻}{78604}
\saveTG{𨢒}{78604}
\saveTG{睯}{78604}
\saveTG{暋}{78604}
\saveTG{䃋}{78620}
\saveTG{𢾞}{78640}
\saveTG{𧩸}{78646}
\saveTG{𥖩}{78666}
\saveTG{𥑺}{78672}
\saveTG{臥}{78700}
\saveTG{𫇉}{78711}
\saveTG{𠥸}{78711}
\saveTG{𦥬}{78711}
\saveTG{䑘}{78712}
\saveTG{𦣪}{78712}
\saveTG{𪕶}{78712}
\saveTG{𩝱}{78712}
\saveTG{𨲠}{78712}
\saveTG{𪕌}{78713}
\saveTG{𣀟}{78714}
\saveTG{𢽼}{78714}
\saveTG{𣱑}{78714}
\saveTG{㽉}{78717}
\saveTG{𢀧}{78722}
\saveTG{𦣡}{78727}
\saveTG{䦇}{78727}
\saveTG{䶃}{78727}
\saveTG{𨲬}{78727}
\saveTG{𪕧}{78727}
\saveTG{鼢}{78727}
\saveTG{𨲼}{78727}
\saveTG{𫌎}{78732}
\saveTG{𪩩}{78732}
\saveTG{𩟨}{78732}
\saveTG{鼸}{78737}
\saveTG{𨱛}{78738}
\saveTG{𢼳}{78740}
\saveTG{𢿛}{78740}
\saveTG{𢾺}{78740}
\saveTG{攺}{78740}
\saveTG{敃}{78740}
\saveTG{𪖄}{78740}
\saveTG{𨱝}{78740}
\saveTG{䦋}{78740}
\saveTG{𢽇}{78740}
\saveTG{𨲶}{78742}
\saveTG{𪕒}{78744}
\saveTG{𦣬}{78744}
\saveTG{𨲡}{78744}
\saveTG{𨱹}{78747}
\saveTG{𨲥}{78756}
\saveTG{𨲛}{78762}
\saveTG{𪕛}{78762}
\saveTG{𦣷}{78764}
\saveTG{臨}{78766}
\saveTG{𨲯}{78766}
\saveTG{㟩}{78772}
\saveTG{𦉞}{78774}
\saveTG{𪖁}{78782}
\saveTG{𨲧}{78782}
\saveTG{𨱯}{78790}
\saveTG{䑐}{78790}
\saveTG{𪥡}{78804}
\saveTG{𤑾}{78809}
\saveTG{𤎩}{78809}
\saveTG{𫎨}{78819}
\saveTG{赚}{78837}
\saveTG{𤊮}{78840}
\saveTG{败}{78840}
\saveTG{赠}{78866}
\saveTG{𧮾}{78868}
\saveTG{赊}{78891}
\saveTG{𦃂}{78903}
\saveTG{𦂳}{78903}
\saveTG{𦄊}{78903}
\saveTG{𣠿}{78904}
\saveTG{𥽏}{78904}
\saveTG{𥽐}{78904}
\saveTG{𣜹}{78927}
\saveTG{𪾕}{79102}
\saveTG{𫘡}{79117}
\saveTG{𡮻}{79120}
\saveTG{𡮼}{79120}
\saveTG{𩨇}{79144}
\saveTG{}{79159}
\saveTG{𫘚}{79181}
\saveTG{𧑽}{79182}
\saveTG{𤑠}{79189}
\saveTG{𡮅}{79200}
\saveTG{𡮤}{79200}
\saveTG{𡮉}{79200}
\saveTG{𨼺}{79211}
\saveTG{颵}{79212}
\saveTG{腃}{79212}
\saveTG{𩖥}{79212}
\saveTG{𨹵}{79212}
\saveTG{𨹂}{79212}
\saveTG{胱}{79212}
\saveTG{𩗵}{79212}
\saveTG{𦞔}{79212}
\saveTG{𠗲}{79213}
\saveTG{隚}{79214}
\saveTG{塍}{79214}
\saveTG{膛}{79214}
\saveTG{𨽓}{79214}
\saveTG{𣍺}{79215}
\saveTG{𧡼}{79216}
\saveTG{䲢}{79216}
\saveTG{}{79216}
\saveTG{䯑}{79217}
\saveTG{颷}{79218}
\saveTG{飈}{79218}
\saveTG{飚}{79218}
\saveTG{𩖧}{79218}
\saveTG{𩙪}{79218}
\saveTG{𩘌}{79218}
\saveTG{𡲸}{79219}
\saveTG{𩙆}{79219}
\saveTG{𩗐}{79219}
\saveTG{𦡎}{79219}
\saveTG{䏚}{79220}
\saveTG{𡮩}{79220}
\saveTG{𩨡}{79220}
\saveTG{隲}{79221}
\saveTG{幐}{79227}
\saveTG{腾}{79227}
\saveTG{勝}{79227}
\saveTG{陗}{79227}
\saveTG{朥}{79227}
\saveTG{㬺}{79227}
\saveTG{𫆷}{79227}
\saveTG{𫆟}{79227}
\saveTG{𪀅}{79227}
\saveTG{𫚲}{79227}
\saveTG{𩩓}{79227}
\saveTG{𣎃}{79227}
\saveTG{騰}{79227}
\saveTG{䏴}{79227}
\saveTG{𣎲}{79231}
\saveTG{黱}{79231}
\saveTG{𤑘}{79231}
\saveTG{𧜜}{79232}
\saveTG{𤫾}{79232}
\saveTG{螣}{79236}
\saveTG{鰧}{79236}
\saveTG{𤔶}{79236}
\saveTG{𦡗}{79238}
\saveTG{𣍹}{79244}
\saveTG{𦝼}{79244}
\saveTG{𣎌}{79244}
\saveTG{媵}{79244}
\saveTG{髅}{79244}
\saveTG{𪢈}{79244}
\saveTG{𦛀}{79247}
\saveTG{胖}{79250}
\saveTG{𪨠}{79252}
\saveTG{𨽃}{79258}
\saveTG{隣}{79259}
\saveTG{膦}{79259}
\saveTG{隣}{79259}
\saveTG{脳}{79260}
\saveTG{謄}{79261}
\saveTG{𩩭}{79262}
\saveTG{𫆦}{79262}
\saveTG{𦡁}{79266}
\saveTG{𨼴}{79266}
\saveTG{𣃉}{79266}
\saveTG{膡}{79268}
\saveTG{𣎒}{79268}
\saveTG{𦛡}{79280}
\saveTG{阦}{79280}
\saveTG{𨺹}{79280}
\saveTG{𨃗}{79282}
\saveTG{}{79282}
\saveTG{䐝}{79286}
\saveTG{𨻈}{79286}
\saveTG{𩪈}{79286}
\saveTG{賸}{79286}
\saveTG{𣎎}{79288}
\saveTG{䐐}{79289}
\saveTG{𪱪}{79289}
\saveTG{𩩧}{79289}
\saveTG{腅}{79289}
\saveTG{𣎝}{79293}
\saveTG{縢}{79293}
\saveTG{脒}{79294}
\saveTG{榺}{79294}
\saveTG{隙}{79296}
\saveTG{𨻶}{79296}
\saveTG{滕}{79299}
\saveTG{駫}{79312}
\saveTG{𩡾}{79320}
\saveTG{𪂗}{79327}
\saveTG{𩦜}{79327}
\saveTG{驣}{79327}
\saveTG{𩻷}{79336}
\saveTG{騨}{79350}
\saveTG{𩢔}{79350}
\saveTG{驎}{79359}
\saveTG{𫘉}{79380}
\saveTG{𩦬}{79382}
\saveTG{𫙝}{79384}
\saveTG{𩧟}{79389}
\saveTG{𩥱}{79399}
\saveTG{𤔵}{79689}
\saveTG{𨲏}{79712}
\saveTG{𪖀}{79715}
\saveTG{𦥰}{79717}
\saveTG{𪓺}{79717}
\saveTG{𪕓}{79717}
\saveTG{𥸿}{79719}
\saveTG{𨲆}{79727}
\saveTG{𨲮}{79727}
\saveTG{𨲓}{79762}
\saveTG{𡮭}{79800}
\saveTG{𡮺}{79800}
\saveTG{𡮸}{79884}
\saveTG{赕}{79889}
\saveTG{入}{80000}
\saveTG{人}{80000}
\saveTG{丷}{80000}
\saveTG{八}{80000}
\saveTG{𠆢}{80000}
\saveTG{}{80000}
\saveTG{䒑}{80001}
\saveTG{气}{80017}
\saveTG{𠔃}{80027}
\saveTG{兮}{80027}
\saveTG{𠆥}{80040}
\saveTG{𠓛}{80100}
\saveTG{亼}{80100}
\saveTG{𨥱}{80100}
\saveTG{𠓞}{80101}
\saveTG{企}{80101}
\saveTG{兰}{80101}
\saveTG{佱}{80101}
\saveTG{仝}{80102}
\saveTG{羞}{80102}
\saveTG{益}{80102}
\saveTG{𠆳}{80102}
\saveTG{㒰}{80102}
\saveTG{盆}{80102}
\saveTG{差}{80102}
\saveTG{並}{80102}
\saveTG{盫}{80102}
\saveTG{𥁧}{80102}
\saveTG{盖}{80102}
\saveTG{盦}{80102}
\saveTG{盒}{80102}
\saveTG{𣥳}{80102}
\saveTG{𥁌}{80102}
\saveTG{𥂝}{80102}
\saveTG{𠓯}{80102}
\saveTG{㿽}{80102}
\saveTG{𥂜}{80102}
\saveTG{𥂴}{80102}
\saveTG{𣥝}{80102}
\saveTG{𠇯}{80102}
\saveTG{𦥾}{80104}
\saveTG{𡌵}{80104}
\saveTG{𠈘}{80104}
\saveTG{坌}{80104}
\saveTG{𡌆}{80104}
\saveTG{𡋛}{80104}
\saveTG{𡌫}{80104}
\saveTG{𤤎}{80104}
\saveTG{𤫆}{80104}
\saveTG{𤥓}{80104}
\saveTG{𦥈}{80104}
\saveTG{𦍌}{80104}
\saveTG{𡉀}{80104}
\saveTG{}{80104}
\saveTG{𦚙}{80104}
\saveTG{全}{80104}
\saveTG{𡍽}{80104}
\saveTG{𨩛}{80105}
\saveTG{𪜽}{80105}
\saveTG{𩚃}{80106}
\saveTG{鲝}{80106}
\saveTG{𧯚}{80108}
\saveTG{𨥄}{80108}
\saveTG{𨦋}{80108}
\saveTG{𠈔}{80108}
\saveTG{𠉓}{80109}
\saveTG{𨤾}{80109}
\saveTG{𨧊}{80109}
\saveTG{釜}{80109}
\saveTG{佥}{80109}
\saveTG{釒}{80109}
\saveTG{金}{80109}
\saveTG{釡}{80109}
\saveTG{𠊍}{80109}
\saveTG{𨥀}{80109}
\saveTG{𨥏}{80109}
\saveTG{釯}{80110}
\saveTG{𠇎}{80110}
\saveTG{𨰞}{80111}
\saveTG{𫕽}{80111}
\saveTG{𨬧}{80112}
\saveTG{𫓕}{80112}
\saveTG{𫒜}{80112}
\saveTG{銃}{80112}
\saveTG{鏡}{80112}
\saveTG{鋶}{80112}
\saveTG{鏕}{80112}
\saveTG{𠐰}{80112}
\saveTG{𢚖}{80112}
\saveTG{𥃣}{80112}
\saveTG{𪼈}{80114}
\saveTG{𪼉}{80114}
\saveTG{𨮞}{80114}
\saveTG{𠓾}{80114}
\saveTG{𨫎}{80114}
\saveTG{𨮻}{80114}
\saveTG{𨫈}{80114}
\saveTG{錱}{80114}
\saveTG{𫒕}{80114}
\saveTG{鉒}{80114}
\saveTG{𩀩}{80115}
\saveTG{𩁳}{80115}
\saveTG{𨫻}{80115}
\saveTG{鐘}{80115}
\saveTG{錐}{80115}
\saveTG{𨬉}{80115}
\saveTG{𫓙}{80115}
\saveTG{鏟}{80115}
\saveTG{𨭖}{80116}
\saveTG{䥝}{80117}
\saveTG{𨪡}{80117}
\saveTG{𨦊}{80117}
\saveTG{𨰠}{80117}
\saveTG{氢}{80117}
\saveTG{氫}{80117}
\saveTG{𣱦}{80117}
\saveTG{氩}{80117}
\saveTG{氬}{80117}
\saveTG{氲}{80117}
\saveTG{氳}{80117}
\saveTG{鈧}{80117}
\saveTG{𨩎}{80117}
\saveTG{𨰊}{80117}
\saveTG{𨯤}{80117}
\saveTG{𠎤}{80117}
\saveTG{𦎶}{80117}
\saveTG{𣱰}{80117}
\saveTG{𨰅}{80117}
\saveTG{𣱠}{80117}
\saveTG{𣱖}{80117}
\saveTG{𣱧}{80117}
\saveTG{𧉁}{80117}
\saveTG{𨩾}{80117}
\saveTG{鉝}{80118}
\saveTG{鑫}{80119}
\saveTG{侴}{80120}
\saveTG{𨪃}{80121}
\saveTG{𦐈}{80121}
\saveTG{𨪆}{80121}
\saveTG{𨮛}{80122}
\saveTG{𨩱}{80122}
\saveTG{鑇}{80123}
\saveTG{鏞}{80127}
\saveTG{翕}{80127}
\saveTG{翁}{80127}
\saveTG{鍗}{80127}
\saveTG{鈰}{80127}
\saveTG{鐫}{80127}
\saveTG{鎸}{80127}
\saveTG{𨧤}{80127}
\saveTG{翦}{80127}
\saveTG{鎬}{80127}
\saveTG{鈁}{80127}
\saveTG{鏑}{80127}
\saveTG{鎊}{80127}
\saveTG{𫓘}{80127}
\saveTG{𠑴}{80127}
\saveTG{𨬺}{80127}
\saveTG{𦏾}{80127}
\saveTG{𨪞}{80127}
\saveTG{𨫢}{80127}
\saveTG{錥}{80127}
\saveTG{𫒚}{80130}
\saveTG{䥋}{80130}
\saveTG{𧔭}{80131}
\saveTG{𫒮}{80131}
\saveTG{𨧻}{80131}
\saveTG{鐎}{80131}
\saveTG{䥰}{80131}
\saveTG{𨨸}{80131}
\saveTG{鑣}{80131}
\saveTG{𨮙}{80132}
\saveTG{銥}{80132}
\saveTG{鉉}{80132}
\saveTG{鑲}{80132}
\saveTG{鉱}{80132}
\saveTG{𠑳}{80132}
\saveTG{鎄}{80132}
\saveTG{𨮹}{80132}
\saveTG{𠑮}{80132}
\saveTG{㒪}{80132}
\saveTG{𨬤}{80134}
\saveTG{𧕶}{80136}
\saveTG{𧔑}{80136}
\saveTG{𧌶}{80136}
\saveTG{𧒃}{80136}
\saveTG{𧊧}{80136}
\saveTG{蠤}{80136}
\saveTG{鐿}{80136}
\saveTG{𧉊}{80136}
\saveTG{𧕙}{80136}
\saveTG{𨯑}{80136}
\saveTG{𧓀}{80136}
\saveTG{蚠}{80136}
\saveTG{鐮}{80137}
\saveTG{鏣}{80137}
\saveTG{䥥}{80137}
\saveTG{鈫}{80140}
\saveTG{鐴}{80141}
\saveTG{𫓅}{80141}
\saveTG{鎨}{80141}
\saveTG{𨫆}{80141}
\saveTG{𨫚}{80141}
\saveTG{𨈿}{80141}
\saveTG{𨯢}{80141}
\saveTG{鋅}{80141}
\saveTG{𨰚}{80142}
\saveTG{䤳}{80142}
\saveTG{𫓆}{80143}
\saveTG{𨫏}{80143}
\saveTG{𨨧}{80144}
\saveTG{𨦩}{80144}
\saveTG{𨪟}{80146}
\saveTG{鏱}{80146}
\saveTG{𨯙}{80147}
\saveTG{鍍}{80147}
\saveTG{𩠮}{80147}
\saveTG{錞}{80147}
\saveTG{𫒤}{80147}
\saveTG{𣀚}{80148}
\saveTG{𩠰}{80148}
\saveTG{鉸}{80148}
\saveTG{錊}{80148}
\saveTG{𨰸}{80151}
\saveTG{𨯇}{80152}
\saveTG{𨬨}{80152}
\saveTG{𨩏}{80152}
\saveTG{𨬦}{80152}
\saveTG{鏲}{80153}
\saveTG{𣼪}{80156}
\saveTG{龲}{80156}
\saveTG{𠩖}{80157}
\saveTG{𨭤}{80157}
\saveTG{錇}{80161}
\saveTG{𨦼}{80161}
\saveTG{𨮀}{80162}
\saveTG{𨬙}{80162}
\saveTG{𨦽}{80164}
\saveTG{鎕}{80165}
\saveTG{𠊋}{80175}
\saveTG{羨}{80182}
\saveTG{䤤}{80182}
\saveTG{羡}{80182}
\saveTG{𨩌}{80184}
\saveTG{𨪏}{80184}
\saveTG{鑛}{80186}
\saveTG{𨯄}{80189}
\saveTG{𨧲}{80194}
\saveTG{𨮍}{80194}
\saveTG{𨬈}{80194}
\saveTG{𨧖}{80194}
\saveTG{鏶}{80194}
\saveTG{鍄}{80196}
\saveTG{𩾌}{80199}
\saveTG{鏮}{80199}
\saveTG{个}{80200}
\saveTG{𩙿}{80200}
\saveTG{𠆤}{80200}
\saveTG{㸘}{80201}
\saveTG{㒱}{80203}
\saveTG{𠇏}{80207}
\saveTG{丫}{80207}
\saveTG{今}{80207}
\saveTG{爹}{80207}
\saveTG{𠭳}{80211}
\saveTG{𡰕}{80211}
\saveTG{乍}{80211}
\saveTG{龕}{80211}
\saveTG{𡰟}{80211}
\saveTG{麄}{80212}
\saveTG{𨭒}{80212}
\saveTG{𠉞}{80212}
\saveTG{兊}{80212}
\saveTG{兌}{80212}
\saveTG{𠓡}{80212}
\saveTG{尣}{80212}
\saveTG{兑}{80212}
\saveTG{𨿄}{80215}
\saveTG{𦏮}{80215}
\saveTG{雂}{80215}
\saveTG{𩀞}{80215}
\saveTG{𨿝}{80215}
\saveTG{𡯷}{80216}
\saveTG{䶳}{80217}
\saveTG{𠆪}{80217}
\saveTG{𫖻}{80217}
\saveTG{𠇃}{80217}
\saveTG{𠔲}{80217}
\saveTG{𪋨}{80217}
\saveTG{𠈑}{80217}
\saveTG{𪛓}{80217}
\saveTG{𠎍}{80217}
\saveTG{𠉼}{80217}
\saveTG{𪎨}{80217}
\saveTG{𠒁}{80217}
\saveTG{𠎛}{80217}
\saveTG{𦍑}{80217}
\saveTG{𣱚}{80217}
\saveTG{𣱞}{80217}
\saveTG{㲵}{80217}
\saveTG{𣱢}{80217}
\saveTG{𦍛}{80217}
\saveTG{𠙠}{80217}
\saveTG{𪚙}{80217}
\saveTG{𪚕}{80217}
\saveTG{氞}{80217}
\saveTG{氚}{80217}
\saveTG{氘}{80217}
\saveTG{氛}{80217}
\saveTG{氪}{80217}
\saveTG{氖}{80217}
\saveTG{氕}{80217}
\saveTG{氰}{80217}
\saveTG{氱}{80217}
\saveTG{𣱕}{80217}
\saveTG{𧡟}{80217}
\saveTG{𣱙}{80217}
\saveTG{氝}{80217}
\saveTG{𪚡}{80217}
\saveTG{𡯂}{80217}
\saveTG{𠈢}{80217}
\saveTG{𡢣}{80217}
\saveTG{𢔆}{80218}
\saveTG{介}{80220}
\saveTG{𠈛}{80220}
\saveTG{前}{80221}
\saveTG{斧}{80221}
\saveTG{俞}{80221}
\saveTG{𢒃}{80222}
\saveTG{㐱}{80222}
\saveTG{𢁰}{80227}
\saveTG{𢁭}{80227}
\saveTG{㠳}{80227}
\saveTG{兯}{80227}
\saveTG{䏌}{80227}
\saveTG{𠋁}{80227}
\saveTG{𠇰}{80227}
\saveTG{𦚷}{80227}
\saveTG{𠎘}{80227}
\saveTG{𩰕}{80227}
\saveTG{侖}{80227}
\saveTG{㸗}{80227}
\saveTG{𠛈}{80227}
\saveTG{𦚶}{80227}
\saveTG{𤕓}{80227}
\saveTG{𩱁}{80227}
\saveTG{𠎚}{80227}
\saveTG{𠌈}{80227}
\saveTG{𠓹}{80227}
\saveTG{𠔑}{80227}
\saveTG{𠑂}{80227}
\saveTG{𦛝}{80227}
\saveTG{分}{80227}
\saveTG{肏}{80227}
\saveTG{弟}{80227}
\saveTG{剪}{80227}
\saveTG{𢅊}{80227}
\saveTG{爷}{80227}
\saveTG{龠}{80227}
\saveTG{𠭻}{80227}
\saveTG{𠞽}{80227}
\saveTG{𢃬}{80227}
\saveTG{𠇮}{80227}
\saveTG{𩢁}{80227}
\saveTG{𩱋}{80227}
\saveTG{𠉙}{80227}
\saveTG{𨷫}{80227}
\saveTG{𩰱}{80227}
\saveTG{禽}{80227}
\saveTG{𩱢}{80227}
\saveTG{𢄔}{80227}
\saveTG{𫜜}{80227}
\saveTG{𢄕}{80227}
\saveTG{𠋅}{80227}
\saveTG{𢂂}{80227}
\saveTG{𪞊}{80227}
\saveTG{𢁥}{80227}
\saveTG{𥜼}{80227}
\saveTG{𠔕}{80227}
\saveTG{𤕨}{80227}
\saveTG{㔦}{80228}
\saveTG{𠓪}{80228}
\saveTG{𠌺}{80228}
\saveTG{养}{80228}
\saveTG{𠁭}{80228}
\saveTG{𠌕}{80228}
\saveTG{𠇍}{80230}
\saveTG{𫇥}{80231}
\saveTG{𨱓}{80231}
\saveTG{兪}{80232}
\saveTG{𣲎}{80232}
\saveTG{𣴲}{80232}
\saveTG{𣴴}{80232}
\saveTG{𣹅}{80232}
\saveTG{𠊺}{80232}
\saveTG{𣴎}{80232}
\saveTG{㒸}{80232}
\saveTG{𠍐}{80233}
\saveTG{𩠘}{80236}
\saveTG{兼}{80237}
\saveTG{𠉽}{80237}
\saveTG{𠇥}{80240}
\saveTG{𡬴}{80241}
\saveTG{𠭶}{80247}
\saveTG{𠬬}{80247}
\saveTG{㿯}{80247}
\saveTG{舞}{80251}
\saveTG{羲}{80253}
\saveTG{𦏁}{80253}
\saveTG{𠎏}{80256}
\saveTG{𪛏}{80261}
\saveTG{𧨁}{80261}
\saveTG{𠐂}{80262}
\saveTG{倉}{80267}
\saveTG{𠔅}{80277}
\saveTG{𠔆}{80277}
\saveTG{𠋓}{80277}
\saveTG{𡰤}{80277}
\saveTG{𤈩}{80289}
\saveTG{亽}{80300}
\saveTG{令}{80302}
\saveTG{𠔇}{80302}
\saveTG{仒}{80303}
\saveTG{𨗜}{80309}
\saveTG{𨾠}{80315}
\saveTG{忥}{80317}
\saveTG{氡}{80317}
\saveTG{𩡨}{80327}
\saveTG{𪀝}{80327}
\saveTG{鶿}{80327}
\saveTG{𩿈}{80327}
\saveTG{𤆋}{80330}
\saveTG{𢙅}{80330}
\saveTG{𢗊}{80330}
\saveTG{𠓣}{80330}
\saveTG{羔}{80331}
\saveTG{𪬐}{80331}
\saveTG{𢗁}{80331}
\saveTG{𨚋}{80331}
\saveTG{𢛪}{80331}
\saveTG{無}{80331}
\saveTG{恙}{80331}
\saveTG{怎}{80331}
\saveTG{𢖴}{80331}
\saveTG{煎}{80332}
\saveTG{忿}{80332}
\saveTG{㥐}{80332}
\saveTG{愈}{80332}
\saveTG{念}{80332}
\saveTG{𢜒}{80332}
\saveTG{忩}{80333}
\saveTG{慈}{80333}
\saveTG{𤋌}{80333}
\saveTG{𪬄}{80333}
\saveTG{㤣}{80334}
\saveTG{𢙽}{80335}
\saveTG{鮺}{80336}
\saveTG{𦏬}{80336}
\saveTG{𩺃}{80336}
\saveTG{𤋃}{80336}
\saveTG{𤎰}{80336}
\saveTG{总}{80336}
\saveTG{鯗}{80336}
\saveTG{𠔥}{80337}
\saveTG{缹}{80337}
\saveTG{𢚱}{80338}
\saveTG{𢝑}{80338}
\saveTG{𪬪}{80338}
\saveTG{𢚒}{80339}
\saveTG{𢞥}{80339}
\saveTG{悆}{80339}
\saveTG{𡬦}{80341}
\saveTG{𦍸}{80344}
\saveTG{尊}{80346}
\saveTG{仐}{80400}
\saveTG{攵}{80400}
\saveTG{父}{80400}
\saveTG{𦍡}{80400}
\saveTG{𠓝}{80400}
\saveTG{𣁁}{80400}
\saveTG{午}{80400}
\saveTG{𦔴}{80401}
\saveTG{𦔵}{80401}
\saveTG{𦗊}{80401}
\saveTG{𢆉}{80401}
\saveTG{𦍍}{80401}
\saveTG{𦔯}{80401}
\saveTG{𠇨}{80401}
\saveTG{𤕎}{80401}
\saveTG{𠈾}{80401}
\saveTG{姜}{80404}
\saveTG{𠌂}{80404}
\saveTG{𦎕}{80404}
\saveTG{𡛑}{80404}
\saveTG{𠌓}{80406}
\saveTG{䓥}{80406}
\saveTG{𠔟}{80407}
\saveTG{𪦻}{80407}
\saveTG{𠓠}{80407}
\saveTG{𠓥}{80407}
\saveTG{𠇀}{80407}
\saveTG{𠬠}{80407}
\saveTG{𢻯}{80407}
\saveTG{复}{80407}
\saveTG{孳}{80407}
\saveTG{夔}{80407}
\saveTG{屰}{80407}
\saveTG{𤕗}{80407}
\saveTG{𡕸}{80407}
\saveTG{𠬰}{80407}
\saveTG{𤕘}{80407}
\saveTG{𠭦}{80407}
\saveTG{𢻶}{80407}
\saveTG{𠦍}{80408}
\saveTG{𠦎}{80408}
\saveTG{𠍘}{80408}
\saveTG{傘}{80408}
\saveTG{伞}{80409}
\saveTG{𡴌}{80409}
\saveTG{𠓧}{80409}
\saveTG{龛}{80414}
\saveTG{𨾝}{80415}
\saveTG{𩀂}{80415}
\saveTG{𨾟}{80415}
\saveTG{𪝶}{80417}
\saveTG{氨}{80417}
\saveTG{𣱗}{80417}
\saveTG{𣱘}{80417}
\saveTG{気}{80417}
\saveTG{𠢬}{80427}
\saveTG{𩫐}{80427}
\saveTG{爺}{80427}
\saveTG{𡦈}{80433}
\saveTG{𢌯}{80440}
\saveTG{𢆏}{80440}
\saveTG{𣁊}{80440}
\saveTG{𤕑}{80441}
\saveTG{并}{80441}
\saveTG{𫒈}{80441}
\saveTG{𠇋}{80441}
\saveTG{弅}{80442}
\saveTG{𢆣}{80442}
\saveTG{𩚇}{80443}
\saveTG{弇}{80446}
\saveTG{𢍜}{80446}
\saveTG{𢍋}{80446}
\saveTG{𠓰}{80447}
\saveTG{𤕖}{80447}
\saveTG{𨥖}{80447}
\saveTG{𦍰}{80448}
\saveTG{𠏸}{80449}
\saveTG{𫅘}{80464}
\saveTG{年}{80500}
\saveTG{𤙻}{80501}
\saveTG{𦍐}{80501}
\saveTG{𫅓}{80501}
\saveTG{}{80501}
\saveTG{羊}{80501}
\saveTG{𠓢}{80502}
\saveTG{𢪘}{80502}
\saveTG{𤘝}{80502}
\saveTG{拿}{80502}
\saveTG{𨌅}{80506}
\saveTG{𠇚}{80506}
\saveTG{䡨}{80506}
\saveTG{𤰒}{80506}
\saveTG{𠇭}{80506}
\saveTG{𨐃}{80506}
\saveTG{单}{80506}
\saveTG{𠓽}{80506}
\saveTG{𠔈}{80506}
\saveTG{𠎃}{80508}
\saveTG{𠔎}{80508}
\saveTG{羌}{80512}
\saveTG{羗}{80513}
\saveTG{𦏆}{80515}
\saveTG{羶}{80516}
\saveTG{氧}{80517}
\saveTG{氠}{80517}
\saveTG{氟}{80517}
\saveTG{㲷}{80517}
\saveTG{𣱮}{80517}
\saveTG{𦏱}{80517}
\saveTG{㲴}{80517}
\saveTG{𦎓}{80517}
\saveTG{羛}{80527}
\saveTG{弚}{80527}
\saveTG{𣫺}{80527}
\saveTG{𦍫}{80531}
\saveTG{𦏨}{80532}
\saveTG{羻}{80547}
\saveTG{𦏐}{80551}
\saveTG{羴}{80551}
\saveTG{搻}{80552}
\saveTG{𤛈}{80552}
\saveTG{義}{80553}
\saveTG{𡈤}{80600}
\saveTG{𡇤}{80600}
\saveTG{𠮦}{80600}
\saveTG{𥃦}{80600}
\saveTG{㕣}{80600}
\saveTG{𠔋}{80600}
\saveTG{𪞌}{80600}
\saveTG{𤰗}{80600}
\saveTG{𥃮}{80600}
\saveTG{㒲}{80600}
\saveTG{𡇷}{80600}
\saveTG{𠓦}{80600}
\saveTG{𠟤}{80600}
\saveTG{𦎍}{80601}
\saveTG{合}{80601}
\saveTG{普}{80601}
\saveTG{善}{80601}
\saveTG{舎}{80601}
\saveTG{𣅑}{80601}
\saveTG{𨧕}{80601}
\saveTG{𦏯}{80601}
\saveTG{𠓤}{80601}
\saveTG{兽}{80601}
\saveTG{䇾}{80601}
\saveTG{㒶}{80602}
\saveTG{𠓮}{80602}
\saveTG{𪉵}{80602}
\saveTG{𤽽}{80602}
\saveTG{𪉤}{80602}
\saveTG{𪉬}{80602}
\saveTG{𥄘}{80602}
\saveTG{𥄟}{80602}
\saveTG{含}{80602}
\saveTG{首}{80602}
\saveTG{𠯨}{80602}
\saveTG{𥅍}{80603}
\saveTG{𠏇}{80603}
\saveTG{𠈂}{80603}
\saveTG{𠲒}{80603}
\saveTG{𤲸}{80603}
\saveTG{酋}{80604}
\saveTG{𤱇}{80604}
\saveTG{𨠭}{80604}
\saveTG{㸙}{80604}
\saveTG{酓}{80604}
\saveTG{𨡣}{80604}
\saveTG{𤱭}{80604}
\saveTG{舍}{80604}
\saveTG{𨢚}{80604}
\saveTG{𠰥}{80604}
\saveTG{𠵊}{80604}
\saveTG{𠔍}{80605}
\saveTG{𠲘}{80605}
\saveTG{着}{80605}
\saveTG{𠋑}{80606}
\saveTG{𣌭}{80606}
\saveTG{㑹}{80606}
\saveTG{曾}{80606}
\saveTG{畣}{80606}
\saveTG{曽}{80606}
\saveTG{會}{80606}
\saveTG{𥅽}{80606}
\saveTG{𠰞}{80606}
\saveTG{𠔌}{80608}
\saveTG{谷}{80608}
\saveTG{𧮫}{80608}
\saveTG{𤳋}{80608}
\saveTG{𠔪}{80609}
\saveTG{𤲞}{80609}
\saveTG{畬}{80609}
\saveTG{畲}{80609}
\saveTG{𧪰}{80611}
\saveTG{兺}{80612}
\saveTG{𫇕}{80614}
\saveTG{𩠳}{80615}
\saveTG{䨄}{80615}
\saveTG{𩁆}{80615}
\saveTG{𨿜}{80615}
\saveTG{䧻}{80615}
\saveTG{䧾}{80615}
\saveTG{𨿹}{80615}
\saveTG{氜}{80617}
\saveTG{𣱣}{80617}
\saveTG{𣱜}{80617}
\saveTG{𣱟}{80617}
\saveTG{𣱩}{80617}
\saveTG{𧯇}{80617}
\saveTG{𣱫}{80617}
\saveTG{㖋}{80617}
\saveTG{氤}{80617}
\saveTG{𫒐}{80617}
\saveTG{氥}{80617}
\saveTG{㲶}{80617}
\saveTG{𤕒}{80621}
\saveTG{𫗻}{80627}
\saveTG{命}{80627}
\saveTG{𠐗}{80632}
\saveTG{𧏨}{80635}
\saveTG{𦏟}{80643}
\saveTG{𡄰}{80648}
\saveTG{𨢅}{80648}
\saveTG{𠏎}{80652}
\saveTG{譱}{80661}
\saveTG{韽}{80661}
\saveTG{𧮟}{80661}
\saveTG{𠑰}{80664}
\saveTG{𩠚}{80682}
\saveTG{铲}{80701}
\saveTG{𡘔}{80704}
\saveTG{钅}{80704}
\saveTG{𠅇}{80710}
\saveTG{仺}{80711}
\saveTG{䭩}{80711}
\saveTG{𠔄}{80711}
\saveTG{㐌}{80712}
\saveTG{仑}{80712}
\saveTG{𩚁}{80712}
\saveTG{𠤒}{80712}
\saveTG{𠓿}{80712}
\saveTG{𠊊}{80712}
\saveTG{𪓵}{80712}
\saveTG{𠆹}{80712}
\saveTG{毓}{80712}
\saveTG{锍}{80712}
\saveTG{𠇜}{80712}
\saveTG{镜}{80712}
\saveTG{仓}{80712}
\saveTG{铳}{80712}
\saveTG{𦈢}{80714}
\saveTG{飳}{80714}
\saveTG{𫄻}{80715}
\saveTG{锥}{80715}
\saveTG{𩁥}{80715}
\saveTG{𩟀}{80715}
\saveTG{𩜑}{80715}
\saveTG{䭚}{80715}
\saveTG{𫔑}{80716}
\saveTG{饘}{80716}
\saveTG{𣱥}{80717}
\saveTG{㐈}{80717}
\saveTG{氙}{80717}
\saveTG{瓮}{80717}
\saveTG{乞}{80717}
\saveTG{𠉲}{80717}
\saveTG{钪}{80717}
\saveTG{瓫}{80717}
\saveTG{甆}{80717}
\saveTG{𪵥}{80717}
\saveTG{𠓟}{80717}
\saveTG{爸}{80717}
\saveTG{𤕕}{80717}
\saveTG{𢀻}{80717}
\saveTG{𨝬}{80717}
\saveTG{𣯠}{80717}
\saveTG{䭗}{80717}
\saveTG{𡄬}{80717}
\saveTG{𣮣}{80718}
\saveTG{䍆}{80721}
\saveTG{𥫮}{80722}
\saveTG{䭣}{80723}
\saveTG{铈}{80727}
\saveTG{镌}{80727}
\saveTG{镐}{80727}
\saveTG{钫}{80727}
\saveTG{镝}{80727}
\saveTG{𩞋}{80727}
\saveTG{𪺜}{80727}
\saveTG{𠋒}{80727}
\saveTG{𦉙}{80727}
\saveTG{𩝾}{80727}
\saveTG{𢎦}{80727}
\saveTG{镛}{80727}
\saveTG{𩝿}{80727}
\saveTG{𩚠}{80727}
\saveTG{𫓾}{80727}
\saveTG{镑}{80727}
\saveTG{𩝝}{80727}
\saveTG{𥫧}{80728}
\saveTG{飰}{80730}
\saveTG{𠫟}{80731}
\saveTG{𠆭}{80731}
\saveTG{𠧦}{80731}
\saveTG{𩚀}{80731}
\saveTG{𩟠}{80731}
\saveTG{𠫥}{80731}
\saveTG{镳}{80731}
\saveTG{铱}{80732}
\saveTG{兿}{80732}
\saveTG{侌}{80732}
\saveTG{衾}{80732}
\saveTG{𧜚}{80732}
\saveTG{𧛯}{80732}
\saveTG{𧙳}{80732}
\saveTG{𩚜}{80732}
\saveTG{𩟮}{80732}
\saveTG{锿}{80732}
\saveTG{兹}{80732}
\saveTG{会}{80732}
\saveTG{公}{80732}
\saveTG{镶}{80732}
\saveTG{食}{80732}
\saveTG{餏}{80732}
\saveTG{饟}{80732}
\saveTG{铉}{80732}
\saveTG{養}{80732}
\saveTG{䭖}{80734}
\saveTG{镱}{80736}
\saveTG{飠}{80737}
\saveTG{镰}{80737}
\saveTG{䭠}{80737}
\saveTG{锌}{80741}
\saveTG{镀}{80747}
\saveTG{}{80747}
\saveTG{餃}{80748}
\saveTG{𩜘}{80748}
\saveTG{䍊}{80748}
\saveTG{铰}{80748}
\saveTG{毎}{80757}
\saveTG{每}{80757}
\saveTG{锫}{80761}
\saveTG{䍌}{80761}
\saveTG{餢}{80761}
\saveTG{饝}{80762}
\saveTG{𩝲}{80763}
\saveTG{餹}{80765}
\saveTG{𡶠}{80771}
\saveTG{㒴}{80772}
\saveTG{𠚗}{80772}
\saveTG{缶}{80772}
\saveTG{仚}{80772}
\saveTG{嵞}{80772}
\saveTG{峹}{80772}
\saveTG{齹}{80772}
\saveTG{岔}{80772}
\saveTG{𡽚}{80772}
\saveTG{𤕏}{80772}
\saveTG{𪗖}{80772}
\saveTG{𠋔}{80772}
\saveTG{屳}{80772}
\saveTG{𪞈}{80774}
\saveTG{𪺛}{80774}
\saveTG{𦥭}{80777}
\saveTG{𦥴}{80777}
\saveTG{𦦓}{80777}
\saveTG{𦥽}{80777}
\saveTG{𠓴}{80777}
\saveTG{𠌆}{80777}
\saveTG{𠔒}{80777}
\saveTG{𠍗}{80777}
\saveTG{𦈲}{80782}
\saveTG{䬵}{80782}
\saveTG{䭧}{80793}
\saveTG{𨱉}{80796}
\saveTG{𪵔}{80796}
\saveTG{𠔁}{80800}
\saveTG{仌}{80800}
\saveTG{𠓫}{80801}
\saveTG{𠔤}{80801}
\saveTG{𦍶}{80801}
\saveTG{兾}{80801}
\saveTG{𩛛}{80801}
\saveTG{𠍺}{80802}
\saveTG{𧾺}{80802}
\saveTG{𨂝}{80802}
\saveTG{𧿍}{80802}
\saveTG{𠈮}{80802}
\saveTG{贫}{80802}
\saveTG{贪}{80802}
\saveTG{𨂮}{80802}
\saveTG{𨅐}{80802}
\saveTG{𦎅}{80803}
\saveTG{𡚈}{80804}
\saveTG{𠇄}{80804}
\saveTG{𪥗}{80804}
\saveTG{𡗯}{80804}
\saveTG{𡙡}{80804}
\saveTG{𡙛}{80804}
\saveTG{𡗳}{80804}
\saveTG{𠓷}{80804}
\saveTG{𠓵}{80804}
\saveTG{𥐄}{80804}
\saveTG{𡙫}{80804}
\saveTG{羹}{80804}
\saveTG{关}{80804}
\saveTG{美}{80804}
\saveTG{矢}{80804}
\saveTG{奠}{80804}
\saveTG{𠓱}{80804}
\saveTG{𠌉}{80805}
\saveTG{𠔯}{80805}
\saveTG{𣍄}{80805}
\saveTG{㑒}{80805}
\saveTG{羮}{80805}
\saveTG{𤴘}{80805}
\saveTG{𠈏}{80806}
\saveTG{𤕙}{80806}
\saveTG{𦎙}{80806}
\saveTG{𧸋}{80806}
\saveTG{貧}{80806}
\saveTG{貪}{80806}
\saveTG{䝷}{80806}
\saveTG{羑}{80807}
\saveTG{𤆅}{80809}
\saveTG{㷢}{80809}
\saveTG{𤋎}{80809}
\saveTG{𦎟}{80809}
\saveTG{𤇇}{80809}
\saveTG{𤎯}{80809}
\saveTG{𤈣}{80809}
\saveTG{炃}{80809}
\saveTG{羙}{80809}
\saveTG{雉}{80815}
\saveTG{𣱛}{80817}
\saveTG{氮}{80817}
\saveTG{氦}{80817}
\saveTG{𧷭}{80821}
\saveTG{𢘈}{80827}
\saveTG{𦢻}{80827}
\saveTG{𥐌}{80827}
\saveTG{𪹨}{80827}
\saveTG{𥏹}{80827}
\saveTG{𥎸}{80831}
\saveTG{𥐐}{80832}
\saveTG{𥏡}{80844}
\saveTG{𠧆}{80848}
\saveTG{𥏮}{80861}
\saveTG{众}{80880}
\saveTG{僉}{80886}
\saveTG{𠊌}{80898}
\saveTG{尒}{80900}
\saveTG{尓}{80900}
\saveTG{𡭗}{80900}
\saveTG{𠈒}{80901}
\saveTG{佘}{80901}
\saveTG{氽}{80902}
\saveTG{羕}{80902}
\saveTG{淾}{80902}
\saveTG{汆}{80902}
\saveTG{𥾈}{80903}
\saveTG{𣏰}{80904}
\saveTG{𠓸}{80904}
\saveTG{𣜫}{80904}
\saveTG{㮍}{80904}
\saveTG{𪜱}{80904}
\saveTG{𣒫}{80904}
\saveTG{𠓲}{80904}
\saveTG{𣔕}{80904}
\saveTG{𣛙}{80904}
\saveTG{𣑠}{80904}
\saveTG{籴}{80904}
\saveTG{枀}{80904}
\saveTG{余}{80904}
\saveTG{𠎳}{80904}
\saveTG{𦍎}{80905}
\saveTG{𦓫}{80905}
\saveTG{𪜬}{80905}
\saveTG{𠑱}{80906}
\saveTG{䨀}{80915}
\saveTG{𠐸}{80915}
\saveTG{雓}{80915}
\saveTG{𣱝}{80917}
\saveTG{𢳖}{80917}
\saveTG{𣱨}{80917}
\saveTG{𣱭}{80917}
\saveTG{𥠠}{80917}
\saveTG{𥠌}{80917}
\saveTG{𢘔}{80917}
\saveTG{氭}{80917}
\saveTG{氯}{80917}
\saveTG{氣}{80917}
\saveTG{樖}{80921}
\saveTG{䅻}{80927}
\saveTG{𥿅}{80931}
\saveTG{𣕙}{80931}
\saveTG{𣁏}{80940}
\saveTG{𠓼}{80941}
\saveTG{𠔢}{80947}
\saveTG{𪯰}{80947}
\saveTG{𥢂}{80994}
\saveTG{𣞣}{80994}
\saveTG{𥣠}{80994}
\saveTG{𠐙}{80994}
\saveTG{𣜈}{80994}
\saveTG{橆}{80994}
\saveTG{𠓭}{80994}
\saveTG{𣡰}{80995}
\saveTG{𨾁}{80999}
\saveTG{𥄰}{81010}
\saveTG{𨥑}{81100}
\saveTG{𥂲}{81102}
\saveTG{𪾗}{81102}
\saveTG{𥂾}{81102}
\saveTG{𨦑}{81111}
\saveTG{𨪹}{81111}
\saveTG{鑨}{81111}
\saveTG{鉦}{81111}
\saveTG{䤵}{81111}
\saveTG{𨦖}{81112}
\saveTG{𨮮}{81112}
\saveTG{銸}{81112}
\saveTG{𨮫}{81112}
\saveTG{𨮷}{81112}
\saveTG{𨨨}{81112}
\saveTG{䤠}{81112}
\saveTG{𫒆}{81112}
\saveTG{鉔}{81112}
\saveTG{鈨}{81112}
\saveTG{錏}{81112}
\saveTG{𫓋}{81112}
\saveTG{鈪}{81112}
\saveTG{鑪}{81112}
\saveTG{鋞}{81112}
\saveTG{釭}{81112}
\saveTG{䥶}{81112}
\saveTG{𨥛}{81112}
\saveTG{𨨙}{81113}
\saveTG{鈺}{81113}
\saveTG{䤷}{81114}
\saveTG{𨰡}{81114}
\saveTG{𨫄}{81114}
\saveTG{銍}{81114}
\saveTG{𨭲}{81115}
\saveTG{𨯟}{81115}
\saveTG{鑩}{81116}
\saveTG{鏂}{81116}
\saveTG{鉕}{81116}
\saveTG{𨰣}{81117}
\saveTG{𨦸}{81117}
\saveTG{𣱤}{81117}
\saveTG{𨩽}{81117}
\saveTG{𨥯}{81117}
\saveTG{𨥵}{81117}
\saveTG{𤮬}{81117}
\saveTG{錿}{81117}
\saveTG{鉅}{81117}
\saveTG{㽂}{81117}
\saveTG{鋀}{81118}
\saveTG{鑎}{81118}
\saveTG{𫓝}{81119}
\saveTG{鉟}{81119}
\saveTG{錒}{81120}
\saveTG{釘}{81120}
\saveTG{鈳}{81120}
\saveTG{𨭶}{81121}
\saveTG{𨮣}{81121}
\saveTG{𨮭}{81121}
\saveTG{鎶}{81121}
\saveTG{䤮}{81122}
\saveTG{𨰳}{81126}
\saveTG{𨥿}{81127}
\saveTG{𨭀}{81127}
\saveTG{𨥚}{81127}
\saveTG{𨨄}{81127}
\saveTG{䥮}{81127}
\saveTG{𫒧}{81127}
\saveTG{𨬆}{81127}
\saveTG{𥐃}{81127}
\saveTG{𥃋}{81127}
\saveTG{𨨍}{81127}
\saveTG{䥎}{81127}
\saveTG{𨯅}{81127}
\saveTG{𨫴}{81127}
\saveTG{𨬩}{81127}
\saveTG{鈵}{81127}
\saveTG{鈣}{81127}
\saveTG{鎘}{81127}
\saveTG{釫}{81127}
\saveTG{錹}{81127}
\saveTG{鐪}{81127}
\saveTG{鎷}{81127}
\saveTG{鑈}{81127}
\saveTG{鑐}{81127}
\saveTG{𨦄}{81127}
\saveTG{鉲}{81131}
\saveTG{𨥕}{81131}
\saveTG{𨭍}{81131}
\saveTG{鐚}{81131}
\saveTG{鐻}{81132}
\saveTG{鋹}{81132}
\saveTG{𨧧}{81132}
\saveTG{𨯂}{81132}
\saveTG{鋠}{81132}
\saveTG{鑢}{81136}
\saveTG{𧐉}{81136}
\saveTG{𨭫}{81137}
\saveTG{𨧠}{81140}
\saveTG{釾}{81140}
\saveTG{銒}{81140}
\saveTG{鈃}{81140}
\saveTG{釪}{81140}
\saveTG{釬}{81140}
\saveTG{鉺}{81140}
\saveTG{𨪝}{81140}
\saveTG{鑷}{81141}
\saveTG{𨩜}{81141}
\saveTG{𨯊}{81141}
\saveTG{𫓍}{81142}
\saveTG{鎒}{81143}
\saveTG{𨮡}{81143}
\saveTG{𨪁}{81144}
\saveTG{𨨵}{81144}
\saveTG{𨬀}{81145}
\saveTG{𨩫}{81145}
\saveTG{鐔}{81146}
\saveTG{鋽}{81146}
\saveTG{𨬷}{81147}
\saveTG{𨩘}{81147}
\saveTG{𨩿}{81147}
\saveTG{㪧}{81147}
\saveTG{㪂}{81147}
\saveTG{𨰮}{81147}
\saveTG{錽}{81147}
\saveTG{鋄}{81147}
\saveTG{𨧴}{81147}
\saveTG{𨬜}{81147}
\saveTG{𨪬}{81147}
\saveTG{𨦐}{81147}
\saveTG{䥳}{81147}
\saveTG{鈙}{81147}
\saveTG{𫓗}{81148}
\saveTG{銔}{81149}
\saveTG{鏬}{81149}
\saveTG{𨥾}{81149}
\saveTG{𫒲}{81150}
\saveTG{𨬲}{81151}
\saveTG{𦉋}{81151}
\saveTG{鎼}{81152}
\saveTG{𨭴}{81152}
\saveTG{𨬳}{81152}
\saveTG{𨪨}{81152}
\saveTG{鐬}{81153}
\saveTG{𨩞}{81157}
\saveTG{鉆}{81160}
\saveTG{鏀}{81160}
\saveTG{鐕}{81161}
\saveTG{𨫌}{81161}
\saveTG{䥄}{81161}
\saveTG{鐳}{81161}
\saveTG{𨫶}{81161}
\saveTG{鋙}{81161}
\saveTG{𨯻}{81161}
\saveTG{鐂}{81162}
\saveTG{𫗖}{81162}
\saveTG{銆}{81162}
\saveTG{鉐}{81162}
\saveTG{鍢}{81166}
\saveTG{䥧}{81168}
\saveTG{𨬴}{81168}
\saveTG{𨧆}{81169}
\saveTG{𨦰}{81170}
\saveTG{鑡}{81172}
\saveTG{𨩝}{81174}
\saveTG{𨮒}{81176}
\saveTG{𨯼}{81181}
\saveTG{鐝}{81182}
\saveTG{𨮗}{81182}
\saveTG{𨩼}{81182}
\saveTG{𫖪}{81182}
\saveTG{𫒖}{81184}
\saveTG{𨨰}{81184}
\saveTG{𪥙}{81184}
\saveTG{䫡}{81186}
\saveTG{𨫫}{81186}
\saveTG{𨯸}{81186}
\saveTG{𨯲}{81186}
\saveTG{𨬗}{81186}
\saveTG{𨮌}{81186}
\saveTG{𨫋}{81186}
\saveTG{𨬰}{81186}
\saveTG{𨰥}{81186}
\saveTG{顉}{81186}
\saveTG{鍞}{81186}
\saveTG{𩔌}{81186}
\saveTG{𩔚}{81186}
\saveTG{𩔱}{81186}
\saveTG{𨭂}{81188}
\saveTG{𨪘}{81189}
\saveTG{𨪳}{81189}
\saveTG{𨰖}{81189}
\saveTG{鈈}{81190}
\saveTG{鏢}{81191}
\saveTG{𨮶}{81191}
\saveTG{𨥮}{81191}
\saveTG{銾}{81192}
\saveTG{銢}{81192}
\saveTG{𨫊}{81194}
\saveTG{䥔}{81194}
\saveTG{𨬡}{81194}
\saveTG{𨪛}{81196}
\saveTG{𨮦}{81199}
\saveTG{尩}{81211}
\saveTG{𠔠}{81212}
\saveTG{䝇}{81212}
\saveTG{𫗎}{81217}
\saveTG{𤬰}{81217}
\saveTG{𤭌}{81217}
\saveTG{𪛔}{81217}
\saveTG{𪛌}{81217}
\saveTG{甉}{81217}
\saveTG{𩒬}{81218}
\saveTG{𪛍}{81221}
\saveTG{㼶}{81227}
\saveTG{𥐒}{81227}
\saveTG{𠁄}{81230}
\saveTG{𢼎}{81247}
\saveTG{㪠}{81247}
\saveTG{𦧎}{81264}
\saveTG{𧈁}{81281}
\saveTG{颁}{81282}
\saveTG{𥸤}{81286}
\saveTG{𩔀}{81286}
\saveTG{龥}{81286}
\saveTG{𩑟}{81286}
\saveTG{𩓂}{81286}
\saveTG{䫄}{81286}
\saveTG{頒}{81286}
\saveTG{𩒉}{81286}
\saveTG{瓴}{81317}
\saveTG{𤮐}{81317}
\saveTG{甒}{81317}
\saveTG{𧆺}{81317}
\saveTG{𠄖}{81320}
\saveTG{𫂨}{81366}
\saveTG{领}{81382}
\saveTG{領}{81386}
\saveTG{䫞}{81386}
\saveTG{𡟿}{81412}
\saveTG{瓶}{81417}
\saveTG{𪪃}{81427}
\saveTG{𠑋}{81444}
\saveTG{𣀔}{81447}
\saveTG{𢼩}{81447}
\saveTG{啎}{81461}
\saveTG{頩}{81485}
\saveTG{𩑿}{81486}
\saveTG{頩}{81486}
\saveTG{𩒭}{81486}
\saveTG{𩒕}{81486}
\saveTG{𩕍}{81486}
\saveTG{𩑤}{81486}
\saveTG{𢹁}{81502}
\saveTG{䍽}{81512}
\saveTG{羥}{81512}
\saveTG{𦍞}{81512}
\saveTG{𦏰}{81512}
\saveTG{𦎢}{81514}
\saveTG{𦎣}{81514}
\saveTG{𦍼}{81516}
\saveTG{𦏀}{81517}
\saveTG{𦍭}{81517}
\saveTG{𦏊}{81517}
\saveTG{𦍘}{81517}
\saveTG{𦎄}{81517}
\saveTG{𦏤}{81521}
\saveTG{羺}{81527}
\saveTG{𦏌}{81527}
\saveTG{𦍠}{81527}
\saveTG{𦎂}{81527}
\saveTG{𦍝}{81531}
\saveTG{𦎜}{81532}
\saveTG{𦎇}{81532}
\saveTG{𢼝}{81547}
\saveTG{䍬}{81549}
\saveTG{𢆫}{81552}
\saveTG{䍼}{81561}
\saveTG{𦏪}{81561}
\saveTG{𦏅}{81582}
\saveTG{𦏥}{81584}
\saveTG{𦏈}{81594}
\saveTG{羱}{81596}
\saveTG{𣉻}{81604}
\saveTG{豅}{81611}
\saveTG{䜫}{81612}
\saveTG{𠼞}{81612}
\saveTG{𧯙}{81612}
\saveTG{𩠛}{81612}
\saveTG{㠮}{81612}
\saveTG{𩠯}{81616}
\saveTG{𣍎}{81617}
\saveTG{𧮽}{81617}
\saveTG{𩞞}{81617}
\saveTG{𤭏}{81617}
\saveTG{㼨}{81617}
\saveTG{甑}{81617}
\saveTG{𣍋}{81630}
\saveTG{𥰊}{81630}
\saveTG{谺}{81640}
\saveTG{㪨}{81647}
\saveTG{𣀖}{81647}
\saveTG{敆}{81647}
\saveTG{𢼽}{81647}
\saveTG{𧇵}{81652}
\saveTG{𧯖}{81668}
\saveTG{𧯕}{81668}
\saveTG{颔}{81682}
\saveTG{颌}{81682}
\saveTG{𧯗}{81684}
\saveTG{頜}{81686}
\saveTG{頷}{81686}
\saveTG{𩔕}{81686}
\saveTG{䭭}{81686}
\saveTG{𩓱}{81686}
\saveTG{𩕊}{81686}
\saveTG{𠇌}{81703}
\saveTG{}{81707}
\saveTG{𩰤}{81711}
\saveTG{钲}{81711}
\saveTG{𫓬}{81712}
\saveTG{𩚢}{81712}
\saveTG{𩜄}{81712}
\saveTG{𩛉}{81712}
\saveTG{缸}{81712}
\saveTG{罏}{81712}
\saveTG{铔}{81712}
\saveTG{𦈮}{81712}
\saveTG{𩚫}{81712}
\saveTG{𦈵}{81712}
\saveTG{𩟽}{81712}
\saveTG{𣬁}{81712}
\saveTG{𨱀}{81712}
\saveTG{}{81712}
\saveTG{钰}{81713}
\saveTG{𫓯}{81714}
\saveTG{𩚽}{81714}
\saveTG{䬹}{81714}
\saveTG{铚}{81714}
\saveTG{𫔓}{81715}
\saveTG{𩟯}{81715}
\saveTG{𦉒}{81716}
\saveTG{饇}{81716}
\saveTG{钷}{81716}
\saveTG{𧆦}{81717}
\saveTG{𩚌}{81717}
\saveTG{𩞊}{81717}
\saveTG{钜}{81717}
\saveTG{𩚋}{81717}
\saveTG{䥻}{81717}
\saveTG{𧇍}{81717}
\saveTG{𩜚}{81717}
\saveTG{䬧}{81717}
\saveTG{𩝨}{81717}
\saveTG{𤭬}{81717}
\saveTG{𤮫}{81717}
\saveTG{𦈪}{81717}
\saveTG{𦉐}{81717}
\saveTG{𪓳}{81717}
\saveTG{𩚚}{81717}
\saveTG{䵹}{81717}
\saveTG{𤭐}{81717}
\saveTG{𩟭}{81717}
\saveTG{缻}{81717}
\saveTG{餖}{81718}
\saveTG{𣱪}{81719}
\saveTG{𩚼}{81719}
\saveTG{钉}{81720}
\saveTG{锕}{81720}
\saveTG{钶}{81720}
\saveTG{飣}{81720}
\saveTG{}{81721}
\saveTG{餰}{81721}
\saveTG{𦈥}{81721}
\saveTG{𦉛}{81721}
\saveTG{钙}{81727}
\saveTG{镉}{81727}
\saveTG{𩛄}{81727}
\saveTG{𫓺}{81727}
\saveTG{𩞪}{81727}
\saveTG{𦉏}{81727}
\saveTG{𫗜}{81727}
\saveTG{𦈣}{81727}
\saveTG{𩚞}{81731}
\saveTG{𨙌}{81731}
\saveTG{}{81731}
\saveTG{𫓵}{81732}
\saveTG{餦}{81732}
\saveTG{𩟃}{81732}
\saveTG{𩞈}{81733}
\saveTG{}{81740}
\saveTG{飦}{81740}
\saveTG{䥺}{81740}
\saveTG{䍂}{81740}
\saveTG{铒}{81740}
\saveTG{餌}{81740}
\saveTG{钘}{81740}
\saveTG{𦉄}{81741}
\saveTG{𩟓}{81742}
\saveTG{𦈨}{81744}
\saveTG{𩜝}{81744}
\saveTG{𩝼}{81744}
\saveTG{镡}{81746}
\saveTG{罈}{81746}
\saveTG{𩟴}{81747}
\saveTG{镊}{81747}
\saveTG{𫓸}{81747}
\saveTG{䍈}{81749}
\saveTG{罅}{81749}
\saveTG{𦈾}{81752}
\saveTG{𦉌}{81752}
\saveTG{饖}{81753}
\saveTG{钻}{81760}
\saveTG{䭙}{81761}
\saveTG{𦉢}{81761}
\saveTG{铻}{81761}
\saveTG{镭}{81761}
\saveTG{䬯}{81762}
\saveTG{𩟬}{81762}
\saveTG{𫜈}{81762}
\saveTG{䍄}{81762}
\saveTG{𩜰}{81766}
\saveTG{𩛷}{81769}
\saveTG{𦈶}{81769}
\saveTG{𦉟}{81772}
\saveTG{𩝟}{81777}
\saveTG{镢}{81782}
\saveTG{颂}{81782}
\saveTG{𫗔}{81784}
\saveTG{餪}{81784}
\saveTG{𩛫}{81784}
\saveTG{𩓫}{81786}
\saveTG{𩔝}{81786}
\saveTG{𩑔}{81786}
\saveTG{𩑡}{81786}
\saveTG{𩒋}{81786}
\saveTG{頌}{81786}
\saveTG{𫔕}{81786}
\saveTG{𩒡}{81786}
\saveTG{𦈧}{81790}
\saveTG{䬪}{81790}
\saveTG{钚}{81790}
\saveTG{𩚪}{81791}
\saveTG{镖}{81791}
\saveTG{}{81794}
\saveTG{𩞸}{81794}
\saveTG{𧸶}{81806}
\saveTG{𪿊}{81812}
\saveTG{𨦍}{81812}
\saveTG{𨧇}{81812}
\saveTG{𥏝}{81812}
\saveTG{𥎹}{81814}
\saveTG{𥐓}{81815}
\saveTG{𨧄}{81815}
\saveTG{矩}{81817}
\saveTG{㽀}{81817}
\saveTG{𥐍}{81818}
\saveTG{短}{81818}
\saveTG{𥎵}{81826}
\saveTG{𥐎}{81827}
\saveTG{𥏴}{81827}
\saveTG{𥏁}{81827}
\saveTG{𥏾}{81827}
\saveTG{𥏼}{81827}
\saveTG{𥏦}{81827}
\saveTG{𧋚}{81831}
\saveTG{𥎿}{81840}
\saveTG{𦗼}{81842}
\saveTG{𦕌}{81842}
\saveTG{𥏥}{81846}
\saveTG{𪠫}{81847}
\saveTG{𥏽}{81849}
\saveTG{𥏒}{81861}
\saveTG{𥏄}{81862}
\saveTG{𥏪}{81862}
\saveTG{𫂪}{81877}
\saveTG{𥏂}{81884}
\saveTG{顩}{81886}
\saveTG{榘}{81904}
\saveTG{𣛊}{81904}
\saveTG{𣱯}{81919}
\saveTG{敍}{81947}
\saveTG{𢿸}{81947}
\saveTG{𩒀}{81986}
\saveTG{𢯍}{82000}
\saveTG{刏}{82000}
\saveTG{𥄅}{82077}
\saveTG{𨨪}{82100}
\saveTG{䤛}{82100}
\saveTG{𠟊}{82100}
\saveTG{𨧢}{82100}
\saveTG{𨦙}{82100}
\saveTG{𠚭}{82100}
\saveTG{𨦿}{82100}
\saveTG{𨦕}{82100}
\saveTG{𨪑}{82100}
\saveTG{𨦀}{82100}
\saveTG{𪩢}{82100}
\saveTG{𠚻}{82100}
\saveTG{鉶}{82100}
\saveTG{剑}{82100}
\saveTG{銂}{82100}
\saveTG{釗}{82100}
\saveTG{𨧘}{82100}
\saveTG{𠛲}{82100}
\saveTG{𠞊}{82100}
\saveTG{㔐}{82100}
\saveTG{𠛮}{82100}
\saveTG{䥷}{82100}
\saveTG{剉}{82100}
\saveTG{𨭚}{82100}
\saveTG{鍘}{82100}
\saveTG{鈏}{82100}
\saveTG{𨮸}{82100}
\saveTG{釧}{82100}
\saveTG{鋓}{82100}
\saveTG{𠠃}{82100}
\saveTG{𠜋}{82100}
\saveTG{𪾘}{82102}
\saveTG{𡐂}{82104}
\saveTG{鈚}{82110}
\saveTG{釓}{82110}
\saveTG{錷}{82110}
\saveTG{鉳}{82110}
\saveTG{𨧥}{82111}
\saveTG{𨧑}{82112}
\saveTG{𫒒}{82112}
\saveTG{鑞}{82112}
\saveTG{銚}{82113}
\saveTG{𨭇}{82114}
\saveTG{𨯐}{82114}
\saveTG{𨧓}{82114}
\saveTG{銋}{82114}
\saveTG{鈓}{82114}
\saveTG{𨮑}{82115}
\saveTG{錘}{82115}
\saveTG{鏙}{82115}
\saveTG{𨰁}{82115}
\saveTG{鍾}{82115}
\saveTG{𨪰}{82115}
\saveTG{𫒇}{82117}
\saveTG{𨤽}{82117}
\saveTG{𣮧}{82117}
\saveTG{𣰅}{82117}
\saveTG{鋵}{82117}
\saveTG{𨨜}{82117}
\saveTG{𨦂}{82117}
\saveTG{𨥼}{82117}
\saveTG{𠃽}{82117}
\saveTG{𨬌}{82117}
\saveTG{𨪉}{82117}
\saveTG{𨧰}{82117}
\saveTG{𨮴}{82117}
\saveTG{鎧}{82118}
\saveTG{𨰘}{82118}
\saveTG{鐙}{82118}
\saveTG{𨰼}{82119}
\saveTG{䤺}{82121}
\saveTG{𨮕}{82121}
\saveTG{𨭠}{82121}
\saveTG{𨰉}{82121}
\saveTG{𨦬}{82121}
\saveTG{𨭩}{82121}
\saveTG{𨬅}{82121}
\saveTG{䤱}{82121}
\saveTG{鏩}{82121}
\saveTG{釿}{82121}
\saveTG{鐁}{82121}
\saveTG{釤}{82122}
\saveTG{䤯}{82122}
\saveTG{𨧗}{82122}
\saveTG{𢒐}{82122}
\saveTG{𨭌}{82122}
\saveTG{鍴}{82127}
\saveTG{鐈}{82127}
\saveTG{鑴}{82127}
\saveTG{銹}{82127}
\saveTG{鐤}{82127}
\saveTG{銟}{82127}
\saveTG{鏰}{82127}
\saveTG{𨰽}{82127}
\saveTG{𨪮}{82127}
\saveTG{𨬚}{82127}
\saveTG{𨨔}{82127}
\saveTG{𫒳}{82127}
\saveTG{𨰛}{82127}
\saveTG{𨦝}{82127}
\saveTG{䤫}{82127}
\saveTG{𨩈}{82127}
\saveTG{䥴}{82127}
\saveTG{𨭽}{82127}
\saveTG{𫒎}{82130}
\saveTG{釽}{82130}
\saveTG{鈲}{82130}
\saveTG{鑂}{82131}
\saveTG{䥀}{82131}
\saveTG{𨰙}{82131}
\saveTG{𨮳}{82131}
\saveTG{𨩪}{82131}
\saveTG{𨥺}{82131}
\saveTG{𨧌}{82132}
\saveTG{鎃}{82132}
\saveTG{銩}{82132}
\saveTG{䤨}{82132}
\saveTG{𧔢}{82136}
\saveTG{𨮐}{82136}
\saveTG{𨥝}{82137}
\saveTG{鏭}{82139}
\saveTG{釺}{82140}
\saveTG{鋋}{82141}
\saveTG{錌}{82141}
\saveTG{鋌}{82141}
\saveTG{𨪙}{82143}
\saveTG{𨪻}{82144}
\saveTG{𡡚}{82144}
\saveTG{錗}{82144}
\saveTG{鋖}{82144}
\saveTG{𨨻}{82147}
\saveTG{𨫁}{82147}
\saveTG{𨦏}{82147}
\saveTG{𨨒}{82147}
\saveTG{鍰}{82147}
\saveTG{鑁}{82147}
\saveTG{𨭅}{82147}
\saveTG{鍐}{82147}
\saveTG{鏺}{82147}
\saveTG{鑀}{82147}
\saveTG{鋢}{82147}
\saveTG{鈑}{82147}
\saveTG{𨰺}{82148}
\saveTG{鋝}{82149}
\saveTG{䤣}{82149}
\saveTG{𨥋}{82150}
\saveTG{鐖}{82153}
\saveTG{鎽}{82154}
\saveTG{銗}{82161}
\saveTG{鍇}{82162}
\saveTG{錙}{82163}
\saveTG{鍿}{82163}
\saveTG{銽}{82164}
\saveTG{錉}{82164}
\saveTG{鍎}{82164}
\saveTG{銛}{82164}
\saveTG{𨩟}{82164}
\saveTG{𨨱}{82164}
\saveTG{𨩢}{82165}
\saveTG{𨧐}{82168}
\saveTG{鐇}{82169}
\saveTG{錔}{82169}
\saveTG{𨥉}{82170}
\saveTG{𨥍}{82170}
\saveTG{𨥥}{82170}
\saveTG{𫓈}{82170}
\saveTG{𨪸}{82172}
\saveTG{鈯}{82172}
\saveTG{鎐}{82172}
\saveTG{𨬭}{82172}
\saveTG{𨪫}{82177}
\saveTG{鍤}{82177}
\saveTG{𨭡}{82177}
\saveTG{䤾}{82177}
\saveTG{鎭}{82181}
\saveTG{鋲}{82181}
\saveTG{𨥜}{82184}
\saveTG{𨪾}{82184}
\saveTG{𨨷}{82184}
\saveTG{鍨}{82184}
\saveTG{鏷}{82185}
\saveTG{𨮓}{82185}
\saveTG{鎻}{82186}
\saveTG{鑕}{82186}
\saveTG{𨪼}{82188}
\saveTG{𫓁}{82191}
\saveTG{銯}{82193}
\saveTG{𫒠}{82193}
\saveTG{鉌}{82194}
\saveTG{𨨫}{82194}
\saveTG{鏫}{82194}
\saveTG{鏁}{82194}
\saveTG{鑠}{82194}
\saveTG{𨧺}{82197}
\saveTG{㓱}{82200}
\saveTG{𠟃}{82200}
\saveTG{𠜑}{82200}
\saveTG{𠝮}{82200}
\saveTG{𠞳}{82200}
\saveTG{𠛢}{82200}
\saveTG{剃}{82200}
\saveTG{創}{82200}
\saveTG{𠚼}{82200}
\saveTG{𠟖}{82200}
\saveTG{毤}{82211}
\saveTG{𡯲}{82211}
\saveTG{𡰡}{82212}
\saveTG{𢒘}{82212}
\saveTG{𡯑}{82213}
\saveTG{毹}{82214}
\saveTG{𣬩}{82217}
\saveTG{𣮆}{82217}
\saveTG{𣯹}{82217}
\saveTG{𧈛}{82217}
\saveTG{䶵}{82217}
\saveTG{𪛊}{82221}
\saveTG{𣂮}{82221}
\saveTG{㶗}{82232}
\saveTG{𤫫}{82233}
\saveTG{𤬓}{82233}
\saveTG{𩻢}{82236}
\saveTG{𪟐}{82257}
\saveTG{睂}{82261}
\saveTG{𥇮}{82261}
\saveTG{龤}{82262}
\saveTG{𣢏}{82282}
\saveTG{龢}{82294}
\saveTG{𥤖}{82294}
\saveTG{刢}{82300}
\saveTG{𠠆}{82300}
\saveTG{𠟚}{82300}
\saveTG{𩙳}{82313}
\saveTG{𣯇}{82317}
\saveTG{𣬹}{82317}
\saveTG{𤫲}{82333}
\saveTG{懖}{82336}
\saveTG{𠝢}{82400}
\saveTG{㔍}{82400}
\saveTG{𠛼}{82400}
\saveTG{𠛶}{82400}
\saveTG{𠞼}{82400}
\saveTG{𦘉}{82401}
\saveTG{𦗦}{82401}
\saveTG{𦗾}{82401}
\saveTG{𦘌}{82402}
\saveTG{𣰚}{82417}
\saveTG{𪟄}{82420}
\saveTG{𣃍}{82421}
\saveTG{𢒑}{82422}
\saveTG{𦍯}{82431}
\saveTG{𦍖}{82437}
\saveTG{𥣃}{82494}
\saveTG{𠠀}{82500}
\saveTG{𦏫}{82517}
\saveTG{𦎞}{82517}
\saveTG{𦎵}{82517}
\saveTG{䍮}{82517}
\saveTG{羏}{82522}
\saveTG{𦎋}{82522}
\saveTG{𦍳}{82533}
\saveTG{羝}{82540}
\saveTG{䍴}{82544}
\saveTG{𦏦}{82548}
\saveTG{𦏑}{82553}
\saveTG{𦎥}{82562}
\saveTG{𦏩}{82569}
\saveTG{羳}{82569}
\saveTG{𦍦}{82572}
\saveTG{羰}{82589}
\saveTG{㓣}{82600}
\saveTG{劄}{82600}
\saveTG{劊}{82600}
\saveTG{𠟏}{82600}
\saveTG{𠟂}{82600}
\saveTG{㓧}{82600}
\saveTG{䜲}{82616}
\saveTG{𣯽}{82617}
\saveTG{𩠔}{82617}
\saveTG{𨡳}{82617}
\saveTG{𣃀}{82621}
\saveTG{㣒}{82622}
\saveTG{𧮮}{82640}
\saveTG{𠎈}{82647}
\saveTG{𫗶}{82649}
\saveTG{㧱}{82650}
\saveTG{𧮶}{82661}
\saveTG{𧯅}{82669}
\saveTG{豀}{82684}
\saveTG{𩕻}{82686}
\saveTG{𦅚}{82693}
\saveTG{钏}{82700}
\saveTG{创}{82700}
\saveTG{𫗛}{82700}
\saveTG{刽}{82700}
\saveTG{刉}{82700}
\saveTG{铏}{82700}
\saveTG{铡}{82700}
\saveTG{钊}{82700}
\saveTG{㓡}{82700}
\saveTG{𠠑}{82700}
\saveTG{𠜮}{82700}
\saveTG{𠛀}{82700}
\saveTG{𠛛}{82700}
\saveTG{}{82710}
\saveTG{钆}{82710}
\saveTG{镴}{82712}
\saveTG{𩚨}{82712}
\saveTG{铫}{82713}
\saveTG{餆}{82713}
\saveTG{罀}{82713}
\saveTG{餁}{82714}
\saveTG{飥}{82714}
\saveTG{𦉖}{82714}
\saveTG{飪}{82714}
\saveTG{锤}{82715}
\saveTG{锺}{82715}
\saveTG{𦉎}{82715}
\saveTG{𩞅}{82715}
\saveTG{𩜀}{82715}
\saveTG{𦉂}{82715}
\saveTG{𫓳}{82717}
\saveTG{𩚂}{82717}
\saveTG{𩚬}{82717}
\saveTG{𩛪}{82717}
\saveTG{飥}{82717}
\saveTG{𣰴}{82717}
\saveTG{𩝹}{82717}
\saveTG{𩜆}{82717}
\saveTG{铠}{82717}
\saveTG{䭓}{82718}
\saveTG{𩞬}{82718}
\saveTG{镫}{82718}
\saveTG{}{82721}
\saveTG{𫗲}{82721}
\saveTG{𩞏}{82721}
\saveTG{钐}{82722}
\saveTG{𩚎}{82722}
\saveTG{𦈤}{82727}
\saveTG{镚}{82727}
\saveTG{锈}{82727}
\saveTG{𩠊}{82727}
\saveTG{䥿}{82727}
\saveTG{𩜵}{82727}
\saveTG{䭨}{82727}
\saveTG{𩟥}{82727}
\saveTG{𫔔}{82727}
\saveTG{𫓱}{82728}
\saveTG{𩟑}{82728}
\saveTG{𩝚}{82731}
\saveTG{𫔒}{82731}
\saveTG{𧖦}{82731}
\saveTG{铥}{82732}
\saveTG{𨱃}{82733}
\saveTG{𩛃}{82733}
\saveTG{}{82734}
\saveTG{𩜡}{82734}
\saveTG{䍇}{82737}
\saveTG{䭡}{82737}
\saveTG{钎}{82740}
\saveTG{餠}{82741}
\saveTG{铤}{82741}
\saveTG{𩛝}{82743}
\saveTG{䬫}{82743}
\saveTG{𩝂}{82744}
\saveTG{餒}{82744}
\saveTG{𩝺}{82744}
\saveTG{餧}{82744}
\saveTG{锾}{82747}
\saveTG{飯}{82747}
\saveTG{钣}{82747}
\saveTG{𩚿}{82747}
\saveTG{𩛞}{82747}
\saveTG{𦈴}{82747}
\saveTG{𩞽}{82747}
\saveTG{𦈽}{82747}
\saveTG{锊}{82749}
\saveTG{饑}{82753}
\saveTG{缿}{82761}
\saveTG{𩞷}{82762}
\saveTG{锴}{82762}
\saveTG{𩜠}{82762}
\saveTG{锱}{82763}
\saveTG{𩜊}{82763}
\saveTG{𩛶}{82764}
\saveTG{铦}{82764}
\saveTG{餂}{82764}
\saveTG{𩛎}{82767}
\saveTG{𫔍}{82769}
\saveTG{𨱄}{82772}
\saveTG{𩛋}{82772}
\saveTG{飿}{82772}
\saveTG{饀}{82777}
\saveTG{锸}{82777}
\saveTG{锧}{82782}
\saveTG{飫}{82784}
\saveTG{镤}{82785}
\saveTG{𩝗}{82786}
\saveTG{𩞉}{82791}
\saveTG{}{82792}
\saveTG{𩜖}{82792}
\saveTG{䋣}{82793}
\saveTG{緐}{82793}
\saveTG{𩝰}{82794}
\saveTG{𩜿}{82794}
\saveTG{𩜓}{82794}
\saveTG{铄}{82794}
\saveTG{𩞛}{82794}
\saveTG{𩟧}{82794}
\saveTG{𠝫}{82800}
\saveTG{𠠬}{82800}
\saveTG{矧}{82800}
\saveTG{𠟬}{82800}
\saveTG{劍}{82800}
\saveTG{剣}{82800}
\saveTG{𪟕}{82800}
\saveTG{㕗}{82803}
\saveTG{𤏼}{82809}
\saveTG{䂑}{82817}
\saveTG{𥏐}{82817}
\saveTG{𥎬}{82817}
\saveTG{𥎺}{82817}
\saveTG{𥏸}{82818}
\saveTG{𣃇}{82821}
\saveTG{䞈}{82827}
\saveTG{𥏗}{82827}
\saveTG{矯}{82827}
\saveTG{矫}{82828}
\saveTG{𣲝}{82830}
\saveTG{𤫴}{82833}
\saveTG{𥏩}{82833}
\saveTG{𥏎}{82841}
\saveTG{矮}{82844}
\saveTG{𥎮}{82847}
\saveTG{𢪽}{82850}
\saveTG{䂐}{82872}
\saveTG{𥎽}{82884}
\saveTG{矨}{82884}
\saveTG{𥣂}{82894}
\saveTG{𥚘}{82901}
\saveTG{䄻}{82917}
\saveTG{𤫿}{82933}
\saveTG{𠔭}{83022}
\saveTG{㦰}{83050}
\saveTG{𢦜}{83058}
\saveTG{鈊}{83100}
\saveTG{釙}{83100}
\saveTG{𡎢}{83100}
\saveTG{𥃒}{83100}
\saveTG{𫒓}{83100}
\saveTG{𥃅}{83102}
\saveTG{𨭮}{83102}
\saveTG{𥃗}{83102}
\saveTG{𡓛}{83104}
\saveTG{鉍}{83104}
\saveTG{鈩}{83107}
\saveTG{鈗}{83112}
\saveTG{鋎}{83112}
\saveTG{鉈}{83112}
\saveTG{𨭏}{83112}
\saveTG{鋺}{83112}
\saveTG{𨨀}{83112}
\saveTG{鑧}{83113}
\saveTG{𫒭}{83114}
\saveTG{𨫐}{83114}
\saveTG{鍹}{83116}
\saveTG{䤩}{83117}
\saveTG{䥉}{83117}
\saveTG{𨫕}{83117}
\saveTG{𨫯}{83117}
\saveTG{𫒯}{83117}
\saveTG{䤞}{83117}
\saveTG{𨥡}{83117}
\saveTG{鑹}{83117}
\saveTG{𨰧}{83117}
\saveTG{𨭙}{83117}
\saveTG{鑏}{83121}
\saveTG{鏒}{83122}
\saveTG{䥱}{83127}
\saveTG{鋪}{83127}
\saveTG{鍽}{83127}
\saveTG{𨫂}{83127}
\saveTG{𨰬}{83127}
\saveTG{䥇}{83127}
\saveTG{}{83127}
\saveTG{𨩮}{83127}
\saveTG{𨥭}{83132}
\saveTG{𨰆}{83132}
\saveTG{鋃}{83132}
\saveTG{鎵}{83132}
\saveTG{鋐}{83132}
\saveTG{𨨕}{83132}
\saveTG{𨬃}{83133}
\saveTG{𨩔}{83134}
\saveTG{𨯶}{83136}
\saveTG{𨥧}{83137}
\saveTG{𨮄}{83137}
\saveTG{𨰔}{83138}
\saveTG{𨫨}{83139}
\saveTG{鋱}{83140}
\saveTG{𨥶}{83140}
\saveTG{釴}{83140}
\saveTG{錻}{83140}
\saveTG{鉽}{83140}
\saveTG{𨫃}{83141}
\saveTG{𨦦}{83141}
\saveTG{鉽}{83141}
\saveTG{𨩴}{83142}
\saveTG{鎛}{83142}
\saveTG{𫒙}{83142}
\saveTG{銨}{83144}
\saveTG{𨨿}{83147}
\saveTG{鈸}{83147}
\saveTG{鋑}{83147}
\saveTG{𫒛}{83147}
\saveTG{䤹}{83147}
\saveTG{𨩵}{83147}
\saveTG{鋨}{83150}
\saveTG{鐵}{83150}
\saveTG{鐡}{83150}
\saveTG{鋮}{83150}
\saveTG{𢦳}{83150}
\saveTG{䤦}{83150}
\saveTG{龯}{83150}
\saveTG{銊}{83150}
\saveTG{鈛}{83150}
\saveTG{𨯱}{83150}
\saveTG{𫒞}{83150}
\saveTG{𢦺}{83150}
\saveTG{鉞}{83150}
\saveTG{韱}{83150}
\saveTG{鏚}{83150}
\saveTG{鍼}{83150}
\saveTG{鉾}{83150}
\saveTG{鑯}{83150}
\saveTG{銭}{83150}
\saveTG{𨨅}{83151}
\saveTG{䥫}{83151}
\saveTG{𨭱}{83151}
\saveTG{𨯒}{83151}
\saveTG{𨫓}{83151}
\saveTG{𨰕}{83151}
\saveTG{𨮯}{83151}
\saveTG{𨬿}{83151}
\saveTG{𨪠}{83152}
\saveTG{𨦭}{83152}
\saveTG{𨭟}{83153}
\saveTG{𫓏}{83153}
\saveTG{錢}{83153}
\saveTG{𨩆}{83154}
\saveTG{𨭓}{83156}
\saveTG{𠁝}{83156}
\saveTG{𨪴}{83160}
\saveTG{鈶}{83160}
\saveTG{𨯀}{83161}
\saveTG{鑙}{83161}
\saveTG{鏥}{83162}
\saveTG{𨧩}{83162}
\saveTG{𫒴}{83164}
\saveTG{鎋}{83165}
\saveTG{𨭋}{83165}
\saveTG{𨬬}{83166}
\saveTG{鎔}{83168}
\saveTG{𨨐}{83172}
\saveTG{錧}{83177}
\saveTG{鑳}{83181}
\saveTG{錠}{83181}
\saveTG{𫓃}{83181}
\saveTG{𨭹}{83184}
\saveTG{𨦛}{83184}
\saveTG{𨧚}{83184}
\saveTG{錑}{83184}
\saveTG{钀}{83184}
\saveTG{𨭢}{83184}
\saveTG{𨰰}{83186}
\saveTG{鏔}{83186}
\saveTG{鑌}{83186}
\saveTG{𨰦}{83186}
\saveTG{𨬄}{83189}
\saveTG{錝}{83191}
\saveTG{鑔}{83191}
\saveTG{鉥}{83194}
\saveTG{𨯈}{83196}
\saveTG{銶}{83199}
\saveTG{𠟐}{83200}
\saveTG{𡯠}{83218}
\saveTG{𢧥}{83250}
\saveTG{𢨍}{83250}
\saveTG{戧}{83250}
\saveTG{𢨠}{83250}
\saveTG{𠔺}{83256}
\saveTG{𨮺}{83261}
\saveTG{𠏓}{83261}
\saveTG{𧷟}{83286}
\saveTG{𪈇}{83327}
\saveTG{𠨉}{83402}
\saveTG{𠬌}{83442}
\saveTG{羦}{83512}
\saveTG{羫}{83512}
\saveTG{䍫}{83517}
\saveTG{羜}{83521}
\saveTG{𦎎}{83527}
\saveTG{羪}{83532}
\saveTG{䍸}{83543}
\saveTG{𦍺}{83547}
\saveTG{羧}{83547}
\saveTG{𦍱}{83547}
\saveTG{𢧐}{83550}
\saveTG{羬}{83550}
\saveTG{羢}{83550}
\saveTG{𦎘}{83551}
\saveTG{𢧚}{83552}
\saveTG{𦎗}{83552}
\saveTG{𦎩}{83561}
\saveTG{𦎱}{83565}
\saveTG{𥗛}{83602}
\saveTG{𩉕}{83602}
\saveTG{谾}{83612}
\saveTG{𨠫}{83617}
\saveTG{㞃}{83617}
\saveTG{舗}{83627}
\saveTG{𩠤}{83627}
\saveTG{舖}{83627}
\saveTG{𢎒}{83640}
\saveTG{𩠕}{83644}
\saveTG{馘}{83650}
\saveTG{𢨉}{83650}
\saveTG{𢨈}{83650}
\saveTG{䤋}{83651}
\saveTG{䜮}{83652}
\saveTG{𧮺}{83653}
\saveTG{𧯃}{83656}
\saveTG{𪽚}{83658}
\saveTG{𨣡}{83662}
\saveTG{𧯆}{83665}
\saveTG{舘}{83677}
\saveTG{𤡿}{83684}
\saveTG{猷}{83684}
\saveTG{𩠵}{83694}
\saveTG{钋}{83700}
\saveTG{𩚅}{83700}
\saveTG{𠧩}{83700}
\saveTG{飶}{83704}
\saveTG{铋}{83704}
\saveTG{铊}{83712}
\saveTG{𩟏}{83712}
\saveTG{𩞜}{83712}
\saveTG{𩝑}{83716}
\saveTG{𩜌}{83717}
\saveTG{𣫼}{83717}
\saveTG{𩝜}{83717}
\saveTG{𦈭}{83717}
\saveTG{䭢}{83721}
\saveTG{𩜽}{83722}
\saveTG{𩞀}{83722}
\saveTG{}{83727}
\saveTG{䦂}{83727}
\saveTG{𩝯}{83727}
\saveTG{铺}{83727}
\saveTG{餔}{83727}
\saveTG{}{83732}
\saveTG{锒}{83732}
\saveTG{𩛡}{83732}
\saveTG{镓}{83732}
\saveTG{𩞼}{83738}
\saveTG{铽}{83740}
\saveTG{}{83740}
\saveTG{𩝪}{83741}
\saveTG{餺}{83742}
\saveTG{镈}{83742}
\saveTG{𦉊}{83743}
\saveTG{𫗒}{83744}
\saveTG{𩛖}{83744}
\saveTG{铵}{83744}
\saveTG{𩜯}{83747}
\saveTG{䥽}{83747}
\saveTG{𩝘}{83747}
\saveTG{𩟒}{83747}
\saveTG{}{83747}
\saveTG{餕}{83747}
\saveTG{钹}{83747}
\saveTG{}{83750}
\saveTG{𩚧}{83750}
\saveTG{䬻}{83750}
\saveTG{𩟦}{83750}
\saveTG{𫓨}{83750}
\saveTG{𨱆}{83750}
\saveTG{铖}{83750}
\saveTG{餓}{83750}
\saveTG{锇}{83750}
\saveTG{戗}{83750}
\saveTG{钺}{83750}
\saveTG{钱}{83750}
\saveTG{𦈸}{83751}
\saveTG{𩟶}{83751}
\saveTG{𫓰}{83751}
\saveTG{𦈻}{83753}
\saveTG{餞}{83753}
\saveTG{𩞿}{83756}
\saveTG{𩞡}{83756}
\saveTG{𩝈}{83756}
\saveTG{镩}{83756}
\saveTG{飴}{83760}
\saveTG{𫔊}{83762}
\saveTG{𩝛}{83765}
\saveTG{𣫾}{83768}
\saveTG{镕}{83768}
\saveTG{𩝒}{83772}
\saveTG{館}{83777}
\saveTG{镔}{83781}
\saveTG{锭}{83781}
\saveTG{𩜦}{83782}
\saveTG{镲}{83791}
\saveTG{𩟔}{83791}
\saveTG{𫓽}{83791}
\saveTG{}{83794}
\saveTG{𩟩}{83796}
\saveTG{𨱇}{83799}
\saveTG{𡰙}{83817}
\saveTG{𥏕}{83831}
\saveTG{𥎱}{83844}
\saveTG{𢨔}{83850}
\saveTG{𤡽}{83884}
\saveTG{𥏖}{83884}
\saveTG{𣚱}{83904}
\saveTG{𢧅}{83950}
\saveTG{𪫦}{84012}
\saveTG{𦓯}{84014}
\saveTG{鈄}{84100}
\saveTG{鉜}{84100}
\saveTG{釮}{84100}
\saveTG{針}{84100}
\saveTG{鑆}{84100}
\saveTG{𨥇}{84103}
\saveTG{𨮬}{84103}
\saveTG{𨥪}{84103}
\saveTG{𨮊}{84103}
\saveTG{𨩥}{84103}
\saveTG{𡓷}{84104}
\saveTG{𡓉}{84104}
\saveTG{釷}{84110}
\saveTG{𨦁}{84110}
\saveTG{𨧀}{84110}
\saveTG{𨮇}{84110}
\saveTG{鈋}{84110}
\saveTG{鋩}{84110}
\saveTG{𨯕}{84111}
\saveTG{鈂}{84112}
\saveTG{鎑}{84112}
\saveTG{鑉}{84112}
\saveTG{銠}{84112}
\saveTG{鐃}{84112}
\saveTG{釶}{84112}
\saveTG{銑}{84112}
\saveTG{𨩯}{84112}
\saveTG{鋴}{84112}
\saveTG{𨫿}{84112}
\saveTG{𨰗}{84112}
\saveTG{𫒦}{84112}
\saveTG{鍷}{84114}
\saveTG{銈}{84114}
\saveTG{錵}{84114}
\saveTG{䥓}{84114}
\saveTG{𨩒}{84114}
\saveTG{𨫲}{84114}
\saveTG{錴}{84114}
\saveTG{𨬶}{84115}
\saveTG{䥃}{84115}
\saveTG{鑵}{84115}
\saveTG{𨰑}{84115}
\saveTG{𨩨}{84116}
\saveTG{𨥃}{84117}
\saveTG{𨨉}{84117}
\saveTG{䤶}{84117}
\saveTG{𨧣}{84117}
\saveTG{𨥐}{84117}
\saveTG{鉪}{84117}
\saveTG{釚}{84117}
\saveTG{䥵}{84117}
\saveTG{𨨇}{84117}
\saveTG{𨦾}{84117}
\saveTG{𫒸}{84117}
\saveTG{鍖}{84118}
\saveTG{𨯷}{84118}
\saveTG{𨦵}{84118}
\saveTG{錡}{84121}
\saveTG{釛}{84127}
\saveTG{鈉}{84127}
\saveTG{勜}{84127}
\saveTG{銪}{84127}
\saveTG{𨦌}{84127}
\saveTG{𨧈}{84127}
\saveTG{䤭}{84127}
\saveTG{𨮩}{84127}
\saveTG{𨬍}{84127}
\saveTG{𫒿}{84127}
\saveTG{𨧨}{84127}
\saveTG{𨦘}{84127}
\saveTG{𨩇}{84127}
\saveTG{銬}{84127}
\saveTG{𨩦}{84127}
\saveTG{錺}{84127}
\saveTG{銙}{84127}
\saveTG{鈽}{84127}
\saveTG{鋤}{84127}
\saveTG{钄}{84127}
\saveTG{𫓀}{84127}
\saveTG{𨨆}{84127}
\saveTG{𨯫}{84127}
\saveTG{鏋}{84127}
\saveTG{𠢄}{84127}
\saveTG{𦟎}{84127}
\saveTG{𫒾}{84127}
\saveTG{㔕}{84127}
\saveTG{䤻}{84127}
\saveTG{𨦲}{84127}
\saveTG{𨰂}{84127}
\saveTG{𨭬}{84127}
\saveTG{𨭯}{84127}
\saveTG{𨬁}{84130}
\saveTG{鈦}{84130}
\saveTG{𨯛}{84131}
\saveTG{䤲}{84131}
\saveTG{𨨤}{84131}
\saveTG{鋕}{84131}
\saveTG{𨯧}{84131}
\saveTG{𨮵}{84132}
\saveTG{鉣}{84132}
\saveTG{𨯝}{84132}
\saveTG{䥦}{84132}
\saveTG{𨩖}{84132}
\saveTG{𫓑}{84132}
\saveTG{鍅}{84132}
\saveTG{鈜}{84132}
\saveTG{鎱}{84132}
\saveTG{鐽}{84135}
\saveTG{鑝}{84135}
\saveTG{𨯌}{84135}
\saveTG{𨯨}{84136}
\saveTG{𨨈}{84137}
\saveTG{錰}{84139}
\saveTG{𨩺}{84140}
\saveTG{𫔌}{84140}
\saveTG{銰}{84140}
\saveTG{釹}{84140}
\saveTG{﨨}{84141}
\saveTG{𨬕}{84141}
\saveTG{𨬟}{84141}
\saveTG{鑄}{84141}
\saveTG{𨫽}{84141}
\saveTG{𨨹}{84142}
\saveTG{鑮}{84142}
\saveTG{𨮾}{84143}
\saveTG{𨫉}{84143}
\saveTG{𨨲}{84143}
\saveTG{䥬}{84143}
\saveTG{䥈}{84144}
\saveTG{錛}{84144}
\saveTG{𨩑}{84144}
\saveTG{鑊}{84147}
\saveTG{𨧾}{84147}
\saveTG{銌}{84147}
\saveTG{鈘}{84147}
\saveTG{鈹}{84147}
\saveTG{錂}{84147}
\saveTG{鋍}{84147}
\saveTG{𨫳}{84147}
\saveTG{𨫭}{84147}
\saveTG{𥀞}{84147}
\saveTG{㿷}{84147}
\saveTG{𨨏}{84147}
\saveTG{𨪇}{84147}
\saveTG{𫔐}{84147}
\saveTG{𨭸}{84147}
\saveTG{﨧}{84147}
\saveTG{𫓤}{84151}
\saveTG{鑻}{84152}
\saveTG{鑖}{84153}
\saveTG{鑶}{84153}
\saveTG{𨫱}{84154}
\saveTG{鏵}{84154}
\saveTG{䥠}{84156}
\saveTG{鍏}{84156}
\saveTG{𫒰}{84157}
\saveTG{𨥰}{84160}
\saveTG{錨}{84160}
\saveTG{鈷}{84160}
\saveTG{鍺}{84160}
\saveTG{𫓞}{84160}
\saveTG{鐑}{84161}
\saveTG{銡}{84161}
\saveTG{鋯}{84161}
\saveTG{𨯠}{84161}
\saveTG{𨪌}{84161}
\saveTG{𨭎}{84161}
\saveTG{𨪲}{84161}
\saveTG{錯}{84161}
\saveTG{鎝}{84161}
\saveTG{𨮚}{84161}
\saveTG{鍣}{84162}
\saveTG{𨦳}{84162}
\saveTG{鍩}{84164}
\saveTG{𨮎}{84164}
\saveTG{鐯}{84164}
\saveTG{𨭈}{84165}
\saveTG{𨮤}{84167}
\saveTG{𨯩}{84167}
\saveTG{𨪓}{84167}
\saveTG{𨮥}{84168}
\saveTG{𨧃}{84169}
\saveTG{鉗}{84170}
\saveTG{𨧅}{84177}
\saveTG{釱}{84180}
\saveTG{𪥞}{84180}
\saveTG{鎮}{84181}
\saveTG{錤}{84181}
\saveTG{鉷}{84181}
\saveTG{𨩅}{84181}
\saveTG{𨭣}{84182}
\saveTG{鏯}{84184}
\saveTG{鏌}{84184}
\saveTG{𨫪}{84185}
\saveTG{𨮟}{84185}
\saveTG{鍈}{84185}
\saveTG{𫓛}{84186}
\saveTG{鐼}{84186}
\saveTG{鐄}{84186}
\saveTG{鑽}{84186}
\saveTG{鑟}{84186}
\saveTG{𨧼}{84187}
\saveTG{鋏}{84188}
\saveTG{𨦗}{84189}
\saveTG{𨭼}{84189}
\saveTG{銝}{84190}
\saveTG{鈢}{84190}
\saveTG{錼}{84191}
\saveTG{𨭺}{84191}
\saveTG{鎍}{84193}
\saveTG{𨦅}{84194}
\saveTG{䥡}{84194}
\saveTG{𨪪}{84194}
\saveTG{𨫟}{84194}
\saveTG{𨬘}{84194}
\saveTG{𨪀}{84194}
\saveTG{𨪩}{84194}
\saveTG{鍱}{84194}
\saveTG{鐷}{84194}
\saveTG{𨰤}{84196}
\saveTG{𨫜}{84196}
\saveTG{鐐}{84196}
\saveTG{錸}{84198}
\saveTG{𫓄}{84203}
\saveTG{𣁺}{84203}
\saveTG{𡯛}{84211}
\saveTG{𡯸}{84211}
\saveTG{𡝸}{84214}
\saveTG{𢼠}{84214}
\saveTG{𡜎}{84214}
\saveTG{𢻐}{84214}
\saveTG{𤿑}{84214}
\saveTG{𢅸}{84227}
\saveTG{勧}{84227}
\saveTG{厁}{84227}
\saveTG{𤿫}{84247}
\saveTG{𢺹}{84247}
\saveTG{𨨭}{84247}
\saveTG{𫒻}{84252}
\saveTG{𡲞}{84260}
\saveTG{𡱨}{84260}
\saveTG{𧷐}{84286}
\saveTG{𩻇}{84336}
\saveTG{𩼫}{84361}
\saveTG{𩽫}{84361}
\saveTG{㔙}{84427}
\saveTG{𣁵}{84503}
\saveTG{𦎺}{84512}
\saveTG{𦎽}{84517}
\saveTG{𦎉}{84517}
\saveTG{𦍔}{84517}
\saveTG{𦎻}{84517}
\saveTG{劷}{84527}
\saveTG{䍩}{84540}
\saveTG{𦍷}{84547}
\saveTG{𦍩}{84560}
\saveTG{𦏔}{84561}
\saveTG{𦎁}{84561}
\saveTG{𦏢}{84586}
\saveTG{𦎳}{84586}
\saveTG{羵}{84586}
\saveTG{𫅗}{84587}
\saveTG{𣁴}{84603}
\saveTG{䜪}{84617}
\saveTG{𠡼}{84627}
\saveTG{𠢡}{84627}
\saveTG{𠢙}{84627}
\saveTG{谹}{84632}
\saveTG{𡞜}{84640}
\saveTG{𥀟}{84647}
\saveTG{𢻆}{84647}
\saveTG{𧮴}{84660}
\saveTG{𫗸}{84661}
\saveTG{𧮳}{84670}
\saveTG{谼}{84681}
\saveTG{𧯀}{84685}
\saveTG{豄}{84686}
\saveTG{针}{84700}
\saveTG{钭}{84700}
\saveTG{𩟡}{84703}
\saveTG{𩜭}{84703}
\saveTG{𩛜}{84703}
\saveTG{𩚭}{84703}
\saveTG{铓}{84710}
\saveTG{钍}{84710}
\saveTG{}{84710}
\saveTG{𩜱}{84711}
\saveTG{𩟟}{84712}
\saveTG{饁}{84712}
\saveTG{𩞵}{84712}
\saveTG{𩟣}{84712}
\saveTG{𩟤}{84712}
\saveTG{饚}{84712}
\saveTG{𩟺}{84712}
\saveTG{铣}{84712}
\saveTG{饒}{84712}
\saveTG{铑}{84712}
\saveTG{𩚉}{84712}
\saveTG{𩝉}{84712}
\saveTG{𦈰}{84714}
\saveTG{𨪅}{84714}
\saveTG{饉}{84715}
\saveTG{罐}{84715}
\saveTG{𦉆}{84716}
\saveTG{餷}{84716}
\saveTG{餣}{84716}
\saveTG{𩜙}{84717}
\saveTG{𨱂}{84717}
\saveTG{𣫸}{84717}
\saveTG{𦉗}{84717}
\saveTG{𩛲}{84717}
\saveTG{𩛔}{84717}
\saveTG{饐}{84718}
\saveTG{𩛠}{84718}
\saveTG{锜}{84721}
\saveTG{𩝳}{84722}
\saveTG{𩞱}{84724}
\saveTG{𩛵}{84727}
\saveTG{餚}{84727}
\saveTG{𫗕}{84727}
\saveTG{𫗏}{84727}
\saveTG{𩚍}{84727}
\saveTG{䭉}{84727}
\saveTG{𩛙}{84727}
\saveTG{𠠶}{84727}
\saveTG{䍎}{84727}
\saveTG{𩟢}{84727}
\saveTG{𩛭}{84727}
\saveTG{𩟞}{84727}
\saveTG{𩞢}{84727}
\saveTG{铹}{84727}
\saveTG{铐}{84727}
\saveTG{锄}{84727}
\saveTG{餝}{84727}
\saveTG{钸}{84727}
\saveTG{铕}{84727}
\saveTG{餙}{84727}
\saveTG{钠}{84727}
\saveTG{勄}{84727}
\saveTG{钛}{84730}
\saveTG{𩟘}{84731}
\saveTG{𩛣}{84731}
\saveTG{𩝤}{84731}
\saveTG{饛}{84732}
\saveTG{𩟆}{84732}
\saveTG{𩟐}{84735}
\saveTG{链}{84735}
\saveTG{𩜫}{84740}
\saveTG{𩚔}{84740}
\saveTG{餀}{84740}
\saveTG{钕}{84740}
\saveTG{铧}{84741}
\saveTG{𩜜}{84741}
\saveTG{𩝽}{84741}
\saveTG{𩟛}{84743}
\saveTG{䭦}{84743}
\saveTG{𩟕}{84743}
\saveTG{𩛏}{84744}
\saveTG{餴}{84744}
\saveTG{锛}{84744}
\saveTG{𩛌}{84744}
\saveTG{𢻂}{84747}
\saveTG{}{84747}
\saveTG{镬}{84747}
\saveTG{铍}{84747}
\saveTG{𫓦}{84747}
\saveTG{餑}{84747}
\saveTG{𩜁}{84747}
\saveTG{𨱋}{84747}
\saveTG{𩜤}{84747}
\saveTG{𩛑}{84747}
\saveTG{𩚡}{84747}
\saveTG{𩞑}{84748}
\saveTG{𩝃}{84748}
\saveTG{𩞴}{84748}
\saveTG{𩞔}{84748}
\saveTG{𩞳}{84748}
\saveTG{𩟙}{84751}
\saveTG{𫓴}{84752}
\saveTG{钴}{84760}
\saveTG{锚}{84760}
\saveTG{锗}{84760}
\saveTG{𩚩}{84760}
\saveTG{饎}{84761}
\saveTG{错}{84761}
\saveTG{锆}{84761}
\saveTG{𨱏}{84761}
\saveTG{}{84761}
\saveTG{𩝙}{84761}
\saveTG{𩜅}{84761}
\saveTG{𫗓}{84761}
\saveTG{䦃}{84764}
\saveTG{𩜼}{84764}
\saveTG{锘}{84764}
\saveTG{}{84764}
\saveTG{𩚵}{84770}
\saveTG{钳}{84770}
\saveTG{𪙄}{84772}
\saveTG{𩝻}{84781}
\saveTG{镇}{84781}
\saveTG{𩛘}{84781}
\saveTG{𫔁}{84782}
\saveTG{𩟌}{84784}
\saveTG{镆}{84784}
\saveTG{𩜈}{84784}
\saveTG{饃}{84784}
\saveTG{𩞶}{84785}
\saveTG{䭊}{84785}
\saveTG{锳}{84785}
\saveTG{𩟲}{84786}
\saveTG{𩞩}{84786}
\saveTG{𨱑}{84786}
\saveTG{饙}{84786}
\saveTG{饡}{84786}
\saveTG{𩟱}{84786}
\saveTG{𩛩}{84788}
\saveTG{𩜪}{84791}
\saveTG{𫔅}{84793}
\saveTG{铩}{84794}
\saveTG{𦉃}{84794}
\saveTG{䭟}{84794}
\saveTG{𩝀}{84794}
\saveTG{𩝇}{84794}
\saveTG{䭎}{84794}
\saveTG{𫄼}{84794}
\saveTG{䭜}{84796}
\saveTG{镣}{84796}
\saveTG{𥎧}{84800}
\saveTG{䂒}{84801}
\saveTG{𪿎}{84803}
\saveTG{𥏌}{84812}
\saveTG{𪿋}{84814}
\saveTG{𥏜}{84821}
\saveTG{𥏤}{84827}
\saveTG{𥎰}{84831}
\saveTG{𥏣}{84841}
\saveTG{𥏢}{84841}
\saveTG{䂔}{84841}
\saveTG{𥐋}{84844}
\saveTG{矱}{84847}
\saveTG{𧸓}{84864}
\saveTG{𦏒}{84864}
\saveTG{𣜟}{84890}
\saveTG{𪿈}{84890}
\saveTG{𫅠}{84894}
\saveTG{𥏭}{84894}
\saveTG{斜}{84900}
\saveTG{𪣸}{84914}
\saveTG{䋾}{84927}
\saveTG{𥅅}{85072}
\saveTG{𢗿}{85090}
\saveTG{鈝}{85100}
\saveTG{𨪦}{85104}
\saveTG{𨥙}{85105}
\saveTG{鉮}{85106}
\saveTG{鉂}{85106}
\saveTG{鋛}{85106}
\saveTG{𫒗}{85106}
\saveTG{𫒨}{85106}
\saveTG{鈡}{85106}
\saveTG{𨧫}{85107}
\saveTG{銉}{85107}
\saveTG{鉎}{85110}
\saveTG{𤯭}{85112}
\saveTG{𨭰}{85114}
\saveTG{𨯏}{85115}
\saveTG{釻}{85117}
\saveTG{鈍}{85117}
\saveTG{𨦤}{85117}
\saveTG{䤜}{85117}
\saveTG{𨰋}{85118}
\saveTG{𫓐}{85118}
\saveTG{𨭳}{85127}
\saveTG{錆}{85127}
\saveTG{𨥦}{85127}
\saveTG{𨥳}{85127}
\saveTG{𠁔}{85127}
\saveTG{鏽}{85127}
\saveTG{鉘}{85127}
\saveTG{鉢}{85130}
\saveTG{鏈}{85130}
\saveTG{𨰍}{85131}
\saveTG{𫓒}{85132}
\saveTG{錶}{85132}
\saveTG{鏸}{85133}
\saveTG{鉵}{85136}
\saveTG{𨫑}{85136}
\saveTG{𨨩}{85136}
\saveTG{𨫍}{85136}
\saveTG{鑓}{85137}
\saveTG{𨯚}{85137}
\saveTG{𨯯}{85138}
\saveTG{𨫩}{85139}
\saveTG{鍵}{85140}
\saveTG{鋳}{85140}
\saveTG{鏄}{85143}
\saveTG{𨪋}{85144}
\saveTG{鏤}{85144}
\saveTG{𨫡}{85147}
\saveTG{𨦎}{85147}
\saveTG{䤡}{85147}
\saveTG{𨯉}{85156}
\saveTG{𨰀}{85156}
\saveTG{𫒩}{85158}
\saveTG{鈾}{85160}
\saveTG{𨩃}{85160}
\saveTG{𨦈}{85165}
\saveTG{鏪}{85166}
\saveTG{鐟}{85168}
\saveTG{鏏}{85177}
\saveTG{𨦶}{85180}
\saveTG{鉄}{85180}
\saveTG{鈌}{85180}
\saveTG{鈇}{85180}
\saveTG{鉠}{85180}
\saveTG{錪}{85181}
\saveTG{𨦇}{85182}
\saveTG{銕}{85182}
\saveTG{𨨯}{85184}
\saveTG{𨦢}{85186}
\saveTG{䥊}{85186}
\saveTG{鑚}{85186}
\saveTG{鐨}{85186}
\saveTG{鐀}{85186}
\saveTG{銖}{85190}
\saveTG{銇}{85190}
\saveTG{𨨓}{85190}
\saveTG{𫓇}{85192}
\saveTG{𨪥}{85192}
\saveTG{𨦉}{85192}
\saveTG{鍊}{85196}
\saveTG{錬}{85196}
\saveTG{鋉}{85196}
\saveTG{𤘨}{85202}
\saveTG{𨍻}{85206}
\saveTG{𤯹}{85212}
\saveTG{𧉇}{85213}
\saveTG{𤒙}{85231}
\saveTG{𣍗}{85265}
\saveTG{𫗙}{85286}
\saveTG{𣗾}{85292}
\saveTG{𫉓}{85296}
\saveTG{𦎖}{85496}
\saveTG{𦍜}{85502}
\saveTG{䍨}{85502}
\saveTG{𦍓}{85517}
\saveTG{𦍣}{85517}
\saveTG{𧒦}{85531}
\saveTG{䍺}{85536}
\saveTG{䍷}{85556}
\saveTG{𦏋}{85568}
\saveTG{羠}{85582}
\saveTG{𫅐}{85582}
\saveTG{𦎸}{85586}
\saveTG{䍪}{85590}
\saveTG{𦎏}{85596}
\saveTG{䍶}{85596}
\saveTG{𤙖}{85602}
\saveTG{𤙣}{85602}
\saveTG{𤯤}{85612}
\saveTG{𧮭}{85617}
\saveTG{𧮹}{85627}
\saveTG{䜬}{85660}
\saveTG{𣍂}{85665}
\saveTG{䬱}{85703}
\saveTG{}{85706}
\saveTG{钟}{85706}
\saveTG{䭄}{85707}
\saveTG{铙}{85712}
\saveTG{䭍}{85712}
\saveTG{𩟝}{85712}
\saveTG{钝}{85717}
\saveTG{飩}{85717}
\saveTG{𩚊}{85717}
\saveTG{𩜎}{85727}
\saveTG{锖}{85727}
\saveTG{}{85727}
\saveTG{缽}{85730}
\saveTG{钵}{85730}
\saveTG{䦄}{85732}
\saveTG{饢}{85732}
\saveTG{𩟊}{85732}
\saveTG{𩞙}{85736}
\saveTG{蝕}{85736}
\saveTG{䭤}{85737}
\saveTG{𩞍}{85739}
\saveTG{键}{85740}
\saveTG{铸}{85740}
\saveTG{𦈫}{85747}
\saveTG{䭈}{85747}
\saveTG{}{85747}
\saveTG{铀}{85760}
\saveTG{𩞒}{85761}
\saveTG{𩞮}{85761}
\saveTG{𣌶}{85765}
\saveTG{𩞄}{85766}
\saveTG{𩞲}{85768}
\saveTG{}{85777}
\saveTG{𩝡}{85777}
\saveTG{缺}{85780}
\saveTG{铁}{85780}
\saveTG{铗}{85780}
\saveTG{𫓧}{85780}
\saveTG{𫓭}{85782}
\saveTG{镄}{85782}
\saveTG{𩚟}{85782}
\saveTG{䬬}{85782}
\saveTG{𩟪}{85786}
\saveTG{饋}{85786}
\saveTG{䬴}{85790}
\saveTG{铼}{85790}
\saveTG{𣔍}{85790}
\saveTG{铢}{85790}
\saveTG{𩝥}{85793}
\saveTG{𩟰}{85794}
\saveTG{餗}{85796}
\saveTG{𨱈}{85796}
\saveTG{𩜍}{85796}
\saveTG{𫔀}{85796}
\saveTG{𩛾}{85799}
\saveTG{𡚃}{85804}
\saveTG{𦘗}{85807}
\saveTG{𦎡}{85812}
\saveTG{𪲿}{85896}
\saveTG{鉫}{86100}
\saveTG{釦}{86100}
\saveTG{鉬}{86100}
\saveTG{銣}{86100}
\saveTG{銦}{86100}
\saveTG{鈤}{86100}
\saveTG{䤧}{86100}
\saveTG{𨮢}{86100}
\saveTG{鈿}{86100}
\saveTG{𫓼}{86100}
\saveTG{𨫵}{86100}
\saveTG{𨯬}{86100}
\saveTG{𨭦}{86100}
\saveTG{𨨛}{86100}
\saveTG{錮}{86100}
\saveTG{𨦱}{86102}
\saveTG{鉑}{86102}
\saveTG{𡒟}{86104}
\saveTG{鉭}{86110}
\saveTG{𨧝}{86110}
\saveTG{鎠}{86111}
\saveTG{鎤}{86112}
\saveTG{鎾}{86112}
\saveTG{鎞}{86112}
\saveTG{錕}{86112}
\saveTG{鋧}{86112}
\saveTG{鋥}{86114}
\saveTG{𨨌}{86114}
\saveTG{鍠}{86114}
\saveTG{𨪽}{86114}
\saveTG{鑸}{86114}
\saveTG{鋰}{86115}
\saveTG{鑼}{86115}
\saveTG{鍟}{86115}
\saveTG{鑺}{86115}
\saveTG{𨪈}{86117}
\saveTG{䥯}{86117}
\saveTG{𨦆}{86117}
\saveTG{𨧏}{86117}
\saveTG{𨦺}{86117}
\saveTG{𨯦}{86117}
\saveTG{䚊}{86117}
\saveTG{𨩄}{86118}
\saveTG{𨯴}{86121}
\saveTG{𫒹}{86122}
\saveTG{鍝}{86127}
\saveTG{鍚}{86127}
\saveTG{𨫖}{86127}
\saveTG{鎉}{86127}
\saveTG{𨮁}{86127}
\saveTG{𨬱}{86127}
\saveTG{𨪒}{86127}
\saveTG{𨮨}{86127}
\saveTG{𫓟}{86127}
\saveTG{𨦻}{86127}
\saveTG{𨩋}{86127}
\saveTG{䤢}{86127}
\saveTG{𨭜}{86127}
\saveTG{鐋}{86127}
\saveTG{鋗}{86127}
\saveTG{蠲}{86127}
\saveTG{錦}{86127}
\saveTG{鐲}{86127}
\saveTG{鍻}{86127}
\saveTG{鍔}{86127}
\saveTG{銱}{86127}
\saveTG{錫}{86127}
\saveTG{鐊}{86127}
\saveTG{鎅}{86128}
\saveTG{𨪜}{86130}
\saveTG{鏓}{86130}
\saveTG{鍶}{86130}
\saveTG{鎴}{86130}
\saveTG{𧔈}{86131}
\saveTG{𨰟}{86131}
\saveTG{𦉡}{86131}
\saveTG{鍡}{86132}
\saveTG{䤼}{86132}
\saveTG{𨬸}{86132}
\saveTG{鐶}{86132}
\saveTG{䥪}{86133}
\saveTG{𨰄}{86133}
\saveTG{𧌲}{86136}
\saveTG{鏹}{86136}
\saveTG{𨨴}{86137}
\saveTG{錍}{86140}
\saveTG{𨰪}{86141}
\saveTG{鍀}{86141}
\saveTG{鐸}{86141}
\saveTG{銲}{86141}
\saveTG{𨬵}{86141}
\saveTG{鍓}{86141}
\saveTG{𨪶}{86144}
\saveTG{𨰃}{86144}
\saveTG{𨨊}{86144}
\saveTG{𨧉}{86144}
\saveTG{𧖿}{86144}
\saveTG{鎫}{86147}
\saveTG{钁}{86147}
\saveTG{𨭵}{86147}
\saveTG{鏝}{86147}
\saveTG{𨰫}{86148}
\saveTG{鉀}{86150}
\saveTG{𨯮}{86154}
\saveTG{鏎}{86154}
\saveTG{𨭐}{86156}
\saveTG{錩}{86160}
\saveTG{𨩗}{86160}
\saveTG{鑘}{86160}
\saveTG{鋁}{86160}
\saveTG{𨫘}{86172}
\saveTG{鋇}{86180}
\saveTG{鉙}{86180}
\saveTG{鋜}{86181}
\saveTG{鍉}{86181}
\saveTG{𨭄}{86181}
\saveTG{𨬹}{86182}
\saveTG{𨩬}{86184}
\saveTG{鋘}{86184}
\saveTG{𨦚}{86184}
\saveTG{𫒱}{86184}
\saveTG{𦒥}{86184}
\saveTG{𨫰}{86185}
\saveTG{𨨋}{86189}
\saveTG{鏍}{86193}
\saveTG{錁}{86194}
\saveTG{𨩚}{86194}
\saveTG{鐰}{86194}
\saveTG{鎳}{86194}
\saveTG{鐛}{86196}
\saveTG{鑤}{86199}
\saveTG{䤽}{86199}
\saveTG{𡯒}{86210}
\saveTG{𡯧}{86211}
\saveTG{𡯱}{86211}
\saveTG{観}{86212}
\saveTG{覦}{86212}
\saveTG{𡰂}{86213}
\saveTG{𨚇}{86217}
\saveTG{𧠝}{86217}
\saveTG{𧢢}{86217}
\saveTG{𩳬}{86217}
\saveTG{𧠚}{86217}
\saveTG{𡯫}{86218}
\saveTG{䶏}{86221}
\saveTG{𠕧}{86227}
\saveTG{𪛋}{86260}
\saveTG{𡃋}{86286}
\saveTG{𪫃}{86286}
\saveTG{𤒸}{86331}
\saveTG{𢤹}{86336}
\saveTG{聟}{86401}
\saveTG{𥏯}{86406}
\saveTG{𦍾}{86502}
\saveTG{𨋧}{86506}
\saveTG{𦎬}{86515}
\saveTG{𦎐}{86515}
\saveTG{𦎔}{86517}
\saveTG{𨚽}{86517}
\saveTG{𦏕}{86527}
\saveTG{羯}{86527}
\saveTG{𦎪}{86527}
\saveTG{𦎲}{86527}
\saveTG{𦏉}{86531}
\saveTG{𦏖}{86532}
\saveTG{䍰}{86560}
\saveTG{𦎑}{86560}
\saveTG{𦎊}{86582}
\saveTG{䍹}{86584}
\saveTG{𦎀}{86584}
\saveTG{𦏛}{86594}
\saveTG{智}{86600}
\saveTG{𩠲}{86600}
\saveTG{𨢮}{86604}
\saveTG{䣽}{86604}
\saveTG{𨞡}{86617}
\saveTG{𧢐}{86617}
\saveTG{䚐}{86617}
\saveTG{𧯐}{86627}
\saveTG{朇}{86640}
\saveTG{𧯌}{86641}
\saveTG{䜱}{86647}
\saveTG{䜰}{86648}
\saveTG{𧯔}{86656}
\saveTG{𧸑}{86680}
\saveTG{𫗷}{86694}
\saveTG{铷}{86700}
\saveTG{钼}{86700}
\saveTG{锢}{86700}
\saveTG{钿}{86700}
\saveTG{𩟍}{86700}
\saveTG{䭅}{86700}
\saveTG{铟}{86700}
\saveTG{𩚣}{86700}
\saveTG{铂}{86702}
\saveTG{𩛇}{86702}
\saveTG{钽}{86710}
\saveTG{锟}{86712}
\saveTG{餛}{86712}
\saveTG{饂}{86712}
\saveTG{𩴽}{86712}
\saveTG{餽}{86713}
\saveTG{餭}{86714}
\saveTG{锽}{86714}
\saveTG{𩛦}{86714}
\saveTG{锃}{86714}
\saveTG{饠}{86715}
\saveTG{㲛}{86715}
\saveTG{𩟹}{86715}
\saveTG{𩞯}{86715}
\saveTG{锂}{86715}
\saveTG{𩚶}{86717}
\saveTG{𫔇}{86717}
\saveTG{𧠡}{86717}
\saveTG{鼅}{86717}
\saveTG{𩚐}{86717}
\saveTG{䭂}{86717}
\saveTG{𩛨}{86717}
\saveTG{𪖬}{86721}
\saveTG{锡}{86727}
\saveTG{锷}{86727}
\saveTG{锅}{86727}
\saveTG{锦}{86727}
\saveTG{锣}{86727}
\saveTG{铞}{86727}
\saveTG{镯}{86727}
\saveTG{餳}{86727}
\saveTG{餲}{86727}
\saveTG{䬼}{86727}
\saveTG{𦈿}{86727}
\saveTG{𫓶}{86727}
\saveTG{𩟉}{86727}
\saveTG{𩞾}{86727}
\saveTG{𩛿}{86727}
\saveTG{𦉀}{86727}
\saveTG{𦉅}{86727}
\saveTG{𩜟}{86727}
\saveTG{锶}{86730}
\saveTG{䭒}{86730}
\saveTG{罎}{86731}
\saveTG{𫔆}{86731}
\saveTG{}{86731}
\saveTG{𩟻}{86732}
\saveTG{𩟁}{86732}
\saveTG{镮}{86732}
\saveTG{餵}{86732}
\saveTG{镪}{86736}
\saveTG{䭞}{86741}
\saveTG{𩛧}{86741}
\saveTG{锝}{86741}
\saveTG{䦆}{86747}
\saveTG{𩞝}{86747}
\saveTG{}{86747}
\saveTG{饅}{86747}
\saveTG{镘}{86747}
\saveTG{钾}{86750}
\saveTG{𩚲}{86750}
\saveTG{饆}{86754}
\saveTG{锠}{86760}
\saveTG{铝}{86760}
\saveTG{𩞓}{86764}
\saveTG{䍉}{86780}
\saveTG{𩞫}{86781}
\saveTG{𦉁}{86782}
\saveTG{𩝊}{86782}
\saveTG{𫔂}{86782}
\saveTG{𩝠}{86784}
\saveTG{𩞗}{86786}
\saveTG{镙}{86793}
\saveTG{𩛴}{86794}
\saveTG{锞}{86794}
\saveTG{餜}{86794}
\saveTG{镍}{86794}
\saveTG{䭋}{86794}
\saveTG{䭘}{86796}
\saveTG{𩟎}{86796}
\saveTG{知}{86800}
\saveTG{𫅖}{86800}
\saveTG{𥎴}{86800}
\saveTG{𥎭}{86800}
\saveTG{𥎶}{86800}
\saveTG{𦏄}{86810}
\saveTG{矲}{86812}
\saveTG{𩲶}{86817}
\saveTG{䂓}{86817}
\saveTG{𧡐}{86817}
\saveTG{𨞀}{86817}
\saveTG{𥏆}{86820}
\saveTG{𥎷}{86820}
\saveTG{𨢰}{86827}
\saveTG{𥏃}{86827}
\saveTG{𥏫}{86827}
\saveTG{𥏅}{86840}
\saveTG{𥐑}{86844}
\saveTG{𥏓}{86844}
\saveTG{𥏠}{86845}
\saveTG{𧸒}{86880}
\saveTG{𣔇}{86904}
\saveTG{𧢪}{86917}
\saveTG{䅰}{86930}
\saveTG{𥡜}{86993}
\saveTG{䳜}{87027}
\saveTG{鳰}{87027}
\saveTG{塑}{87104}
\saveTG{𪣶}{87104}
\saveTG{𨡡}{87108}
\saveTG{𩗩}{87110}
\saveTG{釔}{87110}
\saveTG{釩}{87110}
\saveTG{釠}{87110}
\saveTG{𨭷}{87110}
\saveTG{𨥩}{87110}
\saveTG{𨦨}{87110}
\saveTG{𫕿}{87111}
\saveTG{𡎬}{87112}
\saveTG{鈕}{87112}
\saveTG{鈮}{87112}
\saveTG{鉏}{87112}
\saveTG{錳}{87112}
\saveTG{鋔}{87112}
\saveTG{鉋}{87112}
\saveTG{𨮃}{87112}
\saveTG{鎺}{87112}
\saveTG{𨨃}{87112}
\saveTG{𧗍}{87112}
\saveTG{𫒏}{87112}
\saveTG{𨪐}{87112}
\saveTG{鑱}{87113}
\saveTG{𫒷}{87114}
\saveTG{𫒔}{87114}
\saveTG{鏗}{87114}
\saveTG{𨧵}{87114}
\saveTG{𨩰}{87115}
\saveTG{鑃}{87115}
\saveTG{𨬋}{87117}
\saveTG{𨬫}{87117}
\saveTG{䤥}{87117}
\saveTG{䤟}{87117}
\saveTG{𨥈}{87117}
\saveTG{𨦥}{87117}
\saveTG{𫒪}{87117}
\saveTG{𨭿}{87117}
\saveTG{𦫫}{87117}
\saveTG{釲}{87117}
\saveTG{銫}{87117}
\saveTG{鈀}{87117}
\saveTG{𨥹}{87117}
\saveTG{𨭘}{87117}
\saveTG{𨭕}{87118}
\saveTG{釸}{87120}
\saveTG{鐗}{87120}
\saveTG{釰}{87120}
\saveTG{鍸}{87120}
\saveTG{鍧}{87120}
\saveTG{鉤}{87120}
\saveTG{鈎}{87120}
\saveTG{鋼}{87120}
\saveTG{䤝}{87120}
\saveTG{銅}{87120}
\saveTG{鋾}{87120}
\saveTG{鉰}{87120}
\saveTG{鎙}{87120}
\saveTG{錭}{87120}
\saveTG{鈅}{87120}
\saveTG{鍆}{87120}
\saveTG{鉚}{87120}
\saveTG{鑭}{87120}
\saveTG{鐧}{87120}
\saveTG{銁}{87120}
\saveTG{鐦}{87120}
\saveTG{卸}{87120}
\saveTG{釖}{87120}
\saveTG{錋}{87120}
\saveTG{𨦪}{87120}
\saveTG{䥏}{87120}
\saveTG{𨨶}{87120}
\saveTG{鈞}{87120}
\saveTG{釣}{87120}
\saveTG{銄}{87120}
\saveTG{𨧹}{87121}
\saveTG{𨰎}{87121}
\saveTG{𨥤}{87121}
\saveTG{𨦫}{87121}
\saveTG{𦑺}{87121}
\saveTG{𨥸}{87121}
\saveTG{𨬔}{87121}
\saveTG{龬}{87121}
\saveTG{𨥨}{87121}
\saveTG{𠛺}{87121}
\saveTG{鎀}{87122}
\saveTG{𨧶}{87122}
\saveTG{𨰏}{87122}
\saveTG{鏐}{87122}
\saveTG{𨥘}{87123}
\saveTG{𨥽}{87126}
\saveTG{𨪵}{87126}
\saveTG{𪅲}{87127}
\saveTG{𪅠}{87127}
\saveTG{𨩤}{87127}
\saveTG{𪇩}{87127}
\saveTG{𨪷}{87127}
\saveTG{𪆷}{87127}
\saveTG{𨛏}{87127}
\saveTG{𨦹}{87127}
\saveTG{𨝫}{87127}
\saveTG{鎢}{87127}
\saveTG{钃}{87127}
\saveTG{銿}{87127}
\saveTG{鹢}{87127}
\saveTG{鷁}{87127}
\saveTG{鎁}{87127}
\saveTG{鋣}{87127}
\saveTG{鏅}{87127}
\saveTG{鹟}{87127}
\saveTG{鶲}{87127}
\saveTG{鵭}{87127}
\saveTG{釢}{87127}
\saveTG{釕}{87127}
\saveTG{鎯}{87127}
\saveTG{鐍}{87127}
\saveTG{鋦}{87127}
\saveTG{鷑}{87127}
\saveTG{鍋}{87127}
\saveTG{鈟}{87127}
\saveTG{鉹}{87127}
\saveTG{𨪤}{87127}
\saveTG{𨭾}{87127}
\saveTG{𫓎}{87127}
\saveTG{𨨎}{87127}
\saveTG{䣃}{87127}
\saveTG{𨚶}{87127}
\saveTG{𡗂}{87127}
\saveTG{𨝕}{87127}
\saveTG{𡖼}{87127}
\saveTG{𨜶}{87127}
\saveTG{𨜺}{87127}
\saveTG{𨛈}{87127}
\saveTG{𨫒}{87127}
\saveTG{𪄇}{87127}
\saveTG{𪂽}{87127}
\saveTG{𨩶}{87127}
\saveTG{𨪕}{87127}
\saveTG{𨧪}{87127}
\saveTG{𠁜}{87127}
\saveTG{𫒬}{87127}
\saveTG{𫒉}{87127}
\saveTG{𨧜}{87127}
\saveTG{䥜}{87129}
\saveTG{𨨠}{87129}
\saveTG{銀}{87132}
\saveTG{䤸}{87132}
\saveTG{鐌}{87132}
\saveTG{𨫀}{87132}
\saveTG{䥂}{87132}
\saveTG{𨧟}{87132}
\saveTG{𨯁}{87132}
\saveTG{𨫤}{87132}
\saveTG{鐹}{87132}
\saveTG{鍯}{87132}
\saveTG{鍃}{87132}
\saveTG{鉖}{87133}
\saveTG{鏠}{87135}
\saveTG{螸}{87136}
\saveTG{𨫷}{87136}
\saveTG{鎚}{87137}
\saveTG{𨯭}{87138}
\saveTG{銵}{87140}
\saveTG{釼}{87140}
\saveTG{鋷}{87140}
\saveTG{銏}{87140}
\saveTG{𫒢}{87141}
\saveTG{𫓠}{87142}
\saveTG{𨧞}{87142}
\saveTG{鏘}{87142}
\saveTG{釵}{87143}
\saveTG{𨦞}{87144}
\saveTG{鐞}{87146}
\saveTG{𨦯}{87147}
\saveTG{鋟}{87147}
\saveTG{𨧷}{87147}
\saveTG{𫓊}{87147}
\saveTG{𨥁}{87147}
\saveTG{𨨘}{87147}
\saveTG{鈱}{87147}
\saveTG{鍛}{87147}
\saveTG{𨩷}{87147}
\saveTG{𨥂}{87147}
\saveTG{鎩}{87147}
\saveTG{鎪}{87147}
\saveTG{鍜}{87147}
\saveTG{鈠}{87147}
\saveTG{釨}{87147}
\saveTG{錣}{87147}
\saveTG{𨩻}{87147}
\saveTG{𡦸}{87147}
\saveTG{𨯰}{87147}
\saveTG{鈒}{87147}
\saveTG{𨧍}{87147}
\saveTG{𨬖}{87147}
\saveTG{𫓚}{87148}
\saveTG{鉧}{87150}
\saveTG{鍕}{87152}
\saveTG{𨬯}{87152}
\saveTG{䤿}{87152}
\saveTG{鋒}{87154}
\saveTG{𨦟}{87154}
\saveTG{錚}{87157}
\saveTG{𨪭}{87158}
\saveTG{𨯾}{87158}
\saveTG{鉛}{87161}
\saveTG{鉊}{87162}
\saveTG{銘}{87162}
\saveTG{鎦}{87162}
\saveTG{𨫛}{87162}
\saveTG{鑥}{87163}
\saveTG{䥁}{87163}
\saveTG{鉻}{87164}
\saveTG{鋸}{87164}
\saveTG{鏴}{87164}
\saveTG{鍲}{87164}
\saveTG{𨧙}{87165}
\saveTG{𨪿}{87165}
\saveTG{𨧡}{87167}
\saveTG{鎇}{87167}
\saveTG{𨩲}{87168}
\saveTG{𨧱}{87172}
\saveTG{䥹}{87172}
\saveTG{𫒶}{87172}
\saveTG{䤴}{87172}
\saveTG{𨨳}{87172}
\saveTG{𨥠}{87174}
\saveTG{𨯎}{87174}
\saveTG{鈻}{87177}
\saveTG{錎}{87177}
\saveTG{𫓢}{87181}
\saveTG{𨩸}{87181}
\saveTG{𨯵}{87181}
\saveTG{𨮔}{87181}
\saveTG{𨨣}{87181}
\saveTG{鐉}{87181}
\saveTG{𣣼}{87182}
\saveTG{𣣽}{87182}
\saveTG{歙}{87182}
\saveTG{鏉}{87182}
\saveTG{鍁}{87182}
\saveTG{䥲}{87182}
\saveTG{䥗}{87182}
\saveTG{𨨢}{87182}
\saveTG{𨮈}{87182}
\saveTG{𨰇}{87182}
\saveTG{㰟}{87182}
\saveTG{𪴱}{87182}
\saveTG{欽}{87182}
\saveTG{欫}{87182}
\saveTG{鍥}{87184}
\saveTG{鍭}{87184}
\saveTG{𨩀}{87184}
\saveTG{𨮅}{87184}
\saveTG{𨩉}{87184}
\saveTG{鐭}{87184}
\saveTG{鑦}{87186}
\saveTG{𨰷}{87186}
\saveTG{鏆}{87186}
\saveTG{鈬}{87187}
\saveTG{𨰐}{87189}
\saveTG{𨯺}{87189}
\saveTG{𨰨}{87189}
\saveTG{𨫗}{87189}
\saveTG{𨯣}{87191}
\saveTG{𨨮}{87192}
\saveTG{錄}{87192}
\saveTG{鉨}{87192}
\saveTG{䥛}{87193}
\saveTG{鍒}{87194}
\saveTG{鎟}{87194}
\saveTG{𨮏}{87194}
\saveTG{𨨥}{87194}
\saveTG{𨬏}{87194}
\saveTG{𨦃}{87194}
\saveTG{䤪}{87194}
\saveTG{録}{87199}
\saveTG{鑗}{87199}
\saveTG{𩠟}{87208}
\saveTG{䬇}{87210}
\saveTG{𡯚}{87211}
\saveTG{觎}{87212}
\saveTG{尦}{87212}
\saveTG{𡰊}{87212}
\saveTG{𠻁}{87212}
\saveTG{𪛕}{87212}
\saveTG{𦢢}{87214}
\saveTG{䥨}{87216}
\saveTG{䒊}{87217}
\saveTG{𪕼}{87217}
\saveTG{𠔮}{87217}
\saveTG{𣩭}{87217}
\saveTG{𠒤}{87219}
\saveTG{𦑦}{87221}
\saveTG{𦒈}{87221}
\saveTG{𠞴}{87222}
\saveTG{𩿞}{87227}
\saveTG{𪃎}{87227}
\saveTG{𪈕}{87227}
\saveTG{䲸}{87227}
\saveTG{𨞅}{87227}
\saveTG{𫑩}{87227}
\saveTG{𨟭}{87227}
\saveTG{𡗄}{87227}
\saveTG{𨚂}{87227}
\saveTG{𨚐}{87227}
\saveTG{𪁑}{87227}
\saveTG{鸙}{87227}
\saveTG{鵜}{87227}
\saveTG{𨝙}{87227}
\saveTG{鹈}{87227}
\saveTG{鄃}{87227}
\saveTG{鳹}{87227}
\saveTG{鹣}{87227}
\saveTG{鶼}{87227}
\saveTG{鶬}{87227}
\saveTG{邠}{87227}
\saveTG{鳻}{87227}
\saveTG{䳺}{87227}
\saveTG{𪂥}{87227}
\saveTG{𪈁}{87227}
\saveTG{𨜾}{87227}
\saveTG{𨙽}{87227}
\saveTG{𩛺}{87229}
\saveTG{𪛖}{87231}
\saveTG{𣺩}{87232}
\saveTG{𪛒}{87247}
\saveTG{𫖾}{87250}
\saveTG{𧬑}{87261}
\saveTG{𨩩}{87262}
\saveTG{𠇧}{87281}
\saveTG{歈}{87282}
\saveTG{㰡}{87282}
\saveTG{𣢍}{87282}
\saveTG{𣣡}{87282}
\saveTG{𣢲}{87282}
\saveTG{歉}{87282}
\saveTG{歓}{87282}
\saveTG{欦}{87282}
\saveTG{龡}{87282}
\saveTG{龣}{87299}
\saveTG{𡳽}{87314}
\saveTG{翎}{87320}
\saveTG{𪫶}{87321}
\saveTG{𪅰}{87327}
\saveTG{邻}{87327}
\saveTG{鴒}{87327}
\saveTG{鸰}{87327}
\saveTG{鷡}{87327}
\saveTG{鄦}{87327}
\saveTG{鷷}{87327}
\saveTG{𤊝}{87332}
\saveTG{憌}{87332}
\saveTG{愬}{87332}
\saveTG{𢣰}{87333}
\saveTG{𢜂}{87334}
\saveTG{慾}{87338}
\saveTG{𣣵}{87382}
\saveTG{𢡮}{87382}
\saveTG{𣣈}{87382}
\saveTG{𣢝}{87382}
\saveTG{𪬼}{87386}
\saveTG{𢆗}{87412}
\saveTG{㸖}{87412}
\saveTG{艵}{87417}
\saveTG{朔}{87420}
\saveTG{剏}{87420}
\saveTG{𦑔}{87421}
\saveTG{𦐵}{87421}
\saveTG{𪟳}{87421}
\saveTG{𨳱}{87421}
\saveTG{𦑒}{87421}
\saveTG{𫔦}{87424}
\saveTG{𦎛}{87426}
\saveTG{𪈓}{87427}
\saveTG{𪈢}{87427}
\saveTG{𪃃}{87427}
\saveTG{𪈣}{87427}
\saveTG{𩾿}{87427}
\saveTG{鵧}{87427}
\saveTG{郱}{87427}
\saveTG{𨜰}{87427}
\saveTG{𨚅}{87427}
\saveTG{𫛨}{87427}
\saveTG{𪂻}{87427}
\saveTG{𪇙}{87427}
\saveTG{剙}{87430}
\saveTG{𢍥}{87442}
\saveTG{𣫑}{87447}
\saveTG{𣢴}{87482}
\saveTG{𣤏}{87482}
\saveTG{欮}{87482}
\saveTG{𢵡}{87502}
\saveTG{𨍁}{87506}
\saveTG{羟}{87512}
\saveTG{𦏍}{87514}
\saveTG{䍲}{87517}
\saveTG{䍯}{87517}
\saveTG{羓}{87517}
\saveTG{翔}{87520}
\saveTG{𦍤}{87521}
\saveTG{𦍗}{87521}
\saveTG{𦍨}{87521}
\saveTG{𦍿}{87523}
\saveTG{𦍵}{87526}
\saveTG{𦍻}{87526}
\saveTG{䣡}{87527}
\saveTG{䴊}{87527}
\saveTG{𨛁}{87527}
\saveTG{𦎰}{87527}
\saveTG{𫛴}{87527}
\saveTG{郸}{87527}
\saveTG{鵇}{87527}
\saveTG{鴹}{87527}
\saveTG{𦎈}{87527}
\saveTG{冁}{87532}
\saveTG{𦎆}{87543}
\saveTG{䍳}{87547}
\saveTG{羖}{87547}
\saveTG{𫅒}{87547}
\saveTG{𦎮}{87547}
\saveTG{𦏘}{87552}
\saveTG{䍭}{87554}
\saveTG{𦎾}{87557}
\saveTG{䍵}{87557}
\saveTG{𦏂}{87561}
\saveTG{𦎠}{87564}
\saveTG{䍻}{87581}
\saveTG{𢆘}{87582}
\saveTG{𦍴}{87592}
\saveTG{𦎤}{87594}
\saveTG{𧫋}{87601}
\saveTG{𤉴}{87601}
\saveTG{𨢱}{87604}
\saveTG{𨡢}{87604}
\saveTG{谻}{87610}
\saveTG{𧮬}{87610}
\saveTG{𧮷}{87610}
\saveTG{𩠭}{87614}
\saveTG{𠌑}{87617}
\saveTG{䒏}{87617}
\saveTG{𪓰}{87617}
\saveTG{翖}{87620}
\saveTG{卻}{87620}
\saveTG{朆}{87620}
\saveTG{𧯓}{87620}
\saveTG{𦒂}{87621}
\saveTG{䎊}{87621}
\saveTG{𧥻}{87621}
\saveTG{𣃑}{87621}
\saveTG{䎏}{87621}
\saveTG{𪠁}{87621}
\saveTG{䎖}{87621}
\saveTG{舒}{87622}
\saveTG{豂}{87622}
\saveTG{𧯎}{87622}
\saveTG{𧯍}{87622}
\saveTG{𧯘}{87624}
\saveTG{𧯋}{87624}
\saveTG{𧯑}{87626}
\saveTG{𧮻}{87626}
\saveTG{𩠧}{87627}
\saveTG{鸽}{87627}
\saveTG{鴿}{87627}
\saveTG{郃}{87627}
\saveTG{鹆}{87627}
\saveTG{鵒}{87627}
\saveTG{鄫}{87627}
\saveTG{鄶}{87627}
\saveTG{鄯}{87627}
\saveTG{鵨}{87627}
\saveTG{郤}{87627}
\saveTG{𪃬}{87627}
\saveTG{𨞤}{87627}
\saveTG{𨜟}{87627}
\saveTG{𨛭}{87627}
\saveTG{𨜪}{87627}
\saveTG{𨛣}{87627}
\saveTG{𪈍}{87627}
\saveTG{𪂤}{87627}
\saveTG{𪁟}{87627}
\saveTG{䜯}{87629}
\saveTG{𩠱}{87637}
\saveTG{𪵍}{87647}
\saveTG{𡥫}{87647}
\saveTG{𧮿}{87664}
\saveTG{𦧸}{87681}
\saveTG{𣣔}{87682}
\saveTG{𣣫}{87682}
\saveTG{𣤗}{87682}
\saveTG{㱃}{87682}
\saveTG{𣢺}{87682}
\saveTG{欱}{87682}
\saveTG{歚}{87682}
\saveTG{欲}{87682}
\saveTG{𧯁}{87684}
\saveTG{𧯂}{87684}
\saveTG{𦧺}{87686}
\saveTG{飢}{87710}
\saveTG{钇}{87710}
\saveTG{钒}{87710}
\saveTG{铇}{87712}
\saveTG{飷}{87712}
\saveTG{𩛐}{87712}
\saveTG{𩛀}{87712}
\saveTG{钮}{87712}
\saveTG{锰}{87712}
\saveTG{铌}{87712}
\saveTG{飽}{87712}
\saveTG{镵}{87713}
\saveTG{饞}{87713}
\saveTG{铿}{87714}
\saveTG{𩚖}{87714}
\saveTG{𩜺}{87716}
\saveTG{𩛟}{87717}
\saveTG{𩟖}{87717}
\saveTG{𩚯}{87717}
\saveTG{𩝸}{87717}
\saveTG{𪼾}{87717}
\saveTG{䬿}{87717}
\saveTG{𩜮}{87717}
\saveTG{𩜧}{87717}
\saveTG{钯}{87717}
\saveTG{铯}{87717}
\saveTG{𩚥}{87717}
\saveTG{𩚑}{87717}
\saveTG{𩝮}{87717}
\saveTG{镧}{87720}
\saveTG{锎}{87720}
\saveTG{}{87720}
\saveTG{钧}{87720}
\saveTG{锏}{87720}
\saveTG{钢}{87720}
\saveTG{餬}{87720}
\saveTG{钩}{87720}
\saveTG{罁}{87720}
\saveTG{铜}{87720}
\saveTG{飹}{87720}
\saveTG{钓}{87720}
\saveTG{铆}{87720}
\saveTG{钔}{87720}
\saveTG{飼}{87720}
\saveTG{餇}{87720}
\saveTG{餉}{87720}
\saveTG{缷}{87720}
\saveTG{钥}{87720}
\saveTG{䬨}{87721}
\saveTG{䬢}{87721}
\saveTG{𩚒}{87721}
\saveTG{𫔈}{87721}
\saveTG{䦀}{87721}
\saveTG{𦒟}{87721}
\saveTG{𩜩}{87721}
\saveTG{䥼}{87722}
\saveTG{𠊉}{87722}
\saveTG{𩚦}{87722}
\saveTG{𩚺}{87722}
\saveTG{𩝔}{87722}
\saveTG{镠}{87722}
\saveTG{𩚘}{87723}
\saveTG{𩚈}{87723}
\saveTG{𫗑}{87724}
\saveTG{𩚱}{87726}
\saveTG{𩛯}{87726}
\saveTG{䭇}{87726}
\saveTG{䬲}{87726}
\saveTG{𫓲}{87726}
\saveTG{𩝄}{87726}
\saveTG{𦈺}{87726}
\saveTG{𦉓}{87727}
\saveTG{䥾}{87727}
\saveTG{𩚛}{87727}
\saveTG{䬷}{87727}
\saveTG{𩝣}{87727}
\saveTG{𩛤}{87727}
\saveTG{𩝢}{87727}
\saveTG{𫛾}{87727}
\saveTG{𩜞}{87727}
\saveTG{𩜛}{87727}
\saveTG{𨱍}{87727}
\saveTG{䦁}{87727}
\saveTG{𡗍}{87727}
\saveTG{铘}{87727}
\saveTG{钖}{87727}
\saveTG{钨}{87727}
\saveTG{铴}{87727}
\saveTG{钌}{87727}
\saveTG{郐}{87727}
\saveTG{锔}{87727}
\saveTG{餶}{87727}
\saveTG{鷀}{87727}
\saveTG{鹚}{87727}
\saveTG{鸧}{87727}
\saveTG{𫔎}{87727}
\saveTG{𩛽}{87727}
\saveTG{𩜳}{87727}
\saveTG{𪂾}{87727}
\saveTG{𩝷}{87727}
\saveTG{䲲}{87727}
\saveTG{𩾥}{87727}
\saveTG{𩿽}{87727}
\saveTG{𩾻}{87727}
\saveTG{𩟄}{87731}
\saveTG{䭀}{87731}
\saveTG{𩛓}{87732}
\saveTG{𩟂}{87732}
\saveTG{银}{87732}
\saveTG{䬶}{87732}
\saveTG{餯}{87732}
\saveTG{𩝬}{87732}
\saveTG{锪}{87732}
\saveTG{𩞧}{87732}
\saveTG{𩝎}{87733}
\saveTG{䭔}{87737}
\saveTG{锵}{87742}
\saveTG{𩝴}{87743}
\saveTG{钗}{87743}
\saveTG{𦉇}{87744}
\saveTG{𩝁}{87744}
\saveTG{𩚄}{87744}
\saveTG{𫔉}{87744}
\saveTG{钑}{87747}
\saveTG{锼}{87747}
\saveTG{餿}{87747}
\saveTG{锻}{87747}
\saveTG{锓}{87747}
\saveTG{𨱁}{87747}
\saveTG{𫔏}{87747}
\saveTG{𩜂}{87747}
\saveTG{𩛂}{87747}
\saveTG{餟}{87747}
\saveTG{䬦}{87747}
\saveTG{}{87750}
\saveTG{餫}{87752}
\saveTG{铎}{87754}
\saveTG{锋}{87754}
\saveTG{铮}{87757}
\saveTG{镥}{87761}
\saveTG{𦉜}{87761}
\saveTG{铅}{87761}
\saveTG{𩟋}{87761}
\saveTG{铭}{87762}
\saveTG{䬰}{87762}
\saveTG{}{87762}
\saveTG{餾}{87762}
\saveTG{镏}{87762}
\saveTG{𦉉}{87762}
\saveTG{锯}{87764}
\saveTG{𩜗}{87764}
\saveTG{𩟫}{87764}
\saveTG{餎}{87764}
\saveTG{铬}{87764}
\saveTG{镅}{87767}
\saveTG{𫗗}{87767}
\saveTG{𩜴}{87768}
\saveTG{𨱊}{87772}
\saveTG{𩛍}{87777}
\saveTG{𫓮}{87777}
\saveTG{餡}{87777}
\saveTG{𩜃}{87781}
\saveTG{饌}{87781}
\saveTG{𩜹}{87781}
\saveTG{飮}{87782}
\saveTG{𨱒}{87782}
\saveTG{飲}{87782}
\saveTG{𪴴}{87782}
\saveTG{锨}{87782}
\saveTG{钦}{87782}
\saveTG{缼}{87782}
\saveTG{钡}{87782}
\saveTG{𨱌}{87782}
\saveTG{𣢆}{87782}
\saveTG{𫔋}{87782}
\saveTG{𫓿}{87782}
\saveTG{锲}{87784}
\saveTG{餱}{87784}
\saveTG{𩝍}{87784}
\saveTG{𩝆}{87784}
\saveTG{𩟇}{87784}
\saveTG{}{87784}
\saveTG{罆}{87786}
\saveTG{𩚗}{87786}
\saveTG{𩜷}{87794}
\saveTG{𫔄}{87794}
\saveTG{𨅠}{87802}
\saveTG{𤒡}{87809}
\saveTG{俎}{87812}
\saveTG{𥎾}{87817}
\saveTG{𫂮}{87817}
\saveTG{𦫧}{87817}
\saveTG{劒}{87820}
\saveTG{剱}{87820}
\saveTG{劎}{87820}
\saveTG{𥏊}{87821}
\saveTG{𠝏}{87821}
\saveTG{𦎿}{87821}
\saveTG{𥶂}{87823}
\saveTG{𨷑}{87824}
\saveTG{𥏿}{87826}
\saveTG{𥏨}{87826}
\saveTG{𥐔}{87826}
\saveTG{𥐏}{87827}
\saveTG{郑}{87827}
\saveTG{鄭}{87827}
\saveTG{鴙}{87827}
\saveTG{𩿕}{87827}
\saveTG{𨞿}{87827}
\saveTG{𫛎}{87827}
\saveTG{𪇇}{87827}
\saveTG{𥐇}{87832}
\saveTG{䂕}{87832}
\saveTG{矪}{87840}
\saveTG{劔}{87840}
\saveTG{𧷃}{87847}
\saveTG{𣫍}{87847}
\saveTG{𣫢}{87847}
\saveTG{𥏞}{87847}
\saveTG{𠐖}{87847}
\saveTG{𧷌}{87847}
\saveTG{𥎳}{87854}
\saveTG{𥏵}{87862}
\saveTG{𥖹}{87862}
\saveTG{䂏}{87862}
\saveTG{𥏍}{87862}
\saveTG{歛}{87882}
\saveTG{㰸}{87882}
\saveTG{𥎯}{87882}
\saveTG{𣤉}{87882}
\saveTG{𪿌}{87894}
\saveTG{䌓}{87903}
\saveTG{𦃗}{87903}
\saveTG{𣗰}{87904}
\saveTG{槊}{87904}
\saveTG{𠑝}{87915}
\saveTG{糴}{87915}
\saveTG{𪂘}{87927}
\saveTG{𥝼}{87927}
\saveTG{𥠿}{87927}
\saveTG{𪇝}{87927}
\saveTG{𪇢}{87927}
\saveTG{䣄}{87927}
\saveTG{𫛬}{87927}
\saveTG{叙}{87940}
\saveTG{𪝃}{87947}
\saveTG{𪠩}{87947}
\saveTG{𥫗}{88000}
\saveTG{从}{88000}
\saveTG{𠓜}{88000}
\saveTG{𥮗}{88027}
\saveTG{𨤿}{88100}
\saveTG{釞}{88100}
\saveTG{鉯}{88100}
\saveTG{丛}{88100}
\saveTG{釟}{88100}
\saveTG{竺}{88101}
\saveTG{䇥}{88102}
\saveTG{笡}{88102}
\saveTG{簋}{88102}
\saveTG{笽}{88102}
\saveTG{𥰭}{88102}
\saveTG{𥶞}{88102}
\saveTG{𥷟}{88102}
\saveTG{𥶊}{88102}
\saveTG{𥱔}{88102}
\saveTG{𥮖}{88102}
\saveTG{𥰎}{88102}
\saveTG{𫁱}{88102}
\saveTG{𥬨}{88102}
\saveTG{𥵧}{88102}
\saveTG{𥵁}{88102}
\saveTG{𥲥}{88102}
\saveTG{𥬧}{88102}
\saveTG{𧸵}{88102}
\saveTG{簠}{88102}
\saveTG{篕}{88102}
\saveTG{篮}{88102}
\saveTG{籃}{88102}
\saveTG{笁}{88102}
\saveTG{箜}{88102}
\saveTG{𥵃}{88102}
\saveTG{𥱚}{88102}
\saveTG{䇰}{88103}
\saveTG{䇪}{88104}
\saveTG{𥸀}{88104}
\saveTG{𥫦}{88104}
\saveTG{𥱆}{88104}
\saveTG{䇠}{88104}
\saveTG{𥴀}{88104}
\saveTG{籉}{88104}
\saveTG{簺}{88104}
\saveTG{𥲓}{88104}
\saveTG{簊}{88104}
\saveTG{篁}{88104}
\saveTG{筀}{88104}
\saveTG{𡔗}{88104}
\saveTG{𥭭}{88104}
\saveTG{𥯑}{88104}
\saveTG{𥲠}{88104}
\saveTG{𥵭}{88104}
\saveTG{坐}{88104}
\saveTG{䇸}{88104}
\saveTG{筌}{88104}
\saveTG{𥲔}{88105}
\saveTG{𥭤}{88105}
\saveTG{笙}{88105}
\saveTG{箠}{88105}
\saveTG{𥴁}{88105}
\saveTG{箽}{88105}
\saveTG{篂}{88105}
\saveTG{箮}{88106}
\saveTG{鳘}{88106}
\saveTG{笪}{88106}
\saveTG{𥷛}{88106}
\saveTG{䉡}{88106}
\saveTG{𥬙}{88107}
\saveTG{笠}{88108}
\saveTG{筮}{88108}
\saveTG{簦}{88108}
\saveTG{𥴡}{88108}
\saveTG{䇺}{88108}
\saveTG{𥯀}{88109}
\saveTG{签}{88109}
\saveTG{𥭎}{88110}
\saveTG{鈼}{88111}
\saveTG{𥬻}{88111}
\saveTG{𥸙}{88111}
\saveTG{鍦}{88112}
\saveTG{鉇}{88112}
\saveTG{䈌}{88112}
\saveTG{鎈}{88112}
\saveTG{筂}{88112}
\saveTG{銳}{88112}
\saveTG{鋭}{88112}
\saveTG{笵}{88112}
\saveTG{鑑}{88112}
\saveTG{鑬}{88112}
\saveTG{錓}{88112}
\saveTG{𥶎}{88112}
\saveTG{𡒪}{88112}
\saveTG{鎰}{88112}
\saveTG{𥴐}{88112}
\saveTG{𥱒}{88112}
\saveTG{𨫸}{88112}
\saveTG{𨩣}{88112}
\saveTG{𥯰}{88112}
\saveTG{𥵈}{88112}
\saveTG{箲}{88112}
\saveTG{筄}{88113}
\saveTG{𡓆}{88114}
\saveTG{銼}{88114}
\saveTG{銓}{88114}
\saveTG{𨩳}{88114}
\saveTG{𨦓}{88114}
\saveTG{篞}{88114}
\saveTG{𨫝}{88115}
\saveTG{籦}{88115}
\saveTG{𥷧}{88115}
\saveTG{𥲱}{88115}
\saveTG{𥬾}{88117}
\saveTG{𥸗}{88117}
\saveTG{𥬶}{88117}
\saveTG{釳}{88117}
\saveTG{𫒵}{88117}
\saveTG{𨦣}{88117}
\saveTG{籈}{88117}
\saveTG{𨪢}{88117}
\saveTG{𨫬}{88117}
\saveTG{𥬳}{88117}
\saveTG{𨯥}{88117}
\saveTG{鎎}{88117}
\saveTG{𢘴}{88117}
\saveTG{𨧯}{88117}
\saveTG{𥮸}{88117}
\saveTG{𥰤}{88117}
\saveTG{𥵒}{88117}
\saveTG{𥬑}{88117}
\saveTG{𨥊}{88117}
\saveTG{筑}{88117}
\saveTG{鍂}{88119}
\saveTG{𨰻}{88119}
\saveTG{𨰹}{88119}
\saveTG{𥭃}{88120}
\saveTG{䉧}{88120}
\saveTG{𥷢}{88120}
\saveTG{𫒌}{88120}
\saveTG{䇛}{88120}
\saveTG{鍮}{88121}
\saveTG{鎆}{88121}
\saveTG{𥳷}{88121}
\saveTG{𥬢}{88121}
\saveTG{䇵}{88121}
\saveTG{𥬮}{88121}
\saveTG{𨨞}{88121}
\saveTG{𥱀}{88121}
\saveTG{䉹}{88121}
\saveTG{𨰓}{88121}
\saveTG{𥬛}{88122}
\saveTG{鉁}{88122}
\saveTG{箌}{88123}
\saveTG{䈩}{88123}
\saveTG{銻}{88127}
\saveTG{𥴰}{88127}
\saveTG{䇘}{88127}
\saveTG{𥵇}{88127}
\saveTG{𥬤}{88127}
\saveTG{𨥟}{88127}
\saveTG{𫒑}{88127}
\saveTG{𠓻}{88127}
\saveTG{𥬿}{88127}
\saveTG{𨯖}{88127}
\saveTG{𥳾}{88127}
\saveTG{𨰝}{88127}
\saveTG{𨬎}{88127}
\saveTG{𨯪}{88127}
\saveTG{𫓂}{88127}
\saveTG{篛}{88127}
\saveTG{𥴴}{88127}
\saveTG{𥲟}{88127}
\saveTG{𥬣}{88127}
\saveTG{𥭀}{88127}
\saveTG{𥯙}{88127}
\saveTG{𥮺}{88127}
\saveTG{𨫅}{88127}
\saveTG{𥵚}{88127}
\saveTG{筇}{88127}
\saveTG{筯}{88127}
\saveTG{鑰}{88127}
\saveTG{鎓}{88127}
\saveTG{𨯽}{88127}
\saveTG{笻}{88127}
\saveTG{筠}{88127}
\saveTG{鈐}{88127}
\saveTG{錀}{88127}
\saveTG{鈖}{88127}
\saveTG{笃}{88127}
\saveTG{簜}{88127}
\saveTG{鉓}{88127}
\saveTG{𨯡}{88127}
\saveTG{䈵}{88127}
\saveTG{𥶨}{88127}
\saveTG{䈮}{88127}
\saveTG{𥷏}{88127}
\saveTG{䈳}{88127}
\saveTG{䈬}{88127}
\saveTG{笵}{88127}
\saveTG{𥵥}{88127}
\saveTG{笉}{88127}
\saveTG{筂}{88127}
\saveTG{𨪗}{88127}
\saveTG{𫂟}{88127}
\saveTG{𥬚}{88128}
\saveTG{𥸢}{88131}
\saveTG{}{88131}
\saveTG{鎡}{88132}
\saveTG{鈆}{88132}
\saveTG{錜}{88132}
\saveTG{𩛒}{88132}
\saveTG{𨩭}{88132}
\saveTG{𨪂}{88132}
\saveTG{䥙}{88132}
\saveTG{𨮋}{88132}
\saveTG{鈴}{88132}
\saveTG{𥯠}{88132}
\saveTG{𨭨}{88133}
\saveTG{𨨡}{88133}
\saveTG{鐩}{88133}
\saveTG{𨨟}{88133}
\saveTG{𨭝}{88134}
\saveTG{𥳎}{88136}
\saveTG{𨭪}{88136}
\saveTG{𥴵}{88136}
\saveTG{蠞}{88136}
\saveTG{𥱄}{88136}
\saveTG{𥭥}{88136}
\saveTG{鎌}{88137}
\saveTG{𠁟}{88137}
\saveTG{𨪯}{88137}
\saveTG{鎹}{88138}
\saveTG{𨬊}{88140}
\saveTG{敛}{88140}
\saveTG{𨫾}{88140}
\saveTG{𢿴}{88140}
\saveTG{𫒋}{88140}
\saveTG{𨥞}{88140}
\saveTG{𫒍}{88140}
\saveTG{䥕}{88140}
\saveTG{䥩}{88140}
\saveTG{𨨦}{88140}
\saveTG{䥞}{88140}
\saveTG{𨨽}{88140}
\saveTG{鏾}{88140}
\saveTG{𨬒}{88140}
\saveTG{𨬠}{88140}
\saveTG{𨫼}{88140}
\saveTG{𨯋}{88140}
\saveTG{鐓}{88140}
\saveTG{𥮍}{88141}
\saveTG{鉼}{88141}
\saveTG{𥲂}{88141}
\saveTG{𥶴}{88142}
\saveTG{簿}{88142}
\saveTG{𥴾}{88143}
\saveTG{𥵢}{88143}
\saveTG{𥭿}{88143}
\saveTG{篈}{88143}
\saveTG{𨬻}{88143}
\saveTG{𨩍}{88144}
\saveTG{𥵪}{88144}
\saveTG{𥱼}{88145}
\saveTG{𥭧}{88146}
\saveTG{鐏}{88146}
\saveTG{𥮶}{88146}
\saveTG{𥵓}{88147}
\saveTG{𥸘}{88147}
\saveTG{𥰛}{88147}
\saveTG{𥲝}{88147}
\saveTG{箥}{88147}
\saveTG{鍑}{88147}
\saveTG{䈣}{88148}
\saveTG{𥮭}{88148}
\saveTG{𨦡}{88151}
\saveTG{𥷰}{88151}
\saveTG{𨮉}{88151}
\saveTG{𨰈}{88152}
\saveTG{鎿}{88152}
\saveTG{𨦧}{88152}
\saveTG{籤}{88153}
\saveTG{䈅}{88153}
\saveTG{籛}{88153}
\saveTG{𫂔}{88153}
\saveTG{籖}{88153}
\saveTG{𨬾}{88154}
\saveTG{𫓔}{88155}
\saveTG{𥱐}{88156}
\saveTG{䈽}{88156}
\saveTG{𨮼}{88156}
\saveTG{𥳒}{88156}
\saveTG{𥳘}{88156}
\saveTG{鋂}{88157}
\saveTG{𨮧}{88157}
\saveTG{𥳻}{88157}
\saveTG{𪞍}{88161}
\saveTG{鐥}{88161}
\saveTG{鉿}{88161}
\saveTG{𫓌}{88161}
\saveTG{鐠}{88161}
\saveTG{𨯆}{88161}
\saveTG{𨬼}{88161}
\saveTG{𥱱}{88162}
\saveTG{𥶅}{88162}
\saveTG{𨬣}{88162}
\saveTG{𥮒}{88162}
\saveTG{𥵄}{88162}
\saveTG{鋡}{88162}
\saveTG{箔}{88162}
\saveTG{䈃}{88162}
\saveTG{𨮱}{88163}
\saveTG{箈}{88163}
\saveTG{𥸐}{88164}
\saveTG{𨮿}{88164}
\saveTG{簬}{88164}
\saveTG{𫅡}{88164}
\saveTG{𨩊}{88164}
\saveTG{鏳}{88166}
\saveTG{𨭗}{88166}
\saveTG{鎗}{88167}
\saveTG{𥃙}{88168}
\saveTG{鋊}{88168}
\saveTG{籓}{88169}
\saveTG{𫒘}{88172}
\saveTG{𥲌}{88172}
\saveTG{𥭌}{88174}
\saveTG{𥮢}{88174}
\saveTG{𠉒}{88175}
\saveTG{𥬠}{88177}
\saveTG{篲}{88177}
\saveTG{筜}{88177}
\saveTG{鏦}{88181}
\saveTG{篊}{88181}
\saveTG{鏇}{88181}
\saveTG{𥶛}{88182}
\saveTG{𨬢}{88182}
\saveTG{𥳧}{88182}
\saveTG{𫁴}{88183}
\saveTG{鎂}{88184}
\saveTG{鏃}{88184}
\saveTG{䥖}{88184}
\saveTG{𨫇}{88184}
\saveTG{鉃}{88184}
\saveTG{𨯘}{88186}
\saveTG{鐱}{88186}
\saveTG{𨥎}{88188}
\saveTG{𥲄}{88189}
\saveTG{鉩}{88190}
\saveTG{鏼}{88192}
\saveTG{𥮳}{88192}
\saveTG{𨰭}{88193}
\saveTG{𥴽}{88194}
\saveTG{𥷍}{88194}
\saveTG{𥴓}{88194}
\saveTG{𫂒}{88195}
\saveTG{籙}{88199}
\saveTG{𥫘}{88200}
\saveTG{𠁥}{88200}
\saveTG{𥯢}{88201}
\saveTG{𥫙}{88201}
\saveTG{䇡}{88201}
\saveTG{𥲿}{88202}
\saveTG{篸}{88202}
\saveTG{𥲲}{88202}
\saveTG{𥶃}{88203}
\saveTG{䉕}{88206}
\saveTG{筗}{88206}
\saveTG{𥸖}{88206}
\saveTG{𥭹}{88207}
\saveTG{箩}{88207}
\saveTG{笒}{88207}
\saveTG{𥭋}{88207}
\saveTG{䇙}{88210}
\saveTG{𡰛}{88211}
\saveTG{𥬩}{88211}
\saveTG{筰}{88211}
\saveTG{𥰾}{88211}
\saveTG{籠}{88211}
\saveTG{笮}{88211}
\saveTG{筅}{88212}
\saveTG{笎}{88212}
\saveTG{籚}{88212}
\saveTG{𡯗}{88212}
\saveTG{𥲤}{88212}
\saveTG{籭}{88212}
\saveTG{簏}{88212}
\saveTG{筧}{88212}
\saveTG{笕}{88212}
\saveTG{筦}{88212}
\saveTG{箷}{88212}
\saveTG{篼}{88212}
\saveTG{𥷒}{88212}
\saveTG{𥲜}{88212}
\saveTG{箢}{88212}
\saveTG{𥳁}{88212}
\saveTG{𡰉}{88213}
\saveTG{𥯂}{88214}
\saveTG{𥯅}{88214}
\saveTG{𥮃}{88214}
\saveTG{𥰈}{88214}
\saveTG{𥲃}{88214}
\saveTG{簆}{88214}
\saveTG{𥲎}{88214}
\saveTG{箼}{88214}
\saveTG{𥲪}{88214}
\saveTG{䇮}{88214}
\saveTG{𥵷}{88215}
\saveTG{𥯁}{88215}
\saveTG{𥳌}{88215}
\saveTG{𥯄}{88215}
\saveTG{𥷬}{88215}
\saveTG{籗}{88215}
\saveTG{篧}{88215}
\saveTG{籬}{88215}
\saveTG{籱}{88215}
\saveTG{𥰁}{88215}
\saveTG{籊}{88215}
\saveTG{簅}{88215}
\saveTG{𥸆}{88216}
\saveTG{䉱}{88216}
\saveTG{𥰘}{88217}
\saveTG{𥳱}{88217}
\saveTG{𥱑}{88217}
\saveTG{𥰖}{88217}
\saveTG{𩴨}{88217}
\saveTG{𥸏}{88217}
\saveTG{𥫺}{88217}
\saveTG{𥬱}{88217}
\saveTG{𥷖}{88217}
\saveTG{𥭨}{88217}
\saveTG{𥭬}{88217}
\saveTG{𠒃}{88217}
\saveTG{𥮄}{88217}
\saveTG{䇻}{88217}
\saveTG{𥶢}{88217}
\saveTG{篪}{88217}
\saveTG{𥳸}{88217}
\saveTG{𫂯}{88217}
\saveTG{𥶓}{88217}
\saveTG{箎}{88217}
\saveTG{𥷜}{88217}
\saveTG{𫁼}{88217}
\saveTG{簄}{88217}
\saveTG{𥲁}{88217}
\saveTG{𥴯}{88217}
\saveTG{籯}{88217}
\saveTG{籝}{88217}
\saveTG{竼}{88217}
\saveTG{笐}{88217}
\saveTG{䈈}{88217}
\saveTG{𥷶}{88217}
\saveTG{𥳲}{88218}
\saveTG{𥸠}{88218}
\saveTG{竹}{88220}
\saveTG{𥬭}{88220}
\saveTG{𥬼}{88220}
\saveTG{箭}{88221}
\saveTG{筕}{88221}
\saveTG{箅}{88221}
\saveTG{𥰧}{88221}
\saveTG{𥵆}{88221}
\saveTG{𥷙}{88221}
\saveTG{䉮}{88221}
\saveTG{𥮆}{88221}
\saveTG{𥬊}{88221}
\saveTG{𥭦}{88221}
\saveTG{𥵘}{88221}
\saveTG{𥳑}{88221}
\saveTG{𥫶}{88221}
\saveTG{𥬂}{88221}
\saveTG{𥶽}{88221}
\saveTG{𫂬}{88221}
\saveTG{𥲋}{88221}
\saveTG{𥵬}{88222}
\saveTG{𥱤}{88222}
\saveTG{𥮾}{88222}
\saveTG{簓}{88222}
\saveTG{箾}{88223}
\saveTG{𥫨}{88223}
\saveTG{䉍}{88223}
\saveTG{𥸍}{88224}
\saveTG{𥳐}{88224}
\saveTG{𥴥}{88224}
\saveTG{𥶆}{88226}
\saveTG{𥷄}{88227}
\saveTG{箐}{88227}
\saveTG{筋}{88227}
\saveTG{䇟}{88227}
\saveTG{簥}{88227}
\saveTG{簡}{88227}
\saveTG{简}{88227}
\saveTG{笏}{88227}
\saveTG{篙}{88227}
\saveTG{筩}{88227}
\saveTG{筒}{88227}
\saveTG{篅}{88227}
\saveTG{䉲}{88227}
\saveTG{𥴩}{88227}
\saveTG{𥳔}{88227}
\saveTG{𥲭}{88227}
\saveTG{𠕞}{88227}
\saveTG{𥫳}{88227}
\saveTG{𥷉}{88227}
\saveTG{𥵨}{88227}
\saveTG{𥭊}{88227}
\saveTG{𥮫}{88227}
\saveTG{𥮐}{88227}
\saveTG{𥭞}{88227}
\saveTG{𥬽}{88227}
\saveTG{𥮬}{88227}
\saveTG{𥵀}{88227}
\saveTG{䈧}{88227}
\saveTG{𥭖}{88227}
\saveTG{𥯷}{88227}
\saveTG{䇶}{88227}
\saveTG{𦠗}{88227}
\saveTG{𦠡}{88227}
\saveTG{𦠪}{88227}
\saveTG{𥴱}{88227}
\saveTG{𦢆}{88227}
\saveTG{𥵵}{88227}
\saveTG{𥶿}{88227}
\saveTG{𥸎}{88227}
\saveTG{𥵑}{88227}
\saveTG{𥶧}{88227}
\saveTG{𥳫}{88227}
\saveTG{䈝}{88227}
\saveTG{𥵸}{88227}
\saveTG{䈁}{88227}
\saveTG{𥭕}{88227}
\saveTG{𥱊}{88227}
\saveTG{𥭘}{88227}
\saveTG{𥷽}{88227}
\saveTG{𥷝}{88227}
\saveTG{𥫴}{88227}
\saveTG{𥮉}{88227}
\saveTG{䈒}{88227}
\saveTG{䈻}{88227}
\saveTG{𥲈}{88227}
\saveTG{𥭈}{88227}
\saveTG{𥯔}{88227}
\saveTG{䇖}{88227}
\saveTG{𥳥}{88227}
\saveTG{䈑}{88227}
\saveTG{𥫚}{88227}
\saveTG{𥭩}{88227}
\saveTG{𥲫}{88227}
\saveTG{𥰆}{88227}
\saveTG{𥵣}{88227}
\saveTG{𥵍}{88227}
\saveTG{𥱪}{88227}
\saveTG{𥰀}{88227}
\saveTG{𥮪}{88227}
\saveTG{𥯕}{88227}
\saveTG{𥮩}{88227}
\saveTG{䇤}{88227}
\saveTG{𥬵}{88227}
\saveTG{𥰢}{88227}
\saveTG{第}{88227}
\saveTG{篇}{88227}
\saveTG{𥳜}{88227}
\saveTG{笫}{88227}
\saveTG{箒}{88227}
\saveTG{籥}{88227}
\saveTG{篽}{88227}
\saveTG{簫}{88227}
\saveTG{簘}{88227}
\saveTG{箫}{88227}
\saveTG{筲}{88227}
\saveTG{筛}{88227}
\saveTG{笍}{88227}
\saveTG{篟}{88227}
\saveTG{篣}{88227}
\saveTG{簢}{88227}
\saveTG{籋}{88227}
\saveTG{䈥}{88227}
\saveTG{篱}{88227}
\saveTG{籣}{88227}
\saveTG{𥷹}{88228}
\saveTG{𥴒}{88228}
\saveTG{笇}{88230}
\saveTG{𥬃}{88230}
\saveTG{𥬷}{88231}
\saveTG{𥶔}{88231}
\saveTG{𧐂}{88231}
\saveTG{𡎦}{88231}
\saveTG{𥫫}{88231}
\saveTG{𥮱}{88231}
\saveTG{𥫪}{88232}
\saveTG{𥫸}{88232}
\saveTG{𥱂}{88232}
\saveTG{𥶻}{88232}
\saveTG{䈨}{88232}
\saveTG{䉉}{88232}
\saveTG{𥱉}{88232}
\saveTG{𥭼}{88232}
\saveTG{𥰠}{88232}
\saveTG{𠏧}{88232}
\saveTG{𥴠}{88232}
\saveTG{𥵫}{88232}
\saveTG{笟}{88232}
\saveTG{𥮰}{88232}
\saveTG{籇}{88232}
\saveTG{笊}{88232}
\saveTG{篆}{88232}
\saveTG{𥱭}{88232}
\saveTG{𫁾}{88232}
\saveTG{𥵿}{88232}
\saveTG{䉌}{88232}
\saveTG{𥴧}{88232}
\saveTG{𥵛}{88232}
\saveTG{𥲰}{88232}
\saveTG{𥯐}{88232}
\saveTG{𥶼}{88233}
\saveTG{𥮴}{88233}
\saveTG{箊}{88233}
\saveTG{䇚}{88233}
\saveTG{𥴎}{88233}
\saveTG{笨}{88234}
\saveTG{䉀}{88234}
\saveTG{𥳍}{88236}
\saveTG{𥵖}{88236}
\saveTG{𥶌}{88236}
\saveTG{䈴}{88237}
\saveTG{簾}{88237}
\saveTG{𢾄}{88240}
\saveTG{𥱲}{88240}
\saveTG{𥬖}{88240}
\saveTG{攽}{88240}
\saveTG{敚}{88240}
\saveTG{敓}{88240}
\saveTG{𢼥}{88240}
\saveTG{𥱿}{88241}
\saveTG{𥴬}{88241}
\saveTG{箳}{88241}
\saveTG{簈}{88241}
\saveTG{笌}{88241}
\saveTG{𥯊}{88242}
\saveTG{䉄}{88242}
\saveTG{𥳇}{88242}
\saveTG{𥳀}{88243}
\saveTG{䉃}{88243}
\saveTG{符}{88243}
\saveTG{𥰒}{88243}
\saveTG{笩}{88243}
\saveTG{䈸}{88243}
\saveTG{𥮓}{88244}
\saveTG{𥰅}{88244}
\saveTG{𥴤}{88245}
\saveTG{𥯯}{88245}
\saveTG{簲}{88246}
\saveTG{簰}{88246}
\saveTG{箯}{88246}
\saveTG{𥯖}{88247}
\saveTG{𥭸}{88247}
\saveTG{𥰦}{88247}
\saveTG{𥵩}{88247}
\saveTG{𥭪}{88247}
\saveTG{𫂈}{88247}
\saveTG{𥲬}{88247}
\saveTG{䈔}{88247}
\saveTG{笈}{88247}
\saveTG{箙}{88247}
\saveTG{䈗}{88247}
\saveTG{䉬}{88247}
\saveTG{𥷱}{88247}
\saveTG{𤿤}{88247}
\saveTG{𫂜}{88247}
\saveTG{𥯞}{88247}
\saveTG{𥳊}{88247}
\saveTG{䈜}{88247}
\saveTG{𥲼}{88247}
\saveTG{䉨}{88247}
\saveTG{𥳴}{88247}
\saveTG{𥯼}{88247}
\saveTG{𥳨}{88247}
\saveTG{𥴫}{88247}
\saveTG{𥸇}{88247}
\saveTG{𥸂}{88247}
\saveTG{𫁻}{88247}
\saveTG{䈛}{88247}
\saveTG{𥰙}{88248}
\saveTG{𥰹}{88248}
\saveTG{䉰}{88248}
\saveTG{筱}{88248}
\saveTG{䉷}{88248}
\saveTG{𥳃}{88248}
\saveTG{䉈}{88248}
\saveTG{𥷭}{88248}
\saveTG{𥱙}{88248}
\saveTG{䉠}{88248}
\saveTG{𥷲}{88248}
\saveTG{𥳆}{88248}
\saveTG{𥲖}{88248}
\saveTG{𥲉}{88249}
\saveTG{䉏}{88252}
\saveTG{筬}{88253}
\saveTG{筏}{88253}
\saveTG{箴}{88253}
\saveTG{簚}{88253}
\saveTG{篾}{88253}
\saveTG{𥱡}{88253}
\saveTG{𥱢}{88253}
\saveTG{𥱗}{88253}
\saveTG{𥴸}{88253}
\saveTG{𥯣}{88253}
\saveTG{𥰓}{88253}
\saveTG{𥵞}{88253}
\saveTG{䉔}{88253}
\saveTG{𥮋}{88253}
\saveTG{𨎯}{88256}
\saveTG{𨏇}{88256}
\saveTG{箻}{88257}
\saveTG{𥶒}{88257}
\saveTG{𥳞}{88257}
\saveTG{𥴿}{88261}
\saveTG{𫂕}{88261}
\saveTG{簷}{88261}
\saveTG{𧭨}{88261}
\saveTG{𫂭}{88261}
\saveTG{䈹}{88262}
\saveTG{𥯎}{88262}
\saveTG{𥴕}{88264}
\saveTG{篖}{88265}
\saveTG{𥶰}{88266}
\saveTG{篬}{88267}
\saveTG{𥴔}{88267}
\saveTG{篃}{88267}
\saveTG{𠑐}{88267}
\saveTG{𥶋}{88269}
\saveTG{𥮝}{88272}
\saveTG{𥬍}{88272}
\saveTG{𥶹}{88274}
\saveTG{𥫿}{88277}
\saveTG{𥰽}{88280}
\saveTG{簱}{88281}
\saveTG{篵}{88281}
\saveTG{𥯏}{88281}
\saveTG{簴}{88281}
\saveTG{籏}{88281}
\saveTG{簁}{88281}
\saveTG{𥱰}{88282}
\saveTG{𥲗}{88282}
\saveTG{𥵂}{88284}
\saveTG{𥱌}{88284}
\saveTG{簇}{88284}
\saveTG{篌}{88284}
\saveTG{䈆}{88284}
\saveTG{𥰉}{88284}
\saveTG{𥵯}{88285}
\saveTG{𥵜}{88285}
\saveTG{𥶐}{88286}
\saveTG{籲}{88286}
\saveTG{𥸊}{88286}
\saveTG{𥷎}{88286}
\saveTG{𥳗}{88286}
\saveTG{𥸜}{88286}
\saveTG{𥷅}{88289}
\saveTG{𥭳}{88289}
\saveTG{𫂦}{88291}
\saveTG{𥭫}{88292}
\saveTG{𥮜}{88292}
\saveTG{𥳦}{88292}
\saveTG{籘}{88293}
\saveTG{篠}{88294}
\saveTG{𥱥}{88294}
\saveTG{篨}{88294}
\saveTG{簶}{88299}
\saveTG{𫂞}{88299}
\saveTG{籐}{88299}
\saveTG{𠓺}{88300}
\saveTG{𥳟}{88301}
\saveTG{䉦}{88301}
\saveTG{𥱃}{88301}
\saveTG{𥳠}{88301}
\saveTG{笭}{88302}
\saveTG{簻}{88302}
\saveTG{籩}{88302}
\saveTG{笾}{88302}
\saveTG{𥯧}{88302}
\saveTG{𥸅}{88302}
\saveTG{𥳤}{88302}
\saveTG{𥴪}{88302}
\saveTG{篴}{88303}
\saveTG{𥴦}{88303}
\saveTG{笗}{88303}
\saveTG{籧}{88303}
\saveTG{𥷻}{88304}
\saveTG{篷}{88305}
\saveTG{𥳅}{88306}
\saveTG{簉}{88306}
\saveTG{𥶷}{88308}
\saveTG{𫂗}{88309}
\saveTG{𥱻}{88309}
\saveTG{𥳿}{88311}
\saveTG{𨥷}{88319}
\saveTG{𥮻}{88320}
\saveTG{𢤉}{88327}
\saveTG{篤}{88327}
\saveTG{篶}{88327}
\saveTG{𥲆}{88327}
\saveTG{𥷑}{88327}
\saveTG{䉣}{88327}
\saveTG{𥫩}{88327}
\saveTG{䉆}{88327}
\saveTG{𥶘}{88327}
\saveTG{𪄴}{88327}
\saveTG{𥵌}{88327}
\saveTG{䈡}{88330}
\saveTG{怂}{88330}
\saveTG{}{88330}
\saveTG{𥭜}{88331}
\saveTG{篜}{88331}
\saveTG{𥴏}{88331}
\saveTG{𥤗}{88331}
\saveTG{𥮿}{88331}
\saveTG{𥷀}{88331}
\saveTG{𥷾}{88331}
\saveTG{𥲅}{88331}
\saveTG{𥳬}{88331}
\saveTG{𥭡}{88331}
\saveTG{䉑}{88331}
\saveTG{䵵}{88331}
\saveTG{𢤷}{88332}
\saveTG{𥯱}{88332}
\saveTG{𥴳}{88332}
\saveTG{𫂠}{88333}
\saveTG{𥴺}{88333}
\saveTG{䈚}{88333}
\saveTG{𥳝}{88333}
\saveTG{𥴲}{88334}
\saveTG{𥷌}{88334}
\saveTG{𥵴}{88334}
\saveTG{𢥰}{88334}
\saveTG{慜}{88334}
\saveTG{𥵕}{88336}
\saveTG{鰵}{88336}
\saveTG{𥰝}{88336}
\saveTG{𥯨}{88336}
\saveTG{𥶬}{88337}
\saveTG{𥶙}{88337}
\saveTG{𥲸}{88338}
\saveTG{笗}{88338}
\saveTG{𥮘}{88338}
\saveTG{𥳋}{88338}
\saveTG{𥲡}{88338}
\saveTG{𥮨}{88338}
\saveTG{𥶟}{88338}
\saveTG{𥳚}{88338}
\saveTG{敜}{88340}
\saveTG{等}{88341}
\saveTG{篿}{88343}
\saveTG{筹}{88344}
\saveTG{𥳰}{88346}
\saveTG{𥭐}{88349}
\saveTG{䉳}{88351}
\saveTG{𥸣}{88361}
\saveTG{𥷫}{88361}
\saveTG{𫅜}{88364}
\saveTG{𥵝}{88386}
\saveTG{𦏠}{88389}
\saveTG{𥷊}{88396}
\saveTG{𥫾}{88400}
\saveTG{𠦏}{88400}
\saveTG{竽}{88401}
\saveTG{筵}{88401}
\saveTG{筳}{88401}
\saveTG{耸}{88401}
\saveTG{箿}{88401}
\saveTG{竿}{88401}
\saveTG{筚}{88401}
\saveTG{𢆑}{88401}
\saveTG{𥴚}{88401}
\saveTG{筸}{88401}
\saveTG{𫁵}{88403}
\saveTG{𥫜}{88403}
\saveTG{𥫝}{88403}
\saveTG{簍}{88404}
\saveTG{篓}{88404}
\saveTG{笅}{88404}
\saveTG{𥯦}{88405}
\saveTG{𥴄}{88406}
\saveTG{𥱹}{88406}
\saveTG{𫂍}{88406}
\saveTG{𥲑}{88406}
\saveTG{𥲺}{88406}
\saveTG{𥴃}{88406}
\saveTG{箄}{88406}
\saveTG{𥷦}{88406}
\saveTG{簟}{88406}
\saveTG{䈇}{88406}
\saveTG{𥵙}{88406}
\saveTG{𥫞}{88407}
\saveTG{𥴨}{88407}
\saveTG{𥫣}{88407}
\saveTG{籰}{88407}
\saveTG{𥰞}{88407}
\saveTG{䉶}{88407}
\saveTG{𫂝}{88407}
\saveTG{䈠}{88407}
\saveTG{䈊}{88407}
\saveTG{䈦}{88407}
\saveTG{筟}{88407}
\saveTG{箰}{88407}
\saveTG{𥭟}{88407}
\saveTG{篗}{88407}
\saveTG{𥰺}{88407}
\saveTG{籆}{88407}
\saveTG{夎}{88407}
\saveTG{筊}{88408}
\saveTG{箤}{88408}
\saveTG{䇯}{88411}
\saveTG{笼}{88414}
\saveTG{䉅}{88417}
\saveTG{𥬗}{88417}
\saveTG{𥭙}{88417}
\saveTG{𫁲}{88417}
\saveTG{𥫹}{88417}
\saveTG{𥲇}{88417}
\saveTG{𥷵}{88417}
\saveTG{𥶸}{88417}
\saveTG{𥯇}{88417}
\saveTG{笼}{88417}
\saveTG{𥭴}{88418}
\saveTG{䇷}{88420}
\saveTG{𥮙}{88420}
\saveTG{䈀}{88420}
\saveTG{𥭄}{88421}
\saveTG{𥭇}{88422}
\saveTG{𥷤}{88427}
\saveTG{𥵱}{88427}
\saveTG{𥲊}{88427}
\saveTG{簩}{88427}
\saveTG{竻}{88427}
\saveTG{篘}{88427}
\saveTG{𥯘}{88427}
\saveTG{𪈅}{88427}
\saveTG{𩽘}{88427}
\saveTG{𥳭}{88427}
\saveTG{𥫷}{88427}
\saveTG{𩍸}{88427}
\saveTG{𥰬}{88427}
\saveTG{𥷚}{88427}
\saveTG{𥶶}{88427}
\saveTG{𥰚}{88429}
\saveTG{𥭝}{88429}
\saveTG{箛}{88432}
\saveTG{䉁}{88432}
\saveTG{𢼶}{88440}
\saveTG{𢽑}{88440}
\saveTG{𥯹}{88441}
\saveTG{𥰰}{88441}
\saveTG{𥷂}{88441}
\saveTG{𥵳}{88441}
\saveTG{䈂}{88441}
\saveTG{𥷁}{88441}
\saveTG{䈉}{88441}
\saveTG{䉸}{88441}
\saveTG{簳}{88441}
\saveTG{笄}{88441}
\saveTG{筓}{88441}
\saveTG{筭}{88441}
\saveTG{𥶜}{88441}
\saveTG{𥷨}{88442}
\saveTG{簙}{88442}
\saveTG{𥬒}{88443}
\saveTG{𥫠}{88443}
\saveTG{𥫲}{88443}
\saveTG{𥱈}{88443}
\saveTG{𥫭}{88443}
\saveTG{笲}{88443}
\saveTG{𥶝}{88444}
\saveTG{𥵋}{88444}
\saveTG{𥮈}{88444}
\saveTG{𢍩}{88446}
\saveTG{算}{88446}
\saveTG{𥱅}{88446}
\saveTG{箃}{88447}
\saveTG{𥰩}{88447}
\saveTG{𥫼}{88447}
\saveTG{𥬕}{88447}
\saveTG{𥯿}{88447}
\saveTG{𥴆}{88447}
\saveTG{𥸌}{88447}
\saveTG{𥳣}{88447}
\saveTG{𥱏}{88447}
\saveTG{𥱎}{88447}
\saveTG{𫁷}{88447}
\saveTG{𥶈}{88447}
\saveTG{䈲}{88447}
\saveTG{笯}{88447}
\saveTG{篝}{88447}
\saveTG{笧}{88447}
\saveTG{䉤}{88448}
\saveTG{𥴊}{88448}
\saveTG{𥮎}{88448}
\saveTG{𥲯}{88448}
\saveTG{𥶄}{88448}
\saveTG{𡠅}{88448}
\saveTG{籔}{88448}
\saveTG{䇳}{88453}
\saveTG{𥬪}{88453}
\saveTG{𥴣}{88454}
\saveTG{䇩}{88456}
\saveTG{𥯉}{88462}
\saveTG{笳}{88463}
\saveTG{筎}{88463}
\saveTG{䉋}{88467}
\saveTG{𥶫}{88467}
\saveTG{𡴤}{88472}
\saveTG{𥫢}{88473}
\saveTG{𫂖}{88474}
\saveTG{𥯴}{88478}
\saveTG{𥱩}{88480}
\saveTG{𥴋}{88482}
\saveTG{𥲩}{88483}
\saveTG{𥯺}{88483}
\saveTG{𥸡}{88486}
\saveTG{𥱖}{88493}
\saveTG{𥶭}{88494}
\saveTG{𥽧}{88494}
\saveTG{𥴘}{88494}
\saveTG{𥶁}{88495}
\saveTG{𦍏}{88500}
\saveTG{𥬴}{88501}
\saveTG{𥰪}{88502}
\saveTG{𥳶}{88502}
\saveTG{篫}{88502}
\saveTG{𥬬}{88502}
\saveTG{𥭲}{88502}
\saveTG{𥬄}{88502}
\saveTG{𢲿}{88502}
\saveTG{𥫻}{88503}
\saveTG{笺}{88503}
\saveTG{䇝}{88503}
\saveTG{箋}{88503}
\saveTG{篳}{88504}
\saveTG{䇨}{88504}
\saveTG{𤛎}{88504}
\saveTG{𥭗}{88504}
\saveTG{𥰃}{88506}
\saveTG{𥸈}{88506}
\saveTG{簞}{88506}
\saveTG{箪}{88506}
\saveTG{筻}{88506}
\saveTG{𥫯}{88506}
\saveTG{笚}{88506}
\saveTG{箏}{88507}
\saveTG{𥬁}{88507}
\saveTG{筆}{88507}
\saveTG{笋}{88507}
\saveTG{筝}{88507}
\saveTG{𥲵}{88512}
\saveTG{範}{88512}
\saveTG{箍}{88512}
\saveTG{𥷘}{88515}
\saveTG{𥮼}{88517}
\saveTG{筢}{88517}
\saveTG{笂}{88517}
\saveTG{𫂄}{88517}
\saveTG{䈭}{88517}
\saveTG{𥰳}{88517}
\saveTG{𦍥}{88517}
\saveTG{𥱍}{88517}
\saveTG{𥮚}{88520}
\saveTG{䇽}{88521}
\saveTG{𥷋}{88521}
\saveTG{𠧊}{88521}
\saveTG{羭}{88521}
\saveTG{𦍪}{88522}
\saveTG{笰}{88527}
\saveTG{䈰}{88527}
\saveTG{𥶡}{88527}
\saveTG{𥱴}{88527}
\saveTG{𥷥}{88527}
\saveTG{𥭵}{88527}
\saveTG{𥮇}{88527}
\saveTG{𥷴}{88527}
\saveTG{𫁳}{88527}
\saveTG{𥲾}{88527}
\saveTG{𠁮}{88527}
\saveTG{𥲳}{88527}
\saveTG{𫂅}{88527}
\saveTG{箉}{88527}
\saveTG{簕}{88527}
\saveTG{羒}{88527}
\saveTG{𥭏}{88531}
\saveTG{𦎴}{88531}
\saveTG{𥳺}{88532}
\saveTG{䉖}{88532}
\saveTG{羚}{88532}
\saveTG{䉥}{88536}
\saveTG{籜}{88541}
\saveTG{𥷐}{88541}
\saveTG{𦎭}{88542}
\saveTG{𥸓}{88542}
\saveTG{䉗}{88542}
\saveTG{𥴮}{88543}
\saveTG{𥱮}{88544}
\saveTG{𥴂}{88546}
\saveTG{篺}{88546}
\saveTG{簼}{88547}
\saveTG{𥮁}{88547}
\saveTG{𦏭}{88551}
\saveTG{𥯟}{88551}
\saveTG{𥷪}{88551}
\saveTG{𥲷}{88553}
\saveTG{䉝}{88553}
\saveTG{𥬆}{88553}
\saveTG{箨}{88554}
\saveTG{𥬰}{88555}
\saveTG{𦏣}{88556}
\saveTG{𥰏}{88557}
\saveTG{𥲙}{88557}
\saveTG{𥲶}{88558}
\saveTG{𥬐}{88560}
\saveTG{𦏏}{88561}
\saveTG{𥯤}{88561}
\saveTG{簎}{88561}
\saveTG{籒}{88562}
\saveTG{𥮠}{88562}
\saveTG{𦎌}{88562}
\saveTG{籀}{88562}
\saveTG{筘}{88563}
\saveTG{𥭆}{88563}
\saveTG{䉐}{88564}
\saveTG{簵}{88564}
\saveTG{𥭾}{88566}
\saveTG{𥴻}{88567}
\saveTG{𥶪}{88568}
\saveTG{𨏉}{88568}
\saveTG{䉊}{88568}
\saveTG{箝}{88574}
\saveTG{𥳳}{88582}
\saveTG{簐}{88582}
\saveTG{羷}{88586}
\saveTG{籡}{88586}
\saveTG{䍱}{88594}
\saveTG{𥲢}{88600}
\saveTG{𥭒}{88600}
\saveTG{䇱}{88600}
\saveTG{𥭮}{88600}
\saveTG{𥭚}{88601}
\saveTG{簪}{88601}
\saveTG{笘}{88601}
\saveTG{筶}{88601}
\saveTG{答}{88601}
\saveTG{𥰷}{88601}
\saveTG{𥊃}{88601}
\saveTG{𩁰}{88601}
\saveTG{𥸑}{88601}
\saveTG{䁞}{88601}
\saveTG{𥲞}{88601}
\saveTG{𥶚}{88601}
\saveTG{䉢}{88601}
\saveTG{𥰸}{88601}
\saveTG{𥭠}{88601}
\saveTG{𫂓}{88601}
\saveTG{䈍}{88601}
\saveTG{箁}{88601}
\saveTG{箚}{88602}
\saveTG{𥷠}{88602}
\saveTG{𥱾}{88602}
\saveTG{箵}{88602}
\saveTG{笤}{88602}
\saveTG{𥬝}{88602}
\saveTG{𥱵}{88602}
\saveTG{𥰣}{88602}
\saveTG{筨}{88602}
\saveTG{𥵼}{88602}
\saveTG{箘}{88603}
\saveTG{簂}{88603}
\saveTG{箇}{88603}
\saveTG{笞}{88603}
\saveTG{𥶇}{88603}
\saveTG{𥵉}{88603}
\saveTG{𥱽}{88603}
\saveTG{𥴞}{88603}
\saveTG{𥳙}{88603}
\saveTG{𫂆}{88603}
\saveTG{筃}{88603}
\saveTG{笝}{88603}
\saveTG{𧖻}{88603}
\saveTG{𥬥}{88603}
\saveTG{𥵶}{88604}
\saveTG{筈}{88604}
\saveTG{𥰋}{88604}
\saveTG{𥴉}{88604}
\saveTG{𥭛}{88604}
\saveTG{䇴}{88604}
\saveTG{𥭻}{88604}
\saveTG{𥬡}{88604}
\saveTG{𥯾}{88604}
\saveTG{𥶕}{88604}
\saveTG{𥮷}{88604}
\saveTG{箬}{88604}
\saveTG{𥵟}{88604}
\saveTG{䇧}{88604}
\saveTG{箸}{88604}
\saveTG{笿}{88604}
\saveTG{𥮑}{88604}
\saveTG{𥮂}{88605}
\saveTG{𥰶}{88605}
\saveTG{笛}{88605}
\saveTG{筁}{88605}
\saveTG{𣊦}{88606}
\saveTG{𥵊}{88606}
\saveTG{䈞}{88606}
\saveTG{筥}{88606}
\saveTG{簹}{88606}
\saveTG{䇹}{88607}
\saveTG{㫺}{88608}
\saveTG{簮}{88608}
\saveTG{箺}{88608}
\saveTG{䈶}{88608}
\saveTG{𥰻}{88608}
\saveTG{簭}{88608}
\saveTG{𥰫}{88608}
\saveTG{䉒}{88609}
\saveTG{𥯭}{88611}
\saveTG{𥵏}{88614}
\saveTG{𥭅}{88614}
\saveTG{𥴑}{88615}
\saveTG{𥰗}{88617}
\saveTG{𥶱}{88617}
\saveTG{𥯡}{88617}
\saveTG{𩠓}{88617}
\saveTG{𥵽}{88617}
\saveTG{𥱬}{88617}
\saveTG{𥭔}{88620}
\saveTG{𥰵}{88620}
\saveTG{𥳩}{88620}
\saveTG{𥰮}{88621}
\saveTG{𥯩}{88621}
\saveTG{笴}{88621}
\saveTG{𥰯}{88621}
\saveTG{𥳕}{88621}
\saveTG{𥮀}{88627}
\saveTG{𧮰}{88627}
\saveTG{𪱗}{88627}
\saveTG{𧮱}{88627}
\saveTG{𥶏}{88627}
\saveTG{笱}{88627}
\saveTG{𥯋}{88627}
\saveTG{箶}{88627}
\saveTG{𩠸}{88627}
\saveTG{筍}{88627}
\saveTG{笥}{88627}
\saveTG{篰}{88627}
\saveTG{𥮡}{88627}
\saveTG{𥳉}{88627}
\saveTG{𥰄}{88627}
\saveTG{𫂇}{88627}
\saveTG{𠷜}{88627}
\saveTG{篎}{88629}
\saveTG{𩠠}{88631}
\saveTG{䈋}{88632}
\saveTG{𪗲}{88632}
\saveTG{𣌞}{88632}
\saveTG{𥷿}{88634}
\saveTG{敾}{88640}
\saveTG{豃}{88640}
\saveTG{𩠩}{88640}
\saveTG{㪉}{88640}
\saveTG{𫗺}{88640}
\saveTG{𥷯}{88641}
\saveTG{籌}{88641}
\saveTG{𥍅}{88641}
\saveTG{𡀤}{88643}
\saveTG{䇢}{88643}
\saveTG{𥰇}{88643}
\saveTG{𥱯}{88643}
\saveTG{𥵔}{88645}
\saveTG{𥳂}{88647}
\saveTG{𥲹}{88647}
\saveTG{𫂰}{88651}
\saveTG{𥸔}{88651}
\saveTG{𥯒}{88653}
\saveTG{𥸄}{88661}
\saveTG{𡆃}{88661}
\saveTG{𧮵}{88661}
\saveTG{𥱟}{88661}
\saveTG{䈏}{88661}
\saveTG{𥱠}{88661}
\saveTG{𥶲}{88662}
\saveTG{𩠬}{88662}
\saveTG{𥴜}{88662}
\saveTG{谽}{88662}
\saveTG{𧯒}{88666}
\saveTG{䉪}{88666}
\saveTG{𩠴}{88666}
\saveTG{𥰔}{88666}
\saveTG{𥷇}{88668}
\saveTG{𥵗}{88682}
\saveTG{𥷺}{88684}
\saveTG{𥸝}{88686}
\saveTG{𩚆}{88700}
\saveTG{𫓥}{88700}
\saveTG{飤}{88700}
\saveTG{𥱺}{88702}
\saveTG{䬮}{88707}
\saveTG{𩛮}{88707}
\saveTG{𫓩}{88708}
\saveTG{𥭶}{88710}
\saveTG{笀}{88710}
\saveTG{筐}{88711}
\saveTG{筺}{88711}
\saveTG{飵}{88711}
\saveTG{篚}{88711}
\saveTG{𫁹}{88712}
\saveTG{𫗘}{88712}
\saveTG{䇫}{88712}
\saveTG{𥭯}{88712}
\saveTG{䇭}{88712}
\saveTG{𩝭}{88712}
\saveTG{𥬌}{88712}
\saveTG{镒}{88712}
\saveTG{饈}{88712}
\saveTG{笣}{88712}
\saveTG{}{88712}
\saveTG{}{88712}
\saveTG{𩛆}{88712}
\saveTG{篦}{88712}
\saveTG{笓}{88712}
\saveTG{}{88712}
\saveTG{箆}{88712}
\saveTG{竾}{88712}
\saveTG{篹}{88712}
\saveTG{锐}{88712}
\saveTG{箟}{88712}
\saveTG{箞}{88712}
\saveTG{𩞖}{88712}
\saveTG{𫗝}{88712}
\saveTG{𥫥}{88714}
\saveTG{笔}{88714}
\saveTG{铨}{88714}
\saveTG{𪵧}{88714}
\saveTG{锉}{88714}
\saveTG{𥮟}{88715}
\saveTG{𫂑}{88715}
\saveTG{𥫬}{88715}
\saveTG{𥯃}{88715}
\saveTG{𫂌}{88715}
\saveTG{𥫱}{88715}
\saveTG{筪}{88715}
\saveTG{笸}{88716}
\saveTG{䉭}{88716}
\saveTG{𥭺}{88716}
\saveTG{𥱸}{88716}
\saveTG{篭}{88716}
\saveTG{𥲻}{88717}
\saveTG{𨰿}{88717}
\saveTG{䇼}{88717}
\saveTG{𩛹}{88717}
\saveTG{𥳈}{88717}
\saveTG{𥷳}{88717}
\saveTG{餼}{88717}
\saveTG{笹}{88717}
\saveTG{𥭱}{88717}
\saveTG{笆}{88717}
\saveTG{䬣}{88717}
\saveTG{𩜐}{88717}
\saveTG{𩚤}{88717}
\saveTG{𫔃}{88717}
\saveTG{𥰑}{88717}
\saveTG{䉚}{88717}
\saveTG{𥰕}{88717}
\saveTG{䬽}{88717}
\saveTG{𥫟}{88717}
\saveTG{𥯓}{88718}
\saveTG{籄}{88718}
\saveTG{𥭷}{88718}
\saveTG{𥭣}{88718}
\saveTG{篋}{88718}
\saveTG{箧}{88718}
\saveTG{}{88719}
\saveTG{𨱎}{88720}
\saveTG{𩜶}{88720}
\saveTG{籪}{88721}
\saveTG{簖}{88721}
\saveTG{飻}{88722}
\saveTG{𩝧}{88722}
\saveTG{𨱅}{88722}
\saveTG{𥮞}{88724}
\saveTG{𥬓}{88724}
\saveTG{𥰴}{88727}
\saveTG{䬾}{88727}
\saveTG{𥱳}{88727}
\saveTG{𥴝}{88727}
\saveTG{𫂉}{88727}
\saveTG{𥫤}{88727}
\saveTG{𩞃}{88727}
\saveTG{𥬋}{88727}
\saveTG{𥰐}{88727}
\saveTG{}{88727}
\saveTG{𥭢}{88727}
\saveTG{𥯳}{88727}
\saveTG{𥬯}{88727}
\saveTG{𥫡}{88727}
\saveTG{𫓫}{88727}
\saveTG{𩚝}{88727}
\saveTG{節}{88727}
\saveTG{𩟵}{88727}
\saveTG{𥮽}{88727}
\saveTG{𥬜}{88727}
\saveTG{𥬉}{88727}
\saveTG{䈓}{88727}
\saveTG{𩚕}{88727}
\saveTG{锑}{88727}
\saveTG{籂}{88727}
\saveTG{篩}{88727}
\saveTG{笷}{88727}
\saveTG{钤}{88727}
\saveTG{飾}{88727}
\saveTG{飭}{88727}
\saveTG{𥵦}{88727}
\saveTG{䭃}{88730}
\saveTG{𫓻}{88730}
\saveTG{䇗}{88731}
\saveTG{䉙}{88731}
\saveTG{䭐}{88731}
\saveTG{𥱫}{88731}
\saveTG{𥮔}{88731}
\saveTG{𥬔}{88731}
\saveTG{𥬀}{88731}
\saveTG{𥫛}{88731}
\saveTG{𥭉}{88731}
\saveTG{𩝐}{88731}
\saveTG{餻}{88731}
\saveTG{䈘}{88731}
\saveTG{篒}{88732}
\saveTG{簔}{88732}
\saveTG{籑}{88732}
\saveTG{𥲘}{88732}
\saveTG{𩞭}{88732}
\saveTG{𥮲}{88732}
\saveTG{𩛕}{88732}
\saveTG{𥸒}{88732}
\saveTG{𥯜}{88732}
\saveTG{𥱕}{88732}
\saveTG{𩟼}{88732}
\saveTG{䭥}{88732}
\saveTG{𩚹}{88732}
\saveTG{𥶑}{88732}
\saveTG{𥶉}{88732}
\saveTG{䉵}{88732}
\saveTG{䉴}{88732}
\saveTG{镃}{88732}
\saveTG{䍅}{88732}
\saveTG{𥰟}{88732}
\saveTG{𩝌}{88732}
\saveTG{簑}{88732}
\saveTG{铃}{88732}
\saveTG{筤}{88732}
\saveTG{簒}{88732}
\saveTG{篡}{88732}
\saveTG{}{88733}
\saveTG{𩜏}{88733}
\saveTG{𩟳}{88736}
\saveTG{䭑}{88737}
\saveTG{餸}{88738}
\saveTG{𩞥}{88740}
\saveTG{镦}{88740}
\saveTG{敏}{88740}
\saveTG{饊}{88740}
\saveTG{}{88740}
\saveTG{𨱖}{88740}
\saveTG{𢼏}{88740}
\saveTG{㩿}{88740}
\saveTG{䭛}{88740}
\saveTG{餅}{88741}
\saveTG{缾}{88741}
\saveTG{𥫽}{88742}
\saveTG{𥷮}{88743}
\saveTG{𨱔}{88743}
\saveTG{𩛅}{88744}
\saveTG{𩝅}{88744}
\saveTG{𥷓}{88744}
\saveTG{罇}{88746}
\saveTG{𥷣}{88747}
\saveTG{𩜲}{88747}
\saveTG{笢}{88747}
\saveTG{}{88750}
\saveTG{䬺}{88751}
\saveTG{镎}{88752}
\saveTG{篯}{88753}
\saveTG{𩛸}{88757}
\saveTG{}{88761}
\saveTG{𩞰}{88761}
\saveTG{䦅}{88761}
\saveTG{铪}{88761}
\saveTG{镨}{88761}
\saveTG{饍}{88761}
\saveTG{餄}{88761}
\saveTG{𥬦}{88763}
\saveTG{𩜉}{88764}
\saveTG{䭝}{88766}
\saveTG{𥷷}{88766}
\saveTG{𩝞}{88767}
\saveTG{𩛊}{88772}
\saveTG{䈄}{88772}
\saveTG{䈼}{88772}
\saveTG{笜}{88772}
\saveTG{𥮛}{88773}
\saveTG{䇞}{88774}
\saveTG{𥯥}{88777}
\saveTG{䈱}{88777}
\saveTG{管}{88777}
\saveTG{镟}{88781}
\saveTG{𥸞}{88781}
\saveTG{𫓷}{88782}
\saveTG{𥸟}{88782}
\saveTG{𥸛}{88782}
\saveTG{𩞆}{88782}
\saveTG{餩}{88782}
\saveTG{篏}{88782}
\saveTG{笖}{88783}
\saveTG{𩝏}{88784}
\saveTG{镁}{88784}
\saveTG{镞}{88784}
\saveTG{𩟅}{88786}
\saveTG{𩚸}{88790}
\saveTG{𥱘}{88794}
\saveTG{餘}{88794}
\saveTG{𩜣}{88795}
\saveTG{𥫰}{88800}
\saveTG{䈯}{88801}
\saveTG{箕}{88801}
\saveTG{箑}{88801}
\saveTG{簨}{88801}
\saveTG{籅}{88801}
\saveTG{𥬹}{88801}
\saveTG{𥮏}{88801}
\saveTG{𥬺}{88801}
\saveTG{𥯗}{88801}
\saveTG{筼}{88802}
\saveTG{𥭽}{88802}
\saveTG{䇜}{88802}
\saveTG{𥱁}{88802}
\saveTG{𥷼}{88802}
\saveTG{箦}{88802}
\saveTG{}{88802}
\saveTG{篑}{88802}
\saveTG{𥫵}{88803}
\saveTG{𥮅}{88804}
\saveTG{𫂎}{88804}
\saveTG{䉛}{88804}
\saveTG{𥯪}{88804}
\saveTG{𥯫}{88804}
\saveTG{笶}{88804}
\saveTG{𡚅}{88804}
\saveTG{𥯮}{88804}
\saveTG{𥲦}{88804}
\saveTG{笑}{88804}
\saveTG{𫂁}{88804}
\saveTG{筽}{88804}
\saveTG{𥮯}{88804}
\saveTG{𥰥}{88804}
\saveTG{𥏀}{88804}
\saveTG{𥬇}{88804}
\saveTG{𥳢}{88804}
\saveTG{䇦}{88805}
\saveTG{𥬟}{88805}
\saveTG{𥬘}{88805}
\saveTG{篢}{88806}
\saveTG{簧}{88806}
\saveTG{簤}{88806}
\saveTG{𥳹}{88806}
\saveTG{䈿}{88806}
\saveTG{𩖐}{88806}
\saveTG{𫂢}{88806}
\saveTG{䉯}{88806}
\saveTG{𥳡}{88806}
\saveTG{𥴶}{88806}
\saveTG{筫}{88806}
\saveTG{𥱷}{88806}
\saveTG{籫}{88806}
\saveTG{𥶦}{88806}
\saveTG{簀}{88806}
\saveTG{篔}{88806}
\saveTG{簣}{88806}
\saveTG{𥵎}{88806}
\saveTG{𫁶}{88807}
\saveTG{𥴹}{88808}
\saveTG{𥱶}{88808}
\saveTG{𫁽}{88808}
\saveTG{𠉭}{88808}
\saveTG{筴}{88808}
\saveTG{𤒥}{88809}
\saveTG{𥱇}{88809}
\saveTG{𥏺}{88812}
\saveTG{𥷞}{88812}
\saveTG{𥷔}{88812}
\saveTG{矬}{88814}
\saveTG{𥏧}{88814}
\saveTG{䉜}{88815}
\saveTG{𥸁}{88815}
\saveTG{𥲨}{88817}
\saveTG{𥸚}{88817}
\saveTG{䈟}{88820}
\saveTG{𥵲}{88820}
\saveTG{𪿐}{88820}
\saveTG{𥰨}{88820}
\saveTG{𥲣}{88821}
\saveTG{䈕}{88821}
\saveTG{簛}{88821}
\saveTG{𫂣}{88827}
\saveTG{𥭓}{88827}
\saveTG{𥏻}{88827}
\saveTG{𪿏}{88827}
\saveTG{𥰌}{88831}
\saveTG{斂}{88840}
\saveTG{㪘}{88840}
\saveTG{𥎪}{88840}
\saveTG{𥐊}{88840}
\saveTG{𫂩}{88841}
\saveTG{𥯬}{88842}
\saveTG{𥯝}{88843}
\saveTG{簸}{88847}
\saveTG{籢}{88848}
\saveTG{𥳏}{88853}
\saveTG{𥐅}{88861}
\saveTG{𥯌}{88863}
\saveTG{𥳄}{88865}
\saveTG{矰}{88866}
\saveTG{𥏲}{88867}
\saveTG{𠈌}{88880}
\saveTG{籎}{88881}
\saveTG{𥰲}{88881}
\saveTG{𥳽}{88882}
\saveTG{籨}{88882}
\saveTG{𥬈}{88883}
\saveTG{𠑲}{88886}
\saveTG{𥷡}{88886}
\saveTG{𥱣}{88886}
\saveTG{簽}{88886}
\saveTG{𦏇}{88887}
\saveTG{𥶗}{88894}
\saveTG{𥶺}{88894}
\saveTG{𥰂}{88894}
\saveTG{𥷕}{88894}
\saveTG{𥵐}{88896}
\saveTG{𥮵}{88901}
\saveTG{𫂛}{88901}
\saveTG{篻}{88901}
\saveTG{籞}{88901}
\saveTG{策}{88902}
\saveTG{𥬞}{88902}
\saveTG{瀪}{88902}
\saveTG{𥳮}{88903}
\saveTG{繁}{88903}
\saveTG{𥲕}{88903}
\saveTG{𥰼}{88903}
\saveTG{𥱨}{88903}
\saveTG{𦃚}{88903}
\saveTG{𦅪}{88903}
\saveTG{纂}{88903}
\saveTG{𥭍}{88904}
\saveTG{𥴷}{88904}
\saveTG{𥯲}{88904}
\saveTG{䉎}{88904}
\saveTG{𥲧}{88904}
\saveTG{𥰿}{88904}
\saveTG{𥮣}{88904}
\saveTG{𥱋}{88904}
\saveTG{𥮮}{88904}
\saveTG{𥬸}{88904}
\saveTG{𥱧}{88904}
\saveTG{𥸉}{88904}
\saveTG{𥬲}{88904}
\saveTG{𥲒}{88904}
\saveTG{䈎}{88904}
\saveTG{𥣚}{88904}
\saveTG{築}{88904}
\saveTG{筿}{88904}
\saveTG{筡}{88904}
\saveTG{簗}{88904}
\saveTG{篥}{88904}
\saveTG{筞}{88904}
\saveTG{𥸋}{88904}
\saveTG{𥭁}{88904}
\saveTG{𥷃}{88904}
\saveTG{𥲀}{88905}
\saveTG{䇬}{88905}
\saveTG{𥴼}{88905}
\saveTG{𥬎}{88905}
\saveTG{䇿}{88905}
\saveTG{筙}{88905}
\saveTG{𥵺}{88906}
\saveTG{𥶣}{88906}
\saveTG{䈢}{88906}
\saveTG{簝}{88906}
\saveTG{𫂋}{88907}
\saveTG{箂}{88908}
\saveTG{𥭑}{88909}
\saveTG{箓}{88909}
\saveTG{𥮤}{88909}
\saveTG{𥱜}{88911}
\saveTG{𫂊}{88912}
\saveTG{篐}{88912}
\saveTG{䅬}{88912}
\saveTG{𥱝}{88912}
\saveTG{𥯸}{88914}
\saveTG{𥴛}{88914}
\saveTG{𥮊}{88914}
\saveTG{𥴢}{88915}
\saveTG{𥷈}{88915}
\saveTG{𥵾}{88915}
\saveTG{籮}{88915}
\saveTG{𥷩}{88915}
\saveTG{𥵻}{88915}
\saveTG{𥶮}{88915}
\saveTG{䉩}{88916}
\saveTG{𥰜}{88917}
\saveTG{𥭂}{88917}
\saveTG{𥮕}{88917}
\saveTG{𥳵}{88917}
\saveTG{𥶀}{88917}
\saveTG{𫂏}{88917}
\saveTG{𥳛}{88917}
\saveTG{𥴗}{88917}
\saveTG{𥱞}{88917}
\saveTG{𥮹}{88918}
\saveTG{𥰍}{88920}
\saveTG{𥮥}{88921}
\saveTG{𥶠}{88922}
\saveTG{𥶩}{88922}
\saveTG{箣}{88923}
\saveTG{筣}{88923}
\saveTG{𥶖}{88926}
\saveTG{簃}{88927}
\saveTG{箹}{88927}
\saveTG{𥳓}{88927}
\saveTG{䈾}{88927}
\saveTG{𥳼}{88927}
\saveTG{𥯻}{88927}
\saveTG{𥵰}{88927}
\saveTG{䈫}{88927}
\saveTG{𥴭}{88927}
\saveTG{𥶤}{88927}
\saveTG{𥯆}{88931}
\saveTG{䉂}{88931}
\saveTG{𥸕}{88931}
\saveTG{𥶥}{88932}
\saveTG{𥵅}{88932}
\saveTG{䈺}{88933}
\saveTG{𥣏}{88933}
\saveTG{𥵮}{88938}
\saveTG{敘}{88940}
\saveTG{𥭰}{88941}
\saveTG{𥲛}{88943}
\saveTG{𥱦}{88943}
\saveTG{䈖}{88943}
\saveTG{𥮧}{88943}
\saveTG{䈙}{88943}
\saveTG{𥴟}{88944}
\saveTG{𥴖}{88945}
\saveTG{𥴙}{88946}
\saveTG{𥵤}{88946}
\saveTG{𥸃}{88947}
\saveTG{𥵠}{88947}
\saveTG{𥷆}{88947}
\saveTG{𥵡}{88947}
\saveTG{𥶵}{88947}
\saveTG{𥯈}{88947}
\saveTG{𥶍}{88947}
\saveTG{𥴌}{88948}
\saveTG{𫂙}{88948}
\saveTG{𥰡}{88948}
\saveTG{𥯍}{88948}
\saveTG{䇲}{88952}
\saveTG{𥴈}{88953}
\saveTG{𥴍}{88953}
\saveTG{𫂚}{88954}
\saveTG{𥶳}{88956}
\saveTG{䅾}{88957}
\saveTG{𥴇}{88958}
\saveTG{𥲐}{88961}
\saveTG{𥳯}{88961}
\saveTG{𥳖}{88961}
\saveTG{籍}{88961}
\saveTG{𥲴}{88962}
\saveTG{籕}{88962}
\saveTG{箱}{88963}
\saveTG{𥴅}{88964}
\saveTG{𥱛}{88964}
\saveTG{𫂘}{88964}
\saveTG{𥯚}{88964}
\saveTG{䈷}{88964}
\saveTG{𥱓}{88964}
\saveTG{𥷸}{88968}
\saveTG{䈤}{88974}
\saveTG{簯}{88981}
\saveTG{籁}{88982}
\saveTG{簌}{88982}
\saveTG{䇣}{88983}
\saveTG{筷}{88985}
\saveTG{籟}{88986}
\saveTG{䈐}{88989}
\saveTG{篍}{88989}
\saveTG{䉘}{88991}
\saveTG{𥵹}{88991}
\saveTG{箖}{88994}
\saveTG{𫂡}{88995}
\saveTG{𫂧}{88995}
\saveTG{䉓}{88995}
\saveTG{䉫}{88998}
\saveTG{𫂐}{88998}
\saveTG{𥶾}{88999}
\saveTG{𥷗}{88999}
\saveTG{釥}{89100}
\saveTG{銧}{89112}
\saveTG{錈}{89112}
\saveTG{鎲}{89112}
\saveTG{𨯗}{89114}
\saveTG{鏜}{89114}
\saveTG{𨫠}{89115}
\saveTG{𨬥}{89117}
\saveTG{𨭊}{89117}
\saveTG{𨫺}{89120}
\saveTG{䤬}{89120}
\saveTG{鈔}{89120}
\saveTG{𨦒}{89127}
\saveTG{鋿}{89127}
\saveTG{鏛}{89127}
\saveTG{鐒}{89127}
\saveTG{銷}{89127}
\saveTG{𫒥}{89131}
\saveTG{钂}{89131}
\saveTG{𨬂}{89138}
\saveTG{𨮽}{89142}
\saveTG{𨭃}{89142}
\saveTG{𨪍}{89144}
\saveTG{𨩐}{89144}
\saveTG{𫒽}{89148}
\saveTG{鉡}{89150}
\saveTG{鐣}{89152}
\saveTG{鏻}{89159}
\saveTG{𨩠}{89162}
\saveTG{鐺}{89166}
\saveTG{鍬}{89180}
\saveTG{鈥}{89180}
\saveTG{𨧛}{89184}
\saveTG{鎖}{89186}
\saveTG{鑜}{89186}
\saveTG{錟}{89189}
\saveTG{𨨚}{89189}
\saveTG{鏿}{89194}
\saveTG{鑅}{89194}
\saveTG{銤}{89194}
\saveTG{𠑜}{89212}
\saveTG{㝺}{89220}
\saveTG{𤢯}{89257}
\saveTG{𥽡}{89281}
\saveTG{𡭿}{89397}
\saveTG{𠵵}{89650}
\saveTG{㷕}{89680}
\saveTG{𧯈}{89696}
\saveTG{𧯊}{89696}
\saveTG{锩}{89712}
\saveTG{镋}{89712}
\saveTG{镗}{89714}
\saveTG{饄}{89714}
\saveTG{𩜇}{89717}
\saveTG{钞}{89720}
\saveTG{𩚙}{89720}
\saveTG{𩜋}{89726}
\saveTG{销}{89727}
\saveTG{𩛱}{89727}
\saveTG{𦉘}{89742}
\saveTG{镂}{89744}
\saveTG{𣫷}{89750}
\saveTG{䬳}{89750}
\saveTG{𩞦}{89752}
\saveTG{罉}{89752}
\saveTG{𩞻}{89757}
\saveTG{}{89759}
\saveTG{𩟈}{89766}
\saveTG{铛}{89777}
\saveTG{钬}{89780}
\saveTG{锹}{89780}
\saveTG{𩝋}{89780}
\saveTG{𦈦}{89780}
\saveTG{锁}{89782}
\saveTG{锬}{89789}
\saveTG{餤}{89789}
\saveTG{饓}{89794}
\saveTG{𡮜}{89800}
\saveTG{𡮔}{89800}
\saveTG{𥏙}{89817}
\saveTG{𪿉}{89820}
\saveTG{𥏟}{89827}
\saveTG{𧶈}{89827}
\saveTG{𥐆}{89850}
\saveTG{𥢷}{89966}
\saveTG{𡭔}{90000}
\saveTG{}{90000}
\saveTG{忄}{90000}
\saveTG{小}{90000}
\saveTG{龸}{90000}
\saveTG{龹}{90008}
\saveTG{忙}{90010}
\saveTG{戂}{90011}
\saveTG{𢡒}{90012}
\saveTG{𢛛}{90014}
\saveTG{𪫬}{90014}
\saveTG{𢟴}{90015}
\saveTG{憧}{90015}
\saveTG{𢢓}{90015}
\saveTG{㦃}{90015}
\saveTG{𢥪}{90015}
\saveTG{𢦄}{90015}
\saveTG{惟}{90015}
\saveTG{憻}{90016}
\saveTG{㤝}{90017}
\saveTG{𢝋}{90017}
\saveTG{𢖪}{90017}
\saveTG{㤺}{90017}
\saveTG{㦇}{90017}
\saveTG{忼}{90017}
\saveTG{𢘮}{90018}
\saveTG{𢝜}{90021}
\saveTG{𢞆}{90022}
\saveTG{懠}{90023}
\saveTG{𢥖}{90023}
\saveTG{𢟚}{90027}
\saveTG{𥻿}{90027}
\saveTG{𢟢}{90027}
\saveTG{慵}{90027}
\saveTG{懏}{90027}
\saveTG{弮}{90027}
\saveTG{㤄}{90027}
\saveTG{㥬}{90027}
\saveTG{𢘥}{90027}
\saveTG{𢞟}{90027}
\saveTG{㤃}{90027}
\saveTG{𢢘}{90027}
\saveTG{㥔}{90027}
\saveTG{悙}{90027}
\saveTG{𢙕}{90030}
\saveTG{忭}{90030}
\saveTG{𢚚}{90031}
\saveTG{𢜚}{90031}
\saveTG{憔}{90031}
\saveTG{𪬹}{90031}
\saveTG{𢣗}{90031}
\saveTG{𢙇}{90032}
\saveTG{𢟰}{90032}
\saveTG{𢡏}{90032}
\saveTG{𢜺}{90032}
\saveTG{懹}{90032}
\saveTG{怰}{90032}
\saveTG{懡}{90032}
\saveTG{惤}{90032}
\saveTG{懷}{90032}
\saveTG{𢟦}{90032}
\saveTG{懐}{90032}
\saveTG{𢠫}{90034}
\saveTG{憶}{90036}
\saveTG{𪬸}{90037}
\saveTG{忟}{90040}
\saveTG{𢤈}{90041}
\saveTG{𢡐}{90041}
\saveTG{𢟳}{90043}
\saveTG{𢜙}{90043}
\saveTG{𢙗}{90044}
\saveTG{慞}{90046}
\saveTG{𢥵}{90047}
\saveTG{𢜬}{90047}
\saveTG{𢥠}{90047}
\saveTG{愯}{90047}
\saveTG{惇}{90047}
\saveTG{𢚙}{90047}
\saveTG{恔}{90048}
\saveTG{悴}{90048}
\saveTG{𢣃}{90048}
\saveTG{𢣾}{90052}
\saveTG{惦}{90061}
\saveTG{㥉}{90061}
\saveTG{愔}{90061}
\saveTG{𢚘}{90061}
\saveTG{𡭵}{90061}
\saveTG{慉}{90063}
\saveTG{悋}{90064}
\saveTG{㥫}{90066}
\saveTG{恼}{90072}
\saveTG{㤥}{90082}
\saveTG{愱}{90084}
\saveTG{𪬜}{90084}
\saveTG{懭}{90086}
\saveTG{𢡝}{90086}
\saveTG{懔}{90091}
\saveTG{𢤭}{90094}
\saveTG{𢛐}{90094}
\saveTG{𢥐}{90094}
\saveTG{懍}{90094}
\saveTG{𢠩}{90094}
\saveTG{𢥄}{90094}
\saveTG{惊}{90096}
\saveTG{慷}{90099}
\saveTG{𤆜}{90101}
\saveTG{𡙅}{90102}
\saveTG{𥁠}{90102}
\saveTG{𥁸}{90102}
\saveTG{𤇙}{90102}
\saveTG{𣥺}{90102}
\saveTG{𥻎}{90104}
\saveTG{𪢾}{90104}
\saveTG{堂}{90104}
\saveTG{㷑}{90104}
\saveTG{坣}{90104}
\saveTG{𡓪}{90104}
\saveTG{𡔨}{90104}
\saveTG{𤍢}{90104}
\saveTG{𡭤}{90104}
\saveTG{尘}{90104}
\saveTG{烾}{90104}
\saveTG{𥹿}{90104}
\saveTG{𡮝}{90105}
\saveTG{鲎}{90106}
\saveTG{𤇁}{90106}
\saveTG{鲞}{90106}
\saveTG{𤆂}{90108}
\saveTG{𧯦}{90108}
\saveTG{𩖟}{90110}
\saveTG{𡮛}{90114}
\saveTG{𩀙}{90115}
\saveTG{𪚯}{90117}
\saveTG{𩁫}{90125}
\saveTG{鸴}{90127}
\saveTG{𡭬}{90127}
\saveTG{𡮣}{90127}
\saveTG{𧈨}{90135}
\saveTG{䖭}{90136}
\saveTG{𧒾}{90136}
\saveTG{蛍}{90136}
\saveTG{𤍷}{90168}
\saveTG{𢑐}{90174}
\saveTG{当}{90177}
\saveTG{少}{90200}
\saveTG{㗬}{90207}
\saveTG{𥹆}{90211}
\saveTG{覚}{90212}
\saveTG{灮}{90212}
\saveTG{光}{90212}
\saveTG{覍}{90212}
\saveTG{麊}{90212}
\saveTG{觉}{90212}
\saveTG{党}{90212}
\saveTG{炛}{90214}
\saveTG{雀}{90215}
\saveTG{𩀯}{90215}
\saveTG{𨿰}{90215}
\saveTG{㝹}{90217}
\saveTG{𡮄}{90217}
\saveTG{鼡}{90217}
\saveTG{𫜏}{90217}
\saveTG{𡭪}{90217}
\saveTG{𠓉}{90217}
\saveTG{㝸}{90217}
\saveTG{兤}{90218}
\saveTG{𡭙}{90221}
\saveTG{券}{90227}
\saveTG{帣}{90227}
\saveTG{尚}{90227}
\saveTG{𦙊}{90227}
\saveTG{𥼇}{90227}
\saveTG{𠝂}{90227}
\saveTG{𠓌}{90227}
\saveTG{𡭮}{90227}
\saveTG{𡖹}{90227}
\saveTG{𩱙}{90227}
\saveTG{常}{90227}
\saveTG{𡮆}{90227}
\saveTG{𡭺}{90227}
\saveTG{㡀}{90227}
\saveTG{𡭸}{90227}
\saveTG{𠒿}{90227}
\saveTG{𫆙}{90227}
\saveTG{尙}{90227}
\saveTG{肖}{90227}
\saveTG{觠}{90227}
\saveTG{奍}{90228}
\saveTG{𡮏}{90229}
\saveTG{𡭷}{90232}
\saveTG{𡗑}{90232}
\saveTG{𣲱}{90232}
\saveTG{豢}{90232}
\saveTG{𡭦}{90234}
\saveTG{𣁣}{90240}
\saveTG{牚}{90241}
\saveTG{𠦖}{90241}
\saveTG{𡮌}{90247}
\saveTG{厳}{90248}
\saveTG{㷠}{90257}
\saveTG{粦}{90259}
\saveTG{𣇃}{90268}
\saveTG{𡭣}{90290}
\saveTG{𠒥}{90294}
\saveTG{尐}{90300}
\saveTG{𡮯}{90306}
\saveTG{鴬}{90327}
\saveTG{䲵}{90327}
\saveTG{𩾟}{90327}
\saveTG{𪅓}{90327}
\saveTG{𪀲}{90327}
\saveTG{𪂟}{90327}
\saveTG{駦}{90327}
\saveTG{}{90327}
\saveTG{𢖿}{90330}
\saveTG{𢤤}{90331}
\saveTG{𩏸}{90331}
\saveTG{黨}{90331}
\saveTG{𤍳}{90331}
\saveTG{㦂}{90332}
\saveTG{𢡭}{90332}
\saveTG{𢘤}{90334}
\saveTG{𤏂}{90334}
\saveTG{𢠴}{90335}
\saveTG{鮝}{90336}
\saveTG{𩵮}{90336}
\saveTG{㥕}{90338}
\saveTG{𢘻}{90339}
\saveTG{𪹦}{90339}
\saveTG{𢜡}{90344}
\saveTG{𣁂}{90400}
\saveTG{𥹢}{90401}
\saveTG{𦕉}{90401}
\saveTG{娄}{90404}
\saveTG{𡭼}{90405}
\saveTG{𧄣}{90406}
\saveTG{𤌸}{90406}
\saveTG{𤋬}{90407}
\saveTG{学}{90407}
\saveTG{𤆌}{90407}
\saveTG{𤈬}{90407}
\saveTG{𠦂}{90409}
\saveTG{𡭾}{90417}
\saveTG{𡮑}{90417}
\saveTG{㶢}{90417}
\saveTG{労}{90427}
\saveTG{𠢸}{90427}
\saveTG{劣}{90427}
\saveTG{劵}{90427}
\saveTG{𥾀}{90431}
\saveTG{𡭛}{90440}
\saveTG{𡜈}{90441}
\saveTG{𢍕}{90441}
\saveTG{𥸶}{90441}
\saveTG{𡜽}{90442}
\saveTG{𡝷}{90442}
\saveTG{𢍨}{90443}
\saveTG{𥹻}{90443}
\saveTG{𡜛}{90446}
\saveTG{𡮁}{90446}
\saveTG{𡝣}{90446}
\saveTG{𥸻}{90447}
\saveTG{𥻰}{90447}
\saveTG{𤍚}{90447}
\saveTG{𥻔}{90447}
\saveTG{𢍊}{90448}
\saveTG{灷}{90448}
\saveTG{𥸭}{90449}
\saveTG{単}{90500}
\saveTG{半}{90500}
\saveTG{拳}{90502}
\saveTG{掌}{90502}
\saveTG{挙}{90502}
\saveTG{𨎖}{90506}
\saveTG{韏}{90506}
\saveTG{𡮕}{90506}
\saveTG{単}{90506}
\saveTG{𡭭}{90506}
\saveTG{牶}{90508}
\saveTG{举}{90508}
\saveTG{𨚚}{90517}
\saveTG{𢆤}{90518}
\saveTG{𪟸}{90527}
\saveTG{𫃗}{90531}
\saveTG{𢦢}{90531}
\saveTG{𪨃}{90548}
\saveTG{𡮍}{90572}
\saveTG{𥼈}{90572}
\saveTG{𩏶}{90589}
\saveTG{畄}{90600}
\saveTG{𥃻}{90600}
\saveTG{𥅷}{90600}
\saveTG{𡮢}{90600}
\saveTG{𤎭}{90600}
\saveTG{誊}{90601}
\saveTG{誉}{90601}
\saveTG{𣅱}{90601}
\saveTG{喾}{90601}
\saveTG{嘗}{90601}
\saveTG{𣆤}{90601}
\saveTG{𠹉}{90601}
\saveTG{𪾻}{90601}
\saveTG{𧨲}{90601}
\saveTG{𫃅}{90601}
\saveTG{𪿝}{90602}
\saveTG{省}{90602}
\saveTG{𩉅}{90602}
\saveTG{}{90602}
\saveTG{𡮀}{90602}
\saveTG{𨡔}{90604}
\saveTG{𣋈}{90604}
\saveTG{営}{90606}
\saveTG{當}{90606}
\saveTG{𤌻}{90606}
\saveTG{𥋤}{90607}
\saveTG{𤱵}{90608}
\saveTG{眷}{90608}
\saveTG{畨}{90609}
\saveTG{𫂵}{90609}
\saveTG{㕾}{90617}
\saveTG{𤒀}{90617}
\saveTG{𩏷}{90631}
\saveTG{𡮲}{90643}
\saveTG{𠤠}{90711}
\saveTG{卷}{90712}
\saveTG{𡭠}{90712}
\saveTG{𠤝}{90712}
\saveTG{毟}{90714}
\saveTG{𣭮}{90715}
\saveTG{𦈷}{90717}
\saveTG{瓽}{90717}
\saveTG{巻}{90717}
\saveTG{𨚒}{90717}
\saveTG{㐘}{90719}
\saveTG{𡮊}{90727}
\saveTG{餋}{90732}
\saveTG{裳}{90732}
\saveTG{尝}{90732}
\saveTG{㷃}{90732}
\saveTG{𧙯}{90732}
\saveTG{飬}{90732}
\saveTG{𧚵}{90732}
\saveTG{𤆟}{90732}
\saveTG{𣫶}{90757}
\saveTG{𡷀}{90772}
\saveTG{𡭲}{90772}
\saveTG{齤}{90772}
\saveTG{峃}{90772}
\saveTG{𪨤}{90772}
\saveTG{𤆽}{90772}
\saveTG{甞}{90774}
\saveTG{𦈹}{90774}
\saveTG{火}{90800}
\saveTG{𤆓}{90800}
\saveTG{烡}{90801}
\saveTG{糞}{90801}
\saveTG{粪}{90801}
\saveTG{兴}{90801}
\saveTG{𥻪}{90801}
\saveTG{赏}{90802}
\saveTG{𪸖}{90802}
\saveTG{䟫}{90802}
\saveTG{尖}{90804}
\saveTG{𥏇}{90804}
\saveTG{𤊖}{90804}
\saveTG{𥻜}{90804}
\saveTG{𡭟}{90804}
\saveTG{𡗩}{90804}
\saveTG{类}{90804}
\saveTG{𡙪}{90805}
\saveTG{}{90805}
\saveTG{𠔉}{90805}
\saveTG{賞}{90806}
\saveTG{𧴪}{90806}
\saveTG{𪹝}{90806}
\saveTG{𧶦}{90806}
\saveTG{黉}{90806}
\saveTG{𡭩}{90807}
\saveTG{𤐥}{90809}
\saveTG{𤎫}{90809}
\saveTG{𤇑}{90809}
\saveTG{炎}{90809}
\saveTG{𥸪}{90809}
\saveTG{𤑋}{90809}
\saveTG{𡭞}{90809}
\saveTG{𤓒}{90811}
\saveTG{熝}{90812}
\saveTG{𤐷}{90814}
\saveTG{𤍣}{90814}
\saveTG{㸀}{90814}
\saveTG{𪸱}{90814}
\saveTG{炷}{90814}
\saveTG{𤒲}{90814}
\saveTG{燑}{90815}
\saveTG{焳}{90815}
\saveTG{𪨅}{90815}
\saveTG{𪹰}{90815}
\saveTG{𪹠}{90815}
\saveTG{𪺇}{90815}
\saveTG{炕}{90817}
\saveTG{𤏶}{90817}
\saveTG{煷}{90817}
\saveTG{𤇥}{90818}
\saveTG{㷚}{90821}
\saveTG{㸄}{90823}
\saveTG{𤒱}{90823}
\saveTG{𤈽}{90827}
\saveTG{熇}{90827}
\saveTG{熵}{90827}
\saveTG{焴}{90827}
\saveTG{㷰}{90827}
\saveTG{𪹧}{90827}
\saveTG{𪹚}{90827}
\saveTG{㷪}{90827}
\saveTG{𤎱}{90827}
\saveTG{炞}{90830}
\saveTG{爊}{90831}
\saveTG{燋}{90831}
\saveTG{炫}{90832}
\saveTG{𪨆}{90832}
\saveTG{𤐶}{90832}
\saveTG{爙}{90832}
\saveTG{燱}{90836}
\saveTG{熫}{90837}
\saveTG{燫}{90837}
\saveTG{𤑦}{90837}
\saveTG{𤒄}{90837}
\saveTG{𤑙}{90837}
\saveTG{𥸹}{90840}
\saveTG{炆}{90840}
\saveTG{𤑴}{90841}
\saveTG{𤈼}{90841}
\saveTG{𤍤}{90846}
\saveTG{焲}{90847}
\saveTG{𤊰}{90847}
\saveTG{焞}{90847}
\saveTG{烄}{90848}
\saveTG{焠}{90848}
\saveTG{𤋾}{90851}
\saveTG{𪹜}{90856}
\saveTG{𤉘}{90861}
\saveTG{焙}{90861}
\saveTG{𡮷}{90864}
\saveTG{煻}{90865}
\saveTG{𤉮}{90869}
\saveTG{烗}{90882}
\saveTG{𡮪}{90882}
\saveTG{𤌿}{90884}
\saveTG{㷜}{90884}
\saveTG{𤋋}{90884}
\saveTG{爌}{90886}
\saveTG{焿}{90887}
\saveTG{焱}{90889}
\saveTG{𤊭}{90891}
\saveTG{燷}{90891}
\saveTG{𤉜}{90894}
\saveTG{𪹯}{90894}
\saveTG{𤒢}{90894}
\saveTG{𤉬}{90894}
\saveTG{燺}{90894}
\saveTG{燣}{90894}
\saveTG{𤎖}{90899}
\saveTG{}{90900}
\saveTG{氺}{90900}
\saveTG{䄅}{90901}
\saveTG{泶}{90902}
\saveTG{𡭝}{90902}
\saveTG{絭}{90903}
\saveTG{米}{90904}
\saveTG{燊}{90904}
\saveTG{桊}{90904}
\saveTG{巣}{90904}
\saveTG{𣏦}{90904}
\saveTG{𥽒}{90904}
\saveTG{𣙂}{90904}
\saveTG{䅈}{90904}
\saveTG{𡭢}{90904}
\saveTG{栄}{90904}
\saveTG{棠}{90904}
\saveTG{𪴍}{90904}
\saveTG{𠔗}{90905}
\saveTG{𥼢}{90905}
\saveTG{𡭽}{90906}
\saveTG{𡭴}{90906}
\saveTG{𡮂}{90906}
\saveTG{尜}{90908}
\saveTG{𣳾}{90909}
\saveTG{𡭯}{90909}
\saveTG{䊳}{90911}
\saveTG{䊌}{90912}
\saveTG{𫃎}{90912}
\saveTG{糡}{90912}
\saveTG{𥽄}{90912}
\saveTG{𥺁}{90914}
\saveTG{粧}{90914}
\saveTG{䊒}{90915}
\saveTG{𥼷}{90916}
\saveTG{粇}{90917}
\saveTG{𥹷}{90917}
\saveTG{粒}{90918}
\saveTG{𥻚}{90926}
\saveTG{𥺞}{90927}
\saveTG{𫃍}{90927}
\saveTG{䊞}{90927}
\saveTG{𥺪}{90927}
\saveTG{𥻭}{90927}
\saveTG{𥼚}{90931}
\saveTG{𥽓}{90931}
\saveTG{𥽬}{90932}
\saveTG{𥼌}{90934}
\saveTG{𤐙}{90941}
\saveTG{𥼁}{90943}
\saveTG{𥼊}{90945}
\saveTG{𥽅}{90945}
\saveTG{𥽝}{90947}
\saveTG{𡮋}{90948}
\saveTG{粹}{90948}
\saveTG{𥹜}{90948}
\saveTG{𥽨}{90952}
\saveTG{𥺦}{90962}
\saveTG{𥺨}{90964}
\saveTG{糖}{90965}
\saveTG{𥺧}{90985}
\saveTG{䊯}{90986}
\saveTG{𨤡}{90986}
\saveTG{尛}{90990}
\saveTG{𥽰}{90994}
\saveTG{𤎎}{90994}
\saveTG{糠}{90999}
\saveTG{怔}{91011}
\saveTG{恇}{91011}
\saveTG{悱}{91011}
\saveTG{𢤱}{91011}
\saveTG{𢝎}{91011}
\saveTG{𢗝}{91012}
\saveTG{怃}{91012}
\saveTG{𢞃}{91012}
\saveTG{𢛟}{91012}
\saveTG{㥱}{91012}
\saveTG{𪭃}{91012}
\saveTG{𪫣}{91012}
\saveTG{𢙼}{91012}
\saveTG{㦭}{91012}
\saveTG{𢝙}{91012}
\saveTG{忨}{91012}
\saveTG{慨}{91012}
\saveTG{𢖷}{91012}
\saveTG{恆}{91012}
\saveTG{𢤩}{91012}
\saveTG{𢥈}{91012}
\saveTG{憈}{91012}
\saveTG{𢚬}{91014}
\saveTG{𢚯}{91014}
\saveTG{𢛢}{91014}
\saveTG{𪬉}{91014}
\saveTG{𢛄}{91014}
\saveTG{𢟵}{91014}
\saveTG{愝}{91014}
\saveTG{怄}{91014}
\saveTG{忹}{91014}
\saveTG{恎}{91014}
\saveTG{慪}{91016}
\saveTG{㥾}{91016}
\saveTG{恒}{91016}
\saveTG{怇}{91017}
\saveTG{𢜜}{91017}
\saveTG{𢥬}{91017}
\saveTG{𢣒}{91017}
\saveTG{㤴}{91017}
\saveTG{惬}{91018}
\saveTG{愜}{91018}
\saveTG{𪫯}{91018}
\saveTG{㤱}{91018}
\saveTG{怌}{91019}
\saveTG{忊}{91020}
\saveTG{𢥳}{91021}
\saveTG{㤚}{91021}
\saveTG{𪫧}{91026}
\saveTG{𪫡}{91027}
\saveTG{𢡍}{91027}
\saveTG{𢣚}{91027}
\saveTG{怲}{91027}
\saveTG{懦}{91027}
\saveTG{𢝒}{91027}
\saveTG{𢟡}{91027}
\saveTG{𢗔}{91027}
\saveTG{㥼}{91027}
\saveTG{𢡼}{91027}
\saveTG{𢥻}{91027}
\saveTG{𪫢}{91027}
\saveTG{𢣧}{91027}
\saveTG{𢗙}{91027}
\saveTG{𪬰}{91027}
\saveTG{𢗚}{91027}
\saveTG{𢗃}{91027}
\saveTG{𢚽}{91027}
\saveTG{𢗶}{91027}
\saveTG{𢢛}{91027}
\saveTG{𢡤}{91027}
\saveTG{𪫨}{91028}
\saveTG{𢗄}{91030}
\saveTG{𢡡}{91031}
\saveTG{𢠟}{91032}
\saveTG{忶}{91032}
\saveTG{懅}{91032}
\saveTG{𢤟}{91032}
\saveTG{悵}{91032}
\saveTG{𢡊}{91038}
\saveTG{悿}{91038}
\saveTG{𢟑}{91040}
\saveTG{𢖳}{91040}
\saveTG{忓}{91040}
\saveTG{𢙘}{91041}
\saveTG{懾}{91041}
\saveTG{𢣿}{91042}
\saveTG{㥝}{91042}
\saveTG{㤉}{91042}
\saveTG{𢟹}{91043}
\saveTG{𢣵}{91043}
\saveTG{𢙱}{91044}
\saveTG{𢞅}{91044}
\saveTG{𢝶}{91044}
\saveTG{𢟶}{91045}
\saveTG{𢙾}{91046}
\saveTG{憛}{91046}
\saveTG{悼}{91046}
\saveTG{慑}{91047}
\saveTG{懮}{91047}
\saveTG{𢝠}{91047}
\saveTG{𢥝}{91047}
\saveTG{㦆}{91049}
\saveTG{怦}{91049}
\saveTG{𢟟}{91053}
\saveTG{𢟧}{91060}
\saveTG{怗}{91060}
\saveTG{𢘆}{91060}
\saveTG{𪬕}{91061}
\saveTG{𢤓}{91061}
\saveTG{憯}{91061}
\saveTG{𢣸}{91061}
\saveTG{悟}{91061}
\saveTG{𪫮}{91062}
\saveTG{𢤡}{91062}
\saveTG{愐}{91062}
\saveTG{恓}{91064}
\saveTG{𢙖}{91064}
\saveTG{愊}{91066}
\saveTG{憯}{91067}
\saveTG{㤳}{91069}
\saveTG{𢛯}{91072}
\saveTG{𢞹}{91084}
\saveTG{恹}{91084}
\saveTG{愞}{91084}
\saveTG{懨}{91084}
\saveTG{愩}{91086}
\saveTG{𪭄}{91086}
\saveTG{𢠷}{91086}
\saveTG{㥧}{91086}
\saveTG{怀}{91090}
\saveTG{慓}{91091}
\saveTG{𢡅}{91094}
\saveTG{慄}{91094}
\saveTG{憟}{91094}
\saveTG{𢢙}{91096}
\saveTG{㥳}{91096}
\saveTG{𡔇}{91104}
\saveTG{𧇺}{91212}
\saveTG{𧇲}{91212}
\saveTG{𠒝}{91212}
\saveTG{甐}{91217}
\saveTG{𧇳}{91217}
\saveTG{𩱸}{91227}
\saveTG{𢿦}{91247}
\saveTG{𢼼}{91247}
\saveTG{㪁}{91247}
\saveTG{𢼯}{91247}
\saveTG{𩒚}{91286}
\saveTG{𩕶}{91286}
\saveTG{𩕔}{91286}
\saveTG{𩔫}{91286}
\saveTG{𩓝}{91286}
\saveTG{𪅒}{91327}
\saveTG{𩧔}{91327}
\saveTG{𢝦}{91332}
\saveTG{𢠕}{91336}
\saveTG{𢝴}{91336}
\saveTG{𢛈}{91336}
\saveTG{𢣁}{91338}
\saveTG{㽊}{91417}
\saveTG{𩔗}{91428}
\saveTG{𢽾}{91447}
\saveTG{𩔬}{91486}
\saveTG{𩓍}{91486}
\saveTG{𩒪}{91486}
\saveTG{𢰨}{91502}
\saveTG{頖}{91586}
\saveTG{㽆}{91617}
\saveTG{㼳}{91617}
\saveTG{𩖋}{91686}
\saveTG{顲}{91686}
\saveTG{𣀏}{91747}
\saveTG{𨆜}{91801}
\saveTG{爏}{91811}
\saveTG{爖}{91811}
\saveTG{炡}{91811}
\saveTG{烴}{91812}
\saveTG{灴}{91812}
\saveTG{𤐊}{91812}
\saveTG{𪸔}{91812}
\saveTG{𤎠}{91812}
\saveTG{烥}{91812}
\saveTG{𪹣}{91812}
\saveTG{}{91812}
\saveTG{爐}{91812}
\saveTG{煙}{91814}
\saveTG{}{91814}
\saveTG{𤆦}{91814}
\saveTG{𪸛}{91814}
\saveTG{𪸽}{91814}
\saveTG{𤈜}{91814}
\saveTG{𤎆}{91814}
\saveTG{㸌}{91815}
\saveTG{𤏚}{91815}
\saveTG{𤉁}{91815}
\saveTG{𤐀}{91816}
\saveTG{烜}{91816}
\saveTG{𤎐}{91816}
\saveTG{熰}{91816}
\saveTG{𤬬}{91817}
\saveTG{𤐇}{91817}
\saveTG{炬}{91817}
\saveTG{𤮶}{91817}
\saveTG{𤓧}{91817}
\saveTG{𤍆}{91817}
\saveTG{𪸑}{91817}
\saveTG{𪸒}{91817}
\saveTG{爧}{91818}
\saveTG{𤇨}{91819}
\saveTG{炣}{91820}
\saveTG{灯}{91820}
\saveTG{烆}{91821}
\saveTG{𤏙}{91821}
\saveTG{𤓠}{91826}
\saveTG{𤎄}{91827}
\saveTG{𤌦}{91827}
\saveTG{𤈃}{91827}
\saveTG{燸}{91827}
\saveTG{𦓒}{91827}
\saveTG{爄}{91827}
\saveTG{炳}{91827}
\saveTG{𤆏}{91827}
\saveTG{𤑣}{91827}
\saveTG{𤌬}{91827}
\saveTG{𤇤}{91827}
\saveTG{𤇃}{91827}
\saveTG{𤊬}{91831}
\saveTG{𪹪}{91831}
\saveTG{烼}{91832}
\saveTG{㶹}{91832}
\saveTG{㷾}{91832}
\saveTG{𤍭}{91832}
\saveTG{燯}{91832}
\saveTG{𤊞}{91832}
\saveTG{𪹄}{91832}
\saveTG{爈}{91836}
\saveTG{㶥}{91840}
\saveTG{𪸥}{91841}
\saveTG{㸎}{91842}
\saveTG{𤆹}{91842}
\saveTG{𤍇}{91843}
\saveTG{𤊠}{91844}
\saveTG{𤈆}{91846}
\saveTG{焯}{91846}
\saveTG{燂}{91846}
\saveTG{𪸫}{91846}
\saveTG{𤌤}{91847}
\saveTG{𤆝}{91847}
\saveTG{㷞}{91847}
\saveTG{𣀲}{91847}
\saveTG{𣀰}{91847}
\saveTG{𪺈}{91847}
\saveTG{敥}{91847}
\saveTG{𤇊}{91849}
\saveTG{𤎙}{91856}
\saveTG{炶}{91860}
\saveTG{熸}{91861}
\saveTG{焐}{91861}
\saveTG{𪹓}{91861}
\saveTG{𤑭}{91862}
\saveTG{𤋛}{91862}
\saveTG{炻}{91862}
\saveTG{煔}{91862}
\saveTG{𨟽}{91864}
\saveTG{𤈇}{91864}
\saveTG{𤌜}{91864}
\saveTG{𤐝}{91864}
\saveTG{煏}{91866}
\saveTG{㶺}{91868}
\saveTG{𤊿}{91877}
\saveTG{烦}{91882}
\saveTG{𩖖}{91882}
\saveTG{煗}{91884}
\saveTG{𩕈}{91886}
\saveTG{煩}{91886}
\saveTG{熕}{91886}
\saveTG{𤋺}{91886}
\saveTG{顃}{91886}
\saveTG{𤒟}{91886}
\saveTG{類}{91886}
\saveTG{𪹺}{91886}
\saveTG{𩔖}{91886}
\saveTG{𤌢}{91889}
\saveTG{炋}{91890}
\saveTG{熛}{91891}
\saveTG{𤇋}{91891}
\saveTG{𤌣}{91894}
\saveTG{𤌓}{91894}
\saveTG{𤌪}{91896}
\saveTG{㷧}{91896}
\saveTG{㮡}{91904}
\saveTG{𥺘}{91911}
\saveTG{䉺}{91912}
\saveTG{𥺼}{91912}
\saveTG{𤉑}{91912}
\saveTG{𥾂}{91912}
\saveTG{𥽛}{91912}
\saveTG{𥻛}{91914}
\saveTG{𫂸}{91914}
\saveTG{糎}{91915}
\saveTG{𥽥}{91915}
\saveTG{𥹚}{91916}
\saveTG{虩}{91917}
\saveTG{𧈅}{91917}
\saveTG{粔}{91917}
\saveTG{𥺉}{91918}
\saveTG{𥺫}{91919}
\saveTG{𥹂}{91919}
\saveTG{𥸧}{91920}
\saveTG{𥸰}{91921}
\saveTG{𥹴}{91922}
\saveTG{𥺾}{91927}
\saveTG{𥽮}{91927}
\saveTG{𤐨}{91927}
\saveTG{粫}{91927}
\saveTG{粝}{91927}
\saveTG{糯}{91927}
\saveTG{糥}{91927}
\saveTG{糲}{91927}
\saveTG{𥸯}{91927}
\saveTG{𥸴}{91927}
\saveTG{𥻥}{91927}
\saveTG{𥹘}{91927}
\saveTG{𥼞}{91931}
\saveTG{𥽜}{91931}
\saveTG{𥼳}{91931}
\saveTG{𥼸}{91932}
\saveTG{粻}{91932}
\saveTG{𤐾}{91938}
\saveTG{𤒛}{91943}
\saveTG{粳}{91946}
\saveTG{䊤}{91946}
\saveTG{𢿧}{91947}
\saveTG{𥽟}{91947}
\saveTG{𥽶}{91948}
\saveTG{𥽑}{91949}
\saveTG{𥹒}{91949}
\saveTG{𥽘}{91953}
\saveTG{粘}{91960}
\saveTG{糣}{91961}
\saveTG{𥼦}{91961}
\saveTG{𥽯}{91962}
\saveTG{𫂹}{91962}
\saveTG{粨}{91962}
\saveTG{糆}{91962}
\saveTG{𦧡}{91964}
\saveTG{𥺗}{91964}
\saveTG{粞}{91964}
\saveTG{𥻅}{91966}
\saveTG{𥺖}{91969}
\saveTG{𤒳}{91969}
\saveTG{颣}{91982}
\saveTG{𥻟}{91984}
\saveTG{𩓮}{91986}
\saveTG{𥜛}{91986}
\saveTG{纇}{91986}
\saveTG{頪}{91986}
\saveTG{𥽣}{91989}
\saveTG{𥼠}{91994}
\saveTG{𢢼}{92000}
\saveTG{懰}{92000}
\saveTG{惻}{92000}
\saveTG{悧}{92000}
\saveTG{㤡}{92000}
\saveTG{恻}{92000}
\saveTG{𢣎}{92013}
\saveTG{恌}{92013}
\saveTG{𢗖}{92014}
\saveTG{㤛}{92014}
\saveTG{𪬷}{92014}
\saveTG{悂}{92014}
\saveTG{𢝤}{92014}
\saveTG{𢝆}{92015}
\saveTG{慛}{92015}
\saveTG{𢚇}{92015}
\saveTG{𢛉}{92015}
\saveTG{𪬗}{92015}
\saveTG{𢗵}{92017}
\saveTG{𪬁}{92017}
\saveTG{恺}{92017}
\saveTG{㥴}{92017}
\saveTG{𢖬}{92017}
\saveTG{𢗽}{92017}
\saveTG{𢖲}{92017}
\saveTG{𢗳}{92017}
\saveTG{𢙔}{92017}
\saveTG{𢢯}{92017}
\saveTG{𢟒}{92017}
\saveTG{𢙙}{92017}
\saveTG{𢠎}{92017}
\saveTG{憕}{92018}
\saveTG{愷}{92018}
\saveTG{忻}{92021}
\saveTG{𢠹}{92021}
\saveTG{慚}{92021}
\saveTG{惭}{92021}
\saveTG{𢚵}{92022}
\saveTG{憉}{92022}
\saveTG{𢒆}{92022}
\saveTG{𢢋}{92022}
\saveTG{𢙿}{92027}
\saveTG{𢤮}{92027}
\saveTG{𢥘}{92027}
\saveTG{𢠿}{92027}
\saveTG{𢛷}{92027}
\saveTG{𢘏}{92027}
\saveTG{惴}{92027}
\saveTG{𢢄}{92027}
\saveTG{憍}{92027}
\saveTG{𢣆}{92027}
\saveTG{㤭}{92028}
\saveTG{𢗷}{92030}
\saveTG{𢢥}{92030}
\saveTG{𢟏}{92030}
\saveTG{𢣤}{92031}
\saveTG{㦩}{92031}
\saveTG{𢦈}{92031}
\saveTG{𢘌}{92031}
\saveTG{𢝥}{92032}
\saveTG{𢚭}{92032}
\saveTG{怅}{92034}
\saveTG{懚}{92037}
\saveTG{怟}{92040}
\saveTG{忯}{92040}
\saveTG{忏}{92040}
\saveTG{𢘛}{92041}
\saveTG{𢚀}{92041}
\saveTG{𢚃}{92043}
\saveTG{𢥚}{92043}
\saveTG{𢚶}{92044}
\saveTG{𢛊}{92044}
\saveTG{𢣌}{92047}
\saveTG{𢜻}{92047}
\saveTG{悸}{92047}
\saveTG{惾}{92047}
\saveTG{㤆}{92047}
\saveTG{㥅}{92047}
\saveTG{𢠺}{92047}
\saveTG{懓}{92047}
\saveTG{𢢟}{92047}
\saveTG{𢠭}{92047}
\saveTG{愋}{92047}
\saveTG{𢚼}{92047}
\saveTG{𢡧}{92053}
\saveTG{𪬭}{92057}
\saveTG{𢛵}{92057}
\saveTG{𢞄}{92061}
\saveTG{㤧}{92061}
\saveTG{恉}{92061}
\saveTG{𢝷}{92062}
\saveTG{惱}{92062}
\saveTG{𢞮}{92063}
\saveTG{恬}{92064}
\saveTG{𢡦}{92064}
\saveTG{𢝺}{92064}
\saveTG{惛}{92064}
\saveTG{𢡄}{92068}
\saveTG{𢚺}{92068}
\saveTG{憣}{92069}
\saveTG{𢣊}{92069}
\saveTG{𪬏}{92069}
\saveTG{忷}{92070}
\saveTG{愮}{92072}
\saveTG{𪨀}{92072}
\saveTG{㤕}{92072}
\saveTG{𢝿}{92072}
\saveTG{𢝾}{92072}
\saveTG{𢚰}{92072}
\saveTG{慆}{92077}
\saveTG{愼}{92081}
\saveTG{懻}{92081}
\saveTG{㤇}{92084}
\saveTG{𢣨}{92084}
\saveTG{𢜽}{92084}
\saveTG{慀}{92084}
\saveTG{懫}{92086}
\saveTG{㥽}{92086}
\saveTG{𢠄}{92091}
\saveTG{㦡}{92094}
\saveTG{㥒}{92094}
\saveTG{𢘨}{92094}
\saveTG{𢟊}{92094}
\saveTG{𢢜}{92095}
\saveTG{𢜼}{92095}
\saveTG{𠟅}{92100}
\saveTG{𨨺}{92109}
\saveTG{㓥}{92120}
\saveTG{𨭥}{92195}
\saveTG{𠚺}{92200}
\saveTG{削}{92200}
\saveTG{㓬}{92200}
\saveTG{㔂}{92200}
\saveTG{亃}{92210}
\saveTG{𡮵}{92215}
\saveTG{㲖}{92217}
\saveTG{𦧿}{92217}
\saveTG{亃}{92217}
\saveTG{𧺄}{92218}
\saveTG{斴}{92221}
\saveTG{𣃌}{92221}
\saveTG{𪛑}{92227}
\saveTG{粼}{92230}
\saveTG{𥻘}{92237}
\saveTG{𤏞}{92237}
\saveTG{𡮚}{92270}
\saveTG{𠠣}{92299}
\saveTG{𠟛}{92300}
\saveTG{鵥}{92327}
\saveTG{𪃅}{92327}
\saveTG{𢜦}{92332}
\saveTG{𢣥}{92332}
\saveTG{𢝲}{92339}
\saveTG{𡮖}{92401}
\saveTG{𡞟}{92444}
\saveTG{𢖖}{92447}
\saveTG{𡢐}{92461}
\saveTG{判}{92500}
\saveTG{揱}{92502}
\saveTG{𢱀}{92503}
\saveTG{叛}{92547}
\saveTG{劏}{92600}
\saveTG{㼕}{92633}
\saveTG{𤭰}{92717}
\saveTG{𠟥}{92800}
\saveTG{𤑯}{92800}
\saveTG{𤊶}{92800}
\saveTG{𤆑}{92800}
\saveTG{𤈘}{92800}
\saveTG{𤏬}{92800}
\saveTG{灲}{92800}
\saveTG{剡}{92800}
\saveTG{熴}{92812}
\saveTG{𪹎}{92812}
\saveTG{爉}{92812}
\saveTG{烑}{92813}
\saveTG{灹}{92814}
\saveTG{𤎛}{92814}
\saveTG{熣}{92815}
\saveTG{煄}{92815}
\saveTG{𤆇}{92817}
\saveTG{𤎞}{92817}
\saveTG{𤌶}{92817}
\saveTG{𤆺}{92817}
\saveTG{𤐜}{92817}
\saveTG{𤊚}{92817}
\saveTG{𤍡}{92817}
\saveTG{𤈡}{92817}
\saveTG{𤈯}{92817}
\saveTG{𤍈}{92818}
\saveTG{燈}{92818}
\saveTG{𠝿}{92820}
\saveTG{燍}{92821}
\saveTG{𤍖}{92821}
\saveTG{炘}{92821}
\saveTG{𪹫}{92822}
\saveTG{𪹡}{92822}
\saveTG{烿}{92822}
\saveTG{𪸹}{92822}
\saveTG{𤉒}{92822}
\saveTG{𤐣}{92827}
\saveTG{㲭}{92827}
\saveTG{𤈰}{92827}
\saveTG{𤊷}{92827}
\saveTG{𤎶}{92827}
\saveTG{𪺄}{92827}
\saveTG{燆}{92827}
\saveTG{𤍦}{92827}
\saveTG{𪹋}{92827}
\saveTG{煓}{92827}
\saveTG{𤆩}{92830}
\saveTG{𤍅}{92830}
\saveTG{燻}{92831}
\saveTG{𪸡}{92831}
\saveTG{𤋻}{92832}
\saveTG{𤑿}{92833}
\saveTG{𡮽}{92836}
\saveTG{𤇮}{92837}
\saveTG{𤇹}{92841}
\saveTG{烻}{92841}
\saveTG{烶}{92841}
\saveTG{爝}{92843}
\saveTG{𪸯}{92844}
\saveTG{𤇼}{92844}
\saveTG{𤉦}{92844}
\saveTG{𤏣}{92845}
\saveTG{爝}{92846}
\saveTG{爜}{92847}
\saveTG{𤌁}{92847}
\saveTG{煖}{92847}
\saveTG{𤓡}{92847}
\saveTG{燰}{92847}
\saveTG{𤊐}{92847}
\saveTG{𤏋}{92847}
\saveTG{炍}{92847}
\saveTG{烰}{92847}
\saveTG{𥼰}{92848}
\saveTG{𤓩}{92848}
\saveTG{烀}{92849}
\saveTG{㷨}{92854}
\saveTG{𪸾}{92857}
\saveTG{𤍉}{92861}
\saveTG{𤊲}{92862}
\saveTG{煯}{92862}
\saveTG{焝}{92864}
\saveTG{𪹒}{92864}
\saveTG{𤏾}{92864}
\saveTG{𤈏}{92864}
\saveTG{㷔}{92869}
\saveTG{𤋭}{92869}
\saveTG{𤋠}{92869}
\saveTG{燔}{92869}
\saveTG{灿}{92870}
\saveTG{𤇱}{92872}
\saveTG{熎}{92872}
\saveTG{𤒤}{92872}
\saveTG{炪}{92872}
\saveTG{熖}{92877}
\saveTG{𤌵}{92877}
\saveTG{𤌳}{92884}
\saveTG{𤒊}{92885}
\saveTG{𪹟}{92886}
\saveTG{𪹇}{92889}
\saveTG{爍}{92894}
\saveTG{𤍝}{92894}
\saveTG{烁}{92894}
\saveTG{𤊔}{92894}
\saveTG{秌}{92894}
\saveTG{𤊕}{92894}
\saveTG{𤍒}{92895}
\saveTG{㸁}{92895}
\saveTG{𠟡}{92900}
\saveTG{𠠓}{92900}
\saveTG{𥻃}{92900}
\saveTG{粌}{92900}
\saveTG{𦂗}{92903}
\saveTG{𪲫}{92904}
\saveTG{𣕇}{92904}
\saveTG{粃}{92910}
\saveTG{𥽖}{92913}
\saveTG{粍}{92914}
\saveTG{籷}{92914}
\saveTG{𥻝}{92915}
\saveTG{𥼂}{92915}
\saveTG{𥽊}{92917}
\saveTG{㲜}{92917}
\saveTG{粃}{92917}
\saveTG{𥺺}{92918}
\saveTG{𥻶}{92918}
\saveTG{𥺈}{92921}
\saveTG{䉼}{92921}
\saveTG{𥼤}{92921}
\saveTG{𥺐}{92921}
\saveTG{𥼔}{92921}
\saveTG{𥻁}{92927}
\saveTG{𥼱}{92927}
\saveTG{𥸺}{92927}
\saveTG{𥾃}{92931}
\saveTG{䊝}{92939}
\saveTG{𤎕}{92939}
\saveTG{𥹽}{92940}
\saveTG{粁}{92940}
\saveTG{糭}{92947}
\saveTG{𥺲}{92947}
\saveTG{䉻}{92947}
\saveTG{粄}{92947}
\saveTG{粰}{92947}
\saveTG{糉}{92947}
\saveTG{䉿}{92949}
\saveTG{𥼘}{92958}
\saveTG{𥻢}{92960}
\saveTG{𥻄}{92962}
\saveTG{䊩}{92969}
\saveTG{籼}{92970}
\saveTG{𤋫}{92972}
\saveTG{𢙉}{92972}
\saveTG{𥻺}{92983}
\saveTG{𤐓}{92984}
\saveTG{𥼜}{92985}
\saveTG{𥽸}{92993}
\saveTG{𥼄}{92994}
\saveTG{𥽗}{92994}
\saveTG{㣺}{93000}
\saveTG{㤈}{93000}
\saveTG{怭}{93004}
\saveTG{𢗼}{93007}
\saveTG{𢞲}{93011}
\saveTG{𢠊}{93011}
\saveTG{惋}{93012}
\saveTG{忧}{93012}
\saveTG{憱}{93012}
\saveTG{𪬃}{93012}
\saveTG{悾}{93012}
\saveTG{𢠠}{93014}
\saveTG{𢚍}{93014}
\saveTG{𢟭}{93014}
\saveTG{愃}{93016}
\saveTG{㤞}{93017}
\saveTG{𢘙}{93017}
\saveTG{𢣝}{93017}
\saveTG{㤶}{93017}
\saveTG{𢥼}{93017}
\saveTG{𪫻}{93017}
\saveTG{𪫤}{93017}
\saveTG{𪬍}{93017}
\saveTG{懧}{93021}
\saveTG{㤖}{93021}
\saveTG{慘}{93022}
\saveTG{惨}{93022}
\saveTG{𢛙}{93027}
\saveTG{惼}{93027}
\saveTG{悑}{93027}
\saveTG{𢠌}{93031}
\saveTG{𢤪}{93032}
\saveTG{𢜿}{93032}
\saveTG{悢}{93032}
\saveTG{𢠋}{93033}
\saveTG{憾}{93035}
\saveTG{㦥}{93036}
\saveTG{㦓}{93038}
\saveTG{𢡖}{93038}
\saveTG{𢢈}{93038}
\saveTG{恜}{93040}
\saveTG{𢖺}{93040}
\saveTG{𢟓}{93041}
\saveTG{愽}{93042}
\saveTG{𢘀}{93044}
\saveTG{𪫲}{93044}
\saveTG{悛}{93047}
\saveTG{𢜶}{93047}
\saveTG{𢢒}{93048}
\saveTG{㦐}{93048}
\saveTG{悈}{93050}
\saveTG{恈}{93050}
\saveTG{怴}{93050}
\saveTG{𢟘}{93050}
\saveTG{懺}{93050}
\saveTG{慽}{93050}
\saveTG{懴}{93050}
\saveTG{𢤚}{93050}
\saveTG{惐}{93050}
\saveTG{𢚄}{93052}
\saveTG{㤜}{93053}
\saveTG{㥇}{93053}
\saveTG{𢥇}{93054}
\saveTG{𢜩}{93056}
\saveTG{𢡠}{93056}
\saveTG{𢚕}{93056}
\saveTG{𢜾}{93058}
\saveTG{怡}{93060}
\saveTG{𢚜}{93061}
\saveTG{𢞏}{93062}
\saveTG{𢜥}{93064}
\saveTG{愘}{93064}
\saveTG{𢞐}{93065}
\saveTG{𢠲}{93066}
\saveTG{愹}{93068}
\saveTG{𪬺}{93069}
\saveTG{悺}{93077}
\saveTG{𢟫}{93080}
\saveTG{𢛸}{93081}
\saveTG{𪬚}{93081}
\saveTG{悷}{93084}
\saveTG{𢚝}{93084}
\saveTG{𢝀}{93084}
\saveTG{𢢕}{93084}
\saveTG{𢗗}{93084}
\saveTG{𪬦}{93086}
\saveTG{𢣐}{93086}
\saveTG{𢠡}{93089}
\saveTG{悰}{93091}
\saveTG{𢣼}{93091}
\saveTG{怺}{93092}
\saveTG{𢚗}{93094}
\saveTG{怵}{93094}
\saveTG{㤹}{93099}
\saveTG{𨨁}{93114}
\saveTG{蠽}{93136}
\saveTG{𧓷}{93136}
\saveTG{𧑼}{93136}
\saveTG{㴦}{93166}
\saveTG{𨮘}{93186}
\saveTG{䆪}{93211}
\saveTG{𡰚}{93217}
\saveTG{𢧵}{93250}
\saveTG{𠒬}{93251}
\saveTG{𤡩}{93284}
\saveTG{𢰣}{93502}
\saveTG{戦}{93550}
\saveTG{𢨄}{93650}
\saveTG{𥲏}{93653}
\saveTG{獣}{93684}
\saveTG{䭏}{93727}
\saveTG{𩚳}{93744}
\saveTG{𣫿}{93759}
\saveTG{烞}{93800}
\saveTG{𤆸}{93800}
\saveTG{𤆊}{93800}
\saveTG{𤇩}{93804}
\saveTG{炉}{93807}
\saveTG{烷}{93812}
\saveTG{炨}{93812}
\saveTG{焢}{93812}
\saveTG{焥}{93812}
\saveTG{𤊃}{93812}
\saveTG{𤐒}{93812}
\saveTG{烢}{93814}
\saveTG{煊}{93816}
\saveTG{𤇭}{93817}
\saveTG{𪸗}{93817}
\saveTG{熩}{93817}
\saveTG{𤓍}{93819}
\saveTG{𤆼}{93821}
\saveTG{𤍜}{93822}
\saveTG{𤇖}{93827}
\saveTG{煸}{93827}
\saveTG{烳}{93827}
\saveTG{焪}{93827}
\saveTG{煽}{93827}
\saveTG{𢙭}{93830}
\saveTG{𤈻}{93831}
\saveTG{烺}{93832}
\saveTG{𤏘}{93835}
\saveTG{𤐉}{93835}
\saveTG{燃}{93838}
\saveTG{𤏟}{93838}
\saveTG{烒}{93840}
\saveTG{𪹞}{93841}
\saveTG{𪸵}{93841}
\saveTG{𤑧}{93842}
\saveTG{煿}{93842}
\saveTG{𤈁}{93843}
\saveTG{𤒔}{93843}
\saveTG{炦}{93847}
\saveTG{𪺅}{93847}
\saveTG{焌}{93847}
\saveTG{𤇦}{93850}
\saveTG{煘}{93850}
\saveTG{熾}{93850}
\saveTG{𤊨}{93851}
\saveTG{𤒯}{93851}
\saveTG{𤍬}{93851}
\saveTG{𤇳}{93851}
\saveTG{㸍}{93851}
\saveTG{𤉍}{93852}
\saveTG{𪸤}{93852}
\saveTG{𪸶}{93853}
\saveTG{𤈪}{93854}
\saveTG{𤇽}{93854}
\saveTG{𪸬}{93856}
\saveTG{𤉫}{93858}
\saveTG{炲}{93860}
\saveTG{熍}{93866}
\saveTG{熔}{93868}
\saveTG{𤊟}{93882}
\saveTG{㷝}{93884}
\saveTG{𪺉}{93884}
\saveTG{㶼}{93884}
\saveTG{𪹾}{93891}
\saveTG{𤉳}{93891}
\saveTG{炢}{93894}
\saveTG{㷘}{93894}
\saveTG{爎}{93896}
\saveTG{𥹀}{93900}
\saveTG{𥹅}{93904}
\saveTG{粐}{93907}
\saveTG{𥺥}{93912}
\saveTG{𥼵}{93912}
\saveTG{𥹈}{93912}
\saveTG{𥹳}{93917}
\saveTG{𥺹}{93917}
\saveTG{𥹍}{93921}
\saveTG{糝}{93922}
\saveTG{糁}{93922}
\saveTG{糄}{93927}
\saveTG{䊇}{93927}
\saveTG{糘}{93932}
\saveTG{粮}{93932}
\saveTG{𥽇}{93935}
\saveTG{𥻮}{93941}
\saveTG{𥹨}{93941}
\saveTG{糐}{93942}
\saveTG{𫃀}{93944}
\saveTG{𥹇}{93944}
\saveTG{𥻬}{93947}
\saveTG{𥹔}{93947}
\saveTG{䊱}{93950}
\saveTG{𫂽}{93954}
\saveTG{𥻇}{93956}
\saveTG{𥼀}{93959}
\saveTG{𥹋}{93961}
\saveTG{𥼍}{93962}
\saveTG{糌}{93964}
\saveTG{𥽴}{93964}
\saveTG{𥻞}{93964}
\saveTG{𥻀}{93964}
\saveTG{𥹼}{93982}
\saveTG{𥽕}{93991}
\saveTG{粽}{93991}
\saveTG{䊉}{93994}
\saveTG{𫃁}{93999}
\saveTG{𢟙}{94000}
\saveTG{忖}{94000}
\saveTG{𢗢}{94002}
\saveTG{𢗸}{94003}
\saveTG{㤔}{94003}
\saveTG{𢥓}{94003}
\saveTG{㦠}{94003}
\saveTG{𢠢}{94003}
\saveTG{𢙪}{94003}
\saveTG{𢙳}{94010}
\saveTG{恾}{94010}
\saveTG{𪫞}{94010}
\saveTG{憢}{94012}
\saveTG{慌}{94012}
\saveTG{忱}{94012}
\saveTG{𢣏}{94012}
\saveTG{𢞑}{94012}
\saveTG{㥀}{94012}
\saveTG{忚}{94012}
\saveTG{㥺}{94012}
\saveTG{恅}{94012}
\saveTG{𢡨}{94013}
\saveTG{恠}{94014}
\saveTG{㤬}{94014}
\saveTG{𢘪}{94014}
\saveTG{懛}{94014}
\saveTG{懽}{94015}
\saveTG{懂}{94015}
\saveTG{𢥢}{94015}
\saveTG{𢞕}{94015}
\saveTG{𢟠}{94015}
\saveTG{慬}{94015}
\saveTG{𢟾}{94017}
\saveTG{𢞷}{94017}
\saveTG{㤿}{94017}
\saveTG{𢙝}{94017}
\saveTG{𢘚}{94017}
\saveTG{𢝚}{94017}
\saveTG{𢙧}{94017}
\saveTG{𢚛}{94017}
\saveTG{怈}{94017}
\saveTG{𢗟}{94017}
\saveTG{㦉}{94018}
\saveTG{愖}{94018}
\saveTG{𢚂}{94018}
\saveTG{𢜳}{94018}
\saveTG{㥓}{94021}
\saveTG{𢤦}{94025}
\saveTG{悕}{94027}
\saveTG{恊}{94027}
\saveTG{愶}{94027}
\saveTG{怮}{94027}
\saveTG{㤼}{94027}
\saveTG{𢣛}{94027}
\saveTG{𢟕}{94027}
\saveTG{𢛻}{94027}
\saveTG{恸}{94027}
\saveTG{𢥉}{94027}
\saveTG{𢞧}{94027}
\saveTG{𢞎}{94027}
\saveTG{㦅}{94027}
\saveTG{𦑑}{94027}
\saveTG{𪫭}{94027}
\saveTG{𢤆}{94027}
\saveTG{𢛘}{94027}
\saveTG{𢟼}{94027}
\saveTG{怖}{94027}
\saveTG{惰}{94027}
\saveTG{憜}{94027}
\saveTG{恗}{94027}
\saveTG{忇}{94027}
\saveTG{慲}{94027}
\saveTG{懄}{94027}
\saveTG{懜}{94027}
\saveTG{𪫝}{94027}
\saveTG{𪬢}{94027}
\saveTG{𢢖}{94027}
\saveTG{𢦂}{94027}
\saveTG{㤢}{94027}
\saveTG{𢚆}{94027}
\saveTG{慟}{94027}
\saveTG{忲}{94030}
\saveTG{𢢮}{94031}
\saveTG{𢙺}{94031}
\saveTG{𢣖}{94031}
\saveTG{懗}{94031}
\saveTG{𢗞}{94031}
\saveTG{𢤼}{94031}
\saveTG{㤸}{94031}
\saveTG{懞}{94032}
\saveTG{𢙐}{94032}
\saveTG{憽}{94032}
\saveTG{恡}{94032}
\saveTG{怯}{94032}
\saveTG{𢤺}{94033}
\saveTG{𢤄}{94036}
\saveTG{慥}{94036}
\saveTG{𪫿}{94037}
\saveTG{𢞺}{94038}
\saveTG{㤊}{94040}
\saveTG{𢢩}{94040}
\saveTG{𢖵}{94040}
\saveTG{𢗛}{94040}
\saveTG{懤}{94041}
\saveTG{𢡇}{94041}
\saveTG{悻}{94041}
\saveTG{恃}{94041}
\saveTG{𢝉}{94041}
\saveTG{𢠍}{94043}
\saveTG{𢟨}{94044}
\saveTG{𢜘}{94044}
\saveTG{愺}{94046}
\saveTG{𢙨}{94047}
\saveTG{悖}{94047}
\saveTG{𢜢}{94047}
\saveTG{㥰}{94047}
\saveTG{㦜}{94047}
\saveTG{㥄}{94047}
\saveTG{𢞻}{94047}
\saveTG{怶}{94047}
\saveTG{忮}{94047}
\saveTG{㤒}{94048}
\saveTG{懱}{94053}
\saveTG{㦊}{94054}
\saveTG{愅}{94056}
\saveTG{愇}{94056}
\saveTG{怙}{94060}
\saveTG{㤑}{94060}
\saveTG{悎}{94061}
\saveTG{𢠁}{94061}
\saveTG{㦧}{94061}
\saveTG{𢤬}{94061}
\saveTG{憘}{94061}
\saveTG{恄}{94061}
\saveTG{惜}{94061}
\saveTG{𢢂}{94061}
\saveTG{懎}{94061}
\saveTG{愭}{94061}
\saveTG{𢟉}{94061}
\saveTG{𪬠}{94062}
\saveTG{懵}{94062}
\saveTG{𢞇}{94063}
\saveTG{㥩}{94064}
\saveTG{𢥃}{94064}
\saveTG{𢜪}{94064}
\saveTG{𢙵}{94064}
\saveTG{㦋}{94064}
\saveTG{𢣅}{94064}
\saveTG{㤌}{94070}
\saveTG{忕}{94080}
\saveTG{㤨}{94081}
\saveTG{𢤋}{94081}
\saveTG{懥}{94081}
\saveTG{憷}{94081}
\saveTG{慎}{94081}
\saveTG{㥍}{94081}
\saveTG{愤}{94082}
\saveTG{𢟎}{94082}
\saveTG{慡}{94084}
\saveTG{慔}{94084}
\saveTG{𢠦}{94085}
\saveTG{愥}{94085}
\saveTG{㦫}{94086}
\saveTG{𢢰}{94086}
\saveTG{𢥎}{94086}
\saveTG{憤}{94086}
\saveTG{𢤳}{94086}
\saveTG{㥚}{94087}
\saveTG{悏}{94088}
\saveTG{恢}{94089}
\saveTG{𪫟}{94090}
\saveTG{惏}{94090}
\saveTG{恘}{94090}
\saveTG{㦗}{94091}
\saveTG{𢢲}{94094}
\saveTG{惵}{94094}
\saveTG{𢜮}{94094}
\saveTG{𢘹}{94094}
\saveTG{憭}{94096}
\saveTG{𢜞}{94098}
\saveTG{𡭶}{94100}
\saveTG{墯}{94104}
\saveTG{𡓧}{94104}
\saveTG{𠡱}{94127}
\saveTG{𪟢}{94127}
\saveTG{㙦}{94143}
\saveTG{㒯}{94154}
\saveTG{𡮴}{94181}
\saveTG{𥊣}{94212}
\saveTG{韑}{94215}
\saveTG{黋}{94218}
\saveTG{𩱵}{94227}
\saveTG{𠓚}{94227}
\saveTG{𡮹}{94232}
\saveTG{𢻒}{94247}
\saveTG{𤿨}{94247}
\saveTG{𤿼}{94247}
\saveTG{𠓅}{94294}
\saveTG{𠢁}{94327}
\saveTG{𢝍}{94333}
\saveTG{𠡶}{94427}
\saveTG{𡡭}{94427}
\saveTG{㪵}{94503}
\saveTG{𡭉}{94503}
\saveTG{𡛤}{94540}
\saveTG{𠦻}{94542}
\saveTG{𦎚}{94552}
\saveTG{𤕛}{94552}
\saveTG{𩏗}{94557}
\saveTG{𠢚}{94627}
\saveTG{𣯁}{94718}
\saveTG{勌}{94727}
\saveTG{𩞘}{94727}
\saveTG{𩚰}{94740}
\saveTG{𫓹}{94781}
\saveTG{炓}{94800}
\saveTG{𪸟}{94800}
\saveTG{𪹅}{94803}
\saveTG{𤆭}{94804}
\saveTG{灶}{94810}
\saveTG{𤇧}{94810}
\saveTG{㸆}{94811}
\saveTG{𤍞}{94811}
\saveTG{灺}{94812}
\saveTG{烍}{94812}
\saveTG{燒}{94812}
\saveTG{熆}{94812}
\saveTG{焼}{94812}
\saveTG{𤋨}{94812}
\saveTG{𤊧}{94812}
\saveTG{𤇉}{94812}
\saveTG{𤐩}{94812}
\saveTG{𤓆}{94812}
\saveTG{𤆧}{94813}
\saveTG{𤋑}{94814}
\saveTG{烓}{94814}
\saveTG{煃}{94814}
\saveTG{𪸷}{94814}
\saveTG{𤐽}{94814}
\saveTG{𤏎}{94814}
\saveTG{𤏥}{94815}
\saveTG{𤒽}{94815}
\saveTG{爟}{94815}
\saveTG{𤌍}{94815}
\saveTG{𤓌}{94815}
\saveTG{𤒬}{94816}
\saveTG{㶩}{94817}
\saveTG{𤌥}{94817}
\saveTG{𤆷}{94817}
\saveTG{㷈}{94817}
\saveTG{𤎣}{94817}
\saveTG{𥹗}{94817}
\saveTG{𤏈}{94818}
\saveTG{𤉛}{94818}
\saveTG{煁}{94818}
\saveTG{𪸴}{94821}
\saveTG{爩}{94822}
\saveTG{𤑳}{94824}
\saveTG{𤎻}{94824}
\saveTG{𥻗}{94826}
\saveTG{𤊹}{94827}
\saveTG{𤈸}{94827}
\saveTG{𤐟}{94827}
\saveTG{𤉶}{94827}
\saveTG{烠}{94827}
\saveTG{𤋲}{94827}
\saveTG{熁}{94827}
\saveTG{烯}{94827}
\saveTG{焫}{94827}
\saveTG{煵}{94827}
\saveTG{燤}{94827}
\saveTG{爤}{94827}
\saveTG{烤}{94827}
\saveTG{𤊊}{94827}
\saveTG{𤊈}{94827}
\saveTG{𤓝}{94827}
\saveTG{㶸}{94827}
\saveTG{𤏨}{94827}
\saveTG{𤌃}{94827}
\saveTG{㷲}{94827}
\saveTG{爋}{94827}
\saveTG{烵}{94827}
\saveTG{𥽼}{94827}
\saveTG{㶭}{94827}
\saveTG{㶧}{94827}
\saveTG{𤊍}{94828}
\saveTG{𥽫}{94831}
\saveTG{𤒈}{94831}
\saveTG{𪸘}{94831}
\saveTG{𤓂}{94831}
\saveTG{𤒋}{94831}
\saveTG{𤓦}{94831}
\saveTG{𤌷}{94831}
\saveTG{爑}{94831}
\saveTG{爀}{94831}
\saveTG{焃}{94831}
\saveTG{𤆠}{94831}
\saveTG{𪹴}{94832}
\saveTG{𤑛}{94832}
\saveTG{𤑕}{94832}
\saveTG{𤋶}{94832}
\saveTG{𥺚}{94832}
\saveTG{燪}{94832}
\saveTG{㶶}{94832}
\saveTG{𤑫}{94835}
\saveTG{燵}{94835}
\saveTG{𤑞}{94835}
\saveTG{爡}{94836}
\saveTG{𤑢}{94836}
\saveTG{𤓣}{94840}
\saveTG{烨}{94841}
\saveTG{燽}{94841}
\saveTG{𤓭}{94843}
\saveTG{𤋵}{94843}
\saveTG{𤈝}{94844}
\saveTG{𤇎}{94845}
\saveTG{㷹}{94846}
\saveTG{𤏱}{94847}
\saveTG{𤊥}{94847}
\saveTG{𤉗}{94847}
\saveTG{𪯅}{94847}
\saveTG{𤋓}{94847}
\saveTG{𢻑}{94847}
\saveTG{𤉻}{94847}
\saveTG{𤓄}{94847}
\saveTG{𤐰}{94847}
\saveTG{𤈊}{94847}
\saveTG{㶿}{94847}
\saveTG{𤊫}{94847}
\saveTG{𤑮}{94852}
\saveTG{𤒃}{94854}
\saveTG{𤋙}{94854}
\saveTG{燁}{94854}
\saveTG{煂}{94856}
\saveTG{煒}{94856}
\saveTG{𤒑}{94857}
\saveTG{𤏑}{94857}
\saveTG{𪸚}{94860}
\saveTG{𤇌}{94860}
\saveTG{𤋹}{94860}
\saveTG{熺}{94861}
\saveTG{焟}{94861}
\saveTG{𤐵}{94861}
\saveTG{𤏔}{94861}
\saveTG{𤈄}{94861}
\saveTG{𥻽}{94861}
\saveTG{焅}{94861}
\saveTG{𤎃}{94862}
\saveTG{𤌫}{94862}
\saveTG{𤋼}{94864}
\saveTG{𤌚}{94864}
\saveTG{𤌕}{94864}
\saveTG{𤏸}{94864}
\saveTG{𤌄}{94864}
\saveTG{㶰}{94870}
\saveTG{𤋿}{94872}
\saveTG{𤌭}{94881}
\saveTG{𤊄}{94881}
\saveTG{烘}{94881}
\saveTG{𪹵}{94882}
\saveTG{㷬}{94884}
\saveTG{熯}{94885}
\saveTG{煐}{94885}
\saveTG{燌}{94886}
\saveTG{𤒷}{94886}
\saveTG{𤓎}{94886}
\saveTG{𪺁}{94886}
\saveTG{燲}{94886}
\saveTG{熿}{94886}
\saveTG{烣}{94889}
\saveTG{炑}{94890}
\saveTG{𤊩}{94890}
\saveTG{烌}{94890}
\saveTG{𤐖}{94891}
\saveTG{𤌘}{94893}
\saveTG{𤏗}{94894}
\saveTG{𤓈}{94894}
\saveTG{𤐌}{94894}
\saveTG{煠}{94894}
\saveTG{𤒶}{94894}
\saveTG{煤}{94894}
\saveTG{燎}{94896}
\saveTG{𫃋}{94900}
\saveTG{籵}{94900}
\saveTG{籿}{94900}
\saveTG{料}{94900}
\saveTG{𥹃}{94903}
\saveTG{䊋}{94910}
\saveTG{𫃒}{94911}
\saveTG{粩}{94912}
\saveTG{𫂴}{94912}
\saveTG{糚}{94914}
\saveTG{糀}{94914}
\saveTG{}{94916}
\saveTG{𫃊}{94917}
\saveTG{𥽵}{94917}
\saveTG{䊁}{94917}
\saveTG{𥹑}{94917}
\saveTG{䊦}{94918}
\saveTG{糂}{94918}
\saveTG{𥺿}{94921}
\saveTG{𥺭}{94927}
\saveTG{𥺯}{94927}
\saveTG{糒}{94927}
\saveTG{勬}{94927}
\saveTG{糷}{94927}
\saveTG{䊟}{94927}
\saveTG{䊖}{94927}
\saveTG{𥹬}{94927}
\saveTG{䊪}{94927}
\saveTG{𫄁}{94927}
\saveTG{}{94927}
\saveTG{𥹱}{94927}
\saveTG{𥼓}{94927}
\saveTG{𥻻}{94927}
\saveTG{粏}{94930}
\saveTG{𥹓}{94931}
\saveTG{𥺃}{94931}
\saveTG{糙}{94936}
\saveTG{籹}{94940}
\saveTG{䊭}{94943}
\saveTG{𥻣}{94943}
\saveTG{𫂷}{94943}
\saveTG{𥻵}{94943}
\saveTG{𥹩}{94943}
\saveTG{𥸳}{94947}
\saveTG{𥹖}{94947}
\saveTG{𥺄}{94947}
\saveTG{𥼉}{94947}
\saveTG{𥺙}{94947}
\saveTG{𥹸}{94947}
\saveTG{䊀}{94960}
\saveTG{䊰}{94960}
\saveTG{𥺮}{94961}
\saveTG{𥺊}{94961}
\saveTG{糦}{94961}
\saveTG{𤐞}{94964}
\saveTG{粓}{94970}
\saveTG{𤊼}{94970}
\saveTG{粠}{94981}
\saveTG{粸}{94981}
\saveTG{糢}{94984}
\saveTG{𥽢}{94984}
\saveTG{䊔}{94985}
\saveTG{𥽷}{94986}
\saveTG{䊣}{94986}
\saveTG{𤐗}{94989}
\saveTG{𥽍}{94991}
\saveTG{𥻨}{94993}
\saveTG{𥻈}{94994}
\saveTG{𥻹}{94994}
\saveTG{𥼧}{94995}
\saveTG{𢗒}{95000}
\saveTG{𢗆}{95000}
\saveTG{㤓}{95003}
\saveTG{㤦}{95006}
\saveTG{𢘊}{95006}
\saveTG{㤤}{95006}
\saveTG{𢘽}{95006}
\saveTG{𢚷}{95006}
\saveTG{忡}{95006}
\saveTG{𢘶}{95007}
\saveTG{𢜝}{95007}
\saveTG{性}{95010}
\saveTG{𢦃}{95012}
\saveTG{𢟸}{95012}
\saveTG{𢣺}{95012}
\saveTG{㦎}{95016}
\saveTG{𢙒}{95017}
\saveTG{忳}{95017}
\saveTG{𢢪}{95018}
\saveTG{情}{95027}
\saveTG{怫}{95027}
\saveTG{慩}{95030}
\saveTG{𢘷}{95031}
\saveTG{𢥞}{95031}
\saveTG{𢤢}{95031}
\saveTG{憹}{95032}
\saveTG{憓}{95033}
\saveTG{㥙}{95036}
\saveTG{懳}{95037}
\saveTG{𢠻}{95039}
\saveTG{㤽}{95043}
\saveTG{慱}{95043}
\saveTG{悽}{95044}
\saveTG{慺}{95044}
\saveTG{𢘦}{95047}
\saveTG{𢞡}{95047}
\saveTG{𢤠}{95056}
\saveTG{𢝂}{95057}
\saveTG{𢜗}{95058}
\saveTG{怞}{95060}
\saveTG{𢢺}{95061}
\saveTG{𢝇}{95062}
\saveTG{慒}{95066}
\saveTG{𢝣}{95068}
\saveTG{𢡚}{95068}
\saveTG{𢥱}{95069}
\saveTG{𢠅}{95077}
\saveTG{𢟩}{95077}
\saveTG{怏}{95080}
\saveTG{怢}{95080}
\saveTG{𢗲}{95080}
\saveTG{快}{95080}
\saveTG{㥏}{95081}
\saveTG{𢜀}{95081}
\saveTG{愦}{95082}
\saveTG{恞}{95082}
\saveTG{𢣜}{95086}
\saveTG{憒}{95086}
\saveTG{𢢻}{95086}
\saveTG{怽}{95090}
\saveTG{𢙩}{95090}
\saveTG{𢟐}{95092}
\saveTG{𪬔}{95092}
\saveTG{𢙀}{95092}
\saveTG{愫}{95093}
\saveTG{𢞼}{95094}
\saveTG{𢛔}{95095}
\saveTG{𢟞}{95096}
\saveTG{𢠣}{95096}
\saveTG{悚}{95096}
\saveTG{𢛔}{95096}
\saveTG{𢠮}{95099}
\saveTG{㥆}{95099}
\saveTG{㥭}{95099}
\saveTG{𧓔}{95136}
\saveTG{㷯}{95158}
\saveTG{𪶟}{95182}
\saveTG{𠒗}{95206}
\saveTG{𠓒}{95286}
\saveTG{𠒵}{95296}
\saveTG{𪹸}{95333}
\saveTG{𤑁}{95442}
\saveTG{𠣇}{95456}
\saveTG{𠢣}{95481}
\saveTG{𥹮}{95507}
\saveTG{𦎹}{95544}
\saveTG{𪺊}{95638}
\saveTG{炐}{95800}
\saveTG{㶱}{95803}
\saveTG{𤉖}{95806}
\saveTG{𤈠}{95807}
\saveTG{烧}{95812}
\saveTG{𤇣}{95812}
\saveTG{燼}{95812}
\saveTG{𤉈}{95817}
\saveTG{𡭻}{95817}
\saveTG{𤆚}{95817}
\saveTG{炖}{95817}
\saveTG{𤌛}{95827}
\saveTG{𤌂}{95827}
\saveTG{炥}{95827}
\saveTG{炜}{95827}
\saveTG{熽}{95827}
\saveTG{燶}{95832}
\saveTG{𤒚}{95832}
\saveTG{𤍲}{95836}
\saveTG{烛}{95836}
\saveTG{爞}{95836}
\saveTG{𤑡}{95837}
\saveTG{煡}{95840}
\saveTG{}{95840}
\saveTG{𤍿}{95843}
\saveTG{熡}{95844}
\saveTG{煹}{95847}
\saveTG{𤊡}{95858}
\saveTG{𤊴}{95864}
\saveTG{㷮}{95866}
\saveTG{㸇}{95868}
\saveTG{𤏖}{95868}
\saveTG{𤓗}{95869}
\saveTG{𤆮}{95880}
\saveTG{炴}{95880}
\saveTG{炔}{95880}
\saveTG{𪸸}{95881}
\saveTG{𤊪}{95881}
\saveTG{𤈙}{95882}
\saveTG{𤋷}{95884}
\saveTG{𤏳}{95886}
\saveTG{𤎺}{95887}
\saveTG{𤈞}{95890}
\saveTG{㶬}{95890}
\saveTG{𪸙}{95890}
\saveTG{𤈐}{95892}
\saveTG{𤍊}{95893}
\saveTG{炼}{95894}
\saveTG{𤌴}{95894}
\saveTG{煉}{95896}
\saveTG{𫂶}{95900}
\saveTG{粀}{95900}
\saveTG{𥼏}{95902}
\saveTG{𥹞}{95906}
\saveTG{𥸵}{95917}
\saveTG{𤎒}{95917}
\saveTG{𥽈}{95918}
\saveTG{𥼥}{95927}
\saveTG{䊘}{95927}
\saveTG{精}{95927}
\saveTG{䊥}{95927}
\saveTG{𥺒}{95927}
\saveTG{𥼯}{95931}
\saveTG{𧊾}{95931}
\saveTG{䊕}{95940}
\saveTG{𥺅}{95943}
\saveTG{𤐭}{95943}
\saveTG{䊜}{95943}
\saveTG{𥹧}{95947}
\saveTG{粬}{95960}
\saveTG{粙}{95960}
\saveTG{𥻍}{95961}
\saveTG{糟}{95966}
\saveTG{𥽾}{95969}
\saveTG{𥼩}{95986}
\saveTG{𥼃}{95986}
\saveTG{䊧}{95986}
\saveTG{𥹯}{95990}
\saveTG{粖}{95990}
\saveTG{䊂}{95992}
\saveTG{𥼨}{95994}
\saveTG{𥹵}{95996}
\saveTG{𥻂}{95996}
\saveTG{𢛾}{96000}
\saveTG{慖}{96000}
\saveTG{恛}{96000}
\saveTG{悃}{96000}
\saveTG{怬}{96000}
\saveTG{𢛅}{96000}
\saveTG{𢙫}{96000}
\saveTG{㥵}{96000}
\saveTG{𢛕}{96000}
\saveTG{𢘄}{96000}
\saveTG{𪬱}{96000}
\saveTG{𢛍}{96000}
\saveTG{𢘾}{96000}
\saveTG{怕}{96002}
\saveTG{怛}{96010}
\saveTG{𢙃}{96011}
\saveTG{𢚉}{96011}
\saveTG{𢝈}{96012}
\saveTG{惃}{96012}
\saveTG{愰}{96012}
\saveTG{愠}{96012}
\saveTG{悓}{96012}
\saveTG{慍}{96012}
\saveTG{怳}{96012}
\saveTG{愧}{96013}
\saveTG{悜}{96014}
\saveTG{𢠪}{96014}
\saveTG{𢟗}{96014}
\saveTG{惶}{96014}
\saveTG{𪬮}{96014}
\saveTG{𪬨}{96015}
\saveTG{悝}{96015}
\saveTG{懼}{96015}
\saveTG{惺}{96015}
\saveTG{𢢳}{96015}
\saveTG{㦬}{96015}
\saveTG{𪭈}{96015}
\saveTG{悒}{96017}
\saveTG{𢛺}{96017}
\saveTG{𢚋}{96017}
\saveTG{𢤛}{96017}
\saveTG{愰}{96017}
\saveTG{𢞗}{96017}
\saveTG{𢣦}{96021}
\saveTG{𢙆}{96022}
\saveTG{𢢗}{96027}
\saveTG{𢢔}{96027}
\saveTG{愣}{96027}
\saveTG{惕}{96027}
\saveTG{㥜}{96027}
\saveTG{愒}{96027}
\saveTG{愕}{96027}
\saveTG{愓}{96027}
\saveTG{𢚹}{96027}
\saveTG{𪬀}{96027}
\saveTG{𢠉}{96027}
\saveTG{𢣔}{96027}
\saveTG{𢥅}{96027}
\saveTG{𢤇}{96027}
\saveTG{㥥}{96027}
\saveTG{悁}{96027}
\saveTG{𢚻}{96028}
\saveTG{𢞴}{96030}
\saveTG{愢}{96030}
\saveTG{憁}{96030}
\saveTG{𢡀}{96031}
\saveTG{𢡎}{96032}
\saveTG{𢝓}{96032}
\saveTG{𢟿}{96032}
\saveTG{㦙}{96032}
\saveTG{懁}{96032}
\saveTG{愄}{96032}
\saveTG{㥗}{96040}
\saveTG{𢡋}{96041}
\saveTG{悍}{96041}
\saveTG{懌}{96041}
\saveTG{𢜱}{96042}
\saveTG{㥂}{96043}
\saveTG{𢣀}{96043}
\saveTG{𢛿}{96044}
\saveTG{𢛞}{96045}
\saveTG{𢠧}{96047}
\saveTG{㦍}{96047}
\saveTG{𢢇}{96047}
\saveTG{戄}{96047}
\saveTG{慢}{96047}
\saveTG{𢥴}{96048}
\saveTG{𢘉}{96050}
\saveTG{𢝟}{96052}
\saveTG{𢞙}{96052}
\saveTG{憚}{96056}
\saveTG{𢛽}{96060}
\saveTG{㦒}{96061}
\saveTG{𢥑}{96061}
\saveTG{𢙲}{96062}
\saveTG{怾}{96080}
\saveTG{惿}{96081}
\saveTG{悮}{96084}
\saveTG{悞}{96084}
\saveTG{愪}{96086}
\saveTG{𢠼}{96086}
\saveTG{懆}{96094}
\saveTG{𢥟}{96094}
\saveTG{惈}{96094}
\saveTG{憬}{96096}
\saveTG{懪}{96099}
\saveTG{𨝦}{96117}
\saveTG{𨩡}{96123}
\saveTG{䥟}{96127}
\saveTG{𨰴}{96132}
\saveTG{𣆥}{96210}
\saveTG{尡}{96212}
\saveTG{𤾗}{96214}
\saveTG{𨤢}{96215}
\saveTG{𪞃}{96215}
\saveTG{𠓊}{96215}
\saveTG{䰪}{96217}
\saveTG{𨞁}{96217}
\saveTG{𨜂}{96217}
\saveTG{𨚈}{96217}
\saveTG{𠒼}{96217}
\saveTG{𠓋}{96241}
\saveTG{𠒦}{96262}
\saveTG{𠒪}{96294}
\saveTG{𢜔}{96330}
\saveTG{𠠽}{96427}
\saveTG{𠲑}{96527}
\saveTG{䚇}{96617}
\saveTG{𩴖}{96617}
\saveTG{𩴏}{96617}
\saveTG{𥽙}{96684}
\saveTG{𦉕}{96756}
\saveTG{𪸜}{96800}
\saveTG{焑}{96800}
\saveTG{畑}{96800}
\saveTG{𤇆}{96800}
\saveTG{𤊎}{96800}
\saveTG{𤎍}{96800}
\saveTG{烟}{96800}
\saveTG{𤇢}{96802}
\saveTG{炟}{96810}
\saveTG{𤑻}{96811}
\saveTG{𤌗}{96812}
\saveTG{𤈥}{96812}
\saveTG{炾}{96812}
\saveTG{熀}{96812}
\saveTG{焜}{96812}
\saveTG{煴}{96812}
\saveTG{覢}{96812}
\saveTG{熅}{96812}
\saveTG{𤈑}{96812}
\saveTG{𪹈}{96812}
\saveTG{𤉊}{96812}
\saveTG{𤌒}{96814}
\saveTG{𤌼}{96814}
\saveTG{煌}{96814}
\saveTG{爅}{96814}
\saveTG{𤓓}{96815}
\saveTG{𤎲}{96815}
\saveTG{爠}{96815}
\saveTG{煋}{96815}
\saveTG{𨚊}{96817}
\saveTG{𤈋}{96817}
\saveTG{𤌋}{96817}
\saveTG{䙺}{96817}
\saveTG{𤐡}{96817}
\saveTG{煜}{96818}
\saveTG{𤊋}{96818}
\saveTG{焬}{96827}
\saveTG{煟}{96827}
\saveTG{燭}{96827}
\saveTG{焆}{96827}
\saveTG{𤒌}{96827}
\saveTG{𤒫}{96827}
\saveTG{𤑼}{96827}
\saveTG{𪸝}{96827}
\saveTG{𪸺}{96827}
\saveTG{𤐋}{96827}
\saveTG{㷎}{96827}
\saveTG{𤌙}{96827}
\saveTG{㷒}{96827}
\saveTG{𪸧}{96827}
\saveTG{煬}{96827}
\saveTG{𤋽}{96828}
\saveTG{𤋘}{96830}
\saveTG{熜}{96830}
\saveTG{煾}{96830}
\saveTG{熄}{96830}
\saveTG{𤋉}{96831}
\saveTG{㷵}{96831}
\saveTG{𤍴}{96832}
\saveTG{𤏏}{96832}
\saveTG{𪹷}{96832}
\saveTG{𤓋}{96832}
\saveTG{煨}{96832}
\saveTG{𤓢}{96832}
\saveTG{𤒴}{96833}
\saveTG{𤐴}{96833}
\saveTG{𪹼}{96838}
\saveTG{𤒁}{96838}
\saveTG{𤊢}{96840}
\saveTG{焷}{96840}
\saveTG{焺}{96840}
\saveTG{燡}{96841}
\saveTG{𤏵}{96841}
\saveTG{焊}{96841}
\saveTG{焷}{96845}
\saveTG{熳}{96847}
\saveTG{𤎑}{96847}
\saveTG{𪹢}{96847}
\saveTG{𤆿}{96847}
\saveTG{熶}{96847}
\saveTG{𪸻}{96847}
\saveTG{炠}{96850}
\saveTG{𤌰}{96852}
\saveTG{𤐲}{96852}
\saveTG{熚}{96854}
\saveTG{㷸}{96854}
\saveTG{爗}{96854}
\saveTG{𤒹}{96854}
\saveTG{燀}{96856}
\saveTG{𤌖}{96857}
\saveTG{𤑥}{96858}
\saveTG{焻}{96860}
\saveTG{焒}{96860}
\saveTG{煰}{96860}
\saveTG{𤓊}{96861}
\saveTG{𤋪}{96872}
\saveTG{𤉆}{96872}
\saveTG{𪸭}{96880}
\saveTG{炽}{96880}
\saveTG{煶}{96881}
\saveTG{熼}{96881}
\saveTG{𤈶}{96882}
\saveTG{𤋀}{96884}
\saveTG{𪸮}{96884}
\saveTG{𤋡}{96884}
\saveTG{𤉇}{96884}
\saveTG{熉}{96886}
\saveTG{𤍶}{96893}
\saveTG{𤒺}{96894}
\saveTG{𥺰}{96894}
\saveTG{燥}{96894}
\saveTG{㷄}{96895}
\saveTG{燝}{96896}
\saveTG{爆}{96899}
\saveTG{}{96900}
\saveTG{𥹊}{96900}
\saveTG{𫂾}{96900}
\saveTG{𥹉}{96900}
\saveTG{糰}{96900}
\saveTG{𫂻}{96900}
\saveTG{粕}{96902}
\saveTG{䊫}{96911}
\saveTG{粯}{96912}
\saveTG{𥽞}{96914}
\saveTG{𥺆}{96914}
\saveTG{䊗}{96914}
\saveTG{糧}{96915}
\saveTG{𥽺}{96915}
\saveTG{粴}{96915}
\saveTG{䊐}{96917}
\saveTG{𤒠}{96917}
\saveTG{𥽃}{96917}
\saveTG{糃}{96927}
\saveTG{𥺳}{96927}
\saveTG{㶽}{96927}
\saveTG{𥻉}{96927}
\saveTG{𥻑}{96927}
\saveTG{𥽔}{96927}
\saveTG{𥻏}{96930}
\saveTG{𥼟}{96932}
\saveTG{糫}{96932}
\saveTG{糨}{96936}
\saveTG{𥺛}{96940}
\saveTG{粺}{96940}
\saveTG{𥽉}{96941}
\saveTG{𥻯}{96941}
\saveTG{𥼶}{96941}
\saveTG{𥼺}{96941}
\saveTG{𫃇}{96942}
\saveTG{䊡}{96947}
\saveTG{𥻷}{96948}
\saveTG{𥻸}{96953}
\saveTG{𥺓}{96962}
\saveTG{糬}{96964}
\saveTG{䊓}{96982}
\saveTG{糗}{96984}
\saveTG{𥻱}{96986}
\saveTG{𥼐}{96991}
\saveTG{粿}{96994}
\saveTG{𥼑}{96994}
\saveTG{𥼾}{96996}
\saveTG{𫃏}{96996}
\saveTG{𢙜}{97010}
\saveTG{𢞁}{97010}
\saveTG{忛}{97010}
\saveTG{忆}{97010}
\saveTG{𢤯}{97012}
\saveTG{𢡉}{97012}
\saveTG{𢝘}{97012}
\saveTG{𪬞}{97012}
\saveTG{𢖭}{97012}
\saveTG{恤}{97012}
\saveTG{忸}{97012}
\saveTG{怩}{97012}
\saveTG{悗}{97012}
\saveTG{恑}{97012}
\saveTG{怚}{97012}
\saveTG{怉}{97012}
\saveTG{𢛴}{97012}
\saveTG{怪}{97014}
\saveTG{㥛}{97014}
\saveTG{悭}{97014}
\saveTG{慳}{97014}
\saveTG{𢣷}{97015}
\saveTG{㦕}{97015}
\saveTG{𢗌}{97017}
\saveTG{𢙚}{97017}
\saveTG{𢗑}{97017}
\saveTG{𢖾}{97017}
\saveTG{憴}{97017}
\saveTG{忋}{97017}
\saveTG{𢞌}{97017}
\saveTG{𢖯}{97017}
\saveTG{恦}{97020}
\saveTG{憫}{97020}
\saveTG{悯}{97020}
\saveTG{恫}{97020}
\saveTG{忉}{97020}
\saveTG{惆}{97020}
\saveTG{㥌}{97020}
\saveTG{𢞬}{97020}
\saveTG{𢛠}{97020}
\saveTG{𢜭}{97020}
\saveTG{𢗾}{97020}
\saveTG{𪫸}{97020}
\saveTG{怐}{97020}
\saveTG{恟}{97020}
\saveTG{憪}{97020}
\saveTG{惘}{97020}
\saveTG{恂}{97020}
\saveTG{𢡞}{97021}
\saveTG{㦦}{97021}
\saveTG{㣼}{97021}
\saveTG{𢜠}{97021}
\saveTG{𢢅}{97021}
\saveTG{㥊}{97021}
\saveTG{𢠯}{97021}
\saveTG{𢢀}{97021}
\saveTG{㥃}{97021}
\saveTG{𢗋}{97021}
\saveTG{㣿}{97021}
\saveTG{忬}{97022}
\saveTG{𢗠}{97022}
\saveTG{𢘕}{97022}
\saveTG{𢗘}{97022}
\saveTG{㥘}{97022}
\saveTG{憀}{97022}
\saveTG{㦖}{97023}
\saveTG{𢗇}{97023}
\saveTG{𢗕}{97023}
\saveTG{𢘭}{97024}
\saveTG{𡮥}{97024}
\saveTG{恦}{97026}
\saveTG{𢣻}{97026}
\saveTG{𢝁}{97026}
\saveTG{㤯}{97026}
\saveTG{𢘜}{97026}
\saveTG{恀}{97027}
\saveTG{愵}{97027}
\saveTG{愲}{97027}
\saveTG{𢞜}{97027}
\saveTG{𢞛}{97027}
\saveTG{㥠}{97027}
\saveTG{𢖸}{97027}
\saveTG{𢤜}{97027}
\saveTG{𢜟}{97027}
\saveTG{𢡓}{97027}
\saveTG{𢥋}{97027}
\saveTG{㥮}{97027}
\saveTG{𢞠}{97027}
\saveTG{𢝸}{97027}
\saveTG{𢗉}{97027}
\saveTG{𢥶}{97027}
\saveTG{𢛡}{97027}
\saveTG{憰}{97027}
\saveTG{𢘘}{97027}
\saveTG{}{97027}
\saveTG{愑}{97027}
\saveTG{悀}{97027}
\saveTG{𢝞}{97027}
\saveTG{𢖱}{97027}
\saveTG{𪬖}{97028}
\saveTG{𢡿}{97029}
\saveTG{㦨}{97029}
\saveTG{𢥕}{97031}
\saveTG{𢥾}{97031}
\saveTG{𢟈}{97031}
\saveTG{𢚁}{97031}
\saveTG{𢠆}{97032}
\saveTG{𢢽}{97032}
\saveTG{𢠽}{97032}
\saveTG{𢣄}{97032}
\saveTG{𢞶}{97032}
\saveTG{惚}{97032}
\saveTG{㥟}{97032}
\saveTG{𢥯}{97032}
\saveTG{恨}{97032}
\saveTG{愡}{97032}
\saveTG{𢚴}{97032}
\saveTG{𢟔}{97033}
\saveTG{㤏}{97033}
\saveTG{𢣩}{97033}
\saveTG{㦀}{97035}
\saveTG{𢠐}{97036}
\saveTG{慅}{97036}
\saveTG{𢟋}{97037}
\saveTG{怓}{97040}
\saveTG{𪬑}{97041}
\saveTG{忰}{97041}
\saveTG{憳}{97046}
\saveTG{𢘎}{97047}
\saveTG{㣾}{97047}
\saveTG{𢛏}{97047}
\saveTG{怋}{97047}
\saveTG{𢢁}{97047}
\saveTG{𢝄}{97047}
\saveTG{惙}{97047}
\saveTG{忣}{97047}
\saveTG{𢟖}{97047}
\saveTG{𢝼}{97047}
\saveTG{𢛼}{97047}
\saveTG{惸}{97047}
\saveTG{𢟝}{97047}
\saveTG{𢗎}{97047}
\saveTG{𢟌}{97047}
\saveTG{𢘩}{97047}
\saveTG{𢞸}{97047}
\saveTG{𢚌}{97048}
\saveTG{惲}{97052}
\saveTG{懈}{97052}
\saveTG{𢜲}{97052}
\saveTG{𢘃}{97053}
\saveTG{恽}{97054}
\saveTG{怿}{97054}
\saveTG{𢘸}{97054}
\saveTG{𢤫}{97058}
\saveTG{𢙛}{97060}
\saveTG{憺}{97061}
\saveTG{𢝌}{97062}
\saveTG{𢞓}{97062}
\saveTG{慴}{97062}
\saveTG{怊}{97062}
\saveTG{恪}{97064}
\saveTG{惽}{97064}
\saveTG{𢞦}{97067}
\saveTG{𢞳}{97072}
\saveTG{𢞊}{97072}
\saveTG{𢙬}{97074}
\saveTG{惂}{97077}
\saveTG{㤘}{97077}
\saveTG{}{97077}
\saveTG{𢜁}{97080}
\saveTG{慏}{97080}
\saveTG{惧}{97081}
\saveTG{懙}{97081}
\saveTG{懝}{97081}
\saveTG{𢤽}{97081}
\saveTG{㦏}{97081}
\saveTG{𢣇}{97082}
\saveTG{惞}{97082}
\saveTG{懒}{97082}
\saveTG{惯}{97082}
\saveTG{𢙊}{97082}
\saveTG{𢡜}{97082}
\saveTG{忺}{97082}
\saveTG{𢚾}{97084}
\saveTG{𢙑}{97084}
\saveTG{𢜴}{97084}
\saveTG{懊}{97084}
\saveTG{愌}{97084}
\saveTG{𢜵}{97084}
\saveTG{𢝛}{97084}
\saveTG{𢦅}{97086}
\saveTG{慣}{97086}
\saveTG{𢝢}{97086}
\saveTG{懶}{97086}
\saveTG{𢢾}{97086}
\saveTG{𢤞}{97086}
\saveTG{𪬡}{97086}
\saveTG{𢗜}{97087}
\saveTG{𢘧}{97087}
\saveTG{𢥡}{97089}
\saveTG{憏}{97091}
\saveTG{𢘝}{97092}
\saveTG{㤾}{97094}
\saveTG{㥡}{97094}
\saveTG{𢜸}{97094}
\saveTG{𢟺}{97099}
\saveTG{𪾚}{97102}
\saveTG{𡓢}{97104}
\saveTG{𡮬}{97123}
\saveTG{䣘}{97127}
\saveTG{𩼪}{97136}
\saveTG{𢡌}{97144}
\saveTG{𣪼}{97147}
\saveTG{𫕣}{97174}
\saveTG{𦧾}{97210}
\saveTG{𪞀}{97212}
\saveTG{𫒄}{97215}
\saveTG{耀}{97215}
\saveTG{𠄈}{97217}
\saveTG{䚏}{97217}
\saveTG{𦫢}{97217}
\saveTG{翷}{97220}
\saveTG{𦒪}{97221}
\saveTG{翷}{97221}
\saveTG{𦐶}{97221}
\saveTG{𦐺}{97221}
\saveTG{𠒫}{97221}
\saveTG{𠓂}{97222}
\saveTG{𪇲}{97227}
\saveTG{𪆞}{97227}
\saveTG{䳤}{97227}
\saveTG{䶴}{97227}
\saveTG{𪁺}{97227}
\saveTG{𪀯}{97227}
\saveTG{𨛍}{97227}
\saveTG{䣊}{97227}
\saveTG{𨙹}{97227}
\saveTG{𨞧}{97227}
\saveTG{𫛮}{97227}
\saveTG{鄰}{97227}
\saveTG{𪁎}{97227}
\saveTG{𠓖}{97229}
\saveTG{𢤾}{97247}
\saveTG{𠬵}{97247}
\saveTG{㲂}{97247}
\saveTG{輝}{97252}
\saveTG{辉}{97254}
\saveTG{𦒉}{97262}
\saveTG{𠓑}{97263}
\saveTG{𪄹}{97271}
\saveTG{𦥿}{97277}
\saveTG{𢢚}{97282}
\saveTG{𣢒}{97282}
\saveTG{𪞆}{97294}
\saveTG{𠒽}{97295}
\saveTG{䣣}{97327}
\saveTG{𢦇}{97332}
\saveTG{𢝧}{97333}
\saveTG{𢠒}{97336}
\saveTG{𡥯}{97407}
\saveTG{𡦐}{97407}
\saveTG{𩙜}{97410}
\saveTG{𨛂}{97427}
\saveTG{𡢏}{97440}
\saveTG{𪸲}{97462}
\saveTG{𥹰}{97480}
\saveTG{𠦯}{97532}
\saveTG{𨞴}{97627}
\saveTG{𪇁}{97627}
\saveTG{𨟏}{97627}
\saveTG{𨜜}{97627}
\saveTG{𧼧}{97801}
\saveTG{𩗹}{97810}
\saveTG{飊}{97810}
\saveTG{煈}{97810}
\saveTG{𤊉}{97811}
\saveTG{𤆾}{97811}
\saveTG{炮}{97812}
\saveTG{烅}{97812}
\saveTG{烃}{97812}
\saveTG{𤈓}{97812}
\saveTG{𥺬}{97812}
\saveTG{𤊛}{97812}
\saveTG{炄}{97812}
\saveTG{𤍄}{97812}
\saveTG{𤇶}{97813}
\saveTG{𤏯}{97814}
\saveTG{𤇂}{97814}
\saveTG{𤌆}{97814}
\saveTG{熞}{97814}
\saveTG{㷐}{97814}
\saveTG{𤓛}{97815}
\saveTG{燿}{97815}
\saveTG{𪚱}{97817}
\saveTG{𤒰}{97817}
\saveTG{𤑑}{97817}
\saveTG{𤈦}{97817}
\saveTG{𪸎}{97817}
\saveTG{𤇛}{97817}
\saveTG{𤆤}{97817}
\saveTG{𤒅}{97817}
\saveTG{𦫟}{97817}
\saveTG{𤆵}{97817}
\saveTG{𤊓}{97817}
\saveTG{𤆘}{97817}
\saveTG{熤}{97818}
\saveTG{烱}{97820}
\saveTG{燗}{97820}
\saveTG{燘}{97820}
\saveTG{焖}{97820}
\saveTG{爛}{97820}
\saveTG{熌}{97820}
\saveTG{焩}{97820}
\saveTG{燜}{97820}
\saveTG{𤊵}{97820}
\saveTG{㶯}{97820}
\saveTG{炯}{97820}
\saveTG{爓}{97820}
\saveTG{灱}{97820}
\saveTG{煳}{97820}
\saveTG{焵}{97820}
\saveTG{烔}{97820}
\saveTG{灼}{97820}
\saveTG{焹}{97820}
\saveTG{𪹛}{97821}
\saveTG{𤎜}{97821}
\saveTG{𪸪}{97821}
\saveTG{𤆥}{97821}
\saveTG{𪹆}{97821}
\saveTG{𤑷}{97821}
\saveTG{㶦}{97821}
\saveTG{𤆞}{97822}
\saveTG{𤐛}{97822}
\saveTG{𤆻}{97822}
\saveTG{熮}{97822}
\saveTG{𪹹}{97822}
\saveTG{𤒻}{97822}
\saveTG{㶤}{97823}
\saveTG{𤋢}{97823}
\saveTG{𤓙}{97824}
\saveTG{𪸐}{97824}
\saveTG{𤆳}{97824}
\saveTG{𤈎}{97826}
\saveTG{𤈍}{97826}
\saveTG{𤏐}{97826}
\saveTG{𪸼}{97826}
\saveTG{㶷}{97826}
\saveTG{𪹔}{97826}
\saveTG{𤆈}{97827}
\saveTG{燏}{97827}
\saveTG{炀}{97827}
\saveTG{熓}{97827}
\saveTG{𤇔}{97827}
\saveTG{郯}{97827}
\saveTG{爥}{97827}
\saveTG{焗}{97827}
\saveTG{煱}{97827}
\saveTG{焨}{97827}
\saveTG{煼}{97827}
\saveTG{𪹳}{97827}
\saveTG{𤑺}{97827}
\saveTG{𤏓}{97827}
\saveTG{𤓁}{97827}
\saveTG{𤊳}{97827}
\saveTG{𤍎}{97827}
\saveTG{邩}{97827}
\saveTG{㷌}{97827}
\saveTG{𧣌}{97827}
\saveTG{䲴}{97827}
\saveTG{𤆡}{97827}
\saveTG{㶴}{97827}
\saveTG{䣔}{97827}
\saveTG{𤈤}{97827}
\saveTG{𤍪}{97827}
\saveTG{㷁}{97827}
\saveTG{𤑲}{97827}
\saveTG{熪}{97827}
\saveTG{𤇄}{97828}
\saveTG{𤐤}{97831}
\saveTG{𤊂}{97832}
\saveTG{㷓}{97832}
\saveTG{熥}{97832}
\saveTG{爘}{97832}
\saveTG{𤊺}{97832}
\saveTG{𤉃}{97832}
\saveTG{𤐠}{97832}
\saveTG{𤍑}{97833}
\saveTG{烬}{97833}
\saveTG{炵}{97833}
\saveTG{煺}{97833}
\saveTG{熢}{97835}
\saveTG{燳}{97836}
\saveTG{㷟}{97837}
\saveTG{𤌀}{97837}
\saveTG{𪹙}{97837}
\saveTG{烐}{97840}
\saveTG{炿}{97840}
\saveTG{𤇚}{97841}
\saveTG{𤇜}{97841}
\saveTG{𤋊}{97844}
\saveTG{燖}{97846}
\saveTG{}{97847}
\saveTG{㷆}{97847}
\saveTG{𤓇}{97847}
\saveTG{燬}{97847}
\saveTG{煆}{97847}
\saveTG{煅}{97847}
\saveTG{𤆣}{97847}
\saveTG{𤇜}{97847}
\saveTG{𪹤}{97847}
\saveTG{𤐘}{97847}
\saveTG{𤈧}{97847}
\saveTG{𤒎}{97847}
\saveTG{𤎈}{97847}
\saveTG{𤐢}{97847}
\saveTG{炈}{97847}
\saveTG{𤎋}{97848}
\saveTG{煇}{97852}
\saveTG{𤐃}{97852}
\saveTG{烽}{97854}
\saveTG{𪸩}{97854}
\saveTG{𤑱}{97856}
\saveTG{𤐕}{97856}
\saveTG{𤇺}{97857}
\saveTG{𤌅}{97857}
\saveTG{𤐐}{97861}
\saveTG{𤊻}{97861}
\saveTG{熘}{97862}
\saveTG{熠}{97862}
\saveTG{炤}{97862}
\saveTG{𤉝}{97862}
\saveTG{焔}{97862}
\saveTG{𪸳}{97862}
\saveTG{𤌐}{97863}
\saveTG{𤑟}{97863}
\saveTG{𤉣}{97863}
\saveTG{𤉸}{97864}
\saveTG{烙}{97864}
\saveTG{𤉙}{97867}
\saveTG{煝}{97867}
\saveTG{𤈵}{97870}
\saveTG{煀}{97872}
\saveTG{焰}{97877}
\saveTG{熐}{97880}
\saveTG{㷷}{97881}
\saveTG{𤒩}{97881}
\saveTG{𤑍}{97881}
\saveTG{𤑉}{97881}
\saveTG{𤎏}{97881}
\saveTG{𪸢}{97882}
\saveTG{焮}{97882}
\saveTG{炊}{97882}
\saveTG{歘}{97882}
\saveTG{欻}{97882}
\saveTG{𤏉}{97882}
\saveTG{𤐚}{97882}
\saveTG{𤉚}{97882}
\saveTG{𪹍}{97884}
\saveTG{燠}{97884}
\saveTG{𤋸}{97884}
\saveTG{焕}{97884}
\saveTG{煥}{97884}
\saveTG{㸊}{97886}
\saveTG{𤎽}{97886}
\saveTG{𤏌}{97889}
\saveTG{𤒾}{97889}
\saveTG{𪸞}{97892}
\saveTG{𤏦}{97893}
\saveTG{𪹻}{97894}
\saveTG{𪸨}{97894}
\saveTG{𤌧}{97894}
\saveTG{燦}{97894}
\saveTG{煣}{97894}
\saveTG{𤉾}{97894}
\saveTG{𤊒}{97899}
\saveTG{𤑬}{97899}
\saveTG{𣒓}{97904}
\saveTG{籶}{97910}
\saveTG{䊃}{97910}
\saveTG{𩘽}{97910}
\saveTG{籸}{97910}
\saveTG{粗}{97912}
\saveTG{𤇅}{97912}
\saveTG{𪹁}{97912}
\saveTG{粈}{97912}
\saveTG{㷥}{97913}
\saveTG{𥺣}{97914}
\saveTG{䊮}{97915}
\saveTG{𥺕}{97917}
\saveTG{䊊}{97917}
\saveTG{𥼗}{97917}
\saveTG{𥼡}{97917}
\saveTG{粑}{97917}
\saveTG{𥸱}{97917}
\saveTG{粷}{97920}
\saveTG{糊}{97920}
\saveTG{𥸾}{97920}
\saveTG{粅}{97920}
\saveTG{籾}{97920}
\saveTG{粡}{97920}
\saveTG{𫃐}{97923}
\saveTG{㶲}{97924}
\saveTG{𥹝}{97926}
\saveTG{𥺂}{97926}
\saveTG{𥺝}{97926}
\saveTG{𥼴}{97926}
\saveTG{𫃃}{97927}
\saveTG{𪇒}{97927}
\saveTG{𪂈}{97927}
\saveTG{𥻼}{97927}
\saveTG{𪀿}{97927}
\saveTG{𫃈}{97927}
\saveTG{𥻤}{97927}
\saveTG{𥺵}{97927}
\saveTG{糑}{97927}
\saveTG{糈}{97927}
\saveTG{糏}{97927}
\saveTG{𥸲}{97927}
\saveTG{𥹠}{97927}
\saveTG{𨞇}{97927}
\saveTG{𥸬}{97927}
\saveTG{𥽭}{97929}
\saveTG{𫃌}{97932}
\saveTG{𫃓}{97932}
\saveTG{𥼙}{97933}
\saveTG{𥽩}{97935}
\saveTG{糔}{97936}
\saveTG{䊚}{97937}
\saveTG{粣}{97940}
\saveTG{粋}{97941}
\saveTG{𥺍}{97941}
\saveTG{𥹣}{97944}
\saveTG{𤇒}{97944}
\saveTG{𥸼}{97945}
\saveTG{𥺱}{97947}
\saveTG{𫃂}{97947}
\saveTG{籽}{97947}
\saveTG{𥼹}{97947}
\saveTG{𥺑}{97947}
\saveTG{𣪿}{97947}
\saveTG{𥸩}{97947}
\saveTG{䊛}{97947}
\saveTG{𪺀}{97948}
\saveTG{𥹾}{97954}
\saveTG{𥹤}{97955}
\saveTG{𥹪}{97957}
\saveTG{𥺸}{97961}
\saveTG{䊅}{97962}
\saveTG{𥹙}{97962}
\saveTG{𥺢}{97964}
\saveTG{𥻡}{97967}
\saveTG{𥻓}{97968}
\saveTG{𥺷}{97972}
\saveTG{𤏭}{97977}
\saveTG{𥻩}{97980}
\saveTG{𥺎}{97981}
\saveTG{𣤹}{97982}
\saveTG{𥽁}{97982}
\saveTG{𥺏}{97982}
\saveTG{𥸷}{97982}
\saveTG{糇}{97984}
\saveTG{𥺶}{97984}
\saveTG{𥼻}{97986}
\saveTG{粎}{97987}
\saveTG{𥻐}{97991}
\saveTG{𫂼}{97992}
\saveTG{𥽆}{97994}
\saveTG{糅}{97994}
\saveTG{粶}{97999}
\saveTG{𢗅}{98010}
\saveTG{怍}{98011}
\saveTG{𢚮}{98011}
\saveTG{悦}{98012}
\saveTG{悅}{98012}
\saveTG{𢠏}{98012}
\saveTG{𢥸}{98012}
\saveTG{𢘬}{98012}
\saveTG{㦈}{98012}
\saveTG{怆}{98012}
\saveTG{恱}{98012}
\saveTG{懢}{98012}
\saveTG{𢞂}{98012}
\saveTG{𪬘}{98012}
\saveTG{𢛰}{98012}
\saveTG{𢘐}{98012}
\saveTG{𪬌}{98012}
\saveTG{恮}{98014}
\saveTG{愾}{98017}
\saveTG{忔}{98017}
\saveTG{𢥊}{98017}
\saveTG{𢠥}{98017}
\saveTG{𪬣}{98017}
\saveTG{𢥆}{98017}
\saveTG{𢛹}{98017}
\saveTG{忾}{98017}
\saveTG{𪫺}{98019}
\saveTG{惍}{98019}
\saveTG{忦}{98020}
\saveTG{𢤣}{98021}
\saveTG{愉}{98021}
\saveTG{𢡁}{98027}
\saveTG{𢥣}{98027}
\saveTG{㤋}{98027}
\saveTG{彆}{98027}
\saveTG{惀}{98027}
\saveTG{忴}{98027}
\saveTG{悌}{98027}
\saveTG{慃}{98027}
\saveTG{㦢}{98027}
\saveTG{𢢬}{98027}
\saveTG{慯}{98027}
\saveTG{𢗴}{98027}
\saveTG{𢛌}{98030}
\saveTG{𢛒}{98031}
\saveTG{憮}{98031}
\saveTG{㥞}{98032}
\saveTG{𢢊}{98032}
\saveTG{𢤸}{98032}
\saveTG{𢟣}{98032}
\saveTG{怜}{98032}
\saveTG{惗}{98032}
\saveTG{忪}{98032}
\saveTG{懩}{98032}
\saveTG{𢛨}{98033}
\saveTG{𢢝}{98033}
\saveTG{𢢴}{98034}
\saveTG{慊}{98037}
\saveTG{忤}{98040}
\saveTG{𢚲}{98040}
\saveTG{㦑}{98040}
\saveTG{憞}{98040}
\saveTG{𢠵}{98040}
\saveTG{𢠂}{98040}
\saveTG{𢝽}{98040}
\saveTG{𢠨}{98040}
\saveTG{憿}{98040}
\saveTG{慠}{98040}
\saveTG{𢠇}{98040}
\saveTG{𢣈}{98040}
\saveTG{𢠳}{98040}
\saveTG{𢗡}{98040}
\saveTG{恲}{98041}
\saveTG{𢡩}{98043}
\saveTG{𢜰}{98046}
\saveTG{𢚔}{98047}
\saveTG{𪭆}{98047}
\saveTG{𢝡}{98047}
\saveTG{愎}{98047}
\saveTG{𢙋}{98048}
\saveTG{𢥌}{98051}
\saveTG{𢣂}{98053}
\saveTG{𢤻}{98053}
\saveTG{𪬻}{98053}
\saveTG{𪬶}{98054}
\saveTG{𢤝}{98056}
\saveTG{惮}{98056}
\saveTG{𢠤}{98056}
\saveTG{悔}{98057}
\saveTG{𥣬}{98060}
\saveTG{𪭇}{98061}
\saveTG{𢢏}{98061}
\saveTG{恰}{98061}
\saveTG{𢢆}{98061}
\saveTG{㤷}{98062}
\saveTG{㥢}{98064}
\saveTG{憎}{98066}
\saveTG{𢠑}{98066}
\saveTG{懀}{98066}
\saveTG{愴}{98067}
\saveTG{𢡽}{98068}
\saveTG{𪫹}{98068}
\saveTG{懀}{98068}
\saveTG{𢙹}{98080}
\saveTG{𢠰}{98082}
\saveTG{𢞖}{98084}
\saveTG{𢝹}{98084}
\saveTG{𢠈}{98086}
\saveTG{憸}{98086}
\saveTG{𢣴}{98089}
\saveTG{憡}{98092}
\saveTG{悇}{98094}
\saveTG{𡐞}{98104}
\saveTG{鳖}{98106}
\saveTG{鐅}{98109}
\saveTG{龞}{98117}
\saveTG{𪛆}{98117}
\saveTG{𫜁}{98127}
\saveTG{蟞}{98136}
\saveTG{𧏟}{98136}
\saveTG{㴨}{98184}
\saveTG{𥻋}{98200}
\saveTG{𨮠}{98212}
\saveTG{斃}{98212}
\saveTG{䨆}{98215}
\saveTG{𠒳}{98217}
\saveTG{𧢍}{98217}
\saveTG{𠓇}{98221}
\saveTG{𦠞}{98227}
\saveTG{𠓓}{98227}
\saveTG{𢄞}{98227}
\saveTG{𠟈}{98227}
\saveTG{幣}{98227}
\saveTG{幤}{98227}
\saveTG{𢙓}{98231}
\saveTG{𠓁}{98232}
\saveTG{敝}{98240}
\saveTG{敞}{98240}
\saveTG{𢽐}{98240}
\saveTG{𢿽}{98240}
\saveTG{𢼴}{98240}
\saveTG{𢿻}{98240}
\saveTG{𣱔}{98242}
\saveTG{𠒡}{98243}
\saveTG{𩠦}{98262}
\saveTG{𪨄}{98286}
\saveTG{鷩}{98327}
\saveTG{𩦉}{98327}
\saveTG{𪅶}{98327}
\saveTG{㥹}{98332}
\saveTG{𢢌}{98334}
\saveTG{憋}{98334}
\saveTG{𢞽}{98334}
\saveTG{鱉}{98336}
\saveTG{𩻪}{98336}
\saveTG{𣁢}{98400}
\saveTG{𦗥}{98401}
\saveTG{嫳}{98404}
\saveTG{𡦤}{98407}
\saveTG{𠢪}{98427}
\saveTG{敩}{98440}
\saveTG{数}{98440}
\saveTG{𪪷}{98442}
\saveTG{弊}{98444}
\saveTG{㢢}{98444}
\saveTG{撆}{98502}
\saveTG{暼}{98604}
\saveTG{瞥}{98604}
\saveTG{𤑓}{98617}
\saveTG{氅}{98714}
\saveTG{𣰉}{98715}
\saveTG{鼈}{98717}
\saveTG{鄨}{98717}
\saveTG{𤮕}{98717}
\saveTG{𪔀}{98717}
\saveTG{𣀴}{98717}
\saveTG{𧝬}{98722}
\saveTG{𧝟}{98732}
\saveTG{𦦢}{98777}
\saveTG{蹩}{98801}
\saveTG{𠔷}{98801}
\saveTG{𡚁}{98804}
\saveTG{獘}{98804}
\saveTG{𡙢}{98804}
\saveTG{𧸁}{98806}
\saveTG{𧷍}{98806}
\saveTG{𤎨}{98808}
\saveTG{烂}{98811}
\saveTG{炸}{98811}
\saveTG{爁}{98812}
\saveTG{爦}{98812}
\saveTG{炝}{98812}
\saveTG{炧}{98812}
\saveTG{烇}{98814}
\saveTG{熂}{98817}
\saveTG{𤆢}{98817}
\saveTG{𤑸}{98817}
\saveTG{𤍯}{98817}
\saveTG{𪸏}{98817}
\saveTG{𪸕}{98817}
\saveTG{𤈷}{98819}
\saveTG{炌}{98820}
\saveTG{㷙}{98821}
\saveTG{𤓃}{98821}
\saveTG{𤇪}{98822}
\saveTG{𤓀}{98826}
\saveTG{㷍}{98827}
\saveTG{𤆶}{98827}
\saveTG{𥼝}{98827}
\saveTG{𤐯}{98827}
\saveTG{𤍼}{98827}
\saveTG{爚}{98827}
\saveTG{㸅}{98827}
\saveTG{𤌏}{98827}
\saveTG{焍}{98827}
\saveTG{𤍺}{98827}
\saveTG{熻}{98827}
\saveTG{𤒘}{98827}
\saveTG{㷻}{98831}
\saveTG{𤍏}{98831}
\saveTG{𤐹}{98831}
\saveTG{炩}{98832}
\saveTG{焾}{98832}
\saveTG{煫}{98832}
\saveTG{炂}{98832}
\saveTG{𤎔}{98832}
\saveTG{𤒪}{98832}
\saveTG{𤏢}{98832}
\saveTG{𤐄}{98832}
\saveTG{烩}{98832}
\saveTG{燧}{98833}
\saveTG{焧}{98833}
\saveTG{𤉪}{98833}
\saveTG{𤐑}{98834}
\saveTG{熑}{98837}
\saveTG{𪹲}{98840}
\saveTG{𤋖}{98840}
\saveTG{𢾗}{98840}
\saveTG{𤏮}{98840}
\saveTG{𤒦}{98840}
\saveTG{𤏰}{98840}
\saveTG{𤊦}{98840}
\saveTG{燩}{98840}
\saveTG{炇}{98840}
\saveTG{𪹖}{98840}
\saveTG{燉}{98840}
\saveTG{𣀱}{98840}
\saveTG{𤑈}{98840}
\saveTG{𤏽}{98841}
\saveTG{燇}{98846}
\saveTG{𤋟}{98847}
\saveTG{𤐆}{98851}
\saveTG{㷣}{98851}
\saveTG{烊}{98851}
\saveTG{燨}{98853}
\saveTG{爔}{98853}
\saveTG{𤍃}{98854}
\saveTG{}{98856}
\saveTG{𤏫}{98857}
\saveTG{烸}{98857}
\saveTG{㷽}{98861}
\saveTG{烚}{98861}
\saveTG{𤏧}{98861}
\saveTG{焓}{98862}
\saveTG{煪}{98864}
\saveTG{熷}{98866}
\saveTG{燴}{98866}
\saveTG{熗}{98867}
\saveTG{𤎳}{98868}
\saveTG{𤍸}{98868}
\saveTG{焀}{98868}
\saveTG{𤈒}{98872}
\saveTG{𤌯}{98884}
\saveTG{𪹏}{98884}
\saveTG{烪}{98884}
\saveTG{𥽋}{98886}
\saveTG{㷿}{98886}
\saveTG{𤏆}{98894}
\saveTG{𪹘}{98894}
\saveTG{䌘}{98903}
\saveTG{𥹁}{98911}
\saveTG{𤉔}{98911}
\saveTG{𥸽}{98912}
\saveTG{粚}{98912}
\saveTG{糮}{98912}
\saveTG{𣀽}{98916}
\saveTG{𥺡}{98917}
\saveTG{𥹲}{98917}
\saveTG{籺}{98917}
\saveTG{𥼕}{98918}
\saveTG{𥺻}{98919}
\saveTG{糋}{98921}
\saveTG{𥽪}{98927}
\saveTG{𥺽}{98927}
\saveTG{粉}{98927}
\saveTG{𥺀}{98927}
\saveTG{𫃖}{98927}
\saveTG{𤐎}{98928}
\saveTG{𥺴}{98930}
\saveTG{𥼣}{98931}
\saveTG{𥼎}{98931}
\saveTG{糕}{98931}
\saveTG{𥺩}{98931}
\saveTG{糍}{98932}
\saveTG{𥻖}{98932}
\saveTG{𥹕}{98932}
\saveTG{𫃕}{98933}
\saveTG{𫃄}{98933}
\saveTG{𤓚}{98935}
\saveTG{𥽌}{98936}
\saveTG{𥽎}{98937}
\saveTG{𥻧}{98937}
\saveTG{𤍗}{98940}
\saveTG{𥼼}{98940}
\saveTG{𥼲}{98940}
\saveTG{敉}{98940}
\saveTG{糤}{98940}
\saveTG{𢿵}{98940}
\saveTG{𥹎}{98941}
\saveTG{𢽻}{98948}
\saveTG{䊈}{98957}
\saveTG{粭}{98961}
\saveTG{𤏲}{98962}
\saveTG{糩}{98966}
\saveTG{𥻲}{98967}
\saveTG{𥼪}{98968}
\saveTG{𥼖}{98980}
\saveTG{𥻙}{98984}
\saveTG{䊴}{98986}
\saveTG{𥺌}{98994}
\saveTG{}{99000}
\saveTG{𢖹}{99000}
\saveTG{恍}{99012}
\saveTG{惓}{99012}
\saveTG{𢣙}{99014}
\saveTG{憆}{99014}
\saveTG{𢥂}{99027}
\saveTG{𢛎}{99027}
\saveTG{惝}{99027}
\saveTG{𢛗}{99027}
\saveTG{𪬾}{99027}
\saveTG{悄}{99027}
\saveTG{憦}{99027}
\saveTG{戃}{99031}
\saveTG{𢤨}{99031}
\saveTG{𢞞}{99039}
\saveTG{㥪}{99044}
\saveTG{㦪}{99047}
\saveTG{怑}{99050}
\saveTG{𢣶}{99057}
\saveTG{憐}{99059}
\saveTG{𢜫}{99062}
\saveTG{𢡾}{99062}
\saveTG{𢥏}{99062}
\saveTG{慻}{99068}
\saveTG{𢟷}{99069}
\saveTG{悩}{99072}
\saveTG{愀}{99080}
\saveTG{𢤗}{99086}
\saveTG{𢥥}{99089}
\saveTG{惔}{99089}
\saveTG{𢘺}{99094}
\saveTG{𢟱}{99099}
\saveTG{瑩}{99103}
\saveTG{𤐻}{99104}
\saveTG{塋}{99104}
\saveTG{𤯵}{99105}
\saveTG{𤍩}{99108}
\saveTG{䝁}{99108}
\saveTG{鎣}{99109}
\saveTG{𨧿}{99109}
\saveTG{𢄋}{99117}
\saveTG{𦟴}{99117}
\saveTG{䎕}{99127}
\saveTG{螢}{99136}
\saveTG{𥰱}{99136}
\saveTG{𨨝}{99164}
\saveTG{𥯛}{99164}
\saveTG{𪽸}{99185}
\saveTG{𥲚}{99191}
\saveTG{覮}{99212}
\saveTG{𠓃}{99215}
\saveTG{𡮡}{99217}
\saveTG{𠙦}{99217}
\saveTG{𤈺}{99217}
\saveTG{𠒧}{99217}
\saveTG{焭}{99217}
\saveTG{𡀸}{99227}
\saveTG{䈪}{99227}
\saveTG{𥬅}{99227}
\saveTG{𤍔}{99227}
\saveTG{𤍀}{99227}
\saveTG{膋}{99227}
\saveTG{𤬐}{99232}
\saveTG{𪯞}{99248}
\saveTG{𤌠}{99257}
\saveTG{𥬫}{99257}
\saveTG{𠓐}{99294}
\saveTG{}{99327}
\saveTG{鶯}{99327}
\saveTG{鶑}{99327}
\saveTG{憥}{99332}
\saveTG{䉞}{99335}
\saveTG{𢥒}{99339}
\saveTG{𤇾}{99370}
\saveTG{𣁧}{99400}
\saveTG{𦖽}{99401}
\saveTG{𣂈}{99402}
\saveTG{嫈}{99404}
\saveTG{𥻾}{99407}
\saveTG{𤏻}{99407}
\saveTG{𡦃}{99407}
\saveTG{夑}{99407}
\saveTG{燮}{99407}
\saveTG{𤍛}{99407}
\saveTG{𤎬}{99407}
\saveTG{𣀢}{99408}
\saveTG{煢}{99417}
\saveTG{𪳟}{99417}
\saveTG{㷀}{99417}
\saveTG{𤎤}{99420}
\saveTG{勞}{99427}
\saveTG{𠣁}{99427}
\saveTG{𡠺}{99442}
\saveTG{犖}{99502}
\saveTG{𨍶}{99506}
\saveTG{𥲮}{99520}
\saveTG{𪟶}{99527}
\saveTG{𤌌}{99552}
\saveTG{𤍧}{99553}
\saveTG{䁝}{99601}
\saveTG{䪯}{99601}
\saveTG{㽦}{99601}
\saveTG{謍}{99601}
\saveTG{䃕}{99602}
\saveTG{㽦}{99602}
\saveTG{𥮦}{99603}
\saveTG{醟}{99604}
\saveTG{𥲍}{99606}
\saveTG{營}{99606}
\saveTG{𤏃}{99621}
\saveTG{𪹽}{99621}
\saveTG{𥯶}{99644}
\saveTG{𤐼}{99717}
\saveTG{甇}{99717}
\saveTG{䉇}{99723}
\saveTG{褮}{99732}
\saveTG{嵤}{99772}
\saveTG{𡹚}{99772}
\saveTG{罃}{99772}
\saveTG{𤐔}{99802}
\saveTG{𤌡}{99804}
\saveTG{𪥢}{99804}
\saveTG{𤋍}{99805}
\saveTG{熒}{99809}
\saveTG{㸉}{99809}
\saveTG{爕}{99809}
\saveTG{𤐺}{99809}
\saveTG{𪺃}{99814}
\saveTG{𪺌}{99814}
\saveTG{𤎌}{99814}
\saveTG{𤑚}{99814}
\saveTG{熦}{99815}
\saveTG{𤈛}{99817}
\saveTG{𤉼}{99817}
\saveTG{炒}{99820}
\saveTG{𤉤}{99827}
\saveTG{焇}{99827}
\saveTG{龦}{99827}
\saveTG{𤏪}{99827}
\saveTG{爣}{99831}
\saveTG{𤋩}{99840}
\saveTG{𤋏}{99844}
\saveTG{𤊀}{99847}
\saveTG{𫃘}{99850}
\saveTG{𤐪}{99857}
\saveTG{燐}{99859}
\saveTG{𤑝}{99862}
\saveTG{𤒨}{99868}
\saveTG{𤇻}{99877}
\saveTG{炏}{99880}
\saveTG{煍}{99880}
\saveTG{𤓔}{99884}
\saveTG{燚}{99889}
\saveTG{㷋}{99889}
\saveTG{爃}{99894}
\saveTG{𤒇}{99894}
\saveTG{𤇿}{99894}
\saveTG{𥚡}{99901}
\saveTG{禜}{99901}
\saveTG{滎}{99902}
\saveTG{縈}{99903}
\saveTG{𥼬}{99904}
\saveTG{榮}{99904}
\saveTG{𤎊}{99905}
\saveTG{𤌟}{99909}
\saveTG{糛}{99914}
\saveTG{䊎}{99917}
\saveTG{𫃔}{99917}
\saveTG{𥺇}{99919}
\saveTG{𥻠}{99920}
\saveTG{粆}{99920}
\saveTG{𥹶}{99927}
\saveTG{䊑}{99927}
\saveTG{𫃑}{99927}
\saveTG{𥽻}{99931}
\saveTG{䉽}{99941}
\saveTG{𣓳}{99948}
\saveTG{𥼭}{99957}
\saveTG{𥼽}{99966}
\saveTG{𥹥}{99977}
\saveTG{䊏}{99989}
\saveTG{𡮐}{99990}
\saveTG{𥹫}{99994}
\saveTG{𥼫}{99994}
\saveTG{𥣻}{99994}
\saveTG{𥼮}{99994}
\saveTG{𥻒}{99994}
\saveTG{檾}{99994}
\saveTG{𥻕}{99995}
%%
%% End of file `tetrogonos-database.tex'

		%%
		%% End of file `tetrogonos.sty'
	\end{Verbatim}
	\end{adjustwidth}
\endgroup

汉字到四角号码的映射储存在另外的文件 tetragonos-example.def 中。其正文开头数行如下:

\begingroup
	\begin{adjustwidth}{3em}{3em}
	\footnotesize
	\begin{Verbatim}[numbers=left,numbersep=1ex,gobble=2,formatcom=\color{darkmiku},firstnumber=22]
		\ProvidesFile{tetragonos-database.def}[2019/01/14 v1 tetragonos database]
		\saveTG{亠}{00000}
		\saveTG{弯}{00027}
		\saveTG{亪}{00037}
		\saveTG{疒}{00100}
		\saveTG{韲}{00101}
	\end{Verbatim}
	\end{adjustwidth}
\endgroup

\end{document}
