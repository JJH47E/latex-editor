\documentclass{article}
\usepackage[russian]{babel}

\usepackage{url}

\usepackage{fontspec}
\setmainfont{CMU Serif}

\author{А.\,Б.\,Шипунов\footnote{e-mail: \texttt{plantago at
herba.msu.ru}}}

\title{\texttt{xecyr.sty},\\ русский стиль для \texttt{XeLaTeX}}

\date{}

\begin{document}

\maketitle

Стиль основан на файле А.\,Крюкова <<\texttt{xecyr.tex}>>.

Доступны две пары опций: \texttt{ext/noext} и \texttt{nomis/mis}
(умалчиваемые опции первые). Первая грузит пакет \texttt{xltxtra} (и
заодно \texttt{fontspec}, без которого жизнь в \texttt{XeLaTeX} осложняется),
вторая грузит (если ее задать явно) пакет \texttt{misccorr}, для загрузки
которого нужен некий трюк с кодировками.

Кавычки набираются "<по-бабелевски">, при помощи \verb|"< "> "` "'|,
лигатуры \verb|<< >>| будут работать только если в поставке имеются
последние версии файлов \texttt{tex-text.map} и \texttt{tex-text.tec},
скачанные с \url{http://scripts.sil.org/svn-view/xetex/TRUNK}. В будущем
эта ситуация, скорее всего, изменится.

В системе должны быть
установлены шрифты Computer Modern Unicode
(\url{http://sourceforge.net/projects/cm-unicode}).

\end{document}
