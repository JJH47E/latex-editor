\listfiles
%% 
%%  Ein Beispiel der DANTE-Edition
%% 
%% 
%%  Copyright (C) 2012 Herbert Voss
%% 
%%  It may be distributed and/or modified under the conditions
%%  of the LaTeX Project Public License, either version 1.3
%%  of this license or (at your option) any later version.
%% 
%%  See http://www.latex-project.org/lppl.txt for details.
%% 
%% 
%% ==== 
% Show page(s) 1,3
%% 
\documentclass[landscape,rules]{seminar}
\pagestyle{empty}
\setlength\textwidth{172.40709pt}
% use **dvips -T 297mm,210mm <file>**!
\usepackage[utf8]{inputenc}
\usepackage[scaled]{helvet}
\def\FileDate{June 2002}
\def\FileInfoB{Backgrounds}
\def\FileVersion{1.0}
\usepackage{sem-dem}% General utility macros
\HyperSetUp\hypersetup{pdfpagemode=UseOutlines}% Open the document with bookmarks shown
\newcommand\MySeminarOutlinePresentation{%
  \SeminarOutlinePresentation{Where we are}{LightBlue}{RoyalBlue}{yellow}%
  \SeminarHeader{\LARGE\HLe{Demonstration of \MakeLowercase{\FileInfoB}}}}
\VerbatimFootnotes              % To allow verbatim material in footnotes
\begin{document}
\SeminarFirstSlide  \SeminarListOfSlides
\renewcommand{\slidestretch}{1}% Komprimieren der Zeilen
\begin{slide}
  \slideheading{Introduction}
  \begin{dinglist}{\DingListSymbolA}
    \item The \emph{full screen} backgrounds, which are the ones well suited
    for screen presentations, are managed by the
    \Verb+\SeminarNewSlideFrameBackground+ and
    \Verb+\SeminarSlideFrameBackground+ macros.\footnote{They are built on the
    model of the \Verb[fontsize=\footnotesize]+\newslideframe+ and
    \Verb[fontsize=\footnotesize]+\slideframe+ ones, which cannot be used for
    this goal (there is a design incompatibility in the management of the
    slide headers).}

    \item We can overlap several partial \emph{backgrounds} with the
    `\textsf{fancybox}' package (using it \Verb+\boxput+ macro), calling the
    \Verb+\slideframe[*]+ macro, eventually several times (see the
    \HLe{Seminar} documentation and FAQ). See examples on
    slide~12 and in the file \href{file:sem-dem2.pdf}{sem-dem2.pdf}.

    \item We show here several kinds of backgrounds:
    \begin{dinglist}{\DingListSymbolB}\setlength{\labelwidth}{4mm}
      \item \HLe{solid} backgrounds,
      \item \HLe{gradient} backgrounds, using the \textsf{pst-grad} PSTricks package,
      \item \HLe{gradient} backgrounds, using the \textsf{pst-slpe} PSTricks package,
      \item \HLe{composite} backgrounds, based on PSTricks graphics (you can
      obviously rather use the others \AllTeX{} packages for algorithmic graphics),
      \item backgrounds with \HLe{external images}.
    \end{dinglist}
  \end{dinglist}
\end{slide}
\end{document}
