\documentclass[a4paper,10pt]{artikel1} % or just 'article'
\usepackage{shortvrb}
\newcommand\Lopt[1]{\textsf{#1}}
\newcommand\file[1]{\texttt{#1}}

\begin{document}

\title{An introduction to the Dutch \LaTeX\ document
  classes\thanks{Updated for \LaTeXe\ by Johannes Braams, 6 february
    1994}}
\author{Victor Eijkhout}
\date{3 September 1989}
\maketitle

\section{Background}

The Dutch \LaTeX\ document classes are an attempt to provide \LaTeX\ 
users with facilities of the standard distribution document classes
`article' and `report', but coupled to typography that is geared
towards Dutch usage.  In order to ensure interchangeability, the new
styles implemented the commands of the distribution styles.  In
principle, then, no manual for these styles would be needed.  There
are a few points, however, that can't go without comment.

Table 1 lists all files connected with the Dutch document classes.
These files can be found on the \texttt{tex-nl} fileserver
on the \texttt{hearn} node of Bitnet/Earn and on the \textsc{ctan}.
Not all of them are necessary, the ones with extension \file{.tex}
contain just blah.

Remark: there exists a Dutch version of `letter', which is so
specifically Dutch that I decided not to include it in this list.

\begin{table}[hbtp]
  \begin{center}
    \begin{tabular}{|@{\tt\hspace{1em}}r|l|}
      \hline
      ntgclass.tex & The \LaTeXe\ source for this text\\
      ntgclass.dtx & The source for all the class files\\
      ntgclass.dst & A \textsf{docstrip} program to produce\\
                   & the class files from the source.\\
      artikel1.cls  & article compatible, design 1, straightforward \\
      artikel2.cls  & article compatible, design 2, rather different \\
      artikel3.cls  & article compatible, design 3, 
                      parskip instead of indent\\
      rapport1.cls  & report compatible, design 1 \\
      rapport3.cls  & report compatible, design 3 \\
      book.cls      & book compatible, design 1\\
      ntg10.clo     & 10 point option for all styles \\
      ntg11.clo     & 11 point option for all styles \\
      ntg12.clo     & 12 point option for all styles \\
      artdoc.tex    & the genesis of the `artikel' classes, in Dutch \\
      briefdoc.tex  & the genesis of the `brief class, also in Dutch \\
      rapdoc.tex    & the genesis of the `rapport' classes, also in
                      Dutch.\\
    \hline
    \end{tabular}
  \end{center}
  \caption{List of files}
\end{table}


\section{Languages}

These styles have been developed bearing in mind suggestions of Hubert
Partl for making the language of styles switchable.  Thus, on their
own these styles will produce English captions like `chapter' and
`table of contents', but specifying the options `dutch' or `german',
or any language option that follows the directions of Partl, will
switch these to the language of the option.

\MakeShortVerb{\|}

\section{Design}

At the moment there are styles compatible with `article' and with
`report'. The Dutch names for these are `artikel' and `rapport'. In
contrast to the standard styles, however, the user can now choose from
different visual designs.  Names of the styles are formed by appending
their number to the name, for instance `rapport3'.
\begin{enumerate}
\item Design one is meant to be a universally acceptable style.  It
  has been kept as uncontroversial as possible.  Under heavy protests
  of the implementer (me) the one point that has turned out to be
  controversial, the table of contents, has been made subject to a
  switch that can restore the old \LaTeX\ layout. Explanation of this
  follows.
\item Design two will probably never be heavily used. It is more
  something of a heroic attempt to be different. At the moment only
  available in `artikel' form.
\item Design three meets the wishes of people who like a zero
  |\parindent| and a positive |\parskip|.  As just setting these
  parameters within for instance `artikel1' will give some unwanted
  side effects, I~decided to repair these in a separate style.
\end{enumerate}
Credits for the visual design go to one real-life designer and a
couple of books by designers I consulted.  The full story can be found
elsewhere.

\subsection{User options}

As was mentioned above, the new layout of the table of contents has
turned out to be somewhat controversial.  So, in order to ensure wider
acceptance of these styles I~have incorporated a switch that will
restore the old layout. Just specify the \Lopt{oldtoc} option.  This is
available in designs 1 and~3. Number~2 really insists on being
different.

%I~do no page sizing, so kindly use an option like \texttt{a4}.
%There is a nice one on the server, which is not the one
%by John Pavel.

\subsection{Option files}

Like in the standard styles there exist option files for $10/11/12$
point layout. Until now, however, I~have managed to get away with
using the same option files for both the `artikel' `rapport', and
`boek' styles.  The option files then have to have some neutral names.
Which are at the moment \file{ntg10}, \file{ntg11},
and~\file{ntg12}. `NTG', of course, stands for Nederlandse \TeX\ 
Gebruikersgroep, i.e., Dutch \TeX\ Users Group.

When doing the `rapport' styles, I~needed to modify the `\file{titlepag}'
option file. Thus there is a Dutch version of that, bearing the name
of `voorwerk' (literal translation: `frontwork').  Probably the
majority of Dutch \LaTeX\ users don't even know that this is the
correct term.  You're never too old to learn.

With the \LaTeXe\ version of thes classes the file \file{voorwerk.sty}
has disappeared. It has been turned into the internal option
\Lopt{voorwerk}, just like \file{titlpage.sty} has disappeared into
the internal option \Lopt{titlepage}.
\end{document}

