\documentclass[a4paper]{article}
\usepackage[a4paper]{geometry}
\usepackage{miscdoc}
\usepackage[scaled=0.85]{luximono}
\begin{document}
\title{The \Package{sansmath} package}
\author{Donald Arseneau\thanks{Documentation file assembled by Robin
    Fairbairns}}
\date{2003-08-13, version 1.0}
\maketitle

\section{Outline}
The package is designed to offer sans-serif mathematics in the absence
of proper sans maths fonts.

The package's name could be misconstrued: there was an ambition to
do the job for ``non-standard'' sans fonts (as indicated by the value
of \cs{sfdefault}), but the only good results have been with Computer
Modern and \FontName{cmss}.

\section{Use}

After \cmdinvoke{usepackage}{sansmath}, a new ``math version''
\texttt{sans} is defined, together with a command \cs{sansmath}, which
behaves as \cs{boldmath} does.

There is also a command \cs{unsansmath} (which does what you might
imagine), but if maths are to be sans-serif for a limited area of
document, it is better to limit it to a local group, for example by
\cmdinvoke{begin}{sansmath} \texttt{\dots} \cmdinvoke{end}{sansmath}

Within the scope of the \cs{sansmath} declaration, maths characters
will be taken from the text sans-serif family as much as possible.
The actual sans fonts are OT1 encodings of those indicated by the 
meaning of \cs{sfdefault} \emph{WHEN THE PACKAGE WAS LOADED}, not the
meaning at each maths environment!

Since the \texttt{OT1} text fonts do not provide the lower-case greek letters,
there is a package option \option[eulergreek] to take the lowercase
greek from the Euler maths fonts.

Since some (many) sans fonts have no uppercase greek letters either
(missing characters from the \texttt{OT1} encoding), there is an
option \option[EULERGREEK] to take \emph{all} greek letters from the
euler fonts.  In this case one should also investigate using
\FontName{Euler} fonts for \emph{all maths} in the document, using
package \Package{euler} instead of this one!

\texttt{OT1} encoding is normally required to get the uppercase greek
letters, but if you use the \option[EULERGREEK] option or don't use
any uppercase greek letters, then you are welcome to define
\cs{sansmathencoding} \emph{before} loading this package.  There is
also a package option \option[T1] to perform that particuler
definition.  Note the comment above about only \FontName{cmss} being
good~--- even the \texttt{T1}-encoded \FontName{ec} fonts are poor
substitutes.

The package achieves maths-italic by reloading the slanted version of
the text sans-serif font, and changing a \texttt{fontdimen} parameter
(spaceskip).  This causes the italic correction to be applied between
letters (good) but does not break up the `fi' and `fl' ligatures
(bad).  (Why does a sans font have these ligatures anyway?)  As yet,
nothing is done about this bug.

\end{document}
