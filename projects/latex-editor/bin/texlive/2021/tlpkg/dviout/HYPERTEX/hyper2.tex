\documentstyle[a4j, myhyper]{jarticle}
\title{dviout for Windows\\
       and\\
       Hyper\TeX}
\author{SHIMA}
\date{2004年 5月 1\name{date}{日}}
\begin{document}
\maketitle
\section{ようこそ}
ここから、別のdviファイルやそのほかのファイルへジャンプすることができます。
どれでも選んでください(色の変わった部分をマウスの左ボタンでクリック)。

\ 1. \href{file:hyperdvi.dvi#jump}{hyperdvi}\name{jump}{}: HyperTeX の案内

\ 2. \href{file:..\sample\sample.dvi}{sample}: A sample(英語)

\ 3. \href{file:..\test_a4.dvi}{test\_a4}: A4版印刷での原点調整用(英語)

\ 4. \href{file:..\test_b5.dvi}{test\_b5}: B5版印刷での原点調整用(日本語)

\ 5. \href{file:..\test_b5e.dvi}{test\_b5e}: 欧州式B5版印刷での原点調整用(英語)

\ 6. \href{file:..\DOC\tex_dvioutw.html}{tex\_dvioutw}: dviout のインストールと使い方のヒント(HTML文書)

\ 7. \href{file:..\DOC\dvioutQA.html}{dvioutQA}: dviout におけるトラブル(HTML文書)

\ 8. \href{file:..\DOC\cmode.html}{cmode}: DVIware としての dviout(HTML文書)

\ 9. \href{file:input.dvi}{input}: Help TeX(左クリックでクリップボードにコピー可)

10. \href{file:..\ptex\naochan!.dvi}{naochan}: 縦書き文書の例

11. \href{file:..\graphic\tpic\tpicdoc.dvi}{tpicdoc}: tpic speicials について

12. \href{file:..\graphic\color\color.dvi}{color}: color specials(文字の色付け)色見本

13. \href{file:..\graphic\ps\epsfdoc.dvi}{epsfdoc}: PostScript special について(Ghostscriptの登録が必要)

14. \href{file:..\cfg\optcfg.dvi}{optcfg}: dviprtのプリンタ定義ファイルの仕様

15. \href{file:..\sample\slisampl.dvi}{Presentation-1}: dviout による自動プレゼンテーションの例

16. \href{file:..\sample\slisamp2.dvi}{Presentation-2}: dviout によるプレゼンテーション(スペースキーで進行:pause specials)

17. \href{file:..\sample\slisamp3.dvi}{Presentation-3}: dviout によるプレゼンテーション(カラー紙面)

18. \href{file:..\sample\slisamp4.dvi}{Presentation-4}: dviout によるプレゼンテーション(スペースキーで進行:dviout specials `+)

19. \href{file:..\special\demo.dvi}{掲示}: dviout による自動プレゼンテーションの例([ALT]+[F4] で終了)

20. \href{file:..\readme.txt}{Read me}: dviout for Windowsの解説(テキスト)

21. \href{file:..\history.txt}{History}: dviout for Windowsの開発歴史(テキスト)

22. \href{file:..\graphic\ps\sample2.ps}{Title}: PostScript画像(要GSview)

\section{さようなら}
この文書は、1ページのみで、ここで終わりです。

\end{document}
