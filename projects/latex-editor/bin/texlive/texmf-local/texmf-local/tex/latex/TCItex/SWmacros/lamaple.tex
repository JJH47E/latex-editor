
%
%%%%%%   \copyright Philip B. Yasskin, Texas A\&M Univ., 1994

%%%%%   Proposed TeX Macros for the Maple CalcLab Manual

%%%%%   Written to be used with Maple and LaTeX


%%%%%   FONT DEFINITIONS:

\font\smallrm=cmr8
\font\bigbf=cmbx10 scaled \magstep1
\font\bigcaps=cmcsc10 scaled \magstep2
\font\ssbf=cmssbx10
\font\caps=cmcsc10
\font\smallsl=cmsl8
%\font\mc=cmtt10  %Use this for maple command words.


%%%%%   SHORTHAND MACROS:

\def\n{\noindent }
\def\d{\displaystyle }
\def\h{\hfill}
\def\v{\vfill}
%\def\M{Maple }
%\def\power{{\char94 }}
\def\frac#1#2{{\displaystyle#1\over\displaystyle#2}}
%\def\plotsp{\vskip .4truein}


%%%%%  LAB MACROS:

%%%%%  This goes at the top of the first page of the lab, lab report, demo or drill etc.  
%%%%%  Put the names of the authors, Texas A\&M Univ., and the year.  

%\def\cprt#1{{\smallrm \n \copyright #1 \bigskip}}


%%%%%  Next put in the course and the lab's, demo's  or drill's title.
%%%%%  I suggest:  Calculus I, II, III, Differential Equations, 
%%%%%              Linear Algebra, etc.

%\def\lab\course#1\title#2{\smallskip \hang{\bigbf \n #1 Lab:} \quad {\bigcaps #2}}
%\def\demo\course#1\title#2{\smallskip \hang{\bigbf \n #1 Demo:} \quad {\bigcaps #2}}
%\def\drill\course#1\title#2{\smallskip \hang{\bigbf \n #1 Drill:} \quad {\bigcaps #2}}


%%%%%  If you don't put the copyright, you can put the authors.  
%%%%%  Put the names of the authors, Texas A\&M Univ., and the year.  

%\def\author#1{\goodbreak \smallskip \n {\ssbf Author: } #1 \bigskip}
%\def\authors#1{\goodbreak \smallskip \n {\ssbf Authors: } #1 \bigskip}


%%%%%  Several pages require a subtitle.

\def\labrep{\medskip \centerline{\ssbf Lab Report} \smallskip}
\def\usersguide{\medskip \centerline{\ssbf Users' Guide} \smallskip}


%%%%%  These are the major section headings of the lab:

\def\objectives{\goodbreak \smallskip \n {\ssbf Objectives: }}
\def\before{\goodbreak \smallskip \n {\ssbf Before Lab: }}
\def\prerequisites{\goodbreak \smallskip \n {\ssbf Prerequisites: }}
\def\maplecommands{\goodbreak \smallskip \n {\ssbf Maple Commands: }}
\def\initialization{\goodbreak \smallskip \n {\ssbf Initialization: }}
\def\labreq{%\goodbreak
 \smallskip \n {\ssbf Lab Report Requirements: }}
\def\procedure{\goodbreak \smallskip \n {\ssbf Procedure: }}
\def\method{\goodbreak \smallskip \n {\ssbf Method: }}
\def\files{\goodbreak \smallskip \n {\ssbf Files: } \quad}


%%%%%  Start each problem or step with a problem number.  
%%%%%  Note, I have not put in automatic problem numbering.

%\def\prob#1{\goodbreak \medskip \n #1. }
%\def\prob#1{\goodbreak \medskip \llap{#1. }}


%%%%%  These are subsections that might appear within a problem to 
%%%%%  help the students:

\def\mapleproc#1{\n {\caps Maple Procedure: } {\sl #1}}
\def\comment#1{\n {\sl #1}}
\def\note#1{\hang{} {\caps Note: } {\sl #1}}
\def\inlinenote#1{\n {\caps Note: } {\sl #1}}
\def\notes#1{\hang{} {\caps Notes: } {\sl #1}}
\def\inlinenotes#1{\n {\caps Notes: } {\sl #1}}
\def\hint#1{\hang{} {\caps Hint: } {\sl #1}}
\def\hints#1{\hang{} {\caps Hints: } {\sl #1}}
\def\answer#1{\hang{} {\caps Answer: } {\sl #1}}
\def\caution#1{\hang{} {\caps Caution: } {\sl #1}}

\def\say#1#2{\hang{} {\caps #1: } {\sl #2}}
\def\inlinesay#1#2{\n {\caps #1: } {\sl #2}}


%%%%%  These will format lines of Maple input and output code or leave blanks:

%\def\maplein#1{{\tt \hang{$>$} \quad #1}}		                        %use \in
%\def\mapleout{\hfill $\to$ \quad \undrline{3truein}}			       %use \out
%\def\mapleinout#1{{\tt \hang{$>$} \quad #1} \hfill $\to$ \quad \undrline{3truein}}  %use \inout

%\def\in#1{{\tt \hang{$>$} \quad #1}}		%a line of input
%\def\bin{\in{\undrline{5truein}} }			%a blank for input
%\def\outcent#1{\centerline{\tt #1}}		%a line of output - centered
%\def\outleft#1{{\tt \hang{} \quad #1}}		%a line of output - at the left
\def\out{\par \hfill $\to$ \quad \undrline{3truein} \hfill}	%a blank for output
\def\outagain{\par \hfill \undrline{3truein} \hfill}		%another blank for output
\def\longout{\par \hfill $\to$ \quad \undrline{5truein}}	%a long blank for output
\def\longoutagain{\par \hfill \undrline{5truein}}		%another long blank for output
%\def\inout#1{{\tt \hang{$>$} \quad #1} \hfill $\to$ \quad \undrline{3truein}}
					%a line of input and a blank for output
%\def\binout{\in{\undrline{2truein}} \hfill $\to$ \quad \undrline{3truein}}
					%a blank for input and a blank for output


%%%%%  Use this for a key on the keyboard:

\def\key#1{{\caps $\langle$#1$\rangle$}}

%%%%%  Use this for a menu item on the screen:

\def\menu#1{{\caps #1 }}



%%%%%  LAB REPORT MACROS:


%%%%%  This goes at the top on the first page of the lab report:
%%%%%  2 times for a lab report, 4 times for a group project

\def\nameidsec{ \vskip 5pt
\n     NAME    \undrline{1.5truein} 
\hfill ID      \undrline{1truein}
\hfill Section \undrline{.5truein} \par }


%%%%%  Next put the copyright or authors, the lab course and lab title
%%%%%  and the subtitle Lab Report,
%%%%%  using the \cprt, \lab, \author and \labrep macros from above.

%%%%%  Number each answer to match the problem numbers in the lab,
%%%%%  using the \prob macro.


%%%%%  This will draw a line for fill-in the blank.  
%%%%%  You specify the width of the line.

\def\undrline#1{{\vrule height -.3pt depth .4pt width #1 }}
%\def\uline#1{{\vrule height -.3pt depth .4pt width #1 }}

%%%%%  These leave space for open ended responses.
%%%%%  You specify a prompt, such as Reason or Discussion or etc.
%%%%%  You specify the amount of space to skip.

\def\answerspace#1#2{\n {#1:} \vskip #2 }


%%%%% This asks the student to circle a choice.

\def\circle{\n {\caps Circle:} \quad }


%%%%% This tells the students that this part of a problem is worth extra credit.
%%%%% You give the \% extra credit.

\def\extracredit#1{\n {\caps #1\% Extra Credit:} \quad }



%%%%%  PROJECT MACROS:

%%%%%  At the top of the project put the copyright  
%%%%%  using the \cprt macro from above.


%%%%%  Next put in the course and the project's title.

%\def\project\course#1\title#2{\hang{\bigbf \n #1 Project:} \quad {\bigcaps #2}}


%%%%%  Type the statement of the project.


%%%%%  Start each problem or step with a problem number
%%%%%  using the \prob macro form above.


%%%%%  Use macros from above for hints, comments, Maple input and output, etc.



%%%%%  PROJECT REPORT COVER PAGE MACROS:

%%%%%  First put the lab course and a blank for the project title
%%%%%  using the \project and \uline macros from above, e.g.
%%%%%  \project\course{Calculus I}\title{\uline{4truein}}


%%%%%  Next identify this as the cover page:

\def\cover{\vskip 25pt \centerline{\bigcaps Cover Page}}


%%%%%  Then ask for the instructor, teaching assistant and section number

\def\instrtasec{ \vskip 50pt \n  INSTRUCTOR    \hskip 20pt \undrline{2.5truein} 
                 \vskip 25pt \n  TEACHING ASST \hskip  5pt \undrline{2.5truein}
                 \hfill \n  SECTION \#    \undrline{1truein}
\vskip 50pt \par }


%%%%%  Finally ask for the names and ID's of the students:
%%%%%  2-3 times for a small group project, 4-5 times for a large group project.

\def\nameid{ \vskip 10pt
\n     NAME  \hskip  5pt   \undrline{2.5truein} 
\hfill ID    \hskip  5pt   \undrline{2.0truein}
\hfill \par }



%%%%%%  End of macro definitions
 
