\documentclass[12pt]{article}

\usepackage{sbc-template}

\usepackage{graphicx,url}

\usepackage[brazil]{babel}   
\usepackage[utf8]{inputenc}  

     
\sloppy

\title{Modelo de Proposta de Trabalho de Conclusão de Curso - CEDS - ITA}

\author{Nome Completo\inst{1}}


\address{Divisão de Ciência da Computação -- Instituto Tecnológico de Aeronáutica\\
  Praça Mal. Eduardo Gomes 50 -- São José dos Campos -- SP -- Brasil
 \email{user@ita.br}
}

\begin{document} 

\maketitle

\begin{abstract}
  Resumo em inglês, sempre obrigatório.
\end{abstract}
     
\begin{resumo} 
  Resumo em português, obrigatório apenas para TCC em português. Este meta-artigo descreve o estilo a ser usado na confecção de Trabalhos de Conclusão de Curso (TCC) para o Curso de Especialização em Ciência de Dados fornecido pela Divisão de Ciência da Computação do Instituto Tecnológico de Aeronáutica (ITA). É solicitada a escrita de resumo e abstract apenas para os artigos escritos em português. Artigos em inglês deverão apresentar apenas abstract.   Nos dois casos, o autor deve tomar cuidado para que o resumo (e o abstract) não ultrapassem 10 linhas cada, sendo que ambos devem estar na primeira
  página do artigo.
\end{resumo}

\section{Informação Geral}

Todos os Trabalhos de Conclusão de Curso (TCC)  devem ser escritos em inglês ou em português. O formato do papel deve ser A4 com coluna simples, 3,5 cm para margem, 2,5 cm para a margem inferior e 3,0 cm para as margens laterais, sem cabeçalhos ou rodapés. A fonte principal deve ser Times, tamanho nominal 12, com 6 pontos de espaço antes de cada parágrafo. Os números das páginas devem ser suprimidos. Os trabalhos devem respeitar o limite mínimo de 10 páginas de conteúdo, sem incluir as referências.

Caminho para o \textit{template} em LaTeX:


Caminho para o \textit{template} em Microsoft Word\copyright:

{\small \url{https://1drv.ms/w/s!AnwH34e356I3grxIeG87k-c0MrYi3g?e=WdWDJV}}

\section{Primeira Página} \label{sec:firstpage}

A primeira página deve conter o título do trabalho, o nome e o endereço do
autores, resumo em inglês e ``resumo'' em português (``resumos'' são
exigido apenas para artigos escritos em português). O título deve ser centralizado
em toda a página, em negrito, tamanho 16 e espaço de 12 pontos
antes de si. Os nomes dos autores devem ser centralizados em fonte 12, negrito, todos dispostos na mesma linha, separados por vírgulas e com 12 vírgulas
espaço após o título. Os endereços devem ser centralizados em fonte tamanho 12, também com 12 pontos de espaço após os nomes dos autores. Os endereços de e-mail devem ser escrita na fonte Courier New, tamanho nominal de 10 pontos, com espaço de 6 pontos antes e 6 pontos de espaço depois.

O \textit{abstract} e resumo (se for o caso) devem estar em fonte Times 12,
recuo de 0,8 cm em ambos os lados. A palavra \textbf{Abstract} e \textbf{Resumo},
devem ser escritas em negrito e devem preceder o texto.

\section{Seções e Parágrafos}

Os títulos das seções devem estar em negrito, 13pt, alinhados à esquerda. Deve haver um extra 12 pt de espaço antes de cada título. A numeração das seções é opcional. O primeiro parágrafo de cada seção não deve ser recuado, enquanto as primeiras linhas de parágrafos subsequentes devem ter recuo de 1,27 cm.

\subsection{Subseções}

Os títulos das subseções devem estar em negrito, 12pt, alinhados à esquerda.

\section{Figuras e Legendas}\label{sec:figs}


Legendas de figuras e tabelas devem ser centralizadas se menos de uma linha
(Figura~\ref{fig:exemploFig1}), justificado e recuado por 0,8 cm em ambas as margens, conforme mostrado na Figura~\ref{fig:exemploFig2}. A fonte da legenda deve
ser Helvetica, 10 pontos, negrito, com 6 pontos de espaço antes e depois de cada
rubrica.

\begin{figure}[ht]
\centering
\includegraphics[width=.5\textwidth]{fig1.jpg}
\caption{Uma figura com título curto}
\label{fig:exemploFig1}
\end{figure}

\begin{figure}[ht]
\centering
\includegraphics[width=.3\textwidth]{fig2.jpg}
\caption{Esta figura é um exemplo de uma legenda de figura que leva mais de uma
   linha e justificado considerando as margens mencionadas na Seção~\ref{sec:figs}.}
\label{fig:exemploFig2}
\end{figure}

Nas tabelas, procure evitar o uso de fundos coloridos ou sombreados e evite
linhas de enquadramento grossas, duplicadas ou desnecessárias. Ao relatar dados empíricos, não use mais dígitos decimais do que o permitido por sua precisão e
reprodutibilidade. A legenda da tabela deve ser colocada antes da tabela (ver Tabela \ref{tab:exTable1}) e a fonte utilizada também deve ser Helvetica, 10 pontos, negrito, com 6 pontos de espaço antes e depois de cada legenda.


\begin{table}[ht]
\centering
\begin{tabular}{|c|c|}
\hline
Cor      & Valores Numéricos          \\ \hline
Azul     & R\$ 5,00                   \\ \hline
Amarelo  & 3.1456                     \\ \hline
Vermelho & 5 x 10\textasciicircum{}-2 \\ \hline
\end{tabular}
\caption{Exemplo de Tabela}
\label{tab:exTable1}
\end{table}

\section{Estrutura da Proposta Trabalho de Conclusão de Curso)}

Essa proposta de TCC deverá ser fornecida neste \textit{template} com a seguinte estrutura \textbf{obrigatória}:
\begin{itemize}
\item Seção 1 - Introdução
\item Seção 1.1 - Contextualização
\item Seção 1.2 - Problema de Pesquisa
\item Seção 1.3 - Objetivo Geral
\item Seção 1.4 - Objetivos Específicos
\item Seção 2 - Resultados Esperados
\item Seção 3 - Cronograma de Pesquisa
\end{itemize}

\section{Estrutura do Trabalho de Conclusão de Curso)}

O TCC deverá estar com a seguinte estrutura (mínima)

\begin{itemize}
\item Seção 1 - Introdução
\item Seção 1.1 - Contextualização
\item Seção 1.2 - Problema de Pesquisa
\item Seção 1.3 - Objetivo Geral
\item Seção 1.4 - Objetivos Específicos
\item Seção 2 - Desenvolvimento*
\item Seção 3 - Analise e Discussão dos Resultados*
\item Seção 4 - Conclusão
\end{itemize}

*Essas seções poderão ter a estrutura que for mais adequada e a ser definida com o orientador.

\section{Referências}

As referências bibliográficas devem ser inequívocas e uniformes. É obrigatório fornecer as referências dos nomes dos autores entre colchetes, por exemplo, \cite{knuth:84}, \cite{boulic:91} e \cite{smith:99}.

As referências devem ser listadas usando tamanho de fonte de 12 pontos, com 6 pontos de espaço antes de cada referência. A primeira linha de cada referência não deve ser recuada, enquanto as seguintes devem ser recuadas em 0,5 cm.

\bibliographystyle{sbc}
\bibliography{sbc-template}

\end{document}